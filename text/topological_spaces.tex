\subsection{Topological spaces}\label{subsec:topological_spaces}

\begin{remark}\label{rem:topologies_and_sigma_algebras_as_lattices}
  Some structures like \hyperref[def:topological_space]{topologies} and \hyperref[def:sigma_algebra]{sigma algebras} are \hyperref[def:semilattice/submodel]{bounded sublattices} of the \hyperref[thm:boolean_algebra_of_subsets]{Boolean algebra of subsets} of the ambient space. This allows us to study certain aspects of \hyperref[sec:general_topology]{general topology} and \hyperref[sec:measure_theory]{measure theory} via \hyperref[subsec:lattices]{lattice theory}.
\end{remark}

\begin{definition}\label{def:topological_space}\mcite[11]{Engelking1989}
  A \term{topology} on a set \( X \) is a \hyperref[def:semilattice/lattice]{join-complete} \hyperref[def:semilattice/submodel]{bounded sublattice} of the \hyperref[thm:boolean_algebra_of_subsets]{Boolean algebra of subsets}.

  This is commonly expressed via the following three conditions on an arbitrary family \( \mscrT \) of subsets:
  \begin{thmenum}
    \thmitem[def:topological_space/O1]{O1} \( \mscrT \) contains both \( \varnothing \) and \( X \).
    \thmitem[def:topological_space/O2]{O2} \( \mscrT \) is closed under finite intersections: \( U, V \in \mscrT \) implies \( U \cap V \in \mscrT \).
    \thmitem[def:topological_space/O3]{O3} \( \mscrT \) is closed under arbitrary unions: \( \mscrT' \subseteq \mscrT \) implies \( \bigcup \mscrT' \in \mscrT \).
  \end{thmenum}

  If \( \mscrT \) is a topology on \( X \), we call \( (X, \mscrT) \) a \term{topological space}. When the topology is obvious from the context, we refer to \( X \) itself as a topological space.

  Elements of \( X \) are called \term{points} of the topological space, elements of \( \mscrT \) are called \term{open sets} and \hyperref[thm:boolean_algebra_of_subsets/complement]{set-theoretic complements} of open sets are called \term{closed sets}. We will later prove that open sets are \hyperref[def:fixed_point]{fixed points} of the \hyperref[def:topological_interior_operator]{topological interior operator} and closed sets are fixed points of the \hyperref[def:topological_closure_operator]{topological closure operator}.
\end{definition}

\begin{definition}\label{def:topology_ordering}\mcite[16]{Engelking1989}
  We say that the topology \( \mscrT \) on \( X \) is \term{coarser} than the topology \( \mscrO \) on \( X \) if \( \mscrT \) is a subset of \( \mscrO \). We say that \( \mscrO \) is \term{finer} than \( \mscrT \).

  As we shall see in \fullref{thm:lattice_of_topologies}, this is a \hyperref[def:partially_ordered_set]{partial ordering} in the \hyperref[thm:lattice_of_topologies]{lattice of topologies} on \( X \).
\end{definition}

\begin{definition}\label{def:discrete_topology}\mcite[37]{Kelley1975}
  The \term{discrete topology} on the set \( X \) is simply the Boolean algebra of subsets \( \pow(X) \).

  It is the in the \hyperref[def:topology_ordering]{finest} topology in the \hyperref[thm:lattice_of_topologies]{lattice of topologies} on \( X \), and is \hyperref[def:category_adjunction]{left adjoint} to the \hyperref[def:concrete_category]{forgetful functor} \( U: \cat{Top} \to \cat{Set} \).
\end{definition}

\begin{remark}\label{rem:discrete_topology_and_not_open_sets}
  In \hyperref[def:connected_space]{connected spaces}, being closed implies not being open. At the other end, under the \hyperref[def:discrete_topology]{discrete topology}, every set is open. Statements regarding a set not being open are wrong in general, even if they intuitively seem true.
\end{remark}

\begin{definition}\label{def:indiscrete_topology}\mcite[37]{Kelley1975}
  The \term{trivial} or \term{indiscrete topology} on the set \( X \) is simply the \hyperref[def:boolean_algebra/trivial]{trivial subalgebra} \( \set{ 0, X } \) of \( \pow(X) \).

  It is the in the \hyperref[def:topology_ordering]{coarsest} topology in the \hyperref[thm:lattice_of_topologies]{lattice of topologies} on \( X \), and is \hyperref[def:category_adjunction]{right adjoint} to the \hyperref[def:concrete_category]{forgetful functor} \( U: \cat{Top} \to \cat{Set} \).
\end{definition}

\begin{proposition}\label{thm:empty_set_discrete_and_indiscrete_topologies}
  A set is \hyperref[def:empty_set]{empty} if and only if its \hyperref[def:discrete_topology]{discrete} and \hyperref[def:indiscrete_topology]{indiscrete} topologies coincide.
\end{proposition}
\begin{proof}
  Trivial.
\end{proof}

\begin{definition}\label{def:sierpinski_space}\mcite[exmpl. 2.1.7(e)]{Perrone2019}
  The \hyperref[def:boolean_algebra/trivial]{two-element Boolean algebra} \( \set{ \top, \bot } \) endowed with the topology
  \begin{equation*}
    \set[\Big]{ \varnothing, \set{ \top }, \set{ \top, \bot } }
  \end{equation*}
  is called the \term{Sierpinski space}.

  \begin{figure}[!ht]
    \hfill
    \includegraphics[page=1]{output/def__sierpinski_space.pdf}
    \hfill\hfill
    \caption{A \hyperref[def:hasse_diagram]{Hasse diagram} of the \hyperref[def:discrete_topology]{discrete} and the \hyperref[def:sierpinski_space]{Sierpinski topologies} on \( \set{ \top, \bot } \). }
    \label{fig:def:sierpinski_space}
  \end{figure}
\end{definition}

\begin{remark}\label{rem:constructing_topologies}
  Among others, we have the following ways of constructing topologies:
  \begin{thmenum}
    \thmitem{rem:constructing_topologies/open} By specifying all open sets --- see \fullref{def:topological_space}.

    \thmitem{rem:constructing_topologies/closed} By specifying all closed sets --- see \fullref{thm:topology_from_closed_sets}.

    \thmitem{rem:constructing_topologies/base} By specifying only a \hyperref[def:topological_base]{base} of open sets --- see \fullref{thm:topology_from_base}.

    \thmitem{rem:constructing_topologies/subbase} By specifying only some family of open sets --- see \fullref{thm:topology_from_subbase}.

    This is the approach we use for order topologies in \fullref{def:order_topology}.

    \thmitem{rem:constructing_topologies/local_base} By specifying a \hyperref[def:topological_local_base]{local base} of open sets at every point --- see \fullref{thm:topology_from_local_base}.

    This is the approach we use for metric spaces in \fullref{def:metric_topology} and uniform spaces in \fullref{def:uniform_topology}.

    \thmitem{rem:constructing_topologies/local_subbase} By specifying a \hyperref[def:topological_local_subbase]{local subbase} of open sets at every point --- see \fullref{thm:topology_from_local_subbase}.

    \thmitem{rem:constructing_topologies/closure} By specifying a \hyperref[def:topological_closure_operator]{closure operator} --- see \fullref{thm:topology_from_closure_operator}.

    \thmitem{rem:constructing_topologies/interior} By specifying an \hyperref[def:topological_interior_operator]{interior operator} --- see \fullref{thm:topology_from_interior_operator}.
  \end{thmenum}
\end{remark}

\begin{proposition}\label{thm:topology_from_closed_sets}
  If \( \mscrF \) is a \hyperref[def:semilattice/submodel]{bounded sublattice} of the \hyperref[thm:boolean_algebra_of_subsets]{Boolean algebra of subsets} of \( X \) that is closed under arbitrary intersections, then \( \mscrT \coloneqq \set{ X \setminus F \given F \in \mscrF } \) is a \hyperref[def:topological_space]{topology} on \( X \). This allows us to construct a topology by specifying the closed sets in advance.

  This is commonly expressed via the following three conditions on an arbitrary family \( \mscrF \) of subsets:
  \begin{thmenum}
    \thmitem[thm:topology_from_closed_sets/C1]{C1} \( \mscrF \) contains both \( \varnothing \) and \( X \).
    \thmitem[thm:topology_from_closed_sets/C2]{C2} \( \mscrF \) is closed under finite unions: \( U, V \in \mscrF \) implies \( U \cup V \in \mscrF \).
    \thmitem[thm:topology_from_closed_sets/C3]{C3} \( \mscrF \) is closed under arbitrary intersections: \( \mscrF' \subseteq \mscrF \) implies \( \bigcap \mscrF' \in \mscrF \).
  \end{thmenum}
\end{proposition}
\begin{proof}
  We will show that the axioms \ref{thm:topology_from_closed_sets/C1} --- \ref{thm:topology_from_closed_sets/C3} imply \ref{def:topological_space/O1} --- \ref{def:topological_space/O3}. The converse is also true, but we will not give a proof because it is straightforward.

  \SubProofOf{def:topological_space/O1} Follows directly from \ref{thm:topology_from_closed_sets/C1}.
  \SubProofOf{def:topological_space/O2} If \( F, G \in \mscrF \), then \ref{thm:topology_from_closed_sets/C2} implies \( F \cup G \in \mscrF \).

  By \fullref{thm:de_morgans_laws_for_sets},
  \begin{equation*}
    X \setminus (F \cup G) = (X \setminus F) \cap (X \setminus G),
  \end{equation*}
  hence \( (X \setminus F), (X \setminus G) \in \mscrT \) implies \( X \setminus (F \cup G) \in \mscrT \).

  \SubProofOf{def:topological_space/O3} Follows from \ref{thm:topology_from_closed_sets/C3} via \fullref{thm:de_morgans_laws_for_sets}.
\end{proof}

\begin{definition}\label{def:neighborhood_system}\mcite[39]{Kelley1975}
  Let \( A \) be a nonempty set in a topological space \( (X, \mscrT) \). We say that the set \( U \) is a \term[bg=околност,ru=окрестность]{neighborhood} of \( A \) if \( A \subseteq U \) and \( U \) is open. The definition usually applies to singleton sets \( A = \set{ x } \), in which case we refer to \enquote{neighborhoods of a point}.

  We define the \term{neighborhood system} of \( A \) as
  \begin{equation*}
    \mscrT(A) \coloneqq \set{ U \in \mscrT \given A \subseteq U }.
  \end{equation*}

  It is also called a \term{neighborhood filter} because of \fullref{thm:neighborhood_filter}.

  Note that some authors --- for example \cite[12]{Engelking1989}, \cite[7]{Rudin1991Functional} and \cite[def. 10.2]{ИвановТужилин2017} --- stick to this definition, while others define neighborhoods to be an arbitrary superset of what we call a neighborhood --- for example \cite[8]{Kelley1975}.
\end{definition}

\begin{example}\label{ex:def:neighborhood_system}
  We list several examples of \hyperref[def:neighborhood_system]{neighborhood systems}:
  \begin{thmenum}
    \thmitem{ex:def:neighborhood_system/discrete} With the \hyperref[def:discrete_topology]{discrete topology} on \( X \), the \hyperref[def:neighborhood_system]{neighborhood system} of the point \( x \) is the \hyperref[ex:principal_ultrafilter]{principal ultrafilter} of \( x \).

    \thmitem{ex:def:neighborhood_system/indiscrete} At the other end, with the \hyperref[def:indiscrete_topology]{indiscrete topology} on \( X \), the \hyperref[def:neighborhood_system]{neighborhood system} of any point \( x \) is \( \set{ X } \).
  \end{thmenum}
\end{example}

\begin{proposition}\label{thm:neighborhood_filter}
  The \hyperref[def:neighborhood_system]{neighborhood system} \( \mscrT(A) \) is a \hyperref[def:lattice_ideal/ideal]{filter} in \( \mscrT \).

  For a singleton set \( A = \set{ x } \), \( \mscrT(x) \) is a \hyperref[def:lattice_ideal/submodel]{subfilter} of the \hyperref[ex:principal_ultrafilter]{principal ultrafilter} of \( x \).
\end{proposition}
\begin{proof}
  The intersection of finitely many neighborhoods of \( A \) is again a neighborhood of \( A \).

  The union of a neighborhood of \( A \) with any open set is again a neighborhood of \( A \).
\end{proof}

\begin{definition}\label{def:topological_base}\mcite[12]{Engelking1989}
  Fix a topological space \( (X, \mscrT) \). We say that the family \( \mscrB \subseteq \mscrT \) is a \term{base} for the topology \( \mscrT \) if \( \mscrB \) satisfies any of the equivalent conditions:
  \begin{thmenum}
    \thmitem{def:topological_base/union} Every open set \( U \in \mscrT \) is the union \( U = \bigcup \mscrB' \) of some subset \( \mscrB' \subseteq \mscrB \)
    \thmitem{def:topological_base/subset} For any point \( x \in X \) and for any neighborhood \( U \) of \( x \), there exists a set \( V \in \mscrB \) in the base such that \( x \in V \subseteq U \)
  \end{thmenum}
\end{definition}
\begin{proof}
  \ImplicationSubProof{def:topological_base/union}{def:topological_base/subset} Fix a point \( x \in X \) and a neighborhood \( U \in T \) of \( x \). Let \( \mscrB' \) be a subfamily of \( \mscrB \) such that
  \begin{equation*}
    U = \bigcup \mscrB'.
  \end{equation*}

  Then \( x \in V \) for at least one \( V \in \mscrB' \) by definition of union.

  \ImplicationSubProof{def:topological_base/subset}{def:topological_base/union} Fix an open set \( U \in T \). Then for every \( x \in U \), there exists a set \( V_x \in \mscrB \) such that \( x \in V_x \subseteq U \). We have
  \begin{equation*}
    \bigcup_{x \in U} V_x \subseteq U \subseteq \bigcup_{x \in U} V_x,
  \end{equation*}
  thus
  \begin{equation*}
    U = \bigcup_{x \in U} V_x.
  \end{equation*}
\end{proof}

\begin{example}\label{ex:def:topological_base}
  We list several examples of \hyperref[def:topological_base]{topological bases}:
  \begin{thmenum}
    \thmitem{ex:def:topological_base/topology} Every topology is itself a base. The topology with the empty set removed is also a base.

    \thmitem{ex:def:topological_base/superset} If \( \mscrB \) is a base of \( \mscrT \), then any set between \( \mscrB \) and \( \mscrT \) is also a base.

    \thmitem{ex:def:topological_base/indiscrete} The family \( \set{ X } \) is a base for the \hyperref[def:indiscrete_topology]{indiscrete topology} on a nonempty set \( X \). It is the only proper base.

    For an empty space \( X = \varnothing \), the indiscrete topology is \( \set{ \varnothing } \), and hence the empty family is a base.

    \thmitem{ex:def:topological_base/sierpinski} The \hyperref[def:sierpinski_space]{Sierpinski space} also has only two bases: the topology itself and the set
    \begin{equation*}
      \set[\Big]{ \set{ \top }, \set{ \top, \bot } }.
    \end{equation*}

    \thmitem{ex:def:topological_base/discrete} For the \hyperref[def:discrete_topology]{discrete topology} \( \pow(X) \) on a set \( X \), the following is a base:
    \begin{equation*}
      \mscrB = \set[\Big]{ \set{ x } \given x \in X }.
    \end{equation*}

    Indeed, every subset of \( X \) is a union of members of \( \mscrB \).

    \thmitem{ex:def:topological_base/not_closed_under_intersections} A topological base may or may not be closed under finite intersections. For example, consider the \hyperref[def:order_topology]{order topology} on \( \BbbR \) with base \eqref{eq:def:order_topology/base}.

    We can remove any open interval \( (a, b) \) from the base and the result would still be a base because, for \( a < x < y < b \),
    \begin{equation*}
      (a, b) = (a, y) \cup (x, b).
    \end{equation*}

    Then the base would not be closed under intersections because
    \begin{equation*}
      (a, b) = (-\infty, b) \cap (a, \infty).
    \end{equation*}

    This leads to \ref{thm:topology_from_base/B2}.
  \end{thmenum}
\end{example}

\begin{proposition}\label{thm:topology_from_base}\mcite[12]{Engelking1989}
  Let \( X \) be an arbitrary set and let \( \mscrB \) be a family of subsets that satisfies the conditions
  \begin{thmenum}
    \thmitem[thm:topology_from_base/B1]{B1} \( \mscrB \) \hyperref[def:set_partition]{covers} \( X \), i.e. \( \bigcup \mscrB = X \).
    \thmitem[thm:topology_from_base/B2]{B2} If \( U, V \in \mscrB \) and if \( x \in U \cap V \), there exists \( W \in \mscrB \) such that \( x \in W \) and \( W \subseteq U \cap V \).
  \end{thmenum}

  Then the family
  \begin{equation}\label{eq:thm:topology_from_base/topology}
    \mscrT \coloneqq \set*{ \bigcup \mscrB' \given \mscrB' \subseteq \mscrB }
  \end{equation}
  is the coarsest topology on \( X \) containing \( \mscrB \).

  Furthermore, \( \mscrB \) satisfies \fullref{def:topological_base/union} is thus a \hyperref[def:topological_base]{base} of \( \mscrT \). We say that the base \( \mscrB \) \term{generates} \( \mscrT \).
\end{proposition}
\begin{proof}
  \SubProof{Proof that \( \mscrT \) is a topology}

  \SubProofOf*{def:topological_space/O1} \( \varnothing = \bigcup \varnothing \) and, by \ref{thm:topology_from_base/B1}, \( X = \bigcup \mscrB \).

  \SubProofOf*{def:topological_space/O3} Pick a subfamily \( \mscrT' \subseteq \mscrT \). For each \( U \in \mscrT' \), there exists a subfamily \( \mscrB_U' \) such that \( U = \bigcup \mscrB_U' \). Define
  \begin{equation*}
    \mscrB'
    \coloneqq
    \bigcup_{U \in \mscrT'} \mscrB_U'
    =
    \set{ B \in \mscrB \given \qexists {U \in \mscrT'} B \in \mscrB_U' }.
  \end{equation*}

  It is a subset of \( \mscrB \). Furthermore,
  \begin{align*}
    \bigcup \mscrT'
    &=
    \set{ x \in X \given \qexists{U \in \mscrT'} x \in U }
    = \\ &=
    \set{ x \in X \given \qexists{U \in \mscrT'} x \in \bigcup \mscrB_U' }
    = \\ &=
    \set{ x \in X \given \qexists{U \in \mscrT'} \qexists{B \in \mscrB_U'} x \in B }
    = \\ &=
    \set{ x \in X \given \qexists{B \in \mscrB'} x \in B }
    =
    \bigcup \mscrB'.
  \end{align*}

  Hence, \( \bigcup\mscrT' \in \mscrT \).

  \SubProofOf*{def:topological_space/O2} Fix \( U, V \in T \). Then there exist families \( \mscrB_U, \mscrB_V \subseteq B \) such that \( U = \bigcup \mscrB_U' \) and \( V = \bigcup \mscrB_V' \).

  Furthermore, by \ref{thm:topology_from_base/B2}, for every \( x \in U' \cap V' \) there exists a neighborhood \( W_x \) of \( x \) such that \( W \subseteq U' \cap V' \). We have
  \begin{equation*}
    U' \cap V' \subseteq \bigcup\set{ W_x \given x \in U' \cap V' } \subseteq U' \cap V'.
  \end{equation*}

  Then
  \begin{align*}
    U \cap V
    &=
    \parens*{ \bigcup \mscrB_U } \cap \parens*{ \bigcup \mscrB_V }
    = \\ &=
    \set{ x \in X \given x \in \bigcup \mscrB_U \T{and} x \in \bigcup \mscrB_V }
    = \\ &=
    \set{ x \in X \given \qexists{ U' \in \mscrB_U' } x \in U' \T{and} \qexists{ V' \in \mscrB_V' } x \in V' }
    \reloset {\ref{thm:pulling_quantifiers_out}} = \\ &=
    \set{ x \in X \given \qexists{ U' \in \mscrB_U' } \qexists{ V' \in \mscrB_V' } x \in U' \cap V' }
    = \\ &=
    \bigcup\set{ U' \cap V' \given U' \in \mscrB_U' \T{and} V' \in \mscrB_V' }.
    = \\ &=
    \bigcup\set{ W_x \given U' \in \mscrB_U' \T{and} V' \in \mscrB_V' \T{and} x \in U' \cap V' }.
  \end{align*}

  Therefore, \( U \cap V \) is a union of members of \( \mscrB \), and is thus a member of \( \mscrT \).

  \SubProof{Proof of minimality of \( \mscrT \)} Let \( \mscrO \) be another topology on \( X \) containing \( \mscrB \).

  Since \( \mscrO \) must be closed under arbitrary unions, and since \( \mscrB \) may not be closed under arbitrary unions, it follows that \( \mscrO \) must contain the union \( \bigcup \mscrB' \) of an arbitrary subset \( \mscrB' \subseteq \mscrB \).

  Therefore, \( \mscrT \) is necessarily a subset of \( \mscrO \).
\end{proof}

\begin{proposition}\label{thm:base_can_generate_topology}
  Every \hyperref[def:topological_base]{base} satisfies \ref{thm:topology_from_base/B1} and \ref{thm:topology_from_base/B2}.
\end{proposition}
\begin{proof}
  Let \( (X, \mscrT) \) be a topological space and let \( \mscrB \) be a base for \( \mscrT \).

  \SubProofOf{thm:topology_from_base/B1} Since \( X \) is always a member of the topology, \fullref{def:topological_base/union} implies that \( X \) is a union.

  \SubProofOf{thm:topology_from_base/B2} Let \( U \) and \( V \) be members of \( \mscrB \). Since \( U \cap V \in \mscrT \), there exists a family \( \mscrB' \subseteq \mscrB \) such that \( U \cap V = \bigcup \mscrB' \).

  Thus, for every point \( x \) of \( U \cap V \), by definition of set union there exists a member \( W \) of \( \mscrB' \) such that \( x \in W \). Obviously \( W \subseteq U \cap V \).
\end{proof}

\begin{proposition}\label{thm:set_open_iff_neighborhood_is_contained}
  Let \( \mscrB \) be a \hyperref[def:topological_base]{base} of the space \( (X, \mscrT) \).

  A set \( A \subseteq X \) is open if and only if every point of \( A \) has a neighborhood in \( \mscrB \) contained in \( A \).
\end{proposition}
\begin{proof}
  This holds vacuously for empty sets. Assume that \( A \subseteq X \) is nonempty.

  \SufficiencySubProof Assume that \( A \) is open and let \( x_0 \in A \). Then \( A \) is a neighborhood of \( x_0 \), hence there exists some \( V \in \mscrB \) such that \( x_0 \in V \subseteq A \).

  \NecessitySubProof Assume that every point \( x \in A \) has a neighborhood \( U_x \) in \( \mscrB \) that \( U_x \subseteq A \). Take the union
  \begin{equation*}
    V \coloneqq \bigcup_{x \in A} U_x.
  \end{equation*}

  Obviously \( V \subseteq A \). Conversely, since \( x \) belongs to \( U_x \) for every \( x \in A \), it follows that their union \( V \) is contained in \( A \).

  We conclude that \( A = V \) is open as a union of open sets.
\end{proof}

\begin{definition}\label{def:topological_space_weight}
  We define the \term{weight} of the topological space \( (X, \mscrT) \) as the \hyperref[def:cardinal]{cardinal}
  \begin{equation*}
    w(X, \mscrT) \coloneqq \min \set{ \card(\mscrB) \given \mscrB \T{is a base for} \mscrT }.
  \end{equation*}

  We simply write \( w(X) \) when the topology is clear from the context.

  Spaces for which \( w(X) \leq \hyperref[def:aleph_hierarchy]{\aleph_0} \) are said to be \term{second-countable}.

  See \fullref{thm:topological_space_countability} for how weight relates to other measurements of the space's \enquote{size}.
\end{definition}
\begin{defproof}
  The definition is correct because cardinals are well-ordered when regarded as initial ordinals.
\end{defproof}

\begin{example}\label{ex:def:topological_space_weight}
  We list several examples of \hyperref[def:topological_space_weight]{topological space weights}:
  \begin{thmenum}
    \thmitem{ex:def:topological_space_weight/indiscrete} We have shown in \fullref{ex:def:topological_base/indiscrete} that \( \set{ X } \) is the unique base of the indiscrete topology on a nonempty set \( X \) that is not the topology itself. Hence, the weight of any indiscrete topology is \( 1 \).

    The weight of the indiscrete topology on the empty set is \( 0 \).

    \thmitem{ex:def:topological_space_weight/sierpinski} Similarly to \fullref{ex:def:topological_space_weight/indiscrete}, \fullref{ex:def:topological_base/sierpinski} allows us to conclude that the weight of the Sierpinski space is \( 2 \).

    \thmitem{ex:def:topological_space_weight/discrete} Therefore, the weight of the discrete topology on \( X \) is \( \card X \).

    Let \( \mscrB \) be an arbitrary base of the \hyperref[def:discrete_topology]{discrete topology} \( \pow(X) \) on \( X \).

    There are \( 2^{\card \mscrB} \) subfamilies of \( \mscrB \). The union of two distinct subfamilies may be the same, hence
    \begin{equation*}
      \card(\pow X) = 2^{\card X} \leq 2^{\card \mscrB}.
    \end{equation*}

    \Fullref{thm:cardinal_exponentiation_comparison} then implies that \( \card X \leq \card \mscrB \). The base discussed in \fullref{ex:def:topological_base/discrete} has the cardinality of \( X \) itself, hence it is possible to attain the equality.
  \end{thmenum}
\end{example}

\begin{proposition}\label{thm:base_has_subset_of_minimal_weight}
  Every \hyperref[def:topological_base]{base} of a topological space has a subset that is a base of minimal cardinality.
\end{proposition}
\begin{thmcomment}
  This infinite case of the theorem can be found as \enquote{theorem 2} in \cite{АлександровУрысон1950}. For this reason, some topologists I know refer to it as \enquote{the} Alexandrov-Urysohn theorem, although it is not the focus of the paper.
\end{thmcomment}
\begin{proof}
  Let \( \mscrB \) and \( \mscrC \) be bases of \( (X, \mscrT) \). Suppose that \( \mscrC \) has minimal cardinality. We will find a subset \( \mscrB_0 \) of \( \mscrB \) that is itself a base if minimal cardinality.

  \SubProof{Proof if weight is finite} Suppose that \( \mscrC \) is finite. We will show that \( \mscrC \subseteq \mscrB \).

  For each set \( A \), denote by \( \mscrC(A) \) the subfamily of \( \mscrC \) of all sets containing \( A \). Also, denote by \( C_x \) the intersection of \( \mscrC(x) \), i.e. all sets in \( \mscrC \) that contain \( x \).

  Note that \( C_x \) is a member of \( \mscrB \). Indeed, there must exist some \( U \in \mscrB \) such that \( x \in U \subseteq C_x \). Then there exists some \( V \in \mscrC \) such that \( x \in V \subseteq U \). That is, we have \( C_x \subseteq V \subseteq U \), and hence \( U = C_x \). Then
  \begin{equation*}
    \set{ C_x \given x \in X } \subseteq \mscrB.
  \end{equation*}

  Since \( \mscrB \) was an arbitrary base, we conclude that
  \begin{equation*}
    \set{ C_x \given x \in X } \subseteq \mscrC.
  \end{equation*}

  Now fix some nonempty set \( V \) in \( \mscrC \). Then \( C_x \subseteq V \) for every \( x \in V \), hence
  \begin{equation*}
    V = \bigcup_{x \in V} C_x.
  \end{equation*}

  If \( C_x \neq V \) for every \( x \in V \), then \( \mscrC \setminus \set{ V } \) is also a base, contradicting the minimality of \( \mscrC \). Hence, \( V = C_x \) for at least one \( x \in V \), implying that
  \begin{equation*}
    \mscrC = \set{ C_x \given x \in X }.
  \end{equation*}

  We conclude that \( \mscrC \subseteq \mscrB \).

  \SubProof{Proof if weight is infinite} Suppose that \( \mscrC \) is infinite, and consider the following set:
  \begin{equation*}
    \mscrB_0 \coloneqq \set{ B \in \mscrB \given \qexists {U \in \mscrC} \qexists {V \in \mscrC} U \subseteq B \subseteq V }.
  \end{equation*}

  \Fullref{thm:simplified_cardinal_arithmetic/infinite} implies that
  \begin{equation*}
    \card\mscrB_0 \leq \card \mscrC^2 = \card \mscrC = w(X, \mscrT).
  \end{equation*}

  We will show that \( \mscrB_0 \) is a base. Let \( O \) be an arbitrary nonempty open set from \( \mscrT \), and let \( x \in O \). Then there exists some set \( V \) from \( \mscrC \) such that \( x \in V \subseteq O \). Since \( V \) is open, there exists some nonempty set \( B \) from \( \mscrB \) such that \( x \in B \subseteq V \), and similarly there exists a set \( U \) in \( \mscrB \) such that \( x \in U \subseteq V \). That is, \( U \subseteq B \subseteq V \subseteq O \). Then \( B \in \mscrB_0 \). We have found a neighborhood \( B \) of \( x \) from \( \mscrB_0 \) contained in \( O \) for an arbitrary open set \( O \) and point \( x \) in \( O \).

  Therefore, \( \mscrB_0 \) satisfies \fullref{def:topological_base/subset} is thus a base of \( (X, \mscrT) \).
\end{proof}

\begin{definition}\label{def:topological_subbase}\mcite[12]{Engelking1989}
  Fix a topological space \( (X, \mscrT) \). We say that the family \( \mscrS \subseteq \mscrT \) is a \term{subbase} for the topology \( \mscrT \) if \ref{thm:topology_from_base/B1} holds and, instead of requiring \ref{thm:topology_from_base/B2}, we require that for every \( U \in \mscrT \), there exists some subfamily \( V_1, \ldots, V_n \in \mscrS \) such that
  \begin{equation}\label{eq:def:topological_subbase}
    \varnothing \subsetneq V_1 \cap \cdots \cap V_n \leq U.
  \end{equation}
\end{definition}

\begin{example}\label{ex:def:topological_subbase}
  We list several examples of \hyperref[def:topological_subbase]{topological subbases}:
  \begin{thmenum}
    \thmitem{ex:def:topological_subbase/topology} Every topology is itself a subbase, as well as any base.

    \thmitem{ex:def:topological_subbase/superset} If \( \mscrS \) is a subbase of \( \mscrT \), then any set between \( \mscrS \) and \( \mscrT \) is also a subbase.

    \thmitem{ex:def:topological_subbase/indiscrete} The family \( \set{ X } \) is a subbase for the \hyperref[def:indiscrete_topology]{indiscrete topology} on any set \( X \).

    \thmitem{ex:def:topological_subbase/discrete} For the \hyperref[def:discrete_topology]{discrete topology} \( \pow(X) \) on a set \( X \) with at least three elements, the following is a subbase:
    \begin{equation*}
      \mscrS = \set[\Big]{ \set{ x, y } \given x \in X \T{and} y \in X \T{and} x \neq y }.
    \end{equation*}

    Indeed, \fullref{ex:def:topological_base/discrete} implies that
    \begin{equation*}
      \mscrB = \set[\Big]{ \set{ x } \given x \in X }
    \end{equation*}
    is a base, and, for \( y \neq z \) distinct from \( x \),
    \begin{equation*}
      \set{ x } = \set{ x, y } \cap \set{ x, z }.
    \end{equation*}

    \thmitem{ex:def:topological_subbase/order} \hyperref[def:order_topology]{Order topologies} have a base \eqref{eq:def:order_topology/base} constructed from the subbase \eqref{eq:def:order_topology/subbase}.
  \end{thmenum}
\end{example}

\begin{proposition}\label{thm:topology_from_subbase}
  Fix a set \( X \) and an arbitrary family of subsets \( \mscrS \) of \( X \). Then the family
  \begin{equation}\label{eq:thm:topology_from_subbase/base}
    \mscrB \coloneqq \set*{ \bigcap \mscrS' \colon \mscrS' \T{is a finite nonempty subfamily of} \mscrS }
  \end{equation}
  of finite intersections of \( \mscrS \) generates via \fullref{thm:topology_from_base} the coarsest topology \( \mscrT \) containing \( \mscrS \).

  Furthermore, \( \mscrS \) is a \hyperref[def:topological_subbase]{subbase} of the topology. We say that the subbase \( \mscrS \) \term{generates} the topology. We make no claim that \( \mscrB \) is the smallest base containing \( \mscrS \) --- \fullref{ex:def:topological_base/not_closed_under_intersections} provides a counterexample.

  Unlike in \fullref{thm:filter_from_prefilter}, we require \ref{thm:topology_from_base/B1} to ensure that \( X \) will be in the final topology. An alternative would be to use the empty intersection convention discussed in \fullref{rem:subbase_and_empty_intersection}.
\end{proposition}
\begin{proof}
  \Fullref{thm:topology_from_base} already shows well-definedness and minimality of the topology with respect to the base. Minimality with respect to the subbase can be shown similarly.
\end{proof}

\begin{remark}\label{rem:subbase_and_empty_intersection}
  In the context of an ambient topological space, the intersection of empty sets discussed in \fullref{def:basic_set_operations/intersection} is safe to use. It can be used, for example, in \eqref{eq:thm:topology_from_subbase/base} to drop the requirement \ref{thm:topology_from_base/B1}, but it would also complicate \fullref{thm:global_base_to_local_subbase}.

  Nonetheless, we prefer not to use empty intersections because of possible ambiguity.
\end{remark}

\begin{proposition}\label{thm:lattice_of_topologies}
  The family \( \mfrakT \) of all \hyperref[def:topological_space]{topologies} on a set \( X \) is a \hyperref[def:semilattice/lattice]{complete bounded lattice} with respect to the ordering from \fullref{def:topology_ordering}. Explicitly:

  \begin{thmenum}
    \thmitem{thm:lattice_of_topologies/join} The \hyperref[def:semilattice/join]{join} of an arbitrary family \( \mfrakT' \) of topologies on \( X \) is the topology generated by the family \( \bigcup \mfrakT' \) via \fullref{thm:topology_from_subbase}.

    \thmitem{thm:lattice_of_topologies/top} The \hyperref[def:partially_ordered_set_extremal_points/top_and_bottom]{top element} is the finest topology, the \hyperref[def:discrete_topology]{discrete topology} on \( X \).

    \thmitem{thm:lattice_of_topologies/meet} The \hyperref[def:semilattice/meet]{meet} of an arbitrary family \( \mfrakT' \) of topologies on \( X \) is simply their intersection \( \bigcap \mfrakT' \).

    \thmitem{thm:lattice_of_topologies/bottom} The \hyperref[def:partially_ordered_set_extremal_points/top_and_bottom]{bottom element} is the coarsest topology, \hyperref[def:indiscrete_topology]{indiscrete topology} on \( X \).
  \end{thmenum}
\end{proposition}
\begin{proof}
  \SubProofOf{thm:lattice_of_topologies/join} Minimality follows from \fullref{thm:topology_from_subbase}.

  \SubProofOf{thm:lattice_of_topologies/top} Every topology is by definition a sublattice of the discrete topology.

  \SubProofOf{thm:lattice_of_topologies/meet} We have shown in \fullref{thm:boolean_algebra_of_subsets/meet} that \( \bigcap \mfrakT' \) the meet of \( \mfrakT' \) in the Boolean algebra of sets. It remains only to show that \( \bigcap \mfrakT' \) is a topology.

  \SubProofOf*{def:topological_space/O1} Every topology must contain the empty set and the space itself, hence so does the intersection of any family of topologies.

  \SubProofOf*{def:topological_space/O2} Any finite family of sets from \( \bigcap \mfrakT' \) belongs to every topology in \( \mfrakT' \), hence their intersection also belongs to every topology in \( \mfrakT' \).

  \SubProofOf*{def:topological_space/O3} Similarly, any family of sets from \( \bigcap \mfrakT' \) belongs to every topology in \( \mfrakT' \), hence their union also belongs to every topology in \( \mfrakT' \).

  \SubProofOf{thm:lattice_of_topologies/bottom} The intersection of all topologies on \( X \) is the initial sublattice.
\end{proof}

\begin{definition}\label{def:topological_local_base}\mcite[12]{Engelking1989}
  Fix a topological space \( (X, \mscrT) \) and a point \( x \in X \). We say that the family \( \mscrB(x) \) of open sets is a \term{local base} for \( \mscrT \) at \( x \) if it is a \hyperref[def:prefilter]{prefilter} of the \hyperref[def:neighborhood_system]{neighborhood filter} \( \mscrT(x) \). That is, if \( \mscrB(x) \) consists of neighborhoods of \( x \) and if every neighborhood of \( x \) has an open subset in \( \mscrB(x) \).
\end{definition}

\begin{proposition}\label{thm:global_base_to_local_base}
  For any \hyperref[def:topological_base]{base} \( \mscrB \), the subfamily \( \mscrB(x) \) of all members of \( \mscrB \) containing \( x \) is a \hyperref[def:topological_local_base]{local base} at \( x \).
\end{proposition}
\begin{proof}
  Let \( U \) be some neighborhood of \( x \). Then there exists some neighborhood \( V \) of \( x \) such that \( V \subseteq U \). Since \( U \) is in the neighborhood filter \( \mscrT(x) \) and \( V \) is in \( \mscrB(x) \), after generalizing on \( U \) it follows that \( \mscrB(x) \) is a prefilter for \( \mscrT(x) \).
\end{proof}

\begin{example}\label{ex:def:topological_local_base}
  We list several examples of \hyperref[def:topological_local_base]{topological local bases}:
  \begin{thmenum}
    \thmitem{ex:def:topological_local_base/sierpinski} The \hyperref[def:sierpinski_space]{Sierpinski space} has a local base \( \set{ \set{ \top } } \) at \( \top \) and \( \set{ \set{ \top, \bot } } \) at \( \bot \).

    \thmitem{ex:def:topological_local_base/discrete} \fullref{ex:def:topological_base/discrete} implies that, for the \hyperref[def:discrete_topology]{discrete topology} \( \pow(X) \) on a set \( X \), a simple local base is \( \mscrB(x) = \set{ \set{ x } } \) for every point \( x \).

    \thmitem{ex:def:topological_local_base/metric} For a \hyperref[def:metric_space]{metric space}, we have an explicit countable local base \eqref{def:metric_topology/integer_base} and an explicit uncountable local base \eqref{def:metric_topology/real_base}.
  \end{thmenum}
\end{example}

\begin{proposition}\label{thm:topology_from_local_base}\mcite[13]{Engelking1989}
  Let \( X \) be an arbitrary set and let
  \begin{equation*}
    \seq{ \mscrB(x) }_{x \in X}
  \end{equation*}
  be an indexed family of families of subsets of \( X \) that satisfies the conditions
  \begin{thmenum}
    \thmitem[thm:topology_from_local_base/BP1]{BP1} For every \( x \in X \), the family \( \mscrB(x) \) is nonempty every member \( U \in B(x) \) contains \( x \).
    \thmitem[thm:topology_from_local_base/BP2]{BP2} If \( x \in U \in B(y) \), then there exists some \( V \in B(x) \) such that \( V \subseteq U \).
    \thmitem[thm:topology_from_local_base/BP3]{BP3} For every pair \( U, V \in B(x) \), there exists some \( W \in B(x) \) such that \( W \subseteq U \cap V \).
  \end{thmenum}

  The family
  \begin{equation*}
    \mscrB \coloneqq \bigcup_{x \in X} \mscrB(x)
  \end{equation*}
  generates via \fullref{thm:topology_from_base} the coarsest topology \( \mscrT \) containing \( \mscrB(x) \) for every \( x \in X \).

  Furthermore, \( \mscrB(x) \) is a \hyperref[def:topological_local_base]{local base} for \( \mscrT \) at \( x \).
\end{proposition}
\begin{proof}
  \SubProof{Proof that \( \mscrB \) satisfies \fullref{thm:topology_from_base}}

  \SubProofOf*{thm:topology_from_base/B1} \ref{thm:topology_from_local_base/BP1} implies that \( x \in U \) for every \( U \in \mscrB(x) \), hence the union over \( x \) of all neighborhoods is \( X \).

  \SubProofOf*{thm:topology_from_base/B2} Let \( U, V \in \mscrB \). Suppose that \( U \in \mscrB(u) \) and \( V \in \mscrB(v) \).

  If \( U \cap V \) is empty, \ref{thm:topology_from_base/B2} is vacuously satisfied.

  Otherwise, let \( w \in U \cap V \). \ref{thm:topology_from_local_base/BP2} implies that there exist sets \( U' \in \mscrB(w) \) and \( V' \in \mscrB(w) \) such that \( U' \subseteq U \) and \( V' \subseteq V \). \ref{thm:topology_from_local_base/BP3} implies that there exists some \( W \in \mscrB(w) \) such that
  \begin{equation*}
    W \subseteq U' \cap V' \subseteq U \cap V.
  \end{equation*}

  \SubProof{Proof of minimality of \( \mscrT \)} The proof is similar to that in \fullref{thm:topology_from_base}.

  \SubProof{Proof that \( \mscrB(x) \) is a local base} Consider the \hyperref[def:neighborhood_system]{neighborhood filter} \( \mscrT(x) \). Let \( U \in \mscrT(x) \).

  By construction, \( U \) is a union of some nonempty subfamily \( \mscrB' \subseteq \mscrB \). Since \( U \) contains \( x \), then there exists some nonempty subfamily \( \mscrB^\dprime \) of \( \mscrB' \) of points containing \( x \). \ref{thm:topology_from_base/B1} implies that \( \mscrB^\dprime \) is a subset of \( \mscrB(x) \).

  Then \( V \subseteq U \) for every set \( V \) in \( \mscrB^\dprime \subseteq \mscrB(x) \).
\end{proof}

\begin{proposition}\label{thm:local_base_can_generate_topology}
  Every family of \hyperref[def:topological_local_base]{local bases} satisfies \ref{thm:topology_from_local_base/BP1} -- \ref{thm:topology_from_local_base/BP3}.
\end{proposition}
\begin{proof}
  Let \( (X, \mscrT) \) be a topological space and let \( \mscrB(x) \) be a local base for \( \mscrT \) at \( x \). As a \hyperref[def:prefilter]{prefilter} of \( \mscrT(x) \), \( \mscrB(x) \) is nonempty, consists of neighborhoods of \( x \) and every neighborhood of \( x \) has an open subset in \( \mscrB(x) \).

  \SubProofOf*{thm:topology_from_local_base/BP1} Since \( X \) is a neighborhood of every point, by definition, \( \mscrB(x) \) contains some neighborhood \( U_x \subseteq X \) of \( x \). Hence, \( \mscrB(x) \) is nonempty.

  Furthermore, as a subset of \( \mscrT(x) \), every set in \( \mscrB(x) \) contains \( x \).

  \SubProofOf*{thm:topology_from_local_base/BP2} Suppose that \( x \in U \in \mscrB(y) \).

  Then \( U \) is a neighborhood of \( x \), and thus \( \mscrB(x) \) contains some open subset \( V \) of \( U \).

  \SubProofOf*{thm:topology_from_local_base/BP3} If \( U, V \in \mscrB(x) \), then \( U \cap V \) is a neighborhood of \( x \) and hence \( \mscrB(x) \) contains some open subset \( W \) of \( U \cap V \).
\end{proof}

\begin{definition}\label{def:topological_space_character}\mcite[12]{Engelking1989}
  Fix a topological space \( (X, \mscrT) \). We define the \term{character} of the point \( x \in X \) as the \hyperref[def:cardinal]{cardinal}
  \begin{equation*}
    \chi(X, \mscrT, x) \coloneqq \min \set{ \card \mscrB(x) \given \mscrB(x) \T{is a local base for} \mscrT \T{at} x }.
  \end{equation*}

  We define the \term{character} of the entire space \( (X, \mscrT) \) as
  \begin{equation*}
    \chi(X, \mscrT) \coloneqq \sup \set{ \chi(x) \given x \in X }.
  \end{equation*}

  Note that the supremum of cardinals is well-defined as a consequence of \fullref{thm:union_of_set_of_cardinals}.

  We simply write \( \chi(X) \) when the topology is clear from the context.

  Spaces for which \( \chi(X) \leq \hyperref[def:aleph_hierarchy]{\aleph_0} \) are said to be \term{first-countable}.

  See \fullref{thm:topological_space_countability} for how weight relates to other measurements of the space's \enquote{size}.
\end{definition}

\begin{example}\label{ex:def:topological_space_character}
  We list several examples of \hyperref[def:topological_space_character]{topological space characters}:
  \begin{thmenum}
    \thmitem{ex:def:topological_space_character/indiscrete} We have shown in \fullref{ex:def:topological_base/indiscrete} that \( \set{ X } \) is the unique base of the indiscrete topology on a nonempty set \( X \) that is not the topology itself. Hence, the character of each point is \( 1 \), and the character of the space itself is also \( 1 \).

    \thmitem{ex:def:topological_space_character/sierpinski} \Fullref{ex:def:topological_local_base/sierpinski} implies that the character of both points in the Sierpinski space is \( 1 \), and thus character of the entire space is \( 1 \).

    \thmitem{ex:def:topological_space_character/discrete} Similarly, \fullref{ex:def:topological_local_base/discrete} implies that the character of each point with respect to the discrete topology is \( 1 \), and hence the character of the space itself is \( 1 \).

    \thmitem{ex:def:topological_space_character/metric} \Fullref{thm:def:metric_topology/first_countable} shows that the character of a \hyperref[def:metric_space]{metric space} is \( \aleph_0 \).
  \end{thmenum}
\end{example}

\begin{definition}\label{def:topological_local_subbase}\mimprovised
  Fix a topological space \( (X, \mscrT) \) and a point \( x \in X \). We say that the family \( \mscrS(x) \) of open sets is a \term{local subbase} for \( \mscrT \) at \( x \) if it is a \hyperref[def:filter_subbase]{filter subbase} of the \hyperref[def:neighborhood_system]{neighborhood filter} \( \mscrT(x) \).
\end{definition}

\begin{proposition}\label{thm:global_base_to_local_subbase}
  For any \hyperref[def:topological_subbase]{subbase} \( \mscrS \), the subfamily \( \mscrS(x) \) of all members of \( \mscrS \) containing \( x \) is a \hyperref[def:topological_local_subbase]{local subbase} at \( x \).
\end{proposition}
\begin{proof}
  Follows from \fullref{thm:global_base_to_local_subbase} once we take finite intersections of \( \mscrS \) and \( \mscrS(x) \).
\end{proof}

\begin{proposition}\label{thm:topology_from_local_subbase}
  Let \( X \) be an arbitrary set and let
  \begin{equation*}
    \seq{ \mscrS(x) }_{x \in X}
  \end{equation*}
  be an indexed family of families of subsets of \( X \) that satisfies the condition \ref{thm:topology_from_local_base/BP1} and \ref{thm:topology_from_local_base/BP2}.

  For each \( x \in X \), define
  \begin{equation*}
    \mscrB(x) \coloneqq \set*{ \bigcap \mscrS' \given \mscrS' \T{is a finite nonempty subfamily of} \mscrS(x) }.
  \end{equation*}

  Then the family \( \seq{ \mscrB(x) }_{x \in X} \) generates via \fullref{thm:topology_from_local_base} the coarsest topology \( \mscrT \) containing \( \mscrS(x) \) for every \( x \in X \).

  Furthermore, \( \mscrB(x) \) is a \hyperref[def:topological_local_base]{local base} for \( \mscrT \) at \( x \).
\end{proposition}
\begin{proof}
  \SubProof{Proof that \( \mscrB \) satisfies \fullref{thm:topology_from_local_base}} We require \ref{thm:topology_from_local_base/BP1} and \ref{thm:topology_from_local_base/BP2} to hold, hence it remains only to show \ref{thm:topology_from_local_base/BP3}.

  Let \( U, V \in \mscrB(x) \). Both are intersections of finitely many elements of \( \mscrS(x) \), hence \( U \cap V \) also is. Then \( U \cap V \in \mscrB(x) \).

  \SubProof{Proof of minimality of \( \mscrT \)} The proof is similar to that in \fullref{thm:topology_from_subbase}.

  \SubProof{Proof that \( \mscrB(x) \) is a local subbase} Follows from \fullref{thm:topology_from_local_base} by noting that every member of \( \mscrB(x) \) is a finite intersection of members of \( \mscrS(x) \).
\end{proof}

\begin{proposition}\label{thm:local_subbase_can_generate_topology}
  Every family of \hyperref[def:topological_local_subbase]{local subbases} satisfies \ref{thm:topology_from_local_base/BP1} and \ref{thm:topology_from_local_base/BP2}.
\end{proposition}
\begin{proof}
  Both properties are proven in \fullref{thm:local_base_can_generate_topology} and the proof works here.
\end{proof}

\begin{definition}\label{def:topological_closure_operator}\mcite[thm. 1.1.6]{Engelking1989}
  Let \( (X, \mscrT) \) be a topological space. Define the \term{topological closure} of \( A \) as the intersection of all closed sets containing \( A \):
  \begin{equation*}
    \begin{aligned}
      &\cl: \pow(X) \to \pow(X) \\
      &\cl(A) \coloneqq \bigcap \set{ X \setminus U \given U \in \mscrT \T{and} A \subseteq X \setminus U }.
    \end{aligned}
  \end{equation*}

  As an intersection of closed set, \( \cl(A) \) is closed. It is actually the smallest closed set containing \( A \) because it is a subset of every closed set containing \( A \). In particular, the \hyperref[def:fixed_point]{fixed points} of the operator are the closed sets of \( \mscrT \).

  Furthermore, it is an \hyperref[def:abstract_closure_operator]{abstract closure operator}.
\end{definition}
\begin{defproof}
  \SubProofOf[def:extensive_function]{extensivity} Since \( A \) belongs to any closed superset, it follows that \( A \subseteq \cl(A) \).

  \SubProofOf[def:magma/idempotent]{idempotence} Since \( \cl(A) \) is the smallest closed set containing \( \cl(A) \), we conclude that \( \cl(\cl(A)) = \cl(A) \).

  \SubProofOf[eq:def:partially_ordered_set/homomorphism/nonstrict]{monotonicity} If \( A \subseteq B \), then every closed superset of \( B \) is also a closed superset of \( A \). Hence, \( \cl(A) \subseteq \cl(B) \).
\end{defproof}

\begin{proposition}\label{thm:topology_from_closure_operator}
  Let \( X \) be an arbitrary set and let \( \cl: \pow(X) \to \pow(X) \) be a function that satisfies \term{Kuratowski's axioms}:
  \begin{thmenum}
    \thmitem[thm:topology_from_closure_operator/CO1]{CO1} \( \cl(\varnothing) = \varnothing \),
    \thmitem[thm:topology_from_closure_operator/CO2]{CO2} It is \hyperref[def:extensive_function]{extensive}: \( A \subseteq \cl(A) \),
    \thmitem[thm:topology_from_closure_operator/CO3]{CO3} It preserves finite unions: \( \cl(A \cup B) = \cl(A) \cup \cl(B) \),
    \thmitem[thm:topology_from_closure_operator/CO4]{CO4} It is \hyperref[def:magma/idempotent]{idempotent}: \( \cl(\cl(A)) = \cl(A) \).
  \end{thmenum}

  Then the family
  \begin{equation*}
    \mscrF \coloneqq \set{ A \subseteq X \given A = \cl(A) }
  \end{equation*}
  generates via \fullref{thm:topology_from_closed_sets} the topology
  \begin{equation*}
    \mscrT \coloneqq \set{ X \setminus F \given F \in \mscrF }
  \end{equation*}
  for which \( \mscrF \) is the family of closed sets.

  Furthermore, \( \cl \), is the \hyperref[def:topological_closure_operator]{topological closure operator} on \( (X, \mscrT) \).
\end{proposition}
\begin{proof}
  \SubProofOf[eq:def:partially_ordered_set/homomorphism/nonstrict]{monotonicity} \ref{thm:topology_from_closure_operator/CO3} implies that, if \( A \subseteq B \), then
  \begin{equation*}
    \cl(B) = \cl(A \cup B \setminus A) = \cl(A) \cup \cl(B \setminus A).
  \end{equation*}

  Hence, \( \cl(A) \subseteq \cl(B) \).

  \SubProof{Proof that \( \mscrF \) satisfies \fullref{thm:topology_from_closed_sets}}

  \SubProofOf*{thm:topology_from_closed_sets/C1} \ref{thm:topology_from_closure_operator/CO1} implies that \( \varnothing \in \mscrF \). \ref{thm:topology_from_closure_operator/CO2} implies that \( X \subseteq \cl(X) \), hence \( X = \cl(X) \) and thus \( X \in \mscrF \).

  \SubProofOf*{thm:topology_from_closed_sets/C2} \ref{thm:topology_from_closure_operator/CO3} implies that, if \( F, G \in \mscrF \), then
  \begin{equation*}
    F \cup G = \cl(F) \cup \cl(G) = \cl(F \cup G),
  \end{equation*}
  and thus \( F \cup G \in \mscrF \).

  \SubProofOf*{thm:topology_from_closed_sets/C3} Let \( \mscrF' \subseteq \mscrF \). \ref{thm:topology_from_closure_operator/CO2} implies that
  \begin{equation*}
    \bigcap \mscrF' \subseteq \cl\parens*{ \bigcap \mscrF' }.
  \end{equation*}

  Conversely, monotonicity implies that, for any \( F_0 \in \mscrF' \),
  \begin{equation*}
    \cl\parens*{ \bigcap \mscrF' }
    \subseteq
    \cl(F_0)
    =
    F_0.
  \end{equation*}

  Therefore,
  \begin{equation*}
    \cl\parens*{ \bigcap \mscrF' } \subseteq \bigcap \mscrF'.
  \end{equation*}

  \SubProof{Proof that \( \cl \) is the topological closure operator} Fix a set \( A \) and let \( \mscrF' \subset \mscrF \) be the family of closed sets containing \( A \). Then
  \begin{equation*}
    \cl(A) \subseteq \cl\parens*{ \bigcap \mscrF' } = \bigcap \mscrF'.
  \end{equation*}

  \ref{thm:topology_from_closure_operator/CO2} implies that \( A \subseteq \cl(A) \). \ref{thm:topology_from_closure_operator/CO4} implies that \( \cl(A) \) is a closet set. Hence, \( \cl(A) \) belongs to \( \mscrF' \) and
  \begin{equation*}
    \bigcap \mscrF' \subseteq \cl(A).
  \end{equation*}

  We conclude that \( \cl(A) \) is the closure operator for the induced topology.
\end{proof}

\begin{proposition}\label{thm:topological_closure_operator_can_generate_topology}
  Every \hyperref[def:topological_closure_operator]{topological closure operator} satisfies \ref{thm:topology_from_closure_operator/CO1} -- \ref{thm:topology_from_closure_operator/CO4}.
\end{proposition}
\begin{proof}
  \SubProofOf{thm:topology_from_closure_operator/CO1} Since \( \varnothing \) is closed, \( \cl(\varnothing) = \varnothing \).
  \SubProofOf{thm:topology_from_closure_operator/CO2} Already proven in \fullref{def:topological_closure_operator}.
  \SubProofOf{thm:topology_from_closure_operator/CO3} Fix sets \( A \) and \( B \). Note that \( \cl(A \cup B) \) is a closed set containing both \( A \) and \( B \), hence
  \begin{align*}
    \cl(A) \subseteq \cl(A \cup B)
    &&
    \cl(B) \subseteq \cl(A \cup B).
  \end{align*}

  Then
  \begin{equation*}
    \cl(A) \cup \cl(B) \subseteq \cl(A \cup B).
  \end{equation*}

  Conversely, \( \cl(A) \cup \cl(B) \) is a closed set containing both \( A \) and \( B \), and hence \( A \cup B \), thus
  \begin{equation*}
    \cl(A \cup B) \subseteq \cl(A) \cup \cl(B).
  \end{equation*}

  We conclude that \( \cl \) preserves finite unions.

  \SubProofOf{thm:topology_from_closure_operator/CO4} Already proven in \fullref{def:topological_closure_operator}.
\end{proof}

\begin{definition}\label{def:topological_interior_operator}\mcite[thm. 1.1.6]{Engelking1989}
  Let \( (X, \mscrT) \) be a topological space. Define the \term{interior} of \( A \) as the union of all open sets contained in \( A \):
  \begin{equation*}
    \begin{aligned}
      &\Int: \pow(X) \to \pow(X) \\
      &\Int(A) \coloneqq \bigcup \set{ U \given U \in \mscrT \T{and} U \subseteq A }.
    \end{aligned}
  \end{equation*}

  As a union of open sets, \( \Int(A) \) is open. It is actually the largest open set containing \( A \) because it is a superset of every closed set contained in \( A \). In particular, the \hyperref[def:fixed_point]{fixed points} of the operator are the open sets of \( \mscrT \).

  Points that belong to \( \Int(A) \) are called \term{interior points} of \( A \). \Fullref{thm:properties_via_bases/interior} provides a characterization of interior points.

  The interior is not an \hyperref[def:abstract_closure_operator]{abstract closure operator} --- it is \hyperref[eq:def:partially_ordered_set/homomorphism/nonstrict]{monotone} and \hyperref[def:magma/idempotent]{idempotent}, but \hyperref[def:extensive_function]{anti-extensive} rather than \hyperref[def:extensive_function]{extensive}. It does have a deep relationship with the \hyperref[def:topological_closure_operator]{topological closure operator} -- see \fullref{thm:interior_closure_complement}.
\end{definition}
\begin{defproof}
  \SubProofOf[def:extensive_function]{anti-extensivity} Since \( A \) contains any open subset, it follows that \( \Int(A) \subseteq A \).

  \SubProofOf[def:magma/idempotent]{idempotence} Since \( \Int(A) \) is the largest an open set contained in \( \Int(A) \), we conclude that \( \Int(\Int(A)) = \Int(A) \)

  \SubProofOf[eq:def:partially_ordered_set/homomorphism/nonstrict]{monotonicity} If \( A \subseteq B \), then every open subset of \( A \) is also an open subset of \( B \). Hence, \( \Int(A) \subseteq \Int(B) \).
\end{defproof}

\begin{proposition}\label{thm:interior_closure_complement}
  The \hyperref[def:topological_closure_operator]{topological closure operator} and the \hyperref[def:topological_interior_operator]{topological interior operator} on the same space are related as follows:
  \begin{align}
    X \setminus \Int(A) = \cl(X \setminus A)
    &&
    X \setminus \cl(A) = \Int(X \setminus A)
    \label{eq:thm:interior_closure_complement}
  \end{align}
\end{proposition}
\begin{proof}
  We have
  \begin{balign*}
    X \setminus \Int(A)
    &=
    X \setminus \bigcup \set[\Big]{ U \given* U \in \mscrT \T{and} U \subseteq A }
    \reloset {\eqref{eq:thm:de_morgans_laws_for_sets/complement_of_union}} = \\ &=
    \bigcap \set[\Big]{ X \setminus U \given* U \in \mscrT \T{and} U \subseteq A }
    = \\ &=
    \bigcap \set[\Big]{ X \setminus U \given* U \in \mscrT \T{and} X \setminus A \subseteq X \setminus U }
    = \\ &=
    \cl(X \setminus A).
  \end{balign*}

  The other equality is obtained by noting that
  \begin{equation*}
    X \setminus \cl(A)
    =
    X \setminus (X \setminus \Int(A))
    \reloset {\ref{thm:set_difference/double_difference}} =
    \Int(A).
  \end{equation*}
\end{proof}

\begin{proposition}\label{thm:topology_from_interior_operator}
  Let \( X \) be an arbitrary set and let \( \Int: \pow(X) \to \pow(X) \) be a function that satisfies the conditions:
  \begin{thmenum}
    \thmitem[thm:topology_from_interior_operator/IO1]{IO1} \( \Int(X) = X \)
    \thmitem[thm:topology_from_interior_operator/IO2]{IO2} It is \hyperref[def:extensive_function]{anti-extensive}: \( \Int(A) \subseteq A \),
    \thmitem[thm:topology_from_interior_operator/IO3]{IO3} It preserves finite intersections: \( \Int(A \cap B) = \Int(A) \cap \Int(B) \),
    \thmitem[thm:topology_from_interior_operator/IO4]{IO4} It is \hyperref[def:magma/idempotent]{idempotent}: \( \Int(\Int(A)) = \Int(A) \).
  \end{thmenum}

  Then the family
  \begin{equation*}
    \mscrT \coloneqq \set{ A \subseteq X \given A = \cl(A) }
  \end{equation*}
  is a topology on \( X \).

  Furthermore, \( \Int \), is the \hyperref[def:topological_interior_operator]{topological interior operator} on \( (X, \mscrT) \).
\end{proposition}
\begin{proof}
  Note that \( \cl(A) \coloneqq X \setminus \Int(A) \) satisfies \ref{thm:topology_from_interior_operator/CO1} -- \ref{thm:topology_from_interior_operator/CO4}. Then \( \mscrT \) is a topology, \( \cl(A) \) is the closure operator for \( \mscrT \), and \fullref{thm:interior_closure_complement} implies that \( \Int(A) \) is the interior operator for \( \mscrT \).
\end{proof}

\begin{proposition}\label{thm:topological_interior_operator_can_generate_topology}
  Every \hyperref[def:topological_interior_operator]{topological interior operator} satisfies \ref{thm:topology_from_interior_operator/IO1} -- \ref{thm:topology_from_interior_operator/IO4}.
\end{proposition}
\begin{proof}
  Follows from \fullref{thm:interior_closure_complement} and \fullref{thm:topological_closure_operator_can_generate_topology}.
\end{proof}

\begin{definition}\label{def:topological_boundary_operator}
  Let \( (X, \mscrT) \) be a topological space. Define the \term{boundary operator}
  \begin{equation*}
    \begin{aligned}
      &\fr: \pow(X) \to \pow(X), \\
      &\fr(A) \coloneqq \cl(A) \setminus \Int(A).
    \end{aligned}
  \end{equation*}

  Points that belong to \( \fr(A) \) are called \term{boundary points} of \( A \). \Fullref{thm:properties_via_bases/boundary} provides a characterization of boundary points.
\end{definition}

\begin{proposition}\label{thm:def:topological_boundary_operator}
  The \hyperref[def:topological_boundary_operator]{topological boundary} has the following basic properties:
  \begin{thmenum}
    \thmitem{thm:def:topological_boundary_operator/intersection} \( \fr(A) = \cl(A) \cap \cl(X \setminus A) \).
    \thmitem{thm:def:topological_boundary_operator/complement} \( \fr(A) = \fr(X \setminus A) \).
    \thmitem{thm:def:topological_boundary_operator/closed} \( \fr(A) \) is a closed set.
  \end{thmenum}
\end{proposition}
\begin{proof}
  \SubProofOf{thm:def:topological_boundary_operator/intersection} We have
  \begin{balign*}
    \fr(A)
    &=
    \cl(A) \setminus \Int(A)
    \reloset {\ref{thm:set_difference/intersection}} = \\ &=
    \cl(A) \cap (X \setminus \Int(A))
    \reloset {\ref{thm:interior_closure_complement}} = \\ &=
    \cl(A) \cap \cl(X \setminus A).
  \end{balign*}

  \SubProofOf{thm:def:topological_boundary_operator/complement} Follows from \fullref{thm:def:topological_boundary_operator/intersection}.

  \SubProofOf{thm:def:topological_boundary_operator/closed} \Fullref{thm:def:topological_boundary_operator/intersection} represents \( \fr(A) \) as the intersection of two closed sets. Hence, \( \fr(A) \) is itself closed.
\end{proof}

\begin{definition}\label{def:cluster_point}\mcite[24]{Engelking1989}
  Let \( (X, \mscrT) \) be a topological space. We say that the point \( x \in X \) is a \term{cluster point} or an \term{accumulation point} of the set \( A \subseteq X \) if \( x \in \cl(A \setminus \set{ x }) \). Note that it is not necessary for \( x \) to belong to \( A \).

  Compare this to the similar notion for nets --- \fullref{def:net_convergence/cluster}.

  Points of \( A \) that are not cluster points are said to be \term{isolated points} of \( A \).
\end{definition}

\begin{proposition}\label{thm:def:cluster_point}
  \hyperref[def:cluster_point]{Cluster points} and \hyperref[def:cluster_point]{isolated points} in \( (X, \mscrT) \) have the following basic properties:
  \begin{thmenum}
    \thmitem{thm:def:cluster_point/isolated} A point \( x \in A \) is isolated if and only if \( A \setminus \set{ x } \) is closed.

    \thmitem{thm:def:cluster_point/discrete} A topological space is \hyperref[def:discrete_space]{discrete} if and only if all of its points are \hyperref[def:cluster_point]{isolated}.

    This motivates the term \term{discrete set} of a set consisting only of isolated points.
  \end{thmenum}
\end{proposition}
\begin{proof}
  \SubProofOf{thm:def:cluster_point/isolated} Closed sets are fixed points of the closure operator.
  \SubProofOf{thm:def:cluster_point/discrete} \Fullref{thm:def:cluster_point/isolated} implies that \( x \) is isolated in \( X \) if and only if \( X \setminus \set{ x } \) is closed, i.e. if \( \set{ x } \) is open. \Fullref{ex:def:topological_base/discrete} shows that \( \set[\Big]{ \set{ x } } \) is a basis of a discrete space.
\end{proof}

\begin{example}\label{ex:def:cluster_point}
  We list several examples of \hyperref[def:cluster_point]{cluster points} of sets:
  \begin{thmenum}
    \thmitem{ex:def:cluster_point/sierpinski} In the \hyperref[def:sierpinski_space]{Sierpinski space}, \( \bot \) is an isolated point because
    \begin{equation*}
      \cl(\set{ \top, \bot } \setminus \set{ \bot }) = \cl(\set{ \top }) = \set{ \top },
    \end{equation*}
    while \( \top \) is a cluster point because
    \begin{equation*}
      \cl(\set{ \bot }) = \set{ \top, \bot }.
    \end{equation*}

    \thmitem{ex:def:cluster_point/euclidean} If \( A \) is a finite set in an \hyperref[def:metric_space]{metric space}, then \( A \) is a \hyperref[thm:def:cluster_point/discrete]{discrete set}. That is, all of its points are isolated. This follows from \fullref{thm:def:cluster_point/isolated} by noting that any finite set is closed (a metric space is \ref{def:separation_axioms/T1}).
  \end{thmenum}
\end{example}

\begin{definition}\label{def:derived_set}\mcite[25]{Engelking1989}
  We define the \term{derived set} of a point \( A \) as the set of all cluster points of \( A \). We will denote it via \( \drv(A) \).
\end{definition}

\begin{proposition}\label{thm:union_with_derived_set}
  For any set \( A \) in a topological space,
  \begin{equation*}
    A \cup \drv(A) = \cl(A).
  \end{equation*}
\end{proposition}
\begin{proof}
  Clearly \( A \subseteq \cl(A) \). Furthermore,
  \begin{equation*}
    \drv(A) \subseteq \bigcup_{x \in X} \cl(A \setminus \set{ x }) \subseteq \cl(A).
  \end{equation*}

  Hence,
  \begin{equation*}
    A \cup \drv(A) \subseteq \cl(A).
  \end{equation*}

  Now let \( x \in \cl(A) \). If \( x \not\in A \), then \( A = A \setminus \set{ x } \), hence \( x \in \cl(A) = \cl(A \setminus \set{ x }) \). That is, if \( x \in \cl(A) \), it is a point in \( A \) or a point of \( \drv(A) \) (or both). Therefore,
  \begin{equation*}
    \cl(A) \subseteq A \cup \drv(A).
  \end{equation*}
\end{proof}

\begin{corollary}\label{thm:cluster_point_characterization}
  A set is closed if and only if it contains all of its cluster points.

  Compare this result to \fullref{thm:limit_point_iff_in_closure}.
\end{corollary}
\begin{proof}
  \Fullref{thm:union_with_derived_set} implies that \( \cl(A) = A \cup \drv(A) \).

  If \( A \) is closed, then \( A = \cl(A) = A \cup \drv(A) \) and thus \( \drv(A) \) is a subset of \( A \).

  Conversely, if \( \drv(A) \subseteq A \), then \( \cl(A) = A \cup \drv(A) = A \), implying that \( A \) is closed.
\end{proof}

\begin{definition}\label{def:topologically_dense_set}\mcite[25]{Engelking1989}
  We say that the set \( A \) in a topological space is \term{dense} in \( B \) if \( \cl(A) = B \) and \term{nowhere dense} if \( \Int(\cl(A)) = \varnothing \).

  If we do not specify the set \( B \), for example if we say \enquote{\( A \) is dense}, we assume that \( B \) is the entire space.
\end{definition}

\begin{proposition}\label{thm:def:topologically_dense_set}
  \hyperref[def:topologically_dense_set]{Dense} and \hyperref[def:topologically_dense_set]{nowhere dense} sets have the following basic properties:
  \begin{thmenum}
    \thmitem{thm:def:topologically_dense_set/no_proper_closed_set} A set is dense if and only if no proper closed set contains it.
    \thmitem{thm:def:topologically_dense_set/dense_superset} A superset of a dense set is dense.
    \thmitem{thm:def:topologically_dense_set/nowhere_dense_subset} A subset of a nowhere dense set is nowhere dense.
    \thmitem{thm:def:topologically_dense_set/nowhere_dense_is_in_boundary} Nowhere dense sets are entirely contained in their boundaries.
    \thmitem{thm:def:topologically_dense_set/complement_dense} A set is nowhere dense if and only if the complement of its closure is dense.
  \end{thmenum}
\end{proposition}
\begin{proof}
  \SubProofOf{thm:def:topologically_dense_set/no_proper_closed_set} \( \cl(A) \) is the entire space if and only if it is the smallest closed set containing \( A \).

  \SubProofOf{thm:def:topologically_dense_set/dense_superset} The closure operator is monotone.

  \SubProofOf{thm:def:topologically_dense_set/nowhere_dense_subset} Let \( A \) be a nowhere dense set and let \( B \subseteq A \). Then
  \begin{equation*}
    \Int(\cl(B))
    \subseteq
    \Int(\cl(A))
    =
    \varnothing,
  \end{equation*}
  therefore \( B \) is also nowhere dense.

  \SubProofOf{thm:def:topologically_dense_set/nowhere_dense_is_in_boundary} If \( A \) is nowhere dense, its interior is empty. Then
  \begin{equation*}
    A \subseteq \cl(A) = \Int(A) \cup \fr(A) = \fr(A).
  \end{equation*}

  \SubProofOf{thm:def:topologically_dense_set/complement_dense} \Fullref{thm:interior_closure_complement} implies that
  \begin{equation*}
    \cl(X \setminus \cl(A)) = X \setminus \Int(\cl(A)).
  \end{equation*}

  Then \( \cl(X \setminus \cl(A)) = X \) if and only if \( \Int(\cl(A)) = \varnothing \). That is, the complement of \( \cl(A) \) is dense if and only if \( A \) is nowhere dense.
\end{proof}

\begin{example}\label{ex:def:topologically_dense_set}
  We list several examples of \hyperref[def:topologically_dense_set]{dense} and \hyperref[def:topologically_dense_set]{nowhere dense} sets:
  \begin{thmenum}
    \thmitem{ex:def:topologically_dense_set/discrete} Under the \hyperref[def:discrete_topology]{discrete topology}, no proper subset is dense because each of them is closed. There are no nonempty nowhere dense sets because \( \Int(\cl(A)) = A \) for every set \( A \).

    \thmitem{ex:def:topologically_dense_set/indiscrete} On the other end, under the \hyperref[def:indiscrete_topology]{indiscrete topology}, every nonempty subset is dense because none of them is closed except for the empty set. There are no nonempty nowhere dense sets because \( \Int(\cl(A)) \) is the entire space.

    \thmitem{ex:def:topologically_dense_set/metric} Every \hyperref[def:metric_space]{metric space} is dense in its \hyperref[thm:metric_space_completion]{completion} as shown in \fullref{thm:metric_space_is_dense_in_completion}.

    \thmitem{ex:def:topologically_dense_set/finite} Let \( A \) be an \hyperref[def:set_countability/at_most_countable]{at most countable} set in some \hyperref[def:euclidean_space]{Euclidean spaces}.

    Every set in the local base \eqref{def:metric_topology/real_base} is uncountable, hence, \( A \) does not contain any of them. Then \( A \) does not contain any open set. Since \( A \) is closed, \( A = \cl(A) \), and \( \Int(A) = \varnothing \) implies that \( \Int(\cl(A)) = \varnothing \). Therefore, \( A \) is nowhere dense.

    \Fullref{thm:properties_via_bases/nowhere_dense} generalizes similar reasoning to arbitrary topological spaces.
  \end{thmenum}
\end{example}

\begin{definition}\label{def:topological_space_density}\mcite[12]{Engelking1989}
  We define the \term{density} of the topological space \( (X, \mscrT) \) as the \hyperref[def:cardinal]{cardinal}
  \begin{equation*}
    d(X, \mscrT) \coloneqq \min \set{ \card D \given D \T{is dense} }.
  \end{equation*}

  We simply write \( d(X) \) when the topology is clear from the context.

  Spaces for which \( d(X) \leq \hyperref[def:aleph_hierarchy]{\aleph_0} \) are said to be \term{separable}.

  See \fullref{thm:topological_space_countability} for how weight relates to other measurements of the space's \enquote{size}.
\end{definition}

\begin{example}\label{ex:def:topological_space_density}
  We list several examples of \hyperref[def:topological_space_density]{topological space density}:
  \begin{thmenum}
    \thmitem{ex:def:topological_space_density/discrete} Under the \hyperref[def:discrete_topology]{discrete topology}, the density of a space equals its cardinality. Indeed, \fullref{ex:def:topologically_dense_set/discrete} implies that only the space itself is dense.

    Then the space is separable if and only if it is countable.

    \thmitem{ex:def:topological_space_density/indiscrete} On the other end, under the \hyperref[def:indiscrete_topology]{indiscrete topology}, \fullref{ex:def:topologically_dense_set/indiscrete} shows that every nonempty set is dense, hence the density of the space is \( 1 \).
  \end{thmenum}
\end{example}

\begin{proposition}\label{thm:topological_space_countability}
  For any topological space \( (X, \mscrT) \), the \hyperref[def:topological_space_weight]{weight} \( w(X) \) \hyperref[def:equinumerosity]{dominates} both the \hyperref[def:topological_space_character]{character} \( \chi(X) \) and the \hyperref[def:topological_space_density]{density} \( d(X) \).

  In particular, every \hyperref[def:topological_space_weight]{second-countable space} is \hyperref[def:topological_space_character]{first-countable} and \hyperref[def:topological_space_density]{separable}.
\end{proposition}
\begin{proof}
  \SubProof{Proof that \( \chi(X) \leq w(X) \)} Let \( \mscrB \) be \hyperref[def:topological_base]{base} of minimal cardinality \( w(X) \). Then \( \card \mscrB(x) \leq \card \mscrB \) for any
  \begin{equation*}
    \mscrB(x) = \set{ U \in \mscrB \given x \in U }.
  \end{equation*}

  \Fullref{thm:global_base_to_local_base} implies that \( \seq{ \mscrB(x) }_{x \in X} \) is a family of local bases. Then
  \begin{equation*}
    \chi(X, \mscrT, x) \leq \card \mscrB(x) \leq \card \mscrB = w(X)
  \end{equation*}
  and
  \begin{equation*}
    \chi(X, \mscrT) = \sup_{x \in X} \chi(X, \mscrT, x) \leq \sup_{x \in X} \card \mscrB(x) \leq \card \mscrB = w(X).
  \end{equation*}

  \SubProof{Proof that \( d(X) \leq w(X) \)} The \hyperref[def:zfc/choice]{axiom of choice} gives us a \hyperref[def:choice_function]{choice function}
  \begin{equation*}
    c: \mscrB \setminus \set{ \varnothing } \to X.
  \end{equation*}

  Let \( D \) be the image of this function. Clearly
  \begin{equation*}
    \card D = \card \mscrB \setminus \set{ \varnothing } \leq \card \mscrB = w(X).
  \end{equation*}

  We will show that \( D \) is \hyperref[def:topologically_dense_set]{dense} in \( X \). Indeed, if \( F \) is a proper closed set, then \( X \setminus F \) is open, hence it contains a set \( U \) from \( \mscrB \). The point \( c(U) \) then belongs to \( D \), implying that \( F \) does not belong to \( D \). The only closed set containing \( D \) is \( X \) itself, and \fullref{thm:def:topologically_dense_set/no_proper_closed_set} implies that \( D \) is dense.

  Therefore,
  \begin{equation*}
    d(X) \leq \card D \leq w(X).
  \end{equation*}
\end{proof}

\begin{proposition}\label{thm:properties_via_bases}
  We present characterizations of several definitions entirely via \hyperref[def:topological_local_base]{local bases}. Let \( \set{ \mscrB(x) }_{x \in X} \) be a family of local bases of the space \( (X, \mscrT) \).

  \begin{thmenum}
    \thmitem{thm:properties_via_bases/open} A set \( A \) is open if and only if, for every point \( x \in A \), there must exist a neighborhood in \( \mscrB(x) \) contained in \( A \).

    \thmitem{thm:properties_via_bases/boundary} \( x \) belongs to the \hyperref[def:topological_boundary_operator]{boundary} of \( A \) if and only if every neighborhood in \( \mscrB(x) \) intersects both \( A \) and \( X \setminus A \).

    \thmitem{thm:properties_via_bases/closure} \( x \) belongs to the \hyperref[def:topological_closure_operator]{closure} of \( A \) if and only if every neighborhood in \( \mscrB(x) \) intersects \( A \).

    \thmitem{thm:properties_via_bases/interior} \( x \) belongs to the \hyperref[def:topological_interior_operator]{interior} of \( A \) if and only if \( \mscrB(x) \) contains a set entirely in \( A \).

    \thmitem{thm:properties_via_bases/cluster} \( x \) is a \hyperref[def:cluster_point/cluster_point]{cluster point} of \( A \) if and only if every neighborhood in \( \mscrB(x) \) intersects \( A \setminus \set{ x } \).

    \thmitem{thm:properties_via_bases/isolated} \( x \) is an \hyperref[def:cluster_point/isolated_point]{isolated point} of \( A \) if and only \( \mscrB(x) \) contains a set \( U \) disjoint from \( A \setminus \set{ x } \).

    \thmitem{thm:properties_via_bases/dense} The set \( A \) is \hyperref[def:topologically_dense_set]{dense} in \( B \) if and only if, for every point \( x \in B \), every neighborhood in \( \mscrB(x) \) intersects \( A \).

    \thmitem{thm:properties_via_bases/nowhere_dense} The set \( A \) is \hyperref[def:topologically_dense_set]{nowhere dense} if and only if, for every point \( x \in A \), all members of \( \mscrB(x) \) are supersets of \( A \).

    This statement also holds of we consider \hyperref[def:topological_local_subbase]{local subbases} rather than local bases.
  \end{thmenum}
\end{proposition}
\begin{proof}
  \SubProofOf{thm:properties_via_bases/open}

  \SufficiencySubProof* Suppose that \( A \) is open and let \( x \in A \). Then \( A \) is a neighborhood of \( x \).

  Since \( \mscrB(x) \) is a \hyperref[def:prefilter]{prefilter} of \( \mscrT(x) \), there exists some neighborhood \( U \in \mscrB(x) \) such that \( U \subseteq A \).

  \NecessitySubProof* Suppose that, for every point \( x \in A \), there must exist a neighborhood in \( \mscrB(x) \) contained in \( A \). Let \( \seq{ U_x }_{x \in A} \) be a collection of such neighborhoods. Then
  \begin{equation*}
    A = \bigcup_{x \in A} U_x,
  \end{equation*}
  implying that \( A \) is open as a union of open sets.

  \SubProofOf{thm:properties_via_bases/boundary}

  \SufficiencySubProof* Let \( x \in \fr(A) \).

  Aiming for a contradiction, suppose that there is a neighborhood \( U \in \mscrB(x) \) that is disjoint from \( A \). Then \( A \subseteq X \setminus U \). Since \( X \setminus U \) is closed, it follows that \( \cl(A) \subseteq X \setminus U \). The point \( x \) is not in \( X \setminus U \), therefore \( x \not\in \cl(A) \). But this contradicts \fullref{thm:def:topological_boundary_operator/intersection} because we have assumed that \( x \in \fr(A) \), and \( \fr(A) \subseteq \cl(A) \).

  The obtained contradiction shows that every neighborhood in \( \mscrB(x) \) intersects \( A \).

  By passing to complements, we can reuse this to prove that every neighborhood of \( x \) intersects \( X \setminus A \) using \fullref{thm:interior_closure_complement}.

  \NecessitySubProof* Suppose that every neighborhood in \( \mscrB(x) \) intersects both \( A \) and \( X \setminus A \). \Fullref{thm:properties_via_bases/open} then implies that \( x \) belongs to neither \( \Int(A) \) nor \( \Int(X \setminus A) \). Therefore,
  \begin{equation*}
    x
    \in
    X \setminus \parens[\Big]{ \Int(A) \cup \Int(X \setminus A) }
    \reloset {\eqref{eq:thm:de_morgans_laws_for_sets/complement_of_union}} =
    \cl(X \setminus A) \cap \cl(A)
    =
    \fr(A).
  \end{equation*}

  \SubProofOf{thm:properties_via_bases/closure}
  \SufficiencySubProof* Let \( x \in \cl(A) \), and let \( U \in \mscrB(x) \).

  \Fullref{thm:def:topological_boundary_operator/intersection} implies that \( \cl(A) = A \cup \fr(A) \).
  \begin{itemize}
    \item If \( x \in A \), then \( x \in U \cap A \), and hence \( U \) intersects \( A \).
    \item If \( x \in \fr(A) \), then \fullref{thm:properties_via_bases/boundary} implies that \( U \) intersects \( A \).
  \end{itemize}

  Therefore, every neighborhood in \( \mscrB(x) \) intersects \( A \).

  \NecessitySubProof* Let \( x \in X \) and suppose that every neighborhood in \( \mscrB(x) \) intersects \( A \).

  \begin{itemize}
    \item If \( x \in A \), clearly \( x \in \cl(A) \).
    \item If \( x \in X \setminus A \), every neighborhood in \( \mscrB(x) \) intersects both \( A \) and \( X \setminus A \), and \fullref{thm:properties_via_bases/boundary} implies that \( x \in \fr(A) \).
  \end{itemize}

  In both cases, \( x \in \cl(A) \).

  \SubProofOf{thm:properties_via_bases/interior} \Fullref{thm:interior_closure_complement} implies that \( x \in \Int(A) \) if and only if \( x \in X \setminus \cl(X \setminus A) \). \Fullref{thm:properties_via_bases/closure} implies that \( x \in X \setminus \cl(X \setminus A) \) if and only if some neighborhood in \( \mscrB(x) \) is disjoint from \( X \setminus A \), i.e. entirely in \( A \).

  \SubProofOf{thm:properties_via_bases/cluster} Follows from \fullref{thm:properties_via_bases/closure}.

  \SubProofOf{thm:properties_via_bases/isolated} \hyperref[def:material_implication/inverse]{Inverse} of \fullref{thm:properties_via_bases/cluster}.

  \SubProofOf{thm:properties_via_bases/dense} Follows from \fullref{thm:properties_via_bases/closure}.

  \SubProofOf{thm:properties_via_bases/nowhere_dense}

  \SufficiencySubProof* Suppose that \( A \) is nowhere dense. Let \( x \in A \) and let \( U \) be a neighborhood in \( \mscrB(x) \). Since \( \Int(\cl(A)) = \varnothing \), \( U \) is not contained in \( \cl(A) \).

  Generalizing on \( U \), it follows that every neighborhood in \( \mscrB(x) \) is a superset of \( \cl(A) \).

  \NecessitySubProof* Suppose that, for every point \( x \in A \), all members of \( \mscrB(x) \) are supersets of \( \cl(A) \). Then \( \cl(A) \) contains no nonempty open sets. Therefore, its interior is empty.

  If \( \mscrB(x) \) is only a local subbase, then the result also follows because any intersection of members of \( \mscrB(x) \) is also a superset of \( \cl(A) \),
\end{proof}
