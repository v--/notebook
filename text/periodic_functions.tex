\section{Periodic functions}\label{sec:periodic_functions}

\paragraph{Periodic functions on monoids}

\begin{remark}\label{rem:periodic_functions_and_periods}
  Periodic functions are a ubiquitous concept that is unfortunately handled inconsistently in the literature. We try to provide a systematic approach by first defining the set of all periods in \fullref{def:periodic_function} and then discussing fundamental periods in \fullref{def:fundamental_function_period}.

  We list here some definitions by different authors:
  \begin{itemize}
    \item The quintessential (and unfortunately restrictive) definition in the context of Fourier series is to call \( f: \BbbR \to \BbbC \) periodic with period \( p \) if \( f(x + p) = f(x) \) for every real number \( x \). This definition is given by
    \incite[def. 5.1.1]{Tao2022AnalysisII},
    \incite[371]{Фихтенгольц1968ОсновыАнализаТом2} and
    \incite[299]{ИльинСадовничийСендов1987АнализТом2}.

    Tao requires the period to be nonzero.

    For Fourier series on finite groups, an analogous definition is given when \( f: \BbbZ \to \BbbC \) by \incite[def. 5.1.1]{Steinberg2012RepresentationTheory}.

    In the context of complex trigonometric functions, an analogous definition is given when \( f: \BbbC \to \BbbC \) by \incite[44]{Ahlfors1979ComplexAnalysis} and \incite[324]{Маркушевич1967АналитическиеФункцииТом2}. Mаrkushevich requires the period to be nonzero.

    \item \incite[157]{Apostol1976AnalyticNumberTheory} calls an \hyperref[def:arithmetic_function]{arithmetical function} \enquote{periodic} with \enquote{period} \( k \) if \( f(n + k) = f(n) \) for every integer \( n \). He then defines the \enquote{fundamental period} as the smallest positive period.

    \item In the context of \hyperref[def:orbit_periodicity]{periodicity of orbits} of \hyperref[def:dynamical_system]{dynamical systems}, \incite[88]{Müller2022HandbookOfDynamicsAndProbability} calls a \hi{nonconstant} real-valued function \( f \) periodic if \( f(x + p) = f(x + q) \) for some \( p \neq q \).

    He also discusses restrictions on the set of periods which ensure that it is an infinite cyclic group. The base period is then simply a generator of this group.

    We use the cyclicity when defining base periods, but also allow periodic functions without a base period --- see \fullref{ex:def:function_periodicity}.

    \item Again in the context of dynamical system orbits, \incite[def. 11]{GiuntiMazzola2012DynamicalSystemsOnMonoids} give a definition that reduces to a \hyperref[def:dynamical_system_trajectory]{trajectory} \( f: M \to X \), where \( M \) is a monoid, being periodic with period \( p \) if \( f(x + p) = f(x) \) for every \( x \) in \( M \).

    They also define a base period, which they call \enquote{the period} as opposed to \enquote{a period}, as a period such that every other period is its integer multiple. This is equivalent to our definition of a base period as a cyclic group generator.

    \item In the same context, \incite[2]{BrinStuck2002DynamicalSystems} give a definition that reduces to the trajectory \( f: \BbbR \to X \) being periodic with period \( p \) if \( f(t) = p \) for some nonzero moment \( t \). They define a \enquote{minimal period} as the smallest positive period (ignoring the subtleties of such a period not existing like in \fullref{ex:def:periodic_function/dirichlet}).
  \end{itemize}
\end{remark}

\begin{definition}\label{def:periodic_function}\mimprovised
  Fix a function \( f: M \to X \), where \( M \) is an \hi{infinite} \hyperref[con:additive_semigroup]{additive} \hyperref[def:monoid]{monoid} and \( X \) is an arbitrary set. Define the set
  \begin{equation*}
    P(f) \coloneqq \set{ p \in M \given \qforall* {x \in M} f(x + p) = f(x) }.
  \end{equation*}

  For every \hi{nonzero} element \( p \) of \( P(f) \), we say that \( f \) is \term[bg=периодична (функция) (\cite[318]{ИлинСадовничиСендов1989АнализТом2}), ru=периодичная (функция) (\cite[371]{Фихтенгольц1968ОсновыАнализаТом2}), en=periodic (function) (\cite[def. 5.1.1]{Tao2022AnalysisII})]{periodic} with \term[bg=период (\cite[318]{ИлинСадовничиСендов1989АнализТом2}), ru=период (\cite[371]{Фихтенгольц1968ОсновыАнализаТом2}), en=period (\cite[def. 8.3]{LidlNiederreiter1997FiniteFields}]{period} \( p \).
\end{definition}
\begin{comments}
  \item If \( M \) is finite, periodicity becomes a questionable concept. This becomes apparent when defining fundamental periods in \fullref{def:fundamental_function_period}.

  \item We focus on the set of all periods to provide a more systematic exposition; see \fullref{rem:periodic_functions_and_periods} for how some other authors handle periodicity.

  \item There is an intimately related notion of periodicity of points and orbits in a dynamical system. See \fullref{def:dynamical_system_periodicity}.
\end{comments}

\begin{proposition}\label{thm:monoid_of_function_periods}
  The set \( P(f) \) of \hyperref[def:periodic_function]{periods} of \( f: M \to X \) is an infinite submonoid of \( M \). If \( M \) is a group, then \( P(f) \) is a subgroup.
\end{proposition}
\begin{proof}
  First note that the neutral element \( 0 \) is always a period since \( x + 0 = x \).

  To show that \( P(f) \) is closed under addition, suppose that \( f(x + p) = f(x + q) = f(x) \) for every \( x \) in \( G \). Then
  \begin{equation*}
    f(x + (p + q)) = f((x + p) + q) = f(x + p) = f(x).
  \end{equation*}

  Since \( M \) is infinite, all positive integer multiples of \( p \) must be distinct. Hence, \( P(f) \) is infinite.

  Finally, if \( M \) is a group and if \( f(x + p) = f(x) \) for every \( x \), then
  \begin{equation*}
    f(x) = f(x - p + p) = f(x - p).
  \end{equation*}
\end{proof}

\begin{definition}\label{def:fundamental_function_period}\mimprovised
  Fix a \hyperref[def:periodic_function]{periodic function} \( f: M \to X \). We will present two cases in which we are able to pick a distinguished period from \( P(f) \); we will call this choice the \term[ru=основной период (\cite[102]{Маркушевич1967АналитическиеФункцииТом1}), en=fundamental period (\cite[157]{Apostol1976AnalyticNumberTheory})]{fundamental period} of \( f \).

  \begin{thmenum}
    \thmitem{def:fundamental_function_period/monoid} \Fullref{thm:monoid_of_function_periods} implies that \( P(f) \) is an infinite monoid. In case \( P(f) \) is a \hyperref[def:cyclic_monoid]{cyclic monoid}, by \fullref{thm:infinite_cyclic_monoid_generator}, it has a unique generator, which we designate as the base period.

    Note that this requires \( M \) itself not to be a group, which largely reduces to the cases where \( M \) is the additive monoid of natural numbers or nonnegative real numbers. Even so, a base period may not exist unless \( P(f) \) is a cyclic monoid --- see \fullref{ex:def:periodic_function/dirichlet}.

    \thmitem{def:fundamental_function_period/group} Suppose that \( M \) is the additive group of \hyperref[def:complex_numbers]{complex numbers} or a subgroup such as that of \hyperref[def:real_numbers]{real numbers}.

    Then, by \fullref{thm:monoid_of_function_periods}, \( P(f) \) is an infinite group. If it is a \hyperref[def:cyclic_group]{cyclic group}, by \fullref{thm:def:cyclic_group/infinite_generators}, it has two generators, \( p \) and \( -p \). We pick the one with the smaller \hyperref[def:complex_numbers_trigonometric_form]{complex argument}.

    This gives us the positive real number in case \( p \) is real, and the number with positive imaginary part in case \( p \) is purely imaginary.
  \end{thmenum}
\end{definition}
\begin{comments}
  \item We reuse this definition for the fundamental period of a point in a dynamical system. See \fullref{def:dynamical_system_fundamental_period}.
\end{comments}

\begin{proposition}\label{thm:def:fundamental_function_period}
  \hyperref[def:fundamental_function_period]{Fundamental periods} of \hyperref[def:periodic_function]{periodic functions} have the following basic properties:
  \begin{thmenum}
    \thmitem{thm:def:fundamental_function_period/sequence} If a \hyperref[def:sequence]{infinite sequence} or \hyperref[def:doubly_infinite_sequence]{doubly infinite sequence} is periodic, it has a fundamental period.
  \end{thmenum}
\end{proposition}
\begin{proof}
  \SubProofOf{thm:def:fundamental_function_period/sequence} Trivial.
\end{proof}

\begin{example}\label{ex:def:periodic_function}
  We list examples of \hyperref[def:periodic_function]{periodic functions}:
  \begin{thmenum}
    \thmitem{ex:def:periodic_function/constant} Any constant function \( f(x) = c \) is periodic with every possible period.

    More precisely, the set \( P(f) \) coincides with the monoid on which \( f \) is defined.

    In case \( f \) is defined on the real numbers, it cannot have a fundamental period because the additive group of \( \BbbR \) is not cyclic.

    On the other hand, if \( f \) is defined on the integers, since \( \BbbZ \) is cyclic, \fullref{def:fundamental_function_period/group} is satisfied. Out of the generators \( 1 \) and \( -1 \) of \( \BbbZ \), we pick \( 1 \) as a fundamental period because it has complex argument \( 0 \).

    \thmitem{ex:def:periodic_function/dirichlet}\mcite[88]{Müller2022HandbookOfDynamicsAndProbability} \hyperref[def:dirichlet_function]{Dirichlet's function} is periodic, and its periods are all rational numbers

    Indeed, denote the function by \( f \). Then, for every rational number \( r \), \( x + r \) is rational if \( x \) is rational and, by \fullref{thm:real_number_arithmetic/rational_irrational_sum}, it is irrational if \( x \) is irrational.

    Hence, for any real number \( x \), if \( f(x) = 0 \), then \( f(x + r) = 0 \), and if \( f(x) = 1 \), then \( f(x + r) = 1 \).

    The additive group of rational numbers is not cyclic, hence it cannot have a fundamental period.

    \thmitem{ex:def:periodic_function/exponential} \Fullref{thm:def:exponential_function/periodic} implies that the complex \hyperref[def:exponential_function]{exponential function} \( e^z \) is periodic and that its periods are integer multiples of \( 2i\pi \).

    Then the group of periods is cyclic, \fullref{def:fundamental_function_period/group} is satisfied, and thus the fundamental period is \( 2i\pi \) ---  it has complex argument \( \pi/2 \), unlike \( -2i\pi \) whose argument is \( 3\pi/2 \).

    \thmitem{ex:def:periodic_function/trigonometric} \Fullref{thm:trigonometric_function_period} implies that the complex \hyperref[def:trigonometric_functions]{trigonometric functions} \( \sin(z) \) and \( \cos(z) \) are periodic with periods that are integer multiples of \( 2\pi \).

    By the same reasoning, their fundamental period is \( 2\pi \) since it has complex argument \( 0 \).
  \end{thmenum}
\end{example}

\paragraph{Ultimately periodic sequences}

\begin{definition}\label{def:ultimately_periodic_sequence}\mimprovised
  We say that the \hyperref[def:sequence]{sequence} \( \seq{ x_n }_{n=0}^\infty \) is \term[en=ultimately periodic (\cite[def. 8.3]{LidlNiederreiter1997FiniteFields})]{ultimately periodic} with \term[en=preperiod (\cite[412]{LidlNiederreiter1997FiniteFields})]{preperiod} \( p_0 \) if \( p_0 \) is the smallest nonnegative integer such that the subsequence \( \seq{ x_n }_{n=p_0}^\infty \) is periodic in the sense of \fullref{def:periodic_function}.
\end{definition}
\begin{comments}
  \item \Fullref{thm:def:ultimately_periodic_sequence/preperiod_uniqueness} implies that all preperiods for a sequence coincide; thus, the preperiod is an intrinsic property of the sequence.

  \item \Fullref{thm:def:fundamental_function_period/sequence} implies that every ultimately periodic sequence has a fundamental period.

  \item We base our definition on \bycite[def. 8.3]{LidlNiederreiter1997FiniteFields}, but impose uniqueness on \( p_0 \) by requiring minimality. The latter is in line with \bycite[exerc. 3.1.6(a)]{Knuth1997ArtVol2}.

  \item This notion is related but distinct from \hyperref[def:dynamical_system_eventual_periodicity]{eventually periodic points} in a dynamical system. The two are linked via \fullref{thm:def:dynamical_system_eventual_periodicity/ultimately_periodic}.
\end{comments}

\begin{proposition}\label{thm:def:ultimately_periodic_sequence}
  \hyperref[def:ultimately_periodic_sequence]{Ultimately periodic sequences} have the following basic properties:
  \begin{thmenum}
    \thmitem{thm:def:ultimately_periodic_sequence/preperiod_uniqueness} If a sequence is ultimately periodic with preperiod \( p_0 \) and period \( p \), and if it is also ultimately periodic with preperiod \( q_0 \) and period \( q \), then \( p_0 = q_0 \).

    \thmitem{thm:def:ultimately_periodic_sequence/periodic} A sequence is periodic if and only if it is ultimately periodic with preperiod \( 0 \).
  \end{thmenum}
\end{proposition}
\begin{proof}
  Fix a sequence \( \seq{ x_n }_{n=0}^\infty \) that is ultimately periodic with preperiod \( p_0 \) and period \( p \).

  \SubProofOf{thm:def:ultimately_periodic_sequence/preperiod_uniqueness} Suppose that \( q_0 \) is also a preperiod with a corresponding period \( q \).

  Suppose first that \( p_0 \leq q_0 \). \Fullref{alg:integer_division} gives us integers \( a \) and \( b \) such that \( 0 \leq b < q_0 \)
  \begin{equation*}
    q_0 - p_0 = ap + b.
  \end{equation*}

  \Fullref{thm:alg:integer_division/quot_positivity} implies that \( a \geq 0 \). Thus,
  \begin{equation*}
    x_{q_0} = x_{p_0 + (q_0 - p_0)} = x_{p_0 + ap + b} = x_{p_0 + b}.
  \end{equation*}

  For every nonnegative integer \( m \), we have
  \begin{equation*}
    x_{q_0 + pm} = x_{p_0 + b + pm} = x_{p_0 + b},
  \end{equation*}
  thus \( p \) is a period for the preperiod \( q_0 \). By minimality of \( q_0 \), we conclude that it must equal \( p_0 \).

  Similarly, if \( p_0 \geq q_0 \), we conclude that \( p_0 = q_0 \).

  \SubProofOf{thm:def:ultimately_periodic_sequence/periodic} Trivial.
\end{proof}
