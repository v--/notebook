\section{Geometry}\label{sec:geometry}

Humans possess a strong intuition for visual information like drawings or diagrams. A drawing on a paper is only a medium for communicating ideas and data. \Fullref{fig:sec:geometry/figures} contains some highlighted curves that our mind maps to abstract geometric figures, without considering the size limitations of the page, the precision of the drawings or the thickness of the lines.

\begin{figure}[!ht]
  \centering
  \includegraphics{output/sec__geometry__figures.pdf}
  \caption{A \hyperref[def:triangle]{triangle}, a \hyperref[def:circle]{circle} and a \hyperref[def:affine_line]{line} in the \hyperref[def:euclidean_space]{Euclidean plane}.}\label{fig:sec:geometry/figures}
\end{figure}

Our goal is to introduce formalisms for these mental visualizations. An axiomatic approach for a theory of figures in the place and space was developed by Euclid in the third century BC and can be found in \cite{Fitzpatrick2008}. A few millennia later, several mathematicians proposed systems of axioms that fit the requirements of \hyperref[sec:mathematical_logic]{modern logical systems}. Tarski's system can be found in \cite{Tarski1959}. This is known today as \term{synthetic Euclidean geometry} and is mostly of theoretical interest.

An important distinction between ancient and modern geometry is the introduction of coordinates in the 17th century. Descartes' idea of coordinates connects problems of algebra and geometry in such a way that most of today's mathematics seamlessly switches between algebraic and geometric interpretations of the same problem. The study of classical Greek geometry in terms of coordinates is known as \term{analytic geometry}.

A modern interpretation if the ideas behind analytic geometry leads to \hyperref[def:affine_space]{affine spaces}, which we will discuss in \fullref{subsec:affine_spaces} and \fullref{subsec:convex_sets}, to Euclidean spaces discussed in \fullref{subsec:euclidean_spaces} and to the Euclidean plane discussed in \fullref{subsec:euclidean_plane}, \fullref{subsec:triangles} and \fullref{subsec:quadratic_plane_curves}. As part of our discussion of the \hyperref[def:euclidean_plane]{Euclidean plane}, we will briefly introduce modern concepts related to differential geometry in \fullref{subsec:parametric_curves} and algebraic geometry in \fullref{subsec:quadratic_plane_curves}.
