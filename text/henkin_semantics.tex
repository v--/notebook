\section{Henkin semantics}\label{sec:henkin_semantics}

\paragraph{Frames and interpretations}

\begin{definition}\label{def:hol_frame}\mimprovised
  Fix a \hyperref[def:hol_signature]{signature} \( \Sigma \) of \hyperref[def:higher_order_logic]{higher-order logic} and a \hyperref[con:truth_value_algebra]{truth value algebra} \( \BbbH \).

  A \term[en=frame (\cite[238]{Andrews2002Logic})]{frame} over \( \Sigma \) is an \hyperref[def:indexed_family]{family} \( \set{ U_\tau }_{\tau \in \op*{\BbbQ_\Sigma}} \) of nonempty sets, indexed by the \hyperref[def:quantifiable_type]{quantifiable types} over \( \Sigma \), and subject to some constraints. We call \( U_\tau \) the \term{universe} or \term{domain} of type \( \tau \).

  \begin{thmenum}
    \thmitem{def:hol_frame/propositions} For the type of propositions \( \syn\omicron \), we let \( U_{\syn\omicron} \coloneqq \BbbH \) since the values of propositions are intended to be truth values.

    \thmitem{def:hol_frame/sorts} For every sort \( S \), we allow \( U_S \) to range over any nonempty set.

    \thmitem{def:hol_frame/functions} For every arrow type \( \tau \synimplies \sigma \), we require \( U_{\tau \synimplies \sigma} \) to be a nonempty set of functions from \( U_\tau \) to \( U_\sigma \).
  \end{thmenum}
\end{definition}
\begin{comments}
  \item It is important that the universes in a frame are nonempty; see \cref{rem:predicate_logic_empty_domains}.

  \item We adapt Andrew's definition from \cite[238]{Andrews2002Logic} by allowing arbitrary truth value algebras and arbitrary sorts.

  We refer to the individual sets as \enquote{universes} rather than Andrews' term \enquote{domains}; see \cref{rem:model_theory_universe_terminology}.
\end{comments}

\begin{remark}\label{rem:model_theory_universe_terminology}
  When defining frames in \hyperref[def:higher_order_logic]{higher-order logic} in \cref{def:hol_frame} or structures in \hyperref[def:first_order_logic]{first-order logic} in \cref{def:first_order_structure}, we call the underlying sets \enquote{universes}. A popular alternative, especially for frames, is the word \enquote{domain}.

  \begin{itemize}
    \item \enquote{Universe} is used by \cite[83]{Hinman2005Logic} and \cite[59]{VanDalen2004LogicAndStructure} for first-order structures.

    \item \enquote{Domain} is used by
    \cite[84]{Henkin1950CompletenssInTheoryOfTypes},
    \cite[238]{Andrews2002Logic} and
    \cite[271]{Farmer2008STTVirtues}
    for frames, and by
    \cite[84]{Kleene2002Logic} and \cite[71]{КолмогоровДрагалин2006Логика} (as \enquote{область})
    for first-order structures.

    Andrews admits \enquote{universe} as a synonym for \enquote{domain} for first-order structures in \cite[115]{Andrews2002Logic}.

    \item \enquote{Носитель} (\enquote{bearer}, also translated as \enquote{support} when dealing with functions --- see \cref{def:function_support}) is used in some Russophone literature --- for example \cite[75]{ШеньВерещагин2017ЯзыкиИИсчисления} and \cite[def. 2.2.3]{Герасимов2011Вычислимость}.
  \end{itemize}
\end{remark}

\begin{example}\label{ex:hol_proof_nonempty_domain}
  Consider the \hyperref[def:hol_natural_deduction_proof_tree]{higher-order logic proof tree}
  \begin{equation*}
    \begin{prooftree}
      \hypo{ \qforall {x^\tau} \varphi }
      \infer1[\ref{inf:def:hol_quantifier_rules/terms/forall_elim}]{ \varphi[x^\tau \mapsto M^\tau] }
      \infer1[\ref{inf:def:hol_quantifier_rules/terms/exists_intro}]{ \qexists {x^\tau} \varphi }
    \end{prooftree}
  \end{equation*}

  The obvious problem is that \( \qforall {x^\tau} \varphi \) holds vacuously in an empty universe, but \( \qexists {x^\tau} \varphi \) is false.

  As stated, both rules are applicable with \( M^\tau = x^\tau \). But a variable is a placeholder for a value. One solution is to require \( M^\tau \) to refer to an actual value.

  \incite*[\S 5.1.10]{Mimram2020ProgramEqualsProof} suggests an unorthodox tweak for his formulation of first-order logic. In our formulation, the tweak would require us to modify \ref{inf:def:hol_quantifier_rules/terms/forall_elim} and \ref{inf:def:hol_quantifier_rules/terms/exists_intro} so that the free variables of the term \( M^\tau \) are free in the proof tree (in the sense of \cref{def:hol_natural_deduction_proof_tree/free_variables}). This would disallow \( M^\tau = x^\tau \), and instead require \( M^\tau \) to be a \( \muplambda \)-term not depending on \( x^\tau \).

  We will refrain from these tweaks, as well as from empty universe in general. See \cref{rem:predicate_logic_empty_domains} for a broader discussion.
\end{example}

\begin{remark}\label{rem:predicate_logic_empty_domains}
\end{remark}

\begin{remark}\label{rem:predicate_logic_sentences}
\end{remark}
