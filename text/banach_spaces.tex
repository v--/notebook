\subsection{Banach spaces}\label{subsec:banach_spaces}

\begin{definition}\label{def:normed_vector_space}
  \todo{Define}.
\end{definition}

\begin{definition}\label{def:banach_space}
  A \term{Banach space} is a \hyperref[def:norm]{normed} \hyperref[def:vector_space]{vector space} which is also a \hyperref[def:complete_metric_space]{complete metric spaces} with the metric induced by the \hyperref[def:norm_induced_metric]{norm}.
\end{definition}

\begin{definition}\label{def:topological_duality_pairing}
  Let \( M \) and \( N \) be left \( R \)-modules. A \term{duality pairing} \( \inprod \cdot \cdot: M \times N \to R \) is a \hyperref[def:degenerate_bilinear_form]{nondegenerate} bilinear form.

  The \term{canonical duality pairing} of a vector space \( V \) over \( F \) is
  \begin{balign*}
     & \inprod \cdot \cdot: V^* \times V \to F \\
     & \inprod {x^*} x \mapsto x^*(x).
  \end{balign*}
\end{definition}

\begin{example}\label{ex:noncomplete_normed_space}\mcite{MathCounterExamples:noncomplete_normed_space}
  Consider the polynomial \hyperref[def:polynomial_algebra]{algebra} \( \BbbR[x] \) as a vector space with the supremum norm. We will show that it is not complete. Define the sequence
  \begin{equation*}
    p_n(x) \coloneqq \sum_{k=0}^n \frac{x^k} {2^k}, n = 1, 2, \ldots
  \end{equation*}

  Then the limit of the sequence in \( C([0, 1]) \) is the power series
  \begin{equation*}
    \lim_{n \to \infty} p_n(x)
    =
    \sum_{k=0}^n \frac{x^k} {2^k}
    =
    \frac 2 {2 - x}.
  \end{equation*}

  Since \( \BbbR[x] \) is a subspace of \( C([0, 1]) \), we conclude that \( \BbbR[x] \) has fundamental sequence, but we just demonstrated that its limit is not in \( \BbbR[x] \).
\end{example}

\begin{definition}\label{def:dual_norm}
  Fix two nonempty Banach spaces \( (X, \norm{\cdot}_X) \) and \( (Y, \norm{\cdot}_Y) \). We define the \term{operator norm} \( \norm{\cdot}_{\hom(X, Y)} \) on \( \hom(X, Y) \) equivalently as
  \begin{thmenum}
    \thmitem{def:dual_norm/sup_unit_sphere}
    \begin{equation*}
      \norm{L}_{\hom(X, Y)} \coloneqq \sup_{\norm{x}_X = 1} \norm{Lx}_Y.
    \end{equation*}

    \thmitem{def:dual_norm/sup_unit_ball}
    \begin{equation*}
      \norm{L}_{\hom(X, Y)} \coloneqq \sup_{\norm{x}_X < 1} \norm{Lx}_Y.
    \end{equation*}

    \thmitem{def:dual_norm/sup_nonzero}
    \begin{equation*}
      \norm{L}_{\hom(X, Y)} \coloneqq \sup_{x \neq 0_X} \frac {\norm{Lx}_Y} {\norm{x}_X}.
    \end{equation*}

    \thmitem{def:dual_norm/inf}
    \begin{equation*}
      \norm{L}_{\hom(X, Y)} \coloneqq \inf \left\{ c \geq 0 \colon \norm{Lx}_Y \leq c \norm{x}_X \right\}.
    \end{equation*}
  \end{thmenum}

  In particular, this induces a norm on \( X^* \).
\end{definition}

\begin{definition}\label{def:banach_space_support_function}\mcite[exmpl. 3.2(a)]{Phelps1993})
  Let \( X \) be a Banach space.

  We define the \term{support function \( \sigma_{A^*} \) for the set of functionals \( A^* \subseteq X^* \)} by
  \begin{balign*}
     & \sigma_{A^*}: X \to \BbbR \cup \{ \infty \}                             \\
     & \sigma_{A^*}(x) \coloneqq \sup \{ \inprod {x^*} x \colon x^* \in A^* \}
  \end{balign*}

  and the \term{weak* support function \( \sigma^*_A \) for the set of points \( A \subseteq X \)} by
  \begin{balign*}
     & \sigma^*_A: X^* \to \BbbR \cup \{ \infty \}                          \\
     & \sigma^*_A(x^*) \coloneqq \sup \{ \inprod {x^*} x \colon x \in A \}.
  \end{balign*}
\end{definition}

\begin{definition}\label{def:banach_space_slice}\mcite[def. 2.17]{Phelps1993}
  Given a linear functional \( x^* \), a nonempty subset \( A \) of \( X \) and a \term{diameter} \( \alpha > 0 \), the value \( S(x^*, A, \alpha) \) is called a \term{slice} of \( A \), where
  \begin{balign*}
     & S: X^* \times \pow(X) \times \BbbR_{>0} \mapsto \pow(A)                                      \\
     & S(x^*, A, \alpha) \coloneqq \{ x \in A \colon \inprod {x^*} x > \sigma_A^*(x^*) - \alpha \}.
  \end{balign*}

  We define a weak* slice of \( A^* \subseteq X^* \) as \( S^*(x, A^*, \alpha) \), where
  \begin{balign*}
     & S^*: X \times \pow(X) \times \BbbR_{>0} \mapsto \pow(A)                                            \\
     & S^*(x, A^*, \alpha) \coloneqq \{ x^* \in A^* \colon \inprod {x^*} x > \sigma_{A^*}(x) - \alpha \}.
  \end{balign*}

  If we need to make the underlying space explicit, we will use \( S_X(x^*, A, \alpha) \) and \( S_X^*(x, A^*, \alpha) \).
\end{definition}

\begin{proposition}
  If \( \{ a_k \}_{k=1}^\infty \) and \( \{ b_k \}_{k=1}^\infty \) are sequences a in a \hyperref[def:banach_space]{Banach} \hyperref[def:algebra_over_semiring]{algebra} \( X \), that converge to \( a \) and \( b \), correspondingly, then \( a_k b_k \to a b \).
\end{proposition}
\begin{proof}
  Let \( \delta > 0 \) and let \( k_0 \) be an index such that for \( k \geq k_0 \) we have both \( \norm{a - a_k} < \delta \) and \( \norm{b - b_k} < \delta \). Then
  \begin{balign*}
    ab - a_k b_k
     & =
    (ab - a b_k) + (a b_k - a_k b) + (a_k b - a_k b_k)
    =    \\ &=
    a (b - b_k) + (a b_k - ab + ab - a_k b) + (-a_k)(b_k - b)
    =    \\ &=
    a (b - b_k) + a \underbrace{(b_k - b)} + (a - a_k) b + (-a_k)\underbrace{(b_k - b)}
    =    \\ &=
    a \underbrace{(b - b_k)}_{\in B(0, \delta)} + \underbrace{(a - a_k)}_{\in B(0, \delta)} \underbrace{(b_k - b)}_{\in B(0, \delta)} + \underbrace{(a - a_k)}_{\in B(0, \delta)} b.
  \end{balign*}

  Therefore, \( \norm{ab - a_k b_k} < \delta^2 + \norm{a + b} \delta \). If we require \( \delta \) to be strictly less than \( 1 \), we obtain \( \delta^2 < \delta \) and \( \norm{ab - a_k b_k} < (1 + \norm{a + b}) \delta \).

  Given an arbitrary \( \varepsilon > 0 \), we can choose \( \delta = \tfrac {\min \{\varepsilon, 1 \}} {1 + \norm{a + b}} \) in order to have \( \norm{ab - a_k b_k} < \varepsilon \) for some large enough \( k \).

  Therefore, \( a_k b_k \to a b \).
\end{proof}

\begin{definition}\label{def:banach_algebra}
  \todo{Define}.
\end{definition}
