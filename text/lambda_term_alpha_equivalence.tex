\section{Lambda term alpha equivalence}\label{sec:lambda_term_alpha_equivalence}

\paragraph{\( \alpha \)-equivalence}

\begin{definition}\label{def:lambda_term_alpha_equivalence}\mimprovised
  In accordance with \fullref{thm:recursively_defined_relations}, we will define an equivalence relation between \( \synlambda \)-terms:

  \begin{paracol}{2}
    \begin{leftcolumn}
      \ParacolAlignmentHack
      \begin{equation*}\taglabel[\ensuremath{ \logic{Atom}_\alpha }]{inf:def:lambda_term_alpha_equivalence/atom}
        \begin{prooftree}
          \hypo{ M \in \op*{Atom} }
          \infer1[\ref{inf:def:lambda_term_alpha_equivalence/atom}]{ M \aequiv M }
        \end{prooftree}
      \end{equation*}
    \end{leftcolumn}

    \begin{rightcolumn}
      \ParacolAlignmentHack
      \begin{equation*}\taglabel[\ensuremath{ \logic{App}_\alpha }]{inf:def:lambda_term_alpha_equivalence/app}
        \begin{prooftree}
          \hypo{ A \aequiv C }
          \hypo{ B \aequiv D }
          \infer2[\ref{inf:def:lambda_term_alpha_equivalence/app}]{ AB \aequiv CD }
        \end{prooftree}
      \end{equation*}
    \end{rightcolumn}
  \end{paracol}

  \begin{equation*}\taglabel[\ensuremath{ \logic{Abs}_\alpha }]{inf:def:lambda_term_alpha_equivalence/abs}
    \begin{prooftree}
      \hypo{ A[a \mapsto n] \aequiv B[b \mapsto n] \T{for every} n \not\in \op*{Free}(\qabs a A) }
      \infer1[\ref{inf:def:lambda_term_alpha_equivalence/abs}]{ \qabs a A \aequiv \qabs b B }
    \end{prooftree}
  \end{equation*}
\end{definition}
\begin{comments}
  \item The gist of \ref{inf:def:lambda_term_alpha_equivalence/abs} is that we want \( A[a \mapsto n] \aequiv B[a \mapsto n] \) to hold for every variable \( n \) that is not free in either \( A \) or \( B \), but we still want to allow \( n \) to be \( a \) or \( b \). We do not need a separate condition for \( n \) not to be free in \( \qabs b B \) because, as we will show in \fullref{thm:def:lambda_term_alpha_equivalence/free}, if \( \qabs a A \) and \( \qabs b B \) are equivalence per our definition, they have the same free variables.

  The assumption in the premise will help us immensely in inductive proofs, but it is impossible to mechanize. We will be able to split it into two simplified rules in \fullref{thm:alpha_equivalence_simplified}.

  \item \hyperref[def:typed_lambda_term]{Typed \( \synlambda \)-terms} require modified rules for abstraction --- see \fullref{def:typed_term_alpha_equivalence}.

  \item Except for the explicit rule for constant terms, our rules resemble those in \cite[5]{Pollack2005AlphaConversion}, with two notable differences, both of which allow us to simplify inductive proofs:
  \begin{itemize}
    \item We have restricted \ref{inf:def:lambda_term_alpha_equivalence/atom} to atomic terms only, while Pollack states the rule for general terms.
    \item We have generalized the premise in \ref{inf:def:lambda_term_alpha_equivalence/abs} to hold for all suitable variables simultaneously, and have allowed this variable to be \( a \) or \( b \).
  \end{itemize}

  \item \incite[\S 2.1.11]{Barendregt1984LambdaCalculus} and \incite[def. 1A8]{Hindley1997BasicSTT} call the successive renaming of bound variables \enquote{\( \alpha \)-conversion} and consider \( \synlambda \)-terms \enquote{up to \( \alpha \)-conversion}. Barendregt calls \enquote{\( \alpha \)-congruence} for what we call \( \alpha \)-equivalence, while Hindley says that \enquote{\( M \) \( \alpha \)-converts to \( N \)}.

  \incite[5]{Pollack2005AlphaConversion} use \enquote{\( \alpha \)-conversion} and \enquote{\( \alpha \)-equivalence} interchangeably, while \incite[114]{Mimram2020ProgramEqualsProof} uses \enquote{\( \alpha \)-conversion} for what we call \( \alpha \)-equivalence.

  We will avoid using the term \enquote{\( \alpha \)-conversion}.

  \item The rules themselves can be formalized using \( \synlambda \)-term schemas and instantiations akin to how we have formalized the rules in \fullref{sec:natural_deduction}. We will introduce schemas for \( \synlambda \)-terms in \fullref{sec:simply_typed_lambda_terms}, however they will be more limited than what we need here and in \fullref{sec:lambda_term_reductions}.
\end{comments}

\begin{definition}\label{def:lambda_term_length}\mimprovised
  For inductive proofs, we will find useful to recursively define the \term{length} of a \( \synlambda \)-term as
  \begin{equation*}
    \len(M) \coloneqq \begin{cases}
      1,                     &M \in \op*{Atom}, \\
      2 + \len(N) + \len(K), &M = (NK), \\
      5 + \len(N),           &M = (\qabs x N). \\
    \end{cases}
  \end{equation*}
\end{definition}
\begin{comments}
  \item If the variable names in \( M \) are single-symbol strings, then \( \len(M) \) is exactly the \hyperref[def:formal_language/string_length]{string length} of \( M \). The string length is the definition of term length given by \incite[\S 2.1.3]{Barendregt1984LambdaCalculus}.

  Our definition instead resembles that by \incite[def. 1A2]{Hindley1997BasicSTT}, but we also account for parentheses, dots and \( \synlambda \) itself.

  \item We have introduced this concept mostly for proofs like that of \fullref{thm:def:lambda_term_alpha_equivalence/equivalence} so that the inductive hypothesis holds simultaneously for all terms of the same length.
\end{comments}

\begin{definition}\label{def:lambda_renaming}
  We say that the \hyperref[def:lambda_term_substitution]{substitution} \( \Bbbs \) is a \term{renaming substitution} if \( \Bbbs(x) \) is a variable for every \( x \).
\end{definition}
\begin{comments}
  \item The definition is based on \cite[252]{Mimram2020ProgramEqualsProof}, but restated for \( \synlambda \)-terms.
\end{comments}

\begin{proposition}\label{thm:def:lambda_term_length}
  \( \synlambda \)-terms have the following basic properties regarding their \hyperref[def:lambda_term_length]{length}:
  \begin{thmenum}
    \thmitem{thm:def:lambda_term_length/subterm} Every strict subterm of a \( \synlambda \)-term is strictly shorter.

    \thmitem{thm:def:lambda_term_length/substitution} For any substitution \( \Bbbs \) and \( \synlambda \)-term \( M \), we have
    \begin{equation}\label{eq:thm:def:lambda_term_length/substitution}
      \len(M[\Bbbs]) = \len(M) + \sum_{\mathclap{u \in \op*{Free}(M)}} (\len(\Bbbs(u)) - 1).
    \end{equation}

    In particular, if \( \Bbbs \) is a \hyperref[def:lambda_renaming]{renaming substitution}, then \( \len(M[\Bbbs]) = \len(M) \).

    \thmitem{thm:def:lambda_term_length/equivalent} \hyperref[def:lambda_term_alpha_equivalence]{\( \alpha \)-equivalent}  \( \synlambda \)-terms have the same length.
  \end{thmenum}
\end{proposition}
\begin{proof}
  \SubProofOf{thm:def:lambda_term_length/subterm} Trivial.

  \SubProofOf{thm:def:lambda_term_length/substitution} We proceed via \fullref{thm:induction_on_rooted_trees} on \( M \), simultaneously on all substitutions.
  \begin{itemize}
    \item If \( M \) is a constant, then \( M \) has no free variables, thus \eqref{eq:thm:def:lambda_term_length/substitution} holds vacuously.

    \item If \( M \) is a variable, then
    \begin{equation*}
      \len(M[\Bbbs]) = \len(\Bbbs(M)) = \len(M) + (\len(\Bbbs(M)) - \underbrace{\len(M)}_1).
    \end{equation*}

    \item If \( M = NK \), where the inductive hypothesis holds for \( N \) and \( K \), we have
    \begin{balign*}
      \len(M[\Bbbs])
      &=
      \len(N[\Bbbs] K[\Bbbs])
      = \\ &=
      2 + \len(N[\Bbbs]) + \len(K[\Bbbs])
      \reloset {\T{ind.}} = \\ &=
      2 + \len(N) + \sum_{\mathclap{u \in \op*{Free}(N)}} (\len(\Bbbs(u)) - 1) + \len(K) + \sum_{\mathclap{u \in \op*{Free}(K)}} (\len(\Bbbs(u)) - 1)
      = \\ &=
      2 + \len(N) + \len(K) + \sum_{\mathclap{u \in \op*{Free}(NK)}} (\len(\Bbbs(u)) - 1)
      = \\ &=
      \len(NK) + \sum_{\mathclap{u \in \op*{Free}(NK)}} (\len(\Bbbs(u)) - 1).
    \end{balign*}

    \item Finally, if \( M = \qabs x N \), where the inductive hypothesis holds for \( N \), \fullref{thm:lambda_substitution_single_rule} implies that
    \begin{equation*}
      M[\Bbbs] = \qabs v N[\Bbbs_{x \mapsto v}],
    \end{equation*}
    where \( v \not\in \op*{Free}(N) \cup \op*{Free}(N[\Bbbs]) \).

    Then
    \begin{balign*}
      \len(M[\Bbbs])
      &=
      5 + \len(N[\Bbbs_{x \mapsto u}])
      \reloset {\T{ind.}} = \\ &=
      5 + \len(N) + \sum_{\mathclap{u \in \op*{Free}(N)}} (\len(\Bbbs_{x \mapsto v}(u)) - 1)
      = \\ &=
      5 + \len(N) + \sum_{\mathclap{u \in \op*{Free}(N) \setminus \set{ x }}} (\len(\Bbbs(u)) - 1) + (\len(v) - 1)
      = \\ &=
      \len(M) + \sum_{\mathclap{u \in \op*{Free}(M)}} (\len(\Bbbs(u)) - 1).
    \end{balign*}
  \end{itemize}

  \SubProofOf{thm:def:lambda_term_length/equivalent} Straightforward considering \fullref{thm:def:lambda_term_length/substitution}.
\end{proof}

\begin{proposition}\label{thm:def:lambda_term_alpha_equivalence}
  \hyperref[def:lambda_term_alpha_equivalence]{\( \alpha \)-equivalence} of \( \synlambda \)-terms has the following basic properties:
  \begin{thmenum}
    \thmitem{thm:def:lambda_term_alpha_equivalence/same_kind} If two \( \synlambda \)-terms are \( \alpha \)-equivalent, they are both either variables, applications or abstractions.

    \thmitem{thm:def:lambda_term_alpha_equivalence/free} The free variables of \( \alpha \)-equivalent \( \synlambda \)-terms coincide.

    \thmitem{thm:def:lambda_term_alpha_equivalence/equivalence} \( \alpha \)-equivalence is an \hyperref[def:equivalence_relation]{equivalence relation}.
  \end{thmenum}
\end{proposition}
\begin{proof}
  \SubProofOf{thm:def:lambda_term_alpha_equivalence/same_kind} Follows by inspecting the conclusions of the rules in \fullref{def:lambda_term_alpha_equivalence}.

  \SubProofOf{thm:def:lambda_term_alpha_equivalence/free} We proceed via induction on the length of \( M \) to show that, if \( M \aequiv N \), then the free variables of \( M \) and \( N \) coincide.

  No term has length \( 0 \). If \( M \) has length \( 1 \), it is necessarily an atom, and the only rule that allow us to conclude that \( M \aequiv N \) is \ref{inf:def:lambda_term_alpha_equivalence/atom}. It is satisfied only if \( M = N \). Thus, \( \op*{Free}(M) = \op*{Free}(N) \).

  If the length of \( M \) is strictly greater than \( 1 \), we have the following possibilities:
  \begin{itemize}
    \item If \( M = AB \), the rule \ref{inf:def:lambda_term_alpha_equivalence/app} is applicable if \( N \) is also an application, say \( N = CD \), and if both \( A \aequiv C \) and \( B \aequiv D \). Then
    \begin{equation*}
      \op*{Free}(M)
      =
      \op*{Free}(A) \cup \op*{Free}(B)
      \reloset{\T{ind.}} =
      \op*{Free}(C) \cup \op*{Free}(D)
      =
      \op*{Free}(N).
    \end{equation*}

    \item Otherwise, we have \( M = \qabs a A \). The rule \ref{inf:def:lambda_term_alpha_equivalence/abs} implies that \( N = \qabs b B \), where in particular \( A = A[a \mapsto a] \aequiv B[b \mapsto a] \). Then
    \begin{align*}
      \op*{Free}(\qabs a A)
      &=
      \op*{Free}(A) \setminus \set{ a }
      \reloset {\eqref{eq:thm:lambda_substitution_free_variables}} = \\ &=
      \op*{Free}(A[a \mapsto a]) \setminus \set{ a }
      \reloset{\T{ind.}} = \\ &=
      \op*{Free}(B[b \mapsto a]) \setminus \set{ a }
      \reloset {\eqref{eq:thm:lambda_substitution_free_variables}} = \\ &=
      \op*{Free}(B) \setminus \set{ b }
      = \\ &=
      \op*{Free}(\qabs b B).
    \end{align*}
  \end{itemize}

  \SubProofOf{thm:def:lambda_term_alpha_equivalence/equivalence}

  \SubProofOf*[def:binary_relation/reflexive]{reflexivity} We proceed via \fullref{thm:induction_on_rooted_trees} on the \( \synlambda \)-term \( M \) to show that \( M \aequiv N \) whenever \( M = N \).
  \begin{itemize}
    \item If \( M \) is an atomic term, the rule \ref{inf:def:lambda_term_alpha_equivalence/atom} directly implies that \( M \aequiv M \).
    \item If \( M = AB \) and the inductive hypothesis holds for \( A \) and \( B \), then \( A \aequiv A \) and \( B \aequiv B \), and the rule \ref{inf:def:lambda_term_alpha_equivalence/app} implies that \( M \aequiv M \).
    \item If \( M = \qabs x N \) and if the inductive hypothesis holds for \( N \), then \( N \aequiv N \). The rule \ref{inf:def:lambda_term_alpha_equivalence/abs} implies that \( M \aequiv M \).
  \end{itemize}

  \SubProofOf*[def:binary_relation/symmetric]{symmetry} We will use induction on the length of \( M \) to show that \( M \aequiv N \) implies \( N \aequiv M \).

  There are no terms of length \( 0 \), so in this case symmetry holds vacuously. A term of length \( 1 \) is atomic, and the rule \ref{inf:def:lambda_term_alpha_equivalence/atom} implies that \( \alpha \)-equivalent atomic terms are equal. Hence, symmetry holds for terms of length \( 1 \).

  For terms strictly longer than \( 1 \) symbol, we have two possibilities:
  \begin{itemize}
    \item Suppose first that \( M = AB \). Then, if \( M \aequiv N \), \ref{inf:def:lambda_term_alpha_equivalence/app} implies that \( N = CD \), where \( A \aequiv C \) and \( B \aequiv D \).

    By the inductive hypothesis, \( C \aequiv A \) and \( D \aequiv B \). Applying \ref{inf:def:lambda_term_alpha_equivalence/app}, we conclude that \( N \aequiv M \).

    \item Otherwise, we have \( M = \qabs a A \). The rule \ref{inf:def:lambda_term_alpha_equivalence/abs} implies that \( N = \qabs b B \) and, for every variable \( n \) not in \( M \), \( A[a \mapsto n] \aequiv B[b \mapsto n] \).

    By \fullref{thm:def:lambda_term_length/substitution}, \( A[a \mapsto n] \) has the same length as \( A \), thus the inductive hypothesis holds for the former. Then \( B[b \mapsto n] \aequiv A[a \mapsto n] \).

    Furthermore, \fullref{thm:def:lambda_term_alpha_equivalence/free} implies that \( n \) is not free in \( N \), so we can apply \ref{inf:def:lambda_term_alpha_equivalence/abs} to obtain \( N \aequiv M \).
  \end{itemize}

  \SubProofOf*[def:binary_relation/transitive]{transitivity} We will again use induction on the length of \( M \) that \( M \aequiv N \) and \( N \aequiv K \) imply \( M \aequiv K \).

  The base case and the case where \( M \) is an application are straightforward. For the remaining case, suppose that \( M = \qabs a A \), so that \( N = \qabs b B \) and \( K = \qabs c C \).

  Fix a variable \( n \) not free in \( M \). \Fullref{thm:def:lambda_term_alpha_equivalence/free} implies that \( n \) is also not free in \( N \). Then \( A[a \mapsto n] \aequiv B[b \mapsto n] \) and \( B[b \mapsto n] \aequiv C[c \mapsto n] \).

  The inductive hypothesis applies to \( A \), hence also to \( A[a \mapsto n] \), therefore
  \begin{equation*}
    A[a \mapsto n] \aequiv C[c \mapsto n].
  \end{equation*}

  We conclude via \ref{inf:def:lambda_term_alpha_equivalence/abs} that \( M \aequiv K \).
\end{proof}

\begin{example}\label{ex:def:lambda_term_alpha_equivalence}
  We list examples of \hyperref[def:lambda_term_alpha_equivalence]{\( \alpha \)-equivalence}:
  \begin{thmenum}
    \thmitem{ex:inf:def:lambda_term_alpha_equivalence/combinator} The combinator \( \ref{eq:ex:def:lambda_term/combinator/i} \) has the same essential structure regardless of how we name its variables:
    \begin{equation*}
      I = \qabs \synx \synx \aequiv \qabs \syny \syny \aequiv \qabs \syna \syna \aequiv \cdots
    \end{equation*}

    \thmitem{ex:inf:def:lambda_term_alpha_equivalence/nested} Consider a more complicated case:
    \begin{equation*}
      \qabs \synx \qabs \syny \synx \syny
      \T{and}
      \qabs \syny \qabs \synx \syny \synx.
    \end{equation*}

    In the notation of \ref{inf:def:lambda_term_alpha_equivalence/abs}, we have
    \begin{equation*}
      \qabs {\underbrace{\synx}_a} \underbrace{\qabs \syny \synx \syny}_A
      \T{and}
      \qabs {\underbrace{\syny}_b} \underbrace{\qabs \synx \syny \synx}_B.
    \end{equation*}

    We reduce our \( \alpha \)-equivalence check to \( A[a \mapsto n] \) and \( B[b \mapsto n] \) (note that the bound variables were renamed to \( \syna \) in both):
    \begin{equation*}
      \qabs {\underbrace{\syna}_c} \underbrace{n \syna}_C
      \T{and}
      \qabs {\underbrace{\syna}_d} \underbrace{n \syna}_D.
    \end{equation*}

    We further reduce our check to \( C[\syna \mapsto m] = nm \) and \( D[\syna \mapsto m] = nm \). The rule \ref{inf:def:lambda_term_alpha_equivalence/app} allows us to reduce this problem to checking whether \( n \aequiv n \) and \( m \aequiv m \), while \ref{inf:def:lambda_term_alpha_equivalence/atom} confirms that the latter equivalences hold. Therefore, our entire verification is correct and
    \begin{equation*}
      \qabs \synx \qabs \syny \synx \syny \aequiv \qabs \syny \qabs \synx \syny \synx.
    \end{equation*}

    \thmitem{ex:inf:def:lambda_term_alpha_equivalence/freeing} In \fullref{ex:def:lambda_variable_freeness/both} we discussed how \( \synx \) is both bound and free in \( M = I \synx = (\qabs \synx \synx) \synx \).

    We can use the term \( M' = (\qabs \syny \syny) \synx \) instead, where all occurrences of \( \synx \) are free and all occurrences of \( \syny \) are bound.

    This is generalized by \fullref{alg:separation_of_free_and_bound_variables}.
  \end{thmenum}
\end{example}

\paragraph{Invariance of \( \alpha \)-equivalence under substitution}

\begin{lemma}\label{thm:renaming_composition_is_alpha_equivalent}
  For the composite substitution \( \Bbbt\Bbbs(u) \coloneqq \Bbbs(u)[\Bbbt] \) of the \hyperref[def:lambda_renaming]{\hi{renaming substitutions}} \( \Bbbs \) and \( \Bbbt \), we have
  \begin{equation}\label{eq:thm:renaming_composition_is_alpha_equivalent}
    M[\Bbbt\Bbbs] \aequiv M[\Bbbs][\Bbbt].
  \end{equation}
\end{lemma}
\begin{comments}
  \item We will use this to prove \fullref{thm:renaming_on_alpha_equivalent_terms}, which will later be used to prove the generalization \fullref{thm:substitution_composition_is_alpha_equivalent} to non-renaming substitutions.
\end{comments}
\begin{proof}
  We will use induction on the length of \( M \) simultaneously on all substitutions.

  The base case and the case where \( M \) is an application are straightforward. For the remaining case, suppose that \( M = \qabs x N \). We will utilize \fullref{thm:lambda_substitution_single_rule} thrice:
  \begin{align*}
    M[\Bbbt\Bbbs]   &= \qabs u N[\Bbbt\Bbbs_{x \mapsto u}],                 &&u \not\in \op*{Free}(M) \cup \op*{Free}(M[\Bbbt\Bbbs]), \\
    M[\Bbbs]        &= \qabs v N[\Bbbs_{x \mapsto v}],                      &&v \not\in \op*{Free}(M) \cup \op*{Free}(M[\Bbbs]), \\
    M[\Bbbs][\Bbbt] &= \qabs w N[\Bbbs_{x \mapsto v}][\Bbbt_{v \mapsto w}], &&w \not\in \op*{Free}(M[\Bbbs]) \cup \op*{Free}(M[\Bbbs][\Bbbt]).
  \end{align*}

  Fix a variable \( n \) not free in \( M[\Bbbt\Bbbs] \) and denote by \( \pi \) the composite of \( \Bbbt\Bbbs_{x \mapsto u} \) and \( u \mapsto n \). The inductive hypothesis gives us
  \begin{equation*}
    N[\Bbbt\Bbbs_{x \mapsto u}][u \mapsto n] \aequiv N[\pi].
  \end{equation*}

  Since \( u \) is not free in \( M \) nor in \( M[\Bbbt\Bbbs] \), its only occurrences in \( N[\Bbbt\Bbbs_{x \mapsto u}] \) are those where \( x \) has been substituted for it. \Fullref{thm:lambda_substitution_restriction} then implies that
  \begin{equation*}
    N[\pi] = N[\Bbbt\Bbbs_{x \mapsto n}],
  \end{equation*}
  and the inductive hypothesis on \( N \) allows us to conclude that
  \begin{equation}\label{eq:thm:renaming_composition_is_alpha_equivalent/proof/composite}
    N[\Bbbt\Bbbs_{x \mapsto u}][u \mapsto n] \aequiv N[\Bbbt\Bbbs_{x \mapsto n}].
  \end{equation}

  Since \( \Bbbs \) is a renaming, \fullref{thm:def:lambda_term_length/substitution} implies that \( N[\Bbbs_{x \mapsto v}] \) has the same length as \( N \), thus the inductive hypothesis holds for the former. Then
  \begin{equation*}
    N[\Bbbs_{x \mapsto v}][\Bbbt_{v \mapsto w}][w \mapsto n] \aequiv N[\Bbbs_{x \mapsto v}][\Bbbt_{v \mapsto n}].
  \end{equation*}

  Via the inductive hypothesis on \( N \), we can further simplify this and obtain
  \begin{equation}\label{eq:thm:renaming_composition_is_alpha_equivalent/proof/repeated}
    N[\Bbbs_{x \mapsto v}][\Bbbt_{v \mapsto w}][w \mapsto n] \aequiv N[\Bbbt\Bbbs_{x \mapsto n}].
  \end{equation}

  Combining \eqref{eq:thm:renaming_composition_is_alpha_equivalent/proof/composite} and \eqref{eq:thm:renaming_composition_is_alpha_equivalent/proof/repeated} allows us to conclude that \( M[\Bbbt\Bbbs] \aequiv M[\Bbbs][\Bbbt] \).
\end{proof}

\begin{lemma}\label{thm:renaming_on_alpha_equivalent_terms}
  If \( M \aequiv N \), for any \hyperref[def:lambda_renaming]{\hi{renaming substitution}} \( \Bbbs \), we have \( M[\Bbbs] \aequiv N[\Bbbs] \).
\end{lemma}
\begin{comments}
  \item We will use this to prove \fullref{thm:substitution_composition_is_alpha_equivalent}, which will later be used to prove the generalization \fullref{thm:substitution_on_alpha_equivalent_terms} to non-renaming substitutions.
\end{comments}
\begin{proof}
  We will use induction on the length of \( M \) simultaneously on all substitutions.

  Again, the base case and the case where \( M \) is an application are straightforward. For the remaining case, suppose that \( M = \qabs a A \), so that \( N = \qabs b B \). We will utilize \fullref{thm:lambda_substitution_single_rule} twice:
  \begin{align*}
    M[\Bbbs] &= \qabs u A[\Bbbs_{a \mapsto u}], &&u \not\in \op*{Free}(M) \cup \op*{Free}(M[\Bbbs]), \\
    N[\Bbbs] &= \qabs v B[\Bbbs_{b \mapsto v}], &&v \not\in \op*{Free}(N) \cup \op*{Free}(N[\Bbbs]).
  \end{align*}

  Since \( a \) is not free in \( M \), from \( M \aequiv N \) we conclude that \( A[a \mapsto a] \aequiv B[a \mapsto b] \), where \( A[a \mapsto a] = A \) as a consequence of \fullref{thm:lambda_substitution_noop}.

  Applying the inductive hypothesis to \( A \), we can conclude that
  \begin{equation*}
    A[\Bbbs_{a \mapsto u}] \aequiv B[b \mapsto a][\Bbbs_{a \mapsto u}].
  \end{equation*}

  Furthermore, from \fullref{thm:renaming_composition_is_alpha_equivalent} it follows that
  \begin{equation*}
    B[b \mapsto a][\Bbbs_{a \mapsto u}] \aequiv B[\Bbbs_{b \mapsto u}]
  \end{equation*}
  and
  \begin{equation*}
    B[\Bbbs_{b \mapsto u}] \aequiv B[\Bbbs_{b \mapsto v}][v \mapsto u].
  \end{equation*}

  Fix a variable \( n \) not free in \( M[\Bbbs] \). Since \( \Bbbs \) is a renaming, the inductive hypothesis holds for \( A[\Bbbs_{a \mapsto u}] \). Then, by the inductive hypothesis,
  \begin{equation*}
    A[\Bbbs_{a \mapsto u}][u \mapsto n]
    \aequiv
    B[\Bbbs_{b \mapsto v}][v \mapsto u][u \mapsto n].
  \end{equation*}

  \begin{itemize}
    \item If \( u = v \), \fullref{thm:lambda_substitution_restriction} implies that
    \begin{equation*}
      B[\Bbbs_{b \mapsto v}][v \mapsto u] = B[\Bbbs_{b \mapsto u}],
    \end{equation*}
    thus \fullref{thm:renaming_composition_is_alpha_equivalent} implies that
    \begin{equation*}
      \underbrace{B[\Bbbs_{b \mapsto v}][v \mapsto u]}_{B[\Bbbs_{b \mapsto u}]}[u \mapsto n] = B[\Bbbs_{b \mapsto v}][v \mapsto n].
    \end{equation*}

    \item Otherwise, the same conclusion follows from \fullref{thm:renaming_composition_is_alpha_equivalent} directly:
    \begin{equation*}
      B[\Bbbs_{b \mapsto v}]\underbrace{[v \mapsto u][u \mapsto n]}_{[v \mapsto n]} = B[\Bbbs_{b \mapsto v}][v \mapsto n].
    \end{equation*}
  \end{itemize}

  Therefore,
  \begin{equation*}
    A[\Bbbs_{a \mapsto u}][u \mapsto n]
    \aequiv
    B[\Bbbs_{a \mapsto v}][v \mapsto n]
  \end{equation*}
  and \ref{inf:def:lambda_term_alpha_equivalence/abs} allows us to conclude that \( M[\Bbbs] \aequiv N[\Bbbs] \).
\end{proof}

\begin{proposition}\label{thm:substitution_composition_is_alpha_equivalent}
  For the composite substitution \( \Bbbt\Bbbs(u) \coloneqq \Bbbs(u)[\Bbbt] \), we have
  \begin{equation}\label{eq:thm:substitution_composition_is_alpha_equivalent}
    M[\Bbbt\Bbbs] \aequiv M[\Bbbs][\Bbbt].
  \end{equation}
\end{proposition}
\begin{comments}
  \item This result generalizes \fullref{thm:renaming_composition_is_alpha_equivalent} to arbitrary substitutions.
\end{comments}
\begin{proof}
  We can follow the proof of \fullref{thm:renaming_composition_is_alpha_equivalent} until the point where we use that \( \Bbbs \) and \( \Bbbt \) are renaming substitutions.

  In short, we are following induction on the length of \( M \), and in the case where \( M = \qabs x N \) we have
  \begin{align*}
    M[\Bbbt\Bbbs]   &= \qabs u N[\Bbbt\Bbbs_{x \mapsto u}],                 &&u \not\in \op*{Free}(M) \cup \op*{Free}(M[\Bbbt\Bbbs]), \\
    M[\Bbbs]        &= \qabs v N[\Bbbs_{x \mapsto v}],                      &&v \not\in \op*{Free}(M) \cup \op*{Free}(M[\Bbbs]), \\
    M[\Bbbs][\Bbbt] &= \qabs w N[\Bbbs_{x \mapsto v}][\Bbbt_{v \mapsto w}], &&w \not\in \op*{Free}(M[\Bbbs]) \cup \op*{Free}(M[\Bbbs][\Bbbt]).
  \end{align*}

  Fix again a variable \( n \) not free in \( M[\Bbbt\Bbbs] \). In \fullref{thm:substitution_composition_is_alpha_equivalent} we have used the inductive hypothesis on \( N \) directly to conclude \eqref{eq:thm:renaming_composition_is_alpha_equivalent/proof/composite}, which we will restate here for convenience:
  \begin{equation}\label{eq:thm:substitution_composition_is_alpha_equivalent/proof/composite}
    N[\Bbbt\Bbbs_{x \mapsto u}][u \mapsto n] \aequiv N[\Bbbt\Bbbs_{x \mapsto n}].
  \end{equation}

  On the other hand, we can no longer apply the inductive hypothesis to \( N[\Bbbs_{x \mapsto v}] \) to conclude \eqref{eq:thm:renaming_composition_is_alpha_equivalent/proof/repeated}. Instead, we will start by applying the hypothesis to \( N \) to obtain
  \begin{equation*}
    N[\Bbbs_{x \mapsto v}][\Bbbt_{v \mapsto w}] \aequiv N[\Bbbt\Bbbs_{x \mapsto w}]
  \end{equation*}
  and then \fullref{thm:renaming_on_alpha_equivalent_terms} to obtain
  \begin{equation*}
    N[\Bbbs_{x \mapsto v}][\Bbbt_{v \mapsto w}][w \mapsto n] \aequiv N[\Bbbt\Bbbs_{x \mapsto w}][w \mapsto n].
  \end{equation*}

  We can now apply the inductive hypothesis to \( N \) for the third time to simplify the right-hand side and obtain
  \begin{equation}\label{eq:thm:substitution_composition_is_alpha_equivalent/proof/repeated}
    N[\Bbbs_{x \mapsto v}][\Bbbt_{v \mapsto w}][w \mapsto n] \aequiv N[\Bbbt\Bbbs_{x \mapsto n}].
  \end{equation}

  Finally, we can combine \eqref{eq:thm:substitution_composition_is_alpha_equivalent/proof/composite} and \eqref{eq:thm:substitution_composition_is_alpha_equivalent/proof/repeated} to conclude that \( M[\Bbbt\Bbbs] \aequiv M[\Bbbs][\Bbbt] \) via \ref{inf:def:lambda_term_alpha_equivalence/abs}.
\end{proof}

\begin{corollary}\label{thm:substitution_chain_contraction}
  If \( y \) is not free in \( M \), then, for any substitution \( \Bbbs \), we have
  \begin{equation}\label{eq:thm:substitution_chain_contraction/contraction}
    M[x \mapsto y][\Bbbs_{y \mapsto z}]
    \aequiv
    M[\Bbbs_{x \mapsto z}]
    \aequiv
    M[\Bbbs_{x \mapsto y}][y \mapsto z].
  \end{equation}
\end{corollary}
\begin{comments}
  \item We have already used a similar argument, but now we can state it in our desired generality.
\end{comments}
\begin{proof}
  Follows from \fullref{thm:substitution_composition_is_alpha_equivalent} since, by assumption, \( y \) is not free in \( M \).
\end{proof}

\begin{proposition}\label{thm:substitution_on_alpha_equivalent_terms}
  If \( M \aequiv N \), for an \hyperref[def:lambda_term_substitution]{\hi{arbitrary substitution}} \( \Bbbs \), we have \( M[\Bbbs] \aequiv N[\Bbbs] \).
\end{proposition}
\begin{comments}
  \item This result generalizes \fullref{thm:renaming_on_alpha_equivalent_terms} to arbitrary substitutions.
\end{comments}
\begin{proof}
  We can follow the proof of \fullref{thm:renaming_on_alpha_equivalent_terms} until the point where we use that \( \Bbbs \) and \( \Bbbt \) are renaming substitutions.

  In short, we are following induction on the length of \( M \), and in the case where \( M = \qabs a A \) and \( N = \qabs b B \) and
  \begin{align*}
    M[\Bbbs] &= \qabs u A[\Bbbs_{a \mapsto u}], &&u \not\in \op*{Free}(M) \cup \op*{Free}(M[\Bbbs]), \\
    N[\Bbbs] &= \qabs v B[\Bbbs_{b \mapsto v}], &&v \not\in \op*{Free}(N) \cup \op*{Free}(N[\Bbbs]).
  \end{align*}

  Since \( M \aequiv N \), we can conclude, as in our proof of \fullref{thm:renaming_on_alpha_equivalent_terms}, that
  \begin{equation*}
    A \aequiv B[b \mapsto a].
  \end{equation*}

  Again, fix a variable \( n \) not free in \( M[\Bbbs] \). We can use the inductive hypothesis that
  \begin{equation*}
    A[\Bbbs_{a \mapsto n}] \aequiv B[b \mapsto a][\Bbbs_{a \mapsto n}].
  \end{equation*}

  This already differs from our proof of \fullref{thm:renaming_on_alpha_equivalent_terms} in that we modify \( \Bbbs \) via \( a \mapsto n \) rather than \( a \mapsto u \).

  We can use \fullref{thm:substitution_chain_contraction} to simplify the right-hand side and obtain
  \begin{equation*}
    A[\Bbbs_{a \mapsto n}] \aequiv B[\Bbbs_{b \mapsto n}]
  \end{equation*}
  and then again to expand both sides:
  \begin{equation*}
    A[\Bbbs_{a \mapsto u}][u \mapsto n] \aequiv B[\Bbbs_{b \mapsto v}][v \mapsto n].
  \end{equation*}

  Therefore, \ref{inf:def:lambda_term_alpha_equivalence/abs} allows us to conclude that \( M[\Bbbs] \aequiv N[\Bbbs] \).
\end{proof}

\paragraph{Mechanizing \( \alpha \)-equivalence}

\begin{proposition}\label{thm:alpha_equivalence_simplified}
  The rule \ref{inf:def:lambda_term_alpha_equivalence/abs}, specifying when two abstractions are \( \alpha \)-equivalent, is equivalent to the following pair of rules:
  \begin{paracol}{2}
    \begin{leftcolumn}
      \ParacolAlignmentHack
      \begin{equation*}\taglabel[\ensuremath{ \logic{Lift}_\alpha }]{inf:thm:alpha_equivalence_simplified/lift}
        \begin{prooftree}
          \hypo{ A \aequiv B }
          \infer1[\ref{inf:thm:alpha_equivalence_simplified/lift}]{ \qabs x A \aequiv \qabs x B }
        \end{prooftree}
      \end{equation*}
    \end{leftcolumn}

    \begin{rightcolumn}
      \ParacolAlignmentHack
      \begin{equation*}\taglabel[\ensuremath{ \logic{Ren}_\alpha }]{inf:thm:alpha_equivalence_simplified/ren}
        \begin{prooftree}
          \hypo{ a \neq b }
          \hypo{ a \not\in \op*{Free}(B) }
          \hypo{ A \aequiv B[b \mapsto a] }
          \infer3[\ref{inf:thm:alpha_equivalence_simplified/ren}]{ \qabs a A \aequiv \qabs b B }
        \end{prooftree}
      \end{equation*}
    \end{rightcolumn}
  \end{paracol}
\end{proposition}
\begin{comments}
  \item This check can be found as \identifier{lambda_.untyped.alpha.are_terms_alpha_equivalent} in \cite{notebook:code}.
\end{comments}
\begin{proof}
  \SufficiencySubProof We will show that \ref{inf:thm:alpha_equivalence_simplified/lift} and \ref{inf:thm:alpha_equivalence_simplified/ren} are \hyperref[con:inference_rule_admissibility]{admissible} with respect to \ref{inf:def:lambda_term_alpha_equivalence/abs}.

  \SubProof*{Proof that \ref{inf:thm:alpha_equivalence_simplified/lift} is admissible} Fix a variable \( x \) and terms \( A \aequiv B \). Let \( n \) be any variable not free in \( \qabs x A \). \Fullref{thm:substitution_on_alpha_equivalent_terms} then implies that \( A[x \mapsto n] \aequiv B[x \mapsto n] \).

  The rule \ref{inf:def:lambda_term_alpha_equivalence/abs} then implies that \( \qabs x A \aequiv \qabs x B \).

  Thus, \ref{inf:thm:alpha_equivalence_simplified/lift} is admissible.

  \SubProof*{Proof that \ref{inf:thm:alpha_equivalence_simplified/ren} is admissible} Fix variables \( a \neq b \) and terms \( A \) and \( B \) such that \( a \) is not free in \( B \) and \( A \aequiv B[b \mapsto a] \).

  Let \( n \) be any variable not free in \( \qabs a A \). \Fullref{thm:substitution_on_alpha_equivalent_terms} implies that
  \begin{equation*}
    A[a \mapsto n] \aequiv B[b \mapsto a][a \mapsto n]
  \end{equation*}
  and \fullref{thm:substitution_chain_contraction} implies that
  \begin{equation*}
    A[a \mapsto n] \aequiv B[b \mapsto n].
  \end{equation*}

  The rule \ref{inf:def:lambda_term_alpha_equivalence/abs} then implies that \( \qabs a A \aequiv \qabs b B \).

  \NecessitySubProof We will show that \ref{inf:def:lambda_term_alpha_equivalence/abs} is admissible with respect to \ref{inf:thm:alpha_equivalence_simplified/lift} and \ref{inf:thm:alpha_equivalence_simplified/ren}.

  Fix \( \synlambda \)-terms \( A \) and \( B \) and variables \( a \) and \( b \) and suppose that \( A[a \mapsto n] \aequiv B[b \mapsto n] \) for every variable \( n \) not free in \( \qabs a A \). In particular, we are interested in \( n = a \).

  \begin{itemize}
    \item If \( a = b \), then \fullref{thm:lambda_substitution_noop} implies that \( A[a \mapsto a] = A \) and \( B[b \mapsto a] = B \), thus \ref{inf:thm:alpha_equivalence_simplified/lift} implies that \( \qabs a A \aequiv \qabs b B \).

    \item Otherwise, \( A \aequiv B[b \mapsto a] \), and to use \ref{inf:thm:alpha_equivalence_simplified/ren} we must only verify that \( a \) is not free in \( B \).

    If we suppose that \( a \) is free in \( B \), this will contradict \fullref{thm:def:lambda_term_alpha_equivalence/free} since \( A[a \mapsto b] \aequiv B \). Thus, \( a \) is indeed not free in \( B \), and we can use \ref{inf:thm:alpha_equivalence_simplified/ren} to conclude that \( \qabs a A \aequiv \qabs b B \).
  \end{itemize}
\end{proof}

\begin{corollary}\label{thm:alpha_conversion}
  If \( y \) is not free in \( M \), then
  \begin{equation}\label{eq:thm:alpha_conversion}
    \qabs x M \aequiv \qabs y M[y \mapsto x]
  \end{equation}
\end{corollary}
\begin{comments}
  \item \incite[\S 2.1.11]{Barendregt1984LambdaCalculus} calls this property \enquote{\( \alpha \)-conversion} and uses it as a definition, while \incite[def. 1A8]{Hindley1997BasicSTT} calls it \enquote{change of bound variables} and calls \enquote{\( \alpha \)-conversion} the transitive closure of the relation obtained.
\end{comments}
\begin{proof}
  This is a restatement of \ref{inf:thm:alpha_equivalence_simplified/ren}.
\end{proof}

\begin{corollary}\label{thm:alpha_conversion_modified}
  If \( u \) and \( v \) are not free in \( M \), then, for any substitution \( \Bbbs \),
  \begin{equation}\label{eq:thm:alpha_conversion_modified}
    \qabs u M[\Bbbs_{x \mapsto u}] \aequiv \qabs v M[\Bbbs_{x \mapsto v}].
  \end{equation}
\end{corollary}
\begin{proof}
  \Fullref{thm:def:lambda_term_alpha_equivalence/equivalence} implies that
  \begin{equation*}
    M[\Bbbs_{x \mapsto v}] \aequiv M[\Bbbs_{x \mapsto v}].
  \end{equation*}

  Furthermore, \fullref{thm:substitution_chain_contraction} implies that
  \begin{equation*}
    M[\Bbbs_{x \mapsto v}][v \mapsto u] \aequiv M[\Bbbs_{x \mapsto u}]
  \end{equation*}
  and \ref{inf:thm:alpha_equivalence_simplified/ren} implies that
  \begin{equation*}
    \qabs v M[\Bbbs_{x \mapsto v}] \aequiv \qabs u M[\Bbbt_{x \mapsto u}].
  \end{equation*}
\end{proof}

\begin{algorithm}[Separation of free and bound variables]\label{alg:separation_of_free_and_bound_variables}
  Fix a \( \synlambda \)-term \( M \). We will build an \hyperref[def:lambda_term_alpha_equivalence]{\( \alpha \)-equivalent} term \( M' \) where the \hyperref[def:lambda_variable_freeness]{free and bound variables} are distinct.

  Fix a map \( \sharp \) as in \fullref{def:lambda_term_substitution} for generating fresh variable identifiers.

  We will use the following auxiliary operation, in which \( \Gamma \) is a context of variables that should be avoided:
  \begin{equation*}
    M'_\Gamma \coloneqq \begin{cases}
      M,                                 &M \in \op*{Atom}, \\
      N'_\Gamma \thinspace K'_\Gamma,    &M = NK, \\
      \qabs x N'_{\Gamma,x}              &M = \qabs x N \T{and} x \not\in \Gamma, \\
      \qabs n N'_{\Gamma,n}[x \mapsto n] &M = \qabs x N \T{and} x \in \Gamma,
    \end{cases}
  \end{equation*}
  where \( n \coloneqq \sharp(\op*{Free}(N) \cup \Gamma) \). We claim that \( M \aequiv M'_\Gamma \) and that the bound variables in \( M'_\Gamma \) are disjoint from \( \Gamma \).

  To obtain the desired \( \synlambda \)-term \( M' \), we simply take \( M'_\Gamma \) with \( \Gamma = \op*{Free}(M) \).
\end{algorithm}
\begin{comments}
  \item This algorithm can be found as \identifier{lambda_.untyped.alpha.separate_free_and_bound_variables} in \cite{notebook:code}.
\end{comments}
\begin{defproof}
  We proceed via \fullref{thm:induction_on_rooted_trees} on \( M \) to show that, for any context \( \Gamma \), \( M \aequiv M'_\Gamma \) and the bound variables in \( M'_\Gamma \) are not in \( \Gamma \).

  \begin{itemize}
    \item If \( M \) is an atomic term, then \( M = M'_\Gamma \), hence they are \( \alpha \)-equivalent by \fullref{thm:def:lambda_term_alpha_equivalence/equivalence}. Furthermore, \( M \) has no bound variables.

    \item If \( M = NK \) and if the inductive hypothesis holds for both \( N \) and \( K \), then the rule \ref{inf:def:lambda_term_alpha_equivalence/app} allows us to conclude that \( M \aequiv M'_\Gamma \) and the inductive hypothesis allows us to conclude that the bound variables of \( M'_\Gamma \) are disjoint from \( \Gamma \).

    \item Finally, suppose that \( M = \qabs x N \) and that the inductive hypothesis holds for \( N \). We have the following possibilities:
    \begin{itemize}
      \item If \( x \not\in \Gamma \), then \( M' = \qabs x N'_{\Gamma,x} \).

      By the inductive hypothesis, \( N \aequiv N'_{\Gamma,x} \) and the rule \ref{inf:thm:alpha_equivalence_simplified/lift} allows us to conclude that
      \begin{equation*}
        M = \qabs x N \aequiv \qabs x N'_{\Gamma,x} = M'_\Gamma.
      \end{equation*}

      Furthermore, by the inductive hypothesis,
      \begin{equation*}
        \op*{Bound}(N'_{\Gamma,x}) \cap \parens[\Big]{ \Gamma \cup \set{ x } } = \varnothing,
      \end{equation*}
      hence
      \begin{align*}
        \op*{Bound}(M'_\Gamma) \cap \Gamma
        &=
        \parens[\Big]{ \op*{Bound}(N'_{\Gamma,x}) \cup \set{ x } } \cap \Gamma
        \reloset {\eqref{eq:thm:union_intersection_distributivity/intersection_over_union}} = \\ &=
        \parens[\Big]{ \underbrace{\op*{Bound}(N'_{\Gamma,x}) \cap \Gamma }_\varnothing } \cup \parens[\Big]{ \underbrace{\set{ x } \cap \Gamma}_\varnothing }
      \end{align*}

      \item If \( x \in \Gamma \), then \( M' = \qabs n N'_{\Gamma,n}[x \mapsto n] \), where \( n = \sharp(\op*{Free}(N) \cup \Gamma) \).

      By the inductive hypothesis, we have \( N \aequiv N'_{\Gamma,n} \). \Fullref{thm:substitution_on_alpha_equivalent_terms} implies that
      \begin{equation}\label{eq:alg:separation_of_free_and_bound_variables/proof/final_ind}
        N[x \mapsto n] \aequiv N'_{\Gamma,n}[x \mapsto n].
      \end{equation}

      Let
      \begin{align*}
        a &\coloneqq n && A \coloneqq N'_{\Gamma,n}[x \mapsto n], \\
        b &\coloneqq x && B \coloneqq N.
      \end{align*}
      so that, in the notation of \ref{inf:thm:alpha_equivalence_simplified/ren}, \eqref{eq:alg:separation_of_free_and_bound_variables/proof/final_ind} becomes
      \begin{equation*}
        A \aequiv B[b \mapsto a].
      \end{equation*}

      By definition, \( a = n \) is not free in \( B = N \). Thus, we can apply \ref{inf:thm:alpha_equivalence_simplified/ren} and obtain \( M \aequiv M'_\Gamma \).

      Furthermore, by the inductive hypothesis,
      \begin{equation*}
        \op*{Bound}(N'_{\Gamma,n}) \cap \parens[\Big]{ \Gamma \cup \set{ n } } = \varnothing,
      \end{equation*}
      hence, again,
      \begin{align*}
        \op*{Bound}(M'_\Gamma) \cap \Gamma
        &=
        \parens[\Big]{ \op*{Bound}(N'_{\Gamma,n}) \cup \set{ n } } \cap \Gamma
        \reloset {\eqref{eq:thm:union_intersection_distributivity/intersection_over_union}} = \\ &=
        \parens[\Big]{ \underbrace{\op*{Bound}(N'_{\Gamma,n}) \cap \Gamma }_\varnothing } \cup \parens[\Big]{ \underbrace{\set{ n } \cap \Gamma}_\varnothing }.
      \end{align*}
    \end{itemize}
  \end{itemize}
\end{defproof}
