\section{Complex numbers}\label{sec:complex_numbers}

\Fullref{thm:ordered_field_not_algebraically_closed} implies that the \hyperref[def:field]{field} \( \BbbR \) of \hyperref[def:real_numbers]{real numbers} is not \hyperref[def:algebraically_closed_field]{algebraically closed}. This motivates the introduction of complex numbers.

\paragraph{Complex numbers}

\begin{definition}\label{def:complex_numbers}\mimprovised
  We define the \hyperref[def:field]{field} \( \BbbC \) of \term[bg=комплексни числа (\cite[296]{ИлинСадовничиСендов1984АнализТом1}), ru=комплексные числа (\cite[12]{Маркушевич1967АналитическиеФункцииТом1}), en=complex numbers (\cite[1]{Ahlfors1979ComplexAnalysis})]{complex numbers} as the \hyperref[def:algebra_over_ring/quotient]{quotient algebra}
  \begin{equation}\label{eq:def:complex_numbers/quotient}
    \BbbC \coloneqq \BbbR[i] / \braket{ i^2 + 1 }.
  \end{equation}

  It is indeed a field because, by \fullref{thm:axx_byy_irreducible} and \fullref{thm:homogeneous_polynomial_constant}, the polynomial \( i^2 + 1 \) is irreducible in \( \BbbR \) and, by \fullref{thm:quotient_by_irreducible_polynomial}, \( \BbbC \) is a \hyperref[def:field/submodel]{field extension} of \( \BbbR \) of \hyperref[def:field_extension_degree]{degree} \( 2 \).

  Here \( i \) is an indeterminate with no inherent semantics --- its semantics stem from the algebraic properties of the defined quotient. \Fullref{thm:representatives_in_univariate_polynomial_quotient_set} allows us to unambiguously identify each coset in \( \BbbC \) with a canonical representative; thus, we regard members of \( \BbbC \) as linear polynomials with real coefficients:
  \begin{equation}\label{eq:def:complex_numbers/number}
    z = a + bi.
  \end{equation}

  The constant coefficient is written first by convention since we embed \( \BbbR \) as constant polynomials.

  We denote the constant coefficient of \( z \) by \( \real z \) and call it the \term[bg=реална част (\cite[296]{ИлинСадовничиСендов1984АнализТом1}), ru=действительная часть (\cite[12]{Маркушевич1967АналитическиеФункцииТом1}), en=real (part) (\cite[1]{Ahlfors1979ComplexAnalysis})]{real part}. We denote the linear coefficient by \( \imag z \) and call it the \term[bg=имагинерна част (\cite[296]{ИлинСадовничиСендов1984АнализТом1}), ru=мнимая часть (\cite[12]{Маркушевич1967АналитическиеФункцииТом1}), en=imaginary part (\cite[1]{Ahlfors1979ComplexAnalysis})]{imaginary part}. If \( \real z = 0 \), we say that \( z \) is \term[ru=чисто мнимое (число) (\cite[12]{Маркушевич1967АналитическиеФункцииТом1}), en=purely imaginary (number) (\cite[1]{Ahlfors1979ComplexAnalysis})]{purely imaginary}. We call \( i \) the \term[en=imaginary unit (\cite[1]{Ahlfors1979ComplexAnalysis}), ru=мнимая единица (\cite[12]{Маркушевич1967АналитическиеФункцииТом1})]{imaginary unit}.

  Since, by definition of quotient ring, we have \( i^2 + 1 = 0 \), it follows that
  \begin{equation}\label{eq:def:complex_numbers/i_square}
    i^2 = -1.
  \end{equation}

  Multiplication in \( \BbbC \) is thus given by
  \begin{equation}\label{eq:def:complex_numbers/multiplication}
    (a + bi) (c + di) = ac + adi + bci + bdi^2 = (ac - bd) + (ad + bc) i.
  \end{equation}
\end{definition}
\begin{comments}
  \item We consider the following an indispensable part of \( \BbbR \):
  \begin{itemize}
    \item Complex conjugation defined in \fullref{def:complex_conjugation}.
    \item The absolute value defined in \fullref{def:complex_absolute_value}, as well as the induced topology.
    \item Trigonometric forms of complex numbers defined in \fullref{def:complex_numbers_trigonometric_form}.
    \item The \hyperref[def:algebraically_closed_field]{algebraic closure} shown in \fullref{thm:fundamental_theorem_of_algebra}.
  \end{itemize}

  \item As a consequence of \fullref{thm:ordered_field_not_algebraically_closed}, the complex numbers cannot be \hyperref[def:totally_ordered_set]{totally ordered} in a way that would make it an \hyperref[def:ordered_semiring]{ordered ring}.

  \item The definition as a quotient ring is given after knowledge of complex numbers is already assumed, for which reason complex numbers are often defined in a more elementary form --- by endowing \( \BbbR^2 \) with multiplication defined by \eqref{eq:def:complex_numbers/multiplication}. The quotient ring construction is given in \incite[example III.4.8]{Aluffi2009Algebra}, where it is shown to be equivalent to the vector space construction.

  We discuss an alternative definition via matrices in \fullref{thm:complex_numbers_as_matrices}.
\end{comments}

\begin{proposition}\label{thm:complex_numbers_as_matrices}
  The subset of the \hyperref[thm:matrix_algebra]{matrix algebra} \( \BbbR^{2 \times 2} \) of matrices of the form
  \begin{equation}\label{eq:thm:complex_numbers_as_matrices}
    \begin{pmatrix}
      a  & b \\
      -b & a
    \end{pmatrix}
  \end{equation}
  is a \hyperref[def:algebra_over_ring/submodel]{subalgebra} and is isomorphic to the \( \BbbR \)-algebra \( \BbbC \) of \hyperref[def:complex_numbers]{complex numbers}.
\end{proposition}
\begin{proof}
  Denote by \( C \) the subset of all matrices of the form \eqref{eq:thm:complex_numbers_as_matrices}.

  It is closed under addition and scalar multiplication and contains both the additive and multiplicative identities. It is also closed under additive inverses. Therefore, \( C \) is a vector subspace of \( \BbbR^{2 \times 2} \). The function sending \( a + bi \) to \eqref{thm:complex_numbers_as_matrices} is a vector space isomorphism from \( \BbbC \) to \( C \).

  Furthermore, we have
  \begin{equation*}
    \begin{pmatrix}
      a  & b \\
      -b & a
    \end{pmatrix}
    \cdot
    \begin{pmatrix}
      c  & d \\
      -d & c
    \end{pmatrix}
    =
    \begin{pmatrix}
      ac - bd  & ad + bc \\
      -ad - bc & -bd + ac.
    \end{pmatrix}
  \end{equation*}

  Thus, \( C \) is closed under multiplication, making it a subalgebra of \( \BbbR^{2 \times 2} \). Comparing the product with \eqref{eq:def:complex_numbers/multiplication}, we conclude that \( C \) is isomorphic to \( \BbbC \) as an algebra.
\end{proof}

\paragraph{Complex conjugation}

\begin{definition}\label{def:complex_conjugation}\mcite[]{Маркушевич1967АналитическиеФункцииТом1}
  We define the \term[bg=(комплексно) спрегнато (число) (\cite[298]{ИлинСадовничиСендов1984АнализТом1}), ru=комплексно сопряжённые (числа), en=complex conjugate (\cite[7]{Ahlfors1979ComplexAnalysis})]{complex conjugate} of the \hyperref[def:complex_numbers]{complex number} \( a + bi \) as \( a - bi \).
\end{definition}

\begin{proposition}\label{thm:def:complex_conjugation}
  \hyperref[def:complex_conjugation]{Complex conjugation} has the following basic properties:
  \begin{thmenum}
    \thmitem{thm:def:complex_conjugation/components} We have
    \begin{align*}
      \real z = \frac {z + \oline z} 2
      &&\T{and}&&
      \imag z = \frac {z - \oline z} {2i}.
    \end{align*}

    \thmitem{thm:def:complex_conjugation/automorphism} Conjugation is a \hyperref[def:field/homomorphism]{field automorphism} of the complex numbers.
    \thmitem{thm:def:complex_conjugation/fixed} The \hyperref[def:fixed_point]{fixed points} under complex conjugation are the \hyperref[def:real_numbers]{real numbers}.
  \end{thmenum}
\end{proposition}
\begin{proof}
  \SubProofOf{thm:def:complex_conjugation/components} We have
  \begin{equation*}
    \frac {a + bi + a - bi} 2 = a = \real(a + bi)
  \end{equation*}
  and similarly
  \begin{equation*}
    \frac {a + bi - a + bi} {2i} = b = \imag(a + bi)
  \end{equation*}

  \SubProofOf{thm:def:complex_conjugation/automorphism} Straightforward.
  \SubProofOf{thm:def:complex_conjugation/fixed} Every real number is clearly a fixed point.

  Conversely, we can cancel \( a \) in \( a + bi = a - bi \) to obtain \( 2bi = 0 \). Since \( 2 \neq 0 \) and \( i \neq 0 \), it remains for \( b = 0 \).
\end{proof}

\paragraph{Absolute values}

\begin{definition}\label{def:absolute_value}\mcite[def. 9.1]{Jacobson1989AlgebraPart2}
  An \term{absolute value} on a \hyperref[def:field]{field} \( \BbbK \) is a map \( \abs{\anon}: \BbbK \to \BbbR \) that satisfies the following properties:
  \begin{thmenum}
    \thmitem{def:absolute_value/positive_definite} \hyperref[def:real_function_definiteness]{Positive definiteness}: \( \abs{x} \geq 0 \) for all \( x \) and \( \abs{x} = 0 \) if and only if \( x = 0_\BbbK \).
    \thmitem{def:absolute_value/multiplicative} \hyperref[def:multiplicative_function]{Multiplicativity}: \( \abs{xy} = \abs{x} \cdot \abs{y} \).
    \thmitem{def:absolute_value/subadditive} \hyperref[def:additive_function/sub]{Subadditivity}: \( \abs{x + y} \leq \abs{x} + \abs{y} \).
  \end{thmenum}
\end{definition}
\begin{comments}
  \item Absolute values are used to define \hyperref[def:norm]{norms}, however they are also special cases of norms.
  \item Unless otherwise noted, for complex numbers we will assume that \( \abs{\anon} \) refers to the absolute value defined in \fullref{def:complex_absolute_value}.
\end{comments}

\begin{definition}\label{def:complex_absolute_value}
  We define an \hyperref[def:absolute_value]{absolute value} for \hyperref[def:complex_numbers]{complex numbers} as
  \begin{equation}\label{eq:def:complex_absolute_value}
    \abs{z} \coloneqq \sqrt{z \cdot \oline z},
  \end{equation}
  where \( \oline z \) is the \hyperref[def:complex_conjugation]{complex conjugate} of \( z \).

  More explicitly, if \( z = a + bi \), then
  \begin{equation}\label{eq:def:complex_absolute_value/algebraic}
    \abs{a + bi} = \sqrt{(a + bi) \cdot (a - bi)} = \sqrt{a^2 + b^2}.
  \end{equation}
\end{definition}
\begin{comments}
  \item We will define square roots for general complex numbers in \fullref{def:complex_nth_root}. At this point we only need square roots for real numbers as defined in \fullref{def:real_nth_root}.
\end{comments}
\begin{defproof}
  \SubProofOf[def:real_function_definiteness]{positive definiteness} Clearly \( \abs{0} = 0 \). If \( a \neq 0 \), then \fullref{thm:ordered_ring_power} implies that \( a^2 \) is positive and \( b^2 \) is nonnegative, while \fullref{thm:def:ordered_semiring/strict_sum} implies that their sum is positive. Similarly, if \( b \neq 0 \), the sum \( a^2 + b^2 \) is strictly positive.

  \SubProofOf[def:absolute_value/multiplicative]{multiplicativity} We have
  \begin{balign*}
    \abs{(a + bi) \cdot (c + di)}
    &=
    \abs{(ac - bd) + (ad + bc)i}
    = \\ &=
    \sqrt{(ac - bd)^2 + (ad + bc)^2}
    = \\ &=
    \sqrt{a^2 c^2 - \cancel{2abcd} + b^2 d^2 + a^2 d^2 + \cancel{2abcd} + b^2 c^2}
    = \\ &=
    \sqrt{a^2 (c^2 + d^2) + b^2 (d^2 + c^2)}
    = \\ &=
    \sqrt{(a^2 + b^2) \cdot (c^2 + d^2)}
    \reloset {\eqref{eq:thm:def:real_nth_root/multiplicative}} = \\ &=
    \sqrt{a^2 + b^2} \cdot \sqrt{c^2 + d^2}
    = \\ &=
    \abs{a + bi} \cdot \abs{c + di}.
  \end{balign*}

  \SubProofOf[def:additive_function/sub]{subadditivity} We have
  \begin{balign*}
    \abs{(a + bi) + (c + di)}
    &=
    (a + c)^2 + (b + d)^2
    = \\ &=
    (a^2 + b^2) + (c^2 + d^2) + 2(ac + bd)
    = \\ &=
    \abs{a + bi} + \abs{c + di} + 2(ac + bd).
  \end{balign*}
\end{defproof}

\begin{proposition}\label{thm:complex_multiplicative_inverse}
  The \hyperref[def:semiring]{multiplicative inverse} of the nonzero \hyperref[def:complex_numbers]{complex number} \( z = a + bi \) is
  \begin{equation}\label{eq:thm:complex_multiplicative_inverse}
    \frac {\oline z} {{\abs z}^2} = \frac {a - bi} {\sqrt{a^2 + b^2}}
  \end{equation}
\end{proposition}
\begin{proof}
  Trivial considering how we defined the complex absolute value.
\end{proof}

\paragraph{Fundamental theorem of algebra}

\begin{theorem}[Fundamental theorem of algebra]\label{thm:fundamental_theorem_of_algebra}
  The field \( \BbbC \) of \hyperref[def:complex_numbers]{complex numbers} is \hyperref[def:algebraically_closed_field]{algebraically closed}.
\end{theorem}
\begin{proof}
  \todo{Prove}
\end{proof}
