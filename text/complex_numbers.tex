\section{Complex numbers}\label{sec:complex_numbers}

\Fullref{thm:ordered_field_not_algebraically_closed} implies that the \hyperref[def:field]{field} \( \BbbR \) of \hyperref[def:real_numbers]{real numbers} is not \hyperref[def:algebraically_closed_field]{algebraically closed}. This motivates the introduction of complex numbers.

\paragraph{Complex numbers}

\begin{definition}\label{def:complex_numbers}\mimprovised
  We define the \hyperref[def:field]{field} \( \BbbC \) of \term[bg=комплексни числа (\cite[296]{ИлинСадовничиСендов1984АнализТом1}), ru=комплексные числа (\cite[12]{Маркушевич1967АналитическиеФункцииТом1}), en=complex numbers (\cite[1]{Ahlfors1979ComplexAnalysis})]{complex numbers} as the \hyperref[def:algebra_over_ring/quotient]{quotient algebra}
  \begin{equation}\label{eq:def:complex_numbers/quotient}
    \BbbC \coloneqq \BbbR[i] / \braket{ i^2 + 1 }.
  \end{equation}

  It is indeed a field because, by \fullref{thm:axn_byn_irreducible} and \fullref{thm:homogeneous_polynomial_constant}, the polynomial \( i^2 + 1 \) is irreducible in \( \BbbR \) and, by \fullref{thm:quotient_by_irreducible_polynomial}, \( \BbbC \) is a \hyperref[def:field/submodel]{field extension} of \( \BbbR \) of \hyperref[def:field_extension_degree]{degree} \( 2 \).

  Here \( i \) is an indeterminate with no inherent semantics --- its semantics stem from the algebraic properties of the defined quotient. \Fullref{thm:representatives_in_univariate_polynomial_quotient_set} allows us to unambiguously identify each coset in \( \BbbC \) with a canonical representative; thus, we regard members of \( \BbbC \) as linear polynomials with real coefficients:
  \begin{equation}\label{eq:def:complex_numbers/number}
    z = a + bi.
  \end{equation}

  The constant coefficient is written first by convention since we embed \( \BbbR \) as constant polynomials.

  We denote the constant coefficient of \( z \) by \( \real z \) and call it the \term[bg=реална част (\cite[296]{ИлинСадовничиСендов1984АнализТом1}), ru=действительная часть (\cite[12]{Маркушевич1967АналитическиеФункцииТом1}), en=real (part) (\cite[1]{Ahlfors1979ComplexAnalysis})]{real part}. We denote the linear coefficient by \( \imag z \) and call it the \term[bg=имагинерна част (\cite[296]{ИлинСадовничиСендов1984АнализТом1}), ru=мнимая часть (\cite[12]{Маркушевич1967АналитическиеФункцииТом1}), en=imaginary part (\cite[1]{Ahlfors1979ComplexAnalysis})]{imaginary part}. If \( \real z = 0 \), we say that \( z \) is \term[ru=чисто мнимое (число) (\cite[12]{Маркушевич1967АналитическиеФункцииТом1}), en=purely imaginary (number) (\cite[1]{Ahlfors1979ComplexAnalysis})]{purely imaginary}. We call \( i \) the \term[en=imaginary unit (\cite[1]{Ahlfors1979ComplexAnalysis}), ru=мнимая единица (\cite[12]{Маркушевич1967АналитическиеФункцииТом1})]{imaginary unit}.

  Since, by definition of quotient ring, we have \( i^2 + 1 = 0 \), it follows that
  \begin{equation}\label{eq:def:complex_numbers/i_square}
    i^2 = -1.
  \end{equation}

  Multiplication in \( \BbbC \) is thus given by
  \begin{equation}\label{eq:def:complex_numbers/multiplication}
    (a + bi) (c + di) = ac + adi + bci + bdi^2 = (ac - bd) + (ad + bc) i.
  \end{equation}
\end{definition}
\begin{comments}
  \item We consider the following an indispensable part of \( \BbbR \):
  \begin{itemize}
    \item Complex conjugation defined in \fullref{def:complex_conjugation}.
    \item The absolute value defined in \fullref{def:complex_absolute_value}, as well as the induced topology.
    \item Trigonometric forms of complex numbers defined in \fullref{def:complex_numbers_trigonometric_form}.
    \item The \hyperref[def:algebraically_closed_field]{algebraic closure} shown in \fullref{thm:fundamental_theorem_of_algebra}.
  \end{itemize}

  \item As a consequence of \fullref{thm:ordered_field_not_algebraically_closed}, the complex numbers cannot be \hyperref[def:totally_ordered_set]{totally ordered} in a way that would make it an \hyperref[def:ordered_semiring]{ordered ring}.

  \item The definition as a quotient ring is given after knowledge of complex numbers is already assumed, for which reason complex numbers are often defined in a more elementary form --- by endowing \( \BbbR^2 \) with multiplication defined by \eqref{eq:def:complex_numbers/multiplication}. The quotient ring construction is given in \incite[example III.4.8]{Aluffi2009Algebra}, where it is shown to be equivalent to the vector space construction.

  We discuss an alternative definition via matrices in \fullref{thm:complex_numbers_as_matrices}.
\end{comments}

\begin{proposition}\label{thm:complex_numbers_as_matrices}
  The subset of the \hyperref[thm:matrix_algebra]{matrix algebra} \( \BbbR^{2 \times 2} \) of matrices of the form
  \begin{equation}\label{eq:thm:complex_numbers_as_matrices}
    \begin{pmatrix}
      a  & b \\
      -b & a
    \end{pmatrix}
  \end{equation}
  is a \hyperref[def:algebra_over_ring/submodel]{subalgebra} and is isomorphic to the \( \BbbR \)-algebra \( \BbbC \) of \hyperref[def:complex_numbers]{complex numbers}.
\end{proposition}
\begin{proof}
  Denote by \( C \) the subset of all matrices of the form \eqref{eq:thm:complex_numbers_as_matrices}.

  It is closed under addition and scalar multiplication and contains both the additive and multiplicative identities. It is also closed under additive inverses. Therefore, \( C \) is a vector subspace of \( \BbbR^{2 \times 2} \). The function sending \( a + bi \) to \eqref{thm:complex_numbers_as_matrices} is a vector space isomorphism from \( \BbbC \) to \( C \).

  Furthermore, we have
  \begin{equation*}
    \begin{pmatrix}
      a  & b \\
      -b & a
    \end{pmatrix}
    \cdot
    \begin{pmatrix}
      c  & d \\
      -d & c
    \end{pmatrix}
    =
    \begin{pmatrix}
      ac - bd  & ad + bc \\
      -ad - bc & -bd + ac.
    \end{pmatrix}
  \end{equation*}

  Thus, \( C \) is closed under multiplication, making it a subalgebra of \( \BbbR^{2 \times 2} \). Comparing the product with \eqref{eq:def:complex_numbers/multiplication}, we conclude that \( C \) is isomorphic to \( \BbbC \) as an algebra.
\end{proof}

\paragraph{Algebraic and transcendental numbers}

In this paragraph, we also regard real numbers as atoms and not as lower cuts.

\begin{definition}\label{def:algebraic_number}\mcite[222]{ГеновМиховскиМоллов1991Алгебра}
  We say that a \hyperref[def:complex_numbers]{complex number} is \term[bg=алгебрично (число), ru=алгебраическое (число) (\cite[358]{Курош1968КурсВысшейАлгебры}), en=algebraic (number) (\cite[16]{Carothers2000RealAnalysis})]{algebraic} if it is an \hyperref[def:algebraic_element]{algebraic element} over the field \( \BbbQ \) of \hyperref[def:rational_numbers]{rational numbers}.
\end{definition}

\begin{definition}\label{def:transcendental_number}\mcite[222]{ГеновМиховскиМоллов1991Алгебра}
  We say that a \hyperref[def:complex_numbers]{complex number} is \term[bg=трансцендентно (число), ru=трансцендентное (число) (\cite[358]{Курош1968КурсВысшейАлгебры}), en=transcendental (number) (\cite[277]{Jacobson1985BasicAlgebraI})]{transcendental} if it is not \hyperref[def:algebraic_number]{algebraic}.
\end{definition}

\begin{theorem}[Lindemann-Weierstrass theorem]\label{thm:lindemann_weierstrass}\mcite[277]{Jacobson1985BasicAlgebraI}
  Let \( \alpha_1, \ldots, \alpha_n \) be \hyperref[def:complex_numbers]{complex numbers} that are \hyperref[def:algebraic_number]{algebraic} over \( \BbbQ \). If they are \hyperref[def:linear_dependence]{linearly independent} over \( \BbbQ \), their complex exponentials \( e^{u_1}, \ldots, e^{u_n} \) are \hyperref[def:algebraic_dependence]{algebraically independent} over \( \BbbQ \).
\end{theorem}
\begin{comments}
  \item We will not prove this theorem. A proof can be found in \cite[277]{Jacobson1985BasicAlgebraI}.
\end{comments}

\begin{corollary}\label{thm:eulers_constant_is_transcendental}
  \hyperref[def:exponential_function]{Euler's constant} \( e \) is \hyperref[def:transcendental_number]{transcendental}.
\end{corollary}
\begin{proof}
  Since \( 1 \) is by itself linearly independent over \( \BbbQ \), \fullref{thm:lindemann_weierstrass} implies that \( e \) is by algebraically independent over \( \BbbQ \). Hence, there exists no rational polynomial whose root is \( e \), and thus \( e \) satisfies \fullref{def:transcendental_element}.
\end{proof}

\begin{corollary}\label{thm:pi_is_transcendental}\mcite[454]{Knapp2016BasicAlgebra}
  The number \hyperref[def:pi]{\( \pi \)} is \hyperref[def:transcendental_number]{transcendental}.
\end{corollary}
\begin{proof}
  Suppose that \( \pi \) is algebraic over \( \BbbQ \).

  The complex unit \( i \) is algebraic by construction --- it has minimal polynomial \( X^2 + 1 \). Then \( i\pi \) is also algebraic because it belongs to \( \BbbQ[\pi][i] \), which must be finite-dimensional.

  Furthermore, \( i\pi \) is by itself linearly independent over \( \BbbQ \) --- it can only be linearly dependent if it is the zero vector, which it is not. So \fullref{thm:lindemann_weierstrass} implies that \( e^{i\pi} \) is transcendental. But \( e^{i\pi} \) equals \( -1 \), which is a root of the integer polynomial \( X + 1 \).

  The obtained contradiction shows that our initial assumption of \( \pi \) being algebraic over \( \BbbQ \) is false; that is, \( \pi \) must be transcendental.
\end{proof}

\begin{example}\label{ex:polynomials_over_pi}
  \Fullref{thm:pi_is_transcendental} implies that the polynomial ring \( \BbbQ[X] \) can be embedded into \( \BbbR \) via \( \Phi_\pi: \BbbQ[X] \to \BbbR \). We can thus identify a rational polynomial
  \begin{equation*}
    f(X) = \sum_{k=0}^n a_k X^k
  \end{equation*}
  with the number
  \begin{equation*}
    f(\pi) = \sum_{k=0}^n a_k \pi^k.
  \end{equation*}
\end{example}

\paragraph{Square roots}

\begin{definition}\label{def:principal_real_square_root}\mimprovised
  We extend \hyperref[def:nth_root]{principal square roots} as defined in \fullref{def:principal_nonnegative_nth_root} to arbitrary \hyperref[def:real_numbers]{real numbers} by defining, for any \( x > 0 \),
  \begin{equation}\label{eq:def:principal_real_square_root/negative}
    \sqrt {-x} \coloneqq i \sqrt x.
  \end{equation}
\end{definition}
\begin{comments}
  \item We avoid defining square roots for arbitrary complex numbers, or every higher degree roots for negative real numbers, since it is difficult to make a decision which root should be principal. \incite[16]{АлександровМаркушевичХинчинИПр1951ЭнциклопедияТом1} defines principal roots for any complex number based on their \hyperref[def:complex_numbers_trigonometric_form]{trigonometric form}, however we prefer to avoid such a choice.
\end{comments}

\begin{proposition}\label{thm:principal_real_square_root_sign}
  We can infer the \hyperref[def:signum]{sign} of a real number \( x \) from its \hyperref[def:principal_real_square_root]{principal square root} \( \sqrt x \):
  \begin{thmenum}
    \thmitem{thm:principal_real_square_root_sign/positive} \( x > 0 \) if and only if \( \sqrt x > 0 \).
    \thmitem{thm:principal_real_square_root_sign/zero} \( x = 0 \) if and only if \( \sqrt x = 0 \).
    \thmitem{thm:principal_real_square_root_sign/negative} \( x < 0 \) if and only if \( \sqrt{-x} > 0 \) and \( \sqrt x = i \sqrt{-x} \).
  \end{thmenum}
\end{proposition}
\begin{proof}
  There are two possibilities for \( \sqrt x \):
  \begin{itemize}
    \item If \fullref{def:principal_nonnegative_nth_root} is used, then both \( x \) and \( \sqrt x \) are nonnegative real numbers, and, due to \fullref{thm:def:principal_nonnegative_nth_root/zero}, both are either strictly positive or zero simultaneously.

    This handles \fullref{thm:real_quadratic_discriminant/positive} and \fullref{thm:real_quadratic_discriminant/zero}.

    \item If \fullref{def:principal_real_square_root} is used, then \( x < 0 \) and \( \sqrt x \) is by definition \( i \sqrt (-x) \), where \( \sqrt (-x) \) is defined via \fullref{def:principal_nonnegative_nth_root} and is hence a positive real number.

    This handles \fullref{thm:real_quadratic_discriminant/negative}.
  \end{itemize}
\end{proof}

\begin{proposition}\label{thm:real_quadratic_polynomial_roots}
  The \hyperref[def:root_of_polynomial]{roots} of the \hyperref[def:real_numbers]{real} \hyperref[def:polynomial_degree_terminology]{quadratic} \hyperref[def:polynomial_algebra]{polynomial} \( f(X) = a X^2 + b X + c \) are
  \begin{equation}\label{eq:def:conjugate_quadratic_roots}
    \frac {-b \pm \sqrt{b^2 - 4ac}} {2a}.
  \end{equation}
\end{proposition}
\begin{proof}
  Follows from \fullref{thm:quadratic_polynomial_roots}.
\end{proof}

\begin{remark}\label{rem:conjugate_quadratic_roots}\mimprovised
  Consider a \hyperref[def:rational_numbers]{rational} \hyperref[def:polynomial_degree_terminology]{quadratic} \hyperref[def:polynomial_algebra]{polynomial} \( f(X) = a X^2 + b X + c \).

  If the \hyperref[def:principal_real_square_root]{principal square root} of the \hyperref[def:discriminant]{discriminant} \( D(f) = b^2 - 4ac \) is not rational (i.e. irrational or non-real), \fullref{thm:quadratic_extension_square_root} implies that \( \BbbQ(\sqrt{D(f)}) \) is a \hyperref[def:field_extension_degree]{quadratic field extension}.

  \Fullref{thm:quadratic_extension_conjugate} then implies that the roots
  \begin{equation*}
    \frac {-b \pm \sqrt{D(f)}} {2a}
  \end{equation*}
  given by \fullref{thm:real_quadratic_polynomial_roots} are \hyperref[def:conjugate_algebraic_element]{conjugate algebraic elements} over \( \BbbQ(D(f)) \).

  We will refer to them as \term{conjugate roots}.
\end{remark}
\begin{comments}
  \item This notion coincides with \hyperref[def:complex_conjugation]{complex conjugation} when \( D(f) < 0 \), but also justifies generalizing the terminology to cases where \( D(f) \) has an irrational root (e.g. \( 1 \pm \sqrt 3 \) are conjugate roots of \( X^2 - 2X - 2 \)).
\end{comments}

\begin{corollary}\label{thm:real_quadratic_discriminant}
  Fix a real quadratic polynomial \( f(X) = a X^2 + b X + c \). Based on its \hyperref[def:discriminant]{discriminant} \( D(f) = b^2 - 4ac \), we can infer the following:
  \begin{thmenum}
    \thmitem{thm:real_quadratic_discriminant/positive} \( D(f) > 0 \) if and only if \( f(X) \) has two distinct real roots.
    \thmitem{thm:real_quadratic_discriminant/zero} \( D(f) = 0 \) if and only if \( f(X) \) has a real double root.
    \thmitem{thm:real_quadratic_discriminant/negative} \( D(f) < 0 \) if and only if \( f(X) \) has two distinct non-real roots.
  \end{thmenum}
\end{corollary}
\begin{proof}
  \Fullref{thm:real_quadratic_polynomial_roots} implies that the roots of \( f(X) \) are
  \begin{align*}
    x_+ = \frac {-b + \sqrt{D(f)}} {2a}
    &&
    x_- = \frac {-b - \sqrt{D(f)}} {2a}
  \end{align*}

  \SubProofOf{thm:real_quadratic_discriminant/positive}
  \SufficiencySubProof* If \( D(f) > 0 \), \fullref{thm:principal_real_square_root_sign/positive} implies that \( \sqrt{D(f)} \) is positive, and hence \( x_+ \) and \( x_- \) are distinct real numbers.

  \NecessitySubProof* Note that \( \sqrt{D(f)} = a (x_+ - x_-) \). If \( x_+ \) and \( x_- \) are real and distinct, \( \sqrt{D(f)} \) must be a nonzero real number. By \fullref{thm:principal_real_square_root_sign}, \( \sqrt{D(f)} \) can only be positive, in which case \( D(f) > 0 \).

  \SubProofOf{thm:real_quadratic_discriminant/zero}
  \SufficiencySubProof* If \( D(f) = 0 \), then \( x_+ = x_- \).
  \NecessitySubProof* Conversely, if \( x_+ = x_- \), then \( \sqrt{D(f)} = a(x_+ - x_-) = 0 \).

  \SubProofOf{thm:real_quadratic_discriminant/negative}
  \SufficiencySubProof* If \( D(f) < 0 \), \fullref{thm:principal_real_square_root_sign/negative} implies that \( \sqrt{D(f)} = i \sqrt{-D(f)} \), where \( d \coloneqq \sqrt{-D(f)} \) is positive. Then
  \begin{equation*}
    x_\pm = \frac {-b \pm i d} {2a},
  \end{equation*}
  hence \( x_+ \) and \( x_- \) are distinct complex numbers that are not real.

  \NecessitySubProof Suppose that \( x_+ \) and \( x_- \) are distinct and not real. Since
  \begin{equation*}
    x_\pm = \frac {-b} {2a} + \frac {\pm \sqrt{D(f)}} {2a},
  \end{equation*}
  their nonzero imaginary parts can only come from \( \sqrt{D(f)} \). By \fullref{thm:principal_real_square_root_sign}, \( \sqrt{D(f)} \) can only be non-real if \( D(f) < 0 \).
\end{proof}

\paragraph{Complex conjugation}

\begin{definition}\label{def:complex_conjugation}\mcite[15]{Маркушевич1967АналитическиеФункцииТом1}
  We define the \term[bg=(комплексно) спрегнато (число) (\cite[298]{ИлинСадовничиСендов1984АнализТом1}), ru=сопряжённые (числа), en=complex conjugate (\cite[7]{Ahlfors1979ComplexAnalysis})]{complex conjugate} of the \hyperref[def:complex_numbers]{complex number} \( a + bi \) as \( a - bi \).
\end{definition}
\begin{comments}
  \item It follows from \fullref{thm:quadratic_extension_conjugate} that \( a - bi \) is the unique \hyperref[def:conjugate_algebraic_element]{conjugate algebraic element} of \( a + bi \) over \( \BbbR \).
\end{comments}

\begin{proposition}\label{thm:def:complex_conjugation}
  \hyperref[def:complex_conjugation]{Complex conjugation} has the following basic properties:
  \begin{thmenum}
    \thmitem{thm:def:complex_conjugation/components} We have
    \begin{align*}
      \real z = \frac {z + \oline z} 2
      &&\T{and}&&
      \imag z = \frac {z - \oline z} {2i}.
    \end{align*}

    \thmitem{thm:def:complex_conjugation/automorphism} Conjugation is a \hyperref[def:field/homomorphism]{field automorphism} of the complex numbers.
    \thmitem{thm:def:complex_conjugation/fixed} The \hyperref[def:function_fixed_point]{fixed points} under complex conjugation are the \hyperref[def:real_numbers]{real numbers}.
  \end{thmenum}
\end{proposition}
\begin{proof}
  \SubProofOf{thm:def:complex_conjugation/components} We have
  \begin{equation*}
    \frac {a + bi + a - bi} 2 = a = \real(a + bi)
  \end{equation*}
  and similarly
  \begin{equation*}
    \frac {a + bi - a + bi} {2i} = b = \imag(a + bi)
  \end{equation*}

  \SubProofOf{thm:def:complex_conjugation/automorphism} Straightforward.
  \SubProofOf{thm:def:complex_conjugation/fixed} Every real number is clearly a fixed point.

  Conversely, we can cancel \( a \) in \( a + bi = a - bi \) to obtain \( 2bi = 0 \). Since \( 2 \neq 0 \) and \( i \neq 0 \), it remains for \( b = 0 \).
\end{proof}

\paragraph{Absolute values}

\begin{definition}\label{def:absolute_value}\mcite[def. 9.1]{Jacobson1989BasicAlgebraII}
  An \term{absolute value} on a \hyperref[def:field]{field} \( \BbbK \) is a map \( \abs{\anon}: \BbbK \to \BbbR \) that satisfies the following properties:
  \begin{thmenum}
    \thmitem{def:absolute_value/positive_definite} \hyperref[def:real_function_definiteness]{Positive definiteness}: \( \abs{x} \geq 0 \) for all \( x \) and \( \abs{x} = 0 \) if and only if \( x = 0_\BbbK \).
    \thmitem{def:absolute_value/multiplicative} \hyperref[def:multiplicative_function]{Multiplicativity}: \( \abs{xy} = \abs{x} \cdot \abs{y} \).
    \thmitem{def:absolute_value/subadditive} \hyperref[def:additive_function/sub]{Subadditivity}: \( \abs{x + y} \leq \abs{x} + \abs{y} \).
  \end{thmenum}
\end{definition}
\begin{comments}
  \item Absolute values are used to define \hyperref[def:norm]{norms}, however they are also special cases of norms.
  \item Unless otherwise noted, for complex numbers we will assume that \( \abs{\anon} \) refers to the absolute value defined in \fullref{def:complex_absolute_value}.
\end{comments}

\begin{definition}\label{def:complex_absolute_value}
  We define an \hyperref[def:absolute_value]{absolute value} for \hyperref[def:complex_numbers]{complex numbers} as
  \begin{equation}\label{eq:def:complex_absolute_value}
    \abs{z} \coloneqq \sqrt{z \cdot \oline z},
  \end{equation}
  where \( \oline z \) is the \hyperref[def:complex_conjugation]{complex conjugate} of \( z \).

  More explicitly, if \( z = a + bi \), then
  \begin{equation}\label{eq:def:complex_absolute_value/algebraic}
    \abs{a + bi} = \sqrt{(a + bi) \cdot (a - bi)} = \sqrt{a^2 + b^2}.
  \end{equation}
\end{definition}
\begin{comments}
  \item We will define square roots for general complex numbers in \fullref{def:principal_real_square_root}. At this point we only need square roots for real numbers as defined in \fullref{def:principal_nonnegative_nth_root}.
\end{comments}
\begin{defproof}
  \SubProofOf[def:real_function_definiteness]{positive definiteness} Clearly \( \abs{0} = 0 \). If \( a \neq 0 \), then \fullref{thm:ordered_ring_power} implies that \( a^2 \) is positive and \( b^2 \) is nonnegative, while \fullref{thm:def:ordered_semiring/strict_sum} implies that their sum is positive. Similarly, if \( b \neq 0 \), the sum \( a^2 + b^2 \) is strictly positive.

  \SubProofOf[def:absolute_value/multiplicative]{multiplicativity} We have
  \begin{balign*}
    \abs{(a + bi) \cdot (c + di)}
    &=
    \abs{(ac - bd) + (ad + bc)i}
    = \\ &=
    \sqrt{(ac - bd)^2 + (ad + bc)^2}
    = \\ &=
    \sqrt{a^2 c^2 - \cancel{2abcd} + b^2 d^2 + a^2 d^2 + \cancel{2abcd} + b^2 c^2}
    = \\ &=
    \sqrt{a^2 (c^2 + d^2) + b^2 (d^2 + c^2)}
    = \\ &=
    \sqrt{(a^2 + b^2) \cdot (c^2 + d^2)}
    \reloset {\eqref{eq:thm:def:principal_nonnegative_nth_root/multiplicative}} = \\ &=
    \sqrt{a^2 + b^2} \cdot \sqrt{c^2 + d^2}
    = \\ &=
    \abs{a + bi} \cdot \abs{c + di}.
  \end{balign*}

  \SubProofOf[def:additive_function/sub]{subadditivity} We have
  \begin{balign*}
    \abs{(a + bi) + (c + di)}
    &=
    (a + c)^2 + (b + d)^2
    = \\ &=
    (a^2 + b^2) + (c^2 + d^2) + 2(ac + bd)
    = \\ &=
    \abs{a + bi} + \abs{c + di} + 2(ac + bd).
  \end{balign*}
\end{defproof}

\begin{proposition}\label{thm:complex_multiplicative_inverse}
  The \hyperref[def:semiring]{multiplicative inverse} of the nonzero \hyperref[def:complex_numbers]{complex number} \( z = a + bi \) is
  \begin{equation}\label{eq:thm:complex_multiplicative_inverse}
    \frac {\oline z} {{\abs z}^2} = \frac {a - bi} {\sqrt{a^2 + b^2}}
  \end{equation}
\end{proposition}
\begin{proof}
  Trivial considering how we defined the complex absolute value.
\end{proof}

\begin{proposition}\label{thm:def:complex_absolute_value}
  \hyperref[def:complex_absolute_value]{Complex absolute values} have the following basic properties:
  \begin{thmenum}
    \thmitem{thm:def:complex_absolute_value/idempotent} It is \hyperref[def:idempotent_function]{idempotent}: \( \abs{\abs{z}} = \abs{z} \).
    \thmitem{thm:def:complex_absolute_value/nonnegative} \( z = \abs z \) if and only if \( z \) is a nonnegative real number.
    \thmitem{thm:def:complex_absolute_value/negative} \( z = -\abs z \) if and only if \( z \) is a negative real number.
  \end{thmenum}
\end{proposition}
\begin{proof}
  \SubProofOf{thm:def:complex_absolute_value/idempotent} By positive definiteness, we have \( \abs z \geq 0 \). \Fullref{thm:def:complex_conjugation/fixed} then implies that \( \abs z \) is a fixed point of conjugation, hence \( \abs z = \oline{\abs z} \) and
  \begin{equation*}
    \abs{\abs z}
    =
    \sqrt{ \abs z \cdot \oline{\abs z} }
    =
    \sqrt { \abs{z}^2 }
    \reloset {\eqref{eq:thm:def:principal_nonnegative_nth_root/root_of_power}} =
    \abs z.
  \end{equation*}

  \SubProofOf{thm:def:complex_absolute_value/nonnegative} Trivial considering the positive definiteness of the absolute value.

  \SubProofOf{thm:def:complex_absolute_value/negative} Follows from \fullref{thm:def:complex_absolute_value/nonnegative} by noting that, by \fullref{thm:def:ordered_semiring_positivity/additive_inverse}, a real number is positive if and only if its additive inverse is negative.
\end{proof}

\paragraph{Fundamental theorem of algebra}

\begin{theorem}[Fundamental theorem of algebra]\label{thm:fundamental_theorem_of_algebra}
  The field \( \BbbC \) of \hyperref[def:complex_numbers]{complex numbers} is \hyperref[def:algebraically_closed_field]{algebraically closed}.
\end{theorem}
\begin{proof}
  \todo{Prove}
\end{proof}
