\subsection{Complex numbers}\label{subsec:complex_numbers}

\begin{definition}\label{def:complex_numbers}
  We give a few equivalent definition of the \hyperref[def:field]{field} \( \BbbC \) \term{complex numbers}. Informally, there are numbers of the form \( a + bi \), where \( a, b \in \BbbR \) and \( i = \sqrt{-1} \). In order to find the multiplicative inverse of the nonzero polynomial \( a + bi \), we assume that division is well-defined and proceed as follows:
  \begin{equation}\label{def:complex_numbers/inverse}
    \frac 1 {a + bi} = \frac {a - bi} {(a + bi)(a - bi)} = \frac{a - bi}{a^2 + b^2}.
  \end{equation}

  The closest to this informal definition is \fullref{def:complex_numbers/polynomials}.

  \begin{thmenum}
    \thmitem{def:complex_numbers/polynomials} The most \enquote{algebraic} way to define complex numbers is as the \hyperref[def:polynomial_algebra]{polynomial} quotient ring
    \begin{equation*}
      \BbbC \coloneqq \BbbR[X] / \braket{X^2 + 1}.
    \end{equation*}

    Elements of \( \BbbC \) can be identified with real polynomials of the form \( bX + a \). See \fullref{ex:gaussian_integers/quotient} for a broader discussion.
    Define \( i \coloneqq X \). We have
    \begin{equation*}
      i \cdot i = X^2 = -1 \pmod {X^2 + 1}.
    \end{equation*}

    Thus, \( i \) is indeed the square root of \( -1 \). We will write
    \begin{equation*}
      a + bi = bX + a.
    \end{equation*}

    It is shown in \fullref{ex:gaussian_integers/quotient} that multiplication modulo \( X^2 + 1 \) gives
    \begin{equation}\label{def:complex_numbers/polynomials/multiplication}
      (bX + a) (dX + c) = (ad + bc)X + (ac - bd) \pmod {X^2 + 1}.
    \end{equation}

    The multiplicative inverse of \( a + bi \) is then indeed \fullref{def:complex_numbers/inverse}.

    The canonical embedding \( \iota: \BbbR \to \BbbC \) is then the standard polynomial embedding.

    \thmitem{def:complex_numbers/matrices} The complex numbers can also be defined as the matrix \hyperref[thm:matrix_algebra]{ring}
    \begin{equation*}
      \BbbC \coloneqq \left\{
      \begin{pmatrix}
        a  & b \\
        -b & a
      \end{pmatrix}
      \colon a, b \in \BbbR \right\}
    \end{equation*}
    with the usual matrix multiplication. The canonical embedding \( \iota: \BbbR \to \BbbC \) is then
    \begin{equation*}
      \iota(a) \coloneqq \begin{pmatrix}
        a & 0 \\
        0 & a
      \end{pmatrix}
    \end{equation*}

    \thmitem{def:complex_numbers/tuples} Finally, we can define \( \BbbC \) is the \hyperref[def:algebra_over_semiring]{algebra} obtained from the vector space \( \BbbR^2 \) with the multiplication operation emulating \fullref{def:complex_numbers/polynomials/multiplication} as
    \begin{balign*}
       & \cdot: \BbbC \times \BbbC \to \BbbC                     \\
       & (a, b) \cdot (c, d) \coloneqq (ac - bd, ad + bc).
    \end{balign*}

    The canonical embedding \( \iota: \BbbR \to \BbbC \) is then
    \begin{equation*}
      \iota(a) \coloneqq (a, b).
    \end{equation*}
  \end{thmenum}

  We define the unary \term{complex conjugation} operation as \( \overline{a + bi} \coloneqq a - bi \) and the \term{\hyperref[def:absolute_value]{absolute value}} as
  \begin{equation*}
    \abs{a + bi} \coloneqq \sqrt{a^2 + b^2}.
  \end{equation*}

  For a complex number \( z = a + bi \) we denote
  \begin{balign*}
    \real z = a &  & \imag z = b
  \end{balign*}
  and call them the \term{real} and \term{imaginary} parts of \( z \).
\end{definition}

\begin{theorem}[Fundamental theorem of algebra]\label{thm:fundamental_theorem_of_algebra}
  The field \( \BbbC \) of complex numbers is algebraically \hyperref[def:algebraically_closed_field]{closed}.
\end{theorem}

\begin{theorem}\label{thm:linear_functionals_over_c}
  Let \( X \) be a \hyperref[def:vector_space]{vector space} over \( \BbbC \). There is a bijection between the real-valued and the complex-valued linear functionals on \( X \).
\end{theorem}
\begin{proof}
  Let \( c: X \to \BbbC \) be a complex-valued linear functional. Denote \( a(x) \coloneqq \real c(x) \) and \( b(x) \coloneqq \imag c(x) \). Then \( a: X \to \BbbR \) and \( b: X \to \BbbR \) are linear functionals. We will show that \( a(x) \) uniquely determines \( b(x) \) and hence \( c(x) \).

  Note that \( c(ix) = a(ix) + i b(ix) = i a(x) - b(x) \). Therefore, \( b(x) = a(ix) - c(ix) \) and
  \begin{equation*}
    c(x) = a(x) + i (a(ix) - c(ix)) = a(x) - a(x) + c(x) = c(x).
  \end{equation*}
\end{proof}

\begin{remark}\label{rem:linear_functionals_over_c}
  \Fullref{thm:linear_functionals_over_c} allows us to identify the dual space \( X* \) of a complex vector space \( X \) with \( \hom(X, \BbbR) \) in the case of an algebraic \hyperref[def:dual_vector_space]{dual} or with the corresponding subspace in the case of a \hyperref[def:continuous_dual_space]{continuous dual space}.

  This allows us to reuse some of the theory for real vector spaces, for example hyperplane \hyperref[def:hyperplane_separation]{separation}.
\end{remark}

\begin{definition}\label{def:circle_group}\mimprovised
  The \term{circle group} \( \BbbT \) is any of the following isomorphic groups:
  \begin{thmenum}
    \thmitem{def:circle_group/complex} The set of all complex numbers with unit norm under multiplication.
    \thmitem{def:circle_group/real} The \hyperref[def:group/quotient]{quotient} \( \BbbR / \BbbZ \) under addition.
    \thmitem{def:circle_group/rotations} The set of all \hyperref[def:rigid_motion/rotation]{rotation} in the \hyperref[def:euclidean_plane]{Euclidean plane} under composition.
    \thmitem{def:circle_group/so2} The \hyperref[def:unitary_groups]{special orthogonal group} \( \grp{SO}(2) \).
    \thmitem{def:circle_group/su1} The \hyperref[def:unitary_groups]{special unitary group} \( \grp{SU}(1) \).
  \end{thmenum}
\end{definition}
\begin{defproof}
  \EquivalenceSubProof{def:circle_group/complex}{def:circle_group/real} Define the map
  \begin{equation*}
    \begin{aligned}
      &\varphi: \BbbR / \BbbZ \to \BbbC \\
      &\varphi(t) \coloneqq \frac {e^{it}} {2\pi}.
    \end{aligned}
  \end{equation*}

  It follows from \fullref{thm:def:exponential_function/homomorphism} that it is a homomorphism and from \fullref{thm:def:exponential_function/unit_circle} that it is injective and the values of \( \varphi \) are precisely the complex numbers with norm \( 1 \). Therefore, it is an isomorphism.

  \EquivalenceSubProof{def:circle_group/real}{def:circle_group/so2} Follows from \fullref{thm:plane_rotation_matrix}.

  \EquivalenceSubProof{def:circle_group/rotations}{def:circle_group/so2} This is the definition of rotation in \fullref{def:rigid_motion/rotation}.

  \EquivalenceSubProof{def:circle_group/so2}{def:circle_group/su1} Follows from \fullref{def:complex_numbers/matrices} by noting that the norm of a complex number in matrix form is the determinant of the corresponding matrix.
\end{defproof}
