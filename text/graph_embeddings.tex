\subsection{Graph embeddings}\label{subsec:graph_embeddings}

\paragraph{Geometric realizations of graphs}

\begin{definition}\label{def:graph_geometric_realization}\mimprovised
  Let \( G = (V, A, h, t) \) be a \hyperref[def:directed_multigraph]{directed multigraph}. Our goal is to construct a \hyperref[def:topological_space]{topological space} that translates the connectivity properties of \( G \) into their topological equivalents.

  Consider the \hyperref[def:topological_sum]{topological sum}
  \begin{equation*}
    S \coloneqq \parens[\Bigg]{ \coprod_{e \in A} [0, 1] } \amalg \parens[\Bigg]{ \coprod_{\deg(v) = 0} \set{ v } }.
  \end{equation*}

  The space \( S \) consists of disjoint unit intervals, one for each arc, and of disjoint points, one for each \hyperref[def:graph_cardinality/directed_degree]{isolated vertex}.

  We now want to glue common endpoints of arcs in \( S \). We define the functions
  \begin{equation}\label{eq:def:graph_geometric_realization/rv}
    \begin{aligned}
      &R_V: V \to \pow(S), \\
      &R_V(v) \coloneqq \begin{cases}
        \set[\Big]{ \iota_v(v) },                                                               &\deg(v) = 0 \\
        \set[\Big]{ \iota_e(0) \given h(e) = v } \cup \set[\Big]{ \iota_e(1) \given t(e) = v }, &\deg(v) > 0
      \end{cases}
    \end{aligned}
  \end{equation}
  and
  \begin{equation}\label{eq:def:graph_geometric_realization/ra}
    \begin{aligned}
      &R_A: A \to \pow(\pow(S)), \\
      &R_A(e) \coloneqq \set[\Big]{ \set{ \iota_e(p) } \given 0 < p < 1 }.
    \end{aligned}
  \end{equation}

  Then \( R_V(v) \) is a set in \( S \) whose role is to be a vertex, and \( R_A(e) \) is the interior of an arc, i.e. a set of points between vertices in the topological sense.

  The family
  \begin{equation*}
    X \coloneqq \bigcup \set[\Big]{ R_A(e) \given* e \in A } \cup \set[\Big]{ R_V(v) \given* v \in V }.
  \end{equation*}
  is then a \hyperref[def:set_partition]{partition} of \( S \).

  The partition \( X \) can be endowed with a \hyperref[def:topological_quotient]{quotient topology} \( \mscrT \). We call the topological space \( (X, \mscrT, R_V, R_A) \), endowed with the functions \( R_V \) and \( R_A \), the \term{geometric realization} of \( G \).

  \begin{thmenum}
    \thmitem{def:graph_geometric_realization/undirected} For an \hyperref[def:hypergraph/multigraph]{undirected multigraph} \( G = (V, E, \mscrE) \), the geometric realization is any of the geometric realizations of its \hyperref[def:multigraph_orientation]{orientations}. This construction is dependent on a choice function, but fortunately all the geometric realizations are homeomorphic as shown in \fullref{thm:undirected_multigraph_geometric_realizations_homeomorphic}.

    \thmitem{def:graph_geometric_realization/embedding} We will call any \hyperref[def:homeomorphism]{homeomorphic embedding} with domain \( (X, \mscrT) \) a \term{graph embedding}.
  \end{thmenum}
\end{definition}

\begin{example}\label{ex:def:graph_geometric_realization}
  We will give a few examples of \hyperref[def:graph_geometric_realization]{graph geometric realizations}.

  \begin{thmenum}
    \thmitem{ex:def:graph_geometric_realization/order_zero} The geometric realization of an \hyperref[rem:trivial_graph]{arcless} directed multigraph is the \hyperref[def:discrete_topology]{discrete topological space} on its vertices.

    In particular, the geometric realization of the \hyperref[rem:trivial_graph]{order-zero} directed multigraph (without any arcs and edges) is the empty topological space.

    \thmitem{ex:def:graph_geometric_realization/positive_integers} Consider the reduced positive integer graph \eqref{eq:ex:infinite_integer_graphs/positive}. We start with \( \aleph_0 \) copies of \( [0, 1] \) and glue both ends of each of them except for the first. Thus, we obtain (a space homeomorphic to)
    \begin{equation*}
      \bigcup_{k \geq 0} [k, k + 1] = [0, \infty).
    \end{equation*}

    Therefore, \eqref{eq:ex:infinite_integer_graphs/positive} is a \hyperref[def:linear_graph]{linear graph}.

    \thmitem{ex:def:graph_geometric_realization/k3} The graph with vertices \( V = \set{ a, b, c } \) and arcs \( \set{ \overbrace{a \to b}^{e_1}, \overbrace{b \to c}^{e_2}, \overbrace{c \to a}^{e_3} } \) is more subtle.

    We start with three copies of the interval \( [0, 1] \), depicted in \eqref{eq:ex:def:graph_geometric_realization/k3/relization} as upward-pointing arrows, and use dashed lines to connect the endpoints that we want to glue together.
    \begin{equation}\label{eq:ex:def:graph_geometric_realization/k3/relization}
      \begin{aligned}
        \includegraphics[page=1]{output/ex__def__graph_geometric_realization}
      \end{aligned}
    \end{equation}

    After contracting the dashed lines, we obtain a topological space that can easily be \hyperref[def:graph_geometric_realization/embedding]{embedded} into \( \BbbR^2 \). An obvious embedding corresponds to \enquote{pulling up} \( e_2 \) and \( e_3 \):
    \begin{equation}\label{eq:ex:def:graph_geometric_realization/k3/embedding}
      \begin{aligned}
        \includegraphics[page=2]{output/ex__def__graph_geometric_realization}
      \end{aligned}
    \end{equation}

    It can be expressed in terms of \( R_V \) and \( R_A \):
    \begin{equation}\label{eq:ex:def:graph_geometric_realization/k3/unlabeled}
      \begin{aligned}
        \includegraphics[page=3]{output/ex__def__graph_geometric_realization}
      \end{aligned}
    \end{equation}

    This is only one possible embedding of the geometric realization. It is sufficient, however, for proving that the graph is \hyperref[def:planar_graph]{planar}.

    The corresponding undirected graph is the \hyperref[def:complete_graph]{complete graph} \( K_3 \). We may have different orientations of \( K_3 \). One is \eqref{eq:ex:def:graph_geometric_realization/k3/unlabeled}. Consider another orientation where \( e_1 \) is replaced by an arc from \( b \) to \( a \) denoted by \( e_1' \). This corresponds to the realization
    \begin{equation}\label{eq:ex:def:graph_geometric_realization/k3/undirected_2}
      \begin{aligned}
        \includegraphics[page=4]{output/ex__def__graph_geometric_realization}
      \end{aligned}
    \end{equation}

    The two are homeomorphic as shown in \fullref{thm:undirected_multigraph_geometric_realizations_homeomorphic}.

    \thmitem{ex:def:graph_geometric_realization/k4} \Cref{fig:def:complete_graph/k4} shows that the complete graph \( K_4 \) is planar.

    This is, however, not-at-all obvious from the geometric realization of a particular orientation.
    \begin{equation}\label{eq:ex:def:graph_geometric_realization/k4/realization}
      \begin{aligned}
        \includegraphics[page=5]{output/ex__def__graph_geometric_realization}
      \end{aligned}
    \end{equation}

    This example shows that constructing embeddings can be a tedious task.
  \end{thmenum}
\end{example}

\begin{proposition}\label{thm:undirected_multigraph_geometric_realizations_homeomorphic}
  Let \( G = (V, E, \mscrE) \) be an \hyperref[def:hypergraph/multigraph]{undirected multigraph}.

  Let \( (X_c, \mscrT_c, R_{V_c}, R_{A_c}) \) and \( (X_d, \mscrT_d, R_{V_d}, R_{A_d}) \) be \hyperref[def:graph_geometric_realization/undirected]{geometric realizations} corresponding to the \hyperref[def:multigraph_orientation]{orientations} \( O_c(G) \) and \( O_d(G) \) of \( G \).

  Then the \hyperref[def:topological_space]{topological spaces} \( (X_c, \mscrT_c) \) and \( (X_c, \mscrT_d) \) are \hyperref[def:homeomorphism]{homeomorphic}.
\end{proposition}
\begin{proof}
  Let \( h_c \) and \( h_d \) be the head functions from the \hyperref[def:directed_multigraph]{directed multigraphs} \( O_c(G) \) and \( O_d(G) \).

  Define the function
  \begin{equation*}
    \begin{aligned}
      &f: X_c \to X_d \\
      &f(x) \coloneqq \begin{cases}
        R_{V_d}(v),             &x = R_{V_c}(v), \\
        \set{ \iota_e(p) },     &x = \set{ \iota_e(p) } \T{and} h_c(e) = h_d(e), \\
        \set{ \iota_e(1 - p) }, &x = \set{ \iota_e(p) } \T{and} h_c(e) \neq h_d(e).
      \end{cases}
    \end{aligned}
  \end{equation*}

  This function \enquote{reverses} the direction of some intervals in the construction of the realizations and fixes everything else in place. It is clearly bijective. It is continuous as a composition of \( t \mapsto 1 - t \) with continuous injections and projections. Finally, it is a homeomorphism because the inverse function satisfies the same definition with \( c \) and \( d \) interchanged.
\end{proof}

\begin{proposition}\label{thm:def:graph_geometric_realization}
  Let \( G = (V, A, h, t) \) be a \hyperref[def:directed_multigraph]{directed multigraph} and let \( (X, \mscrT, R_V, R_A) \) be its \hyperref[def:graph_geometric_realization]{geometric realization}. We have the following basic properties:

  \begin{thmenum}
    \thmitem{thm:def:graph_geometric_realization/isolated} A vertex \( v \in V \) is \hyperref[def:isolated_vertex]{isolated} in \( G \), i.e. has degree zero, if and only if \( R_V(v) \) is an \hyperref[def:set_cluster_point]{isolated point} of \( X \).

    \thmitem{thm:def:graph_geometric_realization/t1} If \( G \) is \hyperref[def:graph_cardinality/local]{locally finite}, then the space \( (X, \mscrT) \) satisfies the \ref{def:separation_axioms/T1} separation axiom.
  \end{thmenum}
\end{proposition}
\begin{proof}
  \SubProofOf{thm:def:graph_geometric_realization/isolated}

  \SufficiencySubProof* For every isolated vertex \( v \), the point \( R_V(v) = \set{ \iota(v) } \) is isolated by definition.

  \NecessitySubProof* Now suppose that \( R_V(v) \) is an isolated point but \( v \) is not an isolated vertex.

  Then there exists an arc \( e \) such that \( v = h(e) \) or \( v = t(e) \). In the first case, \( \iota_e(0) \in R_V(v) \), and hence there exists no neighborhood of \( R_V(v) \) disjoint from \( R_A(e) \). In the second case, \( \iota_e(1) \in R_V(v) \), and the same conclusion holds.

  Hence, \( R_V(v) \) cannot be an isolated point of \( X \). The obtained contradiction shows that \( v \) must be an isolated vertex.

  \SubProofOf{thm:def:graph_geometric_realization/t1} Let \( x \in X \).

  \begin{itemize}
    \item If \( x = \set{ \iota_e(p) } \) for some arc \( e \) and \( 0 < p < 1 \), then \( \set{ x } \) is closed because \( [0, 1] \) satisfies \hyperref[def:separation_axioms/T1]{T1} and hence \( \set{ p } \) is closed in \( [0, 1] \).

    \item If \( x = R_V(v) = \set{ \iota_v(v) } \) for some isolated vertex \( v \), then \( R_V(v) \) is clopen by definition.

    \item If \( x = R_V(v) \) for some vertex \( v \) of positive degree, then
    \begin{equation*}
      R_V(v) = \set[\Big]{ \iota_e(0) \given* h(e) = v } \cup \set[\Big]{ \iota_e(1) \given* t(e) = v }.
    \end{equation*}

    Both \( \set{ 0 } \) and \( \set{ 1 } \) are closed in \( [0, 1] \), hence \( \set{ 0, 1 } \) is also closed in \( [0, 1] \). Since \( G \) is locally finite, \( R_V(v) \) is the union of finitely many closed sets and is thus itself closed.
  \end{itemize}
\end{proof}

\paragraph{Connectedness of geometric realizations}

\begin{proposition}\label{thm:directed_multigraph_geometric_realization_paths}
  Let \( G = (V, A, h, t) \) be a \hyperref[def:directed_multigraph]{directed multigraph} and let \( (X, \mscrT, R_V, R_A) \) be its \hyperref[def:graph_geometric_realization]{geometric realization}.

  \begin{thmenum}
    \thmitem{thm:directed_multigraph_geometric_realization_paths/graph_to_realization} If there exists a \hyperref[def:graph_walk/generalized]{generalized walk}
    \begin{equation}\label{eq:thm:directed_multigraph_geometric_realization_paths/graph_to_realization/graph}
      v_0, e_1, v_1, e_2, v_2, \cdots, v_{n-1}, e_n, v_n
    \end{equation}
    then there exists a \hyperref[def:parametric_curve]{continuous curve} \( \gamma: [0, 1] \to X \) from \( R_V(v_0) \) to \( R_V(v_n) \).

    \thmitem{thm:directed_multigraph_geometric_realization_paths/realization_to_graph} If \( G \) is finite and if there exists a simple continuous curve \( \gamma: [0, 1] \to X \) from \( R_V(s) \) to \( R_V(f) \), then there exists a generalized walk from \( s \) to \( f \).
  \end{thmenum}
\end{proposition}
\begin{proof}
  \SubProofOf{thm:directed_multigraph_geometric_realization_paths/graph_to_realization} Suppose that we are given the generalized walk \eqref{eq:thm:directed_multigraph_geometric_realization_paths/graph_to_realization/graph}.

  We will use induction on \( n \) to show that there exists a continuous path between the points \( R_V(s) \) and \( R_V(f)  \) of \( X \).

  The base case \( n = 0 \) is vacuous. Suppose that \( n > 0 \) and that the statement holds for walks of length strictly smaller than \( n \). Then the inductive hypothesis holds for the initial segment
  \begin{equation*}
    v_0, e_1, v_1, e_2, v_2, \cdots, e_{n-1}, v_{n-1},
  \end{equation*}
  hence there exists a continuous curve \( \delta: [0, 1] \to X \) from \( R_V(v_0) \) to \( R_V(v_{n-1}) \).

  \begin{itemize}
    \item If \( e_n \) is positively oriented, then \eqref{eq:thm:directed_multigraph_geometric_realization_paths/graph_to_realization/graph} ends with
    \begin{equation*}
      \cdots v_{n-1} \reloset{e_n} \to v_n.
    \end{equation*}

    Thus, \( \delta \) is a continuous path from \( R_V(v_0) \) to \( R_V(v_{k-1}) \), and \( R_A(e_n) \) is a continuous path from \( R_V(v_{k-1}) \) to \( R_V(v_k) \). We can concatenate the two to obtain a path from \( R_V(s)  \) to \( R_V(f) \).

    \item If \( e_n \) is negatively oriented, then \eqref{eq:thm:directed_multigraph_geometric_realization_paths/graph_to_realization/graph} ends with
    \begin{equation*}
      \cdots v_{n-1} \reloset{e_n} \leftarrow f.
    \end{equation*}

    Thus, \( \delta \) is again a continuous path from \( R_V(s) \) to \( R_V(v_{k-1}) \), but \( R_A(e_n) \) is a continuous path from \( R_V(f) \) to \( R_V(v_{k-1}) \). We can thus concatenate \( \delta \) with the inversely parameterized path of \( R_A(e_n) \) to obtain a path from \( R_V(s) \) to \( R_V(f) \).
  \end{itemize}

  In both cases, we have built a continuous path from \( R_V(s) \) to \( R_V(f)  \).

  \SubProofOf{thm:directed_multigraph_geometric_realization_paths/realization_to_graph} Suppose that \( G \) is a finite directed multigraph and let \( \gamma: [0, 1] \to X \) be a simple continuous curve from \( R_V(s) \) to \( R_V(f) \).

  For every number \( x \in [0, 1] \), define the set
  \begin{equation*}
    A_x \coloneqq \set{ y > x \given \qexists* {v \in V} R_V(v) = \gamma(y) }.
  \end{equation*}

  Let \( n \coloneqq \card(A_0) + 1 \), i.e. the number of vertices lying on the path \( \gamma \). Since \( G \) is finite, \( n \) is also finite. Then we can recursively define the sequence
  \begin{equation*}
    0 = x_0 < x_1 < \cdots < x_{n-1} < x_n = 1
  \end{equation*}
  as follows:
  \begin{equation*}
    x_k \coloneqq \begin{cases}
      0,                &k = 0, \\
      \min A_{x_{k-1}}, &0 < k \leq n. \\
    \end{cases}
  \end{equation*}

  Since the path \( \gamma \) is simple, no vertex can be traversed twice. Thus, we have located the unique vertices lying on \( \gamma \). Denote them via \( v_0, \ldots, v_n \).

  We must show that, for every \( k = 1, \ldots, n \), there exists a unique arc \( e_k \) between \( v_{k-1} \) and \( v_k \) (in any direction) such that every point of \( \gamma \) between \( x_{k-1} \) and \( x_k \) lies on \( R_A(e_k) \). At least one such arc must exist because otherwise we could not construct a continuous curve.

  Aiming at a contradiction, suppose that for some numbers \( y < z \) strictly between \( x_{k-1} \) and \( x_k \), the points \( \gamma(y) \) and \( \gamma(z) \) lie on different arcs, \( R_A(e_y) \) and \( R_A(e_z) \).

  The segment of \( \gamma \) between \( x_{k-1} \) and \( x_k \) is path-connected as a continuous image of the open interval \( (x_{k-1}, x_k) \). Since we have constructed the space \( X \) so that the arcs \( R_A(e_y) \) and \( R_A(e_z) \) are disjoint whenever \( e_y \neq e_z \), a continuous path passing through both \( R_A(e_y) \) and \( R_A(e_z) \) must contain a vertex common to both arcs. But, by construction, there cannot be a vertex between \( x_{k-1} \) and \( x_k \).

  The obtained contradiction shows the uniqueness of \( e_k \). It now remains to define our desired walk in \( G \) as
  \begin{equation*}
    s = v_0, e_1, v_1, e_2, v_2, \cdots, v_{n-1}, e_n, v_n = f.
  \end{equation*}
\end{proof}

\begin{corollary}\label{thm:directed_multigraph_geometric_realization_connectedness}
  A finite \hyperref[def:directed_multigraph]{directed multigraph} is \hyperref[def:graph_connectedness/weak]{weakly connected} if and only if its \hyperref[def:graph_geometric_realization/undirected]{geometric realization} is \hyperref[def:path_connected_space]{path connected}.
\end{corollary}
\begin{proof}
  Follows from \fullref{thm:directed_multigraph_geometric_realization_paths}.
\end{proof}

\begin{corollary}\label{thm:undirected_multigraph_geometric_realization_connectedness}
  A finite \hyperref[def:hypergraph/multigraph]{undirected multigraph} is \hyperref[def:graph_connectedness/undirected]{connected} if and only if its \hyperref[def:graph_geometric_realization/undirected]{geometric realization} is a \hyperref[def:path_connected_space]{path connected}.
\end{corollary}
\begin{proof}
  Follows from \fullref{thm:directed_multigraph_geometric_realization_connectedness}.
\end{proof}

\paragraph{Graph embeddings}

\begin{proposition}\label{thm:linear_directed multigraph_equivalence}
  If a finite simple directed graph \hyperref[def:graph_geometric_realization/embedding]{embeds} in \( \BbbR \), every vertex has \hyperref[def:graph_cardinality/directed_degree]{degree} at most \( 2 \).
\end{proposition}
\begin{proof}
  Let \( G = (V, A) \) be a finite simple directed graph and let \( (X, \mscrT, R_V, R_A) \) be its geometric realization. Let \( f: X \to \BbbR \) be a homeomorphic embedding.

  Suppose that the vertex \( v \) has degree larger than \( 2 \). It is sufficient to consider the case where \( e_1 \), \( e_2 \) and \( e_3 \) are distinct arcs incident to \( v \). Then \( f(R_V(v)) \) must be the point that connects \( f(R_A(e_1)) \), \( f(R_A(e_2)) \) and \( f(R_A(e_3)) \). If \( f(R_A(e_1)) \) is placed to the left of \( f(R_V(v)) \) and \( f(R_A(e_2)) \) --- to the right, then there is nowhere to place \( f(R_A(e_3)) \) without it intersecting the other two curves.

  The obtained contradiction shows that \( \deg(v) \leq 2 \).
\end{proof}

\begin{proposition}\label{thm:moment_curve}\mcite[833]{Rosen1999DiscreteHandbook}
  Consider the \hyperref[def:parametric_curve]{parametric curve}
  \begin{equation*}
    \begin{aligned}
      &\gamma: \BbbR \to \BbbR^n \\
      &\gamma(t) \coloneqq (t, t^2, \ldots, t^n).
    \end{aligned}
  \end{equation*}

  This curve is called the \term{moment curve} of dimension \( n \).

  For any \( t_1 < \ldots < t_n \), the points \( \gamma(t_1), \ldots, \gamma(t_n) \) are \hyperref[def:linear_dependence]{linearly independent}.
\end{proposition}
\begin{proof}
  Follows from \fullref{ex:vandermonde_matrix}.
\end{proof}

\begin{proposition}\label{thm:directed_multigraph_can_be_embedded_into_r3}
  Every finite \hyperref[def:directed_graph]{simple directed graph} can be embedded into \( \BbbR^3 \).
\end{proposition}
\begin{proof}
  Let \( G = (V, A) \) be a finite simple directed graph of order \( n \). By definition of cardinality, there exists a bijection from \( n \) to \( V \).

  Place the vertices of \( G \) along the \hyperref[thm:moment_curve]{moment curve} by using \( \gamma(k) \) as the position for the \( k \)-th vertex of \( V \). Then, by \fullref{thm:moment_curve}, no three of these points are \hyperref[def:collinear_points]{collinear}. Hence, if we connect their vertices using a straight line where there is an arc, the interiors of no two lines would intersect.

  Therefore, this is an embedding.
\end{proof}

\paragraph{Planar graphs}

\begin{definition}\label{def:planar_graph}\mcite[102]{Harary1969Graphs}
  If an \hyperref[rem:arbitrary_kind_graph]{arbitrary-kind graph} can be embedded into \( \BbbR^2 \), we say that it is \term[bg=планарен (граф) (\cite[129]{Мирчев2001Графи}), ru=планарный (граф) (\cite[150]{Емеличев1990Графы}), en=planar (graph) (\cite[93]{Diestel2005Graphs})]{planar}.
\end{definition}

\begin{theorem}[Four color theorem]\label{thm:four_color_theorem}
  Every finite \hyperref[def:planar_graph]{planar} \hyperref[def:undirected_graph]{simple undirected graph} is \( 4 \)-\hyperref[def:graph_coloring/colorable]{colorable}.
\end{theorem}
\begin{comments}
  \item The first proof of this theorem has been published in \cite{AppelHaken1977PlanarPart1} and \cite{AppelHaken1977PlanarPart2}. It relies on computer verification of hundreds on cases. \cite{AppelHaken1977PlanarPart2} lists 63 figures, each containing 35 graphs with some blank spaces left.
\end{comments}

\begin{theorem}[Wagner's theorem]\label{thm:wagners_theorem}\mcite[thm. 11.14]{Harary1969Graphs}
  A finite \hyperref[def:undirected_graph]{simple undirected graph} is \hyperref[def:planar_graph]{planar} if and only if it contains neither the \hyperref[def:complete_graph]{complete graph} \( K_5 \) nor the \hyperref[def:complete_graph]{complete bipartite graph} \( K_{3,3} \) as a \hyperref[def:graph_minor]{minor}.
\end{theorem}
