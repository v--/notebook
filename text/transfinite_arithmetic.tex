\subsection{Transfinite arithmetic}\label{subsec:transfinite_arithmetic}

Our purpose is to extend natural number arithmetic to ordinals and cardinals. It turns out that the two are rather different. We will first introduce some additional concepts, however.

\begin{definition}\label{def:ordinal_arithmetic}
  We recursively define arithmetic operations for arbitrary \hyperref[def:ordinal]{ordinals} as extensions of the corresponding operations of \hyperref[def:peano_arithmetic]{Peano arithmetic}.

  \begin{thmenum}
    \thmitem{def:ordinal_arithmetic/addition}\mcite[def. 6.3.16]{Hinman2005} The \term{sum} of \( \alpha \) and \( \beta \) extends \eqref{eq:def:peano_arithmetic/PA4} and \eqref{eq:def:peano_arithmetic/PA5} with a case for limit ordinals:
    \begin{equation}\label{eq:def:ordinal_arithmetic/addition}
      \alpha + \beta \coloneqq \begin{cases}
        \alpha,                                            &\beta = 0 \\
        \op{succ}(\alpha + \gamma),                        &\beta = \op{succ}(\gamma) \\
        \sup\set{ \alpha + \gamma \given \gamma < \beta }, &\beta \T{is a limit ordinal} \\
      \end{cases}
    \end{equation}

    From \fullref{thm:union_of_set_of_ordinals} it follows that the in limit case \( \alpha + \beta \) is the smallest ordinal strictly larger than \( \alpha + \gamma \) for any \( \gamma < \beta \).

    \thmitem{def:ordinal_arithmetic/multiplication}\mcite[def. 6.3.16]{Hinman2005} Analogously, the \term{product} of \( \alpha \) and \( \beta \) extends \eqref{eq:def:peano_arithmetic/PA6} and \eqref{eq:def:peano_arithmetic/PA7}:
    \begin{equation}\label{eq:def:ordinal_arithmetic/multiplication}
      \alpha \cdot \beta \coloneqq \begin{cases}
        0,                                                     &\beta = 0 \\
        \alpha \cdot \gamma + \alpha,                          &\beta = \op{succ}(\gamma) \\
        \sup\set{ \alpha \cdot \gamma \given \gamma < \beta }, &\beta \T{is a limit ordinal} \\
      \end{cases}
    \end{equation}

    \thmitem{def:ordinal_arithmetic/exponentiation}\mcite[exer. 6.3.34]{Hinman2005} Exponentiation extends \fullref{def:monoid/exponentiation}:
    \begin{equation}\label{eq:def:ordinal_arithmetic/exponentiation}
      \alpha^\beta \coloneqq \begin{cases}
        1,                                               &\beta = 0 \\
        \alpha^\gamma \cdot \alpha,                      &\beta = \op{succ}(\gamma) \\
        \sup\set{ \alpha^\gamma \given \gamma < \beta }, &\beta \T{is a limit ordinal} \\
      \end{cases}
    \end{equation}
  \end{thmenum}
\end{definition}

\begin{remark}\label{rem:ordinal_successor_via_addition}
  For any ordinal \( \alpha \) we have
  \begin{equation*}
    \op{succ}(\alpha)
    \reloset {\ref{eq:def:ordinal_arithmetic/addition}} =
    \op{succ}(\alpha + 0)
    \reloset {\ref{eq:def:ordinal_arithmetic/addition}} =
    \alpha + \op{succ}(0)
    =
    \alpha + 1.
  \end{equation*}

  We will occasionally use the later notation.

  Note that for infinite ordinals \( \op{succ}(\alpha) = 1 + \alpha \) as discussed in \fullref{ex:ordinal_addition}.

  This is an extension of \fullref{rem:natural_number_successor_via_addition}.
\end{remark}

\begin{proposition}\label{thm:ordinal_addition_is_monotone}
  \hyperref[def:ordinal_arithmetic/addition]{Ordinal addition} has the following monotonicity properties:
  \begin{thmenum}
    \thmitem{thm:ordinal_addition_is_monotone/left} Left addition is \hyperref[eq:def:order_function/preserving/strict]{strictly order-preserving}:
    \begin{equation}\label{eq:thm:ordinal_addition_is_monotone/left}
      \alpha < \beta \T{implies} \gamma + \alpha < \gamma + \beta.
    \end{equation}

    \thmitem{thm:ordinal_addition_is_monotone/right} Right addition is \hyperref[def:order_function/preserving]{order-preserving}:
    \begin{equation}\label{eq:thm:ordinal_addition_is_monotone/right}
      \alpha < \beta \T{implies} \alpha + \gamma \leq \beta + \gamma.
    \end{equation}

    See \fullref{ex:ordinal_addition} for examples where the strict inequality fails.
  \end{thmenum}
\end{proposition}
\begin{proof}
  \SubProofOf{thm:ordinal_addition_is_monotone/left} We proceed by induction on \( \beta \).
  \begin{itemize}
    \item The condition \( \alpha < \beta \) is vacuously false for the base case \( \beta = 0 \), hence by \eqref{eq:def:intuitionistic_propositional_deductive_systems/rules/efq} the statement vacuously holds.

    \item Fix some nonzero \( \beta \) and some \( \alpha < \beta \). If \( \gamma + \alpha < \gamma + \beta \), then
    \begin{equation*}
      \gamma + \alpha < \gamma + \beta < \op{succ}(\gamma + \beta) = \gamma + \op{succ}(\beta).
    \end{equation*}

    Since \( \beta < \op{succ}(\beta) \), we have used the inductive hypothesis to conclude that
    \begin{equation*}
      \alpha < \op{succ}(\beta) \T{implies} \gamma + \alpha < \gamma + \op{succ}(\beta).
    \end{equation*}

    \item Let \( \lambda \) be a limit ordinal and suppose that \eqref{eq:thm:ordinal_addition_is_monotone/left} holds for all \( \beta < \lambda \). Let \( \alpha < \lambda \). Then \( \op{succ}(\alpha) < \lambda \) since \( \lambda \) is a limit ordinal and thus
    \begin{equation*}
      \gamma + \alpha
      <
      \gamma + \op{succ}(\alpha)
      \leq
      \sup\set{ \gamma + \beta \given \beta < \lambda }
      =
      \gamma + \lambda.
    \end{equation*}
  \end{itemize}

  \SubProofOf{thm:ordinal_addition_is_monotone/right} We proceed by induction on \( \gamma \).
  \begin{itemize}
    \item The base case \( \gamma = 0 \) is vacuous.
    \item If \( \alpha + \gamma < \beta + \gamma \), then
    \begin{equation*}
      \alpha + \op{succ}(\gamma)
      \reloset {\eqref{eq:def:ordinal_arithmetic/addition}} =
      \op{succ}(\alpha + \gamma)
      \reloset {\ref{thm:ordinal_successor_strictly_monotone_on_ordinals}} <
      \op{succ}(\beta + \gamma)
      \reloset {\eqref{eq:def:ordinal_arithmetic/addition}} =
      \beta + \op{succ}(\gamma).
    \end{equation*}

    \item Let \( \lambda \) be a limit ordinal and suppose that the lemma holds for every \( \gamma < \lambda \). That is, for every \( \gamma < \lambda \) we have
    \begin{equation*}
      \alpha + \gamma < \beta + \gamma.
    \end{equation*}

    Thus,
    \begin{equation*}
      \alpha + \lambda
      =
      \sup\set{ \alpha + \gamma \given \gamma < \lambda }
      \leq
      \sup\set{ \beta + \gamma \given \gamma < \lambda }
      =
      \beta + \lambda.
    \end{equation*}

    We cannot make a stronger conclusion here --- see \fullref{ex:ordinal_addition} for a counterexample.
  \end{itemize}
\end{proof}

\begin{proposition}\label{thm:ordinal_ordering_via_addition}
  For any two ordinals \( \alpha \) and \( \beta \) it holds that \( \alpha \leq \beta \) if and only if there exists an ordinal \( \gamma \) such that \( \alpha + \gamma = \beta \). This ordinal is unique and satisfies \( \gamma \leq \beta \).

  The strict inequality \( \alpha < \beta \) holds if and only if \( \gamma \neq 0 \).
\end{proposition}
\begin{proof}
  \SufficiencySubProof By definition \( \beta + 0 = \beta \), hence we are not interested in the case \( \alpha = \beta \). That is, we will only consider the case \( \alpha < \beta \).

  We will first show uniqueness of \( \gamma \). Suppose that \( \alpha + \gamma_1 = \beta = \alpha + \gamma_2 \). From \eqref{thm:ordinal_addition_is_monotone/left} it follows that if either \( \gamma_1 < \gamma_2 \) or \( \gamma_1 > \gamma_2 \), we would have a strict inequality. Hence, it only remains for \( \gamma_1 = \gamma_2 \) to hold.

  We now use induction on \( \beta \) on prove the existence of \( \gamma \).
  \begin{itemize}
    \item The condition \( \alpha < \beta \) is vacuously false for the base case \( \beta = 0 \), hence by \eqref{eq:def:intuitionistic_propositional_deductive_systems/rules/efq} the statement vacuously holds.

    \item Suppose that \( \alpha < \beta \) and that there exists a unique \( \gamma \leq \beta \) such that \( \alpha + \gamma = \beta \). Then
    \begin{equation*}
      \alpha + \op{succ}(\gamma)
      \reloset {\eqref{eq:def:ordinal_arithmetic/addition}} =
      \op{succ}(\alpha + \gamma)
      =
      \op{succ}(\beta).
    \end{equation*}

    Since \( \alpha < \beta \) and \( \beta < \op{succ}(\beta) \), we have used the inductive hypothesis to conclude that
    \begin{equation*}
      \alpha < \op{succ}(\beta) \T{implies} \qexists {\underbrace{\delta}_{\mathclap{\op{succ}(\gamma)}}} \alpha + \delta = \op{succ}(\beta).
    \end{equation*}

    Furthermore, since \( \gamma \leq \beta \), then also \( \op{succ}(\gamma) \leq \op{succ}(\beta) \).

    \item Suppose that \( \lambda \) is a limit ordinal, \( \alpha < \lambda \) and for each \( \beta < \lambda \) there exists some \( \gamma_\beta \leq \beta \) such that \( \alpha + \gamma_\beta = \beta \). Define
    \begin{equation*}
      \gamma \coloneqq \sup\set{ \gamma_\beta \given \beta < \lambda }.
    \end{equation*}

    By \fullref{thm:union_of_set_of_ordinals} we have that \( \gamma \) is an ordinal and that \( \gamma_\beta \leq \gamma \) for every \( \beta < \lambda \). Thus,
    \begin{align*}
      \lambda
      &\reloset {\ref{thm:ordinal_is_set_of_smaller_ordinals}} =
      \sup\set{ \beta \given \beta < \lambda }
      \reloset {\T{ind.}} = \\ &=
      \sup\set{ \alpha + \gamma_\beta \given \beta < \lambda }
      \reloset {\eqref{eq:thm:ordinal_addition_is_monotone/right}} \leq \\ &\leq
      \sup\set[\Big]{ \alpha + \delta \given \delta < \sup\set{ \gamma_\beta \given \beta < \lambda } }
      = \\ &=
      \sup\set{ \alpha + \delta \given \delta < \gamma }
      \reloset {\eqref{eq:def:ordinal_arithmetic/addition}} = \\ &=
      \alpha + \gamma.
    \end{align*}

    Aiming at a contradiction, suppose that the strict inequality holds. That is, suppose that \( \lambda < \alpha + \gamma \). Then there exists some \( \delta_0 < \gamma \) such that \( \alpha + \delta_0 > \alpha + \gamma_\beta \) for any \( \beta < \lambda \). It follows from \fullref{thm:def:partially_ordered_set} that \( \gamma_\beta < \delta_0 \) for any \( \beta < \lambda \) and thus
    \begin{equation*}
      \underbrace{\sup\set{ \gamma_\beta \given \beta < \lambda }}_{\gamma} \leq \delta_0 < \gamma.
    \end{equation*}

    The obtained contradiction shows that such an ordinal \( \delta_0 \) cannot exist and hence \( \lambda = \alpha + \gamma \).

    Furthermore, since \( \gamma_\beta \leq \beta \) for each \( \beta < \lambda \), we have
    \begin{equation*}
      \gamma
      =
      \sup\set{ \gamma_\beta \given \beta < \lambda }
      \leq
      \sup\set{ \beta \given \beta < \lambda }
      =
      \lambda.
    \end{equation*}
  \end{itemize}

  \NecessitySubProof Suppose that \( \alpha \), \( \beta \) and \( \gamma \leq \beta \) are ordinals and that \( \alpha + \gamma = \beta \). Obviously \( \gamma = 0 \) implies that \( \alpha = \beta \). If \( \gamma > 0 \), then from \eqref{eq:thm:ordinal_addition_is_monotone/left} it follows that
  \begin{equation*}
     \beta = \alpha + \gamma > \alpha + 0 = 0.
  \end{equation*}
\end{proof}

\begin{proposition}\label{thm:ordinal_addition_algebraic_properties}
  Ordinal number addition is \hyperref[def:binary_operation/associative]{associative} and \hyperref[def:binary_operation/cancellative]{left cancellative}.

  As in \fullref{thm:ordinals_are_well_ordered}, we adapt the corresponding axioms due to \fullref{thm:burali_forti_paradox}. The more concrete result is:
  \begin{thmenum}
    \thmitem{thm:ordinal_addition_algebraic_properties/associative} For any three ordinals \( \alpha \), \( \beta \) and \( \gamma \) we have
    \begin{equation*}
      (\alpha + \beta) + \gamma = \alpha + (\beta + \gamma).
    \end{equation*}

    \thmitem{thm:ordinal_addition_algebraic_properties/left_cancellative} For any three ordinals \( \alpha \), \( \beta \) and \( \gamma \) such that \( \gamma + \alpha = \gamma + \beta \), we have \( \alpha = \beta \).
  \end{thmenum}

   See \fullref{ex:ordinal_addition} for counterexamples to \hyperref[def:binary_operation/commutative]{commutativity}.

   Compare this with \fullref{thm:natural_number_addition_properties} and \fullref{thm:cardinal_addition_algebraic_properties}.
\end{proposition}
\begin{proof}
  \SubProofOf{thm:ordinal_addition_algebraic_properties/associative} We will use induction on \( \gamma \). \Fullref{thm:natural_number_addition_properties} already proves the base and successor cases.

  Fix some ordinals \( \alpha \) and \( \beta \). Let \( \lambda \) be a limit ordinal and suppose that
  \begin{equation*}
    (\alpha + \beta) + \gamma = \alpha + (\beta + \gamma)
  \end{equation*}
  holds for all \( \gamma < \lambda \). Then
  \begin{align*}
    (\alpha + \beta) + \lambda
    &\reloset {\eqref{eq:def:ordinal_arithmetic/addition}} =
    \sup\set{ (\alpha + \beta) + \gamma \given \gamma < \lambda }
    \reloset {\T{ind.}} = \\ &=
    \sup\set{ \alpha + (\beta + \gamma) \given \gamma < \lambda }
    \reloset {\eqref{eq:def:order_function/preserving/strict}} = \\ &=
    \sup\set{ \alpha + \delta \given \delta < \beta + \lambda }
    =
    \alpha + (\beta + \lambda).
  \end{align*}

  \SubProofOf{thm:ordinal_addition_algebraic_properties/left_cancellative} Follows from \fullref{thm:def:partially_ordered_set} and \eqref{eq:thm:ordinal_addition_is_monotone/left}.
\end{proof}

\begin{proposition}\label{thm:ordinal_addition_disjoin_union}
  For any two ordinals \( \alpha \) and \( \beta \), their \hyperref[def:ordinal_arithmetic/addition]{sum} satisfies
  \begin{equation*}
    \alpha + \beta = \ord(\alpha \amalg \beta, \prec),
  \end{equation*}
  where \( \prec \) is the \hyperref[def:lexicographic_order]{reverse lexicographic order} on the \hyperref[def:disjoint_union]{disjoint union} \( \alpha \amalg \beta \).
\end{proposition}
\begin{comments}
  \item By our definition of disjoint union, we have
  \begin{equation*}
    \alpha \amalg \beta = \set{ (\gamma, 0) \given \gamma < \alpha } \cup \set{ (\delta, 1) \given \beta < \beta } \subseteq (\alpha \cup \beta) \times \set{ 0, 1 }.
  \end{equation*}

  Under the reverse lexicographic order, we have \( (\gamma, 0) \prec (\delta, 1) \) for every \( \gamma < \alpha \) and \( \delta < \beta \).

  If \( \alpha < \beta \), there are no elements of the form \( (\delta, 0) \), where \( \delta \geq \alpha \). Similarly, if \( \alpha > \beta \), there are no elements of the form \( (\gamma, 1) \).
\end{comments}
\begin{proof}
  We will explicitly build an \hyperref[def:order_function/isomorphism]{order isomorphism} between \( (\alpha + \beta, \in) \) and \( (\alpha \amalg \beta, \prec) \). Define
  \begin{equation*}
    \begin{aligned}
      &f: (\alpha + \beta) \to (\alpha \amalg \beta) \\
      &f(\gamma) \coloneqq \begin{cases}
        (\gamma, 0), &\gamma < \alpha \\
        (\delta, 1), &\qexists \delta (\gamma = \alpha + \delta).
      \end{cases}
    \end{aligned}
  \end{equation*}

  From \fullref{thm:ordinal_ordering_via_addition} it follows that the existence of \( \delta \) such that \( \gamma = \alpha + \delta \) is equivalent to the condition \( \gamma \geq \alpha \). Since \( \gamma < \alpha + \beta \), we have \( \alpha + \delta < \alpha + \beta \) and from \fullref{thm:ordinal_addition_algebraic_properties/left_cancellative} we have \( \delta < \beta \). Therefore, \( f \) is a total function. Furthermore, it is single-valued because of the uniqueness of \( \delta \).

  We will first show that \( f \) is a strict order homomorphism. Let \( \gamma_1 < \gamma_2 \). We have the following possibilities:
  \begin{itemize}
    \item If \( \gamma_2 < \alpha \), then \( f(\gamma_1) = (\gamma_1, 0) < (\gamma_2, 0) = f(\gamma_2) \).
    \item If \( \gamma_1 \geq \alpha \), then \( f(\gamma_1) = (\gamma_1, 1) < (\gamma_2, 1) = f(\gamma_2) \).
    \item If \( \gamma_1 < \alpha \leq \gamma_2 \), then \( f(\gamma_1) = (\gamma_1, 0) < (\gamma_2, 1) = f(\gamma_2) \).
  \end{itemize}

  Therefore, \( f \) is a strict order homomorphism and from \fullref{thm:def:totally_ordered_set/embedding_iff_strict} it follows that \( f \) is an order embedding. Due to \fullref{thm:def:totally_ordered_set/embedding_iff_strict}, in order to show that \( f \) is an order isomorphism it only remains to show that it is a surjective function.

  Let \( (\gamma, k) \in \alpha \amalg \beta \).
  \begin{itemize}
    \item If \( k = 0 \), then \( f(\gamma) = (\gamma, k) \) since \( \gamma \in \alpha \).
    \item If \( k = 1 \), then \( \gamma \in \beta \) and by \eqref{eq:thm:ordinal_addition_is_monotone/left} we have \( \alpha + \gamma < \alpha + \beta \), so \( \alpha + \gamma \) is within the domain of \( f \). Furthermore, as shown in \fullref{thm:ordinal_ordering_via_addition}, if \( \alpha + \delta = \alpha + \gamma \), then \( \delta = \gamma \), Thus, \( f(\alpha + \gamma) = (\gamma, 1) \) .
  \end{itemize}

  Therefore, \( f \) is an order isomorphism between \( (\alpha + \beta, \in) \) and \( (\alpha \amalg \beta, \prec) \) and hence
  \begin{equation*}
    \alpha + \beta = \ord(\alpha \amalg \beta, \prec).
  \end{equation*}
\end{proof}

\begin{example}\label{ex:ordinal_addition}
  The distinction between \( \alpha + \beta \) and \( \beta + \alpha \) is important. A simple example is provided by any limit ordinal \( \lambda \), in particular by \( \omega \). The examples are inconvenient to demonstrate with the recursive definition, however \fullref{thm:ordinal_addition_disjoin_union} eases us.

  In particular, \fullref{thm:ordinal_addition_disjoin_union} highlights that adding one ordinal to another, in an informal sense, \enquote{appending} a copy of the second to a copy the first.

  It is clear that
  \begin{equation*}
    0 + \lambda = \ord(0 \sqcap \lambda) = \ord(\lambda) = \lambda.
  \end{equation*}

  That is, we \enquote{append} \( \lambda \) to an empty well-ordered set only to obtain \( \lambda \) again.

  This operation seems different from \( 1 + \lambda \), which \enquote{appends} \( \lambda \) to a well-ordered singleton set. But this operation only \enquote{shifts} \( \lambda \) --- the function
  \begin{equation*}
    \begin{aligned}
      &f: \ord(1 \sqcap \lambda) \to \ord(0 \sqcap \lambda) \\
      &f(k, \gamma) \coloneqq \begin{cases}
        (0, 0),          &k = 0 \\
        (0, \gamma + 1), &k = 1.
      \end{cases}
    \end{aligned}
  \end{equation*}
  is an order isomorphism and thus
  \begin{equation*}
    1 + \lambda = \ord(1 \sqcap \lambda) = \ord(\lambda) = \lambda.
  \end{equation*}

  What \fullref{thm:def:partially_ordered_set} gives us is that
  \begin{equation*}
    \lambda \leq 1 + \lambda = \lambda.
  \end{equation*}

  This inequality is, of course, strict when dealing with finite ordinals exclusively, but for limit ordinals its results may be counterintuitive.

  What is more interesting is that, as a consequence of \fullref{thm:def:partially_ordered_set}, we have \( \lambda < \lambda + 1 \). This can be explained as follows. Instead of \enquote{appending} an infinite set to a finite one, we append a finite set to an infinite one. This way \( \lambda \) cannot \enquote{absorb} \( 1 \) like it does in \( 1 + \lambda \).

  As a consequence of this example, addition of ordinals is not commutative and also not right-cancellative.

  As discussed in our proof of \fullref{thm:cardinal_addition_algebraic_properties}, this is only a restriction of well-orders and not of the resulting sets themselves.
\end{example}

\begin{proposition}\label{thm:ordinal_multiplication_cartesian_product}
  For any two ordinals \( \alpha \) and \( \beta \), their \hyperref[def:ordinal_arithmetic/multiplication]{product} satisfies
  \begin{equation*}
    \alpha \cdot \beta = \ord(\alpha \times \beta, \prec),
  \end{equation*}
  where \( \prec \) is the \hyperref[def:lexicographic_order]{reverse lexicographic order} on the \hyperref[def:cartesian_product]{Cartesian product} \( \alpha \times \beta \).
\end{proposition}
\begin{proof}
  We will build an order isomorphism between \( (\alpha \cdot \beta, \in) \) and \( (\alpha \times \beta, \prec) \) using recursion on \( \beta \).
  \begin{itemize}
    \item Both sets \( \alpha \cdot 0 \) and \( \alpha \times 0 \) are empty, and the empty function is an order isomorphism.
    \item Suppose that \( f: \alpha \cdot \beta \to \alpha \times \beta \) is an order isomorphism. Consider the function
    \begin{equation*}
      \begin{aligned}
        &\widehat f: \alpha \cdot (\beta + 1) \to \alpha \times (\beta + 1) \\
        &\widehat f \coloneqq \begin{cases}
          f(\gamma),       &\gamma < \alpha \cdot \beta \\
          (\delta, \beta), &\qexists \delta (\gamma = \alpha \cdot \beta + \delta).
        \end{cases}
      \end{aligned}
    \end{equation*}

    In complete analogy with \fullref{thm:ordinal_ordering_via_addition}, we can prove that \( \widehat f \) is an order isomorphism.

    \item Let \( \lambda \) be a limit ordinal and let \( f_\beta: \alpha \cdot \beta \to \alpha \times \beta \) be an order isomorphism for every \( \beta < \lambda \). Take their union
    \begin{equation*}
      f \coloneqq \bigcup\set{ f_\beta \given \beta < \lambda }.
    \end{equation*}

    The uniqueness of each \( f_\beta \) from \fullref{thm:well_ordered_order_type_existence} shows that \( f_{\beta_1} \subseteq f_{\beta_2} \) for each pair \( \beta_1 < \beta_2 \). Therefore, the union \( f \) is a single-valued partial function. It is also total because every ordinal \( \gamma < \alpha \cdot \lambda \) is contains in the image of the function \( f_{\gamma + 1} \).

    The function \( f \) is an order embedding by construction. It is also surjective because, if \( (\delta, \beta) \in \alpha \times \lambda \), then from the successor step we can conclude that \( f(\alpha \cdot \beta + \delta) = (\delta, \beta) \).

    Therefore, \( f \) is an order isomorphism.
  \end{itemize}
\end{proof}

\begin{proposition}\label{thm:ordinal_multiplication_algebraic_properties}
  Similarly to \fullref{thm:ordinal_addition_algebraic_properties} for ordinal number addition, multiplication is also \hyperref[def:binary_operation/associative]{associative} and \hyperref[def:binary_operation/cancellative]{left cancellative}:
  \begin{thmenum}
    \thmitem{thm:ordinal_multiplication_algebraic_properties/associative} For any three ordinals \( \alpha \), \( \beta \) and \( \gamma \) we have
    \begin{equation*}
      (\alpha \cdot \beta) \cdot \gamma = \alpha \cdot (\beta \cdot \gamma).
    \end{equation*}

    \thmitem{thm:ordinal_multiplication_algebraic_properties/left_cancellative} For any three ordinals \( \alpha \), \( \beta \) and \( \gamma \) such that \( \gamma \cdot \alpha = \gamma \cdot \beta \), we have \( \alpha = \beta \).
  \end{thmenum}

  Compare this with \fullref{thm:natural_number_multiplication_properties} and \fullref{thm:cardinal_multiplication_algebraic_properties}.
\end{proposition}
\begin{proof}
  \SubProofOf{thm:ordinal_multiplication_algebraic_properties/associative} Associativity follows easily from the obvious isomorphisms between \( (\alpha \times \beta) \times \gamma \) and \( \alpha \times (\beta \times \gamma) \).

  \SubProofOf{thm:ordinal_multiplication_algebraic_properties/left_cancellative} Now suppose that \( \gamma \cdot \alpha = \gamma \cdot \beta \). Let \( f \) the unique order isomorphism between \( \gamma \times \alpha \) and \( \gamma \times \beta \).

  Suppose that \( \alpha \neq \beta \). Without loss of generality, suppose that \( \beta \subsetneq \alpha \). Then \( f\restr_{\gamma \times \alpha} \) is the identity mapping and hence any set from \( \gamma \times (\beta \setminus \alpha) \) would make \( f \) not injective.

  The obtained contradiction shows that \( \alpha = \beta \).
\end{proof}

\begin{example}\label{ex:countable_limit_ordinals}
  We already know that \( \omega \) is a limit ordinal. It is clear from \fullref{ex:ordinal_addition} that \( \omega + n \) is a successor ordinal for every natural number \( n \).

  What about \( \omega + \omega \)? This corresponds to \enquote{placing} two copies of the natural numbers one after another.

  Suppose that \( \omega + \omega = \omega \cdot 2 \) is the successor of \( \alpha \). Then \( \alpha < \omega + \omega \) and we can show by induction on the natural numbers that \( \alpha + n < \omega + \omega \). But \( \alpha + 1 = \omega \) by assumption, which contradicts trichotomy of ordinals.

  Therefore, \( \omega + \omega \) is a limit ordinal. Furthermore, \( \omega + \omega \) is the second smallest limit ordinal since only ordinals of the form \( \omega + n \) for nonzero finite \( n \) satisfy \( \omega < \omega + n < \omega + \omega \).

  Both ordinals \( \omega \) and \( \omega + \omega \) are countable by \fullref{thm:omega_equinumerous_with_omega_squared}.

  Another limit ordinal is \( \omega \cdot \omega = \omega^2 \). It is also countable by \fullref{thm:countable_product_of_countable_sets}. Actually \( \omega^n \) for any natural number \( n \) is countable by the same theorem.

  Therefore, any \enquote{\hyperref[def:polynomial_algebra/polynomials]{polynomial}} of the form
  \begin{equation*}
    \alpha_n \omega^n + \alpha_{n-1} \omega^{n-1} + \cdots + \alpha_1 \omega + \alpha_0
  \end{equation*}
  with countable coefficients is also countable.
\end{example}

\begin{proposition}\label{thm:square_of_infinite_ordinal}
  If \( \alpha \) is an infinite ordinal, then \( \alpha = \alpha^2 \).
\end{proposition}
\begin{proof}
  We will show that \( \alpha \) is order isomorphic to its \hyperref[def:lexicographic_order]{lexicographically ordered} square
  \begin{equation*}
    S \coloneqq \alpha \times \alpha.
  \end{equation*}

  Aiming at a contradiction, suppose that \( \alpha \) is the smallest ordinal for which this does not hold. \Fullref{thm:omega_equinumerous_with_omega_squared} implies that \( \alpha \geq \omega \).

  Since \( \alpha < \ord(\alpha \times \alpha) \), there \fullref{thm:initial_segment_of_ordinal} implies the existence of some segment \( S_{<(\beta_1, \beta_2)} \) such that
  \begin{equation*}
    \alpha = \ord(S_{<(\beta_1, \beta_2)}),
  \end{equation*}
  where
  \begin{equation*}
    S_{<(\beta_1, \beta_2)} = \set[\Big]{ (\gamma_1, \gamma_2) \in \alpha \times \alpha \given* \gamma_1 < \beta_1 \T{or} (\gamma_1 = \beta_1 \T{and} \gamma_2 < \beta_2) }
  \end{equation*}
  is the corresponding \hyperref[def:order_interval/unbounded]{initial segment}.

  Let \( \beta \coloneqq \max\set{ \beta_1, \beta_2 } \). Then
  \begin{equation*}
    S_{<(\beta_1, \beta_2)} \subsetneq \beta \times \beta.
  \end{equation*}

  Since \( \beta < \alpha \), the lemma holds for \( \beta \), and thus
  \begin{equation*}
    \alpha
    =
    \ord(S_{<(\beta_1, \beta_2)})
    \leq
    \ord(\beta \times \beta)
    =
    \beta
    <
    \alpha.
  \end{equation*}

  The obtained contradiction shows that \( \alpha = \ord(\alpha \times \alpha) \) for every infinite ordinal \( \alpha \).
\end{proof}

\begin{definition}\label{def:cardinal_arithmetic}
  We will define arithmetic operations for them. Unlike in \fullref{def:ordinal_arithmetic}, we will directly define the operations as \hyperref[thm:cardinality_existence]{cardinal numbers} of some sets rather than via some form of recursion.

  Fix two ordinals \( \kappa \) and \( \mu \).
  \begin{thmenum}
    \thmitem{def:cardinal_arithmetic/addition}\mcite[def. 6.4.29]{Hinman2005} Based on \fullref{thm:ordinal_addition_disjoin_union}, we define their \term{sum} as
    \begin{equation*}
      \kappa + \mu \coloneqq \card(\kappa \amalg \mu),
    \end{equation*}
    where \( \kappa \amalg \mu \) is their \hyperref[def:disjoint_union]{disjoint union}.

    \thmitem{def:cardinal_arithmetic/multiplication}\mcite[def. 6.4.29]{Hinman2005} Based on \fullref{thm:ordinal_multiplication_cartesian_product}, we define their \term{product} as
    \begin{equation*}
      \kappa \cdot \mu \coloneqq \card(\kappa \times \mu).
    \end{equation*}

    \thmitem{def:cardinal_arithmetic/exponentiation}\mcite[exer. 6.4.46]{Hinman2005} We define \term{exponentiation} as
    \begin{equation*}
      \kappa^\mu \coloneqq \card(\fun(\kappa, \mu)).
    \end{equation*}
  \end{thmenum}
\end{definition}

\begin{proposition}\label{thm:cardinal_addition_algebraic_properties}
  Cardinal number addition is \hyperref[def:binary_operation/associative]{associative}, \hyperref[def:binary_operation/associative]{commutative} and \hyperref[def:binary_operation/cancellative]{cancellative}.

  Compare this with \fullref{thm:natural_number_addition_properties} and \fullref{thm:ordinal_addition_algebraic_properties}.
\end{proposition}
\begin{proof}
  Associativity and left cancellation is inherited from the ordinals. Commutativity and right cancellation hold because we are considering arbitrary bijective functions rather than the more restrictive order isomorphisms. Indeed, \( \kappa \amalg \mu \) and \( \mu \amalg \kappa \) may have different order types as demonstrated in \fullref{ex:ordinal_addition}, however there is an obvious bijective function between them.
\end{proof}

\begin{proposition}\label{thm:cardinal_multiplication_algebraic_properties}
  Cardinal number multiplication is \hyperref[def:binary_operation/associative]{associative}, \hyperref[def:binary_operation/associative]{commutative} and \hyperref[def:binary_operation/cancellative]{cancellative}.

  Compare this with \fullref{thm:natural_number_multiplication_properties} and \fullref{thm:ordinal_addition_algebraic_properties}.
\end{proposition}
\begin{proof}
  The result follows from the same considerations as in \fullref{thm:cardinal_addition_algebraic_properties}.
\end{proof}

\begin{proposition}\label{thm:double_and_square_of_cardinal}
  For every cardinal \( \kappa \) we have
  \begin{thmenum}
    \thmitem{thm:double_and_square_of_cardinal/double} \( \kappa + \kappa = 2\kappa \).
    \thmitem{thm:double_and_square_of_cardinal/square} \( \kappa \cdot \kappa = \kappa^2 \).
  \end{thmenum}
\end{proposition}
\begin{proof}
  \SubProofOf{thm:double_and_square_of_cardinal/double} Obviously \( \kappa \amalg \kappa = 2 \times \kappa \).
  \SubProofOf{thm:double_and_square_of_cardinal/square} The function
  \begin{equation*}
    \begin{aligned}
      &T: \fun(2, \kappa) \to \kappa \times \kappa \\
      &T(f) \coloneqq (f(0), f(1))
    \end{aligned}
  \end{equation*}
  is clearly injective. It is also surjective because for any ordered pair \( (\gamma, \delta) \in \kappa \times \kappa \) we can define the function
  \begin{equation*}
    \begin{aligned}
      &f: 2 \to \kappa \\
      &f(k) \coloneqq \begin{cases}
        \gamma, k = 0 \\
        \delta, k = 1
      \end{cases}
    \end{aligned}
  \end{equation*}

  Then \( T(f) = (\gamma, \delta) \).
\end{proof}

\begin{proposition}\label{thm:simplified_cardinal_arithmetic}
  Unlike \hyperref[def:ordinal_arithmetic]{ordinal arithmetic} with its intricacies like \fullref{ex:ordinal_addition}, \hyperref[def:cardinal_arithmetic]{cardinal arithmetic} has a simpler behavior:
  \begin{thmenum}
    \thmitem{thm:simplified_cardinal_arithmetic/finite} If \( \kappa \) and \( \mu \) are finite cardinals, then \( \kappa + \mu \) and \( \kappa \cdot \mu \) are the familiar operations on \hyperref[def:natural_numbers]{natural numbers}.

    \thmitem{thm:simplified_cardinal_arithmetic/infinite}\mcite[6.4.35]{Hinman2005} If either \( \kappa \) or \( \mu \) is infinite, then
    \begin{equation*}
      \kappa + \mu = \max\set{ \kappa, \mu }.
    \end{equation*}

    If, additionally, both are nonzero, then
    \begin{equation*}
      \kappa \cdot \mu = \max\set{ \kappa, \mu }.
    \end{equation*}
  \end{thmenum}
\end{proposition}
\begin{proof}
  \SubProofOf{thm:simplified_cardinal_arithmetic/finite} The addition of ordinals defined in \fullref{def:ordinal_arithmetic/addition} is an extension of addition of natural numbers, hence the two are equivalent for finite ordinals. The equivalence with cardinal addition defined in \fullref{def:cardinal_arithmetic/addition} comes from \fullref{thm:ordinal_addition_disjoin_union} and the fact that every finite ordinal is a cardinal as demonstrated in \fullref{thm:natural_numbers_are_cardinals}.

  Analogously, equivalence of cardinal and ordinal multiplication follows from \fullref{thm:ordinal_multiplication_cartesian_product}.

  \SubProofOf{thm:simplified_cardinal_arithmetic/infinite} Suppose that either \( \kappa \) or \( \mu \) is infinite and let \( \nu \coloneqq \max\set{ \kappa, \mu } \). The cases where either of them is zero are trivial, hence suppose that both are nonzero.

  We have
  \begin{equation*}
    \kappa \amalg \mu \subseteq \nu \amalg \nu = 2 \amalg \nu \subseteq \nu \times \nu,
  \end{equation*}
  hence \( \kappa + \mu \leq \nu \cdot \nu = \nu^2 \).

  Furthermore there exists an obvious injective function from \( \nu \) to \( \kappa \amalg \mu \) (which is different depending on whether \( \nu = \kappa \) or \( \nu = \mu \)). Therefore,
  \begin{equation*}
    \nu \leq \kappa + \mu \leq \nu^2.
  \end{equation*}

  For multiplication we have \( \kappa \times \mu \subseteq \nu \times \nu \), hence
  \begin{equation*}
    \nu \leq \kappa \cdot \mu \leq \nu^2.
  \end{equation*}

  The rest follows from \fullref{thm:square_of_infinite_ordinal}.
\end{proof}

\begin{corollary}\label{thm:aleph_zero_is_strong_limit}
  The first infinite cardinal \( \aleph_0 \) is a \hyperref[def:successor_and_limit_cardinal/strong_limit]{strong limit cardinal}.
\end{corollary}
\begin{comments}
  \item See also \fullref{thm:aleph_zero_is_regular}
\end{comments}
\begin{proof}
  \Fullref{thm:simplified_cardinal_arithmetic/finite} states that cardinal exponentiation extends natural number exponentiation. Hence, we can conclude that \( 2^n < \aleph_0 \) for any \( n < \aleph_0 \) since the former are finite and the latter is not.
\end{proof}

\begin{lemma}\label{thm:power_set_via_subsets}
  Fix a set \( A \). The power set \( \pow(A) \) is equinumerous with the set of \hyperref[def:boolean_operator]{Boolean-valued functions} \( \pow(A, \set{ T, F }) \).

  More precisely, then the operator
  \begin{equation*}
    \begin{aligned}
      &T: \fun(A, \set{ T, F }) \to \pow(A) \\
      &T(f) \coloneqq \set{ f(x) = T \given x \in A }.
    \end{aligned}
  \end{equation*}
  is bijective.
\end{lemma}
\begin{proof}
  Injectivity is clear. To see surjectivity, fix some subset \( B \subset A \) and define
  \begin{equation*}
    \begin{aligned}
      &f: A \to \set{ T, F } \\
      &f(x) \coloneqq \begin{cases}
        T, &x \in B \\
        F, \T{otherwise}.
      \end{cases}
    \end{aligned}
  \end{equation*}

  Clearly \( f \in \fun(A, \set{ T, F }) \) and \( T(f) = B \).
\end{proof}

\begin{proposition}\label{thm:cardinal_exponentiation_power_set}
  For every set \( A \) we have
  \begin{equation*}
    \card(\pow(A)) = 2^{\card(A)}.
  \end{equation*}
\end{proposition}
\begin{proof}
  The proof is similar to \fullref{thm:power_set_via_subsets}, but much more convoluted.

  Let \( \varphi: A \to \card(A) \) be \( \psi: \pow(A) \to \card(\pow(A)) \) bijective functions. Note that
  \begin{equation*}
    2^{\card(A)} = \card(\fun(\card(A), \set{ 0, 1 }))
  \end{equation*}
  by definition. Let \( \theta: \fun(\card(A), \set{ 0, 1 }) \to 2^{\card(A)} \) be a bijective function.

  Define the operator
  \begin{equation*}
    \begin{aligned}
      &T: 2^{\card(A)} \to \card(\pow(A)) \\
      &T(p) \coloneqq \psi\parens{ \set{ \varphi^{-1}(\gamma) \given \gamma \in \card(A) \T{and} \theta^{-1}(p)(\gamma) = 1 } }.
    \end{aligned}
  \end{equation*}

  This operator is bijective since for any \( \delta \in \card(\pow(A)) \) we can define
  \begin{equation*}
    \begin{aligned}
      &f: \card(A) \to \set{ 0, 1 } \\
      &f(\gamma) \coloneqq \begin{cases}
        1, &\varphi^{-1}(\gamma) \in \psi^{-1}(\delta) \\
        0, &\T{otherwise.}
      \end{cases}
    \end{aligned}
  \end{equation*}
  so that \( T(\theta(f)) = \delta \).

  Therefore, \( \card(A) \) and \( \card(\pow(A)) \) are equinumerous.
\end{proof}

\begin{corollary}\label{thm:cardinal_exponentiation_comparison}
  We have \( \card A < \card B \) if and only if \( 2^{\card A} < 2^{\card B} \).
\end{corollary}
\begin{proof}
  \SufficiencySubProof Suppose that \( \card A < \card B \).

  Let \( f: A \to B \) be an injective function. It is not surjective by assumption. Then \( f(A) \subsetneq B \).

  Clearly \( \pow(f(A)) \subsetneq \pow(B) \). \Fullref{thm:cardinal_exponentiation_power_set} then implies that \( 2^{\card A} < 2^{\card B} \).

  \NecessitySubProof Suppose that \( 2^{\card A} < 2^{\card B} \).
  \begin{itemize}
    \item If \( \card A = \card B \), then \( 2^{\card A} = 2^{\card B} \), which is a contradiction.
    \item If \( \card A > \card B \), then we have shown that \( 2^{\card A} > 2^{\card B} \), which is again a contradiction.
    \item It remains for \( \card A \) to be strictly less than \( \card B \).
  \end{itemize}
\end{proof}

\begin{proposition}\label{thm:strong_limit_cardinal_is_weak_limit}
  If \( \mu \) is the successor cardinal of \( \kappa \), then \( \mu \leq 2^\kappa \).

  In particular, every \hyperref[def:successor_and_limit_cardinal/strong_limit]{strong limit cardinal} is a \hyperref[def:successor_and_limit_cardinal/strong_limit]{weak limit cardinal}.

  Furthermore, if \( \kappa \) is infinite, then \fullref{hyp:generalized_continuum_hypothesis} implies that \( \mu = 2^\kappa \).
\end{proposition}
\begin{proof}
  It is clear from \fullref{thm:cantor_power_set_theorem} and \fullref{thm:cardinal_exponentiation_power_set} that \( \kappa < 2^\kappa \). By definition, \( \mu \) is the smallest cardinal such that \( \kappa < \mu \). Therefore, \( \mu \leq 2^\kappa \).
\end{proof}

\begin{conjecture}[Generalized continuum hypothesis]\label{hyp:generalized_continuum_hypothesis}\mcite[166]{Enderton1977Sets}
  For every ordinal \( \alpha \) we have
  \begin{equation*}
    \aleph_{\alpha + 1} = 2^{\aleph_\alpha},
  \end{equation*}

  For the definition and the properties of the \( \aleph \) hierarchy, see \fullref{def:aleph_hierarchy}. For the related \( \beth \) hierarchy, see \fullref{def:beth_hierarchy}.

  This is a vast generalization of \fullref{hyp:continuum_hypothesis} from the case \( \alpha = 0 \) to arbitrary ordinals.
\end{conjecture}

\begin{corollary}\label{thm:limit_cardinals_and_gch}
  \Fullref{hyp:generalized_continuum_hypothesis} implies that every \hyperref[def:successor_and_limit_cardinal/weak_limit]{weak limit cardinal} is a \hyperref[def:successor_and_limit_cardinal/strong_limit]{strong limit cardinal}.

  The converse holds in \logic{ZFC} as shown in \fullref{thm:strong_limit_cardinal_is_weak_limit}.
\end{corollary}
\begin{proof}
  Follows from \fullref{thm:infinite_cardinal_is_aleph} and \fullref{hyp:generalized_continuum_hypothesis}.
\end{proof}

\begin{definition}\label{def:beth_hierarchy}\mcite[214]{Enderton1977Sets}
  Similarly to \fullref{def:aleph_hierarchy}, we use transfinite recursion to define, for each ordinal \( \alpha \), the cardinal
  \begin{equation}\label{eq:def:beth_hierarchy}
    \beth_\alpha \coloneqq \begin{cases}
      \omega,                                       &\alpha = 0 \\
      2^\beta,                                      &\alpha = \op{succ}(\beta) \\
      \sup\set{ \beth_\beta \given \beta < \alpha } &\alpha \T{is a limit ordinal}.
    \end{cases}
  \end{equation}

  Unlike \fullref{def:aleph_hierarchy}, it is able to explicitly build the successor of any member of the hierarchy. \Fullref{hyp:generalized_continuum_hypothesis} states that \( \aleph_\alpha = \beta_\alpha \) for every ordinal \( \alpha \), however in general it is only provable that \( \aleph_\alpha \leq \beta_\alpha \) --- see \fullref{thm:strong_limit_cardinal_is_weak_limit}.
\end{definition}
