\section{Logical consequence}\label{sec:logical_consequence}

We introduce several notions that are common for different flavors of logic we will consider in this monograph.

\begin{concept}\label{con:logical_system}
  The phrases \term{formal system} and \term{logical system} may refer to a \hyperref[def:consequence_relation]{consequence relation} (or the less practical \hyperref[def:consequence_operator]{consequence operators}), \hyperref[def:entailment_system]{entailment system}, \hyperref[def:institution]{institution}, \hyperref[def:general_logic]{general logic} or another notion which allows performing deductive reasoning.

  This may also encompass different \hyperref[def:rewriting_system]{rewriting systems} --- for example, \incite{Barendregt1974SurjectivePairing} assumes that what he calls a \enquote{logical system} has a syntax allowing \hyperref[def:lambda_term]{\( \muplambda \)-application}.

  Entailment systems and general logics are not established concepts and the usage that we will present is an attempt to systemize different logical systems.
\end{concept}
\begin{comments}
  \item The general idea of such systems is attributed in \cite[140]{GowersEtAl2008PrincetonCompanion} to Gottlob Frege:
  \begin{displayquote}
    Frege formulated the idea of a formal system, in which one defines in advance all the allowable symbols, all the rules that produce well-formed formulas, all axioms (i.e., certain preselected, well-formed formulas), and all the rules of inference. In such systems any deduction can be checked syntactically --- in other words, by purely symbolic means.
  \end{displayquote}

  \item \incite[def. 1.1.2]{Герасимов2014Вычислимость} defines a \enquote{deductive system} (\enquote{дедуктивная система}) as a \hyperref[def:formal_language/language]{formal language} \( L \) over a fixed \hyperref[def:formal_language/alphabet]{alphabet} \( \Sigma \) (what we would call a \hyperref[con:logical_system_signature]{signature}) with a subset \( Ax \), whose elements he calls \enquote{axioms} (\enquote{аксиомы}), and a set \( R \) of \hyperref[def:relation]{relations} on \( L \) of arity at least \( 2 \), which he calls \enquote{inference rules} (\enquote{правила вывода}).

  Although this definition is more precise than our informal description, it is also restrictive. For example, the entities in \hyperref[def:abstract_type_system]{type systems} are \hyperref[def:type_assertion]{type assertions} rather than strings of symbols, while the inference rules we will consider will be standalone syntactic entities. See \cref{def:inference_rule} for the latter.
\end{comments}

\begin{remark}\label{rem:named_logical_systems}
  Several \hyperref[con:logical_system]{logical systems} have established names:
  \begin{itemize}
    \thmitem{rem:named_logical_systems/gentzen} \incite*[176]{Gentzen1935LogischeSchließen} introduces \hyperref[def:first_order_natural_deduction_system]{natural deduction systems} (which he called \enquote{Kalk\"ul des nat\"urlichen Schliessens}), and called the corresponding \hyperref[con:intuitionistic_logic]{intuitionistic} and \hyperref[con:classical_logic]{classical} variants \logic{NJ} and \logic{NK}. We formulate them in \cref{def:propositional_natural_deduction}.

    In the same paper he also introduces \hyperref[def:abstract_sequent_calculus_system]{sequent calculus}, whose variants he called \logic{LJ} (\enquote{Intuitionistische Pr\"adikatenlogik}) and \logic{LK} (\enquote{Klassik Pr\"adikatenlogik}). We formulate them in \cref{def:classical_propositional_sequent_calculus}.

    \thmitem{rem:named_logical_systems/q0} \incite*[\S 51]{Andrews2002Logic} introduces the system \( \logic{Q}_0 \), which we use as an inspiration for our \hyperref[def:higher_order_logic]{higher-order logic} --- see \cref{rem:higher_order_logic_and_type_theory/terms}.

    \thmitem{rem:named_logical_systems/peano_robinson} \incite*[def. 4.1.1]{Hinman2005Logic} calls \hyperref[def:peano_arithmetic]{Peano arithmetic} \( \logic{P} \) (which we prefer to abbreviate as \logic{PA}), and its weaker form, the \hyperref[def:peano_arithmetic]{Robinson arithmetic}, \( \logic{Q} \).

    \thmitem{rem:named_logical_systems/barendregt} \incite*{Barendregt1992LambdaCalculiWithTypes} introduces several \hyperref[def:abstract_type_system]{type systems} for \( \muplambda \)-calculus, which revolve around the letter \( \muplambda \). For example, he denotes his of \hyperref[con:simple_type_theory]{simple type theory} by \( \muplambda_{\rightarrow} \) and \hyperref[def:lambda_cube/type_on_term]{his flavor} of \hyperref[def:polymorphic_typed_lambda_calculus]{second-order polymorphic \( \muplambda \)-calculus} by \( \synlambda 2 \). Some others are listed in \cref{def:lambda_cube}.

    \thmitem{rem:named_logical_systems/f} Girard is attributed with introducing \hyperref[def:polymorphic_typed_lambda_calculus]{second-order polymorphic \( \muplambda \)-calculus} in a variant called \enquote{System \( F \)} --- see \cref{rem:polymorphic_type_origin}.
  \end{itemize}
\end{remark}

\paragraph{Consequence operators and relations}

\begin{definition}\label{def:consequence_operator}\mimprovised
  Fix a set \( S \), whose elements we will call \term[ru=предложения (\cite[103]{КолмогоровДрагалин2006Логика}), en=sentences (\cite[62]{Tarski1983MethodologyOfDeductiveSciences})]{sentences}, as per \cref{con:proposition}. If \( \op*{Cn} \) is a \hyperref[def:moore_closure_operator]{Moore closure operator} on the power set of \( S \), we say that it is a \term{Tarski consequence operator} on \( S \).

  \begin{thmenum}
    \thmitem{def:consequence_operator/compactness}\mcite[def. 4.2.1]{CitkinMuravitsky2022ConsequenceRelations} We say that \( \op*{Cn} \) is \term{compact} if the following condition holds:
    \begin{equation}\label{eq:def:consequence_operator/compactness}
      \op*{Cn}(\Gamma) = \bigcup\set{ \op*{Cn}(\Delta) \given \Delta \T{is a finite subset of} \Gamma }.
    \end{equation}
  \end{thmenum}
\end{definition}
\begin{comments}
  \item Our main source of consequence operators will be \hyperref[def:consequence_relation]{consequence relations}. Will consider two types of consequence relations:
  \begin{itemize}
    \item Syntactic consequence based on building proof trees. This will be formalized via entailment systems in \cref{def:entailment_system/entailment}.
    \item Semantic consequence based on interpreting theories as \hyperref[con:judgment]{judgments} about metatheoretic objects. This will be formalized via institutions in \cref{def:institutional_entailment}.
  \end{itemize}

  \item Tarski himself defines several variations of consequence operators. Our definition is based on \cite[64]{Tarski1983MethodologyOfDeductiveSciences}, but, following \incite[\S 4.2.1]{CitkinMuravitsky2022ConsequenceRelations}, we replace the mandatory compactness condition with the weaker \hyperref[def:order_function/preserving]{monotonicity} property. Our main motivation for this is that we want to provide a general definition that is useful without a tedious proof of compactness, which may even not exist. Tarski proves monotonicity as a consequence of compactness.

  Another distinction from Tarski's work is that we impose no restrictions on the set \( S \), while Tarski defines it to be \hyperref[def:set_countability/at_most_countable]{at most countable}.

  In \cite[31]{Tarski1983FundamentalConceptsOfMetamathematics}, which chronologically precedes the aforementioned definition, Tarski additionally requires that there exists a sentence \( \varphi \) such that \( S = \op*{Cn}(\set{ \varphi }) \).

  \item The notation \( \op*{Cn} \) is due to Tarski. We also use \( \op*{Th} \) when dealing with \hyperref[def:logical_theory]{logical theories}.

  \item Sentences themselves are not, in practice, atoms. In propositional logic, they are formulas, which themselves have an elaborate structure discussed in \cref{def:propositional_formula}. In first-order logic their structure is even more elaborate --- see \cref{def:closed_fol_formula}.

  \item We are concerned with the epistemology of the judgments used to assert properties of sentences and consequence operators. At the same time, we are not concerned with epistemology of the sentences themselves, but regard them as autonomous abstract objects which can either be manipulated mechanically.
\end{comments}

\begin{definition}\label{def:logical_context}\mimprovised
  A \term{logical context} is a syntactic list generated by the corresponding grammar rule from the \hyperref[def:formal_grammar/schema]{schema}
  \begin{bnf*}
    \bnfprod{nonempty context}    {\bnfpn{context entry} \bnfor} \\
    \bnfmore                      {\bnfpn{nonempty context} \bnfsp \bnftsq{,} \bnfsp \bnfpn{nonempty context}} \\
    \bnfprod{empty context}       {\bnfves} \\
    \bnfprod{context}             {\bnfpn{empty context} \bnfor \bnfpn{nonempty context}}
  \end{bnf*}

  We have purposely not specified rules for the nonterminal \( \bnfpn{context entry} \) in order to encompass entries with different syntax like \hyperref[def:fol_term]{first-order variables} and \hyperref[def:type_assertion]{type assertions}. We assume that, whenever this definition is used, the syntax for entries is understood from the setting.

  We require that the resulting grammar is \hyperref[def:grammar_ambiguity]{unambiguous} so that we can treat contexts as lists of entries.
\end{definition}
\begin{comments}
  \item Contexts are sometimes treated as sets rather than lists --- see \cref{rem:logical_contexts_as_sets}.

  \item Within the metalanguage, we denote abstract formulas via capital Greek letters, most notably \( \Gamma \) and \( \Delta \). This convention will later lead us to a formal definition of context schemas in \cref{def:logical_context_schema}.

  \item This definition generalizes propositional contexts from \cite[45]{Mimram2020ProgramEqualsProof} and \hyperref[def:type_context]{type contexts} from \cite[159]{Mimram2020ProgramEqualsProof} and \cite[def. 2A5]{Hindley1997BasicSTT}, but formalized via grammars and not tied to variables.
\end{comments}

\begin{remark}\label{rem:logical_contexts_as_sets}
  It is possible to treat a \hyperref[def:logical_context]{logical context} \( \Gamma = \varphi_1, \ldots, \varphi_n \) as an unordered collection, i.e. a set, thus retaining only the metalingual notation. As discussed in \cref{rem:sequent_notation}, this is an established practice for \hyperref[def:consequence_relation]{consequence relations} (but not, for example for \hyperref[def:type_derivation_relation]{type derivation relations}).

  Both ordered and unordered contexts are discussed in \cite[\S 2.2.10]{Mimram2020ProgramEqualsProof} and \cite[\S 4.1.7.2]{Mimram2020ProgramEqualsProof}, in the context of proof trees. Mimram highlights some non-uniqueness problems that will not concern us, as we will see in \cref{ex:def:classical_propositional_sequent_calculus/proof_tree_non_uniqueness}.
\end{remark}

\begin{definition}\label{def:sequent_alphabet}\mimprovised
  The \hyperref[def:formal_language/alphabet]{alphabet} of \hyperref[con:improper_symbol]{improper symbols} for sequents consists only of the sequent relation symbol, \enquote{\( \synvdash \)}.
\end{definition}

\begin{definition}\label{def:sequent}\mimprovised
  A \term[ru=секвенция (\cite[66]{Герасимов2014Вычислимость})]{sequent} is a string consisting of two \hyperref[def:logical_context]{contexts} joined by some infix relational symbol. They can be described by the \hyperref[def:formal_grammar]{grammar}
  \begin{bnf*}
    \bnfprod{sequent} {\bnfpn{context} \bnfsp \bnftsq{\( \synvdash \)} \bnfsp \bnfpn{context}}.
  \end{bnf*}
\end{definition}
\begin{comments}
  \item This definition is based on \bycite[290]{Gentzen1964LogicalDeduction}, but formalized via grammars.

  \item Although we have not put restrictions for the context on the right, we may assume that it consists of a single entry.
  \begin{itemize}
    \item An empty context on the right is often meaningless. The intended semantics of a sequent is that of a \hyperref[con:judgment]{judgment} asserting that the left context entails the right one, and if the right context is empty, there is nothing to imply.

    \item General contexts on the right are used in \fullref{sec:sequent_calculus}.
  \end{itemize}
\end{comments}

\begin{remark}\label{rem:sequent_notation}
  As mentioned in \cref{rem:logical_contexts_as_sets}, even though logical contexts in sequents are defined to be ordered, we often treat them as unordered.

  By \term{sequent notation} for a binary relation \( \vdash \), we will mean utilizing the grammar of contexts from \cref{def:logical_context} to simplify notation for \hyperref[def:consequence_relation]{consequence relations}. Given a set \( \set{ \varphi_1, \ldots, \varphi_n } \) of logical formulas which entail \( \psi \), we may write
  \begin{equation*}
    \varphi_1, \ldots, \varphi_n \vdash \psi.
  \end{equation*}

  Correspondingly, if \( \psi \) can be derived from an empty set, we write
  \begin{equation*}
    \vdash \psi.
  \end{equation*}

  In fact, across this chapter, we will use the sequent notation
  \begin{equation*}
    \Gamma \vdash \psi
  \end{equation*}
  even when \( \Gamma \) is a possibly infinite set.
\end{remark}

\begin{definition}\label{def:consequence_relation}\mcite[def. 4.1.1]{CitkinMuravitsky2022ConsequenceRelations}
  As in \cref{def:consequence_operator}, fix a set \( S \) of sentences. We say that the relation \( \vdash \) from \( \pow(S) \) to \( S \) is a \term{consequence relation} if the following conditions hold\fnote{We use the sequent notation discussed in \cref{rem:sequent_notation}, even though we allow \( \Gamma \), \( \Delta \) and \( \Epsilon \) to be infinite sets}:
  \begin{thmenum}
    \thmitem{def:consequence_relation/reflexivity} The following \hyperref[def:binary_relation/reflexive]{reflexivity}-like condition:
    \begin{equation}\label{eq:def:consequence_relation/reflexivity}
      \varphi \in \Gamma \T{implies} \Gamma \vdash \varphi.
    \end{equation}

    This corresponds to Gentzen's \hyperref[inf:def:abstract_sequent_calculus_system/init]{initialization rule} in sequent calculus.

    \thmitem{def:consequence_relation/monotonicity} The following \hyperref[def:order_function/preserving]{monotonicity}-like condition:
    \begin{equation}\label{eq:def:consequence_relation/monotonicity}
      \Gamma \vdash \varphi \T{implies} \Gamma, \Delta \vdash \varphi
    \end{equation}

    This corresponds to \hyperref[def:abstract_propositional_sequent_calculus_system/rules/weak]{weakening rules} in sequent calculus.

    \thmitem{def:consequence_relation/transitivity} The following \hyperref[def:binary_relation/transitive]{transitivity}-like condition:
    \begin{equation}\label{eq:def:consequence_relation/transitivity}
      (\qforall* {\psi \in \Delta} \Gamma \vdash \psi) \T{and} \Delta, \Epsilon \vdash \varphi \T{imply} \Gamma, \Epsilon \vdash \varphi.
    \end{equation}

    This corresponds to the \hyperref[def:abstract_propositional_sequent_calculus_system/rules/cut]{cut rule} in sequent calculus.

    \thmitem{def:consequence_relation/compactness} Optionally, we may require the following to hold, in which case we call the relation \term{compact}:
    \begin{equation}\label{eq:def:consequence_relation/compactness}
      \Gamma \vdash \varphi \T{implies that there exists a finite subset} \Delta \T{such that} \Delta \vdash \varphi.
    \end{equation}
  \end{thmenum}
\end{definition}
\begin{comments}
  \item Regarded as a binary relation between sentences on a fixed signature, consequence is a \hyperref[def:preordered_set]{preorder}. This motivates Lindenbaum-Tarski algebras --- see \cref{def:lindenbaum_tarski_algebra}.

  \item The conditions are stated so that they hold if \( \Gamma \) is either a set or a list. The former will be used through this chapter, while the latter will be used as a basis for defining type derivation relations in \cref{def:type_derivation_relation}.
\end{comments}

\begin{proposition}\label{thm:consequence_operators_and_relations}
  Fix a set \( S \), a function \( \op*{Cn}: \pow(S) \to \pow(S) \) and a relation \( \vdash \) from \( \pow(S) \) to \( S \). Suppose that
  \begin{equation}\label{eq:thm:consequence_operators_and_relations}
    \varphi \in \op*{Cn}(\Gamma) \T{if and only if} \Gamma \vdash \varphi.
  \end{equation}

  Then \( \op*{Cn} \) is a \hyperref[def:consequence_operator]{consequence operator} if and only if \( {\vdash} \) is a \hyperref[def:consequence_relation]{consequence relation}.

  Furthermore, \( \op*{Cn} \) is compact in the sense of \cref{def:consequence_operator/compactness} if and only if \( \vdash \) is compact in the sense of \cref{def:consequence_relation/compactness}.
\end{proposition}
\begin{proof}
  \SubProof{Proof that operators give rise to relations} Suppose that we are given a consequence operator and that we have defined a relation, whose properties we wish to prove.

  \SubProofOf*[def:consequence_relation/reflexivity]{reflexivity} \( \op*{Cn} \) is idempotent, thus \( \op*{Cn}(\Gamma) = \op*{Cn}(\op*{Cn}(\Gamma)) \) and \( \Gamma \vdash \varphi \) implies \( \op*{Cn}(\Gamma) \vdash \varphi \).

  \SubProofOf*[def:consequence_relation/monotonicity]{monotonicity} \( \op*{Cn} \) preserves order, thus \( \op*{Cn}(\Gamma) \subseteq \op*{Cn}(\Gamma \cup \Delta) \) and \( \Gamma \vdash \varphi \) implies \( \Gamma \cup \Delta \vdash \psi \).

  \SubProofOf*[def:consequence_relation/transitivity]{transitivity} First note that \( \Gamma \vdash \psi \) for every \( \psi \in \Delta \) implies \( \Delta \subseteq \op*{Cn}(\Gamma) \).

  Furthermore, since \( \op*{Cn} \) preserves order, we have \( \op*{Cn}(\Gamma) \subseteq \op*{Cn}(\Gamma \cup \Eta) \), and since it is extensive, we have \( E \subseteq D \cup E \subseteq \op*{Cn}(\Gamma \cup \Eta) \). Combining the two inequalities, we obtain
  \begin{equation}\label{eq:thm:consequence_operators_and_relations/op_to_rel/transitivity/intermediate}
    \op*{Cn}(\Gamma) \cup E \subseteq \op*{Cn}(\Gamma \cup \Eta).
  \end{equation}

  Then
  \begin{equation*}
    \op*{Cn}(\Delta \cup \Epsilon)
    \subseteq
    \op*{Cn}(\op*{Cn}(\Gamma) \cup \Epsilon)
    \reloset {\eqref{eq:thm:consequence_operators_and_relations/op_to_rel/transitivity/intermediate}} \subseteq
    \op*{Cn}(\op*{Cn}(\Gamma \cup \Eta))
    =
    \op*{Cn}(\Gamma \cup \Eta)
  \end{equation*}
  and \eqref{eq:def:consequence_relation/transitivity} follows directly.

  \SubProofOf*[def:consequence_relation/compactness]{compactness} Suppose that \( \op*{Cn} \) satisfies the compactness property \eqref{eq:def:consequence_operator/compactness}.

  Suppose that \( \Gamma \vdash \varphi \). Then \( \varphi \in \op*{Cn}(\Gamma) \), and, by \eqref{eq:def:consequence_operator/compactness}, there exists at least one finite subset \( \Delta \) of \( \Gamma \) such that \( \varphi \in \op*{Cn}(\Delta) \). This implies \( \Delta \vdash \varphi \).

  \SubProof{Proof that relations give rise to operators} Suppose that we are given a consequence relation and that we have defined an operator, whose properties we wish to prove.

  \SubProofOf*[def:extensive_function]{extensiveness} If \( \varphi \in \Gamma \), the reflexivity property \eqref{eq:def:consequence_relation/reflexivity} implies that \( \Gamma \vdash \varphi \), and thus \( \varphi \in \op*{Cn}(\Gamma) \).

  \SubProofOf*[def:idempotent_function]{idempotence} If \( \varphi \in \op*{Cn}(\op*{Cn}(\Gamma)) \), that is, if \( \op*{Cn}(\Gamma) \vdash \varphi \), then the transitivity property \eqref{eq:def:consequence_relation/transitivity} implies that \( \Gamma \vdash \varphi \) and thus \( \varphi \in \op*{Cn}(\Gamma) \).

  \SubProofOf*[def:order_function/preserving]{monotonicity} Suppose that \( \Gamma \subseteq \Delta \). If \( \varphi \in \op*{Cn}(\Gamma) \), that is, if \( \Gamma \vdash \psi \), the monotonicity property \eqref{eq:def:consequence_relation/monotonicity} implies that \( \Delta \vdash \psi \), that is, \( \varphi \in \op*{Cn}(\Delta) \).

  \SubProofOf*[def:consequence_operator/compactness]{compactness} Suppose that \( {\vdash} \) satisfies the compactness property \eqref{eq:def:consequence_relation/compactness}.

  Let \( \varphi \) be a sentence in \( \op*{Cn}(\Gamma) \). Then \( \Gamma \vdash \varphi \), and \eqref{eq:def:consequence_relation/compactness} implies that there exists a finite subset \( \Delta \) of \( \Gamma \) such that \( \Delta \vdash \varphi \). Thus, \( \varphi \in \op*{Cn}(\Delta) \).

  Generalizing on \( \varphi \), we obtain that
  \begin{equation*}
    \op*{Cn}(\Gamma) \subseteq \bigcup\set{ \op*{Cn}(\Delta) \given \Delta \T{is a finite subset of} \Gamma }.
  \end{equation*}

  The converse inclusion follows from the fact that \( \op*{Cn} \) preserves order.
\end{proof}

\begin{remark}\label{rem:logical_compactness_theorems}
  We prove \hyperref[def:consequence_operator/compactness]{consequence operator compactness} in the following:
  \begin{thmenum}
    \thmitem{rem:logical_compactness_theorems/propositional_semantic} \Cref{thm:classical_propositional_semantic_compactness}.
    \thmitem{rem:logical_compactness_theorems/propositional_natural_deduction} \Cref{thm:propositional_natural_deduction_derivation_compact}.
    \thmitem{rem:logical_compactness_theorems/implicational} \Cref{thm:propositional_axiomatic_derivation_entailment_compact}.
    \thmitem{rem:logical_compactness_theorems/sequent_calculus} \Cref{thm:sequent_calculus_derivation_compact}.
    \thmitem{rem:logical_compactness_theorems/first_order_syntactic} \Cref{thm:fol_natural_deduction_derivation_compact}.
  \end{thmenum}
\end{remark}

\begin{definition}\label{def:logical_theory}\mcite[96]{CitkinMuravitsky2022ConsequenceRelations}
  Fix a compatible \hyperref[def:consequence_operator]{consequence operator} \( \op*{Cn} \) and \hyperref[def:consequence_relation]{consequence relation} \( {\vdash} \) on the same set of sentences \( S \). We say that a set of sentences \( \Gamma \) is a \term[ru=теория (\cite[def. 3.1.1]{Герасимов2014Вычислимость})]{theory} with respect to \( \op*{Cn} \) if any of the following equivalent conditions hold:
  \begin{thmenum}[series=def:logical_theory]
    \thmitem{def:logical_theory/operator} \( \Gamma \) is closed, that is, the set \( \op*{Cn}(\Gamma) \) coincides with \( \Gamma \).

    \thmitem{def:logical_theory/relation} \( \varphi \in \Gamma \) if and only if \( \Gamma \vdash \varphi \)
  \end{thmenum}

  We say that \( \Gamma \) \term{generates} or \term{axiomatizes} the theory \( \op*{Cn}(\Gamma) \). This justifies the notation \( \op*{Th}(\Gamma) \) instead of \( \op*{Cn}(\Gamma) \) for the consequence closure of \( \Gamma \).
\end{definition}
\begin{comments}
  \item We will mostly rely on the more structured approach to theories in \hyperref[def:entailment_system]{entailment systems} --- \cref{def:entailment_system_theory} and \cref{def:category_of_theories} --- or in \hyperref[def:institution]{institutions} --- see \cref{def:theory_of_institutional_model}. For \hyperref[def:propositional_logic]{propositional logic} we will refine several related notions in \cref{def:propositional_theory}, while for \hyperref[def:first_order_logic]{first-order logic} we will do so in \cref{def:fol_theory}.

  \item In accordance with \cref{rem:implicit_quantification_and_deduction}, in \hyperref[con:predicate_logic]{predicate logic}, we require theories to consist of closed formulas (i.e. formulas without \hyperref[con:variable_binding]{free variables}).

  \item Woodger, translating Tarski's \cite[70]{Tarski1983MethodologyOfDeductiveSciences} instead calls these theories \enquote{deduction systems}, but we avoid this term to prevent any confusion with \hyperref[def:propositional_natural_deduction]{natural deduction systems}.
\end{comments}

\begin{remark}\label{rem:logical_theory_closed}
  We have defined a logical theory in \cref{def:logical_theory} as a set of sentences closed under consequence. This is due to \tcite{#2, where #1}[96]{CitkinMuravitsky2022ConsequenceRelations}, like us, consider arbitrary consequence operators.

  Similarly, theories are required to be closed under consequence by \incite[def. 1.4.5]{Hinman2005Logic} (for propositional logic), \incite[def. 3.1.4(i)]{VanDalen2004LogicAndStructure} and \incite[21]{Andrews2002Logic}. \incite{Эдельман1975Логика} even distinguishes between syntactic and semantic theories of first-order logic.

  On the other hand, the requirement that a theory is closed is sometimes lifted, for example by \incite[7]{TroelstraSchwichtenberg2000BasicProofTheory}, \incite[def. 3.1.1]{Герасимов2014Вычислимость} and \incite[146]{ШеньВерещагин2017ЯзыкиИИсчисления}.

  In \cite[284]{Meseguer1989GeneralLogics}, when discussing \hyperref[def:entailment_system]{entailment systems}, Meseguer comments on this dichotomy as follows:
  \begin{displayquote}
    Given a signature \( \Sigma \), a theory is presented by a set \( \Gamma \) if \( \Sigma \)-sentences called its axioms. We can therefore define a \textit{theory} as a pair \( T = (\Sigma, \Gamma) \). For some purposes, one deals not with the original axioms \( \Gamma \) but rather with their closure under entailment \( \Gamma^* \), so that it is tempting to call \( T = (\Sigma, \Gamma) \) a \textit{presentation} of the theory \( T = (\Sigma, \Gamma^*) \). However, the view of theories as presentations allows us to make finer distinctions that are important for both proof-theoretic and computational purposes.
  \end{displayquote}

  \tcite{#2 take the opposite approach in #1}[def. 2]{GoguenBurstall1992Institutions} --- they distinguish between closed theories and possibly non-closed presentations of theories.

  We assume that theories are closed and try to be explicit about how a given theory is \hyperref[def:logical_theory/generated]{axiomatized}.
\end{remark}

\begin{definition}\label{def:logical_equivalence}\mimprovised
  Fix a compatible \hyperref[def:consequence_operator]{consequence operator} \( \op*{Cn} \) and \hyperref[def:consequence_relation]{consequence relation} \( {\vdash} \) on the same set of sentences \( S \). We say that two sets of sentences \( \Gamma \) and \( \Delta \) are \term[ru=равносильные / эквивалентные (формулы) (\cite[44]{КолмогоровДрагалин2006Логика}), en=equivalent (sets of sentences) (\cite[def. 2]{Tarski1983FundamentalConceptsOfMetamathematics})]{equivalent} if any of the following tantamount conditions hold:
  \begin{thmenum}[series=def:logical_equivalence]
    \thmitem{def:logical_equivalence/operators}\mcite[def. 2]{Tarski1983FundamentalConceptsOfMetamathematics} The \hyperref[def:logical_theory]{logical theories} \hyperref[def:logical_theory/generated]{axiomatized} by \( \Gamma \) and \( \Delta \) coincide, i.e. \( \op*{Cn}(\Gamma) = \op*{Cn}(\Delta) \).

    \thmitem{def:logical_equivalence/relations} We have if \( \Gamma \vdash \psi \) for every \( \psi \) in \( \Delta \) and \( \Delta \vdash \varphi \) for every \( \varphi \) in \( \Gamma \).
  \end{thmenum}
\end{definition}
\begin{defproof}
  \ImplicationSubProof{def:logical_equivalence/relations}{def:logical_equivalence/operators} Suppose that \( \varphi \vDash_\Sigma \psi \) and \( \psi \vDash_\Sigma \varphi \) hold.

  Then the theory
  \begin{equation*}
    \op*{Cn}(\set{ \varphi }) = \set{ \theta \in S \given \varphi \vDash_\Sigma \theta }
  \end{equation*}
  contains \( \psi \). Since \( \op*{Cn} \) preserves order,
  \begin{equation*}
    \op*{Cn}(\set{ \psi }) \subseteq \op*{Cn}(\op*{Cn}(\set{ \varphi })) = \op*{Cn}(\set{ \varphi }).
  \end{equation*}

  The converse can be shown similarly.

  \ImplicationSubProof{def:logical_equivalence/operators}{def:logical_equivalence/relations} Suppose that \( \op*{Cn}(\set{ \varphi }) = \op*{Cn}(\set{ \psi }) \).

  Since \( \op*{Cn} \) is \hyperref[def:extensive_function]{extensive}, it follows that \( \set{ \varphi } \subseteq \op*{Cn}(\set{ \varphi }) \). Then \( \varphi \in \op*{Cn}(\set{ \psi }) \), that is, \( \psi \vdash \varphi \).

  The converse can be shown similarly.
\end{defproof}

\paragraph{Entailment systems}

\begin{concept}\label{con:logical_system_signature}
  A single \hyperref[con:logical_system]{logical system} is generally designed to encode multiple \hyperref[con:metalogic]{object languages}\fnote{An exception is propositional logic, which we discuss in \fullref{sec:propositional_logic}. It provides only a way to study logical relations between \hyperref[con:proposition]{propositions}, irrespective of how these propositions are structured.}. A particular symbol may or may not be common for all languages of the system. For example, the addition symbol \( {\synplus} \) appears in \hyperref[def:peano_arithmetic]{Peano arithmetic} or the \hyperref[def:semiring/theory]{theory of semirings}, while some \hyperref[con:improper_symbol]{improper symbols} such as the conditional connective \( {\synimplies} \) is part of the \hyperref[def:propositional_alphabet]{propositional alphabet} and may thus appear in all formulas of \hyperref[def:propositional_logic]{propositional logic}, \hyperref[def:first_order_logic]{first-order logic} and \hyperref[def:higher_order_logic]{higher-order logic}.

  We call a \term[ru=сигнатура (логики первого порядка) (\cite[def. 2.1.1]{Герасимов2014Вычислимость}), en=signature (\cite[97]{GoguenBurstall1992Institutions})]{signature} the collection of symbols specific to an object language but not the entire system. Thus, in our example with Peano arithmetic, the signature consists of \( {\synplus} \), \( {\syntimes} \), \( \syn0 \) and \( \syns \).

  Signatures may also specify some properties of their symbols. \hyperref[def:fol_signature]{First-order signatures}, for example, specify the kind and arity of the symbols --- in Peano arithmetic, \( {\synplus} \) and \( {\syntimes} \) are binary operation, \( \syns \) is a unary operation, while \( \syn0 \) is a nullary operation.

  Finally, the (cardinal) number of symbols of a signature is important for several reasons outlined in \cref{rem:language_alphabet_cardinality}. Even when signatures are specified using multiple sets, we use adjectives like \enquote{finite} and \enquote{countably infinite} for the signature itself as per \cref{rem:cardinality_auxiliary_terminology}.
\end{concept}
\begin{comments}
  \item \incite[97]{GoguenBurstall1992Institutions} describe a signature as \enquote{vocabularies for use in constructing sentences in a logical system}.

  \item Some authors use \enquote{language} for what we call \enquote{signature}, for example \incite[def. 2.1.2]{Hinman2005Logic} and \incite[270]{Farmer2008STTVirtues}.

  We avoid this usage because it conflicts with our formalisms from \fullref{ch:formal_language_theory}.
\end{comments}

\begin{definition}\label{def:entailment_system}\mcite[def. 1]{Meseguer1989GeneralLogics}
  An \term{entailment system} consists of the following:
  \begin{thmenum}
    \thmitem{def:entailment_system/signatures} A \hyperref[def:category]{category} \( \cat{Sign} \), whose objects we will call \term{signatures}.

    \thmitem{def:entailment_system/sentences} A \hyperref[def:functor]{functor} \( \op*{Sen}: \cat{Sign} \to \cat{Set} \). We call the elements of the set \( \op*{Sen}(\Sigma) \) \term{sentences}.

    For a signature morphism \( t: \Sigma \to \Theta \), we refer to the both the action of \( t \) and of the induced function \( \op*{Sen}(t) \) as \term{translation}.

    \thmitem{def:entailment_system/entailment} For every signature \( \Sigma \), a \hyperref[def:consequence_relation]{consequence relation} \( \vdash_\Sigma \) on \( \op*{Sen}(\Sigma) \) such that, for every signature translation \( t: \Sigma \to \Theta \), the following compatibility condition holds:
    \begin{equation}\label{eq:def:entailment_system/entailment}
      \Gamma \vdash_\Sigma \varphi \T{implies} \op*{Sen}(t)[\Gamma] \vdash_\Theta \op*{Sen}(t)(\varphi).
    \end{equation}

    We say that \( \Gamma \) \term{entails} \( \varphi \).
  \end{thmenum}
\end{definition}
\begin{comments}
  \item For entailment relations we use the sequent notation discussed in \cref{rem:sequent_notation}.
\end{comments}

\begin{definition}\label{def:interderivability}\mimprovised
  We refer to \hyperref[def:logical_equivalence]{logical equivalence} in \hyperref[def:entailment_system]{entailment systems} as \term[ru=синтактическая эквивалентность (\cite[136]{Герасимов2014Вычислимость})]{interderivability}, especially when the entailment system is the syntactic part of a \hyperref[def:general_logic]{general logic}.
\end{definition}

\begin{definition}\label{def:entailment_system_theory}\mimprovised
  \hyperref[def:entailment_system]{Entailment systems} allow systematizing some aspects of logical theories that could not be captured in \cref{def:logical_theory}.

  First and foremost, the definition only makes sense over a fixed signature \( \Sigma \). A theory is thus a set \( \Gamma \) of sentences over \( \Sigma \) that are closed under \( {\vdash_\Sigma} \), i.e. \( \varphi \) must belong to \( \Gamma \) whenever \( \Gamma \vdash_\Sigma \varphi \).

  The signature  \( \Sigma \) may be ambiguous considering \hyperref[def:fol_signature_extension]{first-order signature extensions} and the like; thus, it is sometimes convenient to make \( \Sigma \) explicit by regarding a theory as a pair \( (\Sigma, \Gamma) \).

  \begin{thmenum}
    \thmitem{def:entailment_system_theory/morphism}\mcite[def. 5]{GoguenBurstall1992Institutions} A signature morphism \( t: \Sigma \to \Theta \) allows us to translate sentences.

    We say that \( t \) is a \term{theory morphism} from \( (\Sigma, \Gamma) \) to \( (\Theta, \Delta) \) if the translation \( \op*{Sen}(t)[\Gamma] \) of the sentences from \( \Gamma \) is a subset of \( \Delta \).

    \thmitem{def:entailment_system_theory/conservative}\mcite[115]{GoguenBurstall1992Institutions} We say that the theory morphism \( t: (\Sigma, \Gamma) \to (\Theta, \Delta) \) is \term{conservative} when \( \Gamma \nvDash_\Sigma \varphi \) implies \( \Delta \nvDash_\Theta \op*{Sen}(t)(\varphi) \).
  \end{thmenum}
\end{definition}
\begin{comments}
  \item We have defined what \cite[def. 2]{Meseguer1989GeneralLogics} calls \enquote{axiom-preserving morphisms}. He also suggests a weaker alternative --- merely requiring \( \Delta \vdash \op*{Sen}(t)(\varphi) \) for every sentence \( \varphi \) in \( \Gamma \).

  \item In accordance with \cref{rem:implicit_quantification_and_deduction}, in \hyperref[con:predicate_logic]{predicate logic}, we require theories to consist of closed formulas (i.e. formulas without \hyperref[con:variable_binding]{free variables}).
\end{comments}

\begin{definition}\label{def:category_of_theories}\mcite[def. 2]{Meseguer1989GeneralLogics}
  Fix an \hyperref[def:entailment_system]{entailment system} \( \mscrE \) and a \hyperref[def:grothendieck_universe]{Grothendieck universe} \( \mscrU \), assumed by default to be the universe \hyperref[def:universe_of_hereditary_finite_sets]{\( V_\omega \)} of hereditarily finite sets.

  We define the \term{\hyperref[def:category]{category} of theories} \( \ucat{Th}(\mscrE) \) as follows:
  \begin{thmenum}
    \thmitem{def:category_of_theories/objects} The objects are pairs \( (\Sigma, \Gamma) \), where \( \Gamma \) is a \( \mscrU \)-small \hyperref[def:entailment_system_theory]{theory} over the \( \mscrU \)-small signature \( \Sigma \).

    \thmitem{def:category_of_theories/morphisms} A morphism \( t: (\Sigma, \Gamma) \to (\Theta, \Delta) \) is a compatible signature translation, defined as in \cref{def:entailment_system_theory/morphism}.

    \thmitem{def:category_of_theories/composition} The composition of theory morphisms is simply the composition of the underlying signature morphisms.

    \thmitem{def:category_of_theories/identity} The theory identity on \( (\Sigma, \Gamma) \) is the underlying identity signature morphism on \( \Sigma \).
  \end{thmenum}
\end{definition}

\paragraph{Institutions}

\begin{concept}\label{con:logical_model}
  A \term[ru=модель (\cite[71]{КолмогоровДрагалин2006Логика}), en=model (\cite[25]{Kleene1971Metamathematics})]{model} of a collection of \hyperref[con:judgment]{judgments} is a \hyperref[con:metalogic]{metatheoretic} object that satisfies them.

  Having somehow encoded a collection of judgments in an object language, we obtain a set of \hyperref[con:proposition]{sentences} --- symbolic strings that offers a very precise formulation of the judgments, but are by themselves devoid of meaning.

  To restore this lost meaning, we must provide a systematic way of interpreting models in the object language, so that a model of some collection of judgments is also a model of their encoding in the object language.

  Institutions provide a general formalism for how models relate to sentences; we define them in \cref{def:institution}.
\end{concept}
\begin{comments}
  \item The correspondence between logical formulas and models is an instance of the syntax-semantics duality described in \cref{con:syntax_semantics_duality}.
\end{comments}

\begin{definition}\label{def:institution}\mcite[def. 1]{GoguenBurstall1992Institutions}
  An \term{institution} consists of the following:
  \begin{thmenum}
    \thmitem{def:institution/signatures} A \hyperref[def:category]{category} \( \cat{Sign} \), whose objects we will call \term{signatures}.

    \thmitem{def:institution/sentences} A \hyperref[def:functor]{functor} \( \op*{Sen}: \cat{Sign} \to \cat{Set} \). We call the elements of the set \( \op*{Sen}(\Sigma) \) \term{sentences}.

    \thmitem{def:institution/models} A \hyperref[def:functor]{functor} \( \cat{Mod}: \cat{Sign} \to \cat{Cat}^{\oppos} \)\fnote{As an elucidation for why this functor is required to be contravariant, see \fullref{thm:fol_structure_reduct_denotation}.}.
    \begin{thmenum}
      \thmitem{def:institution/models/obj} We call the objects of the category \( \op*{Mod}(\Sigma) \) \term{models} and the morphisms --- \term{homomorphisms}.

      \thmitem{def:institution/models/hom} \mcite[94]{MossakowskiKrumnackMaibaum2015DerivedSignatureMorphisms} For a given signature morphism \( t: \Sigma \to \Theta \), we call \( \cat{Mod}(t): \cat{Mod}(\Theta) \to \cat{Mod}(\Sigma) \) the \term{model reduct} functor.
    \end{thmenum}

    \thmitem{def:institution/satisfaction} For every signature \( \Sigma \), a \hyperref[def:relation]{relation} \( {\vDash_\Sigma} \) between models from \( \op*{Mod}(\Sigma) \) and sentences from \( \op*{Sen}(\Sigma) \) such that, for every signature morphism \( t: \Sigma \to \Theta \), every sentence \( \varphi \) in \( \op*{Sen}(\Sigma) \) and every model \( \mscrY \) in \( \op*{Mod}(\Theta) \), the following condition holds:
    \begin{equation}\label{eq:def:institution/satisfaction}
      \mscrY \vDash_\Theta \underbrace{\op*{Sen}(t)(\varphi)}_{\mathclap{\T*{translation of} \varphi}}
      \T{if and only if}
      \underbrace{\op*{Mod}(t)(\mscrY)}_{\mathclap{\T*{reduct of} \mscrY}} \vDash_\Sigma \varphi
    \end{equation}

    We call \( \vDash_\Sigma \) the \term{satisfaction} relation for \( \Sigma \).
  \end{thmenum}
\end{definition}
\begin{comments}
  \item If needed, we will distinguish between the institutions \( \mscrI \) and \( \mscrJ \) via subscripts like \( \cat{Sign}_\mscrI \) and \( \cat{Sign}_\mscrJ \).
\end{comments}

\begin{definition}\label{def:institutional_satisfaction}\mcite[def. 2]{GoguenBurstall1992Institutions}
  Fix a signature \( \Sigma \) in some \hyperref[def:institution]{institution}. We will consider sentences and models correspond over \( \Sigma \). Our goal is to introduce some general terminology that will be useful in several contexts.

  We say that the model \( \mscrX \) \term{satisfies} the set of sentences \( \Gamma \) if \( \mscrX \vDash_\Sigma \varphi \) for every \( \varphi \) in \( \Gamma \). If \( \mscrX \) satisfies \( \Gamma \), we call it a \term[ru=модель (теории первого порядка) (\cite[def. 2.6.21]{Герасимов2014Вычислимость})]{model} \hi{of}\fnote{The terminological difference between \enquote{model} and \enquote{model of a theory} can be confusing, for which reason we try to avoid it --- see \cref{rem:model_theory_structure_terminology}.} \( \Gamma \).

  An alternative terminology suggests saying that \( \mscrX \) \term[en=(first-order structure) validates (\cite[39]{CitkinMuravitsky2022ConsequenceRelations})]{validates} \( \Gamma \) and that \( \Gamma \) is \term[en=valid (in general model) (\cite[239]{Andrews2002Logic})]{valid} in \( \mscrX \).
\end{definition}

\begin{definition}\label{def:institutional_entailment}\mcite[def. 2(9)]{GoguenBurstall1992Institutions}
  Fix a signature \( \Sigma \) in some \hyperref[def:institution]{institution}. We will consider sentences and models correspond over \( \Sigma \).

  We say that the set of sentences \( \Gamma \) \term{semantically entails} the sentence \( \varphi \) and write \( \Gamma \vDash_\Sigma \varphi \) if, whenever a model \( \mscrX \) satisfies \( \Gamma \), it also satisfies \( \varphi \).
\end{definition}
\begin{comments}
  \item Note that we now have two relations denoted by \( \vDash_\Sigma \) --- one is the satisfaction relation from \cref{def:institutional_satisfaction}, the other one is the entailment relation from this definition. Since the relations are fundamentally different, this does not cause ambiguity.

  \item Similarly to other entailment relations, we use the sequent notation discussed in \cref{rem:sequent_notation}.
\end{comments}

\begin{definition}\label{def:semantic_equivalence}\mimprovised
  In an \hyperref[def:institution]{institution}, we say that the sets of sentences \( \Gamma \) and \( \Delta \) are semantically equivalent if any of the following equivalent conditions hold:
  \begin{thmenum}
    \thmitem{def:semantic_equivalence/satisfaction} For any model \( \mscrX \), we have \( \mscrX \vDash \Gamma \) if and only if \( \mscrX \vDash \Delta \).

    \thmitem{def:semantic_equivalence/entailment} The sets are logically equivalent in the sense of \cref{def:logical_equivalence}; that is, \( \Gamma \vDash \psi \) for every \( \psi \) in \( \Delta \) and vice versa.
  \end{thmenum}
\end{definition}

\begin{proposition}\label{thm:derivability_relations_are_consequence}
  The \hyperref[def:institutional_entailment]{semantic entailment relations} in an \hyperref[def:institution]{institution} are \hyperref[def:consequence_relation]{consequence relations}.
\end{proposition}
\begin{proof}
  Fix a signature \( \Sigma \) and suppose that all sentences and models correspond to \( \Sigma \).

  \SubProofOf[def:consequence_relation/reflexivity]{reflexivity} Let \( \varphi \in \Gamma \). If some model \( \mscrX \) satisfies \( \Gamma \), by definition it satisfies every sentence in \( \Gamma \), including \( \varphi \). Therefore, \( \Gamma \) entails \( \varphi \).

  \SubProofOf[def:consequence_relation/monotonicity]{monotonicity} Suppose that \( \Gamma \) entails \( \varphi \) and that \( \mscrX \) satisfies \( \Gamma \cup \Delta \). Then \( \mscrX \) satisfies every sentence in \( \Gamma \), and, since \( \Gamma \) entails \( \varphi \), \( \mscrX \) satisfies \( \varphi \). Generalizing on \( \mscrX \), we conclude that \( \Gamma \cup \Delta \) entails \( \varphi \).

  \SubProofOf[def:consequence_relation/transitivity]{transitivity} Suppose that \( \Gamma \vdash \psi \) for every \( \psi \in \Delta \). Fix a model \( \mscrX \) that satisfies \( \Gamma \cup \Epsilon \). By our first assumption, \( \mscrX \) also satisfies \( \Delta \).

  If \( \Delta \cup \Epsilon \) entails \( \varphi \), then \( \mscrX \) satisfies \( \varphi \) because it satisfies both \( \Delta \) and \( \Epsilon \).

  Generalizing on \( \mscrX \), we conclude that, if \( \Delta \cup \Epsilon \) entails \( \varphi \), then \( \Gamma \cup \Epsilon \) also does.
\end{proof}

\begin{corollary}\label{thm:institution_is_entailment_system}
  Every \hyperref[def:institution]{institution} becomes an \hyperref[def:entailment_system]{entailment system} via its \hyperref[def:institutional_entailment]{semantic entailment relations}.
\end{corollary}
\begin{comments}
  \item Although every institution is an entailment system, we will, following \cite{Meseguer1989GeneralLogics}, use entailment systems for syntactic entailment that complement institutions.
\end{comments}
\begin{proof}
  We can easily generalize \cref{thm:derivability_relations_are_consequence} on signatures, but we must also show that \eqref{eq:def:entailment_system/entailment} holds.

  Fix a signature translation \( t: \Sigma \to \Theta \) and suppose that \( \Gamma \vDash_\Sigma \varphi \).

  Suppose that some model \( \mscrT \) in \( \op*{Mod}(\Theta) \) satisfies \( t[\Gamma] \). Due to \eqref{eq:def:institution/satisfaction}, \( \op*{Mod}(t)(\mscrT) \) is a model in \( \op*{Mod}(\Sigma) \) that satisfies \( \Gamma \), and hence also \( \varphi \). Again via \eqref{eq:def:institution/satisfaction} we conclude that \( \mscrT \) satisfies \( t(\varphi) \).

  This demonstrates \eqref{eq:def:entailment_system/entailment}.
\end{proof}

\begin{corollary}\label{thm:institutional_satisfaction_transitivity}
  In an \hyperref[def:institution]{institution}, \( \mscrX \vDash \Gamma \) and \( \Gamma \vDash \psi \), then \( \mscrX \vDash \psi \).
\end{corollary}
\begin{comments}
  \item This is simply a clearly stated connection between \hyperref[def:institutional_satisfaction]{institutional satisfaction} and \hyperref[def:institutional_entailment]{institutional entailment}.
\end{comments}
\begin{proof}
  If \( \Gamma \vDash \psi \), then by definition \( \mscrX \vDash \psi \) whenever \( \mscrX \vDash \Gamma \) for every model \( \mscrX \).
\end{proof}

\begin{corollary}\label{thm:institutional_satisfaction_closure}
  In an \hyperref[def:institution]{institution}, if \( \mscrX \vDash \Gamma \), then \( \mscrX \vDash \op*{Th}(\Gamma) \).
\end{corollary}
\begin{proof}
  Follows by applying \cref{thm:institutional_satisfaction_transitivity} to every formula in \( \Gamma \).
\end{proof}

\begin{definition}\label{def:theory_of_institutional_model}\mimprovised
  In an \hyperref[def:institution]{institution}, we denote by \( \op*{Th}(\mscrX) \) the set of all sentences \hyperref[def:propositional_semantics/satisfaction]{valid} in \( \mscrX \). We call it the \term{theory} \hi{of} \( \mscrX \).
\end{definition}
\begin{defproof}
  We must justify the name by showing that \( \op*{Th}(\mscrX) \) is closed under entailment.

  Let \( \Sigma \) be the underlying signature. Suppose that \( \op*{Th}(\mscrX) \vDash_\Sigma \varphi \), i.e. every model \( \mscrY \) that satisfies \( \op*{Th}(\mscrX) \) must also satisfy \( \varphi \). One such model is \( \mscrX \) itself. So, by definition of \( \op*{Th}(\mscrX) \) must contain \( \varphi \).
\end{defproof}

\begin{definition}\label{def:tautology}\mimprovised
  In an \hyperref[def:institution]{institution}, we say that the sentence is a \term[en=tautology (\cite[12]{Kleene2002Logic})]{tautology} if it is \hyperref[def:propositional_semantics/satisfaction]{satisfied} in every model (over the same signature). Dually, we call a sentence a \term{contradiction} if no model satisfies it.
\end{definition}
\begin{comments}
  \item In \cite[419]{Tarski1983LogicalConsequence}, an English translation of Tarski, \enquote{tautology} is described as \enquote{a statement which \enquote*{says nothing about reality}}.
\end{comments}

\begin{definition}\label{def:equisatisfiability}\mimprovised
  In an \hyperref[def:institution]{institution}, we say that two sets \( \Gamma \) and \( \Delta \) of sentences over a common signature are \term[en=equisatisfiable (\cite[237]{Mimram2020ProgramEqualsProof})]{equisatisfiable} if \( \Gamma \) and \( \Delta \) either both have (possibly different) models or if both do not.
\end{definition}

\begin{proposition}\label{thm:equisatisfiability_is_equivalence_relation}
  \hyperref[def:equisatisfiability]{Equisatisfiability} is an \hyperref[def:equivalence_relation]{equivalence relation} on the set of all sentences of a fixed signature.
\end{proposition}
\begin{proof}
  \hyperref[def:binary_relation/reflexive]{Reflexivity} and \SubProofOf[def:binary_relation/symmetric]{symmetry} are obvious.

  It remains to show \hyperref[def:binary_relation/transitive]{transitivity}. If \( \Gamma \) and \( \Delta \) are equisatisfiable, and so are \( \Delta \) and \( \Epsilon \), we have two cases:
  \begin{itemize}
    \item If \( \Gamma \) and \( \Delta \) both have a model, then \( \Epsilon \) must also have one.
    \item If \( \Gamma \) and \( \Delta \) both do not have a model, then \( \Epsilon \) cannot have one.
  \end{itemize}

  Thus, either \( \Gamma \) and \( \Epsilon \) both have a model or both do not.
\end{proof}

\paragraph{General logics}

\begin{definition}\label{def:general_logic}\mimprovised
  A \term{general logic} consists of an \hyperref[def:institution]{institution} and an \hyperref[def:entailment_system]{entailment system} over a common category \( \cat{Sign} \) of signatures. We will call the institutional entailment relation \term{(semantic) entailment} and denote it by \( \Gamma \vDash \varphi \), and we will call the relation in the dedicated entailment system \term{(syntactic) derivability} and denote it by \( \Gamma \vdash \varphi \). If a formula is derivable from the empty set, we simply say that it is \term{derivable} or \term{provable}.

  We require derivability to be \term[ru=корректное (исличление) (\cite[48]{Герасимов2014Вычислимость}), en=sound (\cite[def. 6]{Meseguer1989GeneralLogics})]{sound} with respect to entailment, meaning that, whenever \( \Gamma \vdash \varphi \), we have \( \Gamma \vDash \varphi \)

  \begin{thmenum}
    \thmitem{def:general_logic/completeness}\mcite[def. 6]{Meseguer1989GeneralLogics} We say that derivability is \term[ru=полнота (исличления) (\cite[57]{Герасимов2014Вычислимость})]{complete} with respect to entailment if, whenever \( \Gamma \vDash \varphi \), we have \( \Gamma \vdash \varphi \).
  \end{thmenum}
\end{definition}
\begin{comments}
  \item Our definition is based in the notion of \enquote{logic} from \cite[def. 6]{Meseguer1989GeneralLogics}, but with different terminology. For example, since the paper is called \enquote{General Logics}, we prefer the term \enquote{general logic} to \enquote{logic}. The same paper introduces \enquote{proof calculi} and \enquote{logical systems}, however these definitions are unnecessarily technical for our purposes.

  The term \enquote{logical framework} is a fine candidate, but is unfortunately also ambiguous --- according to \incite{Pfenning2002LogicalFrameworks},
  \begin{displayquote}
    A logical framework is a meta-language for the specification of deductive systems.
  \end{displayquote}

  Even though we introduce such a metalanguage in \fullref{sec:propositional_natural_deduction}, we will refrain from using the phrase \enquote{logical framework} as it is fundamentally distinct from our notion of general logic that unifies syntax and semantics.

  \item We will later describe several general logics --- \hyperref[con:intuitionistic_logic]{intuitionistic} and \hyperref[con:classical_logic]{classical logic} differ in their behavior, while \hyperref[def:propositional_logic]{propositional} and \hyperref[rem:predicate_logic]{predicate logic} differ in what they can describe. Outside this chapter, we will almost exclusively use classical first-order logic.
\end{comments}

\begin{proposition}\label{thm:completeness_implies_compactness}
  In a \hyperref[def:general_logic/completeness]{complete general logic}, if syntactic derivability is \hyperref[def:consequence_relation/compactness]{compact}, so is semantic entailment.
\end{proposition}
\begin{proof}
  Suppose that \( \Gamma \vDash \varphi \). By completeness, we have \( \Gamma \vdash \varphi \). Since the derivability is compact, there exists a finite subset \( \Gamma_0 \) of \( \Gamma \) such that \( \Gamma_0 \vdash \varphi \). By soundness, then, we have \( \Gamma_0 \vDash \varphi \).
\end{proof}

\begin{remark}\label{rem:soundness_and_completeness_theorem_list}
  The following is a list of different proofs for \hyperref[def:general_logic]{soundness} and \hyperref[def:general_logic/completeness]{completeness}:
  \begin{itemize}
    \item \Fullref{thm:minimal_implicational_logic_soundness}.
    \item \Fullref{thm:propositional_natural_deduction_soundness}.
    \item \Fullref{thm:classical_propositional_sequent_calculus_soundness}.
    \item \Fullref{thm:intuitionistic_propositional_completeness}.
    \item \Fullref{thm:classical_propositional_completeness}.
    \item \Fullref{thm:fol_natural_deduction_soundness}.
  \end{itemize}
\end{remark}

\begin{definition}\label{def:general_logic_theory}\mimprovised
  \hyperref[def:general_logic]{General logics} allow us to further refine \hyperref[def:logical_theory]{logical theories}, already refined for entailment systems in \cref{def:entailment_system_theory}.

  For a fixed signature \( \Sigma \), a \hyperref[def:general_logic]{general logic} provides two related entailment systems; thus, a set \( \Gamma \) of sentences over \( \Sigma \) can be a \term{syntactic theory} if it is closed under \( {\vdash_\Sigma} \) or a \term{semantic theory} if it is closed under \( {\vDash_\Sigma} \).
\end{definition}
\begin{comments}
  \item In complete general logics like \hyperref[def:propositional_logic]{propositional logic} or \hyperref[def:first_order_logic]{first-order logic}, syntactic and semantic theories coincide. We may nonetheless find it important which of the two notions we consider.
\end{comments}

\begin{concept}\label{con:classical_logic}\mimprovised
  By \term[ru=классическая логика (\cite[58]{ШеньВерещагин2017ЯзыкиИИсчисления}), en=classical logic (\cite[8]{TroelstraSchwichtenberg2000BasicProofTheory})]{classical logic} we will mean either the \hyperref[def:general_logic]{general logic} defined in \cref{def:propositional_logic} for \hyperref[def:propositional_formula]{propositional formulas}, with \hyperref[def:truth_value_algebra/classical]{Boolean} \hyperref[def:propositional_institution]{propositional institution} and the corresponding \hyperref[def:propositional_natural_deduction]{classical natural deduction system}, or the general logic defined in \cref{def:first_order_logic} for \hyperref[def:fol_formula]{first-order formulas}, with the \hyperref[def:fol_institution]{first-order institution} and the corresponding \hyperref[def:fol_natural_deduction]{natural deduction system}.
\end{concept}
\begin{comments}
  \item Classical logic is characterized by the ability to use \fullref{thm:propositional_semantic_dne}. A more popular (but less accurate due to \cref{thm:minimal_propositional_negation_laws}) characterization is that \fullref{thm:propositional_semantic_lem} holds.

  \item Within the metatheory, the law of the excluded middle states that either a \hyperref[con:judgment]{judgment} or its negation holds.
\end{comments}

\begin{concept}\label{con:intuitionistic_logic}\mimprovised
  By \term[ru=интуиционисткая логика (\cite[58]{ШеньВерещагин2017ЯзыкиИИсчисления}), en=intuitionistic logic (\cite[8]{TroelstraSchwichtenberg2000BasicProofTheory})]{intuitionistic logic} we will mean the generalization of \hyperref[con:classical_logic]{classical logic} where instead of \fullref{thm:propositional_semantic_lem}, we have the strictly weaker \fullref{thm:propositional_semantic_efq} stating that everything can be proven from a contradiction.

  We will define a \hyperref[def:general_logic]{general logic} for \hyperref[def:propositional_formula]{propositional formulas} in \cref{def:propositional_logic}, with the \hyperref[def:truth_value_algebra/intuitionistic]{intuitionistic} \hyperref[def:propositional_institution]{propositional institution} and the corresponding \hyperref[def:propositional_natural_deduction]{natural deduction system}. In \cref{def:fol_natural_deduction}, we will extend intuitionistic natural deduction to \hyperref[def:fol_formula]{first-order formulas}.
\end{concept}
\begin{comments}
  \item Intuitionistic logic can also be called \enquote{constructive logic} due to the \hyperref[con:brouwer_heyting_kolmogorov_interpretation]{Brouwer-Heyting-Kolmogorov interpretation}.
\end{comments}

\begin{concept}\label{con:minimal_logic}\mimprovised
  By (Johansson's) \term[en=minimal logic (\cite[1]{VanDerMolen2016MinimalLogic}]{minimal logic} we will mean the generalization of \hyperref[con:intuitionistic_logic]{intuitionistic logic} where we reject both \fullref{thm:propositional_semantic_lem} and the strictly weaker \fullref{thm:propositional_semantic_efq}. More technically, we ignore the special role of the \hyperref[def:propositional_alphabet/constants/falsum]{falsum} \( \synbot \) in hope to avoid the \enquote{paradoxes of \hyperref[rem:minimal_semantics_and_bottom]{material implication}}.

  We will only sketch semantics for \hyperref[def:propositional_formula]{propositional formulas} in \cref{rem:minimal_logic_semantics}, but will present a full-fledged natural deduction system in \cref{def:propositional_natural_deduction}, and extend it to \hyperref[def:fol_formula]{first-order formulas} in \cref{def:fol_natural_deduction}.
\end{concept}
