\section{Untyped lambda terms}\label{sec:untyped_lambda_terms}

\paragraph{Untyped \( \synlambda \)-terms}

\begin{definition}\label{def:lambda_term}\mimprovised
  Fix a (finite) \hyperref[def:formal_language/alphabet]{alphabet} \( \op*{Const} \), whose elements we will call \term[en=constant (\cite[202]{Andrews2002STT})]{constant terms}.

  We will introduce a \hyperref[def:formal_grammar/schema]{grammar schema} whose rules and generated languages we will collectively call the \enquote{\hyperref[con:syntax_semantics_duality]{syntax} of untyped lambda calculus}:
  \begin{bnf*}
    \bnfprod{variable}    {\bnfpn{Small Latin identifier}}, \\
    \bnfprod{atom}        {\bnfpn{variable} \bnfor \bnfpn{constant}}, \\
    \bnfprod{application} {\bnftsq{\( ( \)} \bnfsp \bnfpn{term} \bnfsp \bnfpn{term} \bnfsp \bnftsq{\( ) \)}}, \\
    \bnfprod{abstraction} {\bnftsq{\( ( \)} \bnfsp \bnftsq{\( \synlambda \)} \bnfsp \bnfpn{variable} \bnfsp \bnftsq{.} \bnfsp \bnfpn{term} \bnfsp \bnftsq{\( ) \)}}, \\
    \bnfprod{term}        {\bnfpn{atom} \bnfor \bnfpn{application} \bnfor \bnfpn{abstraction}}.
  \end{bnf*}

  We have used variable identifier rules from \fullref{def:variable_identifier}.

  To distinguish constants from the other symbols used in the grammar schema, we will call the latter \term[en=logical symbol (\cite[2]{Hinman2005Logic})]{logical symbols}. We require the logical and nonlogical symbols to be distinct.

  \begin{thmenum}
    \thmitem{def:lambda_term/const} Technically, the definition of \( \synlambda \)-term depends on the set of constants. Constants are an important part of type systems, where they are treated as part of the type system. In general untyped lambda calculus, we will not need them.

    Thus, unless explicitly noted otherwise, we will not consider constant terms, i.e. we will suppose that \( \op*{Const} \) is empty. Still, we want to explicitly handle constants in our definitions and proofs, so that they hold without modification in typed lambda calculus.

    If disambiguation is needed, we will refer to \( \synlambda \)-terms with no constants as \enquote{pure}.

    \thmitem{def:lambda_term/var} We will denote by \( \op*{Var} \) the set of all variables.

    \thmitem{def:lambda_term/atom} We will denote by \( \op*{Atom} \) the set of all atomic formulas, i.e.
    \begin{equation*}
      \op*{Atom} \coloneqq \op*{Const} \cup \op*{Var}.
    \end{equation*}

    \thmitem{def:lambda_term/term} Similarly, we will denote by \( \op*{Term} \) the set of all \( \synlambda \)-terms.
  \end{thmenum}
\end{definition}
\begin{comments}
  \item Except for constant terms, these notions mostly coincide with their original definition in \cite[352]{Church1932Untyped}.

  Later, in \cite[56]{Church1940STT}, Church extends the syntax with constant terms and type annotations for what we now call \enquote{simply typed lambda calculus}, discussed in \fullref{sec:simply_typed_lambda_terms}.

  The terminology itself is based on \cite[def. 1A1]{Hindley1997STT}, with the necessary changes made in order to accommodate constants.

  \item When \hyperref[def:typed_lambda_term]{typed \( \synlambda \)-term} are involved, we will refer to the ones defines here as \enquote{untyped}.

  \item We need all the parentheses so that we can prove unambiguity in \fullref{thm:lambda_term_grammar/unambiguous}. In the metalanguage, we will use the conventions from \fullref{rem:lambda_term_parentheses} regarding parentheses.

  \item The dot is not strictly necessary, and was in fact not used in \incite{Church1932Untyped} where \( \synlambda \)-calculus was introduced. It later became standard to include it. Our main reference books --- \incite[def. 1A1]{Hindley1997STT}, \incite[22]{Barendregt1984LambdaCalculus}, \incite[6]{BarendregtDekkersStatman2013Types}, \incite[\S 4.1.3]{Mimram2020Types} and \incite[188]{Герасимов2011Вычислимость} --- all use it.
\end{comments}

\begin{proposition}\label{thm:lambda_term_grammar}
  The \hyperref[def:formal_grammar]{grammar} of \hyperref[def:lambda_term]{\( \synlambda \)-terms} has the following basic properties:
  \begin{thmenum}
    \thmitem{thm:lambda_term_grammar/unambiguous} It is \hyperref[def:grammar_ambiguity]{unambiguous}.

    \thmitem{thm:lambda_term_grammar/balanced} The \hyperref[def:formal_grammar/language]{generated language} has \hyperref[def:paired_delimiters]{balanced parentheses}.

    \thmitem{thm:lambda_term_grammar/not_regular} The grammar is not \hyperref[def:chomsky_hierarchy/regular]{regular}.
  \end{thmenum}
\end{proposition}
\begin{proof}
  Similar to \fullref{thm:propositional_formula_grammar}.
\end{proof}

\begin{definition}\label{def:lambda_term_ast}
  We implicitly associate with each \( \synlambda \)-term \( M \) an \hyperref[con:abstract_syntax_tree]{abstract syntax tree} \( T(M) \) as follows:
  \begin{thmenum}
    \thmitem{def:lambda_term_ast/variable} If \( M \) is a variable or constant, we define \( T(M) \) as the \hyperref[def:canonical_singleton_tree]{canonical singleton tree} with label \( M \).

    \thmitem{def:lambda_term_ast/application} If \( M = NK \), assuming we have already built \( T(N) \) and \( T(K) \), we define \( T(M) \) by \hyperref[def:ordered_tree_grafting_product]{grafting} \( T(N) \) and \( T(K) \) to a new root labeled with \( \cdot \):
    \begin{equation*}
      \includegraphics[page=1]{output/def__lambda_term_ast}
    \end{equation*}

    \thmitem{def:lambda_term_ast/abstraction} If \( \varphi = \qabs x N \), assuming we have already built \( T(N) \), we define \( T(M) \) by \hyperref[def:ordered_tree_grafting_product]{grafting} \( T(N) \) to a new root labeled with \( \qabs* x \):
    \begin{equation*}
      \includegraphics[page=2]{output/def__lambda_term_ast}
    \end{equation*}
  \end{thmenum}
\end{definition}
\begin{comments}
  \item Formally our ASTs for abstractions are not ideal because we need to \enquote{parse} the string at the root. They are, however, convenient for defining variable occurrences in \fullref{def:lambda_variable_occurrence}, and are also easier to parse visually.
\end{comments}

\begin{remark}\label{rem:lambda_term_parentheses}
  We will use some \enquote{abuse-of-notation} \hyperref[con:metalogic]{metalingual} syntactic conventions somewhat resembling those from \fullref{rem:propositional_formula_parentheses}:
  \begin{thmenum}
    \thmitem{rem:lambda_term_parentheses/outermost} As in \fullref{rem:propositional_formula_parentheses/outermost}, we avoid writing the outermost parentheses in terms like \( (\synx \syny) \) or \( ((\synx \syny) \synz) \).

    \thmitem{rem:lambda_term_parentheses/abstraction} We generally avoid writing parentheses around \( \synlambda \)-abstractions.

    Actually, it is not necessary to put parentheses around \( \synlambda \)-abstractions in order for the grammar to be unambiguous. Not requiring them, however, leads to the following unintuitive artifact.

    Consider the term \( \qabs \synx \syny \synz \). If parentheses are required only for \( \synlambda \)-abstractions but not for \( \synlambda \)-applications, it would unambiguously correspond to the following \hyperref[def:lambda_term_ast]{abstract syntax tree}:
    \begin{equation*}
      \includegraphics[page=1]{output/rem__lambda_term_parentheses}
    \end{equation*}

    But it is reasonable to expect instead the following tree:
    \begin{equation*}
      \includegraphics[page=2]{output/rem__lambda_term_parentheses}
    \end{equation*}

    Since we will generally expect the latter, it is simpler to require parentheses around \( \synlambda \)-abstractions in the formal syntax and then, when convenient, avoid writing them within the metalogic.

    \thmitem{rem:lambda_term_parentheses/left_associative} We suppose that \( \synlambda \)-application is \hyperref[rem:binary_operation_syntax_trees/associativity]{left-associative}, which allows us to simplify notation in some cases like \eqref{eq:ex:def:lambda_term/combinator/s}, where we use \( \synx \synz (\syny \synz) \) rather than \( (\synx \synz)(\syny \synz) \).

    This convention is used by Church himself in \incite[58]{Church1940STT}, as well as our main sources --- \incite[1]{Hindley1997STT}, \incite[notation 2.1.3]{Barendregt1984LambdaCalculus} and \incite[112]{Mimram2020Types}.
  \end{thmenum}
\end{remark}

\begin{definition}\label{def:lambda_subterm}\mcite[def. 1A3]{Hindley1997STT}
  We define the set of all \term[ru=подтерм (\cite[189]{Герасимов2011Вычислимость})]{subterms} of a term \( M \) as follows:
  \begin{equation*}
    \op*{Subterm}(M) \coloneqq \begin{cases}
      \set{ M },                                             &M \in \op*{Atom}, \\
      \set{ M } \cup \op*{Subterm}(N) \cup \op*{Subterm}(K), &M = N K, \\
      \set{ M } \cup \op*{Subterm}(N),                       &M = \qabs x N.
    \end{cases}
  \end{equation*}
\end{definition}
\begin{comments}
  \item Note that the analog of \fullref{thm:propositional_formula_characterization} no longer holds --- a variable that is a substring of a term is not necessarily a subterm --- see \fullref{ex:def:lambda_term/naive_subterm}.
\end{comments}

\begin{proposition}\label{thm:lambda_subterm_characterization}
  Suppose that the \hyperref[def:formal_language/substring]{substring} \( N \) of the \( \synlambda \)-term \( M \) is \hi{not a variable}. Then \( N \) is a \hyperref[def:lambda_subterm]{subterm} of \( M \) if and only if \( N \) is itself a \( \synlambda \)-term.
\end{proposition}
\begin{proof}
  We can give a proof similar to \fullref{thm:propositional_formula_characterization}.
\end{proof}

\begin{proposition}\label{thm:lambda_term_ast_subterm}
  The \( \synlambda \)-term \( N \) is a \hyperref[def:lambda_subterm]{subterm} of \( M \) if and only if the \hyperref[def:lambda_term_ast]{abstract syntax tree} \( T(M) \) has a \hyperref[def:tree/subtree]{subtree} \hyperref[def:labeled_tree/homomorphism]{isomorphic} to \( T(N) \).
\end{proposition}
\begin{proof}
  Trivial.
\end{proof}

\begin{example}\label{ex:def:lambda_term}
  We list examples of \hyperref[def:lambda_term]{\( \synlambda \)-terms}:
  \begin{thmenum}
    \thmitem{ex:def:lambda_term/var} The simplest \( \synlambda \)-terms are the constants and variables.

    \thmitem{ex:def:lambda_term/naive_subterm} Suppose that, as in \fullref{def:propositional_subformula} for propositional subformulas, we say that \( N \) is a subterm of \( M \) if it is itself a \hyperref[def:formal_language/substring]{substring} of \( M \).

    Then the term \( \qabs \synx \syny \) would have both \( \synx \) and \( \syny \) as subterms, while according to \fullref{def:lambda_subterm}, only \( \syny \) is a subterm.

    \thmitem{ex:def:lambda_term/combinator}\mcite[def. 1A10.1]{Hindley1997STT} The following terms have established names:
    \begin{align*}
      \ref{eq:ex:def:lambda_term/combinator/i} &\coloneqq \qabs \synx \synx,                                                           \taglabel[I]{eq:ex:def:lambda_term/combinator/i} \\
      \ref{eq:ex:def:lambda_term/combinator/k} &\coloneqq \qabs \synx \qabs \syny \synx,                                               \taglabel[K]{eq:ex:def:lambda_term/combinator/k} \\
      \ref{eq:ex:def:lambda_term/combinator/s} &\coloneqq \qabs \synx \qabs \syny \qabs \synz \synx \synz (\syny \synz),               \taglabel[S]{eq:ex:def:lambda_term/combinator/s} \\
      \ref{eq:ex:def:lambda_term/combinator/y} &\coloneqq \qabs \synx (\qabs \syny \synx \syny \syny) (\qabs \syny \synx \syny \syny). \taglabel[Y]{eq:ex:def:lambda_term/combinator/y}
    \end{align*}

    They are combinators in the sense that will be defined in \fullref{def:lambda_combinator}, and are referred to as such. We will further discuss them in \fullref{ex:def:beta_eta_reduction}.

    Another useful family of combinators is the following, taken from \cite[\S 3.1.21; \S 6.2.1]{Barendregt1984LambdaCalculus}:
    \begin{align*}
      \ref{eq:ex:def:lambda_term/combinator/omega_n}   &\coloneqq \qabs \synx \synx^n,     \taglabel[\ensuremath{ \omega_n }]{eq:ex:def:lambda_term/combinator/omega_n} \\
      \ref{eq:ex:def:lambda_term/combinator/omega}     &\coloneqq \qabs \synx \synx \synx, \taglabel[\ensuremath{ \omega }]{eq:ex:def:lambda_term/combinator/omega} \\
      \ref{eq:ex:def:lambda_term/combinator/big_omega} &\coloneqq \omega \omega.           \taglabel[\ensuremath{ \Omega }]{eq:ex:def:lambda_term/combinator/big_omega}
    \end{align*}
  \end{thmenum}
\end{example}

\paragraph{Variable scope}

\begin{definition}\label{def:lambda_abstractor}\mcite[def. 1A5]{Hindley1997STT}
  Given an abstraction \( M = \qabs x N \), we call \( \qabs* x \) the \term{abstractor} and \( N \) --- the \term{body} of \( M \).

  We say that the body \( N \) is the \term[ru=область действия (\cite[64]{Герасимов2011Вычислимость})]{scope} of the abstractor \( \quantifier* \synlambda \synx \) and that the abstractor \term{binds} the variable \( \synx \) in \( N \).
\end{definition}

\begin{definition}\label{def:lambda_variable_occurrence}\mimprovised
  An \term[ru=вхождение (\cite[64]{Герасимов2011Вычислимость}), en=occurrence (\cite[9A2]{Hindley1997STT})]{occurrence} of a variable \( x \) in a \( \synlambda \)-term \( M \) is a variable node with label \( x \) in the \hyperref[def:lambda_term_ast]{abstract syntax tree} of \( M \).

  Variable occurrences in \( \synlambda \)-terms can be free and bound. We say that the occurrence is \term[ru=свободное (вхождение) (\cite[64]{Герасимов2011Вычислимость})]{free} in \( M \) if the root of the AST can be reached without passing through a corresponding abstractor. If a variable is not free, there exists an abstractor that binds it, and we say that the occurrence is \term[ru=связанное (вхождение) (\cite[64]{Герасимов2011Вычислимость})]{bound} in \( M \).
\end{definition}

\begin{definition}\label{def:lambda_variable_freeness}\mcite[16A]{Hindley1997STT}
  We say that a variable \( \synx \) is \term{free} in a \( \synlambda \)-term \( M \) if \( \synx \) has at least one \hyperref[def:lambda_variable_occurrence]{free occurrence} in \( M \) and \term{bound} if it has a \hyperref[def:lambda_variable_occurrence]{bound occurrence} in \( M \).
\end{definition}
\begin{comments}
  \item We may use the recursive definitions from \fullref{thm:lambda_variable_freeness_characterization} instead.
\end{comments}

\begin{example}\label{ex:def:lambda_variable_freeness}
  We list examples of free and bound variables and variables occurrences:
  \begin{thmenum}
    \thmitem{ex:def:lambda_variable_freeness/abstractor} The term \( \ref{eq:ex:def:lambda_term/combinator/i} = \qabs \synx \synx \) has exactly one occurrence of \( \synx \):
    \begin{equation*}
      \includegraphics[page=1]{output/def__lambda_variable_freeness}
    \end{equation*}

     This occurrence is bound in the term \( I \) and free in the subterm \( \synx \). Then \( \synx \) is a bound variable in \( I \) and a free variable in \( I \).

    \thmitem{ex:def:lambda_variable_freeness/both} The term \( M = I \synx = (\qabs \synx \synx) \synx \) has two occurrences of the variable \( \synx \):
    \begin{equation*}
      \includegraphics[page=2]{output/def__lambda_variable_freeness}
    \end{equation*}

    One of the occurrences is free, which makes \( \synx \) a free variable of \( M \), and one of the occurrences if bound, which makes \( \synx \) a bound variable of \( M \).

    \thmitem{ex:def:lambda_variable_freeness/binds} In the term \( \qabs \synx \qabs \synx \synx \), only the second abstractor is binding for \( \synx \).
  \end{thmenum}
\end{example}

\begin{remark}\label{rem:barendregt_convention}\mcite[148]{Mimram2020Types}
  Although a variable can be both free and bound, we will avoid such cases as much as possible. This is called the \term{Barendregt convention} because it is stated in \bycite[\S 2.1.12]{Barendregt1984LambdaCalculus}.
\end{remark}

\begin{proposition}\label{thm:lambda_variable_freeness_characterization}
  The set of all \hyperref[def:lambda_variable_freeness]{free variables} of a \( \synlambda \)-term can be characterized as follows:
  \begin{equation*}
    \op*{Free}(M) \coloneqq \begin{cases}
      \varnothing,                       &M \in \op*{Const}, \\
      \set{ M },                         &M \in \op*{Var}, \\
      \op*{Free}(N) \cup \op*{Free}(K),  &M = N K, \\
      \op*{Free}(N) \setminus \set{ x }, &M = \qabs x N. \\
    \end{cases}
  \end{equation*}

  Similarly, the bound variables can be characterized via
  \begin{equation*}
    \op*{Bound}(M) \coloneqq \begin{cases}
      \varnothing,                        &M \in \op*{Const}, \\
      \varnothing,                        &M \in \op*{Var}, \\
      \op*{Bound}(N) \cup \op*{Bound}(K), &M = N K, \\
      \op*{Bound}(N) \cup \set{ x },      &M = \qabs x N. \\
    \end{cases}
  \end{equation*}
\end{proposition}
\begin{proof}
  Straightforward.
\end{proof}

\begin{definition}\label{def:lambda_combinator}\mcite[def. 1A10]{Hindley1997STT}
  If a \( \synlambda \)-term has no \hyperref[def:lambda_variable_freeness]{free variables}, we say that it is \term{closed}. Closed terms are also called \term[ru=комбинаторы (\cite[188]{Герасимов2011Вычислимость})]{combinators}.
\end{definition}

\paragraph{Simultaneous substitution}

\begin{concept}\label{con:lambda_conversion}
  A fundamental feature of \hyperref[def:lambda_term]{\( \synlambda \)-terms} is \enquote{conversion} --- the ability to transform terms into other terms with simpler syntax and similar semantics. Thus, \enquote{conversion} is synonymous with \enquote{rewriting} in the \hyperref[def:rewriting_system]{rewriting systems} for \( \synlambda \)-terms that we will discuss.

  This usage originates from Alonzo Church himself. When introducing \( \lambda \)-calculus in \cite[357]{Church1932Untyped}, he defines \hyperref[def:lambda_substitution]{single-variable substitution} as the first three of five postulates, and later states
  \begin{displayquote}
    If \( M \) and \( N \) are well-formed and if \( N \) can be derived from \( M \) by successive applications of the rules of procedure \logic{I}, \logic{II}, and \logic{III}, then \( M \) is said to be convertible into \( N \), and the process is spoken of as a conversion of \( M \) into \( N \).
  \end{displayquote}

  Later, different authors introduced \hyperref[def:lambda_term_alpha_equivalence]{\( \alpha \)-equivalence}, \hyperref[def:beta_eta_reduction]{\( \beta \)-reduction} and \hyperref[def:beta_eta_reduction]{\( \eta \)-reduction}. This lead to variations of the word \enquote{conversion} which refer to other rewriting systems and not only substitution. For example, \enquote{\( \alpha \)-conversion} is used by \incite[\S 2.1.11]{Barendregt1984LambdaCalculus}, \incite[def. 1A8]{Hindley1997STT}, \incite[114]{Mimram2020Types} and \incite[5]{Pollack2005AlphaConversion} (with subtle differences as discussed in \fullref{def:lambda_term_alpha_equivalence}).

  When the intention of a conversion procedure is for terms to become shorter, we call the procedure a \enquote{reduction}, and we call the reverse procedure an \enquote{expansion}. If both are possible, we call the procedure an \enquote{equivalence}.
\end{concept}

\begin{definition}\label{def:lambda_substitution}\mimprovised
  A \term{simultaneous substitution} of \( \synlambda \)-terms is a pair \( (\Bbbs, \sharp) \), where
  \begin{thmenum}[series=def:lambda_substitution]
    \thmitem{def:lambda_substitution/sigma} \( \Bbbs: \op*{Var} \to \op*{Term} \) is a function specifying how variables need to be replaced with terms. We allow only finitely many variables to not be fixed by \( \Bbbs \).

    \thmitem{def:lambda_substitution/sharp} \( \sharp: F \to \op*{Var} \), where \( F \) is the family of all variable contexts (i.e. finite subsets of \( \op*{Var} \)), is a function providing us with a fresh variable identifier not present in some given context.

    As discussed in \fullref{def:variable_identifier}, unless otherwise noted, will suppose that this function simply provides the smallest variable identifier with respect to the \hyperref[def:lexicographic_order]{reverse lexicographic order}.
  \end{thmenum}

  We will now use substitutions to define operations on all \( \synlambda \)-terms, but we will first need several auxiliary definitions:
  \begin{thmenum}[resume=def:lambda_substitution]
    \thmitem{def:lambda_substitution/modified} We will find useful the concept of \term{modifying} \( \Bbbs \) at \( x \) with \( y \):
    \begin{equation}\label{eq:def:lambda_substitution/modified}
      \Bbbs_{x \mapsto y}(u) \coloneqq \begin{cases}
        y,         &u = x, \\
        \Bbbs(u), &u \neq x.
      \end{cases}
    \end{equation}

    \thmitem{def:lambda_substitution/free} In order to be able to define substitutions for arbitrary terms, we will also need the following auxiliary definition:
    \begin{equation}\label{eq:def:lambda_substitution/free}
      \op*{Free}_\Bbbs(M) \coloneqq \bigcup_{\mathclap{v \in \op*{Free}(M)}} \op*{Free}(\Bbbs(v)).
    \end{equation}

    \thmitem{def:lambda_substitution/operation} Finally, for an arbitrary \( \synlambda \)-term \( M \), we define the following operation (for which will use the notational convention from \fullref{rem:substitution_notation}):
    \begin{empheq}[left={M[\Bbbs]} \coloneqq \empheqlbrace]{align}
      &M,                              &&M \in \op*{Const},                                    \label{eq:def:lambda_substitution/const}               \\
      &\Bbbs(M),                       &&M \in \op*{Var},                                      \label{eq:def:lambda_substitution/var}                 \\
      &N[\Bbbs] \thinspace K[\Bbbs],   &&M = NK,                                               \label{eq:def:lambda_substitution/application}         \\
      &\qabs x N[\Bbbs_{x \mapsto x}], &&M = \qabs x N \T{and} x \not\in \op*{Free}_\Bbbs(M), \label{eq:def:lambda_substitution/abstraction/direct}  \\
      &\qabs n N[\Bbbs_{x \mapsto n}], &&M = \qabs x N \T{and} x \in \op*{Free}_\Bbbs(M),     \label{eq:def:lambda_substitution/abstraction/renaming}
    \end{empheq}
    where \( n = \sharp(\op*{Free}(N) \cup \op*{Free}_\Bbbs(N)) \).
  \end{thmenum}
\end{definition}
\begin{comments}
  \item In \eqref{eq:def:lambda_substitution/abstraction/direct} we modify \( \Bbbs \) to fix \( x \), while in \eqref{eq:def:lambda_substitution/abstraction/renaming} we modify it to send \( x \) to \( n \). We will refer to the latter rule as \enquote{substitution with renaming}. Whenever it will be sufficient, \fullref{thm:lambda_substitution_single_rule} will allow us to compress them into one rule with a looser condition.

  \item This definition is loosely based on \cite[def. 1A7]{Hindley1997STT}, but is modified to allow substituting multiple variables simultaneously, as well as to handle constants. This results in less rules since \eqref{eq:def:lambda_substitution/abstraction/direct} subsumes three of the four cases. Furthermore, our substitution rules also allow proving \fullref{thm:lambda_substitution_noop}.
\end{comments}

\begin{remark}\label{rem:renaming_substitution_rules}
  The substitution rules for abstractions from \fullref{def:lambda_substitution} are adjusted so that \fullref{thm:lambda_substitution_free_variables} and \fullref{thm:lambda_substitution_noop} hold.

  The bare minimum we need is to avoid \enquote{variable capturing} as discussed in \fullref{ex:def:lambda_substitution/capture}. As we can see in the proof of \fullref{thm:lambda_substitution_free_variables}, the renaming rule \eqref{eq:def:lambda_substitution/abstraction/renaming} is by itself sufficient for this. Thus, we may choose to always rename the abstractor variable.

  In practice, however, we would prefer to not rename bound variables, especially in \hyperref[def:lambda_combinator]{combinators}; the rule \eqref{eq:def:lambda_substitution/abstraction/direct} allow us to prove \fullref{thm:lambda_substitution_noop}.

  See \fullref{ex:def:lambda_substitution/ignoring} for concrete examples of ignoring the latter rule.
\end{remark}

\begin{proposition}\label{thm:lambda_substitution_restriction}
  For any \( \synlambda \)-term \( M \), if the substitutions \( \Bbbs \) and \( \Bbbt \) agree on the free variables on \( M \), then \( M[\Bbbs] = M[\Bbbt] \).
\end{proposition}
\begin{proof}
  Straightforward.
\end{proof}

\begin{proposition}\label{thm:lambda_substitution_noop}
  We have \( M[\Bbbs] = M \) if and only if the free variables of \( M \) are fixed by the substitution \( \Bbbs \).
\end{proposition}
\begin{proof}
  \SufficiencySubProof Suppose that \( M[\Bbbs] = M \). We will use \fullref{thm:induction_on_rooted_trees} on \( M \) to show that \( \Bbbs(x) = x \) whenever \( x \) is free in \( M \):
  \begin{itemize}
    \item If \( M \) is a constant, it has no free variables, so the statement vacuously holds.
    \item If \( M \) is a variable, then \( \Bbbs(M) = M[\Bbbs] = M \), proving the statement.
    \item If \( M = NK \) and if the inductive hypothesis holds for \( N \) and \( K \), then
    \begin{equation*}
      N[\Bbbs] \thinspace K[\Bbbs]
      =
      M[\Bbbs]
      =
      M
      =
      N \thinspace K.
    \end{equation*}

    Thus, \( N[\Bbbs] = N \) and \( K[\Bbbs] = K \). Applying the inductive hypothesis, we conclude that \( \Bbbs \) fixes the free variables of both \( N \) and \( K \), and hence of \( M \).

    \item Suppose that \( M = \qabs x N \) and the inductive hypothesis holds for \( N \).

    \Fullref{thm:lambda_substitution_single_rule} implies that, for some variable \( v \) not free in \( M \) or \( M[\Bbbs] \), we have
    \begin{equation*}
      M[\Bbbs] = \qabs v N[\Bbbs_{x \mapsto v}].
    \end{equation*}

    Since \( M = M[\Bbbs] \), however, \( v \) must be equal to \( x \) and \( N[\Bbbs_{x \mapsto x}] \) to \( N \).

    The inductive hypothesis on \( N \) implies that \( \Bbbs_{x \mapsto x} \) fixes all free variables of \( N \). Then \( \Bbbs \) fixes all free variables of \( M \).
  \end{itemize}

  \NecessitySubProof We will use \fullref{thm:induction_on_rooted_trees} on \( M \) to show that \( M[\Bbbs] = M \) for all \( \Bbbs \) that fix the free variables of \( M \):
  \begin{itemize}
    \item If \( M \) is a constant, then \( M[\Bbbs] = M \) by definition.
    \item If \( M \) is a variable, then \( M[\Bbbs] = \Bbbs(M) = M \) by assumption.
    \item If \( M = NK \) and if the inductive hypothesis holds for \( N \) and \( K \), then
    \begin{equation*}
      M[\Bbbs]
      \reloset {\eqref{eq:def:lambda_substitution/application}} =
      N[\Bbbs] \thinspace K[\Bbbs]
      \reloset {\T{ind.}} =
      N K
      =
      M.
    \end{equation*}

    \item Suppose that \( M = \qabs x N \) and the inductive hypothesis holds for \( N \).

    Aiming at a contradiction, suppose that \( x \) is in \( \op*{Free}_\Bbbs(M) \), that is, there exists a free variable \( y \) of \( M \) such that \( x \) is free in \( \Bbbs(y) \). We have assumed that \( \Bbbs \) fixes the free variables of \( M \), thus \( y = \Bbbs(y) \), and we thus conclude that \( x = y \). In particular, \( x \) is free in \( M \). But, by \fullref{thm:lambda_variable_freeness_characterization}, \( x \) cannot be free in \( M \).

    Therefore, \( x \) cannot be in \( \op*{Free}_\Bbbs(M) \), and we must use the rule \eqref{eq:def:lambda_substitution/abstraction/direct}. The free variables of \( N \) are those of \( N \) with the eventual addition of \( x \), all of which are fixed by \( \Bbbs_{x \mapsto x} \). Hence,
    \begin{equation*}
      M[\Bbbs]
      =
      \qabs x N[\Bbbs_{x \mapsto x}]
      \reloset {\T{ind.}} =
      \qabs x N
      =
      M.
    \end{equation*}
  \end{itemize}
\end{proof}

\begin{corollary}\label{thm:lambda_substitution_combinators}
  For any \hyperref[def:lambda_combinator]{combinator} \( M \) and any substitution \( \Bbbs \), we have \( M[\Bbbs] = M \).
\end{corollary}
\begin{proof}
  Vacuously follows from \fullref{thm:lambda_substitution_noop} since combinators simply have no free variables.
\end{proof}

\begin{proposition}\label{thm:lambda_substitution_free_variables}
  For any \( \synlambda \)-term \( M \) and any substitution \( \Bbbs \), we have
  \begin{equation}\label{eq:thm:lambda_substitution_free_variables}
    \op*{Free}( M[\Bbbs] ) = \overbrace{\bigcup_{\mathclap{v \in \op*{Free}(M)}} \op*{Free}(\Bbbs(v))}^{\op*{Free}_\Bbbs(M)}.
  \end{equation}

  This also holds if we always use the renaming rule \eqref{eq:def:lambda_substitution/abstraction/renaming}.
\end{proposition}
\begin{proof}
  We will use \fullref{thm:induction_on_rooted_trees} on \( M \) simultaneously for all possible substitutions:
  \begin{itemize}
    \item If \( M \) is a constant, then both sides of \eqref{eq:thm:lambda_substitution_free_variables} are the empty set.

    \item If \( M \) is a variable, say \( M = x \), then \( \op*{Free}(M) = \set{ x } \) and thus
    \begin{equation*}
      \op*{Free}(M[\Bbbs])
      \reloset {\eqref{eq:def:lambda_substitution/var}} =
      \op*{Free}(\Bbbs(x)).
    \end{equation*}

    Hence, \eqref{eq:thm:lambda_substitution_free_variables} is satisfied.

    \item Suppose that \( M = NK \), where the inductive hypothesis holds for \( N \) and \( K \).

    We have
    \begin{equation*}
      \op*{Free}(M[\Bbbs])
      \reloset {\eqref{eq:def:lambda_substitution/application}} =
      \op*{Free}(N[\Bbbs] \thinspace K[\Bbbs])
      =
      \op*{Free}(N[\Bbbs]) \cup \op*{Free}(K[\Bbbs]).
    \end{equation*}

    Since, by definition, \( \op*{Free}(M) = \op*{Free}(N) \cup \op*{Free}(K) \), from the inductive hypothesis it follows that \eqref{eq:thm:lambda_substitution_free_variables} holds.

    \item Suppose that \( M = \qabs x N \) and that the inductive hypothesis holds for \( N \). We have the following possibilities:
    \begin{itemize}
      \item Suppose first that \( x \not\in \op*{Free}_\Bbbs(M) \). We must thus use \eqref{eq:def:lambda_substitution/abstraction/direct}. We have
      \begin{align*}
        \op*{Free}(M[\Bbbs])
        &\reloset{\eqref{eq:def:lambda_substitution/abstraction/direct}} =
        \op*{Free}(\qabs x N[\Bbbs_{x \mapsto x}])
        = \\ &=
        \op*{Free}(N[\Bbbs_{x \mapsto x}]) \setminus \set{ x }
        \reloset{\T{ind.}} = \\ &=
        \parens[\Big]{ \bigcup_{\mathclap{v \in \op*{Free}(N)}} \op*{Free}(\Bbbs_{x \mapsto x}(v)) } \setminus \set{ x }
        = \\ &=
        \bigcup_{\mathclap{v \in \op*{Free}(N)}} \parens[\Big]{ \op*{Free}(\Bbbs_{x \mapsto x}(v)) \setminus \set{ x } }
        = \\ &=
        \bigcup_{\mathclap{v \in \op*{Free}(N) \setminus \set{ x }}} \parens[\Big]{ \op*{Free}(\underbrace{\Bbbs_{x \mapsto x}(v)}_{\Bbbs(v)}) \setminus \set{ x } } \cup \underbrace{\parens[\Big]{ \op*{Free}(\Bbbs_{x \mapsto x}(x)) \setminus \set{ x } }}_\varnothing.
        = \\ &=
        \bigcup_{\mathclap{v \in \op*{Free}(N) \setminus \set{ x }}} \parens[\Big]{ \op*{Free}(\Bbbs(v)) },
      \end{align*}
      where at the last step we have used that, for \( v \in \op*{Free}(N) \), \( x \) is free in \( \Bbbs(v) \) only when \( v = x \).

      This demonstrates \eqref{eq:thm:lambda_substitution_free_variables}.

      \item Otherwise, \( x \in \op*{Free}_\Bbbs(M) \), and we must use \eqref{eq:def:lambda_substitution/abstraction/renaming}. As discussed in \fullref{rem:renaming_substitution_rules}, the assumption itself is irrelevant for proving \eqref{eq:thm:lambda_substitution_free_variables}.

      Let \( n \) be the renamed abstractor variable. Then
      \begin{align*}
        \op*{Free}(M[\Bbbs])
        &=
        \op*{Free}(\qabs n N[x \mapsto n][\Bbbs])
        = \\ &=
        \op*{Free}(\qabs n N[\Bbbs_{x \mapsto n}])
        = \\ &=
        \op*{Free}(N[\Bbbs_{x \mapsto n}]) \setminus \set{ n }
        \reloset {\T{ind.}} = \\ &=
        \parens[\Big]{ \bigcup_{\mathclap{v \in \op*{Free}(N)}} \op*{Free}(\Bbbs_{x \mapsto n}(v)) } \setminus \set{ n }
        = \\ &=
        \bigcup_{\mathclap{v \in \op*{Free}(N)}} \parens[\Big]{ \op*{Free}(\Bbbs_{x \mapsto n}(v)) \setminus \set{ n } }
        = \\ &=
        \parens[\Big]{ \bigcup_{\mathclap{v \in \op*{Free}(N) \setminus \set{ x }}} \parens[\Big]{ \op*{Free}(\Bbbs_{x \mapsto n}(v)) \setminus \set{ n } } } \cup \parens[\Big]{ \underbrace{\op*{Free}(\Bbbs_{x \mapsto n}(x)) \setminus \set{ n }}_\varnothing }
        = \\ &=
        \bigcup_{\mathclap{v \in \op*{Free}(N) \setminus \set{ x }}} \parens[\Big]{ \op*{Free}(\Bbbs(v)) \setminus \set{ x } },
        = \\ &=
        \bigcup_{\mathclap{v \in \op*{Free}(M) \setminus \set{ x }}} \op*{Free}(\Bbbs(v)),
      \end{align*}
      where in the last step we have used that \( n \) is not free in \( \Bbbs(v) \) because it does not, by definition, belong to \( \op*{Free}_\Bbbs(N) \).

      This demonstrates \eqref{eq:thm:lambda_substitution_free_variables}.
    \end{itemize}
  \end{itemize}

  The induction is complete.
\end{proof}

\begin{corollary}\label{thm:lambda_substitution_free_variables_single}
  For any two \( \synlambda \)-terms \( M \) and \( L \) and any variable \( u \), we have
  \begin{equation}\label{eq:thm:lambda_substitution_free_variables_single}
    \op*{Free}( M[u \mapsto L] ) = \begin{cases}
      \parens[\Big]{ \op*{Free}(M) \setminus \set{ u } } \cup \op*{Free}(L), &u \in \op*{Free}(M) \\
      \op*{Free}(M),                                                         &\T{otherwise.}
    \end{cases}
  \end{equation}

  In both cases,
  \begin{equation}\label{eq:thm:lambda_substitution_free_variables_single/subset}
    \op*{Free}( M[u \mapsto L] ) \subseteq \parens[\Big]{ \op*{Free}(M) \setminus \set{ u } } \cup \op*{Free}(L).
  \end{equation}
\end{corollary}
\begin{proof}
  Denote by \( \Bbbs \) the substitution sending \( u \) to \( L \). If \( u \) is free in \( M \), then
  \begin{equation*}
    \op*{Free}(M[u \mapsto L])
    \reloset{\eqref{eq:thm:lambda_substitution_free_variables}} =
    \bigcup_{\mathclap{v \in \op*{Free}(M)}} \op*{Free}(\Bbbs(v))
    =
    \op*{Free}(\overbrace{\Bbbs(u)}^L) \cup \bigcup_{\mathclap{v \in \op*{Free}(M) \setminus \set{ u }}} \overbrace{\op*{Free}(\Bbbs(v))}^{\set{ v }}.
  \end{equation*}

  Otherwise,
  \begin{equation*}
    \op*{Free}(M[v \mapsto L])
    \reloset{\eqref{eq:thm:lambda_substitution_free_variables}} =
    \bigcup_{\mathclap{v \in \op*{Free}(M)}} \overbrace{\op*{Free}(\Bbbs(v))}^{\set{ v }}
    =
    \op*{Free}(M).
  \end{equation*}
\end{proof}

\begin{example}\label{ex:def:lambda_substitution}
  We list examples of \hyperref[def:lambda_substitution]{substitution} of \( \synlambda \)-terms:
  \begin{thmenum}
    \thmitem{ex:def:lambda_substitution/simultaneous} We start with an example where we analyze the abstraction rules in detail. Consider
    \begin{equation*}
      \parens[\Big]{ \qabs \synx (xyz) }[\synx \mapsto \syna, \syny \mapsto \synb, \synz \mapsto \sync].
    \end{equation*}

    We have \( \op*{Free}_\Bbbs(M) = \set{ b, c } \). Since \( \synx \) does not belong to this set, we must apply \eqref{eq:def:lambda_substitution/abstraction/direct}:
    \begin{equation*}
      \parens[\Big]{ \qabs \synx (xyz) }[\synx \mapsto \syna, \syny \mapsto \synb, \synz \mapsto \sync]
      =
      \qabs \synx \parens[\Big]{ (xyz)[\syny \mapsto \synb, \synz \mapsto \sync] }
      =
      \qabs \synx \synx \synb \sync.
    \end{equation*}

    \thmitem{ex:def:lambda_substitution/nested_noop} Consider the substitution \( \ref{eq:ex:def:lambda_term/combinator/k}[\syny \mapsto \synx] \). Since \( K \) is a combinator, \fullref{thm:lambda_substitution_combinators} holds and \( K[\syny \mapsto \synx] = K \). We will show this explicitly:
    \begin{align*}
      K[\syny \mapsto \synx]
      &=
      (\qabs \synx \qabs \syny \synx)[\syny \mapsto \synx]
      \reloset {\eqref{eq:def:lambda_substitution/abstraction/direct}} = \\ &=
      \qabs \synx \parens[\Big]{ (\qabs \syny \synx)[\syny \mapsto \synx] }
      \reloset {\eqref{eq:def:lambda_substitution/abstraction/direct}} = \\ &=
      \underbrace{\qabs \synx (\qabs \syny \synx)}_{K}.
    \end{align*}

    \thmitem{ex:def:lambda_substitution/capture} The gist of \fullref{thm:lambda_substitution_free_variables} is that substitution avoids \enquote{capturing} free variables under the scope of some abstraction.

    Inappropriately using the rule \eqref{eq:def:lambda_substitution/abstraction/direct} in
    \begin{equation*}
      (\qabs \synx \synx \syny)[\syny \mapsto \synx]
    \end{equation*}
    would give
    \begin{equation*}
      \qabs \synx (\synx \syny[\syny \mapsto \synx]) = \qabs \synx \synx \synx,
    \end{equation*}
    which \enquote{captures} the corresponding occurrence of \( \synx \) in the scope of the closest abstractor.

    The renaming rule \eqref{eq:def:lambda_substitution/abstraction/renaming} instead gives
    \begin{equation*}
      (\qabs \synx \synx \syny)[\syny \mapsto \synx] = \qabs \syna (\synx \syny[\synx \mapsto \syna][\syny \mapsto \synx]) = \qabs \syna \syna \synx.
    \end{equation*}

    \thmitem{ex:def:lambda_substitution/ignoring} We discussed in \fullref{rem:renaming_substitution_rules} that we can ignore \eqref{eq:def:lambda_substitution/abstraction/direct} and always use the renaming rule \eqref{eq:def:lambda_substitution/abstraction/renaming}.

    \Fullref{thm:lambda_substitution_combinators} would immediately fail:
    \begin{equation*}
      \ref{eq:ex:def:lambda_term/combinator/i}[\syny \mapsto \synz]
      =
      (\qabs \synx \synx)[\syny \mapsto \synz]
      \reloset {\eqref{eq:def:lambda_substitution/abstraction/renaming}} =
      \qabs \syna (\synx[\synx \mapsto \syna][\syny \mapsto \synz])
      =
      \qabs \syna \syna.
    \end{equation*}

    In a slightly more complicated example where substitution is actually performed, we would obtain
    \begin{equation*}
      (\qabs \synx \synx \syny)[\syny \mapsto \synz]
      \reloset {\eqref{eq:def:lambda_substitution/abstraction/renaming}} =
      \qabs \syna (\synx \syny[\synx \mapsto a][\syny \mapsto \synz])
      =
      \qabs \syna \syna \synz
      =
      \qabs \syna \syna \synz.
    \end{equation*}
    rather than
    \begin{equation*}
      \qabs \synx \synx \synz.
    \end{equation*}
  \end{thmenum}
\end{example}

\begin{proposition}\label{thm:lambda_substitution_single_rule}
  For any abstraction \( M = \qabs x N \) and substitution \( \Bbbs \), there exists a variable \( v \not\in \op*{Free}(M) \cup \op*{Free}(M[\Bbbs]) \) such that
  \begin{equation}\label{eq:thm:lambda_substitution_single_rule}
    M[\Bbbs] = \qabs v N[\Bbbs_{x \mapsto v}].
  \end{equation}
\end{proposition}
\begin{proof}
  We have two cases to consider:
  \begin{itemize}
    \item If \( x \in \op*{Free}_\Bbbs(M) \), we must use the renaming rule \eqref{eq:def:lambda_substitution/abstraction/renaming}. Let \( n \) be the renamed abstractor variable. Then
    \begin{equation*}
      M[\Bbbs] = \qabs n N[\Bbbs_{x \mapsto n}].
    \end{equation*}

    Furthermore, \( n \) is not in \( \op*{Free}(N) \cup \op*{Free}_\Bbbs(N) \) --- the first is a superset of \( \op*{Free}(M) \), and the second --- of \( \op*{Free}_\Bbbs(M) \).

    \item Otherwise, we must use \eqref{eq:def:lambda_substitution/abstraction/direct}:
    \begin{equation*}
      M[\Bbbs] = \qabs x N[\Bbbs_{x \mapsto x}].
    \end{equation*}

    The variable \( x \) is clearly not free in \( M \). Furthermore, we have assumed that \( x \not\in \op*{Free}_\Bbbs(M) \).
  \end{itemize}

  Finally, \fullref{thm:lambda_substitution_free_variables} implies that \( \op*{Free}_\Bbbs(M) = \op*{Free}(M[\Bbbs]) \).
\end{proof}
