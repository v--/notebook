\section{Measure theory}\label{sec:measure_theory}

\begin{definition}\label{def:sigma_complete_lattice}\mimprovised
  We say that a \hyperref[def:semilattice/lattice]{lattice} is \term{countably complete} or \( \sigma \)-\term{complete} if its operations are defined for \hyperref[def:set_countability/countably_infinite]{countably infinite} sets.
\end{definition}

\begin{remark}\cite{MathOF:what_does_the_sigma_in_sigma_algebra_stand_for}
  The prefix \enquote{\( \sigma \)-} stands for \enquote{closed under countable unions}.

  Felix Hausdorff chose the letter \( \sigma \) because of the German word \enquote{summe}, which he used for unions, and \( \delta \) because of the German word \enquote{durchschnitt}, which he used for intersections.
\end{remark}

\begin{definition}\label{def:sigma_algebra}
  \todo{Define}.
\end{definition}

\begin{definition}\label{def:measure}
  \todo{Define}.
\end{definition}

\begin{definition}\label{def:lebesgue_measure}
  \todo{Define}.
\end{definition}
