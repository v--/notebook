\chapter{Order theory}\label{ch:order_theory}

\term{Orders} are special \hyperref[def:binary_relation]{binary relations} which, surprisingly, are used to compare elements in a \hyperref[def:set]{set}. Order theory studies pairs \( X = (X, \leq) \), where \( X \) is a set and \( \leq \) is a \hyperref[def:preordered_set]{preorder}.

We denote orders using symbols rather than letters because it is customary to write orders using \hyperref[rem:first_order_formula_conventions/infix]{infix notation}, e.g. \( a \leq b \) rather than \( (a, b) \in {\leq} \).

\cref{fig:ordered_sets_hierarchy} features a hierarchy of ordered sets we will consider here. The dashed lines indicate that the objects under consideration are discussed in other sections.

\begin{figure}[!ht]
  \caption{Some important kinds ordered sets}\label{fig:ordered_sets_hierarchy}
  \smallskip
  \hfill
  \begin{forest}
    [
      {\hyperref[def:preordered_set]{Preordered set}}
        [{\hyperref[def:directed_set]{Directed set}}]
        [
          {\hyperref[def:partially_ordered_set]{Partially ordered set}}
            [
              {\hyperref[def:totally_ordered_set]{Totally ordered set}}
                [
                  {\hyperref[def:well_ordered_set]{Well-ordered set}}, edge=dashed
                  [{\hyperref[def:ordinal]{Ordinal}}, edge=dashed]
                ]
            ]
            [
              {\hyperref[def:lattice]{Semilattice}}
                [
                  {\hyperref[def:lattice]{Lattice}}
                    [
                      {\hyperref[def:heyting_algebra]{Heyting algebra}}
                        [
                          {\hyperref[def:boolean_algebra]{Boolean algebra}}
                          [{\hyperref[def:sigma_algebra]{\( \sigma \)-algebra}}, edge=dashed]
                        ]
                        [{\hyperref[def:category_of_small_locales]{Locale}}, edge=dashed]
                        [{\hyperref[def:topological_space]{Topology}}, edge=dashed]
                    ]
                ]
            ]
        ]
        [{\hyperref[def:equivalence_relation]{Equivalence partition}}]
      ]
  \end{forest}
  \hfill\hfill
\end{figure}

General (semi)lattices also admit algebraic definitions, however these algebraic descriptions have some drawbacks:
\begin{itemize}
  \item There is no general way to extend algebraic operations from finitary to infinitary.

  \item We often implicitly rely on the order structure, for example in \hyperref[def:heyting_algebra]{the definition for Heyting algebra}.
\end{itemize}
