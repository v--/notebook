\subsection{Parametric curves}\label{subsec:parametric_curves}

\begin{definition}\label{def:parametric_curve}\mimprovised
  A \term{parametric curve} in a \hyperref[def:topological_space]{topological space} \( X \) is a set \( C \) of points for which there exists a \hyperref[def:global_continuity]{continuous function} \( \gamma: I \to X \), where \( I \) is a nonempty interval in \( \BbbR \) with potentially infinite endpoints, such that \( \gamma(I) = C \).

  Multiple functions correspond to the same curve, the most trivial example being \( t \mapsto \gamma(-t) \). We call any concrete function \( \gamma: I \to X \) a \term{parametrization} of \( C \). We prefer the interval \( I \) to be either \( [0, 1] \), \( (-\infty, 0] \), \( [0, \infty) \) or \( (-\infty, \infty) \).

  \begin{figure}[!ht]
    \centering
    \includegraphics{output/def__parametric_curve}
    \caption{Part of the Lissajous curve parametrized by \( (\sin(2t), \cos(3t)) \) for \( 3 \pi / 4 \leq t \leq 9 \pi / 4 \).}\label{fig:def:parametric_curve}
  \end{figure}
\end{definition}

\begin{definition}\label{def:simple_curve}\mimprovised
  We say that a \hyperref[def:parametric_curve]{parametric curve} is \term{simple} if it has an injective parametrization.
\end{definition}

\begin{example}\label{ex:def:simple_curve}
  \hfill
  \begin{thmenum}
    \thmitem{ex:def:simple_curve/segment} \hyperref[def:line_segment]{Line segments} in \hyperref[def:topological_vector_space]{topological vector spaces} are simple curves. The segment between \( x \) and \( y \) can be parametrized via the bijective function
    \begin{equation*}
      \begin{aligned}
        &\gamma: [0, 1] \to X \\
        &\gamma(t) \coloneqq ty + (1 - t)x.
      \end{aligned}
    \end{equation*}

    \thmitem{ex:def:simple_curve/lissajous} The Lissajous curve
    \begin{equation*}
      \gamma \coloneqq \begin{pmatrix} \sin(2t) \\ \cos(3t) \end{pmatrix}
    \end{equation*}
    for \( 0 \leq t \leq 2\pi \) is not simple. Part of this curve is shown in \cref{fig:def:parametric_curve}.

    To see that it is not simple, note that the curve is zero at both \( t = \pi / 2 \) and \( t = 3\pi / 2 \). For small enough \( \varepsilon > 0 \), the restrictions of \( \gamma(t) \) to an \( \varepsilon \)-neighborhood around \( \pi / 2 \) and around \( 3\pi / 2 \) are simple curves that intersect only at zero --- see \fullref{fig:ex:def:simple_curve/lissajous}. An injective parametrization of \( \gamma \) cannot be continuous because it needs to pass through zero from different directions. But parametrizations are continuous by definition.

    Therefore, \( \gamma \) has no injective parametrization and is thus not a simple curve.

    \begin{figure}[!ht]
      \centering
      \includegraphics{output/ex__def__simple_curve__lissajous}
      \caption{Self-intersection at zero of a Lissajous curve.}\label{fig:ex:def:simple_curve/lissajous}
    \end{figure}
  \end{thmenum}
\end{example}

\begin{example}\label{ex:curve_moving_backwards}
  One thing we wish to avoid is \enquote{moving backwards} during parametrization of a \hyperref[def:parametric_curve]{parametric curve}.

  Consider the \hyperref[def:line_segment]{line segment}
  \begin{equation*}
    \gamma(t) \coloneqq (1 - t) x + ty,
  \end{equation*}
  where \( 0 \leq t \leq 1 \).

  \begin{figure}[!ht]
    \centering
    \includegraphics{output/ex__curve_moving_backwards}
    \caption{A line segment being traversed twice in opposite directions.}\label{fig:ex:curve_moving_backwards}
  \end{figure}

  We can also parametrize it via the function
  \begin{equation*}
    \delta(s) \coloneqq \begin{cases}
      (1 - s)x + sy,       &0 \leq s \leq 1, \\
      (s - 1)x + (2 - s)y, &1 \leq s \leq 2.
    \end{cases}
  \end{equation*}

  In this case, we have the symmetry \( \delta(1 + s) = \delta(1 - s) \) for \( 0 \leq s \leq 1 \). Hence, when regarding the parameter as time, after moving from \( x \) to \( y \), we start \enquote{moving backwards} via the same path.

  This can cause problems with certain definitions. For example, if we regard merely continuous periodic curves as \hyperref[def:closed_curve]{closed}, \( \delta \) can be extended to a periodic function, making the segment a closed curve.

  Fortunately, this can be avoided easily by requiring that \( \delta \) is \hyperref[def:differentiability/frechet]{differentiable}. In our case,
  \begin{equation*}
    \delta'(s) = \begin{cases}
      -x + y, &0 \leq s < 1, \\
      x - y,  &1 < s \leq 2.
    \end{cases}
  \end{equation*}

  But \( \delta \) is not differentiable at \( s = 1 \). On the other hand, \( \gamma \) is differentiable, and it avoids \enquote{moving backwards}.

  Among other pathologies, this leads to \fullref{def:smooth_curve}.
\end{example}

\begin{definition}\label{def:smooth_curve}\mcite[8]{ИвановТужилин2017}
  An \( r \) times \term{smooth curve} is a \hyperref[def:parametric_curve]{parametric curve} for which there exists an \( r \) times \hyperref[def:differentiability/frechet]{differentiable} parametrization \( \gamma: I \to \BbbR^n \). If
  \begin{equation*}
    \gamma(t) = \begin{pmatrix}
      \gamma_1(t) \\
      \vdots \\
      \gamma_n(t)
    \end{pmatrix},
  \end{equation*}
  then the \( k \)-th pointwise derivative is at \( t \in I \) is the vector
  \begin{equation*}
    \gamma^{(k)}(t) = \begin{pmatrix}
      \gamma_1^{(k)}(t) \\
      \vdots \\
      \gamma_n^{(k)}(t)
    \end{pmatrix}.
  \end{equation*}

  By \enquote{parametrization of a smooth curve} we implicitly assume one that is \( r \) times differentiable.

  We call the first derivative \( \gamma'(t) \) --- \term{speed vector} or \term{tangent vector} and \( \gamma^\dprime(t) \) --- the \term{acceleration vector}. Both concepts obviously depend on the parametrization.
\end{definition}

\begin{remark}\label{rem:smooth_curve}
  We will use the term \enquote{smooth} in the context of \hyperref[def:smooth_curve]{smooth curves} and implicitly mean \( r \)-times smooth for large enough \( r \). It is possible for us to require the curve to be infinitely differentiable, in which case we write \( r = \infty \).
\end{remark}

\begin{proposition}\label{thm:tangent_vector_dependent}
  Let \( \gamma: I \to \BbbR^n \) and \( \delta: J \to \BbbR^n \) be two parametrizations of the same smooth curve. Let \( \gamma(t_0) = \delta(s_0) \). Furthermore, suppose that the curve does not intersect itself at this point. That is, suppose that \( s_0 \) is the only solution to \( \gamma(t_0) = \delta(s) \).

  Then the speed vectors \( \gamma'(t_0) \) and \( \delta'(s_0) \) are linearly dependent.
\end{proposition}
\begin{proof}
  If either \( \gamma'(t_0) \) or \( \delta'(s_0) \) is the zero vector, they are vacuously linearly dependent.

  Suppose that both are nonzero. \Fullref{thm:inverse_function_theorem} implies that there exist some neighborhoods \( I \) of \( t_0 \) and \( U \) of \( \gamma(t_0) \) in which there exists a continuously differentiable local inverse \( g: U \to I \) of \( \gamma \). Let \( J \) be the preimage of \( U \) under \( \delta \). Then \( t \coloneqq g \bincirc \delta \) is a smooth map from \( J \) to \( I \). Hence, for \( s \in J \),
  \begin{equation*}
    \delta(s) = \gamma(t(s)).
  \end{equation*}

  Consequently,
  \begin{equation*}
    \delta'(s) = \gamma'(t(s)) \cdot t'(s).
  \end{equation*}

  We have assumed that \( s_0 \) is the only parameter for which \( t_0 = t(s_0) \). We conclude that \( \gamma'(t_0) \) and \( \delta'(s_0) \) are linearly dependent.
\end{proof}

\begin{definition}\label{def:tangent_line}
  Let \( \gamma: I \to \BbbR^n \) be a parametrization of a smooth curve. Suppose that the curve does not intersect itself at \( \gamma(t_0) \) (restrict the domain if necessary). Also suppose that \( \gamma'(t_0) \) is not zero.

  Then we define the \term{tangent line} at \( \gamma(t_0) \) as the \hyperref[def:affine_line]{affine line} with origin \( \gamma(t_0) \) and direction vector \( \gamma'(t_0) \).

  \begin{figure}[!ht]
    \centering
    \includegraphics{output/def__tangent_line}
    \caption{Several tangent lines to the same curve at different points}\label{fig:def:tangent_line}
  \end{figure}

  \Fullref{thm:tangent_vector_dependent} implies that tangent lines, unlike tangent vectors, do not depend on the parametrization.

  If the curve intersects itself at \( \gamma(t_0) \), the concept is ambiguous.
\end{definition}

\begin{definition}\label{def:closed_curve}\mimprovised
  We say that the \hyperref[def:smooth_curve]{smooth curve} \( C \) is \term{closed} if it can be parametrized via a smooth \hyperref[def:periodic_function]{periodic function} \( \gamma: \BbbR \to X \).

  Suppose that \( \gamma \) has period \( p \). Then every interval \( [a, a + p) \) is also a parametrization of \( C \), and for convenience we often take the closed interval \( [a, a + p] \). In particular, if \( \gamma: [a, b) \) is a closed curve that is also \hyperref[def:simple_curve]{simple}, we sometimes refer to \( \gamma: [a, b] \to X \) as a \enquote{simple closed curve}. Here \( \gamma(a) = \gamma(b) \) hods for \( a \neq b \) and \( \gamma \) not injective, however the curve admits an injective parametrization obtained by simply removing \( b \) from the domain of \( \gamma \).

  Of course, different periodic parametrizations may have different periods.

  Conversely, if \( \delta: [a, b] \to X \) is some parametrization of \( C \) such that \( \delta(a) = \delta(b) \), then
  \begin{equation*}
    \begin{aligned}
      &\widetilde{\delta}: \BbbR \to X \\
      &\widetilde{\delta}(t) \coloneqq \delta(a + \rem(t - a, b - a)).
    \end{aligned}
  \end{equation*}
  is a differentiable periodic function.

  Consequently, for a general closed curve, we have no obvious choice of parametrization.

  \begin{figure}[!ht]
    \hfill
    \includegraphics{output/def__closed_curve__a}
    \hfill
    \hfill
    \includegraphics{output/def__closed_curve__b}
    \hfill
    \hfill
    \caption{Two different parametrizations of the same closed trefoil curve.}\label{fig:def:closed_curve}
  \end{figure}
\end{definition}

\begin{example}\label{ex:def:closed_curve}
  \hfill
  \begin{thmenum}
    \thmitem{ex:def:closed_curve/segment} \hyperref[def:line_segment]{Line segments} are not closed curves because the canonical parametrization is not periodic. Furthermore, differentiability avoids the pathologies discussed in \fullref{ex:curve_moving_backwards}, which would technically make line segments closed curves.

    \thmitem{ex:def:closed_curve/lissajous} The parametrization
    \begin{equation*}
      \gamma \coloneqq \begin{pmatrix} \sin(2t) \\ \cos(3t) \end{pmatrix}
    \end{equation*}
    of a Lissajous curve is \( 2\pi \)-periodic, hence the curve is closed.
  \end{thmenum}
\end{example}

\begin{example}\label{ex:curve_staying_in_place}
  We have avoided \hyperref[ex:curve_moving_backwards]{\enquote{moving backwards}} by requiring that curves are \hyperref[def:smooth_curve]{smooth}, however we have not avoided \enquote{staying in place}.

  For example, the graph of the absolute value function can be parametrized as follows:
  \begin{equation*}
    \gamma(t) \coloneqq \begin{cases}
      ( t^2     , t^2 )^T,     &-1 \leq t \leq 0, \\
      ( 0       , 0 )^T,       &0 < t < 1, \\
      ( (t-1)^2 , (t-1)^2 )^T, &1 \leq t \leq 2, \\
    \end{cases}
  \end{equation*}

  Its derivative is
  \begin{equation*}
    \gamma'(t) = \begin{cases}
      ( 2t     , 2t )^T,     &-1 \leq t \leq 0, \\
      ( 0      , 0 )^T,      &0 < t < 1, \\
      ( 2(t-1) , 2(t-1) )^T, &1 \leq t \leq 2, \\
    \end{cases}
  \end{equation*}

  Clearly \( \gamma(t) \) is continuously differentiable. We can consider \( t^{2r} \) instead of \( t^2 \) in order to make it \( 2r - 1 \) times continuously differentiable.

  We do have, however, \( \gamma(t_1) = \gamma(t_2) \) for \( 0 \leq t_1 \leq t_2 \leq 1 \). This is one of the things we generally wish to avoid when working with curves. For this reason, \hyperref[def:regular_curve]{regular curves} impose the additional requirement that \( \gamma'(t) \) should be nonzero at any point.
\end{example}

\begin{definition}\label{def:curve_speed}
  Given a parametrization of a \hyperref[def:smooth_curve]{smooth curve}, its \term{speed} at a point is defined as the norm of its speed vector at that point. The concept depends on the parametrization, and for this reason we prefer \hyperref[thm:natural_parametrization_existence]{natural parameters} with constant speed.
\end{definition}

\begin{example}\label{ex:nonregular_curve}
  Consider the smooth curve
  \begin{equation*}
    \gamma(t)
    \coloneqq
    \begin{pmatrix}
      t^3 \\ t^2
    \end{pmatrix},
  \end{equation*}
  where \( t \) is any real number. For any \( t \), the speed vector is
  \begin{equation*}
    \gamma'(t)
    \coloneqq
    \begin{pmatrix}
      3t^2 \\ 2t
    \end{pmatrix}.
  \end{equation*}

  \begin{figure}[!ht]
    \centering
    \includegraphics{output/ex__nonregular_curve}
    \caption{The curve \( (t^3, t^2) \) has a cusp at zero.}\label{fig:ex:nonregular_curve}
  \end{figure}

  The \hyperref[def:curve_speed]{speed} is
  \begin{equation*}
    \norm{\gamma'(t)} = \abs{t} \sqrt{9 t^2 + 4}.
  \end{equation*}

  Fix some \( \varepsilon > 0 \). Then, for any point \( t \) such that \( 0 < 9 t^2 + 4 < \varepsilon^2 \), we have
  \begin{equation*}
    \norm{\gamma'(t)} < \varepsilon.
  \end{equation*}

  Hence, we technically avoid the \enquote{staying in place} discussed in \fullref{ex:curve_staying_in_place}, but the speed of movement becomes arbitrarily slow until it eventually reaches zero.

  In this particular case, the curve does not allow a parametrization with entirely nonzero speed. Indeed, suppose that there exists a parametrization \( \delta: J \to \BbbR^2 \) that has nonzero speed at every point. Let \( s_0 \) be a value for which \( \delta(s_0) = \gamma(0) = (0, 0) \).

  \Fullref{thm:inverse_function_theorem} implies that there exist neighborhoods \( (a, b) \) of \( s_0 \) and \( U \) of \( (0, 0) \) in which \( \delta \) is invertible via some continuously differentiable function \( f: U \to (a, b) \).

  For small enough \( t > 0 \), the point \( \gamma(t) = (t^3, t^2) \) is in \( U \), hence \( a < f(\gamma(t)) < b \). Define
  \begin{equation*}
    s(t) \coloneqq f(\gamma(t)).
  \end{equation*}

  Then
  \begin{equation*}
    s'(t) = f'(\gamma(t)) \cdot \gamma'(t).
  \end{equation*}

  This derivative is zero at \( t = 0 \), implying that \( s(t) \) is constant. But \( \gamma \) and \( f \) are both injective, hence \( s(t) \) should also be injective.

  The obtained contradiction shows that \( \delta \) must have zero speed at \( 0 \).
\end{example}

\begin{definition}\label{def:regular_curve}
  A \hyperref[def:smooth_curve]{smooth curve} is called \term{regular} if it has a parametrization whose speed is always nonzero.
\end{definition}

\begin{example}\label{ex:nonregular_parametrization}\mcite[rem. 1.16]{ИвановТужилин2017}
  Consider the \hyperref[def:smooth_curve]{smooth curve}
  \begin{equation*}
    \gamma(t)
    \coloneqq
    \begin{pmatrix}
      t \\ t^2
    \end{pmatrix},
  \end{equation*}
  where \( t \) is any real number. For any \( t \), the speed vector is
  \begin{equation*}
    \gamma'(t)
    \coloneqq
    \begin{pmatrix}
      1 \\ 2t
    \end{pmatrix}.
  \end{equation*}

  It is never the zero vector, hence the curve is \hyperref[def:regular_curve]{regular}.

  We can also reparametrize via \( t(s) = s^3 \) so that
  \begin{equation*}
    \delta(s)
    \coloneqq
    \gamma(t(s))
    =
    \begin{pmatrix}
      s^3 \\ s^6
    \end{pmatrix}.
  \end{equation*}

  Then the derivative
  \begin{equation*}
    \delta'(s)
    =
    \begin{pmatrix}
      3t^2 \\ 6t^5
    \end{pmatrix}
  \end{equation*}
  is the zero vector when \( t = 0 \). Hence, \( \delta \) is not a regular parametrization of a regular curve.

  \begin{figure}[!ht]
    \centering
    \includegraphics{output/ex__nonregular_parametrization}
    \caption{Part of the curve \( (t, t^2) \) from \( t = -1 \) to \( t = 1 \).}\label{fig:ex:nonregular_parametrization}
  \end{figure}
\end{example}

\begin{proposition}\label{thm:natural_parametrization_existence}
  Every \hyperref[def:regular_curve]{regular} \hyperref[def:parametric_curve]{parametric curve} has a parametrization for which the \hyperref[def:curve_speed]{speed} is always \( 1 \). We call such a parametrization a \term{natural parametrization} or, due to \fullref{thm:length_of_smooth_curves}, an \term{arc length parametrization}.
\end{proposition}
\begin{proof}
  Consider the regular curve \( \gamma: I \to \BbbR^n \). Define the function
  \begin{equation*}
    s(t) \coloneqq \int_0^t \norm{\gamma'(\tau)} \cdot \dl \tau
  \end{equation*}
  over the same interval.

  \Fullref{thm:fundamental_theorem_of_calculus} implies that
  \begin{equation*}
    s'(t) = \norm{\gamma'(t)}.
  \end{equation*}

  Since \( \gamma \) is regular, \( t'(s) \) is nonzero. Hence, its inverse has a derivative
  \begin{equation*}
    t'(s) = \frac 1 {\norm{\gamma'(s)}}.
  \end{equation*}

  For \( \delta(s) \coloneqq \gamma(t(s)) \), \fullref{thm:chain_rule} implies
  \begin{equation*}
    \delta'(s)
    =
    t'(s) \cdot \gamma'(s)
    =
    \frac {\gamma'(s)} {\norm{\gamma'(s)}}.
  \end{equation*}

  Since \( t_0 \) was arbitrary, we conclude that this holds for the entire interval \( I \).
\end{proof}

\begin{example}\label{ex:natural_reparametrization_of_line}
  We can demonstrate the trick from \fullref{thm:natural_reparametrization_existence} to reparametrize the line \( \gamma(t) = O + td \). Define
  \begin{equation*}
    s(t) \coloneqq \int_0^t \norm{\gamma'(\tau)} \cdot \dl \tau = t \norm{d}.
  \end{equation*}

  Its inverse is
  \begin{equation*}
    t(s) = \frac s {\norm{d}}.
  \end{equation*}

  Then
  \begin{equation*}
    \gamma(t(s)) = O + s \frac d {\norm{d}},
  \end{equation*}
  and this is obviously a natural reparametrization.
\end{example}

\begin{proposition}\label{thm:natural_reparametrization_existence}
  For two \hyperref[thm:natural_parametrization_existence]{natural parameters} \( t_1: J_1 \to I \) and \( t_2: J_2 \to I \) of \( \gamma: I \to \BbbR^n \), there exists some constant \( a \) such that \( J_2 = a + J_1 \) and, for all \( s \in J_2 \), \( t_2(s) = t_1(a + s) \).
\end{proposition}
\begin{proof}
  Define
  \begin{equation*}
    \delta_1(s) \coloneqq \gamma(t_1(s))
  \end{equation*}
  and similarly for \( \delta_2(s) \).

  We have
  \begin{equation*}
    \delta_1'(s) = t_1'(s) \cdot \gamma'(t_1(s)).
  \end{equation*}

  Since \( \delta_1 \) is natural, it follows that
  \begin{equation*}
    1 = \norm{\delta_1'(s)} = \abs{t_1'(s)} \cdot \norm{\gamma'(s)},
  \end{equation*}
  and similarly for \( \delta_2(s) \)

  Furthermore, since \( t_1 \) and \( t_2 \) are strictly monotone, their derivatives positive. Therefore,
  \begin{equation*}
    t_1'(s) = t_2'(s) = \frac 1 {\norm{\gamma'(s)}}.
  \end{equation*}

  Then \( t_1'(s) - t_2'(s) = 0 \), implying that the different between \( t_1(s) \) and \( t_2(s) \) is a constant.
\end{proof}

\begin{definition}\label{def:regular_curve_curvature}
  If the \hyperref[thm:natural_reparametrization_existence]{naturally parametrized} curve \( \gamma: I \to \BbbR^n \) does not intersect itself at the point \( \gamma(t_0) \), we call the norm \( \norm{\gamma^\dprime(t_0)} \) of the \hyperref[def:smooth_curve]{acceleration vector} the \term{curvature} of \( \gamma \) at \( \gamma(t_0) \). \Fullref{thm:natural_reparametrization_existence} ensures that this definition does not depend on the parametrization, as long as it is natural.

  If the curve intersects itself at \( \gamma(t_0) \), the concept is ambiguous.
\end{definition}

\begin{proposition}\label{thm:line_curvature}
  The \hyperref[def:regular_curve_curvature]{curvature} of the \hyperref[thm:natural_reparametrization_existence]{naturally parametrized} \( \gamma: I \to \BbbR^n \) is always \( 0 \) if and only if \( \gamma(t) = O + td \).

  That is, \( \gamma(t) \) is an
  \begin{itemize}
    \item \hyperref[def:affine_line]{affine line} in case \( I \) is unbounded.
    \item \hyperref[def:geometric_ray]{ray} in case \( I \) is bounded from one side.
    \item \hyperref[def:affine_line]{affine line} in case \( I \) is bounded from both sides.
  \end{itemize}
\end{proposition}
\begin{proof}
  \SufficiencySubProof Suppose that \( \norm{\gamma^\dprime(t)} = 0 \). Then the acceleration vector is zero, implying that the speed vector \( d \coloneqq \gamma'(t) \) is a constant and \( \gamma \) itself is
  \begin{equation*}
    \gamma(t) = O + t d
  \end{equation*}
  for some point \( O \).

  Determining both \( O \) and \( d \) requires additional information.

  \NecessitySubProof Clearly \( \gamma'(t) = d \) and thus \( \gamma^\dprime(t) = \vect 0 \).
\end{proof}

\begin{definition}\label{def:arc_length}
  \begin{figure}[!ht]
    \centering
    \includegraphics[page=1]{output/def__parametric_curve_length__approximation}
    \caption{A rough approximation of a curve using three line segments.}\label{def:arc_length/approximation}.
  \end{figure}

  Let \( X \) be a \hyperref[def:banach_space]{Banach space}, let \( C \) be a \hyperref[def:parametric_curve]{parametric curve} and let \( \gamma: [a, b] \to X \) be some parametrization of \( C \).

  For each \hyperref[def:riemann_partition/tagged]{tagged Riemann partition}
  \begin{equation*}
    \begin{aligned}
      &\Delta: a = t_0 < t_1 < \ldots < t_n = b \\
      &\Tau: \tau_k \in [t_{k-1}, t_k], k = 1, \ldots, n,
    \end{aligned}
  \end{equation*}
  assign
  \begin{equation*}
    \len(\gamma, \Delta, \Tau) \coloneqq \sum_{k=1}^n \norm{\gamma(\tau_k) - \gamma(\tau_{k-1})}.
  \end{equation*}

  If the limit over all tagged partition with respect to the order \fullref{def:riemann_partition/order/diameter} exists, we call it the \term{arc length} of \( C \) with respect to \( \gamma \) and say that \( C \) is \term{rectifiable}.

  This concept unfortunately depends on the parametrization in the general case.
\end{definition}

\begin{proposition}\label{thm:length_of_smooth_curves}
  For a \hyperref[def:smooth_curve]{smooth curve} \( C \) with parametrization \( \gamma: [a, b] \to \BbbR^n \), we have
  \begin{equation*}
    \len(\gamma) = \int_a^b \norm{\gamma'(t)} \dl t.
  \end{equation*}
\end{proposition}
\begin{proof}
  By \fullref{thm:lagranges_mean_value_theorem}, given a Riemann partition
  \begin{equation*}
    \Delta: a = t_0 < \cdots < t_n = b,
  \end{equation*}
  for each \( k = 1, \ldots, n \) there exists a point \( \tau_k \in [t_{k-1}, t_k] \) such that
  \begin{equation*}
    \gamma(t_k) - \gamma(t_{k-1}) = \gamma'(\tau_k) (t_k - t_{k-1}).
  \end{equation*}

  Then
  \begin{equation*}
    \len(\gamma, \Delta, \Tau)
    =
    \sum_{k=1}^n \norm{\gamma(t_k) - \gamma(t_{k-1})}
    =
    \sum_{k=1}^n \norm{\gamma'(\tau_k)} (t_k - t_{k-1}),
  \end{equation*}
  which is a standard Riemann sum for the function \( t \mapsto \norm{\gamma'(t)} \). Therefore, \fullref{thm:countinuous_functions_integrable} implies that the curve is rectifiable and
  \begin{equation*}
    \len(\gamma) = \int_a^b \norm{\gamma'(t)} \dl t.
  \end{equation*}
\end{proof}

\begin{corollary}\label{thm:length_of_piecewise_smooth_curves}
  The \hyperref[def:arc_length]{arc length} of a piecewise smooth curve is the sum of the lengths of its components.
\end{corollary}

\begin{corollary}\label{thm:length_of_function_graph}
  The \hyperref[def:arc_length]{length} of the \hyperref[def:set_valued_map/graph]{graph} of a \hyperref[def:differentiability/frechet]{differentiable} function \( f: [a, b] \to \BbbR \), if it exists, is given by
  \begin{equation*}
    \len(\gph(f)) \coloneqq \int_a^b \sqrt{1 + [f'(x)]^2} \dl x.
  \end{equation*}
\end{corollary}
\begin{proof}
  Apply \fullref{thm:length_of_smooth_curves} for the parametric curve \( t \mapsto (t, f(t)) \).
\end{proof}

\begin{definition}\label{def:perimeter}
  Consider a \hyperref[con:geometric_shape]{geometric shape}, i.e. a set of points in an Euclidean space. Suppose that the \hyperref[def:topological_boundary_operator]{boundary} of the shape is a \hyperref[def:parametric_curve]{parametric curve}. Suppose also that the curve is \hyperref[def:arc_length]{rectifiable}.

  In this case, we call the arc length of this curve the \term{perimeter} of the shape.
\end{definition}
