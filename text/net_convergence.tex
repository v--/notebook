\subsection{Net convergence}\label{subsec:net_convergence}

\subsubsection{Nets}

\begin{definition}\label{def:topological_net}\mcite[65]{Kelley1975}
  A \term{net} or \term{generalized sequence} is a family of elements of a \hyperref[def:topological_space]{topological space} \hyperref[def:cartesian_product/indexed_family]{indexed} by a \hyperref[def:directed_set]{directed set}.

  We reuse the notation for indexed families from \fullref{def:cartesian_product/indexed_family}:
  \begin{equation*}
    \set{ x_k }_{k \in \mscrK}
  \end{equation*}
\end{definition}
\begin{comments}
  \item This definition does not require a topology, but, unlike sequences, this concept of a net is only discussed in the context of topological spaces.
\end{comments}

\begin{definition}\label{def:net_eventually_in}\mcite[65]{Kelley1975}
  We say that the \hyperref[def:topological_net]{net} \( \set{ x_k }_{k \in \mscrK} \) is \term{eventually in} the set \( A \) if there exists an index \( k_0 \in \mscrK \) such that, for every \( k \geq k_0 \), we have \( x_k \in A \).
\end{definition}

\begin{proposition}\label{thm:eventually_in_implies_nonempty}
  If the net \( \set{ x_k }_{k \in \mscrK} \) is \hyperref[def:net_eventually_in]{eventually in} some set \( A \), then both \( \mscrK \) and \( A \) are nonempty.
\end{proposition}
\begin{proof}
  By definition, there exists some index \( k_0 \in \mscrK \) such that \( x_{k_0} \in A \).
\end{proof}

\begin{proposition}\label{thm:sequence_eventally_in}
  An \hyperref[def:sequence]{infinite sequence} is eventually in some set if and only if only finitely many sequence elements are outside the set.
\end{proposition}
\begin{proof}
  A sequence \( \seq{ x_k }_{k=1}^\infty \) is eventually in \( A \) if there exists a positive integer \( k_0 \) such that \( x_k \in A \) whenever \( k \geq k_0 \). Since the positive integers are totally ordered, the condition is equivalent to requiring that there exists some upper limit for the set \( \set{ k \given x_k \not\in A } \) --- every such limit can be taken to be \( k_0 \). \Fullref{thm:natural_number_cofinal_subsets} implies that the latter condition is equivalent to requiring that only finitely many sequence elements are outside \( A \).
\end{proof}

\begin{definition}\label{def:net_frequently_in}\mcite[65]{Kelley1975}
  We say that the \hyperref[def:topological_net]{net} \( \set{ x_k }_{k \in \mscrK} \) is \term{frequently in} the set \( A \) if, for every index \( k_0 \in \mscrK \), there exists some \( k \geq k_0 \) such that \( x_k \in A \).
\end{definition}

\begin{proposition}\label{thm:empty_index_set_implies_frequently_in}
  A net \( \set{ x_k }_{k \in \varnothing} \) over the empty set is \hyperref[def:net_frequently_in]{frequently in} any set.
\end{proposition}
\begin{proof}
  Any condition with quantifier \enquote{for every index in \( \varnothing \)} is vacuously true.
\end{proof}

\begin{proposition}\label{thm:eventually_in_implies_frequently_in}
  If a net is \hyperref[def:net_eventually_in]{eventually in} a set, it is also \hyperref[def:net_frequently_in]{frequently in} it.
\end{proposition}
\begin{proof}
  Suppose that \( \seq{ x_k }_{k \in \mscrK} \) is eventually in \( A \). Then there exists some index \( k_e \in \mscrK \) such that \( x_k \in A \) for \( k \geq k_e \).

  Fix some arbitrary index \( k_f \in \mscrK \). Then, since \( \mscrK \) is directed, there must exist an upper bound \( k_u \) for \( k_e \) and \( k_f \).

  Since \( k_u \geq k_e \), we have \( x_{k_u} \in A \). Therefore, \( \seq{ x_k }_{k \in \mscrK} \) is frequently in \( A \).
\end{proof}

\begin{proposition}\label{thm:eventually_and_frequently_in}
  The net \( \seq{ x_k }_{k \in \mscrK} \) in the space \( X \) is \hyperref[def:net_eventually_in]{eventually in} \( A \) if and only if it is not \hyperref[def:net_frequently_in]{frequently in} \( X \setminus A \).
\end{proposition}
\begin{proof}
  \Fullref{thm:relativized_first_order_quantifiers_are_dual} implies that
  \begin{equation*}
    \qexists {k_0 \in \mscrK} \qforall {k \geq k_0} x_k \in A
  \end{equation*}
  if and only if
  \begin{equation*}
    \neg \qforall {k_0 \in \mscrK} \qexists {k \geq k_0} \neg (x_k \in A).
  \end{equation*}
\end{proof}

\begin{proposition}\label{thm:cofinal_iff_frequently_in}
  Let \( \mscrK \) be a directed set, let \( A \subseteq \mscrK \) be a subset and let \( \seq{ x_k }_{k \in \mscrK} \) be a net.

  Then \( A \) is \hyperref[def:cofinal_set]{cofinal} if and only if \( x \) is \hyperref[def:net_frequently_in]{frequently in} the \( \set{ x_a \given a \in A } \).
\end{proposition}
\begin{proof}
  The net \( \seq{ x_k }_{k \in \mscrK} \) is frequently in \( \set{ x_a \given a \in A } \) if
  \begin{equation*}
    \qforall {k_0 \in \mscrK} \qexists {a_0 \in \mscrK} k \leq a_0 \T{and} x_{a_0} \in A,
  \end{equation*}
  which can be rewritten as
  \begin{equation*}
    \qforall {k_0 \in \mscrK} \qexists {a_0 \in A} k \leq a_0.
  \end{equation*}

  This is exactly the condition for \( A \) to be cofinal in \( \mscrK \).
\end{proof}

\begin{example}\label{ex:def:topological_net}
  We list several examples of \hyperref[def:topological_net]{nets}:
  \begin{thmenum}
    \thmitem{ex:def:topological_net/empty} The most basic example of a net is the empty set. Indeed, an empty set is vacuously directed.

    It is also vacuously \hyperref[def:net_frequently_in]{frequently in} any set as a consequence of \fullref{thm:empty_index_set_implies_frequently_in}, but it is not \hyperref[def:net_eventually_in]{eventually in} any set because of \fullref{thm:eventually_in_implies_nonempty}.

    \thmitem{ex:def:topological_net/finite_sequence} Any \hyperref[def:sequence]{finite sequence} \( x_1, \ldots, x_n \) is a net.

    It is eventually in (and hence also frequently in) \( \set{ x_n } \) and any of its supersets.

    \thmitem{ex:def:topological_net/sequence} For a finite sequence it suffices for \( A \) to contain the last element, but for an infinite sequence \( x_1, x_2, x_3, \ldots \), this condition is more complicated. \Fullref{thm:cofinal_iff_frequently_in} implies that the sequence is frequently in some set \( A \) if and only if \( A \) contains a \hyperref[def:cofinal_set]{cofinal subset} of the sequence (when regarding the sequence as a well-ordered set).

    \thmitem{ex:def:topological_net/reverse} Given any point \( x_0 \), its \hyperref[def:neighborhood_filter]{neighborhood filter} \( \mscrT(x_0) \) is naturally a \hyperref[def:semilattice/lattice]{lattice} with respect to inclusion. Indeed, the union and intersection of any pair of neighborhoods of \( x_0 \) is again a neighborhood of \( x_0 \).

    We can also consider the \hyperref[def:semilattice/duality]{dual lattice} of \( \mscrT(x_0) \), that is, the \enquote{reverse inclusion lattice} where \( V \leq U \) if \( V \supseteq U \).

    The latter is useful as an index set of nets, for example in the proofs of \fullref{thm:def:net_limit_point/cluster_point_subnet_limit} and \fullref{thm:cluster_point_iff_in_closure} or the proof of equivalence in \fullref{def:local_convergence}.
  \end{thmenum}
\end{example}

\subsubsection{Subnets}

It is unfortunately an established convention for subsequences to be defined differently compared to subnets. There are actually different definitions of subnets --- see \cite{Lehuta2009} --- but we will base our exposition on that of John Kelley in \cite[ch. 2]{Kelley1975}.

\begin{definition}\label{def:subsequence}\mcite[63]{Kelley1975}
  We say that
  \begin{equation}\label{eq:def:subsequence/sub}
    y_1, y_2, y_3, y_4, \ldots
  \end{equation}
  is a \term{subsequence} of
  \begin{equation}\label{eq:def:subsequence/original}
    x_1, x_2, x_3, x_4, \ldots
  \end{equation}
  if there exists a strictly increasing sequence of indices
  \begin{equation*}
    i_1 < i_2 < i_3 < i_4 < \cdots
  \end{equation*}
  such that \( y_k = x_{i_k} \) for every \( k = 1, 2, \ldots \).
\end{definition}
\begin{comments}
  \item While Kelley in \cite[63]{Kelley1975} calls this definition \enquote{unnecessarily stringent} before introducing nets, other authors like Walter Rudin in \cite[def. 3.5]{Rudin1976Principles} prefer this definition.

  \item If we regard \( x \) and \( y \) as functions from \( \BbbZ_{>0} \) to \( X \), then \( y \) is a subsequence of \( x \) if and only if there exists a \hyperref[def:order_homomorphism/increasing]{strictly increasing} inclusion \hyperref[def:function/endofunction]{endofunction} \( \iota: \BbbZ_{>0} \to \BbbZ_{>0} \) such that \( y = x \cdot \iota \).
\end{comments}

\begin{definition}\label{def:subnet}\mcite[70]{Kelley1975}
  We say that the net \( \seq{ y_m }_{m \in \mscrM} \) is a \term{subnet} of \( \seq{ x_k }_{k \in \mscrK} \) if there exists a family \( \seq{ k_m }_{m \in \mscrM} \) of indices from \( \mscrK \) such that
  \begin{thmenum}
    \thmitem[def:subnet/composition]{SN1} For every \( m \in \mscrM \), we have \( x_{k_m} = y_m \).

    This condition ensures that the subnet consists of members of the supernet.

    \thmitem[def:subnet/cofinality]{SN2} For every \( k_0 \in \mscrK \), there exists some index \( m_0 \in \mscrM \) such that, for every \( m \geq m_0 \), we have \( k_m \geq k_0 \).

    This condition ensures that the subnet \enquote{follows} the direction of the supernet. It involves only the index sets and not the actual nets.
  \end{thmenum}
\end{definition}
\begin{comments}
  \item \ref{def:subnet/cofinality} states that the subnet must be eventually in the tail \( \set{ x_k \given k \geq k_0 } \) for every \( k_0 \in \mscrK \).
\end{comments}

\begin{example}\label{ex:def:subnet}
  We list several examples of \hyperref[def:subnet]{subnets}:
  \begin{thmenum}
    \thmitem{ex:def:subnet/empty} The empty net has no empty subnet.

    \thmitem{ex:def:subnet/subsequence} Consider the \hyperref[def:sequence]{sequence}
    \begin{equation*}
      x_1, x_2, x_3, \ldots
    \end{equation*}

    The sequence
    \begin{equation*}
      x_1, x_3, x_5, \ldots
    \end{equation*}
    with only odd indices is a \hyperref[def:subsequence]{subsequence}, and so is
    \begin{equation*}
      x_2, x_3, x_4, \ldots
    \end{equation*}

    The above are also subnets.

    \thmitem{ex:def:subnet/padded_sequence} The following is a subnet but not subsequence:
    \begin{equation*}
      \underbrace{0}_{y_1}, \underbrace{x_1}_{y_2}, \underbrace{x_2}_{y_3}, \underbrace{x_3}_{y_4}, \underbrace{x_4}_{y_5}, \cdots.
    \end{equation*}

    Indeed, consider the family of indices
    \begin{equation*}
      k_m = \begin{cases}
        0,     &m = 1, \\
        m - 1, &m > 1.
      \end{cases}
    \end{equation*}

    The family is not monotone, yet it defines a subnet because it satisfies \fullref{def:subnet}:
    \begin{itemize}
      \item Since \( y_m = x_{k_m} \) for every \( m = 1, 2, \ldots \), \ref{def:subnet/composition} holds.
      \item For every \( k_0 \), we have an index \( m_0 = k_0 + 1 \) such that \( k_m = m - 1 \geq m_0 - 1 = k_0 \) for \( m \geq m_0 \); hence \ref{def:subnet/cofinality} also holds.
    \end{itemize}

    \thmitem{ex:def:subnet/real_subnet} Consider again some sequence
    \begin{equation*}
      x_1, x_2, x_3, \ldots
    \end{equation*}

    A subnet needs not be a sequence. For example, consider the \hyperref[def:totally_ordered_set]{totally ordered set} \( \BbbR_{>0} \) of positive \hyperref[def:real_numbers]{real numbers}. For every positive real number \( r \), let \( k_r \coloneqq \ceil(r) \) be the smallest integer not smaller than \( r \), and define the net \( y_r \coloneqq x_{k_r} \).

    Then \ref{def:subnet/composition} is automatically satisfied.

    Fix some positive integer \( k_0 \). Then we have an index \( r_0 \coloneqq k_0 \) such that, for \( r \geq r_0 \),
    \begin{equation*}
      k_r = \ceil(r) \geq \ceil(r_0) = r_0 = k_0.
    \end{equation*}

    Therefore, \ref{def:subnet/cofinality} holds and \( \seq{ y_r }_{r \in \BbbR} \) is a subnet.

    \thmitem{ex:def:subnet/divisibility_subnet} Furthermore, a subnet of a sequence needs not even be totally ordered. Given some sequence, if we order its index set by divisibility\footnote{The natural number divisibility lattice is described in \fullref{thm:natural_number_divisibility_lattice}.} rather than by the conventional order, we will obtain a subnet of the original.

    Indeed, for every positive integer \( k_0 \), we have an index \( m_0 \coloneqq k_0 \) such that, for \( m_0 \mid m \), we have \( m \geq k_0 \).
  \end{thmenum}
\end{example}

\subsubsection{Cluster points}

We will now work in an ambient topological space \( (X, \mscrT) \).

\begin{definition}\label{def:net_cluster_point}\mcite[71]{Kelley1975}
  We say that the point \( x_0 \) is an \term{cluster point} or \term{accumulation point} of the net \( \seq{ x_k }_{k \in \mscrK} \) if it is \hyperref[def:net_frequently_in]{frequently in} every neighborhood of \( x_0 \).
\end{definition}
\begin{comments}
  \item \Fullref{thm:net_cluster_point_base} shows that it is sufficient to only consider bases.
  \item This is related to a similar notion for subsets, \fullref{def:set_cluster_point}, via \fullref{thm:closed_iff_contains_all_net_cluster_points}.
\end{comments}

\begin{example}\label{ex:def:net_cluster_point}
  We give several examples of \hyperref[def:net_cluster_point]{net cluster points}:

  \begin{thmenum}
    \thmitem{ex:def:net_cluster_point/empty} \Fullref{thm:empty_index_set_implies_frequently_in} implies that the empty net \( \seq{ x_k }_{k \in \varnothing} \) is frequently in every set, including every neighborhood of every point.

    Therefore, every point is a cluster point of the empty net.

    \thmitem{ex:def:net_cluster_point/finite_sequence} As discussed in \fullref{ex:def:topological_net/finite_sequence}, any finite sequence \( x_1, x_2, \ldots, x_n \) is frequently in \( \set{ x_n } \) and its supersets, hence also in every neighborhood of \( x_n \).

    The last element of a finite sequence is thus a cluster point. Furthermore, it is the only cluster point of the sequence.

    \thmitem{ex:def:net_cluster_point/sequence} An easy example of a sequence with two cluster points is
    \begin{equation*}
      0, 1, 0, 1, 0, 1, 0, 1, \cdots
    \end{equation*}

    It is frequently in \( \set{ 0 } \) and in \( \set{ 1 } \), hence in all of their neighborhoods (independent of the topology). Thus, both \( 0 \) and \( 1 \) are cluster points.

    \thmitem{ex:def:net_cluster_point/sierpinski} Consider the \hyperref[def:sierpinski_space]{Sierpinski space} \( \set{ \top, \bot } \). The constant sequence
    \begin{equation*}
      \top, \top, \top, \top, \top, \top
    \end{equation*}
    obviously has \( \top \) as a cluster point since it is frequently in the neighborhoods \( \set{ \top } \) and \( \set{ \top, \bot } \) of \( \top \).

    Furthermore, since \( \set{ \top, \bot } \) is the only neighborhood of \( \bot \), the latter is also a cluster point of the sequence (despite not belonging to it).

    \thmitem{ex:def:net_cluster_point/natural_divisibility} Consider the sequence
    \begin{equation*}
      p_1, p_2, p_3, \cdots,
    \end{equation*}
    where \( p_k \) is the smallest \hyperref[def:prime_number]{prime number} that divides \( k \). For completeness, let \( p_1 \coloneqq 1 \).

    The first terms are
    \begin{equation*}
      1, 2, 3, 2, 5, 2, 7, 2, 3, 2, 11, 2, 13, 2, 3, \ldots
    \end{equation*}

    The sequence is frequently in \( \set{ 2 } \) since, for any number \( n \), \( 2n \) is a strictly larger even number. Analogously, it is frequently in \( \set{ p } \) for every prime \( p \) because, for any number \( n \), \( pn \) is a strictly larger number divisible by \( p \).

    Therefore, under any topology, the prime numbers are cluster points, and no other numbers except \( 0 \) and \( 1 \) occur.

    \thmitem{ex:def:net_cluster_point/sin} Consider the net \( \seq{ \sin(r) }_{r \in \BbbR} \).

    Every real number in the interval \( [-1, 1] \) is a cluster point under any topology. Indeed, fix \( x_0 \in [-1, 1] \).

    For any index \( r_0 \in \BbbR \), let \( n_0 \) be the smallest integer such that \( r_0 \leq 2 n_0 \pi \). Then
    \begin{equation*}
      \sin(r_0) = \sin(\underbrace{2 n_0 \pi + \arcsin(x_0)}_r).
    \end{equation*}

    Thus, \( r \geq r_0 \) and \( \sin(r) = x_0 \), meaning that the net \( \seq{ \sin(r) }_{r \in \BbbR} \) is frequently in \( \set{ x_0 } \) (and hence any neighborhood of \( x_0 \)).

    Therefore, every number in the interval \( [-1, 1] \) is a cluster point for the net.
  \end{thmenum}
\end{example}

\begin{proposition}\label{thm:net_cluster_point_base}
  Fix some point \( x_0 \) in a topological space and let \( \mscrB(x) \) be a \hyperref[def:topological_local_base]{local base} at \( x_0 \).

  Then \( x_0 \) is a \hyperref[def:net_cluster_point]{cluster point} of the net \( \seq{ x_k }_{k \in \mscrK} \) if and only if it is \hyperref[def:net_frequently_in]{frequently in} every set of \( \mscrB(x) \).
\end{proposition}
\begin{proof}
  \SufficiencySubProof Every set in \( \mscrB(x) \) is itself a neighborhood of \( x_0 \) by \ref{thm:topology_from_local_base/BP1}, hence if \( x_0 \) is a cluster point, the net is frequently in every set from \( \mscrB \) containing \( x_0 \).

  \NecessitySubProof Suppose that the net is frequently in every set in \( \mscrB(x) \).

  Let \( U \) be a neighborhood of \( x_0 \). By definition of a local base, there exists some set \( V \) in \( \mscrB(x) \) such that
  \begin{equation*}
    x_0 \in V \subseteq U.
  \end{equation*}

  For every index \( k_0 \in \mscrK \) then there exists some \( k \geq k_0 \) such that \( x_k \in V \subseteq U \).

  Since \( U \) was arbitrary, we conclude that \( x_0 \) is a cluster point of the net.
\end{proof}

\begin{proposition}\label{thm:cluster_point_iff_in_closure}\mcite[prop. 1.6.3]{Engelking1989}
  A point \( x_0 \) belongs to the closure of the set \( A \) if and only if there exists a net \( \seq{ x_k }_{k \in \mscrK} \) of members of \( A \) for which \( x_0 \) is a cluster point.
\end{proposition}
\begin{proof}
  The statement holds vacuously if \( A \) is the empty set. Assume that \( A \) is nonempty.

  \SufficiencySubProof Suppose that \( x_0 \) is in the closure of \( A \).
  \begin{itemize}
    \item If \( x_0 \) belongs to \( A \) itself, then the singleton net \( (x_0) \) has \( x_0 \) as a cluster point.
    \item Otherwise, \fullref{thm:union_with_derived_set} implies that \( x_0 \) belongs to the boundary of \( A \). \Fullref{thm:properties_via_bases/boundary} then implies that, for every neighborhood \( U \) of \( x_0 \), there exists some point \( x_U \) from \( A \). We can thus build a reverse inclusion net \( \seq{ x_U }_{U \in \mscrT(x_0)} \) in the style of \fullref{ex:def:topological_net/reverse} that is frequently in every neighborhood of \( x_0 \). The point \( x_0 \) is then a cluster point of this net.
  \end{itemize}

  In both cases, we have constructed a net with elements of \( A \) for which \( x_0 \) is a cluster point.

  \NecessitySubProof Let \( x_0 \) be a cluster point of \( \seq{ x_k }_{k \in \mscrK} \).

  Let \( F \) be any closed superset of \( A \). We will show that \( x_0 \) belongs to \( F \), thus allowing us to conclude that \( x_0 \) is in the closure of \( A \).

  Aiming at a contradiction, suppose that \( x_0 \) belongs to \( X \setminus F \). Note that the latter is then a neighborhood of \( x_0 \).

  Since \( x_0 \) is a cluster point of \( \seq{ x_k }_{k \in \mscrK} \), there exists some index \( k_0 \in \mscrK \) such that \( x_{k_0} \) belongs to \( X \setminus F \). But the net consists of members of \( A \), hence \( x_0 \) belongs to
  \begin{equation*}
    A \cap (X \setminus F) \subseteq F \cap (X \setminus F) = \varnothing.
  \end{equation*}

  The obtained contradiction shows that \( x_0 \) belongs to every closed superset of \( A \), hence also to their intersection --- the closure of \( A \).
\end{proof}

\begin{corollary}\label{thm:closed_iff_contains_all_net_cluster_points}
  A set is closed if and only if it contains the cluster points of all of its nets.
\end{corollary}
\begin{proof}
  By \fullref{thm:cluster_point_characterization}, the set \( A \) is closed if and only if it contains all of its cluster points in the sense of \fullref{def:set_cluster_point}. By \fullref{thm:cluster_point_iff_in_closure}, this is equivalent to \( A \) containing all cluster points of its nets in the sense of \fullref{def:net_cluster_point}.
\end{proof}

\subsubsection{Limit points}

Assume again that we work in an ambient topological space \( X \).

\begin{definition}\label{def:net_limit_point}\mcite[66]{Kelley1975}
  We say that the point \( x_0 \) is a \term{limit point} of the net \( \seq{ x_k }_{k \in \mscrK} \) if it is \hyperref[def:net_eventually_in]{eventually in} every neighborhood of \( x_0 \).

  We say that a net is \term{convergent} if it has a limit point and \term{divergent} otherwise.
\end{definition}
\begin{comments}
  \item \Fullref{thm:net_limit_point_subbase} shows that it is sufficient to only consider subbases.

  \item In general, there can exist multiple limit points --- see \fullref{ex:def:net_limit_point}. In Hausdorff spaces, however, limits are unique as shown in \fullref{thm:t2_iff_singleton_limits}.
\end{comments}

\begin{example}\label{ex:def:net_limit_point}
  We give several examples of \hyperref[def:net_limit_point]{net limit points}:

  \begin{thmenum}
    \thmitem{ex:def:net_limit_point/empty} The empty net cannot have limit points as a consequence of \fullref{thm:eventually_in_implies_nonempty}.

    \thmitem{ex:def:net_limit_point/finite_sequence} As in the case of cluster points in \fullref{ex:def:net_cluster_point/finite_sequence}, a finite sequence \( x_1, \ldots, x_n \) can have only one limit point --- \( x_n \).

    \thmitem{ex:def:net_limit_point/sin} We saw in \fullref{ex:def:net_cluster_point/sin} that the net \( \seq{ \sin(r) }_{r \in \BbbR} \) has a lot of cluster points. It does not have a limit point, however.

    Indeed, suppose that \( x_0 \) is a limit point. Fix some neighborhood \( U \) of \( x_0 \) in the topological space \( [-1, 1] \) that is not the whole space.

    There exists some \( r_0 \) such that \( \sin(r) \in U \) for \( r \geq r_0 \). But
    \begin{equation*}
      \sin([r_0, r_0 + 2\pi)) = [-1, 1]
    \end{equation*}
    and we have explicitly assumed that \( U \) is a strict subset of \( [-1, 1] \).

    Therefore, no point in \( [-1, 1] \) is a limit point of the net \( \seq{ \sin(r) }_{r \in \BbbR} \).

    \thmitem{ex:def:net_limit_point/sierpinski_constant} As in \fullref{ex:def:net_cluster_point/sierpinski}, consider the \hyperref[def:sierpinski_space]{Sierpinski space} \( \set{ \top, \bot } \) and the constant sequence
    \begin{equation*}
      \top, \top, \top, \top, \top, \top.
    \end{equation*}

    Both \( \top \) and \( \bot \) are cluster points, but they are also limit points. They are distinct and even \hyperref[def:topologically_indistinguishable]{topologically distinguishable}.

    \thmitem{ex:def:net_limit_point/sierpinski_alternating} Consider instead the Sierpinski space with the alternating sequence
    \begin{equation*}
      \top, \bot, \top, \bot, \top, \bot.
    \end{equation*}

    Both \( \top \) and \( \bot \) are cluster points. Furthermore, \( \bot \) is a limit point, while \( \top \) is not.

    \thmitem{ex:def:net_limit_point/multiple} Limits of sequences need not be unique in general topological spaces. Let \( X = \set{ a, b } \) be a binary set with the \hyperref[def:indiscrete_topology]{indiscrete topology} \( \set{ \varnothing, X } \).

    Define the following \hyperref[def:sequence]{sequence}
    \begin{equation*}
      x_k \coloneqq \begin{cases}
        a, &k \T{is even}, \\
        b, &k \T{is odd}.
      \end{cases}
    \end{equation*}

    The only neighborhood of \( y \), the whole space \( X \), contains all members of the sequence, therefore \( y \) is a limit point of the sequence. The same is true for \( z \), however.
  \end{thmenum}
\end{example}

\begin{proposition}\label{thm:def:net_limit_point}
  \hyperref[def:net_limit_point]{Net limit points} have the following basic properties:

  \begin{thmenum}
    \thmitem{thm:def:net_limit_point/limit_point_is_cluster_point} Every limit point is a cluster point.

    The converse is true for converging nets in \hyperref[def:separation_axioms/T2]{Hausdorff spaces}, but not in general --- see \fullref{ex:def:net_limit_point/sierpinski_alternating}.

    \thmitem{thm:def:net_limit_point/cluster_point_in_subnet} Every cluster point of a subnet is also a cluster point of the entire net.

    \thmitem{thm:def:net_limit_point/limit_and_subnet_limit} A point is a limit point of a net if and only if it is a limit point of all \hyperref[def:subnet]{subnets}.

    \thmitem{thm:def:net_limit_point/cluster_point_subnet_limit} A point is a cluster point of a net if and only if it is a limit point of some subnet.

    \thmitem{thm:def:net_limit_point/cluster_and_limit_point} Given a limit point and a cluster point of the same net, every neighborhood of the first intersects every neighborhood of the second and vice versa.

    In a general topological space, the two points may still be \hyperref[def:topologically_indistinguishable]{topologically distinguishable} as shown in \fullref{ex:def:net_limit_point/sierpinski_constant}.
  \end{thmenum}
\end{proposition}
\begin{proof}
  \SubProofOf{thm:def:net_limit_point/limit_point_is_cluster_point} Follows from \fullref{thm:eventually_in_implies_frequently_in}.

  \SubProofOf{thm:def:net_limit_point/cluster_point_in_subnet} Let \( \seq{ y_m }_{k \in \mscrK} \) be a subnet of \( \seq{ x_k }_{k \in \mscrK} \) with index translation \( \seq{ k_m }_{m \in \mscrM} \). Suppose that \( y_0 \) is a cluster point of \( \seq{ y_m }_{k \in \mscrK} \).

  Let \( U \) be a neighborhood of \( y_0 \) and fix some index \( k_0 \in \mscrK \). \ref{def:subnet/cofinality} implies that there exists some index \( m_0 \) such that \( k_m \geq k_0 \) for \( m \geq m_0 \).

  Since \( y_0 \) is a cluster point of \( \seq{ k_m }_{m \in \mscrM} \), there exists some index \( m' \geq m_0 \) such that the point \( y_{m'} = x_{k_m'} \) belongs to \( U \). Since we have \( k_m' \geq k_0 \), we conclude that the entire net \( \seq{ x_k }_{k \in \mscrK} \) is frequently in \( U \).

  The neighborhood \( U \) was chosen arbitrarily, hence we conclude that \( y_0 \) is a cluster point of \( \seq{ x_k }_{k \in \mscrK} \).

  \SubProofOf{thm:def:net_limit_point/limit_and_subnet_limit}

  \SufficiencySubProof* Let \( x_0 \) be a limit point of \( \seq{ x_k }_{k \in \mscrK} \) and let \( \seq{ y_m }_{m \in \mscrM} \) be a subnet with index translation \( \seq{ k_m }_{m \in \mscrM} \).

  Let \( U \) be a neighborhood of \( x_0 \). Then there exists an index \( k_0 \) such that \( x_k \in U \) for \( k \geq k_0 \). Since \( \seq{ y_m }_{m \in \mscrM} \) is a subnet, \ref{def:subnet/cofinality} states that there exists some index \( m_0 \) such that \( k_m \geq k_0 \) for \( m \geq m_0 \).

  We showed that the subnet is eventually in an arbitrary neighborhood of \( x_0 \), implying that \( x_0 \) is a limit point of the subnet.

  \NecessitySubProof* Suppose that \( x_0 \) is a limit point of all subnets of \( \seq{ x_k }_{k \in \mscrK} \). Every net is a subnet of itself, hence \( x_0 \) is a limit point of \( \seq{ x_k }_{k \in \mscrK} \).

  \SubProofOf{thm:def:net_limit_point/cluster_point_subnet_limit}

  \SufficiencySubProof* Let \( x_0 \) be a cluster point of the net \( \seq{ x_k }_{k \in \mscrK} \).

  For every neighborhood \( U \) of \( x_0 \), there exists some index \( k_U \) such that \( x_{k_U} \in U \).

  Consider the \hyperref[ex:def:topological_net/reverse]{dual filter} \( (\mscrT(x_0), \supseteq) \). For every concrete neighborhood \( U_0 \), there exists some index \( k_{U_0} \) such that \( x_{k_V} \in U_0 \) whenever the neighborhood \( V \) is a subset of \( U_0 \), that is, whenever \( V \geq U_0 \).

  Therefore, the net \( \seq{ k_U }_{U \in \mscrT(x_0)} \) is a subnet of \( \seq{ x_k }_{k \in \mscrK} \).

  Furthermore, it converges to \( x_0 \) because it is eventually in each neighborhood of \( x_0 \).

  \NecessitySubProof* Follows from \fullref{thm:def:net_limit_point/limit_point_is_cluster_point} and \fullref{thm:def:net_limit_point/cluster_point_in_subnet}.

  \SubProofOf{thm:def:net_limit_point/cluster_and_limit_point} Let \( l \) be a limit point of \( \seq{ x_k }_{k \in \mscrK} \) and let \( c \) be a cluster point.

  Let \( U \) be a neighborhood of \( l \). Then there exists an index \( k_l \) such that \( x_k \in U \) for \( k > k_l \).

  Let \( V \) be a neighborhood of \( c \). Then there exists an index \( k_c \geq k_l \) such that \( x_{k_c} \in V \). Furthermore, \( x_{k_c} \in U \cap V \).

  Therefore, \( U \) and \( V \) intersect. Since they are arbitrary neighborhoods, we conclude that the proposition holds.
\end{proof}

\begin{proposition}\label{thm:net_limit_point_subbase}
  Fix some point \( x_0 \) in a topological space and let \( \mscrS(x) \) be a \hyperref[def:topological_subbase]{subbase} at \( x_0 \).

  Then \( x_0 \) is a \hyperref[def:net_limit_point]{limit point} of the net \( \seq{ x_k }_{k \in \mscrK} \) if and only if it is \hyperref[def:net_eventually_in]{eventually in} every set of \( \mscrS(x) \).
\end{proposition}
\begin{proof}
  \SufficiencySubProof Every set in \( \mscrS \) is itself a neighborhood of \( x_0 \), hence if \( x_0 \) is a limit point, the net is eventually in every set from \( \mscrS \) containing \( x_0 \).

  \NecessitySubProof Suppose that the net is eventually in every set in \( \mscrS \) containing \( x_0 \).

  Let \( U \) be a neighborhood of \( x_0 \). By definition of local subbase, there exists some family \( V_1, \ldots, V_n \) in \( \mscrS \) such that
  \begin{equation*}
    x_0 \in V_1 \cap \cdots \cap V_n \subseteq U.
  \end{equation*}

  For every \( i = 1, \ldots, n \), there exists some \( k_i \in \mscrK \) such that \( k \geq k_i \) such that \( x_k \in V_i \subseteq U \).

  Since \( \mscrK \) is a directed set, there exists some upper bound \( k_0 \) for \( k_1, \ldots, k_n \). Then, if \( k \geq k_0 \), we have
  \begin{equation*}
    x_0 \in V_1 \cap \cdots \cap V_n \subseteq U.
  \end{equation*}

  Since \( U \) was arbitrary, we conclude that \( x_0 \) is a limit point of the net.
\end{proof}

\begin{proposition}\label{thm:first_countable_space_limit_points}
  Let \( X \) be a \hyperref[def:topological_space_character]{first-countable space} and let \( \seq{ x_k }_{k \in \mscrK} \) be a net in \( X \). Suppose that \( x_0 \) is a \hyperref[def:net_cluster_point]{cluster point} of the net.

  Then there exists an \hyperref[def:order_homomorphism/increasing]{order-preserving} sequence of indices
  \begin{equation*}
    k_1 \leq_\mscrK k_2 \leq_\mscrK k_3 \leq_\mscrK k_4 \leq_\mscrK \cdots
  \end{equation*}
  from \( K \) such that \( x_0 \) is a \hyperref[def:net_limit_point]{limit point} of the \hyperref[def:sequence]{sequence}
  \begin{equation*}
    x_{k_1}, x_{k_2}, x_{k_3}, x_{k_4}, \ldots.
  \end{equation*}
\end{proposition}
\begin{comments}
  \item First-countable spaces, such as \hyperref[def:metric_space]{metric spaces}, allow restricting ourselves to only \hyperref[def:sequence]{sequences} rather than arbitrary \hyperref[def:topological_net]{nets}.
  \item If \( \mscrK \) is a \hyperref[def:totally_ordered_set]{totally ordered set}, this sequence is a \hyperref[def:subnet]{subnet}.
\end{comments}
\begin{proof}
  Consider any countable neighborhood basis \( U_1, U_2, U_3, \ldots \) of \( x_0 \). Define the following sequence of sets:
  \begin{equation*}
    V_m \coloneqq \cap_{i=1}^m U_i.
  \end{equation*}

  Then we have the reverse inclusion chain
  \begin{equation*}
    V_1 \supseteq V_2 \supseteq V_3 \supseteq \cdots
  \end{equation*}

  Every element of this chain is a neighborhood of \( x_0 \). Fix an arbitrary index \( k_0 \in K \) --- the concrete choice does not matter; it is only a technicality. Since \( x_0 \) is a cluster point of the net \( \seq{ x_k }_{k \in \mscrK} \), there exists some index \( k_1 \geq k_0 \) such that \( x_{k_1} \in V_1 \).

  For \( i > 1 \), we similarly pick an index such that \( k_i \geq k_{i-1} \) and \( x_{k_i} \in V_i \).

  It remains to show that \( x_0 \) is a cluster point of the obtained sequence
  \begin{equation*}
    x_{k_1}, x_{k_2}, x_{k_3}, x_{k_4}, \ldots
  \end{equation*}

  Let \( W \) be an arbitrary neighborhood of \( x_0 \). By definition of local filter, there exists an index \( j_W \) such that \( U_{j_W} \subseteq W \). Then, whenever \( j \geq j_W \),
  \begin{equation*}
    x_{k_j} \in V_j = U_j \cap U_{j_W} \cap \cdots \subseteq U_{j_W} \subseteq W.
  \end{equation*}

  Thus, the sequence is eventually in \( W \). Generalizing on \( W \), we conclude that \( x_0 \) is a limit point of the sequence.
\end{proof}
