\subsection{Ordered set completion}\label{subsec:ordered_set_completion}

\paragraph{Dedekind-MacNeille completion}

\begin{definition}\label{def:dedekind_macnielle_closure}\mimprovised
  Let \( (P, \leq) \) be a \hyperref[def:partially_ordered_set]{partially ordered set}. For a given subset \( A \) of \( P \), define the following sets of all \hyperref[def:extremal_points/bounds]{upper bounds} and of all \hyperref[def:extremal_points/bounds]{lower bounds} of \( A \):
  \begin{align*}
    A^U \coloneqq \overbrace{\bigcap_{a \in A} P_{\geq a}}^{\mathclap{\set{ x \in L \given \qforall {a \in A} x \leq a }}}
    &&
    A^L \coloneqq \overbrace{\bigcap_{a \in A} P_{\leq a}}^{\mathclap{\set{ x \in L \given \qforall {a \in A} a \leq x }}}
  \end{align*}

  Then \( A \mapsto A^{UL} \) is a \hyperref[def:moore_closure_operator]{Moore closure operator}. We call \( A^{UL} \) the \term{Dedekind-MacNaille closure} of \( A \).

  \begin{figure}[!ht]
    \centering
    \includegraphics{output/def__dedekind_macnielle_closure}
    \caption{The \hyperref[def:dedekind_macnielle_completion]{Dedekind-MacNeille closure} of \( \set{ b, c } \) is \( \set{ a, b, c } \).}\label{fig:def:dedekind_macnielle_closure}
  \end{figure}
\end{definition}
\begin{defproof}
  \SubProofOf[def:extensive_function]{extensiveness} Fix a subset \( A \) of \( P \) and a member \( x \) of \( A \).

  For every \( u \in A^U \), by definition of upper bound we have \( x \leq u \). Then \( x \) is a lower limit of \( A^U \), hence \( x \in A^{UL} \).

  Generalizing on \( x \in A \), we conclude that \( A \subseteq A^{UL} \).

  \SubProofOf[def:idempotent_function]{idempotence} Fix again a subset \( A \) of \( P \).

  We have shown that \( B \subseteq B^{UL} \) for any subset \( B \) of \( P \), in particular for \( B = A \). Dually, we can show that \( B \subseteq B^{LU} \), in particular for \( B = A^U \). Furthermore, \( A \subseteq A^{UL} \) implies \( A^U \supseteq A^{ULU} \) because an upper bound for a set is necessarily an upper bound for a subset.

  Thus,
  \begin{equation*}
    A^U \subseteq A^{ULU} \subseteq A^U.
  \end{equation*}

  Therefore, \( A^U = A^{ULU} \) and \( A^{UL} = A^{ULUL} \).

  \SubProofOf[def:order_function/preserving]{monotonicity} Fix subsets \( A \subseteq B \) of \( P \) and a member \( a \) of \( A^{UL} \). We will show that \( a \) is also a member of \( B^{UL} \).

  First note that \( B^U \subseteq A^U \) because an upper limit of \( B \) is necessarily also an upper limit of \( A \).

  For every member \( u \) of \( A^U \), we have \( a \leq u \), in particular for those \( u \) that belong to \( B^U \). Therefore, \( a \) is a lower bound of \( B^U \), that is, a member of \( B^{UL} \).
\end{defproof}

\begin{proposition}\label{thm:def:dedekind_macnielle_closure}
  \hyperref[def:dedekind_macnielle_completion]{Dedekind-MacNeille closures} have the following basic properties:
  \begin{thmenum}
    \thmitem{thm:def:dedekind_macnielle_closure/downward_closed} The Dedekind-MacNeille closure of any set is \hyperref[def:closed_ordered_subset]{downward closed}.

    \thmitem{thm:def:dedekind_macnielle_closure/ideal} The Dedekind-MacNeille closure of any nonempty set in a lattice is a \hyperref[def:lattice_ideal]{lattice ideal}.

    \thmitem{thm:def:dedekind_macnielle_closure/point_in_closure} In a \hyperref[def:totally_ordered_set]{totally ordered set}, every point \( x \) in the closure \( A^{UL} \) of a \hyperref[def:closed_ordered_subset]{downward closed} set \( A \), either \( x \) is in \( A \) or is the supremum of \( A \).

    \thmitem{thm:def:dedekind_macnielle_closure/totally_ordered} In a totally ordered set, if a subset \( A \) is downward closed and, in case \( A \) has a supremum, it contains its supremum, then \( A \) is Dedekind-MacNeille closed.
  \end{thmenum}
\end{proposition}
\begin{proof}
  \SubProofOf{thm:def:dedekind_macnielle_closure/downward_closed} If \( x \in A^{UL} \) and \( y \leq x \), then \( y \leq u \) for every upper bound \( u \) of \( A \). Thus, \( y \) is a lower bound of \( A^U \), i.e. it belongs to \( A^{UL} \).

  Therefore, \( A^{UL} \) is downward closed.

  \SubProofOf{thm:def:dedekind_macnielle_closure/ideal} Let \( A \) be an arbitrary subset of the lattice \( X \). \Fullref{thm:def:dedekind_macnielle_closure/downward_closed} shows that \( A^{UL} \) is downward closed. We will show that \( A^{UL} \) is upward directed. Then from \fullref{thm:def:lattice_ideal/directed_and_closed} it will follow that \( A^{UL} \) is a lattice ideal.

  For any two elements \( x \) and \( y \) of \( A^{UL} \), their supremum \( x \wedge y \) is an upper bound of \( \set{ x, y } \subseteq A^{UL} \). For any upper bound \( u \) of \( A \), we have \( x \wedge y \leq u \), hence \( x \wedge y \) is a lower bound of \( A^{ULU} \). Then \( x \wedge y \) belongs to \( A^{ULUL} = A^{UL} \), demonstrating that \( A^{UL} \) is upward directed.

  \SubProofOf{thm:def:dedekind_macnielle_closure/point_in_closure} Let \( P \) be a totally ordered set. Fix some downward closed subset \( A \) of \( P \).

  The case where \( A \) is empty is vacuous --- every element of \( P \) is an upper bound and so \( A^{UL} \) is either empty or contains the bottom element of \( P \) (which is the supremum of the empty set \( A \)).

  Suppose that \( A \) is not empty and let \( x \) be a point in \( A^{UL} \). We have the following possibilities:
  \begin{itemize}
    \item There exists a point \( y \) in \( A \) such that \( x \leq y \). Then \( x \) is itself in \( A \) because the latter is downward closed by assumption.
    \item Otherwise, \( x \) is an upper bound of \( A \). Let \( u \) be any upper bound of \( A \). Then \( z \leq u \) for any element \( z \) of \( A^{UL} \), in particular \( x \leq u \). Therefore, \( x \) is the least upper bound of \( A \).
  \end{itemize}

  \SubProofOf{thm:def:dedekind_macnielle_closure/totally_ordered} Follows from \fullref{thm:def:dedekind_macnielle_closure/point_in_closure}.
\end{proof}

\begin{definition}\label{def:dedekind_macnielle_completion}\mcite[166]{DaveyPriestley2002}
  Let \( (P, \leq) \) be a \hyperref[def:partially_ordered_set]{partially ordered set}. Using the closure operator from \fullref{def:dedekind_macnielle_closure}, we define the \term{Dedekind-MacNeille completion} of \( P \) as the family
  \begin{equation*}
    D(P) \coloneqq \set{ A \subseteq P \given A = A^{UL} }
  \end{equation*}
  of \hyperref[def:dedekind_macnielle_closure]{Dedekind-MacNeille-closed sets} ordered via \hyperref[def:subset]{set inclusion}.

  The set \( P \) itself can be embedded via the \hyperref[def:order_function/preserving]{order-preserving map} which sends each member of \( P \) to its \hyperref[def:lattice_ideal/principal]{principal ideal}, that is,
  \begin{equation*}
    \begin{aligned}
      &\iota: P \to D(P), \\
      &\iota(x) \coloneqq P_{\leq x}.
    \end{aligned}
  \end{equation*}
\end{definition}
\begin{comments}
  \item Note that \( P_{\leq x} = \set{ x }^{UL} \).
  \item Joins may misbehave without the closure provided by \( A \mapsto A^{UL} \) --- see \fullref{ex:dedekind_macnielle_join_closure}.
\end{comments}

\begin{theorem}[Dedekind-MacNeilla completion]\label{thm:def:dedekind_macnielle_completion}
  The \hyperref[def:dedekind_macnielle_completion]{Dedekind-MacNeille completion} \( D(P) \) of a partially ordered set \( (P, \leq) \) is a \hyperref[def:lattice]{complete lattice} with the following operations:
  \begin{align*}
    \bigvee \mscrA   &\coloneqq \parens[\Big]{ \bigcup \mscrA }^{UL} \\
    \bigwedge \mscrA &\coloneqq \bigcap \mscrA
  \end{align*}

  Furthermore, joins and meets from \( P \) (whenever they exist) are preserved in \( D(P) \).
\end{theorem}
\begin{proof}
  It is obvious that \( D(P) \) is partially ordered by \( \subseteq \).

  \SubProof{Proof that \( D(P) \) is join-complete} Let \( \mscrA \) be a family of members of \( D(P) \). Then, since we have shown idempotence in \fullref{def:dedekind_macnielle_closure}, we have
  \begin{equation*}
    \parens[\Big]{ \bigvee \mscrA }^{UL}
    =
    \parens[\Big]{ \bigcup \mscrA }^{ULUL}
    =
    \parens[\Big]{ \bigcup \mscrA }^{UL}
    =
    \bigvee \mscrA.
  \end{equation*}

  Therefore, the join \( \bigvee \mscrA \) of members of \( P \) also belongs to \( D(P) \).

  \SubProof{Proof that \( D(P) \) is meet-complete} For any family \( \mscrB \) of members of \( D(P) \), we have
  \begin{align*}
    \parens[\Big]{ \bigcap \mscrB }^U
    &=
    \set[\Big]{ u \in P \given* \qforall {x \in \bigcap \mscrB} x \leq u }
    = \\ &=
    \set[\Big]{ u \in P \given* \qforall {B \in \mscrB} \qforall {x \in B} x \leq u }
    = \\ &=
    \bigcap_{B \in \mscrB} B^U
  \end{align*}
  and similarly
  \begin{equation*}
    \parens[\Big]{ \bigcap \mscrB }^L
    =
    \bigcap_{B \in \mscrB} B^L.
  \end{equation*}

  Therefore, for a fixed family \( \mscrA \subseteq D(P) \), we have
  \begin{equation*}
    \parens[\Big]{ \bigwedge \mscrA }^{UL}
    =
    \parens[\Big]{ \bigcap \mscrA }^{UL}
    =
    \parens[\Big]{ \bigcap \mscrA }^U
    =
    \bigcap \mscrA.
  \end{equation*}

  Therefore, the meet \( \bigwedge \mscrA \) is also a member of \( D(P) \).

  \SubProof{Proof that \( D(P) \) preserves joins} Let \( A \) be a set in \( P \) such that \( a_0 \coloneqq \bigvee^P A \) exists.

  Then
  \begin{equation*}
    \bigvee\nolimits^{D(P)} \set{ \iota(a) \given a \in A }
    =
    \bigvee\nolimits^{D(P)} \set{ P_{\leq a} \given a \in A }
    =
    \parens[\Big]{ \bigcup_{a \in A} P_{\leq a} }^{UL}
    =
    \parens[\Big]{ P_{\geq a_0} }^L
    =
    P_{\leq a_0}
    =
    \iota(a_0).
  \end{equation*}

  \SubProof{Proof that \( D(P) \) preserves meets} Let \( A \) be a set in \( P \) such that \( a_0 \coloneqq \bigwedge^P A \) exists.

  Then
  \begin{equation*}
    \bigwedge\nolimits^{D(P)} \set{ \iota(a) \given a \in A }
    =
    \bigwedge\nolimits^{D(P)} \set{ P_{\leq a} \given a \in A }
    =
    \bigcap_{a \in A} P_{\leq a}
    =
    P_{\leq a_0}
    =
    \iota(a_0).
  \end{equation*}
\end{proof}

\begin{example}\label{ex:dedekind_macnielle_join_closure}
  Taking the closure of the union in the definition of join for \hyperref[def:dedekind_macnielle_completion]{Dedekind-MacNeille completions} can be justified as follows.

  Consider the sequence of \hyperref[def:rational_numbers]{rational numbers}
  \begin{equation*}
    a_n \coloneqq -\frac 1 n.
  \end{equation*}

  The supremum of this sequence is zero --- it is clearly an upper bound and, for any negative rational number \( -r \), we have a larger negative number from the sequence:
  \begin{equation*}
    -r
    =
    -\frac 1 {1 / r}
    \leq
    -\frac 1 {\floor(1 / r)}
    <
    0.
  \end{equation*}

  Now consider the completion \( D(\BbbQ) \) and take the embeddings of \( \seq{ a_n }_{n=1}^\infty \), the sequence
  \begin{equation*}
    a_n' \coloneqq \iota(a_n) = \set*{ x \in \BbbQ \given* x \leq -\frac 1 n }.
  \end{equation*}

  Then
  \begin{equation*}
    \bigcup a_n' = \set{ x \in \BbbQ \given x < 0 },
  \end{equation*}
  but this is not an upper bound because the set does not itself belong to \( D(\BbbQ) \).

  Taking the closure, however, we obtain
  \begin{equation*}
    \parens[\Big]{ \bigcup a_n' }^{UL}
    =
    \set{ x \in \BbbQ \given x \geq 0 }^L
    =
    \set{ x \in \BbbQ \given x \leq 0 }
    =
    \bigcup a_n' \cup \set{ 0 }
    =
    \iota(0),
  \end{equation*}
  which is actually the supremum of \( \seq{ a_n' }_{n=1}^\infty \) in \( D(\BbbQ) \).
\end{example}

\begin{theorem}[Dedekind-MacNeille completion universal property]\label{thm:dedekind_macneille_completion_universal_property}
  The \hyperref[def:dedekind_macnielle_completion]{Dedekind-MacNeille completion} \( D(P) \) of a \hyperref[def:partially_ordered_set]{partially ordered set} \( P \) satisfies the following \hyperref[rem:universal_mapping_property]{universal mapping property}:
  \begin{displayquote}
    For every \hyperref[def:lattice]{complete lattice} \( L \), every \hyperref[def:order_function/preserving]{order-preserving map} \( \varphi: P \to L \) \hyperref[def:factors_through]{uniquely factors through} \( D(P) \). More precisely, there exists a unique \hyperref[def:lattice/homomorphism]{lattice homomorphism} \( \widetilde{\varphi}: D(P) \to L \) such that the following diagram commutes:
    \begin{equation}\label{eq:thm:dedekind_macneille_completion_universal_property/diagram}
      \begin{aligned}
        \includegraphics[page=1]{output/thm__dedekind_macneille_completion_universal_property}
      \end{aligned}
    \end{equation}
  \end{displayquote}
\end{theorem}
\begin{comments}
  \item Via \fullref{rem:universal_mapping_property}, this functor becomes \hyperref[def:category_adjunction]{left adjoint} to the \hyperref[def:concrete_category]{forgetful functor} from complete lattices to partially ordered sets.
\end{comments}
\begin{proof}
  In order for a homomorphism \( \widetilde{\varphi} \) to satisfy the theorem, it must satisfy
  \begin{equation*}
    \widetilde{\varphi}(\iota(x)) = \varphi(x).
  \end{equation*}

  Thus, \( \widetilde{\varphi}(\iota(x)) \) depends on \( \varphi(x) \), which suggests the definition
  \begin{equation*}
    \widetilde{\varphi}(A) \coloneqq \bigvee\nolimits^L \set{ \varphi(a) \given a \in A }.
  \end{equation*}

  Since \( L \) is a complete lattice, it has arbitrary joins and hence the homomorphism is well-defined. Furthermore,
  \begin{equation*}
    \widetilde{\varphi}(\iota(x))
    =
    \widetilde{\varphi}(P_{\leq x})
    =
    \bigvee\nolimits^L \set{ \varphi(a) \given a \in P_{\leq x} }
    \reloset {\eqref{eq:def:order_function/preserving}} =
    \varphi(x).
  \end{equation*}

  The last equality holds because \( \varphi \) is order-preserving and hence \( \varphi(x) \geq \varphi(a) \) for every \( a \leq x \).
\end{proof}

\begin{proposition}\label{thm:dedekind_macnielle_closure_is_totally_ordered}
  The Dedekind-MacNeille completion of a totally ordered set is also totally ordered.
\end{proposition}
\begin{proof}
  Fix a totally ordered set \( (P, \leq) \) and denote by \( D(P) \) its Dedekind-MacNeille completion.

  By \fullref{thm:def:dedekind_macnielle_completion}, \( D(P) \) is a complete lattice, and hence a partially ordered set. It remains only to prove trichotomy on \( D(P) \).

  Fix members \( A \) and \( A' \) of \( D(P) \). Note that both are \hyperref[def:closed_ordered_subset]{downward closed} because \( A = A^{UL} \) and similarly for \( A' \).

  Aiming at a contradiction, suppose that both \( A \setminus A' \) and \( A' \setminus A \) are nonempty, and let \( a \) and \( a' \) be members of the corresponding sets.

  \begin{itemize}
    \item If \( a = a' \), then \( a \) belongs both \( A \) and \( P \setminus A \), which is a contradiction.
    \item If \( a < a' \), then \( a \) is a member of \( A' \) because the latter is downward closed --- again a contradiction.
    \item If \( a > a' \), then \( a' \) is a member of \( A \), which is similar to the above case.
  \end{itemize}

  The obtained contradictions show that either \( A \subseteq A' \) or \( A' \subseteq A \).
\end{proof}

\paragraph{Dedekind cuts}

\begin{definition}\label{def:dedekind_cut}\mcite[sec. I.IV]{Beman1901Dedekind}
  A \term[ru=(Дедекиндово) сечение (\cite[34]{Тагамлицки1971Диф}), ru=(Дедекиндово) сечение (\cite[34]{Александров1977Введение})]{Dedekind cut} in a \hyperref[def:totally_ordered_set]{totally ordered set} \( (P, \leq) \) is pair \( (A, B) \) of \hi{nonempty sets} that \hyperref[def:set_partition]{partitions} \( P \), such that \( a < b \) for every \( a \in A \) and \( b \in B \).

  \begin{figure}[!ht]
    \centering
    \includegraphics{output/def__dedekind_cut}
    \caption{Dedekind cuts over the \hyperref[def:rational_numbers]{rational numbers}}
    \label{fig:def:dedekind_cut}
  \end{figure}
\end{definition}
\begin{comments}
  \item Dedekind's goal was to define the \hyperref[def:real_numbers]{real numbers} as a certain family of cuts of \hyperref[def:rational_numbers]{rational numbers}. We will only use the cuts for defining completeness and instead rely on the \hyperref[def:dedekind_macnielle_completion]{Dedekind-MacNeille completion} of the rationals. Hence, we will not be interested in the properties of Dedekind cuts. See \fullref{rem:dedekind_completion_through_dedekind_macneille_closures} for a discussion of the side effects of this decision.

  \item We use the original definition given by \incite[sec. I.IV]{Beman1901Dedekind} (although in greater generality). His definition has been refined, for example by \incite[113]{Enderton1977Sets} and \incite[17]{Rudin1976Principles}, which defines a cut as a \hyperref[def:closed_ordered_subset]{downward closed} nonempty set \( A \) with no maximum (which uniquely defines \( B \) as its complement). We will not have use for such refinements.
\end{comments}

\begin{definition}\label{def:dedekind_completeness}\mcite[sec. I.IV]{Beman1901Dedekind}
  We say that the \hyperref[def:totally_ordered_set]{totally ordered set} \( (P, \leq) \) is \term{Dedekind complete} if, for every \hyperref[def:dedekind_cut]{Dedekind cut} \( (A, B) \), either \( A \) has a maximum or \( B \) has a minimum.
\end{definition}

\begin{proposition}\label{thm:dedekind_completeness_unbounded_characterization}
  An \hi{\hyperref[def:extremal_points/bounds]{unbounded}} totally ordered set is \hyperref[def:dedekind_completeness]{Dedekind complete} if and only if every \hi{bounded} \hi{nonempty} subset has both a supremum and an infimum.
\end{proposition}
\begin{proof}
  Let \( (P, \leq) \) be an unbounded totally ordered set.

  \SufficiencySubProof Suppose that \( P \) is Dedekind complete. Let \( S \) be a bounded nonempty set in \( P \).

  \SubProof{Proof that \( S \) has a supremum} If \( S \) has a maximum, \fullref{tthm:def:extremal_points/greatest_is_supremum} implies that it is also a supremum.

  We can further assume that \( S \) has no maximum, and thus all upper bounds of \( S \) are strict. Consider the set \( B \coloneqq S^U \) of all upper bounds of \( S \) and also the set \( A \coloneqq B^L \setminus B \) of all strict lower bounds of \( B \).

  As already discussed, every element \( x \) of \( S \) satisfies \( x < u \) for every upper bound \( u \) of \( S \), thus \( S \subseteq A \).

  Since \( P \) is Dedekind complete, we have two possibilities:
  \begin{itemize}
    \item If \( A \) has a maximum it is an upper bound of \( S \) since \( S \subseteq A \). But we have defined \( A \) so that it contains no upper bounds of \( S \). Thus, \( A \) cannot have a maximum.

    \item If \( B \) has a minimum, it is the least upper bound of \( S \).
  \end{itemize}

  \SubProof{Proof that \( S \) has an infimum} Simply take the supremum of the set of lower bounds of \( S \).

  \NecessitySubProof Suppose that every bounded nonempty subset of \( P \) has both a supremum and an infimum.

  Let \( (A, B) \) be a Dedekind cut. Let \( a \) be a member of \( A \) that is not a maximum --- such an element exists because \( A \) is unbounded from below. Then \( P_{>a} \cap A \) is a nonempty bounded set, and our assumption implies that it has a supremum. Furthermore, this supremum is the supremum of \( A \).

  By the totality of the order, \( \sup A \) it is either the maximum of \( A \) or the minimum of \( B \).

  Generalizing on \( (A, B) \), we conclude that \( P \) is Dedekind complete.
\end{proof}

\begin{definition}\label{def:dedekind_completion}\mimprovised
  We define the \term{Dedekind completion} of an \hyperref[def:extremal_points/bounds]{unbounded} \hyperref[def:totally_ordered_set]{totally ordered set} as its \hyperref[def:dedekind_macnielle_completion]{Dedekind-MacNeille completion} with the top and bottom elements removed.
\end{definition}
\begin{comments}
  \item \Fullref{thm:dedekind_macnielle_closure_is_totally_ordered} implies that the Dedekind completion itself is totally ordered.
  \item We need to remove the top and bottom elements in order to approximate Dedekind's construction from \cite[sec. I.IV]{Beman1901Dedekind} --- we discuss this in \fullref{rem:dedekind_completion_through_dedekind_macneille_closures}. Otherwise, we could have just used the Dedekind-MacNeille completion itself, and we would obtain the \hyperref[def:extended_real_numbers]{extended real numbers} rather than the \hyperref[def:real_numbers]{real numbers}.
\end{comments}

\begin{remark}\label{rem:dedekind_completion_through_dedekind_macneille_closures}
  Consider the \hyperref[def:dedekind_completion]{Dedekind completion} of an unbounded totally ordered set. \incite[17]{Rudin1976Principles} and \incite[113]{Enderton1977Sets} instead define the completion as the set of all nonempty downward closed sets without a maximum. These authors call their definitions \enquote{Dedekind cuts}, but Dedekind's original definition that we use here is different. Within this remark, we will call them \enquote{Dedekind lower sets}.

  \Fullref{thm:def:dedekind_macnielle_closure/totally_ordered} implies that, in an unbounded totally ordered set, a subset is \hyperref[def:dedekind_macnielle_closure]{Dedekind-MacNeille closed} if and only if it is downward closed and, if it has a supremum, the supremum is a maximum.

  Here are two discrepancies between using Dedekind-MacNeille closed sets and using Dedekind lower sets.
  \begin{itemize}
    \item Both the empty set and the entire ambient set are Dedekind-MacNeille closed. Excluding them from the definition of Dedekind completion ensures compatibility with the nonemptiness condition of the Dedekind lower sets.

    \item Dedekind-MacNeille closed sets contain their supremum (if it exists), hence they correspond to closed initial segments like \( P_{\leq x} \), while Dedekind lower sets do not contain their supremum, hence they correspond to open initial segments \( P_{< x} \).

    If a supremum does not exist, both are open initial segments.
  \end{itemize}
\end{remark}

\begin{theorem}[Dedekind completion]\label{thm:def:dedekind_completion}
  The \hyperref[def:dedekind_completion]{Dedekind completion} of an \hyperref[def:extremal_points/bounds]{unbounded} \hyperref[def:totally_ordered_set]{totally ordered set} is, up to an \hyperref[def:preordered_set/homomorphism]{order isomorphism}, the smallest such set that is \hyperref[def:dedekind_completeness]{Dedekind complete}.
\end{theorem}
\begin{proof}
  \SubProof{Proof of completeness} Let \( (\mscrA, \mscrB) \) be a \hyperref[def:dedekind_cut]{cut} in the Dedekind completion. We will show that \( \mscrA \) has a maximum or \( \mscrB \) has a minimum.

  Since the Dedekind-MacNeille completion is a complete lattice, the family \( \mscrA \) has a join (supremum). Since the family is nonempty by definition of Dedekind cut, this supremum is not the bottom element, hence it belongs to the Dedekind completion.
  \begin{itemize}
    \item If \( \bigvee \mscrA \) belongs to \( \mscrA \), then it is the maximum of \( \mscrA \).
    \item If \( \bigvee \mscrA \) belongs to \( \mscrB \), then it is the minimum of \( \mscrB \).

    Indeed, if \( B \subseteq \bigvee \mscrA \) for some member \( B \) of \( \mscrB \), then this member is an upper bound of \( \mscrA \) smaller than \( \bigvee \mscrA \), which contradicts the minimality of the supremum.
  \end{itemize}

  \SubProof{Proof of minimality} Follows from \fullref{thm:def:dedekind_macnielle_completion} and \fullref{thm:dedekind_completeness_unbounded_characterization} by noting that the top is the supremum for unbounded from above sets, and similarly for the bottom.
\end{proof}

\begin{example}\label{ex:thm:def:dedekind_completion}
  We list examples of \hyperref[def:dedekind_completion]{Dedekind completion}:
  \begin{thmenum}
    \thmitem{ex:thm:def:dedekind_completion/integers} The set of integers is Dedekind complete. This follows from \fullref{thm:def:dedekind_completion} via \fullref{thm:dedekind_completeness_unbounded_characterization} by noting that every nonempty bounded set of integers has a maximum and a minimum. The latter is a direct consequence of a bounded set of integers being finite.

    \thmitem{ex:thm:def:dedekind_completion/rational_numbers} Most famously, the entire theory of Dedekind and Dedekind-MacNeille completions is derived from Dedekind's original construction of the real numbers from the rationals.
  \end{thmenum}
\end{example}

\begin{definition}\label{def:dense_total_order}\mcite[31]{Birkhoff1948}
  We say that a subset \( A \) of a \hyperref[def:totally_ordered_set]{totally ordered set} \( (P, \leq) \) is \term{dense} in \( P \) if, whenever \( a < c \) for some members \( a \) and \( c \) of \( P \), there exists some \( b \) in \( A \) such that \( a < b < c \).
\end{definition}

\begin{proposition}\label{thm:dedekind_completion_dense}
  A \hyperref[def:dense_total_order]{dense-in-itself} unbounded totally ordered set is \hyperref[def:dense_total_order]{dense} in its \hyperref[def:dedekind_completion]{Dedekind completion}.
\end{proposition}
\begin{proof}
  Suppose that \( D(P) \) is the completion of \( (P, \leq) \). Consider the Dedekind-MacNeille closed nonempty proper subsets \( A \) and \( B \) of \( P \) and suppose that \( A \subsetneq B \).

  Let \( u \) be a value in \( B \) not in \( A \). It is an upper bound of \( A \), but it cannot be a least upper bound because that would contradict \fullref{thm:def:dedekind_macnielle_closure/totally_ordered}. Hence, there must exist a smaller upper bound \( l \) of \( A \).

  Since \( P \) is dense-in-itself, there exists a value \( r \) strictly between \( l \) and \( u \). Then
  \begin{equation*}
    A \subsetneq P_{\leq r} \subsetneq B.
  \end{equation*}
\end{proof}
