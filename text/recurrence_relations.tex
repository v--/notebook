\section{Recurrence relations}\label{sec:recurrence_relations}

\paragraph{General recurrence relations}

\begin{definition}\label{def:recurrence_relation}\mimprovised
  Fix a set \( S \) and a function \( F: S^n \to S \). We are interested in a \hyperref[def:sequence]{sequence} \( \seq{ x_k }_{k=1}^\infty \) such that the following holds when \( k > n \)\fnote{For zero-based sequences, \eqref{eq:def:recurrence_relation} must additionally hold for \( k = n \).}:
  \begin{equation}\label{eq:def:recurrence_relation}
    x_k = F(x_{k-1}, \ldots, x_{k-n}).
  \end{equation}

  With fixed initial values \( x_1, \ldots, x_n \), we call \eqref{eq:def:recurrence_relation} a \term[en=recurrence relation (\cite[def. 2.4.4]{Rosen2019DiscreteMathematics})]{recurrence relation} of \term[en=order (of linear recurring sequence) (\cite[395]{LidlNiederreiter1997FiniteFields})]{order} \( n \).

  \begin{thmenum}
    \thmitem{def:recurrence_relation/solution} We call the uniquely determined sequence \( \seq{ x_k }_{k=1}^\infty \) the \term{solution} of the recurrence. If convenient, we may also refer to the solution itself as a \enquote{recurrence relation}.

    \thmitem{def:recurrence_relation/algebraic} If a recurrence relation is determined by a \hyperref[def:polynomial_algebra/polynomial]{polynomial}, we refer to the recurrence as \term{algebraic} and use the prefixes from the polynomial degree terminology described in \fullref{def:polynomial_degree_terminology}.

    \thmitem{def:recurrence_relation/linear}\mcite[395]{LidlNiederreiter1997FiniteFields} Of particular importance are algebraic recurrences determined by linear polynomials, i.e. \term[en=linear recurring sequence (\cite[395]{LidlNiederreiter1997FiniteFields})]{linear recurrence relations}.

    As with \hyperref[rem:system_of_equations]{systems of linear equations}, we say that the recurrence is \term{homogeneous} if its constant coefficient is zero.

    Inhomogeneous linear recurrences can be expressed via matrices --- see \fullref{thm:linear_recurrence_relation}.
  \end{thmenum}
\end{definition}
\begin{comments}
  \item The general definition is a formalization of \cite[def. 2.4.4]{Rosen2019DiscreteMathematics}.

  \item If \( F \) can be expressed as a \hyperref[def:first_order_syntax/term]{first-order term} \( \tau \), we can regard \eqref{eq:def:recurrence_relation} as the \hyperref[def:first_order_equation]{first-order equation} \( x \doteq \tau \) (where we suppose that \( x \) is not a variable in \( \tau \)). The solutions to \( x \doteq \tau \) in the sense of \fullref{def:first_order_equation} are \( (n + 1) \)-tuples, however, which only determine an initial segment of the sequences that we call solutions to \eqref{eq:def:recurrence_relation}.
\end{comments}

\begin{proposition}\label{thm:linear_recurrence_relation}
  Consider the \hyperref[def:recurrence_relation/linear]{inhomogeneous linear} \hyperref[def:recurrence_relation]{recurrence relation}
  \begin{equation*}
    x_k = \sum_{i=1}^n a_i x_{k-i}.
  \end{equation*}

  We can express it in matrix form as follows:
  \begin{equation}\label{eq:thm:linear_recurrence_relation}
    \begin{pmatrix}
      x_{k-n+m}     \\
      x_{k-(n-1)+m} \\
      \vdots        \\
      x_{k-1+m}     \\
      x_{k+m}
    \end{pmatrix}
    =
    \begin{pmatrix}
      0      & 1      & 0      & \cdots & 0      \\
      0      & 0      & 1      & \cdots & 0      \\
      \vdots & \vdots & \vdots & \ddots & \vdots \\
      0      & 0      & 0      & \cdots & 1      \\
      a_1    & a_2    & a_3    & \cdots & a_n
    \end{pmatrix}^m
    \begin{pmatrix}
      x_{k-n}     \\
      x_{k-(n-1)} \\
      \vdots      \\
      x_{k-1}     \\
      x_k
    \end{pmatrix}.
  \end{equation}
\end{proposition}
\begin{proof}
  Straightforward induction on \( m \).
\end{proof}

\paragraph{Progressions}\hfill

The \( n \)-th term of a \hyperref[def:recurrence_relation]{recurrence relation} can sometimes be computed directly. We call some of these recurrences \enquote{progressions}.

\begin{definition}\label{def:arithmetic_progression}\mcite[595]{ButlerEtAl2016Progressions}
  In an \hyperref[def:abelian_group]{abelian group}, by default taken to be the field of complex numbers, we call the \hyperref[def:recurrence_relation]{recurrence relation}
  \begin{equation}\label{eq:def:arithmetic_progression}
    x_k = x_{k-1} + d
  \end{equation}
  an \term[ru=арифметическая прогрессия (\cite[143]{АлександровМаркушевичХинчин1952ЭнциклопедияТом3}), en=arithmetic progression (\cite[def. 2.4.3]{Rosen2019DiscreteMathematics})]{arithmetic progression} with \term[en=common difference (\cite[def. 2.4.3]{Rosen2019DiscreteMathematics})]{difference} \( d \).
\end{definition}

\begin{proposition}\label{thm:arithmetic_progression_common_term}
  The \( n \)-th term of the \hyperref[def:arithmetic_progression]{arithmetic progression} with difference \( d \) is
  \begin{equation*}
    x_n = a_0 + nd,
  \end{equation*}
  where \( kd \) is \hyperref[con:additive_semigroup/multiplication]{iterated semigroup addition}.
\end{proposition}

\begin{proposition}\label{thm:arithmetic_progression_partial_sums}
  The \hyperref[def:convergent_series]{series} constructed from the arithmetic progression \( \seq{ a_0 + kd }_{k=0}^\infty \) has partial sums
  \begin{equation}\label{eq:thm:arithmetic_progression_partial_sums}
    2 \sum_{k=0}^n a_k = (n + 1) (a_n - a_0).
  \end{equation}
\end{proposition}
\begin{proof}
  \begin{balign*}
    2 \sum_{k=0}^n a_k
     & =
    2 \sum_{k=0}^n (a_0 + kd)
    =    \\ &=
    \sum_{k=0}^n (a_0 + kd) + \sum_{k=0}^n (a_0 + (n-k)d)
    =    \\ &=
    \sum_{k=0}^n (2 a_0 + nd)
    =    \\ &=
    (n + 1) (a_0 + a_n).
  \end{balign*}
\end{proof}

\begin{corollary}\label{thm:numeric_arithmetic_progression_partial_sums}
  For any positive integer \( n \), we have
  \begin{equation}\label{eq:thm:numeric_arithmetic_progression_partial_sums}
    \sum_{k=0}^n k = \sum_{k=1}^n k = \frac {n (n + 1)} 2 = \binom {n+1} 2.
  \end{equation}
\end{corollary}
\begin{comments}
  \item These are precisely the \hyperref[def:triangular_number]{triangular numbers}, as shown in \fullref{thm:triangular_point_configuration_cardinality}.
\end{comments}
\begin{proof}
  Follows from either \fullref{thm:arithmetic_progression_partial_sums} or \fullref{thm:vandermonde_convolution}.
\end{proof}

\begin{definition}\label{def:geometric_progression}\mcite[595]{ButlerEtAl2016Progressions}
  In a \hyperref[def:field]{field}, by default taken to be the field of complex numbers, we call the \hyperref[def:recurrence_relation]{recurrence relation}
  \begin{equation}\label{eq:def:geometric_progression}
    x_k = x_{k-1} q
  \end{equation}
  a \term[ru=геометрическая прогрессия (\cite[144]{АлександровМаркушевичХинчин1952ЭнциклопедияТом3}), en=geometric progression (\cite[def. 2.4.2]{Rosen2019DiscreteMathematics})]{geometric progression} with \term[ru=знаменатель (прогрессии) (\cite[\S 227]{Киселёв2004Геометрия}), en=common ratio (\cite[def. 2.4.2]{Rosen2019DiscreteMathematics})]{quotient} \( q \).
\end{definition}

\begin{proposition}\label{thm:arithmetic_to_geometric_progression}
  Given a complex arithmetic progression \( \seq{ a_k }_{k=0}^\infty \) with difference \( d \), for any complex number \( z \), the sequence \( \seq{ z^{a_k} }_{k=0}^\infty \) is a geometric progression with quotient \( z^d \).
\end{proposition}
\begin{proof}
  Trivial.
\end{proof}

\paragraph{Catalan numbers}

\begin{definition}\label{def:catalan_number}\mcite[188]{Stanley2023EnumCombinatoricsVol2}
  We define the \hyperref[def:sequence]{sequence} of \term[ru=число Каталана (\cite[\S 5.7.4]{Новиков2013ДискретнаяМатематика})]{Catalan numbers} \hyperref[rem:natural_number_recursion]{recursively} as
  \begin{equation}\label{eq:def:catalan_number}
    C_n \coloneqq \begin{cases}
      1,                                    &n = 0, \\
      \sum_{k=0}^{n-1} C_k \cdot C_{n-k-1}, &n > 0.
    \end{cases}
  \end{equation}
\end{definition}
\begin{comments}
  \item Richard Stanley lists many characterizations of Catalan numbers in the exercises to \cite[ch. 6]{Stanley2023EnumCombinatoricsVol2}.

  \item We can regard \eqref{eq:def:catalan_number} as a generalized \hyperref[def:recurrence_relation]{recurrence relation} in which the \( n \)-th term depends on all \( n \) previous terms.
\end{comments}

\begin{proposition}\label{thm:catalan_number_via_binomial_coefficients}
  We can characterize \hyperref[def:catalan_number]{Catalan numbers} via \hyperref[def:binomial_coefficient]{binomial coefficients}:
  \begin{equation}\label{eq:thm:catalan_number_via_binomial_coefficients}
    C_n = \frac 1 {n + 1} \binom {2n} n
  \end{equation}
\end{proposition}
\begin{proof}
  \todo{Prove}
\end{proof}

\begin{proposition}\label{thm:full_binary_tree_count}
  The number of \hyperref[def:n_ary_tree]{full binary trees} with \( n \) \hyperref[def:rooted_tree/internal]{internal nodes} is described by the \hyperref[def:catalan_number]{Catalan number} \( C_n \).
\end{proposition}
\begin{proof}
  We will use induction on \( n \). The case \( n = 0 \) is vacuous. For the inductive step, suppose that the hypothesis holds for less than \( n \) internal nodes.

  Let \( T \) be a full binary tree with \( n \) internal nodes. The \hyperref[def:rooted_tree/subtree]{immediate subtrees} of \( T \) have a total of \( n - 1 \) internal nodes. If the left subtree has \( k \) internal nodes, the right subtree has \( n - k - 1 \) internal nodes.

  Denote by \( t_n \) the number of full binary trees with \( n \) internal nodes. Then, by what we have shown,
  \begin{equation*}
    t_n = \sum_{k=0}^{n-1} t_k t_{n-k-1}.
  \end{equation*}

  By the inductive hypothesis, \( t_n \) satisfies the recursive clause in \eqref{eq:def:catalan_number}, hence it equals \( C_n \).
\end{proof}

\begin{lemma}\label{thm:dyck_language_length}
  Every string in the \hyperref[def:dyck_language]{Dyck language} has even length.
\end{lemma}
\begin{proof}
  We will use \fullref{thm:induction_on_rooted_trees} on the \hyperref[def:parse_tree]{parse tree} of the word \( w \):
  \begin{itemize}
    \item If \( w \) is the empty word, its length is clearly even.

    \item If \( w = (w') \), then \( w \) is longer than \( w' \) by two symbols. By the inductive hypothesis, \( w \) has even length.

    \item If \( w = w_1 w_2 \), then \( w \) has even length if and only if both \( w_1 \) and \( w_2 \) have even length.

    Again, by the inductive hypothesis, \( w \) has even length.
  \end{itemize}
\end{proof}

\begin{corollary}\label{thm:dyck_word_count}
  The number of \hyperref[def:dyck_language]{Dyck words} of length \( 2n \) is described by the \hyperref[def:catalan_number]{Catalan number} \( C_n \).
\end{corollary}
\begin{proof}
  \Fullref{thm:dyck_words_and_binary_trees} describes a bijective correspondence between Dyck words and full binary trees.

  Let \( w \) be a Dyck word of length \( 2n \). \Fullref{thm:dyck_language_alternative_grammar} implies that either \( w \) is empty or \( w = (w_1) w_2 \). \Fullref{thm:dyck_language_length} implies that both \( w_1 \) and \( w_2 \) have even length. We can then use induction on the length of \( w \) to show that the tree constructed via \fullref{thm:dyck_words_and_binary_trees/word_to_tree} has \( n \) internal nodes:
  \begin{itemize}
    \item The base case where \( w \) is empty is clear.
    \item If \( w = (w_1) w_2 \) and if the inductive hypothesis holds for \( w_1 \) and \( w_2 \), then the trees \( T_{w_1} \) and \( T_{w_2} \) have \( n - 1 \) internal nodes in common, and, by adding a new root, we obtain \( T_w \) with \( n \) internal nodes.
  \end{itemize}

  \Fullref{thm:full_binary_tree_count} then implies that there are \( C_n \) Dyck words of length \( 2n \).
\end{proof}
