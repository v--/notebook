\subsection{Undirected graphs}\label{subsec:undirected_graphs}

\begin{remark}\label{rem:graph_etymology}
  Unfortunately, the term \enquote{graph} has at least several distinct established meanings:
  \begin{itemize}
    \item The \hyperref[def:multi_valued_function/graph]{graph of a valued function} (or relation).
    \item An \hyperref[def:undirected_multigraph]{undirected graph}, with or without loops or parallel edges.
    \item A \hyperref[def:quiver]{directed graph}, which may be a simple directed graph or a quiver.
  \end{itemize}

  The graph of a function is fundamentally different from the others, which we call \enquote{combinatorial graphs}. Although in \fullref{def:undirected_multigraph} we define undirected graphs as functions, the concepts are distinct and the formalisms are related by accident.
\end{remark}

\begin{definition}\label{def:undirected_multigraph}
  An \term{undirected multigraph} is a \hyperref[def:hypergraph]{hypergraph} \( G = (V, E, \mscrE) \) in which all hyperedges are either edges or loops, i.e. every hyperedge \( e \) has either one or two endpoints. We will use the term \enquote{edge} to include loops, although this in general varies with the author.

  \Cref{fig:def:undirected_multigraph} illustrates the concepts from this definition. The figure is not merely illustrative --- it is a \hyperref[def:quiver_geometric_realization/embedding]{graph embedding}.

  \begin{figure}[!ht]
    \begin{equation}\label{eq:fig:def:undirected_multigraph}
      \begin{aligned}
        \includegraphics[page=1]{output/def__undirected_multigraph.pdf}
      \end{aligned}
    \end{equation}
    \caption{An undirected multigraph, which becomes simple after removing the dashed edges.}\label{fig:def:undirected_multigraph}
  \end{figure}

  \begin{thmenum}
    \thmitem{def:undirected_multigraph/simple} A common assumption is that an undirected multigraph has no loops and no parallel edges. In this case, an established terminology is \term{simple undirected graph}.

    For a simple undirected graph, we often write \( G = (V, E) \), where \( E \) is a set of unordered pairs \( \set{ u, v } \) of distinct vertices. We can also regard simple undirected graphs as symmetric directed graphs --- see \fullref{rem:symmetric_directed_graphs}.

    \thmitem{def:undirected_multigraph/submodel} The \term{subgraphs} of \( G \) are the precisely \hyperref[def:hypergraph/submodel]{sub-hypergraphs} of \( G \). In particular, the \term{full subgraphs} are the \hyperref[eq:def:hypergraph/submodel/full]{full sub-hypergraphs}, that are maximal among the subgraphs with the same vertices.

    \thmitem{def:undirected_multigraph/categories} For a Grothendieck universe \( \mscrU \), we denote the \hyperref[def:category]{category} of undirected (multi)graphs by \( \ucat{MultGph} \) and \( \ucat{SimpGph} \), respectively. Both are strict \hyperref[def:subcategory]{subcategories} of the category \hyperref[def:category_of_small_hypergraphs]{\( \cat{HypGph} \)} of hypergraphs.
  \end{thmenum}
\end{definition}

\begin{example}\label{ex:complete_graph}
  For every set \( V \), the \term{complete graph} \( K_V \) is the \hyperref[def:undirected_multigraph]{simple undirected graph} with an edge between any pair of distinct vertices.

  \begin{figure}[!ht]
    \begin{equation}\label{eq:fig:ex:complete_graph}
      \begin{aligned}
        \includegraphics[page=1]{output/ex__complete_graph.pdf}
      \end{aligned}
    \end{equation}
    \caption{The complete graph \( K_4 \).}\label{fig:ex:complete_graph}
  \end{figure}

  \Fullref{thm:ordinal_is_set_of_smaller_ordinals} justifies using the notation \( K_n \) for complete graphs of \hyperref[def:hypergraph/order]{finite order}.
\end{example}

\begin{definition}\label{def:graph_adjacency_matrix}\mcite[ch. 1, sec. 2.3]{GondranMinoux1984Graphs}
  Let \( G = (V, E, \mscrE) \) be a \hyperref[def:undirected_multigraph]{simple undirected graph}. Its \term{adjacency matrix}
  \begin{equation*}
    M = \seq{ a_{uv} }_{u, v \in V}
  \end{equation*}
  has elements
  \begin{equation*}
    M_{uv} \coloneqq \begin{cases}
      1, &u \T{is adjacent to} v \\
      0, &\T{otherwise.}
    \end{cases}
  \end{equation*}
\end{definition}

\begin{example}\label{ex:def:graph_adjacency_matrix}
  The \hyperref[def:graph_adjacency_matrix]{adjacency matrix} of the complete graph \( K_4 \) (\cref{fig:ex:complete_graph}) is
  \begin{equation}\label{eq:ex:def:graph_adjacency_matrix}
    \begin{blockarray}{ccccc}
        & a & b & c & d \\
    \begin{block}{c(cccc)}
      a & 0 & 1 & 1 & 1 \\
      b & 1 & 0 & 1 & 1 \\
      c & 1 & 1 & 0 & 1 \\
      d & 1 & 1 & 1 & 0 \\
    \end{block}
    \end{blockarray}
  \end{equation}
\end{example}

\begin{definition}\label{def:undirected_multigraph_path}
  Let \( G = (V, E, \mscrE) \) be an \hyperref[def:undirected_multigraph]{undirected multigraph}. We will adapt the definition of a \term{directed path} from \fullref{def:quiver_path}. Different authors use similar terms like \term{walk} and \term{trail} and put different restrictions on them. We will not put as few restrictions as possible.

  \begin{figure}[!ht]
    \begin{equation}\label{eq:fig:def:undirected_multigraph_path}
      \begin{aligned}
        \includegraphics[page=1]{output/def__undirected_multigraph_path.pdf}
      \end{aligned}
    \end{equation}
    \caption{The dashed solid lines indicate the path with head \( a \) and edges \( (e_2, e_4, e_6) \) in the graph \eqref{eq:fig:def:undirected_multigraph}.}\label{fig:def:undirected_multigraph_path}
  \end{figure}

  Analogously to \hyperref[def:quiver_path]{quiver path}, consider a sequence \( p = (v, e_1, e_2, \ldots) \), where \( e_1, e_2, \ldots \) are edges and \( v \) is a vertex incident to \( e_1 \).

  Unlike arcs in quivers, however, edges have no head and tail. Fortunately, we can define them via mutual recursion, with the caveat that every edge may have a different head and tail with respect to another path:
  \begin{itemize}
    \item If \( e_k \) is a loop at \( u \), then \( u \) is both the head and the tail of \( e_k \).
    \item If \( e_1 \) is not a loop, its head is \( v \) and the tail is the other endpoint of \( e_1 \)
    \item If \( e_k \neq e_1 \) is not a loop, if \( e_{k-1} \) has a tail \( t(e_{k-1}) \) and if \( t(e_{k-1}) \) is an endpoint of \( e_k \), we define the head of \( e_k \) to be \( t(e_{k-1}) \) and the tail of \( e_k \) is the other endpoint of \( e_k \).
  \end{itemize}

  If the head and tail is defined for every edge in \( p \), and if \( e_k \neq e_{k-1} \) but \( h(e_k) = t(e_{k-1}) \) for every index \( k \), we say that \( p \) is a \term{path}.

  Note that it is not sufficient for \( e_k \) and \( e_{k-1} \) to be adjacent. Consider the following scenario, in which \( (v, e_1, e_2, e_3) \) is not a path because the tail of \( e_2 \) is not incident to \( e_3 \):
  \begin{equation}\label{eq:def:undirected_multigraph_path/almost_directed}
    \begin{aligned}
      \includegraphics[page=2]{output/def__undirected_multigraph_path.pdf}
    \end{aligned}
  \end{equation}

  We introduce the following terminology:
  \begin{thmenum}
    \thmitem{def:undirected_multigraph_path/endpoints} The \term{head} of the entire path \( p = (v, e_1, \ldots, e_n) \) is \( h(p) \coloneqq v \) and the \term{tail} is \( t(p) \coloneqq t(e_n) \). Note that the tail is only defined for finite paths.

    \thmitem{def:undirected_multigraph_path/empty} The \term{empty path} at \( v \) is the path \( p = (v) \). We call it empty because it has no edges. It is not an empty sequence because we want it to have a head and tail. The primary motivation for different empty paths are the identity morphisms in the free category for a \hyperref[def:quiver]{quiver} defined in \fullref{def:quiver_free_category}.

    \thmitem{def:undirected_multigraph_path/length} The \term{length} \( \len(p) \) of a path \( p \) is simply the number of edges. Hence, empty paths have length zero.

    \thmitem{def:undirected_multigraph_path/domain} The \term{domain} \( \dom(p) \) of \( p \) is the set of all vertices that belong to at least one edge in \( p \). We say that the path \term{visits} each member of \( \dom(p) \).

    The domain of the path \eqref{eq:fig:def:undirected_multigraph_path} is the set \( \set{ e_2, e_4, e_6 } \).

    Note that
    \begin{equation}\label{eq:def:undirected_multigraph_path/length_and_domain}
      \card(\dom(p)) + 1 \leq \len(p)
    \end{equation}
    in general since a vertex can be an endpoint of many edges.

    \thmitem{def:undirected_multigraph_path/concatenation} Similarly to concatenation of words defined in \fullref{def:formal_language/concatenation}, the concatenation of the paths
    \begin{align*}
              p &= (v, e_1, \ldots, e_n) \\
      \T{and} q &= (t(p), e_{n+1}, e_{n+1}, \ldots)
    \end{align*}
    is simply the path
    \begin{equation*}
      p \cdot q \coloneqq (v, e_1, e_2, \ldots, e_n, e_{n+1}, e_{n+1}, \ldots).
    \end{equation*}

    Note that the head of \( q \) must be the tail of \( p \) and that \( p \) is necessarily finite.

    \thmitem{def:undirected_multigraph_path/subpath} We say that the \hyperref[def:sequence]{subsequence} \( q \) of \( p \) is a \term{subpath} of \( p \) if the edges of \( q \) are consecutive in \( p \), i.e. there exists some index \( n \geq 0 \) such that \( q_k = e_{n + k} \) for \( 0 < k < \len(q) \).

    In \eqref{eq:fig:def:undirected_multigraph_path}, the subsequence \( (a, e_2, e_4) \) is a subpath and \( (a, e_2, e_6) \) is not.

    \thmitem{def:undirected_multigraph_path/cycle} A nonempty path \( p \) is a \term{cycle} if \( h(p) = t(p) \). Cycles are also called \term{closed paths}.

    Consider the graph \eqref{eq:fig:def:undirected_multigraph_path}. The path \( (a, e_2, e_4, e_3, e_1) \) is a cycle and so is and so is \( (a, e_2, \widehat{e_2}) \). We have forbidden for the same edge to appear twice consecutively, which means that \( (a, e_2, e_2) \) is not a cycle because it is not a valid path.

    A graph without cycles is called \term{acyclic}.

    \thmitem{def:undirected_multigraph_path/simple} The path \( p \) is \term{simple} if no subpath is a cycle.

    An alternative characterization is that every vertex in \( \dom(p) \) is visited only once, i.e. for every vertex in \( \dom(p) \), there exist at most two edges of the path which are incident to it.

    Another characterization is that equality holds in \eqref{eq:def:undirected_multigraph_path/length_and_domain}.

    \thmitem{def:undirected_multigraph_path/converse} If the chain \eqref{eq:fig:def:undirected_multigraph_path} is finite of length \( n \), we define its \term{converse} as
    \begin{equation*}
      p^{-1} \coloneqq \parens[\Big]{ t(p), e_n, e_{n-1}, \cdots, e_2, e_1 }.
    \end{equation*}

    \thmitem{def:undirected_multigraph_path/characteristic_vector} The \term{characteristic vector} \( \vect{p} \) of a path \( p \) is defined as the characteristic vector of \( \dom(p) \) in the \hyperref[def:hypergraph_vector_spaces/edge]{edge space} \( \BbbF_2^E \).
  \end{thmenum}
\end{definition}

\begin{definition}\label{def:undirected_multigraph_connectedness}
  Let \( G = (V, E, \mscrE) \) be an undirected multigraph.

  \begin{thmenum}
    \thmitem{def:undirected_multigraph_connectedness/reachability} We say that the vertices \( u \) and \( v \) are \term{connected} if there exists a path from \( u \) to \( v \).

    In relation to \fullref{def:quiver_condensation}, we also say that \( v \) is \term{reachable} from \( v \) and vice versa.

    \thmitem{def:undirected_multigraph_connectedness/components} Connectedness is obviously an equivalence relation in \( E \). Denote it by \( {\sim} \). For each vertex \( v \), the equivalence class \( [v] \) of all vertices reachable from \( v \) is called a \term{connected component}.

    The \term{connectivity number} of \( G \) is the cardinality \( \card(V / {\sim}) \) of the quotient set.

    If \( G \) has only one connected component, we say that it itself is \term{connected}. Equivalently, \( G \) is connected if every vertex is reachable from every other vertex.

    \thmitem{def:undirected_multigraph_connectedness/condensation} The \term{condensation} of \( G \) is the edgeless graph with vertices \( V / {\sim} \).

    This concept is useless for undirected graphs but important for \hyperref[def:quiver]{quivers} --- see \fullref{def:quiver_condensation}.
  \end{thmenum}

  Compare this definition to \fullref{def:quiver_connectedness}.
\end{definition}
