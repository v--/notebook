\begin{definition}\label{def:undirected_multigraph_connectedness}
  Let \( G = (V, E, \mscrE) \) be an undirected multigraph.

  \begin{thmenum}
    \thmitem{def:undirected_multigraph_connectedness/reachability} We say that the vertices \( u \) and \( v \) are \term{connected} if there exists a path from \( u \) to \( v \).

    In relation to \fullref{def:directed_multigraph_condensation}, we also say that \( v \) is \term{reachable} from \( v \) and vice versa.

    \thmitem{def:undirected_multigraph_connectedness/components} Connectedness is obviously an equivalence relation in \( E \). Denote it by \( {\sim} \). For each vertex \( v \), the equivalence class \( [v] \) of all vertices reachable from \( v \) is called a \term{connected component}.

    The \term{connectivity number} of \( G \) is the cardinality \( \card(V / {\sim}) \) of the quotient set.

    If \( G \) has only one connected component, we say that it itself is \term{connected}. Equivalently, \( G \) is connected if every vertex is reachable from every other vertex.

    \thmitem{def:undirected_multigraph_connectedness/condensation} The \term{condensation} of \( G \) is the edgeless graph with vertices \( V / {\sim} \).

    This concept is useless for undirected graphs but important for \hyperref[def:directed_multigraph]{directed multigraphs} --- see \fullref{def:directed_multigraph_condensation}.
  \end{thmenum}

  Compare this definition to \fullref{def:graph_connectedness}.
\end{definition}
