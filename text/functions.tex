\subsection{Functions}\label{subsec:functions}

\begin{remark}\label{rem:function_definition}
  It is not straightforward to formalize the notion of correspondence between two values. We will synonymously use the terms \term{mapping}, \term{function}, \term{transformation} and \term{operator}. We will define functions as special binary relations in \fullref{def:function}. Despite this being a standard practice, this has several drawbacks:

  \begin{itemize}
    \item There are some more mappings than the functions defined in \fullref{def:function}. For example, assigning to a set \( A \) its \hyperref[def:basic_set_operations/power_set]{power set} \( P(A) \) can be cannot be regarded as a function because its domain and range should both be the set of all sets whose existence is inconsistent with \hyperref[def:zfc]{\logic{ZFC}} by \fullref{thm:simple_foundation_theorems/member_of_itself}.

    \item We often work with spaces that have some additional structure in addition to being sets. In this case, we are often only interested in maps that preserve this structure. This is the case with \hyperref[def:group/homomorphism]{group homomorphisms}, for example.

    In terms of \hyperref[def:first_order_structure]{first-order structures}, not every function is a \hyperref[def:first_order_homomorphism]{homomorphism}.

    This is a motivating example for the benefits of \hyperref[sec:category_theory]{category theory}, where the notion of \hyperref[def:category/morphisms]{morphism} is able to capture this additional structure.

    It is often important to consider functions that are not homomorphisms, however. For example, function spaces over \( \BbbR \) contain some very complicated functions that are not field homomorphisms, order homomorphisms nor continuous functions and thus do not aim to preserve structural properties of \( \BbbR \).

    \item Several generalizations of the standard notion of a function are often used. These include \hyperref[def:multi_valued_function]{multi-valued} and \hyperref[def:partial_function]{partial functions}. Both are not functions, strictly speaking.

    For simplicity of exposition, we take multi-valued functions as primitive notions and define single-valued functions as special cases. This is actually done in \cite[def. 2.31]{OpenLogicFull} and \cite[8]{Kelley1975} except that the corresponding authors conflate multi-valued functions and relations.

    \item Set-theoretic functions are often used in contexts where they do not refer to the intuitive notion of a mapping. Examples include \hyperref[def:cartesian_product]{Cartesian products} and \hyperref[def:cartesian_product/indexed_family]{indexed families}.
  \end{itemize}
\end{remark}

\begin{definition}\label{def:multi_valued_function}
  A \term{multi-valued function} from \( A \) to \( B \) is simply a \hyperref[def:binary_relation]{binary relation} \( (F, A, B) \). For us the difference between a multi-valued function and a relation is merely in how we treat them. Multi-valued function are also called set-valued.

  As discussed in \fullref{def:relation/graph}, the common practice of identifying a multi-valued function \( (F, A, B) \) with its \hyperref[def:relation/graph]{graph} \( F \) with no regard to its \hyperref[def:relation/signature]{signature} \( (A, B) \) has serious drawbacks that we wish to avoid.

  We use the more established notation \( F: A \multto B \) rather than \( (F, A, B) \) and call the string of symbols \( A \multto B \) the signature of \( F \) rather than the pair \( (A, B) \).

  \begin{thmenum}[series=def:multi_valued_function]
    \thmitem{def:multi_valued_function/value} The \term{value} of \( F \) at \( x \) is
    \begin{equation*}
      F(x) \coloneqq \set{ y \in B \given (x, y) \in F }.
    \end{equation*}

    In case \( x \) is not a concrete value, then \( F(x) \) stands for the function \( F \) itself. In other words, \( F(x) \) refers to a member of \( B \) if \( x \) is a \hyperref[def:first_order_syntax/formula_bound_variables]{bound variable} and \( F(x) \) refers to the function \( F \) if \( x \) is a \hyperref[def:first_order_syntax/formula_free_variables]{free variable}.

    \thmitem{def:multi_valued_function/set_value} We also define the value of \( F \) at a subset \( X \) of \( A \) as
    \begin{equation*}
      F[X] \coloneqq \bigcup \set{ F(x) \given x \in X }.
    \end{equation*}

    This is also called the \term{action} of \( F \) on the set \( X \) or the \term{image} of \( X \) under \( F \). We also refer to \( F(x) \) as the image of \( x \) under \( F \) because \( F(x) = F[\set{x}] \).

    The notation \( f[X] \) is mostly used withing set theory, for example in \cite[def. 2.31]{OpenLogicFull} and \cite[44]{Enderton1977Sets}, however the following notation is usually used preferred of set theory:
    \begin{equation*}
      f(X) = \bigcup \set{ F(x) \given x \in X }
    \end{equation*}

    We use \( f[X] \) almost exclusively for \fullref{sec:set_theory} because of the following nasty ambiguity: If \( A \) is a \hyperref[def:transitive_set]{transitive set}, then \( x \in A \) implies that \( x \subseteq A \) yet \( f(x) \neq f[x] \). For example, if \( A = \set{ \varnothing, \set{ \varnothing } } \), then
    \begin{equation*}
      f[\set{ \varnothing }] = \set{ f(\varnothing) } \neq f(\set{ \varnothing }).
    \end{equation*}

    This may be very confusing when dealing with \hyperref[def:ordinal]{ordinals}.

    \thmitem{def:multi_valued_function/arguments} As mentioned in \fullref{def:multi_valued_function/value}, given a function \( f: A \to B \), we sometimes use the notation \( f(x) \) where \( x \) is a free variable in the sense of \fullref{def:first_order_syntax/formula_free_variables}.

    If \( A = A_1 \times \cdots \times A_n \) is a \hyperref[def:cartesian_product/product]{finite Cartesian product}, we instead use the notation \( f(x_1, \ldots, x_n) \) and regard \( x_1, \ldots, x_n \) as free variables that have no assigned value.

    The variables are called \term{arguments} or sometimes \term{parameters}, although the latter term is a bit overloaded. This notion is somewhat informal and depends on the context since \( A \) can usually be represented as a Cartesian product in different ways and with different arities. For example, if \( A = B \times C \), we can write both \( f(a) \) and \( f(b, c) \) and the function has a different number of parameters in each case. In practice the number of arguments is usually clear from the context. We sometimes use \( \vect{a} \) when we regard \( a \) as a tuple.

    For example, in \hyperref[def:first_order_semantics]{classical first-order semantics}, to each \( n \)-ary functional symbol there corresponds an \( n \)-ary function with the unambiguous signature \( X^n \to X \).

    When working over a vector space like \( \BbbR^2 \), on the other side, depending on the context we regard \hyperref[rem:functional]{functionals} as either unary or binary functions.

    We sometimes refer to \( f \) as a \term{dependent variable} since it depends on its arguments. In this later case, we call the arguments \term{independent variables}.

    Common terms for functions with low arity:
    \begin{itemize}
      \item \term{Nullary} for \( n = 0 \).
      \item \term{Unary} or \term{univariate} for \( n = 1 \).
      \item \term{Binary} or \term{bivariate} for \( n = 2 \).
      \item \term{Ternary} for \( n = 3 \).
      \item \term{Multivariate} for \( n > 1 \).
    \end{itemize}
  \end{thmenum}

  The following terminology is consistent with relations:
  \begin{thmenum}[resume=def:multi_valued_function]
    \thmitem{def:multi_valued_function/graph} The \term{graph} \( \gph(F) \) of \( F \) is the graph of the relation \( F \). This is consistent with \fullref{def:relation/graph}.

    \thmitem{def:multi_valued_function/restriction} The \term{domain restriction} or simply \term{restriction} of \( F: A \multto B \) to \( X \subseteq A \) is the multi-valued function \( F\restr_{X}: X \rightarrow B \). We say that \( F \) is an \term{extension} of \( F\restr_{X} \). This is consistent with \fullref{def:binary_relation/restriction}.

    Dually, if we restrict \( B \) to its subset \( Y \), we say that the obtained function is a \term{range restriction} of \( F \).

    \thmitem{def:multi_valued_function/composition} The \term{composition} \( G \bincirc F \) of two multi-valued functions \( F: A \multto B \) and \( G: B \multto C \) is the function
    \begin{equation*}
      [G \bincirc F](x) \coloneqq G(F(x)).
    \end{equation*}

    The square brackets around \( G \bincirc F \) are not a special notation, but rather another pair of delimiters that looks different from parentheses for the sake of reducing visual clutter.

    This definition is consistent with \fullref{def:binary_relation/composition}.

    \thmitem{def:multi_valued_function/empty} A function \( F: A \multto B \) is \term{empty} if \( \gph(F) = \varnothing \). This is consistent with \fullref{def:binary_relation/empty}.

    Note that if \( F \) is a \hyperref[def:multi_valued_function/total]{total} empty multi-valued function, then \( A = \varnothing \) because otherwise \( F \) would itself be nonempty.
  \end{thmenum}

  The following terminology is inconsistent with relations:
  \begin{thmenum}[resume=def:multi_valued_function]
    \thmitem{def:multi_valued_function/arity} The \term{arity} of a multi-valued function is its number of arguments. This is not to be confused with \hyperref[def:relation/arity]{relation arity} --- functions are always binary relations.

    \thmitem{def:multi_valued_function/total} The term \term{total multi-valued function} means that \( \dom(F) = A \), i.e. that \( F(x) \neq \varnothing \) for all \( x \in A \). This is very different from total binary relations as defined in \fullref{def:binary_relation/total}.

    \thmitem{def:multi_valued_function/inverse} The \term{inverse} \( F^{-1}: B \multto A \) of a multi-valued function \( F: A \multto B \) is the multi-valued function in which assigns to every element \( y \) of the image \( \img(F) \) the set of all \( x \in A \) such that \( y \in F(x) \). It is therefore the \hyperref[def:binary_relation/converse]{converse binary relation} of \( F \).

    \thmitem{def:multi_valued_function/identity}\mcite[exer. 3.1.1]{Leinster2016Basic} The function corresponding to the diagonal relation \( \increment A \) as defined in \fullref{def:binary_relation/diagonal} is called the \term{identity function} on \( A \) and denoted by \( \id_A \). The \term{diagonal function} is instead defined as \( f: A \to A^2 \) is defined as \( f(x) \coloneqq (x, x) \). This is mostly used in category theory.
  \end{thmenum}

  We define some additional terminology:
  \begin{thmenum}[resume=def:multi_valued_function]
    \thmitem{def:multi_valued_function/image} The \term{image} \( \img(F) \) of \( F \) is the set of all \( y \in B \) that belong to the set \( F(x) \) for at least one \( x \in A \). It is the same as the value \( F(A) \).

    \thmitem{def:multi_valued_function/domain} The \term{domain} \( \dom(F) \) of \( F \) is the set of all values \( x \) for which \( F(x) \neq \varnothing \). When regarding \( F \) as a relation, the domain can be defied as the set
    \begin{equation*}
      \dom(R) \coloneqq \set{ x \in A \given \qexists {y \in B} (x, y) \in R }.
    \end{equation*}

    \thmitem{def:multi_valued_function/range} The \term{range} \( \range(F) \) is simply the set \( B \). It is also called the \term{codomain} of \( F \).

    \thmitem{def:multi_valued_function/superposition} Although the terms \enquote{composition} and \enquote{superposition} are used interchangeably, for example in \cite[\textnumero 25]{Фихтенгольц1968Том1}, the term \enquote{superposition} often refers to a certain generalization of \hyperref[def:multi_valued_function/composition]{function composition}. The latter usage can be found in \cite[16]{Яблонский1986}.

    If we are given the family of functions \( \seq{ F_k: A \multto B_k }_{k = 1, \ldots, n} \) and the function \( G: B_1 \times \cdots \times B_n \multto C \), we defined their \term{superposition} is
    \begin{equation*}
      \begin{aligned}
        &H: A \to C
        &H(x) \coloneqq G(F_1(x), \ldots, F_n(x)).
      \end{aligned}
    \end{equation*}

    \thmitem{def:multi_valued_function/endofunction} Functions from a set to itself (e.g. \( F: A \multto A \)) are called \term{endofunctions}.

    \thmitem{def:multi_valued_function/large_preimage} For a fixed set \( Y \subseteq B \), its \term{large preimage} or simply \term{preimage} under \( F: A \to B \) is the image of \( Y \) under the inverse function \( F^{-1}: B \multto A \). For a single value \( y \in B \), we call \( F^{-1}(y) \) the \term{fiber} of \( y \) under \( F \).

    \thmitem{def:multi_valued_function/small_preimage} Analogously, we define its \term{small preimage} as
    \begin{equation*}
      F_{-1}[Y] \coloneqq \set{ x \in A \colon F(x) \subseteq Y }.
    \end{equation*}

    Obviously \( F_{-1}[Y] \subseteq F^{-1}[Y] \).
  \end{thmenum}
\end{definition}

\begin{proposition}\label{thm:def:multivalued_function}
  \hyperref[def:multi_valued_function]{Multi-valued functions} have the following basic properties:

  \begin{thmenum}
    \thmitem{thm:def:multivalued_function/associative} \hyperref[def:multi_valued_function/composition]{Composition} is associative. That is, for any three functions \( F: A \to B \), \( G: B \to C \) and \( H: C \to D \) we have
    \begin{equation*}
      H \bincirc [G \bincirc F] = [H \bincirc G] \bincirc F.
    \end{equation*}

    We will henceforth simply write \( H \bincirc G \bincirc F \).

    \thmitem{thm:def:multivalued_function/composition_inverse} If \( F: A \to B \) and \( G: B \to C \) are \hyperref[def:multi_valued_function]{multi-valued functions}, then
    \begin{equation*}
      [G \bincirc F]^{-1} = F^{-1} \bincirc G^{-1}.
    \end{equation*}
  \end{thmenum}
\end{proposition}
\begin{proof}
  \SubProofOf{thm:def:multivalued_function/associative} Let \( a \in A \). Then in order for \( d \in D \) to belong to \( [[H \bincirc G] \bincirc F](a) \), there must exist values \( b \in B \) and \( c \in C \) such that \( b \in F(a) \) and \( c \in G(b) \) and \( d \in H(c) \). Clearly this is also the condition for \( d \) to belong to \( [H \bincirc [G \bincirc F]](a) \).

  \SubProofOf{thm:def:multivalued_function/composition_inverse} Since \( G \bincirc F \) has signature \( A \to C \), clearly \( [G \bincirc F]^{-1} \) has signature \( C \to A \). Let \( c \in C \).

  \begin{itemize}
    \item If \( [G \bincirc F]^{-1}(c) \) is empty, \( c \not\in \img(G \bincirc F) \), hence either \( G^{-1}(c) \) is empty or is nonempty, but disjoint from \( \img(F) \). Hence, \( [F^{-1} \bincirc G^{-1}](c) \) is also empty.

    \item Suppose that \( [G \bincirc F]^{-1}(c) \) is not empty and let \( a \in [G \bincirc F]^{-1}(c) \).

    By definition, there exists \( b \in B \) and such that \( b \in F(a) \) and \( c \in G(b) \). Hence, \( b \in G^{-1}(c) \) and \( a \in F^{-1}(b) \), which implies that the image of \( c \) under the composition \( F^{-1} \bincirc G^{-1} \) also contains \( a \).
  \end{itemize}

  In both cases, for every \( c \in C \) we have
  \begin{equation*}
    [G \bincirc F]^{-1}(c) = [F^{-1} \bincirc G^{-1}](c).
  \end{equation*}

  Hence, the two multi-valued functions are equal.
\end{proof}

\begin{definition}\label{def:function}
  Although \hyperref[def:multi_valued_function]{multi-valued functions} are very general, they are not nearly as useful nor are not studied nearly as extensively as single-valued functions.

  The \hyperref[def:multi_valued_function]{multi-valued function} \( F: A \multto B \) is called a \term{single-valued function} if \( F(x) \) is a \hyperref[rem:singleton_sets]{singleton set} for each \( x \in A \). In this case, we write \( F: A \to B \) rather than \( F: A \multto B \).

  All terminology from \fullref{def:multi_valued_function} holds for single-valued functions.

  By convention, when both single-valued and multi-valued functions are involved, the former are denoted using lowercase letters and the latter using uppercase letters.

  Strictly speaking, the value \( f(x) \) of a single-valued function \( f: A \to B \) is a singleton set. It is common practice (e.g. in \cite[def. 3.1]{OpenLogicFull} and \cite[10]{Kelley1975}) to define the value of a single-valued function to be an element of \( B \) rather than a subset of \( B \). Unless this would be confusing, we identify \( f(x) \) with its only element due to the convention established in \fullref{rem:singleton_sets}.

  More precisely, single-valued functions satisfy the following \hyperref[rem:predicate_formula]{predicate formula}, which states that the free variable \( \chi \) is a function from \( \alpha \) to \( \beta \):
  \begin{equation*}\taglabel[\op{IsFun}]{eq:def:function/predicate}
    \ref{eq:def:function/predicate}[\chi, \alpha, \beta] \coloneqq \qforall {\xi \in \alpha} \qExists {\eta \in \beta} \underbrace{ \qexists {\zeta \in \chi} \ref{eq:def:cartesian_product/kuratowski_pair_predicate}[\zeta, \xi, \eta] }_{\chi(\xi) = \eta}.
  \end{equation*}

  Unless otherwise noted, we will now conflate the terms \enquote{function} and \enquote{single-valued function}.

  \begin{thmenum}
    \thmitem{def:function/set_value} We can simplify \fullref{def:multi_valued_function/set_value} drastically:
    \begin{equation*}
      f[A]
      =
      \bigcup \set[\Big]{ \set{ f(x) } \given x \in A }
      =
      \set{ f(x) \given x \in A }.
    \end{equation*}

    As mentioned in \fullref{def:multi_valued_function/set_value}, we usually prefer the notation \( f[A] \) outside \fullref{sec:set_theory} where we are less prone to ambiguity.

    \thmitem{def:function/currying} Given the \hyperref[def:multi_valued_function/arguments]{two-argument} function \( f: A \times B \to C \), we can define another function \( g: A \to \fun(B, C) \) as
    \begin{equation*}
      g(x) \coloneqq (y \mapsto f(x, y)).
    \end{equation*}

    This process is called \term{currying} after Haskell Curry. Obtaining \( f \) from \( g \) is instead called \term{uncurrying}.

    Currying is useful if we have somehow fixed a value \( x_0 \in A \), in which case we can \enquote{get rid} of one argument by introducing some shortcut for the function \( g(x_0): B \to C \) for the sake of reducing notational clutter. See \fullref{def:differentiability/first_variation} and the proof of \fullref{thm:countably_infinite_union_of_countably_infinite_sets} for example of how this is useful in the wild.

    See \fullref{thm:currying_is_natural_isomorphism} for a categorical description of currying.

    \thmitem{def:function/selection}\mcite[52]{DontchevRockafellar2014} If \( f: A \to B \) is a single-valued function, \( F: A \multto B \) is a multi-valued function and \( \gph(f) \subseteq \gph(F) \), we say that \( f \) is a \term{selection} of \( F \).

    \thmitem{def:function/set_of_functions} We denote the set of all single-valued total functions from \( A \) to \( B \) by \( \fun(A, B) \). Other accepted notation is either \( \cat{Set}(A, B) \) (which is consistent with \hyperref[def:category_of_small_sets]{morphisms in the category of sets}) or by \( B^A \) (which is consistent with \hyperref[thm:cardinal_exponentiation_power_set]{cardinal exponentiation}). We abbreviate \( \fun(A, A) \) as \( \fun(A) \)
  \end{thmenum}
\end{definition}

\begin{proposition}\label{thm:def:function}
  \hyperref[def:function]{Single-valued functions} have the following properties when regarded as \hyperref[def:multi_valued_function]{multi-valued functions}:
  \begin{thmenum}
    \thmitem{thm:def:function/total} Single-valued functions are \hyperref[def:multi_valued_function/total]{total} as multi-valued functions. As a consequence, a multi-valued function cannot have a selection unless it is \hyperref[def:multi_valued_function/total]{total}.

    This explains why there is no established terminology for the set \( A \) analogous to \enquote{range} for \( B \) --- we rarely need to differentiate between \( A \) and \( \dom(f) \).

    \thmitem{thm:def:function/composition} The \hyperref[def:multi_valued_function/composition]{composition} of single-valued functions is a single-valued function.

    \thmitem{thm:def:function/preimage} The \hyperref[def:multi_valued_function/large_preimage]{large preimages} and \hyperref[def:multi_valued_function/small_preimage]{small preimages} of single-valued functions are identical. We restrict ourselves to large preimages with the notation \( f^{-1}(x) \) and refer to them simply as preimages.

    Note that preimages are multi-valued in general.
  \end{thmenum}
\end{proposition}
\begin{proof}
  Trivial.
\end{proof}

\begin{definition}\label{def:partial_function}
  A \hyperref[def:multi_valued_function]{Multi-valued function} that is otherwise single-valued, but not necessarily \hyperref[def:multi_valued_function/total]{total} is called a \term{partial function}. That is, \( f: A \to B \) is a partial function if \( f(x) \) has at most one element for every \( x \in A \).
\end{definition}

\begin{definition}\label{def:function_invertibility}
  We introduce the following terminology for invertibility of a (single-valued) function \( f: A \to B \):
  \begin{thmenum}
    \begin{minipage}[t]{0.43\textwidth}
      \thmitem{def:function_invertibility/injective} We say that \( f \) is \term{injective} or \term{one-to-one} if any of the following equivalent conditions hold:
      \begin{thmenum}
        \thmitem{def:function_invertibility/injective/existence} For any \( y \in B \) there exists at \hi{most} one \( x \in A \) such that \( f(x) = y \).

        That is, each point in \( B \) is the image of at most one point in \( A \).
        \newline

        \thmitem{def:function_invertibility/injective/equality} For all \( x_1, x_2 \in A \), the equality \( f(x_1) = f(x_2) \) implies \( x_1 = x_2 \).

        The contrapositive of this statement is that different points in \( A \) have different images under \( f \).

        \thmitem{def:function_invertibility/injective/inverse} The inverse is a partial single-valued function.
      \end{thmenum}
    \end{minipage}
    \hfill
    \begin{minipage}[t]{0.44\textwidth}
      \thmitem{def:function_invertibility/surjective} \( f \) is called \term{surjective} or \term{onto} if any of the following equivalent conditions hold:
      \newline
      \begin{thmenum}[leftmargin=0.9cm]
        \thmitem{def:function_invertibility/surjective/existence} For any \( y \in B \) there exists at \hi{least} one \( x \in A \) such that \( f(x) = y \).

        That is, each point in \( B \) is the image of at least one point in \( A \). Hence, the image of \( f \) is the entire range \( B \).

        \thmitem{def:function_invertibility/surjective/equality} For all \( y_1, y_2 \in B \), the equality \( f^{-1}(y_1) = f^{-1}(y_2) \) implies \( y_1 = y_2 \).

        Without surjectivity, the above holds only for the points in the image of \( f \).

        \thmitem{def:function_invertibility/surjective/inverse} The inverse is a total multi-valued function.
      \end{thmenum}
    \end{minipage}

    \thmitem{def:function_invertibility/bijective} Finally, \( f \) is called \term{bijective} if any of the following equivalent conditions hold:
    \begin{thmenum}
      \thmitem{def:function_invertibility/bijective/direct} It is both injective and surjective.
      \thmitem{def:function_invertibility/bijective/existence} For any \( y \in B \) there exists exactly one \( x \in A \) such that \( f(x) = y \).
      \thmitem{def:function_invertibility/bijective/inverse} The inverse is a total single-valued function.
    \end{thmenum}
  \end{thmenum}
\end{definition}
\begin{defproof}
  The equivalences are trivial to verify.
\end{defproof}

\begin{proposition}\label{thm:function_composition_invertibility}
  The composition of injective (resp. surjective or bijective) functions is injective (resp. surjective or bijective).

  Compare this result to \fullref{thm:def:morphism_invertibility/invertible_composition} and \fullref{thm:function_superposition_invertibility}.
\end{proposition}
\begin{proof}
  Let \( f: A \to B \) and \( g: B \to C \) be arbitrary functions and define \( h \coloneqq g \circ f: A \to C \).

  \SubProofOf[def:function_invertibility/injective/equality]{injectivity} Suppose that \( f \) and \( g \) are injective. We will use \fullref{def:function_invertibility/injective/equality}.

  Let \( x_1, x_2 \in A \) and suppose that \( h(x_1) = h(x_2) \), that is, \( g(f(x_1)) = g(f(x_2)) \). Then \( f(x_1) = f(x_2) \) since \( g \) is injective and \( x_1 = x_2 \) since \( f \) is injective.

  Since \( x_1 \) and \( x_2 \) were arbitrary, we conclude that \( h \) is also injective.

  \SubProofOf[def:function_invertibility/surjective/existence]{surjectivity} Suppose that \( f \) and \( g \) are surjective. We will use \fullref{def:function_invertibility/injective/existence}.

  Let \( z \in C \). Then there exists some \( y \in B \) such that \( g(y) = z \) because \( g \) is surjective and similarly there exists some \( x \in B \) such that \( f(x) = y \) because \( f \) is surjective. Thus, \( h(x) = g(f(x)) = z \).

  Since \( z \) was arbitrary, we conclude that \( h \) is also surjective.

  \SubProofOf[def:function_invertibility/bijective/direct]{bijectivity} We have shown that if \( f \) and \( g \) are both injective or surjective, then, so is \( h \). Hence, if \( f \) and \( g \) are bijective, so is \( h \).
\end{proof}

\begin{lemma}\label{thm:diagonal_product_injectivity}
  Let \( \seq{ f_k }_{k \in \mscrK} \) be an indexed family of functions where \( f_k: A \to B_k \) in \( k \in \mscrK \). Then if at least one of the functions is injective, the \hyperref[def:topological_product]{diagonal product}
  \begin{equation*}
    \begin{aligned}
      f: A \to \prod B_k \\
      f(x) \coloneqq \seq{ f_k(x) }_{k \in \mscrK}
    \end{aligned}
  \end{equation*}
  is also injective.
\end{lemma}
\begin{proof}
  Suppose that \( f_{k_0} \) is injective. Let \( f(x_1) = f(x_2) \). Then \( \seq{ f_k(x_1) }_{k \in \mscrK} = \seq{ f_k(x_2) }_{k \in \mscrK} \) and thus \( f_{k_0}(x) = f_{k_0}(y) \). Since \( f_{k_0} \) is injective, we conclude that \( x_1 = x_2 \). Since \( x_1 \) and \( x_2 \) were arbitrary, we conclude that \( f \) is also injective.
\end{proof}

\begin{proposition}\label{thm:function_superposition_invertibility}
  The superposition of injective functions is injective.

  Compare this result to \fullref{thm:function_composition_invertibility}.
\end{proposition}
\begin{proof}
  Suppose that we are given injective functions \( f_k: A \multto B_k, k = 1, \ldots, n \) and \( g: B_1 \times \cdots \times B_n \multto C \). From \fullref{thm:diagonal_product_injectivity} it follows that the function
  \begin{equation*}
    \begin{aligned}
      d: A \to B_1 \times \cdots \times B_n
      d(x) \coloneqq (f_1(x), \ldots, f_n(x))
    \end{aligned}
  \end{equation*}
  is injective.

  Then from \fullref{thm:function_composition_invertibility} it follows that the desired superposition
  \begin{equation*}
    \begin{aligned}
      &h: A \to C
      &h(x) \coloneqq g(f_1(x), \ldots, f_n(x)) = g(d(x)).
    \end{aligned}
  \end{equation*}
  is injective.
\end{proof}

\begin{remark}\label{rem:multi_valued_functions}
  We can represent a \hyperref[def:multi_valued_function]{multi-valued functions} \( F: A \multto B \) as the \hyperref[def:cartesian_product/indexed_family]{indexed family} \( \set{ F(a) }_{a \in A} \). This indexed family is itself a \hyperref[def:function]{single-valued function} \( G \) from \( A \) to \( \pow(B) \). This is an alternative to our approach to define \hyperref[def:multi_valued_function]{multi-valued functions} as a basic notion. The latter approach is used, for example, in \cite[def. 2.3]{Phelps1993}. The pair \( (x, y) \in A \times B \) belongs to the relation \( F \) if and only if there exists a subset \( Y \subseteq B \) such that \( y \in Y \) and the pair \( (x, Y) \) belongs to the relation \( G \).

  The downside of the latter approach is that notions such as the \hyperref[def:multi_valued_function/image]{image}, \hyperref[def:multi_valued_function/range]{range} and \hyperref[def:multi_valued_function/inverse]{inverse} of the multi-valued function have a very different and much less useful meaning and notions such as \hyperref[def:multi_valued_function/endofunction]{endofunctions} cannot even be defined.
\end{remark}

\begin{proposition}\label{thm:function_image_preimage_composition}
  For any function \( f: A \to B \) we have
  \begin{thmenum}
    \thmitem{thm:function_image_preimage_composition/preimage_of_image} If \( X \subseteq A \), then \( X \subseteq f^{-1}[f[X]] \) with equality holding if \( f \) is injective.
    \thmitem{thm:function_image_preimage_composition/image_of_preimage} If \( Y \subseteq B \), then \( f[f^{-1}[Y]] \subseteq Y \) with equality holding if \( f \) is surjective.
  \end{thmenum}
\end{proposition}
\begin{proof}
  \SubProofOf{thm:function_image_preimage_composition/preimage_of_image} If \( x \in X \), clearly \( x \in f^{-1}(f(x)) \). Thus,
  \begin{equation*}
    X \subseteq f^{-1}[f[X]].
  \end{equation*}

  Now suppose that \( f \) is injective and let \( x \in f^{-1}[f[X]] \). There exists some \( y \in f[X] \) such that \( x \in f^{-1}(y) \) and some \( z \in X \) such that \( y = f(z) \). Since \( f \) is injective and \( f(x) = y = f(z) \), it follows that \( x = z \) and thus \( x \in X \). Since \( x \) was chosen arbitrarily from \( f^{-1}[f[X]] \), we conclude that
  \begin{equation*}
    f^{-1}[f[X]] \subseteq X.
  \end{equation*}

  \SubProofOf{thm:function_image_preimage_composition/image_of_preimage} If \( y \in f[f^{-1}[Y]] \), there exists some \( x \in f^{-1}[Y] \) such that \( f(x) = y \). Furthermore, there also exists some \( t \in Y \) such that \( x \in f^{-1}(t) \). Hence, \( y = f(x) = t \) and \( y \in Y \). Therefore,
  \begin{equation*}
    f[f^{-1}[Y]] \subseteq Y.
  \end{equation*}

  Now suppose that \( f \) is surjective and let \( y \in Y \). Then from surjectivity if follows that there exists some \( x \in X \) such that \( f(x) = y \). Hence, \( x \in f^{-1}(y) \) and \( y = f(x) \in f[f^{-1}(y)] \). Since \( y \) was chosen arbitrarily from \( Y \), we conclude that
  \begin{equation*}
    Y \subseteq f[f^{-1}[Y]].
  \end{equation*}
\end{proof}

\begin{proposition}\label{thm:def:function_image}
  \hyperref[def:multi_valued_function/set_value]{Images of sets} under \( f: A \to B \) have the following basic properties:
  \begin{thmenum}
    \thmitem{thm:def:function_image/monotonicity} If \( A_1 \subseteq A_2 \), then \( f[A_1] \subseteq f[A_2] \).

    \thmitem{thm:def:function_image/union} For any \hyperref[def:cartesian_product/indexed_family]{indexed family} \( \seq{ A_k }_{k \in \mscrK} \subseteq A \) of subsets of \( A \) we have the equality
    \begin{equation}\label{eq:thm:def:function_image/union}
      f\bracks*{ \bigcup_{k \in \mscrK} A_k } = \bigcup_{k \in \mscrK} f[A_k].
    \end{equation}

    \thmitem{thm:def:function_image/intersection} For any indexed family \( \seq{ A_k }_{k \in \mscrK} \) of subsets of \( A \) we have the inclusion
    \begin{equation}\label{eq:thm:def:function_image/intersection}
      f\bracks*{ \bigcap_{k \in \mscrK} A_k } \subseteq \bigcap_{k \in \mscrK} f[A_k].
    \end{equation}

    Equality in \eqref{eq:thm:def:function_image/intersection} holds if \( f \) is \hyperref[def:function_invertibility/injective]{injective}. If \( f \) is not injective, for example if both \( A \) and \( B \) are nonempty, \( A_1 \) and \( A_2 \) are disjoint subsets of \( A \) and \( f[A_1] = f[A_2] = B \), then
    \begin{equation*}
      f[A_1 \cap A_2] = f[\varnothing] = \varnothing \subsetneq f[A_1] \cap f[A_2] = B.
    \end{equation*}

    \thmitem{thm:def:function_image/difference} For any two subsets \( A_1 \) and \( A_2 \) of \( A \) we have the inclusion
    \begin{equation}\label{eq:thm:def:function_image/difference}
      f[A_1] \setminus f[A_2] \subseteq f[A_1 \setminus A_2].
    \end{equation}

    Equality in \eqref{eq:thm:def:function_image/difference} holds if \( f \) is injective. If \( f \) is not injective, for example if \( A_1 \subsetneq A_2 \), but \( f[A_1] = f[A_2] \), then
    \begin{equation*}
      f[A_1] \setminus f[A_2] = \varnothing \subsetneq f[A_1 \setminus A_2].
    \end{equation*}
  \end{thmenum}

  Compare this result to the more well-behaved properties of \hyperref[thm:def:function/preimage]{preimages} described in \fullref{thm:def:function_preimage}.
\end{proposition}
\begin{proof}
  \SubProofOf{thm:def:function_image/monotonicity} If \( x \in A_1 \), then \( x \in A_2 \) and hence \( f(x) \in f[A_2] \). Therefore, \( f[A_1] \subseteq f[A_2] \).

  \SubProofOf{thm:def:function_image/union} If \( x_0 \in A_{k_0} \) for some \( k_0 \in \mscrK \), clearly
  \begin{equation*}
    f(x_0) \in f[A_{k_0}] \subseteq \bigcup_{k \in \mscrK} f[A_k].
  \end{equation*}

  Therefore,
  \begin{equation*}
    f\bracks*{ \bigcup_{k \in \mscrK} A_k } \subseteq \bigcup_{k \in \mscrK} f[A_k].
  \end{equation*}

  Conversely, if \( y_0 \in f[A_{k_0}] \) for some \( k_0 \in \mscrK \), by \fullref{thm:def:function_image/monotonicity} obviously
  \begin{equation*}
    y_0 \in f\bracks*{ \bigcup_{k \in \mscrK} A_k }.
  \end{equation*}

  Therefore,
  \begin{equation*}
    f\bracks*{ \bigcup_{k \in \mscrK} A_k } \supseteq \bigcup_{k \in \mscrK} f[A_k].
  \end{equation*}

  Hence, \eqref{eq:thm:def:function_image/union} holds.

  \SubProofOf{thm:def:function_image/intersection} If \( x_0 \) belongs to \( \bigcap_{k \in \mscrK} A_k \), then \( x_0 \) belongs to \( A_k \) for all \( k \in \mscrK \). It follows that \( f(x_0) \) belongs to \( f[A_k] \) for all \( k \in \mscrK \) and hence to their intersection. Therefore, the inclusion \eqref{eq:thm:def:function_image/intersection} holds.

  Now suppose that \( f \) is injective. Let \( y_0 \) be a point in the intersection \( \bigcap_{k \in \mscrK} f[A_k] \). We thus have \( y_0 \in f[A_k] \) for all \( k \in \mscrK \). Since \( f \) is injective, for each \( k \in \mscrK \) there exists a unique \( x_k \in A_k \) such that \( f(x_k) = y_0 \). Again because of injectivity of \( f \), all these elements are equal because \( f(x_k) = f(x_m) = y_0 \) for \( k, m \in \mscrK \). Hence,
  \begin{equation*}
    y_0 \in f\bracks*{ \bigcap_{k \in \mscrK} A_k }.
  \end{equation*}

  Therefore, the reverse inclusion in \eqref{eq:thm:def:function_image/intersection} holds if \( f \) is injective.

  \SubProofOf{thm:def:function_image/difference} If \( f[A_1] \setminus f[A_2] \) is empty, \eqref{eq:thm:def:function_image/difference} obviously holds. Suppose that it is nonempty and let \( y_0 \in f[A_1] \setminus f[A_2] \).

  Then there exists a point \( x_0 \in A_1 \) such that \( f(x_0) = y_0 \). It cannot be that \( x_0 \in A_2 \) because otherwise \( y_0 = f(x_0) \in f[A_2] \), which would contradict our choice of \( y_0 \). Hence, \( x_0 \in A_1 \setminus A_2 \) and \( y_0 \in f(A_1 \setminus A_2) \).

  Since \( y_0 \) was chosen arbitrarily, we conclude that the inclusion \eqref{eq:thm:def:function_image/difference} holds.

  Conversely, suppose that \( f \) is injective. If \( f(A_1 \setminus A_2) \) is empty, by \eqref{eq:thm:def:function_image/difference} the set \( f[A_1] \setminus f[A_2] \) is also empty and the converse holds.

  Now suppose that it is nonempty and let \( y_0 \in f(A_1 \setminus A_2) \). Then there exists a point \( x_0 \in A_1 \setminus A_2 \) such that \( f(x_0) = y_0 \). Furthermore, since \( f \) is injective, \( x_0 \) is the only preimage of \( y_0 \) and hence \( f(x_0) \in f[A_1] \setminus f[A_2] \), which proves the reverse inclusion in \eqref{eq:thm:def:function_image/difference}.
\end{proof}

\begin{proposition}\label{thm:def:function_preimage}
  Function \hyperref[thm:def:function/preimage]{preimages} have the following basic properties:
  \begin{thmenum}
    \thmitem{thm:def:function_preimage/monotonicity} If \( B_1 \subseteq B_2 \), then \( f^{-1}[B_1] \subseteq f^{-1}[B_2] \).

    \thmitem{thm:def:function_preimage/union} For any \hyperref[def:cartesian_product/indexed_family]{indexed family} \( \seq{ B_k }_{k \in \mscrK} \subseteq B \) of subsets of \( B \) we have the equality
    \begin{equation}\label{eq:thm:def:function_preimage/union}
      f^{-1}\bracks*{ \bigcup_{k \in \mscrK} B_k } = \bigcup_{k \in \mscrK} f^{-1}[B_k].
    \end{equation}

    \thmitem{thm:def:function_preimage/intersection} For any indexed family \( \seq{ B_k }_{k \in \mscrK} \) of subsets of \( B \) we have the equality
    \begin{equation}\label{eq:thm:def:function_preimage/intersection}
      f^{-1}\bracks*{ \bigcap_{k \in \mscrK} B_k } = \bigcap_{k \in \mscrK} f^{-1}[B_k].
    \end{equation}

    \thmitem{thm:def:function_preimage/difference} For any two subsets \( B_1 \) and \( B_2 \) of \( B \) we have the equality
    \begin{equation}\label{eq:thm:def:function_preimage/difference}
      f^{-1}[B_1] \setminus f^{-1}[B_2] = f^{-1}[B_1 \setminus B_2].
    \end{equation}
  \end{thmenum}

  Compare this result to the less well-behaved properties of images described in \fullref{thm:def:function_image}.
\end{proposition}
\begin{proof}
  \SubProofOf{thm:def:function_image/monotonicity} Analogous to \fullref{thm:def:function_image/monotonicity}.

  \SubProofOf{thm:def:function_image/union} Analogous to \fullref{thm:def:function_image/union}.

  \SubProofOf{thm:def:function_image/intersection} If \( y_0 \) belongs to \( \bigcap_{k \in \mscrK} B_k \), then \( y_0 \) belongs to \( B_k \) for all \( k \in \mscrK \). It follows that \( f(y_0) \subseteq f^{-1}[B_k] \) for all \( k \in \mscrK \) and hence it is also a subset of their intersection. Therefore,
  \begin{equation*}
    f^{-1} \bracks*{ \bigcap_{k \in \mscrK} B_k } \subseteq \bigcap_{k \in \mscrK} f^{-1}[B_k].
  \end{equation*}

  Conversely, if \( x_0 \in \bigcap_{k \in \mscrK} f^{-1}[B_k] \), it belongs to \( f^{-1}[B_k] \) for all \( k \in \mscrK \). Clearly then \( f(x_0) \in B_k \) for all \( k \in \mscrK \) and thus \( f(x_0) \in \bigcap_{k \in \mscrK} B_k \). Hence, by \fullref{thm:def:function_preimage/monotonicity} we have
  \begin{equation*}
    f^{-1}(f(x_0))
    \subseteq
    f^{-1}\bracks*{ \bigcap_{k \in \mscrK} B_k }.
  \end{equation*}

  Since \( x_0 \in f^{-1}(f(x_0)) \),
  \begin{equation*}
    x_0 \in f^{-1}\bracks*{ \bigcap_{k \in \mscrK} B_k }.
  \end{equation*}

  Since \( x_0 \) was chosen arbitrarily from \( \bigcap_{k \in \mscrK} f^{-1}[B_k] \), we can conclude that
  \begin{equation*}
    \bigcap_{k \in \mscrK} f^{-1}[B_k] \in f^{-1}\bracks*{ \bigcap_{k \in \mscrK} B_k }.
  \end{equation*}

  Hence, \eqref{eq:thm:def:function_preimage/intersection} holds.

  \SubProofOf{thm:def:function_preimage/difference} If \( y_0 \in B_1 \setminus B_2 \), there exists a point \( x_1 \in B_1 \) such that \( f(x_1) = y_0 \). Aiming at a contradiction, suppose that there exists a point \( x_2 \in f^{-1}[B_2] \) such that \( f(x_2) = y_0 \). Then \( y_0 = f(x_1) = f(x_2) \) implies that \( f^{-1}(y_0) \subseteq f^{-1}[B_1] \cap f^{-1}[B_2] \). \Fullref{thm:def:function_preimage/intersection} then in turn implies that \( f^{-1}(y_0) \subseteq f^{-1}[B_1 \cap B_2] \) and hence by \fullref{thm:def:function_image/monotonicity}
  \begin{equation*}
    y_0 = f(f^{-1}(y_0)) \in f[f^{-1}[B_1 \cap B_2]] = B_1 \cap B_2,
  \end{equation*}
  which contradicts our choice of \( y_0 \). Since the choice of \( y_0 \in B_1 \setminus B_2 \), \( x_1 \in f^{-1}(y_0) \cap B_1 \) and \( x_2 \in f^{-1}(y_0) \cap B_2 \) was arbitrary, the obtained contradiction shows that
  \begin{equation*}
    f^{-1}(B_1 \setminus B_2) \subseteq f^{-1}[B_1] \setminus f^{-1}[B_2].
  \end{equation*}

  Conversely, we have
  \begin{equation*}
    f(f^{-1}[B_1] \setminus f^{-1}[B_2])
    \reloset {\eqref{eq:thm:def:function_image/difference}} \subseteq
    f(f^{-1}(B_1 \setminus B_2))
    \reloset {\ref{thm:function_image_preimage_composition/image_of_preimage}} \subseteq
    B_1 \setminus B_2.
  \end{equation*}

  Hence,
  \begin{equation*}
    f^{-1}[B_1] \setminus f^{-1}[B_2]
    \reloset {\ref{thm:function_image_preimage_composition/preimage_of_image}} \subseteq
    f^{-1}\parens[\Big]{ f\parens[\Big]{ f^{-1}[B_1] \setminus f^{-1}[B_2] } }
    \reloset {\eqref{thm:def:function_preimage/monotonicity}} \subseteq
    f^{-1}(B_1 \setminus B_2).
  \end{equation*}
\end{proof}

\begin{theorem}[Recursion theorem]\label{thm:omega_recursion}\mcite[73]{Enderton1977Sets}
  Let \( A \) be a nonempty set. Suppose that we are given some member \( a_0 \) of \( A \) and some transformation \( T: A \to A \). Then there exists a \hyperref[def:sequence]{sequence} \( f: \omega \to A \) such that
  \begin{itemize}
    \item \( f(n) = a_0 \).
    \item For every \( n \in \omega \) we have \( f(\op{succ}(n)) = T(f(n)) \).
  \end{itemize}

  Note that we do not yet use the notation \( n + 1 \) because we will use this theorem to define addition in the first place.

  See \fullref{rem:natural_number_recursion} for a simpler and more conventional notation for recursion on \( \omega \).
\end{theorem}
\begin{proof}
  Let \( G \subseteq \pow(\omega \times A) \) be the set of all \hyperref[def:partial_function]{partial single-valued functions} \( g: \omega \to A \) such that
  \begin{itemize}
    \item If \( g \) is defined at \( \varnothing \), then \( g(\varnothing) = a_0 \).
    \item For every \( n \in \omega \), if \( g \) is defined at \( \op{succ}(n) \), then \( g \) is also defined at \( n \) and
    \begin{equation*}
      g(\op{succ}(n)) = T(f(n)).
    \end{equation*}
  \end{itemize}

  Clearly \( G \) is nonempty because the function \( \set{ (\varnothing, a_0) } \) belongs to \( G \).

  The conditions imposed on the functions in \( G \) ensure that every function is defined in some \hyperref[def:partially_ordered_set_interval/ray]{initial segment} of the natural numbers. A more obvious approach is to require \( g \) to be defined at \( \op{succ}(n) \) if it is defined at \( n \), however we are trying to prove that such a function exists in the first place.

  Define \( f \coloneqq \bigcup G \). At this point \( f \) is a \hyperref[def:multi_valued_function]{multi-valued function}. We must now show that \( f \) has all the properties that we want.

  \SubProofOf[def:multi_valued_function/total]{totality} First, we will use \fullref{thm:omega_induction} to show that \( f \) is total. Clearly \( \varnothing \in \dom f \).

  Now fix \( n \in \dom f \). Then there exists a function \( g \in G \) defined at \( n \).

  \begin{itemize}
    \item If \( g \) is also defined at \( \op{succ}(n) \), this directly proves that \( \op{succ}(n) \in \dom f \).
    \item If \( g \) is not defined at \( \op{succ}(n) \), consider
    \begin{equation*}
      \widehat g \coloneqq g \cup \set{ (\op{succ}(n), T(g(n)) }.
    \end{equation*}

    The function \( \widehat g \) is again a single-valued partial function, and thus it belongs to \( G \), hence \( \op{succ}(n) \in \dom f \).
  \end{itemize}

  Therefore, \fullref{thm:omega_induction} allows us to conclude \( f: \omega \multto A \) is a total multi-valued function.

  \SubProofOf[def:function]{single-valuedness} Now that we know that \( f \) is total, we will prove that it is single-valued and thus is a function in the usual sense of the term.

  Clearly \( f \) is single-valued at \( \varnothing \).

  Now suppose that \( f \) is single-valued at \( n \). Since \( f \) is total, there exist at least one partial function in \( G \) that is defined at \( \op{succ}(n) \), from which it follows that it is also defined at \( n \).  Let \( g \) and \( h \) both be such (single-valued partial) functions.

  Then
  \begin{equation*}
    g(\op{succ}(n)) = T(g(n)) = T(f(n)) = T(h(n)) = h(\op{succ}(n)),
  \end{equation*}
  hence \( g \) and \( h \) coincide at \( \op{succ}(n) \), which in turn implies that \( f \) is single-valued at \( \op{succ}(n) \).

  Therefore, \fullref{thm:omega_induction} allows us to conclude that \( f \) is a single-valued total function.

  \SubProofOf[def:function]{uniqueness} Now that it is clear that \( f \) satisfies the theorem, we must verify that it is unique.

  Suppose that \( f_1 \) and \( f_2 \) both satisfy the theorem. Clearly \( \varnothing \in H \). Fix some \( n \neq \varnothing \) and suppose that \( f_1(n) = f_2(n) \). Then
  \begin{equation*}
    f_2(\op{succ}(n)) = T(f_1(n)) = T(f_2(n)) = f_2(\op{succ}(n)).
  \end{equation*}

  Therefore, \fullref{thm:omega_induction} allows us to conclude that \( f_1 = f_2 \). So there is at most one function that satisfies the theorem, and we have already shown that \( f \) is such a function.
\end{proof}

\begin{definition}\label{def:omega_operations}
  We will use \fullref{thm:omega_recursion} for defining arithmetic operations for natural numbers. There constructions will be more elaborate than the basic recursive sequences defined in e.g. \fullref{thm:banach_fixed_point_theorem}.

  \begin{thmenum}
    \thmitem{def:omega_operations/addition} We will represent the addition operation \( \oplus \) as follows: fix the first summand \( n \) and then define a function \( \oplus_n: \omega \to \omega \) such that \( k = \oplus_n(m) \) gives us \( n \oplus m = k \). This is a particular instance of \hyperref[def:function/currying]{currying}.

    Fix \( n \in \omega \), let \( A = \omega \) and define
    \begin{equation*}
      \begin{aligned}
        &T_n: \omega \to \omega \\
        &T_n(k) \coloneqq \op{succ}(k).
      \end{aligned}
    \end{equation*}

    Now we use the initial point \( a_n = n \) to construct the function \( \oplus_n \).

    Define the addition function \( \oplus: \omega \times \omega \to \omega \) via its graph
    \begin{equation*}
      \set{ ((n, m), \oplus_n(m)) \given n, m \in \omega }.
    \end{equation*}

    \thmitem{def:omega_operations/multiplication} We define natural number multiplication analogously. For each \( n \in \omega \), define \( \odot_n \) via
    \begin{equation*}
      \begin{aligned}
        &T_n: \omega \to \omega \\
        &T_n(k) \coloneqq k \oplus n.
      \end{aligned}
    \end{equation*}
    and \( a_n = 0 \) and then define \( \odot: \omega \times \omega \to \omega \) via its graph
    \begin{equation*}
      \set{ ((n, m), \odot_n(m)) \given n, m \in \omega }.
    \end{equation*}
  \end{thmenum}
\end{definition}

\begin{theorem}\label{thm:omega_is_model_of_pa}
  The \hyperref[thm:smallest_inductive_set_existence]{smallest inductive set} \( \omega \) is a model of \hyperref[def:peano_arithmetic]{Peano arithmetic} with the following interpretation:
  \begin{thmenum}
    \thmitem{thm:omega_is_model_of_pa/zero} \hyperref[def:peano_arithmetic/zero]{Zero} is interpreted as \( \varnothing \).

    \thmitem{thm:omega_is_model_of_pa/succ} The \hyperref[def:peano_arithmetic/succ]{successor} operation \( s \) is interpreted as \( \op{succ} \).

    \thmitem{thm:omega_is_model_of_pa/plus} \hyperref[def:peano_arithmetic/plus]{Addition} is interpreted by the \( \oplus \) function given in \fullref{def:omega_operations/addition}.

    \thmitem{thm:omega_is_model_of_pa/mult} Similarly, \hyperref[def:peano_arithmetic/mult]{multiplication} is interpreted by \( \odot \) from \fullref{def:omega_operations/multiplication}.
  \end{thmenum}

  This is an extension of \fullref{thm:omega_is_model_of_pa_without_operations}.
\end{theorem}
\begin{proof}
  We have already shown in \fullref{thm:omega_is_model_of_pa_without_operations} that \( \omega \) satisfies the axioms \eqref{eq:def:peano_arithmetic/PA1}-\eqref{eq:def:peano_arithmetic/PA3}.

  \SubProofOf{eq:def:peano_arithmetic/PA4} For each \( n \in \omega \) the starting condition (i.e. \( m = \varnothing \)) in \fullref{def:omega_operations/addition} implies that \( n \oplus \varnothing = n \).

  \SubProofOf{eq:def:peano_arithmetic/PA5} For each \( n \in \omega \) the transformation \( T_n \) in \fullref{def:omega_operations/addition} is defined, so that if \( \oplus_n(m) = k \), then \( \oplus_n(\op{succ}(m)) = \op{succ}(k) \).

  It follows that \( n \oplus \op{succ}(m) = \op{succ}(n \oplus m) \) for all \( n, m \in \omega \).

  \SubProofOf{eq:def:peano_arithmetic/PA6} Analogously to \( \oplus \), the starting condition in \fullref{def:omega_operations/multiplication} implies that \( n \odot \varnothing = \varnothing \) for every \( n \in \omega \).

  \SubProofOf{eq:def:peano_arithmetic/PA7} Analogously to \( \oplus \), for each \( n \in \omega \) the transformation \( T_n \) in \fullref{def:omega_operations/multiplication} is defined, so that if \( \odot_n(m) = k \), then \( \odot_n(\op{succ}(m)) = k \oplus n \).

  It follows that \( n \odot \op{succ}(m) = n \odot m + n \) for all \( n, m \in \omega \).
\end{proof}

\begin{remark}\label{rem:natural_number_recursion}
  With the availability of natural numbers, instead of the tedious constructions in \fullref{def:omega_operations/addition}, we can use a more conventional notation when applying \fullref{thm:omega_recursion}.

  As an example, we can define the Fibonacci sequence. The sequence is motivated by the problem in \fullref{ex:fibonacci_rabbits}. In the notation of \fullref{thm:omega_recursion}, we can define the Fibonacci sequence as follows:
  \begin{equation*}
    \begin{aligned}
      &T: \BbbN \times \BbbN \to \BbbN \times \BbbN \\
      &T(a, b) \coloneqq (b, a + b)
    \end{aligned}
  \end{equation*}
  and
  \begin{equation*}
    a_0 \coloneqq (0, 1).
  \end{equation*}

  The recursion theorem gives us a sequence of pairs
  \begin{equation*}
    \underbrace{(0, 1)}_{a_0}, \underbrace{(1, 1)}_{a_1}, \underbrace{(1, 2)}_{a_2}, \underbrace{(2, 3)}_{a_3}, \underbrace{(3, 5)}_{a_4}, \underbrace{(5, 8)}_{a_5}, \underbrace{(8, 13)}_{a_6}, \ldots
  \end{equation*}

  The pairs are only a technicality because otherwise we would not be able to define the sequence \( \seq{ a_k }_{k=1}^\infty \).

  By taking the second element of each pair, we obtain the sequence
  \begin{equation*}
    \underbrace{1}_{b_0}, \underbrace{1}_{b_1}, \underbrace{2}_{b_2}, \underbrace{3}_{b_3}, \underbrace{5}_{b_4}, \underbrace{8}_{b_5}, \underbrace{13}_{b_6}, \ldots
  \end{equation*}

  In order to obtain the Fibonacci sequence, we must prefix the sequence \( \seq{ b_k }_{k=0}^\infty \) with \( 0 \).

  This is undoubtedly much more complicated than writing
  \begin{equation*}
    b_k \coloneqq \begin{cases}
      0,                &k = 0, \\
      1,                &k = 1, \\
      b_{k-1} + b_{k-2} &k > 1.
    \end{cases}
  \end{equation*}

  To see that the latter notation is merely syntax sugar, note that the other sequence can be written as
  \begin{equation*}
    a_k \coloneqq \begin{cases}
      (0, 1),     &k = 0 \\
      T(a_{k-1}), &k > 1.
    \end{cases}
  \end{equation*}
\end{remark}
