\subsection{Real differentiability}\label{subsec:real_differentiability}

\begin{proposition}\label{thm:real_valued_differentiability}
  Let \( U \subseteq \BbbR^n \) be an open set. A real-valued function \( f: U \to \BbbR \) is differentiable at \( x \) in the direction \( h \) if and only if \( \varphi(t) = f(x + th) \) is right-differentiable at \( 0 \).
\end{proposition}
\begin{proof}
  \begin{equation*}
    f_+'(x)(h) \coloneqq \lim_{t \downarrow 0} \frac {f(x + th) - f(x)} t = \varphi_+'(0)(1).
  \end{equation*}
\end{proof}

\begin{example}[Weierstrass' nowhere differentiable function]\label{ex:weierstrass_nowhere_differentiable_function}\mcite[\textnumero 271]{ФихтенгольцОсновыТом2}
  Let \( a \in (0, 1) \) and \( b \) is a positive odd integer such that
  \begin{equation*}
    ab > 1 + \frac 3 2 \pi.
  \end{equation*}

  Define the function
  \begin{equation*}
    f(x) \coloneqq \sum_{k=0}^\infty a^k \cos(b^k \pi x).
  \end{equation*}

  \begin{figure}[!ht]
    \centering
    \includegraphics{output/ex__weierstrass_nowhere_differentiable_function.pdf}
    \caption
    {
      Plot of the third partial sum of the Weierstrass function with \( a = 0.9 \) and \( b = 7 \) from \( -\sfrac \pi 8 \) to \( \sfrac \pi 8 \).
    }
    \label{fig:ex:weierstrass_nowhere_differentiable_function/plot}
  \end{figure}

  Since \( \cos \) is bounded for real arguments and \( a \in (0, 1) \), each term is uniformly bounded by \( 1 \) and by \fullref{thm:weierstrass_series_criterion}, \( f \) is continuous. However, it is not \hyperref[def:differentiability]{differentiable} at any point. The proof of the latter is involved and will not be given here.
\end{example}

\begin{theorem}[Leibniz' rule]\label{thm:leibniz_rule}
  \todo{Prove}.
\end{theorem}

\begin{theorem}[Lagrange's mean value theorem]\label{thm:lagranges_mean_value_theorem}
  \todo{Prove}.
\end{theorem}

\begin{theorem}[Constant rank theorem]\label{thm:constant_rank_theorem}
  \todo{Prove}.
\end{theorem}

\begin{theorem}[Inverse function theorem]\label{thm:inverse_function_theorem}
  \todo{Prove}.
\end{theorem}

\begin{theorem}[Fundamental theorem of calculus]\label{thm:fundamental_theorem_of_calculus}
  \todo{Prove}.
\end{theorem}
