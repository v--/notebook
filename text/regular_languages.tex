\section{Regular languages}\label{sec:regular_languages}

\paragraph{Finite automata}

\begin{definition}\label{def:finite_automaton}\mcite[27]{Salomaa1973FormalLanguages}
  Fix an \hyperref[def:formal_language]{alphabet} \( \Sigma \). Let \( Q \) be a finite nonempty set, whose members we will call \term{states}. Let \( \delta: Q \times \Sigma \multto Q \) be a \hyperref[def:set_valued_map]{set-valued map}, which we will call a \term{transition function} because, depending on a state in \( Q \) and a symbol in \( \Sigma \), \( \delta \) gives us the possible states towards we can transition.

  Finally, let \( S \) and \( T \) be nonempty sets of states, which we call an \term{initial} and \term{terminal states}, correspondingly.

  We call this entire contraption \( (\Sigma, Q, \delta, I, T) \) a \term[ru=конечный автомат (\cite[159]{Гладкий1973ГрамматикиИЯзыки})]{finite automaton}. It models a real-world device that starts its work from some initial state and, via a sequence of state transitions, reaches some terminal state.

  \begin{figure}[!ht]
    \hfill
    \includegraphics[page=1]{output/def__finite_automaton}
    \hfill
    \includegraphics[page=2]{output/def__finite_automaton}
    \hfill\hfill
    \caption{A \hyperref[def:finite_automaton/determinism]{nondeterministic finite automaton} and its \hyperref[alg:determinization_of_finite_automata]{determinization}, both accepting the language \( \set{ a } \cup \set{ b^n \given n > 0 } \cup \set{ aa b^n \given n > 0 } \).}
    \label{fig:def:finite_automaton}
  \end{figure}

  \begin{thmenum}
    \thmitem{def:finite_automaton/graph}\mimprovised Regard \( \delta \) as a set of triples \( (h_0, l_0, t_0) \). Denote by \( h: \delta \to Q \), \( l: \delta \to \Sigma \) and \( t: \delta \to Q \) the functions that take the corresponding entry out of each triple.

    Then the quadruple \( (Q, \delta, h, t) \) is a \hyperref[def:directed_multigraph]{directed multigraph}, whose arcs are \hyperref[def:labeled_set]{labeled} by \( l \).

    We can identify the automaton with its (multi)graph. When drawing this graph, for example in \cref{fig:def:finite_automaton}, we denote initial states via inward arrows without a source and terminal states via double circles.

    \thmitem{def:finite_automaton/determinism} If there is only one initial state and if \( \delta \) is a \hyperref[def:set_valued_map/partial]{single-valued partial function}, we say that the automata is \term{deterministic}.

    Determinism ensures that there is at most one possible state to transition to, given a current state and a symbol.

    \thmitem{def:finite_automaton/recognition}\mcite[def. 4.1.1]{Savage2008ModelsOfComputation} We say that the automaton \term{accepts} or \term{recognizes} the \hyperref[def:formal_language/string]{string} \( a_1 \cdots a_n \) over \( \mscrA \) if there exists a \hyperref[def:graph_walk/directed]{walk}
    \begin{equation*}
      s \reloset {e_1} \to \anon \reloset {e_2} \to \cdots \reloset {e_{n-1}} \to \anon \reloset {e_n} \to \anon.
    \end{equation*}
    such that
    \begin{itemize}
      \item The label \( l(e_k) \) is \( a_k \) for every \( k = 1, \ldots, n \).
      \item The tail \( t(e_n) \) of the walk is a terminal state from \( T \).
    \end{itemize}

    \thmitem{def:finite_automaton/language} The set of all strings recognized by the automaton is called the language \term{accepted} or \term[ru=(язык) распознается (автоматом) (\cite[45]{Гладкий1973ГрамматикиИЯзыки})]{recognized} by the automaton. We denote this language via \( \mscrL(F) \).

    \thmitem{def:finite_automaton/equivalent}\mcite[152]{Savage2008ModelsOfComputation} We say that two finite automata are \term{equivalent} if they recognize the same language.
  \end{thmenum}
\end{definition}

\begin{definition}\label{def:reverse_language}\mimprovised
  We define the \term{reverse language} of \( \mscrL \) as
  \begin{equation*}
    \op{rev}(\mscrL) \coloneqq \set{ \op{rev}(w) \given w \in \mscrL }.
  \end{equation*}
\end{definition}

\begin{proposition}\label{thm:reverse_language_involution}
  The \hyperref[def:reverse_language]{reverse} of the reverse of a language is the original language.
\end{proposition}
\begin{proof}
  Trivial.
\end{proof}

\begin{definition}\label{def:reverse_finite_automaton}\mimprovised
  We define the \term{reverse automaton} of \( F = (\Sigma, Q, \delta, I, T) \) as
  \begin{equation*}
    \op{rev}(F) \coloneqq (\Sigma, Q, \op{rev}(\delta), T, I),
  \end{equation*}
  where
  \begin{equation*}
    \op{rev}(\delta) \coloneqq \set{ (q, a, p) \given (p, a, q) \in \delta }.
  \end{equation*}
\end{definition}
\begin{comments}
  \item We have
  \begin{equation*}
    p \in \op{rev}(\delta)(q, a) \T{if and only if} q \in \delta(p, a).
  \end{equation*}
\end{comments}

\begin{proposition}\label{thm:reverse_finite_automaton_graph}
  The \hyperref[def:finite_automaton/graph]{multigraph} of a \hyperref[def:reverse_finite_automaton]{reverse automaton} is its \hyperref[def:opposite_directed_multigraph]{opposite multigraph}.
\end{proposition}
\begin{proof}
  Trivial.
\end{proof}

\begin{proposition}\label{thm:reverse_finite_automaton_language}
  For a given \hyperref[def:finite_automaton]{finite automaton} \( F = (\Sigma, Q, \delta, I, T) \), we have
  \begin{equation*}
    \mscrL(\op{rev}(F)) = \op{rev}(\mscrL(F)).
  \end{equation*}
\end{proposition}
\begin{proof}
  Trivial.
\end{proof}

\begin{example}\label{ex:def:finite_automaton}
  We list several examples of \hyperref[def:finite_automaton]{finite automata}:
  \begin{thmenum}
    \thmitem{ex:def:finite_automaton/even} Consider the language form \fullref{ex:def:formal_language/leucine} describing even binary numbers. The language can be recognized by the automaton with multigraph
    \begin{equation*}
      \includegraphics[page=2]{output/ex__def__finite_automaton}
    \end{equation*}

    \thmitem{ex:def:finite_automaton/leucine} Consider the language form \fullref{ex:def:formal_language/leucine} describing Leucine. It can be recognized by the nondeterministic finite automaton
    \begin{equation*}
      \includegraphics[page=1]{output/ex__def__finite_automaton}
    \end{equation*}

    \thmitem{ex:def:finite_automaton/anbn} Consider the language
    \begin{equation*}
      \mscrL \coloneqq \set{ a^n b^n \given n \geq 0 }
    \end{equation*}
    from \fullref{ex:def:formal_language/anbn}. We will shown in \fullref{ex:thm:regular_pumping_lemma/anbn} that no finite automaton recognizes it, but we will give a direct proof here.

    As we will see in \fullref{alg:determinization_of_finite_automata}, a deterministic automaton exists accepting a language if and only if a nondeterministic one exists. Aiming at a contradiction, suppose that there exists some deterministic finite automaton \( F = (\Sigma, Q, \delta, \set{ s }, T) \) whose language is \( \mscrL \). Let \( G \) be its multigraph.

    Since \( s \) is the only initial state, and since \( \mscrL \) contains the empty string, then \( s \) must also be a terminal state.

    Furthermore, \( \mscrL \) contains the string \( ab \), hence there must exist a terminal state \( t_1 \) and an intermediate state \( q_1 \) such that
    \begin{equation*}
      \includegraphics[page=3]{output/ex__def__finite_automaton}
    \end{equation*}

    Furthermore, the above is a \hyperref[def:induced_subgraph]{induced subgraph} of \( G \) --- none of the states above are interconnected by any additional arcs.

    Then, in order for \( F \) to accept \( aabb \), it must have more states. Since the automaton is deterministic, there cannot be another arc with label \enquote{\( a \)} starting at \( s \). Hence, \( q_1 \) is the only node where it is possible to have another arc with label \enquote{\( a \)}.

    Hence, \( G \) either has as a subgraph either
    \begin{equation*}
      \includegraphics[page=4]{output/ex__def__finite_automaton}
    \end{equation*}
    or
    \begin{equation*}
      \includegraphics[page=5]{output/ex__def__finite_automaton}
    \end{equation*}

    In particular, \( F \) must have at least five states in order to recognize \( a^2 b^2 \).

    Continuing by induction, we conclude that in order for \( F \) to recognize \( a^n b^n \), it must have at least \( 2n + 1 \) states. For example, the following automaton recognizes \( a^n b^n \):
    \begin{equation*}
      \includegraphics[page=6]{output/ex__def__finite_automaton}
    \end{equation*}

    But \( \mscrL \) contains strings of arbitrary length. Therefore, no finite automaton is able to recognize \( \mscrL \).
  \end{thmenum}
\end{example}

\begin{algorithm}[Determinization of finite automata]\label{alg:determinization_of_finite_automata}
  Let \( N = (\Sigma, Q, \delta, I, T) \) be a \hyperref[def:finite_automaton]{finite automaton}. We will build an \hyperref[def:finite_automaton/equivalent]{equivalent} \hyperref[def:finite_automaton/determinism]{deterministic automaton} \( \det(N) \).

  This can be achieved by grouping states that would otherwise make the automaton nondeterministic. We will first recursively construct the operator \( \mscrD(P, V) \), which, given a set of states \( P \) and a family of \enquote{visited} sets of states \( V \), produces a family of triples describing the transitions of the deterministic automaton. The family \( V \) helps us avoid cycles when traversing \( N \).

  \begin{thmenum}
    \thmitem{alg:determinization_of_finite_automata/step} Suppose we are given a subset \( P \) of \( P \) and a subset \( V \) of \( \pow(Q) \).

    For every symbol \( a \) in \( \Sigma \), consider the set of all states that we can transition to via \( a \) from some state in \( P \):
    \begin{equation*}
      \delta(P, a) = \bigcup_{p \in P} \delta(p, a) = \set{ q \in Q \given \qexists* {p \in P} q \in \delta(p, a) }.
    \end{equation*}

    Define the set of triples that would become part the graph of the new transition function:
    \begin{equation*}
      E_P \coloneqq \set{ (P, a, \delta(P, a)) \given \delta(P, a) \neq \varnothing }.
    \end{equation*}

    Finally, define
    \begin{equation*}
      \mscrD(P, V) \coloneqq E_P \cup \bigcup \set[\Big]{ \mscrD\parens[\Big]{ \delta(P, a), V \cup \set{ P } } \given* \delta(P, a) \T{is nonempty and is not in} V }.
    \end{equation*}

    Note how we used \( V \) to filter out only those sets of states that have not yet been visited.

    \thmitem{alg:determinization_of_finite_automata/run} Let \( \delta' \coloneqq \mscrD(I, \varnothing) \). Define the new set of states
    \begin{equation*}
      Q' \coloneqq \set[\Big]{ P \given* (P, a, O) \in \delta' } \cup \set[\Big]{ O \given* (P, a, O) \in \delta' }.
    \end{equation*}

    Define also the set of initial states \( I' \coloneqq \set{ I } \) and of terminal states
    \begin{equation*}
      T' \coloneqq \set{ P \in Q' \given P \cap T \neq \varnothing }.
    \end{equation*}

    Then \( \det(N) \coloneqq (\Sigma, Q', \delta', I', T') \) is the desired finite automaton.
  \end{thmenum}
\end{algorithm}
\begin{comments}
  \item This algorithm can be found as \texttt{automata.finite.determinize} in \cite{notebook:code}.
\end{comments}
\begin{defproof}
  We have explicitly made \( I \) the only initial state, and we have grouped arcs with identical labels. Hence, \( F \) is indeed deterministic.

  We must show that \( \mscrL(N) = \mscrL(F) \). We have introduced a special state for every occurrence of several vertices that have incoming arcs with identical labels. Hence, we replace every such group of arcs with a single arc. The two automata should then accept identical languages.
\end{defproof}

\paragraph{Finite automata and regular grammars}

\begin{definition}\label{def:reverse_grammar}\mcite[17]{Salomaa1973FormalLanguages}
  We define the \term{reverse grammar} of the \hyperref[def:chomsky_hierarchy/context_free]{context-free} \( G = (V, \Sigma, \to, S) \) as
  \begin{equation*}
    \op{rev}(G) \coloneqq (V, \Sigma, \to_{\op{rev}}, S),
  \end{equation*}
  where
  \begin{equation*}
    A \to_{\op{rev}} \op{rev}(w) \T{if} A \to w.
  \end{equation*}
\end{definition}
\begin{comments}
  \item For example, the rule \( A \to Ba_1 \cdots a_n \) becomes \( A \to_{\op{rev}} a_n \cdots a_1 B \).
\end{comments}

\begin{proposition}\label{thm:reverse_linear_grammar}
  The \hyperref[def:reverse_grammar]{reverse} of a \hyperref[def:chomsky_hierarchy/regular]{left linear grammar} is \hyperref[def:chomsky_hierarchy/regular]{right linear} and vice versa.
\end{proposition}
\begin{proof}
  Trivial.
\end{proof}

\begin{proposition}\label{thm:reverse_grammar_language}
  For a given \hyperref[def:chomsky_hierarchy/context_free]{context-free} grammar \( G = (V, \Sigma, \to, S) \), we have
  \begin{equation*}
    \mscrL(\op{rev}(G)) = \op{rev}(\mscrL(G)).
  \end{equation*}
\end{proposition}
\begin{proof}
  Trivial.
\end{proof}

\begin{algorithm}[Regular grammar to finite automaton]\label{alg:regular_grammar_to_automaton}
  Let \( G = (V, \Sigma, \to, S) \) be a \hyperref[def:chomsky_hierarchy/regular]{regular grammar}. We will construct a \hyperref[def:finite_automaton]{finite automaton} that \hyperref[def:finite_automaton/language]{accepts} \( \mscrL(G) \).

  \begin{thmenum}
    \thmitem{alg:regular_grammar_to_automaton/init} Let \( G_1 = (V, \Sigma, \to_1, S) \) be \( G_1 \) if it is right linear and \( \op{rev}(G) \) if it is left linear. Then \( G_1 \) is necessarily right linear.

    \thmitem{alg:regular_grammar_to_automaton/epsilon} Let \( G_2 = (V_2, \Sigma, \to_2, S_2) \) be the grammar obtained from \( G_1 \) by removing \( \bnfes \) rules via \Fullref{alg:epsilon_rule_removal}.

    \thmitem{alg:regular_grammar_to_automaton/collapse} Let \( G_3 = (V_2, \Sigma, \to_3, S_2) \) be the grammar obtained from \( G_2 \) by collapsing renaming rules via \fullref{alg:renaming_rule_collapse}.

    \thmitem{alg:regular_grammar_to_automaton/intermediate} Build another intermediate grammar \( G_4 = (V_4, \Sigma, \to_4, S_2) \) as follows:
    \begin{itemize}
      \item Add each \( \bnfes \) rule as-is. There should be at most one \( \bnfes \) rule after \fullref{alg:regular_grammar_to_automaton/epsilon}.
      \item For each rule \( A \to_3 w \), \( w = a_1 \cdots a_n \), consider the sequence of rules
      \begin{align*}
        A       &\to_4 a_1 A_1, \\
        A_1     &\to_4 a_2 A_2, \\
                &\vdots \\
        A_{n-1} &\to_4 a_n,
      \end{align*}
      where \( A_1, \ldots, A_{n-1} \) are new nonterminals.

      \item For each rule \( A \to_3 wB \) with \( \len(w) > 0 \), consider a similar sequence, but the last rule being
      \begin{equation*}
        A_{n-1} \to_4 a_n B.
      \end{equation*}
    \end{itemize}

    Thus, every rule in \( G_4 \) has one of the forms \( A \to_4 a \) or \( A \to_4 a B \) or \( A \to_4 B \).

    \thmitem{alg:regular_grammar_to_automaton/automaton}\mcite[thm. 4.10.1]{Savage2008ModelsOfComputation} Let \( F \) be some new nonterminal symbol. Then the following is a finite automaton that accepts \( \mscrL(G_4) \), and hence also \( G_2 \) and \( G_1 \):
    \begin{itemize}
      \item \( \Sigma \) is the alphabet.
      \item \( V_4 \cup \set{ F } \) is the set of states.
      \item \( S_2 \) the only starting state.
      \item \( F \) is a final state. \( S_2 \) is also a final state if \( \bnfes \in \mscrL(G) \).
      \item Add the following transitions:
      \begin{itemize}
        \item \( \delta(A, a) \coloneqq F \) if \( A \to_4 a \).
        \item \( \delta(A, a) \coloneqq B \) if \( A \to_4 aB \).
      \end{itemize}
    \end{itemize}

    \thmitem{alg:regular_grammar_to_automaton/reverse} If \( G \) is right linear, then \( F \) is the desired automaton because
    \begin{equation*}
      \mscrL(F) = \mscrL(G) = \mscrL(G_4).
    \end{equation*}

    Otherwise, we take the \hyperref[def:reverse_finite_automaton]{reverse automaton} \( \op{rev}(F) \) because
    \begin{equation*}
      \mscrL(\op{rev}(F))
      \reloset {\ref{thm:reverse_finite_automaton_language}} =
      \op{rev}(\mscrL(F))
      =
      \op{rev}(\mscrL(G_4))
      =
      \op{rev}(\mscrL(\op{rev}(G)))
      \reloset {\ref{thm:reverse_grammar_language}} =
      \op{rev}(\op{rev}(\mscrL(G)))
      \reloset {\ref{thm:reverse_language_involution}} =
      \mscrL(G).
    \end{equation*}
  \end{thmenum}
\end{algorithm}
\begin{comments}
  \item This algorithm can be found as \identifier{grammars.regular.to_finite_automaton} in \cite{notebook:code}.
\end{comments}

\begin{algorithm}[Finite automaton to right-linear grammar]\label{alg:finite_automaton_to_right_linear_grammar}\mcite[thm. 4.10.1]{Savage2008ModelsOfComputation}
  Let \( F = (\Sigma, Q, \delta, I, T) \) be a \hyperref[def:finite_automaton]{finite automaton}. We will build a \hyperref[def:chomsky_hierarchy/regular]{right linear grammar} \( G = (V, \Sigma, \to, S) \) that \hyperref[def:formal_grammar/language]{generates} \( \mscrL(F) \).

  \begin{thmenum}
    \thmitem{alg:finite_automaton_to_right_linear_grammar/determinize} Use \fullref{alg:determinization_of_finite_automata} to obtain a deterministic automaton \( \det(F) = (\Sigma, Q', \delta', \set{ I }, T') \) equivalent to \( F \).

    \thmitem{alg:finite_automaton_to_right_linear_grammar/grammar} The following describes the desired grammar:
    \begin{itemize}
      \item \( \Sigma \) is the set of terminals.
      \item \( Q' \) is the set of nonterminals.
      \item \( I \) is the starting nonterminal.
      \item The following are rules:
      \begin{itemize}
        \item \( A \to aB \) if \( \delta'(A, a) = B \).
        \item \( A \to \bnfes \) for each terminal state \( A \in T' \).
      \end{itemize}
    \end{itemize}
  \end{thmenum}
\end{algorithm}
\begin{comments}
  \item This algorithm can be found as \identifier{grammars.regular.from_finite_automaton} in \cite{notebook:code}.
\end{comments}

\paragraph{Regular language characterization}

\begin{proposition}\label{thm:regular_languages}
  The following are equivalent for a given \hyperref[def:formal_language/language]{language}:
  \begin{thmenum}
    \thmitem{thm:regular_languages/right} It is \hyperref[def:formal_grammar/language]{generated} by a \hyperref[def:chomsky_hierarchy/regular]{right linear grammar}.
    \thmitem{thm:regular_languages/left} It is \hyperref[def:formal_grammar/language]{generated} by a \hyperref[def:chomsky_hierarchy/regular]{left linear grammar}.
    \thmitem{thm:regular_languages/nfa} It is \hyperref[def:finite_automaton/language]{recognized} by a (possibly nondeterministic) \hyperref[def:finite_automaton]{finite automaton}.
    \thmitem{thm:regular_languages/dfa} It is \hyperref[def:finite_automaton/language]{recognized} by a \hyperref[def:finite_automaton/determinism]{deterministic finite automaton}.
  \end{thmenum}
\end{proposition}
\begin{proof}
  \ImplicationSubProof{thm:regular_languages/right}{thm:regular_languages/left} Let \( G = (V, \Sigma, \to, S) \) be a right-regular grammar. We will describe a procedure for obtaining an equivalent left-regular grammar.

  \begin{itemize}
    \item \Fullref{alg:regular_grammar_to_automaton} gives us a finite automaton \( F = (\Sigma, Q, \delta, \set{ S }, T) \) such that \( \mscrL(F) = \mscrL(G) \).

    \item Take the \hyperref[def:reverse_finite_automaton]{reverse automaton} \( \op{rev}(F) \) of \( F \).

    \item Determinize \( \op{rev}(F) \) via \fullref{alg:determinization_of_finite_automata}.

    \item Use \fullref{alg:finite_automaton_to_right_linear_grammar} to convert \( \det(\op{rev}(F)) \) to a right-regular grammar \( G' \). At this point, we have
    \begin{equation*}
      \mscrL(G')
      =
      \mscrL(\det(\op{rev}(F)))
      =
      \mscrL(\op{rev}(F))
      \reloset {\ref{thm:reverse_finite_automaton_language}} =
      \op{rev}(\mscrL(F)).
    \end{equation*}

    \item Finally, take the \hyperref[def:reverse_grammar]{reverse grammar} \( \op{rev}(G') \). It is a left-regular grammar as a consequence of \fullref{thm:reverse_linear_grammar}. Furthermore,
    \begin{equation*}
      \mscrL(\op{rev}(G'))
      \reloset {\ref{thm:reverse_grammar_language}} =
      \op{rev}(\mscrL(G'))
      =
      \op{rev}(\op{rev}(\mscrL(F)))
      \reloset {\ref{thm:reverse_language_involution}} =
      \mscrL(F).
    \end{equation*}

    Hence, \( \op{rev}(G') \) is the desired left linear grammar.
  \end{itemize}

  \ImplicationSubProof{thm:regular_languages/left}{thm:regular_languages/nfa} \Fullref{alg:regular_grammar_to_automaton} allows us to convert a left linear grammar to a finite automaton.

  \ImplicationSubProof{thm:regular_languages/nfa}{thm:regular_languages/dfa} \Fullref{alg:determinization_of_finite_automata} allows us to convert a general finite automata to a deterministic one.

  \ImplicationSubProof{thm:regular_languages/dfa}{thm:regular_languages/right} \Fullref{alg:finite_automaton_to_right_linear_grammar} allows us to convert a finite automaton to a right linear grammar.
\end{proof}

\begin{lemma}[Pumping lemma for regular languages]\label{thm:regular_pumping_lemma}\mcite[lemma 4.5.1]{Savage2008ModelsOfComputation}
  For every \hyperref[def:chomsky_hierarchy/regular]{regular language} \( \mscrL \) there exists a constant \( p \) such that any string \( w \) in \( \mscrL \) with \( \len(w) \geq p \) can be decomposed as \( w = x y z \), where \( y \) is nonempty, \( \len(xy) \leq p \) and, for any \hi{nonnegative} integer \( n \), the string \( x y^n z \) belongs to \( \mscrL \).
\end{lemma}
\begin{comments}
  \item In simpler terms, in any regular language some part of a sufficiently long string can be repeated indefinitely.
  \item If the language \( \mscrL \) is finite, the lemma is vacuous because we may simply take \( p \) to be longer than the longest string of \( \mscrL \).
\end{comments}
\begin{proof}
  Fix a deterministic finite automaton \( F = (\Sigma, Q, \delta, \set{ S }, T) \) recognizing \( \mscrL \) and denote by \( p \) the number of states in \( Q \).

  Let \( w \) be a string in \( \mscrL \) of length at least \( p \), and let
  \begin{equation*}
    S = q_0 \reloset {a_1} \to q_1 \reloset {a_2} \to \cdots \reloset {a_n} \to q_n
  \end{equation*}
  be a \hyperref[def:graph_walk/directed]{walk} through the graph of \( F \) witnessing \( w \), that is,
  \begin{equation*}
    w = a_1 \cdots a_n.
  \end{equation*}

  \Fullref{thm:pigeonhole_principle/simple} implies that, since \( n \geq p \), at least one state is visited twice. Let \( i \) be the smallest index such that \( q_i \) is repeated, and let \( j > i \) be the smallest index such that \( q_j = q_i \). Then the states \( q_0, \ldots, q_{j-1} \) are distinct by construction, implying \( j - 1 \leq p \).

  Let
  \begin{equation*}
    \underbrace{a_1 \cdots a_{i-1}}_x \underbrace{a_i \cdots a_j}_y \underbrace{a_{j+1} \cdots a_n}_z.
  \end{equation*}

  Since \( i < j \), the string \( y \) is nonempty. Furthermore, \( \len(xy) = j - 1 \leq p \). To complete the proof of the lemma, we must show that, for any nonnegative integer \( n \), the string \( xy^nz \) is in \( \mscrL \).

  \begin{equation*}
    \begin{aligned}
      \includegraphics[page=1]{output/thm__regular_pumping_lemma}
    \end{aligned}
  \end{equation*}

  The \hyperref[def:graph_walk]{walk}
  \begin{equation*}
    q_i \reloset {a_{i+1}} \to q_{i+1} \reloset {a_{i+2}} \to \cdots \reloset {a_j} \to q_j
  \end{equation*}
  is then \hyperref[def:graph_walk/closed]{closed}, and so it can be traversed as many times as desired (including zero).

  Therefore, for any nonnegative integer \( n \), the string \( xy^nz \) is recognized by \( F \), and thus it is in \( \mscrL \).
\end{proof}

\begin{example}\label{ex:thm:regular_pumping_lemma}
  We list some examples related to \fullref{thm:regular_pumping_lemma}:
  \begin{thmenum}
    \thmitem{ex:thm:regular_pumping_lemma/anbn} Consider the language
    \begin{equation*}
      \mscrL \coloneqq \set{ a^n b^n \given n \geq 0 }
    \end{equation*}
    from \fullref{ex:def:formal_language/anbn}. We have shown in \fullref{ex:def:finite_automaton/anbn} that no finite automaton recognizes it, and thus \fullref{thm:regular_languages} implies that it is not regular. We will give another proof here, based on \fullref{thm:regular_pumping_lemma}.

    Fix any positive integer \( p \) and let \( w = a^n b^n \) be a string of length at least \( p \). Let \( w = xyz \) be an arbitrary decomposition such that \( y \) is nonempty and \( \len(xy) \leq p \).

    \begin{itemize}
      \item If \( y = a^m \) for some positive integer \( m \), then \( x y^2 z = a^{n + m} b^n \), which is not in \( \mscrL \).
      \item If instead \( y = b^m \), then \( x y^2 z = a^n b^{n + m} \), which is again not in \( \mscrL \).
      \item If instead \( y = a^k b^m \), then \( x y^2 z = a^{n-k} a^k b^m a^k b^m b^{n - m} \), which is again not in \( \mscrL \).
    \end{itemize}

    Therefore, the conclusion of \fullref{thm:regular_pumping_lemma} does not hold, and thus \( \mscrL \) is not regular.

    \thmitem{ex:thm:regular_pumping_lemma/even} Consider the language
    \begin{equation*}
      \mscrL \coloneqq \set[\Big]{ w \syn0 \given w \in \set{ \syn0, \syn1 }^* }.
    \end{equation*}
    of even numbers in binary notation from \fullref{ex:def:formal_language/even}. This language is obviously regular. We will use it to validate \fullref{thm:regular_pumping_lemma}.

    For any integer \( p > 2 \) and any string \( w = a_1 \cdots a_n \) of length at least \( p \), we can take \( x = \bnfes \), \( y = a_1 \cdots a_{p-1} \) and \( z = a_p \cdots a_n \). Since \( p \leq n \), \( z \) always contains a trailing \( \syn0 \), and thus, for any positive integer \( k \), the string \( xy^kz \) belongs to \( \mscrL \).

    \thmitem{ex:thm:regular_pumping_lemma/balanced_parentheses} \hyperref[def:chomsky_hierarchy/regular]{Regular grammars} cannot express languages with arbitrarily nested \hyperref[def:paired_delimiters]{balanced delimiters} such as the language of propositional logic described in \fullref{def:propositional_syntax/language} or the language of untyped lambda calculus described in \fullref{def:lambda_term}.

    This will be shown in \fullref{thm:paired_delimiters_not_regular}.
  \end{thmenum}
\end{example}
