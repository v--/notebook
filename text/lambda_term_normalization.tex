\section{Lambda term normalization}\label{sec:lambda_term_normalization}

We will state and prove the results here for untyped \( \synlambda \)-terms. The proofs will hold just as well for typed terms, with the corresponding rules for \( \alpha \)-equivalence and \( \beta \)- and \( \eta \)-reduction replaced by those from \fullref{sec:type_derivation_algorithms}.

\paragraph{Conversion and redexes}

\begin{proposition}\label{thm:beta_eta_redexes}
  Let \enquote{\( {\Anon} \)} be a combination of \enquote{\( \beta \)}, \enquote{\( \eta \)} and possibly \enquote{\( \delta \)}. We will be interested in how redexes of the corresponding kinds behave under \hyperref[con:lambda_conversion]{conversion}.

  \begin{thmenum}
    \thmitem{thm:beta_eta_redexes/substitution} If \( M \) contains a redex, so does \( M[\Bbbs] \) for any substitution \( \Bbbs \).

    Furthermore, if \( M \) is itself a redex, so is \( M[\Bbbs] \).

    \thmitem{thm:beta_eta_redexes/alpha} If \( M \) contains a redex, so does \( N \) whenever \( M \aequiv N \).

    Furthermore, if \( M \) is itself a redex, so is \( N \).

    \thmitem{thm:beta_eta_redexes/reducibility} For a fixed term \( M \), there exists a term \( N \) such that \( M \pred N \) if and only if \( M \) contains a redex of the corresponding kind.
  \end{thmenum}
\end{proposition}
\begin{proof}
  \SubProofOf{thm:beta_eta_redexes/substitution} We will use \fullref{thm:induction_on_rooted_trees} on \( M \) to show, simultaneously on all substitutions \( \Bbbs \), that, if \( M \) contains a redex, so does \( M[\Bbbs] \), and that if \( M \) has a redex, so does \( M[\Bbbs] \).

  \begin{itemize}
    \item If \( M \) is atomic, we have several possibilities:
    \begin{itemize}
      \item If \( M \) is a \( \delta \)-redex, so is \( M[\Bbbs] \) because constants are invariant under substitution.

      \item Otherwise, if \( M \) is a constant, then \( M = M[\Bbbs] \) and neither are \( \beta \)- nor \( \eta \)-redexes.

      \item If \( M \) is a variable, then \( M[\Bbbs] \) can be any term, including a redex, however \( M \) is not a redex and so the proposition holds.
    \end{itemize}

    \item If \( M = AB \), where the inductive hypothesis holds for \( A \) and \( B \), we have several possibilities:
    \begin{itemize}
      \item If \( A = \qabs x E \), then \( M \) itself is a \( \beta \)-redex.

      \Fullref{thm:lambda_substitution_single_rule} implies that
      \begin{equation*}
        A[\Bbbs] = \qabs v E[\Bbbs_{x \mapsto v}]
      \end{equation*}
      for some variable \( v \) such that \( v \not\in \op*{Free}(A) \cup \op*{Free}(A[\Bbbs]) \).

      Then \( M[\Bbbs] = A[\Bbbs] B[\Bbbs] \) is also a \( \beta \)-redex.

      \item Otherwise, if \( A \) contains a redex, by the inductive hypothesis, so does \( A[\Bbbs] \), and thus so does \( M[\Bbbs] \).

      \item Otherwise, \( B \) must contain a redex, and hence so do \( B[\Bbbs] \) and \( M[\Bbbs] \).
    \end{itemize}

    \item If \( M = \qabs a A \), where the inductive hypothesis holds for \( A \), we again have several possibilities.

    We start by noting that
    \begin{equation*}
      M[\Bbbs] = \qabs v A[\Bbbs_{a \mapsto v}]
    \end{equation*}
    for some variable \( v \) such that \( v \not\in \op*{Free}(M) \cup \op*{Free}(M[\Bbbs]) \).

    \begin{itemize}
      \item If \( A = Ea \) and \( a \) is not free in \( E \), then \( M \) is an \( \eta \)-redex.

      \Fullref{thm:lambda_substitution_free_variables} implies that \( v \) is not free in \( E[\Bbbs_{a \mapsto v}] \), thus \( M[\Bbbs] \) is also an \( \eta \)-redex.

      \item Otherwise, \( A \) must contain a redex. By the inductive hypothesis, so does \( A[\Bbbs_{a \mapsto v}] \), and hence also \( M[\Bbbs] \).
    \end{itemize}
  \end{itemize}

  \SubProofOf{thm:beta_eta_redexes/alpha} We use \fullref{thm:induction_on_recursively_defined_relations} on \( M \aequiv N \) simultaneously for all \( N \):
  \begin{itemize}
    \item If \( M \aequiv N \) due to \ref{inf:def:lambda_term_alpha_equivalence/atom}, then \( M \) and \( N \) are identical atomic terms, which cannot contain \( \beta \)- or \( \eta \)-redexes.

    If \( M \) is a \( \delta \)-redex, so is \( N \).

    \item If \( M \aequiv N \) due to \ref{inf:def:lambda_term_alpha_equivalence/app}, then \( M = AB \) and \( N = CD \), where \( A \aequiv C \) and \( B \aequiv D \) and the inductive hypothesis holds for the latter two.

    We have the following possibilities:
    \begin{itemize}
      \item If \( A = (\qabs x E) \), \( M \) itself is a redex. In this case \( C \) is also an abstraction since \( A \aequiv C \) due to either \ref{inf:thm:alpha_equivalence_simplified/lift} or \ref{inf:thm:alpha_equivalence_simplified/ren}. Then \( N \) is also a redex.

      \item Otherwise, either \( A \) or \( B \) or possibly both contain a redex, and the inductive hypothesis allows us to conclude the same for \( C \) and/or \( D \).
    \end{itemize}

    \item If \( M \aequiv N \) due to \ref{inf:thm:alpha_equivalence_simplified/lift}, then \( M = \qabs x A \) and \( N = \qabs x B \) and \( A \aequiv B \), where the inductive hypothesis holds for the latter.

    We have the following possibilities:
    \begin{itemize}
      \item If \( A = Ea \) with \( a \) not free in \( E \), then \( B \) is also an application term whose second subterm is \( a \). It follows that \( B = Fa \) for some term \( F \). \Fullref{thm:def:lambda_term_alpha_equivalence/free} implies that \( a \) is not free in \( F \), hence both \( M \) and \( N \) are \( \eta \)-redexes.

      \item Otherwise, \( A \) must contain a redex and, by the inductive hypothesis, so does \( B \).
    \end{itemize}

    On both cases it follows that \( N \) contains a redex.

    \item If \( M \aequiv N \) due to \ref{inf:thm:alpha_equivalence_simplified/ren}, then \( M = \qabs a A \) and \( N = \qabs b B \), where \( a \) is not free in \( B \) and \( A \aequiv B[b \mapsto a] \). We suppose that the inductive hypothesis holds for the latter.

    We have the following possibilities:
    \begin{itemize}
      \item If \( A = Ea \), then \( B[b \mapsto a] \) is an application term whose second subterm is \( a \). It follows that \( B = Fb \) for some term \( F \). \Fullref{thm:def:lambda_term_alpha_equivalence/free} implies that \( a \) is not free in \( B[b \mapsto a] \), and \fullref{thm:lambda_substitution_free_variables_single} implies that \( b \) is not free in \( B \).

      Then both \( M \) and \( N \) are \( \eta \)-redexes.

      \item Otherwise, \( A \) must contain a redex.

      \Fullref{thm:substitution_on_alpha_equivalent_terms} implies that
      \begin{equation*}
        A[a \mapsto b] \aequiv B[b \mapsto a][a \mapsto b]
      \end{equation*}
      and \fullref{thm:substitution_chain_contraction} implies that, since \( b \) is not free in \( B[b \mapsto a] \) (due to \fullref{thm:def:lambda_term_alpha_equivalence/free}),
      \begin{equation*}
        B[b \mapsto a][a \mapsto b] \aequiv B[\id] = B.
      \end{equation*}

      \Fullref{thm:beta_eta_redexes/substitution} implies that, since \( A \) contains a redex, so does \( A[a \mapsto b] \). The inductive hypothesis implies that \( B \) also does.
    \end{itemize}

    Again, in both cases it follows that \( N \) contains a redex.
  \end{itemize}

  \SubProofOf{thm:beta_eta_redexes/reducibility}

  \SufficiencySubProof* We will use \fullref{thm:induction_on_recursively_defined_relations} on \( M \pred N \) to show that \( M \) contains a redex.

  \begin{itemize}
    \item If \( M \pred N \) due to \ref{inf:def:lambda_term_reduction/app_left}, then \( M = AB \) and \( N = CB \) with \( A \pred C \), where the inductive hypothesis holds for the latter.

    Then \( A \) contains a redex, and so does \( M \).

    \item If \( M \pred N \) due to \ref{inf:def:lambda_term_reduction/app_right}, we proceed analogously.

    \item If \( M \pred N \) due to \ref{inf:def:lambda_term_reduction/abs}, then \( M = \qabs x A \) and \( \qabs x B  \) with \( A \pred B \). Via the inductive hypothesis, we conclude that \( A \), and hence also \( M \), contain a redex.

    \item If \( M \pred N \) due to \ref{inf:def:lambda_term_reduction/alpha}, then \( M \aequiv A \) and \( B \aequiv N \), where \( A \pred B \). \Fullref{thm:beta_eta_redexes/alpha} implies that \( A \) contains a redex. Via the inductive hypothesis, we conclude that \( B \) also contains a redex, and then via \Fullref{thm:beta_eta_redexes/alpha} again we conclude that \( N \) contains a redex.

    \item If \( M \pred N \) due to \ref{inf:def:beta_eta_reduction/beta}, \ref{inf:def:beta_eta_reduction/eta} or \ref{inf:def:delta_reduction}, \( M \) itself is a redex.
  \end{itemize}

  \NecessitySubProof* We will use \fullref{thm:induction_on_rooted_trees} on \( M \) to show that, if \( M \) contains a redex, then there exists a term \( N \) such that \( M \pred N \).

  \begin{itemize}
    \item If \( M \) is atomic, it cannot contain a \( \beta \)- or \( \eta \)-redex.

    If \( M \) is a \( \delta \)-redex, \( M \) reduces to the corresponding \( \delta \)-contractum.

    \item If \( M = AB \) where the inductive hypothesis holds for \( A \) and \( B \), we have two possibilities:
    \begin{itemize}
      \item If \( A = \qabs x E \), then \( M \) is itself a \( \beta \)-redex, and \( M \pred E[x \mapsto B] \) due to \ref{inf:def:beta_eta_reduction/beta}.
      \item Otherwise, either \( A \) or \( B \) contain a redex. If \( A \) does, there exists a term \( C \) such that \( A \pred C \), and then \( M \pred CB \) due to \ref{inf:def:lambda_term_reduction/app_left}. If \( A \) contains no redex, we proceed similarly with \( B \) and \ref{inf:def:lambda_term_reduction/app_right}.
    \end{itemize}

    \item If \( M = \qabs a A \), where the inductive hypothesis holds for \( A \), we again have two possibilities:
    \begin{itemize}
      \item If \( A = Ex \) and \( x \) is not free in \( E \), the \( M \) is itself an \( \eta \)-redex, and \( M \pred E \) due to \ref{inf:def:beta_eta_reduction/eta}.

      \item Otherwise, \( A \) must contain a redex, and thus there exists a term \( B \) such that \( A \pred B \). Then \( M \pred \qabs a B \) due to \ref{inf:thm:alpha_equivalence_simplified/lift}.
    \end{itemize}
  \end{itemize}
\end{proof}

\paragraph{Normalization}

\begin{definition}\label{def:lambda_term_normal_form}\mimprovised
  Let \enquote{\( {\Anon} \)} be a combination of \enquote{\( \beta \)}, \enquote{\( \eta \)} and possibly \enquote{\( \delta \)}.

  We say that the term \( N \) is a \term{\( \Anon \)-normal form} if it satisfies any of the following equivalent conditions:
  \begin{thmenum}
    \thmitem{def:lambda_term_normal_form/redex}\mcite[def. 2.1.34]{Barendregt1984LambdaCalculus} \( N \) contains no redexes of the corresponding kinds.

    \thmitem{def:lambda_term_normal_form/irreducible} There exists no term \( K \) such that \( N \pred K \).
  \end{thmenum}

  If \( M \pred* N \) and \( N \) is a normal form, we refer to it is a normal form of \( M \). If \( M \) has a normal form, we say that it is \term{\( \Anon \)-normalizing}.
\end{definition}
\begin{comments}
  \item \Fullref{def:lambda_term_normal_form/redex} is based on \bycite[def. 2.1.34]{Barendregt1984LambdaCalculus} and \bycite[def. 1B6]{Hindley1997BasicSTT}, who discuss only \( \beta \)-normal forms.

  \item \Fullref{def:lambda_term_normal_form/irreducible} generalizes easily to \hyperref[def:rewriting_system]{abstract rewriting systems}, at which level of generality no concept of redex exists.
\end{comments}
\begin{defproof}
  The equivalence of the definition follows from \fullref{thm:beta_eta_redexes/reducibility}.
\end{defproof}

\begin{proposition}\label{thm:normalization_of_alpha_equivalent_term}
  If \( M \) is \hyperref[def:lambda_term_normal_form]{normalizing} and if \( M \aequiv N \), then \( N \) is also normalizing.
\end{proposition}
\begin{proof}
  If \( N \) is not normalizing, it contains a redex, and, by \fullref{thm:beta_eta_redexes/alpha}, so does \( M \). But we have assumed that \( M \) is normalizing.

  The obtained contradiction shows that \( N \) is normalizing.
\end{proof}

\begin{proposition}\label{thm:lambda_term_normal_form_uniqueness}
  If \( M \) is \hyperref[def:lambda_term_normal_form]{\( \Anon \)-normalizing}, all its normal forms are \hyperref[def:lambda_term_alpha_equivalence]{\( \alpha \)-equivalent}.
\end{proposition}
\begin{proof}
  Let \( N \) and \( K \) be normal forms of \( M \).

  By \fullref{thm:church_rosser_theorem}, there exists a \( \synlambda \)-term \( L \) such that \( N \pred* L \) and \( K \pred* L \).

  But \( N \) is a normal form, hence it cannot reduce via the single-step reduction relation \( {\pred} \). We have required the multi-step relation \( {\pred*} \) to be \hyperref[def:alpha_reflexive]{\( \alpha \)-reflexive}, hence it remains for \( N \) and \( K \) to be \hyperref[def:lambda_term_alpha_equivalence]{\( \alpha \)-equivalent} to \( L \).

  It follows that \( N \aequiv K \) due to the transitivity and symmetry of \( {\aequiv} \).
\end{proof}

\paragraph{Strong normalization}

\begin{definition}\label{def:lambda_term_reduction_graph}\mimprovised
  Fix a \hyperref[def:lambda_term_reduction]{reduction relation} \( {\pred} \) for \( \synlambda \)-terms (with the modified rules from \fullref{def:typed_reduction_rules} in case the terms are typed).

  We define the \term{reduction graph} \( G_{\Anon}(M) \) of the \( \synlambda \)-term \( M \) as the \hyperref[def:directed_graph]{directed graph} \hyperref[def:directed_graph_induced_by_relation]{induced} by the relation \( {\pred} \).

  We call \hyperref[def:graph_walk/walk]{walks} in this graph starting from \( M \) \term{reduction chains}.
\end{definition}
\begin{comments}
  \item The definition and terminology is based on \bycite[def. 2.3.18]{Barendregt1992LambdaCalculiWithTypes}, however Barendregt distinguishes between reductions obtained using different strategies, even if their result coincides. Such a distinction will complicate us, for which reason we will avoid it.
\end{comments}

\begin{proposition}\label{thm:term_normalizing_iff_exists_finite_reduction_chain}
  A \( \synlambda \)-term is \hyperref[def:lambda_term_normal_form]{\( \Anon \)-normalizing} if and only if it has a (possibly empty) \hyperref[def:lambda_term_reduction_graph]{reduction chain} whose tail is a \( \Anon \)-normal form.
\end{proposition}
\begin{proof}
  This is simply a reformulation of \fullref{def:lambda_term_normal_form}.
\end{proof}

\begin{definition}\label{def:strongly_normalizing_lambda_term}\mimprovised
  Let \enquote{\( {\Anon} \)} be a combination of \enquote{\( \beta \)}, \enquote{\( \eta \)} and possibly \enquote{\( \delta \)}.

  We say that the \( \synlambda \)-term \( M \) is \term{strongly \( \Anon \)-normalizing} if the following equivalent conditions hold:
  \begin{thmenum}
    \thmitem{def:strongly_normalizing_lambda_term/chains} All \hyperref[def:lambda_term_reduction_graph]{reduction chains} starting at \( M \) are finite.

    \thmitem{def:strongly_normalizing_lambda_term/graph} The \hyperref[def:lambda_term_reduction_graph]{reduction graph} \( G_{\Anon}(M) \) is finite and \hyperref[def:acyclic_graph]{acyclic}.
  \end{thmenum}
\end{definition}
\begin{comments}
  \item \Fullref{def:strongly_normalizing_lambda_term/chains} is based on \bycite[\S 4.3]{Barendregt1992LambdaCalculiWithTypes} and \bycite[\S 4.2]{Mimram2020ProgramEqualsProof}, both of whom provide no formal definition.
\end{comments}
\begin{defproof}
  The graph \( G_{\Anon}(M) \) is necessarily (weakly) connected, hence, if no infinite walk exists, the graph must be finite. The equivalence then follows from \fullref{thm:acyclic_graph_walk_length}.
\end{defproof}

\begin{proposition}\label{thm:strongly_normalizing_term_is_normalizing}
  Every \hyperref[def:strongly_normalizing_lambda_term]{strongly normalizing} \( \synlambda \)-term is also \hyperref[def:lambda_term_normalization]{normalizing}.
\end{proposition}
\begin{proof}
  Follows from \fullref{thm:term_normalizing_iff_exists_finite_reduction_chain}.
\end{proof}

\begin{proposition}\label{thm:strong_normalization_of_alpha_equivalent_term}
  If \( M \) is \hyperref[def:strongly_normalizing_lambda_term]{strongly normalizing} and if \( M \aequiv N \), then \( N \) is also strongly normalizing.
\end{proposition}
\begin{proof}
  Suppose that \( N \) has an infinite \hyperref[def:lambda_term_reduction_graph]{reduction chain}
  \begin{equation*}
    N = N_0 \pred N_1 \pred N_2 \pred \cdots
  \end{equation*}

  Due to \ref{inf:def:lambda_term_reduction/alpha}, it follows that
  \begin{equation*}
    M \pred N_1 \pred N_2 \pred \cdots
  \end{equation*}

  But \( M \) is strongly normalizing, contradicting the existence of such an infinite chain. Therefore, \( N \) must be strongly normalizing.
\end{proof}

\begin{definition}\label{def:strong_normalization_reducibility_candidate}\mcite[172]{Mimram2020ProgramEqualsProof}
  Let \enquote{\( {\Anon} \)} be a combination of \enquote{\( \beta \)} and \enquote{\( \eta \)}.

  To prove strong normalization in \fullref{thm:simply_typable_terms_are_strongly_normalizing}, we will need to define, for every \hyperref[def:simple_type]{simple type} \( \tau \), a set \( R_\tau \) whose elements we will call \term{\( \Anon \)-reducibility candidates}.

  \begin{thmenum}
    \thmitem{def:strong_normalization_reducibility_candidate/base} For every base type \( \tau \), we define \( R_\tau \) as the set of all \hyperref[def:strongly_normalizing_lambda_term]{strongly \( \Anon \)-normalizing} terms.

    \thmitem{def:strong_normalization_reducibility_candidate/arrow} For every arrow type \( \tau \synimplies \sigma \) , we define \( R_{\tau \synimplies \sigma} \) as the set of all terms \( M \) such that \( MN \) is in \( R_\sigma \) whenever \( N \) is in \( R_\tau \).
  \end{thmenum}
\end{definition}
\begin{comments}
  \item This definition is based on \bycite[172]{Mimram2020ProgramEqualsProof}, which in turn refines \bycite[49]{GirardEtAl1989ProofsAndTypes}. In our definition we allow not only \( \beta \)-, but also \( \eta \)- and \( \beta\eta \)-reduction.
\end{comments}

\begin{lemma}\label{thm:alpha_equivalent_reducibility_candidates}
  \hyperref[def:strong_normalization_reducibility_candidate]{\( \Anon \)-reducibility candidate} sets are closed under \hyperref[def:lambda_term_alpha_equivalence]{\( \alpha \)-equivalence}: if \( M \) belongs to \( R_\tau \) and if \( M \aequiv N \), then \( N \) also belongs to \( R_\tau \).
\end{lemma}
\begin{proof}
  We will use \fullref{thm:induction_on_rooted_trees} on \( \tau \):
  \begin{itemize}
    \item If \( \tau \) is a base type, \( R_\tau \) contains all strongly \( \Anon \)-normalizing terms. If \( M \) is in \( R_\tau \) and \( M \aequiv N \), \fullref{thm:strongly_normalizing_term_is_normalizing} implies that \( N \) is also in \( R_\tau \).

    \item Suppose that \( \tau = \sigma \synimplies \rho \) and that the inductive hypothesis holds for \( \sigma \) and \( \rho \).

    Fix some \( M \) in \( R_{\sigma \synimplies \rho} \) and \( K \) in \( R_\sigma \). By definition, \( MK \) is in \( R_\rho \).

    Suppose that \( M \aequiv N \). We can apply \ref{inf:def:lambda_term_alpha_equivalence/app} to conclude that \( MK \aequiv NK \). By the inductive hypothesis on \( \rho \), we conclude that \( NK \) is also in \( R_\rho \).

    Generalizing on \( K \), we conclude that the condition for \( N \) to belong to \( R_{\sigma \synimplies \rho} \) is satisfied.
  \end{itemize}
\end{proof}

\begin{proposition}\label{thm:def:strong_normalization_reducibility_candidate}\mcite[prop. 4.2.2.1]{Mimram2020ProgramEqualsProof}
  \hyperref[def:strong_normalization_reducibility_candidate]{\( \Anon \)-reducibility candidates} have the following basic properties:
  \begin{thmenum}
    \thmitem[thm:def:strong_normalization_reducibility_candidate/normalizing]{CR1} Every term in \( R_\tau \) is \hyperref[def:strongly_normalizing_lambda_term]{strongly normalizing}.

    \thmitem[thm:def:strong_normalization_reducibility_candidate/reduction]{CR2} For every term \( M \) from \( R_\tau \), if \( M \pred N \), then \( N \) is also in \( R_\tau \).

    \thmitem[thm:def:strong_normalization_reducibility_candidate/reduction_reflection]{CR3} If \( M \) is not an abstraction and if \( N \) is in \( R_\tau \) whenever \( M \pred N \), then \( M \) is also in \( R_\tau \).

    In case \( M \) is a normal form, for example a variable, the condition is vacuous, and the conclusion is simply that \( M \) belongs to \( R_\tau \).
  \end{thmenum}
\end{proposition}
\begin{proof}
  We will use \fullref{thm:induction_on_rooted_trees} on \( \tau \) to prove the statements simultaneously.

  \SubProof{Proof of base case} Suppose that \( \tau \) is a base type.

  \SubProofOf*{thm:def:strong_normalization_reducibility_candidate/normalizing} By definition, \( R_\tau \) is the set of all strongly normalizing terms.

  \SubProofOf*{thm:def:strong_normalization_reducibility_candidate/reduction} Suppose that \( M \) is in \( R_\tau \) and that \( M \pred N \).

  Suppose that there exists an infinite reduction chain
  \begin{equation*}
    N = N_0 \pred N_1 \pred N_2 \pred \cdots
  \end{equation*}

  If we prepend \( M \), we obtain a reduction chain starting at \( M \). But \( M \) is, by \ref{thm:def:strong_normalization_reducibility_candidate/normalizing}, strongly normalizing, hence it cannot have an infinite reduction chain.

  The obtain contradiction implies that \( N \) is also strongly normalizing, i.e. \( N \) is in \( R_\tau \).

  \SubProofOf*{thm:def:strong_normalization_reducibility_candidate/reduction_reflection} Suppose that \( M \) is not an abstraction term (we will not need this assumption how, but will find it very useful when proving the inductive step). Suppose that \( K \) is in \( R_{\sigma \synimplies \rho} \) whenever \( M \pred K \).

  Suppose that there exists an infinite reduction chain
  \begin{equation}\label{eq:thm:def:strong_normalization_reducibility_candidate/proof/base/reduction_reflection}
    M = M_0 \pred M_1 \pred M_2 \pred \cdots
  \end{equation}

  The term \( M_1 \) is in \( R_\tau \) and, by \ref{thm:def:strong_normalization_reducibility_candidate/normalizing}, it is strongly normalizing. This contradicts the existence of \eqref{eq:thm:def:strong_normalization_reducibility_candidate/proof/base/reduction_reflection} since the subchain starting at \( M_1 \) is again infinite.

  The obtained contradiction demonstrates that \( M \) is strongly normalizing.

  \SubProof{Proof of inductive step} Suppose that \( \tau = \sigma \synimplies \rho \), where the inductive hypothesis holds for \( \sigma \) and \( \rho \).

  \SubProofOf*{thm:def:strong_normalization_reducibility_candidate/normalizing} Fix a term \( M \) from \( R_{\sigma \synimplies \rho} \). By definition, \( MK \) belongs to \( R_\rho \) whenever \( K \) is in \( R_\sigma \).

  Fix some variable \( x \). \ref{thm:def:strong_normalization_reducibility_candidate/reduction_reflection} on \( \sigma \), for which the inductive hypothesis holds, implies that \( x \) is in \( R_\sigma \). Then \( Mx \) belongs to \( R_\rho \).

  Suppose that \( M \) has an infinite reduction chain
  \begin{equation*}
    M = M_0 \pred M_1 \pred M_2 \pred \cdots
  \end{equation*}

  Applying \ref{inf:def:lambda_term_reduction/app_left}, we obtain
  \begin{equation*}
    Mx = M_0 x \pred M_1 x \pred M_2 x \pred \cdots
  \end{equation*}

  But, via the inductive hypothesis on \( \rho \), \ref{thm:def:strong_normalization_reducibility_candidate/normalizing} implies that \( Mx \) is strongly normalizing, contradicting the existence of the latter reduction chain.

  It remains for \( M \) to be strongly normalizing.

  \SubProofOf*{thm:def:strong_normalization_reducibility_candidate/reduction} Suppose that \( M \) is in \( R_{\sigma \synimplies \rho} \) and that \( M \pred N \).

  Fix a term \( K \) from \( R_\sigma \). By definition of \( R_{\sigma \synimplies \rho} \), \( MK \) must be in \( R_\rho \).

  We can apply \ref{inf:def:lambda_term_reduction/app_left} to conclude that \( MK \pred NK \). By the inductive hypothesis on \( \rho \), \ref{thm:def:strong_normalization_reducibility_candidate/reduction} allows us to conclude that \( NK \) is also in \( R_\rho \).

  Generalizing on \( K \), we conclude that the condition for \( N \) to belongs to \( R_{\sigma \synimplies \rho} \) is satisfied.

  \SubProofOf*{thm:def:strong_normalization_reducibility_candidate/reduction_reflection} Let \( M \) be a variable or application term. Suppose that \( K \) is in \( R_{\sigma \synimplies \rho} \) whenever \( M \pred K \).

  We must show that \( M \) also belongs to \( R_{\sigma \synimplies \rho} \). For this, we need to show that \( MN \) is in \( R_\rho \) whenever \( N \) is in \( R_\sigma \).

  Fix a term \( N \). At this point it is important that \( M \) is not an abstraction because \( MN \) cannot be a \( \beta \)-redex. If \( MN \pred P \), \fullref{thm:single_step_reduction_deconstruction/app} implies that \( P = KL \) and either \( M \pred K \) and \( N \aequiv L \) or \( M \aequiv K \) and \( N \pred L \).

  Suppose that \( N \) belongs to \( R_\sigma \). By the inductive hypothesis on \( \sigma \), \ref{thm:def:strong_normalization_reducibility_candidate/normalizing} implies that \( N \) is strongly normalizing. Then there exists a maximal length \( n \) of reduction chains starting at \( N \). We will use induction on \( n \) in a subtle way to show that \( KL \) belongs to \( R_\rho \).

  \begin{itemize}
    \item Both in the base case \( n = 0 \) and in the inductive step, we will need to handle the following possibility:

    If \( M \pred K \) and \( N \aequiv L \), by assumption, \( K \) is in \( R_{\sigma \synimplies \rho} \). Since we have chosen \( N \) to belong to \( R_\sigma \), by \fullref{thm:alpha_equivalent_reducibility_candidates}, so does \( L \).

    Then \( KL \) must belong to \( R_\rho \) by definition of \( R_{\sigma \synimplies \rho} \).

    \item If \( n > 0 \), it is also possible that \( M \aequiv K \) and \( N \pred L \). We suppose that the inductive hypothesis holds for choices of \( N \) with a maximum reduction chain length shorter than \( n \) --- such a term is \( L \).

    \ref{thm:def:strong_normalization_reducibility_candidate/reduction} on \( \sigma \) implies that \( L \) is in \( R_\sigma \). We can apply \ref{inf:def:lambda_term_reduction/app_left} to conclude that \( ML \pred KL \). The inductive hypothesis on \( L \) then implies that \( ML \) belongs to \( R_\rho \).

    Since \( M \aequiv K \), applying \ref{inf:def:lambda_term_alpha_equivalence/app}, we get \( ML \aequiv KL \). \Fullref{thm:alpha_equivalent_reducibility_candidates} then implies that \( KL \) belongs to \( R_\rho \).
  \end{itemize}

  With this the induction on \( n \) is complete. At this point we have concluded that \( P = KL \) belongs to \( R_\rho \) whenever \( MN \pred P \). By the inductive hypothesis on \( \rho \), \ref{thm:def:strong_normalization_reducibility_candidate/reduction_reflection} implies that, since \( KL \) was arbitrary, \( MN \) also belongs to \( R_\rho \).

  Generalizing on \( N \), we conclude that the condition for \( M \) to belongs to \( R_{\sigma \synimplies \rho} \) is satisfied.
\end{proof}

\begin{lemma}\label{thm:reducibility_candidate_abstraction}\mcite[lemma 4.2.2.2]{Mimram2020ProgramEqualsProof}
  Fix a term \( M \) and a variable \( x \).

  If \( M[x \mapsto N] \) belongs to \( R_\sigma \) for every term \( N \) in \( R_\tau \), then \( \qabs x M \) (resp. \( \qabs {x^\tau} M \)) belongs to \( R_{\tau \synimplies \sigma} \).
\end{lemma}
\begin{comments}
  \item Unlike \incite[thm. 4.2.2.5]{Mimram2020ProgramEqualsProof}, we handle \( \eta \)- and \( \beta\eta \)-reduction.
\end{comments}
\begin{proof}
  Denote \( \qabs x M \) by \( P \). To show that \( P \) belongs to \( R_{\tau \synimplies \sigma} \), we must show that \( P N \) belongs to \( R_\sigma \) for every term \( N \) in \( R_\tau \).

  By \ref{thm:def:strong_normalization_reducibility_candidate/normalizing}, both \( M \) and \( N \) are strongly normalizing. Let \( m \) be the maximum reduction chain length for \( M \), and similarly let \( n \) be the length for \( N \). We will use induction on \( m + n \) to show that, if \( P N \pred S \), then \( P N \) is in \( R_\sigma \).

  In the base case \( m + n = 0 \), \( P N \) must be a normal form. \ref{thm:def:strong_normalization_reducibility_candidate/reduction_reflection} then implies that \( P N \) is in \( R_\sigma \).

  If \( m + n > 0 \), we must consider several cases. \Fullref{thm:single_step_reduction_deconstruction/app} lists the following possibilities for \( P N \pred S \):
  \begin{itemize}
    \item In the case \fullref{thm:single_step_reduction_deconstruction/app/beta} for \( P N \pred S \), we have \( S \aequiv M[x \mapsto N] \).

    Since \( M[x \mapsto N] \) is in \( R_\sigma \) by supposition, \fullref{thm:alpha_equivalent_reducibility_candidates} implies that \( S \) is also in \( R_\sigma \).

    \item In the case \fullref{thm:single_step_reduction_deconstruction/app/left} for \( P N \pred S \), we have \( S = Q L \), where \( P \pred Q \) and \( N \aequiv L \).

    \Fullref{thm:single_step_reduction_deconstruction/abs} lists the following possibilities for \( P \pred Q \):
    \begin{itemize}
      \item In the case \fullref{thm:single_step_reduction_deconstruction/abs/eta} for \( P \pred Q \), we have \( M \aequiv Qx \) and \( x \not\in \op*{Free}(Q) \), i.e. we have \( P \aequiv \qabs x Qx \).

      By supposition, \( M[x \mapsto N] = QN \) belongs to \( R_\sigma \), and \fullref{thm:alpha_equivalent_reducibility_candidates} implies that \( S = Q L \) is also in \( R_\sigma \).

      \item In the case \fullref{thm:single_step_reduction_deconstruction/abs/lift} for \( P \pred Q \), we have \( Q \aequiv \qabs x T \), where \( M \pred T \).

      By the inductive hypothesis on (the maximal lengths of reduction chains on) \( T \) and \( N \), we conclude that \( Q N \) is in \( R_\sigma \).

      Then \fullref{thm:alpha_equivalent_reducibility_candidates} implies that \( S = Q L \) is also in \( R_\sigma \).
    \end{itemize}

    \item In the case \fullref{thm:single_step_reduction_deconstruction/app/right} for \( P N \pred S \), we have \( S = Q L \), where \( P \aequiv Q \) and \( N \pred L \).

    By the inductive hypothesis on \( M \) and \( L \), \( P L \) is in \( R_\sigma \). \Fullref{thm:alpha_equivalent_reducibility_candidates} then implies that \( S = Q L \) is also in \( R_\sigma \).
  \end{itemize}
\end{proof}

\begin{lemma}\label{thm:reducibility_candidate_substitution}\mcite[lemma 4.2.2.3]{Mimram2020ProgramEqualsProof}
  Suppose we are given the \hyperref[def:simple_type_derivability]{derivation} \( \Gamma \vdash M: \tau \).

  For every \hyperref[def:lambda_term_substitution]{substitution} \( \Bbbs \) such that \( \Bbbs(v) \) is in \( R_\sigma \) for every assumption \( v: \sigma \) in \( \Gamma \), the term \( M[\Bbbs] \) belongs to \( R_\tau \).
\end{lemma}
\begin{proof}
  Let \( T \) be a \hyperref[def:type_derivation_tree]{derivation tree} for \( M: \tau \) from \( \Gamma \). Fix a substitution \( \Bbbs \) such that, whenever \( v: \sigma \), \( \Bbbs(v) \) belongs to \( R_\sigma \).

  Note that we say nothing about the type of \( \Bbbs(v) \). Thus, \Fullref{alg:simply_typed_substitution} is inapplicable.

  We will use \fullref{thm:induction_on_rooted_trees} on \( T \), simultaneously on all substitutions, to conclude that \( M[\Bbbs] \) belongs to \( R_\tau \).
  \begin{itemize}
    \item If \( T \) is an assumption tree, \( M \) is a variable, and \( M[\Bbbs] = \Bbbs(M) \) belongs to \( R_\tau \) by supposition.

    \item If \( T \) is an application tree for \ref{inf:def:arrow_typing_rules/elim}, then \( M = NK \) and \( T \) has subtrees \( T_N \) and \( T_K \) deriving \( N: \sigma \synimplies \tau \) and \( K: \sigma \) for some type \( \sigma \). We suppose that the inductive hypothesis holds for \( T_N \) and \( T_K \).

    Since \( M[\Bbbs] = N[\Bbbs] \thinspace K[\Bbbs] \), the inductive hypothesis implies that \( N[\Bbbs] \) is in \( R_{\sigma \synimplies \tau} \) and \( K[\Bbbs] \) is in \( R_\sigma \).

    By definition of \( R_{\sigma \synimplies \tau} \), \( M[\Bbbs] \) must belong to \( R_\tau \).

    \item If \( T \) is an assumption tree for \ref{inf:def:arrow_typing_rules/intro/explicit}, then \( \tau = \sigma \synimplies \rho \) and \( M = \qabs x N \). The tree \( T \) must have a discharge assertion \( x: \sigma \) and subtree \( T_N \) deriving \( N: \rho \).

    \Fullref{thm:lambda_substitution_single_rule} implies that
    \begin{equation*}
      M[\Bbbs] = \qabs v N[\Bbbs_{x \mapsto v}]
    \end{equation*}
    for some variable \( v \) such that \( v \not\in \op*{Free}(M) \cup \op*{Free}(M[\Bbbs]) \).

    The inductive hypothesis on \( T_N \) implies that \( N[\Bbbs_{x \mapsto K}] \) belongs to \( R_\rho \) whenever \( K \) belongs to \( R_\sigma \) (in particular, if \( K = v \)).

    We use \fullref{alg:simply_typed_substitution} to construct from \( T_N \) a tree \( S \) for \( N[\Bbbs_{x \mapsto v}] \). Compared to \( T_N \), in \( S \) the assumption \( x: \sigma \) gets replaced with \( v: \sigma \). Since \( v \) is free in \( N \), \fullref{thm:substitution_chain_contraction} implies that
    \begin{equation*}
      N[\Bbbs_{x \mapsto K}] \aequiv N[\Bbbs_{x \mapsto v}][v \mapsto K].
    \end{equation*}

    Thus, for any term \( K \) in \( R_\sigma \), \fullref{thm:alpha_equivalent_reducibility_candidates} implies that \( N[\Bbbs_{x \mapsto v}][v \mapsto K] \) belongs to \( R_\rho \).

    We can then use \fullref{thm:reducibility_candidate_abstraction} to conclude that \( M[\Bbbs] \) is in \( R_{\sigma \synimplies \tau} \), as desired.
  \end{itemize}
\end{proof}

\begin{proposition}\label{thm:reducibility_candidate_adequacy}\mcite[prop. 4.2.2.4]{Mimram2020ProgramEqualsProof}
  If \( \Gamma \vdash M: \tau \), then \( M \) belongs to the set of \hyperref[thm:def:strong_normalization_reducibility_candidate]{reducibility candidates} \( R_\tau \).
\end{proposition}
\begin{proof}
  \ref{thm:def:strong_normalization_reducibility_candidate/reduction_reflection} implies that, for every assumption \( v: \sigma \) in \( \Gamma \), \( v \) belongs to \( R_\sigma \).

  \Fullref{thm:reducibility_candidate_substitution} then implies that \( M = M[\id] \) belongs to \( R_\tau \).
\end{proof}

\begin{theorem}[Simply typable terms are strongly normalizing]\label{thm:simply_typable_terms_are_strongly_normalizing}\mcite[thm. 4.2.2.5]{Mimram2020ProgramEqualsProof}
  Let \enquote{\( {\Anon} \)} be a combination of \enquote{\( \beta \)} and \enquote{\( \eta \)}.

  Then every (\hyperref[def:typed_lambda_term]{typed} or \hyperref[def:lambda_term]{untyped}) \( \synlambda \)-term \hyperref[def:typability]{typable} via the arrow typing rules from \fullref{def:arrow_typing_rules} is \hyperref[def:strongly_normalizing_lambda_term]{\( \Anon \)-strongly normalizing}.
\end{theorem}
\begin{comments}
  \item Unlike \incite[thm. 4.2.2.5]{Mimram2020ProgramEqualsProof}, we handle \( \eta \)- and \( \beta\eta \)-reduction. This is done in \fullref{thm:reducibility_candidate_abstraction}.
\end{comments}
\begin{proof}
  If \( \Gamma \vdash M: \tau \), \fullref{thm:reducibility_candidate_adequacy} implies that \( M \) belongs to \( R_\tau \). \Fullref{thm:def:strong_normalization_reducibility_candidate/normalizing} then implies that \( M \) is strongly normalizing.
\end{proof}
