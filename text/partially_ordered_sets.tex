\subsection{Partially ordered sets}\label{subsec:partially_ordered_sets}

\hyperref[def:preordered_set]{Preordered sets} are simple to define and arise naturally, for example \fullref{def:lindenbaum_tarski_algebra} or \fullref{thm:semiring_divisibility_order}. Unfortunately, they require certain uniqueness considerations, as discussed in \fullref{ex:preorder_nonuniqueness}. If we want to overcome them, we arrive at the concept of a partially ordered set.

\begin{definition}\label{def:partially_ordered_set}\mcite[2]{Roman2008}
  We define a \term{partially ordered set} to be a \hyperref[def:preordered_set]{preordered set} that also satisfies the antisymmetry condition \eqref{eq:def:binary_relation/antisymmetric}.

  The theory, homomorphisms and category \( \cat{Pos} \) are inherited from \fullref{def:preordered_set} with one additional axiom added.
\end{definition}
\begin{comments}
  \item The category \( \cat{Pos} \) is isomorphic to that of small skeletal preorder categories --- see \fullref{thm:order_category_isomorphism}.
\end{comments}

\begin{proposition}\label{thm:comparables_reflect_inequalities}
  For any \hyperref[def:order_homomorphism/increasing]{increasing function} \( f: P \to Q \) between partially ordered sets, if \( x \) and \( y \) are comparable elements, \( f(x) < f(y) \) implies \( x < y \).
\end{proposition}
\begin{proof}
  Let \( f(x) <_Q f(y) \) and suppose that \( x \geq y \). Since \( f \) is a monotone map, we have \( f(x) \geq_Q f(y) \), which is a contradiction.
\end{proof}

\begin{proposition}\label{thm:strict_partial_order}
  For an arbitrary set \( P \) and binary relation \( \leq \) on \( P \), the pair \( (P, \leq) \) is a \hyperref[def:partially_ordered_set]{partially ordered set} if and only the relation \( < \), defined via \eqref{eq:def:preordered_set/compatibility_strict}, is \hyperref[def:binary_relation/irreflexive]{irreflexive} and \hyperref[def:binary_relation/transitive]{transitive}.

  We refer to \( < \) as a \term{strict partial order}.
\end{proposition}
\begin{proof}
  \SufficiencySubProof Let \( \leq \) be a partial order. We will show that \( < \) is a strict partial order.

  \SubProofOf*[def:binary_relation/transitive]{transitivity} The relation \( < \) is \hyperref[def:binary_relation/transitive]{transitive}. To see this, let \( x < y \) and \( y < z \). In particular, \( x \leq y \) and \( y \leq z \). From transitivity, we have \( x \leq z \).

  Additionally, \( x \neq y \) and \( y \neq z \). Assume that \( x = z \). From reflexivity of \( \leq \) we have \( z \leq x \) and, since \( y \leq z \), from transitivity we obtain \( y \leq x \). But since \( x \leq y \), from the antisymmetry of \( \leq \), we have \( x = y \), which contradicts the assumption that \( x < y \).

  Therefore, \( x < z \).

  \SubProofOf*[def:binary_relation/irreflexive]{irreflexivity}. Follows directly from reflexivity of \( \leq \) and the compatibility condition.

  Since the right side is false, the left side \( x < x \) is also false.

  \NecessitySubProof Let \( < \) be a strict partial order. We will show that \( \leq \) is a partial order.

  \SubProofOf*[def:binary_relation/reflexive]{reflexivity} Fix \( x \in P \) and assume that \( x \not\leq x \). Then \( x \neq x \) which contradicts the reflexivity of equality. Hence, \( x \leq x \).

  \SubProofOf*[def:binary_relation/antisymmetric]{antisymmetry} Let \( x \leq y \) and \( y \leq x \), that is, either \( x = y \) or both \( x < y \) and \( y < x \) hold. Assume the latter. By the transitivity of \( \leq \), we have \( x < x \), which contradicts the irreflexivity of \( < \). Hence, \( x = y \).

  \SubProofOf*[def:binary_relation/transitive]{transitivity} Let \( x \leq y \) and \( y \leq z \). Then we have four cases depending on which of \( x \), \( y \) and \( z \) are equal. Since both relations \( < \) and \( = \) are transitive, it follows that in all four cases \( x \leq z \).
\end{proof}

\begin{definition}\label{def:hasse_diagram}\mcite[4]{Roman2008}
  It is usually easier to define small finite partially ordered sets by drawing graphs than by enumerating all relation pairs. Let \( (P, \leq) \) be a finite partially ordered set. The relation \( \leq \) may also be regarded as the set of edges of a \hyperref[def:quiver/simple]{directed graph}. The graph \( (P, \red^T(\leq)) \), whose edges are the \hyperref[def:relation_closures/transitive]{transitive reduction} of \( \leq \), is called the \term{Hasse graph} or \term{Hasse diagram} of \( P \).
\end{definition}
\begin{comments}
  \item The term \enquote{Hasse diagram} is usually associated with drawings. By convention, no arrowheads for denoting directions are drawn on the Hasse graph despite the graph being directed; instead, edges always point upwards. See \fullref{ex:def:hasse_diagram}.
\end{comments}

\begin{example}\label{ex:def:hasse_diagram}
  Consider the partial order over \( \set{ a, b, c, d, e } \) defined via
  \begin{equation}\label{eq:ex:def:hasse_diagram/partially ordered set}
    \hi{a \leq c},\quad \hi{a \leq d},\quad a \leq e,\quad \hi{b \leq d},\quad b \leq e,\quad \hi{d \leq e}.
  \end{equation}

  The corresponding Hasse graph includes only the highlighted edges. The rest of the edges can be restored from transitivity. In this case, the Hasse graph has edges
  \begin{equation}\label{eq:ex:def:hasse_diagram/hasse_graph}
    \set{ a \to c, a \to d, b \to d, d \to e }
  \end{equation}

  \begin{figure}[!ht]
    \centering
    \includegraphics[page=1]{output/ex__def__hasse_diagram.pdf}
    \caption{A drawing of the Hasse diagram \eqref{eq:ex:def:hasse_diagram/hasse_graph}}
    \label{fig:ex:def:hasse_diagram}
  \end{figure}
\end{example}

\begin{example}\label{ex:preorder_nonuniqueness}
  Consider the preordered set \( P \) in \cref{fig:ex:preorder_nonuniqueness} in which \( b \leq c \) and \( c \leq b \), but \( b \neq c \). We cannot properly draw a \hyperref[def:hasse_diagram]{Hasse diagram} because we have the restriction that \( c \) is drawn (strictly) higher than \( b \) if \( c > b \) and that \( c \) is drawn lower than \( b \) if \( c < b \). We face a similar problem formally, for example in the definition of a \hyperref[def:lindenbaum_tarski_algebra]{Lindenbaum-Tarski algebra} of a \hyperref[def:first_order_theory]{logical theory}, where the preorder \( \vdash \) allows \( \varphi \vdash \psi \) and \( \psi \vdash \varphi \), but still \( \varphi \neq \psi \). Thus, we have nonuniqueness --- every tautology is a largest element with respect to \( \vdash \), while we want to have a single largest element for the sake of building a tidier theory.

  If we are only interested in members of \( P \) up to the equivalence relation from \fullref{thm:preorder_to_partial_order}, it is easy to factor \( P \) by the equivalence relation and obtain a partially ordered set. In the language of graph theory, if we have \hyperref[def:quiver_path/cycle]{directed cycles} that we may wish to avoid, we can contract each directed cycle into a single vertex, at which point the graph becomes acyclic. This corresponds to \hyperref[def:quiver_condensation]{graph condensation}.

  The formulation and proof of correctness of this process can be found in \fullref{thm:preorder_to_partial_order} and an example can be found in \cref{fig:ex:preorder_nonuniqueness}.

  \begin{figure}[!ht]
    \hfill
    \includegraphics[page=1]{output/ex__preorder_nonuniqueness.pdf}
    \hfill
    \includegraphics[page=2]{output/ex__preorder_nonuniqueness.pdf}
    \hfill\hfill
    \caption{A preordered set and its induced partially ordered set.}
    \label{fig:ex:preorder_nonuniqueness}
  \end{figure}
\end{example}

\begin{proposition}\label{thm:preorder_to_partial_order}
  Let \( (P, \leq) \) be a preordered set. Define the relation \( x \sim y \) to hold if \( x \leq y \) and \( y \leq x \).
  \begin{equation}\label{eq:thm:preorder_to_partial_order/equivalence}
    x \cong y \T{if and only if} x \leq y \T{and} y \leq x.
  \end{equation}

  Define the relation \( [x] \preceq [y] \) on the \hyperref[def:equivalence_relation/quotient]{quotient set} \( P / \sim \) to hold if \( x \leq y \).

  The pair \( (P / \sim, \preceq) \) is then a \hyperref[def:partially_ordered_set]{partially ordered set}.
\end{proposition}
\begin{comments}
  \item The relation \( \sim \) is the intersection of the relation \( \leq \) with its \hyperref[def:binary_relation/converse]{inverse}.
\end{comments}
\begin{proof}
  The relation \( \preceq \) is well-defined. Indeed, let \( x \sim x' \) and \( y \sim y' \), that is, both \( x \leq x' \) and \( x' \leq x \), and similarly for \( y \). If \( x \leq y \), transitivity implies \( x \leq y \leq y' \). But \( x' \leq x \), hence \( x' \leq y' \).

  It is then clear that \( \preceq \) is a partial order because it inherits reflexivity and transitivity from \( \leq \) and antisymmetry is imposed by taking quotient sets --- equality in \( P / \sim \) holds precisely when \( \sim \) holds in \( P \).

  Thus, \( (P / \sim, \preceq) \) is indeed a partially ordered set.
\end{proof}

\begin{definition}\label{def:extremal_points}
  We introduce the following terminology for extremal elements of a partially ordered set \( P \). Analogous definition can be given for preordered sets, but the nonuniqueness problems outlined in \fullref{ex:preorder_nonuniqueness} highlight that there are sometimes difficulties in doing so.

  The notions on the left and on the right are \hyperref[thm:preorder_duality]{dual}, but we discuss both nonetheless.

  \begin{thmenum}
    \thmitem{def:extremal_points/upper_and_lower_bounds}\mcite[7]{Roman2008}
    \begin{minipage}[t]{0.45\textwidth}
      An \term{upper bound} for the set \( A \subseteq P \) is an element \( x_0 \in P \) such that \( x \leq x_0 \) for every \( x \in A \). Note that \( x_0 \) does not in general belong to \( A \). An upper bound is called \term{strict} if it does not belong to \( A \).

      If \( A \) has at least one upper bound, it is called \term{bounded from above}.

      Every element is vacuously an upper bound of \( A = \varnothing \).

      In \cref{fig:ex:def:hasse_diagram}, the set \( A = \set{ a, b } \) is bounded from above by both \( d \) and \( e \), but the entire partially ordered set has no upper bound.
    \end{minipage}
    \hspace{0.02\textwidth}
    \begin{minipage}[t]{0.45\textwidth}
      Dually, \( x_0 \in P \) is a \term{lower bound} of \( A \) if \( x_0 \leq x \) for every \( x \in A \). A lower bound is called \term{strict} if it does not belong to \( A \). If \( A \) has a lower bound, it is called \term{bounded from below}.

      If \( A \) is bounded both from below and from above, we say that \( A \) is \term{bounded}.

      Every element is vacuously a lower bound of \( A = \varnothing \). Hence, the empty set is bounded.

      In \cref{fig:ex:def:hasse_diagram}, the entire partially ordered set has no lower bound. The set \( A = \set{ c, d } \) is bounded from below by \( a \), but not from above, hence \( A \) is not bounded.
    \end{minipage}

    \thmitem{def:extremal_points/maximal_and_minimal_element}\mcite[6]{Roman2008}
    \begin{minipage}[t]{0.45\textwidth}
      A \term{maximal element} for the set \( A \subseteq P \) is a member \( x_0 \) of \( A \) such that there is no greater element in \( A \) than \( x_0 \). More precisely, \( x_0 \) is a maximal element of \( A \) if for every element \( x \in A \) such that \( x \leq x_0 \) we have \( x = x_0 \).

      In \cref{fig:ex:def:hasse_diagram}, the entire partially ordered set has two incomparable maximal elements --- \( c \) and \( e \).
    \end{minipage}
    \hspace{0.02\textwidth}
    \begin{minipage}[t]{0.45\textwidth}
      The member \( x_0 \in A \) is a \term{minimal element} of \( A \) if for every \( x \in A \) such that \( x \leq x_0 \) we have \( x = x_0 \).

      The empty set cannot have maximal or minimal elements because it has no members.

      In \cref{fig:ex:def:hasse_diagram}, the entire partially ordered set has two incomparable minimal elements --- \( a \) and \( b \).
    \end{minipage}

    \thmitem{def:extremal_points/maximum_and_minimum}\mcite[6]{Roman2008}
    \begin{minipage}[t]{0.45\textwidth}
      The \term{maximum} \( \max A \) or \term{greatest element} of \( A \subseteq P \), if it exists, is an upper bound of \( A \) that belongs to \( A \). A maximum is necessarily a maximal element because \( x_0 \leq x \) only holds for \( x = x_0 \), which also demonstrates uniqueness of \( x_0 \). See \fullref{ex:unique_maximal_element_that_is_not_maximum} for a unique maximal element that is not a maximum.

      In a totally ordered set, maxima and maximal elements coincide --- see \fullref{thm:totally_ordered_minimal_element_is_minimum}.
    \end{minipage}
    \hspace{0.02\textwidth}
    \begin{minipage}[t]{0.45\textwidth}
      The \term{minimum} \( \min A \) of \( A \), also called \term{smallest element} or \term{least element}, is a lower bound that belongs to \( A \).

      The empty set cannot have a maximum or minimum because it has no members.

      In \cref{fig:ex:def:hasse_diagram}, the entire partially ordered set has no minimum, but the set \( A = P \setminus \set{ b } = \set{ a, b, d, e } \) has \( a \) as its minimum.
    \end{minipage}

    \thmitem{def:extremal_points/supremum_and_infimum}\mcite[7]{Roman2008}
    \begin{minipage}[t]{0.45\textwidth}
      The \term{supremum} \( \sup A \) of \( A \subseteq P \), if it exists, is its least upper bound of \( A \), i.e. the \hyperref[def:extremal_points/maximum_and_minimum]{minimum} of the set of its \hyperref[def:extremal_points/upper_and_lower_bounds]{upper bounds}.

      In \cref{fig:ex:def:hasse_diagram}, the entire partially ordered set has no upper bound, so it cannot possibly have a supremum. Obviously every supremum is a maximum, but the converse is not true. We already noted that both \( d \) and \( e \) are upper bounds of the set \( \set{ a, b } \) and since \( d \leq e \), we conclude that \( d \) is the supremum of \( \set{ a, b } \), yet the set \( \set{ a, b } \) has no maximum.
    \end{minipage}
    \hspace{0.02\textwidth}
    \begin{minipage}[t]{0.45\textwidth}
      The \term{infimum} \( \inf A \) of \( A \subseteq P \) is its greatest lower bound.

      An infimum may fail to exist because the set of lower bounds is nonempty, but has no maximum. This can happen if, for example, we had \( b \leq c \) in \cref{fig:ex:def:hasse_diagram}, in which case both \( a \) and \( b \) would be maximal lower bounds of \( \set{ c, d } \), but none of them would be a greatest lower bound because they are incomparable. This arises in practice in \fullref{rem:lattice_of_principal_ideals}.
    \end{minipage}

    \thmitem{def:extremal_points/top_and_bottom}\mcite[6]{Roman2008}
    \begin{minipage}[t]{0.45\textwidth}
      If it exists, the maximum of the entire partially ordered set \( P \) is usually denoted by \( \top \) and called the \term{global maximum} or \term{top} element of \( P \). Since \( \top \) is the maximum of \( P \), it is also the supremum of \( P \).

      Since every member of \( P \) is a lower bound of \( \varnothing \), the greatest lower bound is the maximum of \( P \).

      In conclusion,
      \begin{equation*}
        \top = \max P = \sup P = \inf \varnothing.
      \end{equation*}

      If \( \top \) exists, we say that the partially ordered set \( P \) itself is bounded from above.
    \end{minipage}
    \hspace{0.02\textwidth}
    \begin{minipage}[t]{0.45\textwidth}
      Dually, the minimum of \( P \) is usually denoted by \( \bot \) and called the \term{global minimum} or \term{bottom} element of \( P \).

      The supremum of the empty set is the least of the upper bounds of the empty set, i.e. the minimum of \( P \), which is \( \bot \).

      In conclusion,
      \begin{equation*}
        \bot = \min P = \inf P = \sup \varnothing.
      \end{equation*}

      If \( \bot \) exists, we say that the partially ordered set \( P \) itself is bounded from below.
    \end{minipage}
  \end{thmenum}
\end{definition}

\begin{example}\label{ex:unique_maximal_element_that_is_not_maximum}\mcite{MathSE:unique_maximal_element_that_is_not_maximum}
  For a more extreme example of the interplay between maximal elements and maxima, adjoin \( \BbbZ \) under the usual order with a new sentinel element \( \star \). Let \( \star \) satisfy reflexivity, but not be in relation with any integer. Then \( \star \) is a unique maximal element of \( \BbbZ \cup \set{ \star } \), yet the latter set has no largest element.

  This phenomenon is impossible in finite partially ordered sets where a unique maximal element is always a maximum. Totally ordered sets also prevent this --- see \fullref{thm:totally_ordered_minimal_element_is_minimum}.
\end{example}

\begin{definition}\label{def:partial_order_chain}\mcite[5]{Roman2008}
  Fix a partially ordered set \( (P, \leq) \).

  \begin{thmenum}
    \thmitem{def:partial_order_chain/chain} We say that the set \( A \) is a \term{chain} if every pair of elements of \( A \) is comparable.

    \thmitem{def:partial_order_chain/length} We define the \term{length} \( \len(A) \) of a chain \( A \) as the unique \hyperref[def:cardinal]{cardinal} satisfying \( \len(A) + 1 = \card(A) \). That is, \( \len(A) = \card(A) - 1 \) for finite chains and \( \len(A) = \card(A) \) otherwise.

    \thmitem{def:partial_order_chain/height}\mcite[15]{Roman2008} We define the \term{height} of \( P \) as the supremum among the lengths of all its chains.

    \thmitem{def:partial_order_chain/antichain} We say that \( A \) is an \term{antichain} if no pair of elements of \( A \) is comparable.

    \thmitem{def:partial_order_chain/width} We define the \term{width} of \( P \) as the supremum of the cardinalities of antichains.
  \end{thmenum}
\end{definition}
\begin{comments}
  \item Unlike for the height, for the width we do not subtract \( 1 \) from the cardinality of antichains.
  \item \Fullref{thm:union_of_set_of_cardinals} implies that the height and width are well-defined cardinals.
\end{comments}

\begin{example}\label{ex:def:partial_order_chain}
  The height of the partially ordered set in \cref{fig:ex:def:hasse_diagram} is \( 2 \), and it is reached by the chains \( \set{ a, d, e } \) and \( \set{ b, d, e } \). The width is \( 2 \) are is reached by \( \set{ a, b } \), \( \set{ b, e } \), \( \set{ c, d } \) and \( \set{ c, e } \).
\end{example}

\begin{definition}\label{def:order_interval}\mcite[2]{Roman2008}
  Fix a \hyperref[def:partially_ordered_set]{partially ordered set} \( (P, \leq) \). For any \( a, b \in P \) with \( a \leq b \), we define the following related partially ordered sets:

  \begin{thmenum}
    \thmitem{def:order_interval/ray}\mcite[28]{Roman2008} The \term{open rays}, also called the \term{open initial segment} and \term{open final segment}, are defined as
    \begin{equation*}
      \begin{aligned}
        (a, \infty) \coloneqq \set{ x \in P \given b > a } \eqqcolon P_{>a},
        \\
        (-\infty, b) \coloneqq \set{ x \in P \given x < b } \eqqcolon P_{<b}.
      \end{aligned}
    \end{equation*}

    The notation on the left assumes that the sentinel symbols \( \infty \) and \( -\infty \) are adjoined to \( P \) as in the case of the \hyperref[def:extended_real_numbers]{extended real numbers}. This convention is widespread for unbounded ordered rings representing numbers --- for example \hyperref[def:integers]{\( \BbbZ \)}, \hyperref[def:rational_numbers]{\( \BbbQ \)} and \hyperref[def:real_numbers]{\( \BbbR \)}. The term \enquote{ray} is used in this context due to the connection with \hyperref[def:geometric_ray]{geometric rays}.

    The notation on the right is more general and is widespread for abstract partial orders, most notably \hyperref[def:well_ordered_set]{well-ordered sets}. In the latter context, they are usually referred to as \enquote{initial/final segments}.

    The \term{closed rays} and \term{closed initial/final segments} are defined analogously as
    \begin{equation*}
      \begin{aligned}
        [a, \infty) \coloneqq \set{ x \in P \given x \geq a } \eqqcolon P_{\geq a},
        \\
        (-\infty, b] \coloneqq \set{ x \in P \given x \leq b } \eqqcolon P_{\leq b}.
      \end{aligned}
    \end{equation*}

    \thmitem{def:order_interval/closed} The \term{closed interval} with endpoints \( a \) and \( b \) is
    \begin{equation*}
      [a, b] \coloneqq \set{ x \in P \given a \leq x \leq b } = P_{\geq a} \cap P_{\leq b}.
    \end{equation*}

    We implicitly assume that \( a \leq b \), but this is not strictly necessary --- \( [a, b] \) is an empty set otherwise. We also assume that \( a \neq \bot \) and \( b \neq \top \).

    \thmitem{def:order_interval/open} The \term{open interval} with endpoints \( a \) and \( b \) is
    \begin{equation*}
      (a, b) \coloneqq \set{ x \in P \given a < x < b } = P_{> a} \cap P_{< b}.
    \end{equation*}

    We implicitly assume that \( a < b \), but this is also not strictly necessary. We also assume that \( a \neq \bot \) and \( b \neq \top \).

    \thmitem{def:order_interval/half_open} The \term{half-open intervals} are
    \begin{equation*}
      \begin{aligned}
        (a, b] \coloneqq \set{ x \in P \given a < x \leq b } = P_{> a} \cap P_{\leq b},
        \\
        [a, b) \coloneqq \set{ x \in P \given a \leq x < b } = P_{\geq a} \cap P_{< b}.
      \end{aligned}
    \end{equation*}

    As with the other cases, we assume that \( a < b \) and that \( a \neq \bot \) and \( b \neq \top \).
  \end{thmenum}
\end{definition}

\begin{proposition}\label{thm:partially_ordered_cofinal_equivalences}
  Let \( (P, \leq) \) be a \hyperref[def:extremal_points/upper_and_lower_bounds]{bounded from above} partially ordered set and let \( A \subseteq P \). Then \( A \) is \hyperref[def:cofinal_set]{cofinal} if and only if it contains the \hyperref[def:extremal_points/top_and_bottom]{top element} \( \top \).
\end{proposition}
\begin{comments}
  \item There is a somewhat similar result case of totally ordered sets in \fullref{thm:totally_ordered_cofinal_equivalences}.
\end{comments}
\begin{proof}
  \SufficiencySubProof Let \( A \) be a cofinal set. Then \( A \) must contain an element \( x \) such that \( \top \leq x \). But \( \top \) is a maximum and hence \( x = \top \) and thus \( \top \in A \).

  \NecessitySubProof Let \( A \) be a set containing \( \top \). Then for any \( x \in P \) we have \( x \leq \top \) and hence \( A \) is cofinal.
\end{proof}

\begin{definition}\label{def:lexicographic_order}\mcite[exmpl. 1.1]{Roman2008}
  Let \( (P, \leq_P) \) and \( (Q, \leq_Q) \) be partially ordered sets.

  We define the \term{lexicographic order} on \( P \times Q \) as
  \begin{equation}\label{eq:def:lexicographic_order}
    (a, b) \prec (c, d) \thickspace \T{if and only if} \thickspace \parens[\Big]{ a <_P c \T{or} \parens[\Big]{ a = c \T{and} b <_Q d } }
  \end{equation}
  and the \term{reverse lexicographic order} as
  \begin{equation}\label{eq:def:lexicographic_order/reverse}
    (a, b) \prec (c, d) \thickspace \T{if and only if} \thickspace \parens[\Big]{ b <_Q d \T{or} \parens[\Big]{ b = d \T{and} a <_P c } }.
  \end{equation}

  We can now use natural number recursion to extend this to arbitrary \( n \)-tuples.
\end{definition}
\begin{comments}
  \item Lexicographic orders are also known as \term{dictionary orders}.
\end{comments}

\begin{proposition}\label{thm:lexicographic_order_is_partial_order}
  The \hyperref[eq:def:lexicographic_order]{lexicographic} and \hyperref[eq:def:lexicographic_order/reverse]{reverse lexicographic} orders are \hyperref[def:partially_ordered_set]{strict partial order} relations.
\end{proposition}
\begin{comments}
  \item An analogous result holds for total orders (\fullref{thm:total_lexicographic_order_is_total_order}) and well-ordered sets (\fullref{thm:well_ordered_lexicographic_order_is_well_ordered}).
\end{comments}
\begin{proof}
  \SubProofOf[def:binary_relation/irreflexive]{irreflexivity} Trivial.

  \SubProofOf[def:binary_relation/transitive]{transitivity} Let \( \prec \) be a lexicographic order on \( P \times Q \).

  If \( (a, b) \prec (c, d) \) and \( (c, d) \prec (e, f) \), then
  \begin{itemize}
    \item If \( a < c \), then \( a < c \leq e \) and thus \( (a, b) \prec (e, f) \).

    \item If \( a = c \) and \( b < d \), then \( a \leq e \) and \( b < d \leq f \) and thus \( (a, b) \prec (e, f) \).
  \end{itemize}

   The proof for the reverse lexicographic order is analogous.
\end{proof}

\begin{example}\label{ex:def:lexicographic_order}
  We list several examples of \hyperref[def:lexicographic_order]{lexicographic ordering}:
  \begin{thmenum}
    \thmitem{ex:def:lexicographic_order/words} In a real-world dictionary like a thesaurus, the word \enquote{bright} comes after \enquote{blight}.

    Denote the English alphabet by \( A \). Then the lexicographic ordering on the Cartesian power \( A^6 \) suggests that since the first letters of \enquote{bright} and \enquote{blight} are equal, we should compare the second letters, which would lead us to the usual thesaurus order.

    \thmitem{ex:def:lexicographic_order/heterogenous_words} Real-world dictionaries list words with different numbers of letters. It is not clear how to compare \enquote{car} to \enquote{carboy}.

    We can add a sentinel symbol \( \star \) to the English alphabet \( A \) to denote empty spaces. That is, \enquote{car} in \( A^6 \) becomes \enquote{car\( \star \star \star \)}.

    Now, if we specify how \( \star \) should compare to the letters, we can use lexicographic ordering to compare the two words. For example, making \( \star \) the bottom of the alphabet will make \enquote{car} precede \enquote{carboy}, while making it the top would do the converse.

    \thmitem{ex:def:lexicographic_order/rectangle} A slightly more relevant example for mathematics is the lexicographic ordering
    \begin{equation*}
      AB < AD < BC < CD
    \end{equation*}
    of the names of the edges of a \hyperref[def:parallelogram/rectangle]{rectangle}. The reverse lexicographic ordering is
    \begin{equation*}
      AB < BC < AD < CD.
    \end{equation*}

    \thmitem{ex:def:lexicographic_order/triangle} For the sides of a \hyperref[def:triangle]{triangle}, we have \( AB < AC < BC \) for both orderings.

    \thmitem{ex:def:lexicographic_order/quiver} The edges of the graph in \eqref{eq:fig:def:quiver} are numbered in lexicographic order, which also happens to be the reverse lexicographic order.

    \thmitem{ex:def:lexicographic_order/ordinals} \Fullref{thm:ordinal_addition_disjoin_union} and \fullref{thm:ordinal_multiplication_cartesian_product} contain more interesting applications of lexicographic orders.
  \end{thmenum}
\end{example}

\begin{definition}\label{def:inflationary_function}\mcite[28]{Roman2008}
  An \hyperref[def:multi_valued_function/endofunction]{endofunction} \( f \) on a \hyperref[def:partially_ordered_set]{partially ordered set} \( (P, \leq) \) is said to be \term{inflationary} if \( x \leq f(x) \) for every \( x \in P \) and \term{anti-inflationary} if \( f(x) \leq x \) instead.
\end{definition}

\begin{definition}\label{def:closure_operator}\mcite[55]{Roman2008}
  Let \( (P, \leq) \) be a \hyperref[def:partially_ordered_set]{partially ordered set}. We say that the function \( \cl: P \to P \) is an \term{closure operator} if it is \hyperref[def:inflationary_function]{inflationary}, \hyperref[def:magma/idempotent]{idempotent} and \hyperref[def:order_homomorphism/increasing]{order-preserving}.

  We say that \( x \) is \term{closed} with respect to \( \cl \) if \( x = \cl(x) \).
\end{definition}

\begin{proposition}\label{thm:closure_operator_minimality}\mcite[55]{Roman2008}
  For any \hyperref[def:closure_operator]{closure operator} \( \cl: P \to P \) on any \hyperref[def:partially_ordered_set]{partially ordered set} \( (P, \leq) \) and any \( x \in P \) it holds that \( \cl(x) \) is the smallest closed element of \( P \) containing \( x \).
\end{proposition}
\begin{proof}
  Since \( f \) is \hyperref[def:inflationary_function]{inflationary}, it is clear that \( x \leq \cl(x) \).

  Let \( y \in P \) be a closed element such that \( x \leq y \leq \cl(x) \). From the monotonicity of \( f \) we have that
  \begin{equation*}
    \cl(x) \leq \underbrace{\cl(y)}_{y} \leq \underbrace{\cl(\cl(x))}_{\cl(x)}.
  \end{equation*}

  Therefore, \( y = \cl(x) \).
\end{proof}

\begin{lemma}[Zorn's lemma]\label{thm:zorns_lemma}\mcite[thm. 5.4]{Jech2003}
  If every \hyperref[def:partial_order_chain]{chain} in a nonempty \hyperref[def:partially_ordered_set]{partially ordered set} has an \hyperref[def:extremal_points/upper_and_lower_bounds]{upper bound}, then the entire set has a \hyperref[def:extremal_points/maximal_and_minimal_element]{maximal element}.
\end{lemma}
\begin{comments}
  \item Within \hyperref[def:zfc]{\logic{ZF}}, this theorem is equivalent to the \hyperref[def:zfc/choice]{axiom of choice} --- see \fullref{thm:axiom_of_choice_equivalences/zorns_lemma}.

  \item Zorn's lemma is usually stated and used only in a \hyperref[thm:boolean_algebra_of_subsets]{lattice of sets}.
\end{comments}
\begin{proof}
  \ImplicationSubProof[def:zfc/choice]{the axiom of choice}[thm:zorns_lemma]{Zorn's lemma} Let \( (P, \leq) \) be a partially ordered set in which every chain has an upper bound. Aiming at a contradiction, suppose that \( P \) has no maximal elements.

  Let \( \mscrC \) be the set of all chains in \( P \). Define the multi-valued map \( F: \mscrC \multto P \) that assigns to each set in \( P \) the set of all its strict upper bounds.

  Since every chain \( C \) has an upper bound, say \( u \), and since \( P \) has no maximal element, there exists some element strictly larger than \( u \). Hence, \( F \) is a total multi-valued map. By \fullref{thm:existence_of_multi_valued_function_selection}, there exists a single-valued selection \( f: \pow(P) \to P \) of \( F \).

  By \fullref{thm:hartogs_lemma}, there exists a smallest \hyperref[def:ordinal]{ordinal} \( \alpha \) such that no function from \( \alpha \) to \( P \) is injective. Using \fullref{thm:bounded_transfinite_recursion}, we can define
  \begin{equation*}
    \begin{aligned}
      &g: \alpha \to P, \\
      &g(\beta) \coloneqq f(\set{ g(\gamma) \given \gamma < \beta }) = f(g[\beta]). \\
    \end{aligned}
  \end{equation*}

  If \( \beta_1 < \beta_2 \), then
  \begin{equation*}
    \set{ g(\gamma) \given \gamma < \beta_1 }
    \subsetneq
    \set{ g(\gamma) \given \gamma < \beta_2 },
  \end{equation*}
  and thus
  \begin{equation*}
    g(\beta_1) < g(\beta_2).
  \end{equation*}

  Then the images under \( g \) of different members \( \beta \) of \( \alpha \) are different, i.e. \( g \) is injective. But this contradicts our choice of \( \alpha \).

  The obtained contradiction shows that \( P \) has a maximal element.

  \ImplicationSubProof[thm:zorns_lemma]{Zorn's lemma}[def:zfc/choice]{axiom of choice} Let \( \mscrA \) be a family of nonempty sets. Let \( \mscrF \) be the set of all \hyperref[def:partial_function]{partial single-valued functions} from \( \mscrA \) to \( \bigcup \mscrA \) with the subset ordering. That is, \( f \leq g \) if \( \dom(f) \subseteq \dom(g) \) for \( f, g \in \mscrF \).

  Clearly every chain has a maximum - a total single-valued function. Then \( \mscrF \) itself has a maximal element by Zorn's lemma. This maximal element is necessarily a total function because otherwise it would not be maximal.

  Then this is the desired choice function for the family \( \mscrA \).
\end{proof}
