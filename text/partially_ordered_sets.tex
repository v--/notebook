\subsection{Partially ordered sets}\label{subsec:partially_ordered_sets}

\paragraph{Partial orders}

\hyperref[def:preordered_set]{Preordered sets} are simple to define and arise naturally, for example \fullref{def:lindenbaum_tarski_algebra} or \fullref{thm:semiring_divisibility_order}. Unfortunately, they require certain uniqueness considerations, as discussed in \fullref{ex:preorder_nonuniqueness}. If we want to overcome them, we arrive at the concept of a partially ordered set.

\begin{definition}\label{def:partially_ordered_set}\mcite[1]{Birkhoff1967Lattices}
  We define a \term[ru=частично упорядоченное множество (\cite[72]{Гуров2013Решётки})]{partially ordered set} to be a \hyperref[def:preordered_set]{preordered set} that also satisfies the antisymmetry condition \eqref{eq:def:binary_relation/antisymmetric}.
\end{definition}
\begin{comments}
  \item The metamathematical properties are inherited from \fullref{def:preordered_set} with this additional axiom added. We denote the corresponding category via \( \cat{Pos} \).

  \item \incite[72]{Гуров2013Решётки}, \incite[10]{Проданов1982ФункАнализ}, \incite[4]{Engelking1989Topology}, \incite[2]{DaveyPriestley2002Lattices}, \incite[13]{Harzheim2005OrderedSets}, \incite[1]{Gratzer2011Lattices} and \incite[1]{Birkhoff1967Lattices} define partial orders like us. \incite[62]{Enderton1977Sets} defines partial orders as what we call \enquote{\hyperref[def:strict_partial_order]{strict partial orders}}, while \incite[13]{Kelley1975Topology} defines partial orders as simply \hyperref[def:binary_relation/transitive]{transitive} relations.

  \item The category \( \cat{Pos} \) is isomorphic to that of small skeletal preorder categories --- see \fullref{thm:order_category_isomorphism}.
\end{comments}

\begin{definition}\label{def:strict_partial_order}\mcite[13]{Harzheim2005OrderedSets}
  We call an \hyperref[def:binary_relation/irreflexive]{irreflexive} and \hyperref[def:binary_relation/transitive]{transitive} \hyperref[def:binary_relation]{binary relation} a \term{strict partial order}.
\end{definition}

\begin{proposition}\label{thm:strict_partial_order}
  For an arbitrary set \( P \) and binary relation \( \leq \) on \( P \), the pair \( (P, \leq) \) is a \hyperref[def:partially_ordered_set]{partially ordered set} if and only the relation \( < \), defined via \eqref{eq:def:preordered_set/compatibility_strict}, is a \hyperref[def:strict_partial_order]{strict partial order}.
\end{proposition}
\begin{proof}
  \SufficiencySubProof Let \( \leq \) be a partial order. We will show that \( < \) is a strict partial order.

  \SubProofOf*[def:binary_relation/transitive]{transitivity} The relation \( < \) is \hyperref[def:binary_relation/transitive]{transitive}. To see this, let \( x < y \) and \( y < z \). In particular, \( x \leq y \) and \( y \leq z \). From transitivity, we have \( x \leq z \).

  Additionally, \( x \neq y \) and \( y \neq z \). Assume that \( x = z \). From reflexivity of \( \leq \) we have \( z \leq x \) and, since \( y \leq z \), from transitivity we obtain \( y \leq x \). But since \( x \leq y \), from the antisymmetry of \( \leq \), we have \( x = y \), which contradicts the assumption that \( x < y \).

  Therefore, \( x < z \).

  \SubProofOf*[def:binary_relation/irreflexive]{irreflexivity}. Follows directly from reflexivity of \( \leq \) and the compatibility condition.

  Since the right side is false, the left side \( x < x \) is also false.

  \NecessitySubProof Let \( < \) be a strict partial order. We will show that \( \leq \) is a partial order.

  \SubProofOf*[def:binary_relation/reflexive]{reflexivity} Fix \( x \in P \) and assume that \( x \not\leq x \). Then \( x \neq x \) which contradicts the reflexivity of equality. Hence, \( x \leq x \).

  \SubProofOf*[def:binary_relation/antisymmetric]{antisymmetry} Let \( x \leq y \) and \( y \leq x \), that is, either \( x = y \) or both \( x < y \) and \( y < x \) hold. Assume the latter. By the transitivity of \( \leq \), we have \( x < x \), which contradicts the irreflexivity of \( < \). Hence, \( x = y \).

  \SubProofOf*[def:binary_relation/transitive]{transitivity} Let \( x \leq y \) and \( y \leq z \). Then we have four cases depending on which of \( x \), \( y \) and \( z \) are equal. Since both relations \( < \) and \( = \) are transitive, it follows that in all four cases \( x \leq z \).
\end{proof}

\begin{proposition}\label{thm:def:partially_ordered_set}
  \hyperref[def:partially_ordered_set]{Partially ordered sets} have the following basic properties:
  \begin{thmenum}
    \thmitem{def:partially_ordered_set/comparables_reflect_inequalities} For any \hyperref[def:order_function/preserving]{order-preserving map} \( f: P \to Q \) between partially ordered sets, if \( x \) and \( y \) are comparable elements, \( f(x) < f(y) \) implies \( x < y \).
  \end{thmenum}
\end{proposition}
\begin{proof}
  \SubProofOf{def:partially_ordered_set/comparables_reflect_inequalities} Let \( f(x) <_Q f(y) \) and suppose that \( x \geq y \). Since \( f \) is order-preserving, we have \( f(x) \geq_Q f(y) \), which is a contradiction.
\end{proof}

\begin{definition}\label{def:hasse_diagram}\mcite[11]{DaveyPriestley2002Lattices}
  It can be easier to define small finite partially ordered sets by drawing graphs than by enumerating all relation pairs. Let \( (P, \leq) \) be a finite partially ordered set. The relation \( \leq \) may also be regarded as the set of edges of a \hyperref[def:directed_graph]{directed graph}. We will instead consider the \hyperref[def:transitive_reduction]{transitive reduction} \( \red^T(\leq) \).

  We call the graph \( (P, \red^T(\leq)) \) the \term[ru=диаграмма Хассе (\cite[78]{Гуров2013Решётки})]{Hasse diagram} of \( (P, \leq) \).
\end{definition}
\begin{comments}
  \item When drawing Hasse diagrams, we draw no arrowheads; instead, arcs point upwards. See \fullref{ex:def:hasse_diagram}.
\end{comments}

\begin{example}\label{ex:def:hasse_diagram}
  Consider the partial order over \( \set{ a, b, c, d, e } \) defined via
  \begin{equation}\label{eq:ex:def:hasse_diagram/partially ordered set}
    \hi{a \leq c},\quad \hi{a \leq d},\quad a \leq e,\quad \hi{b \leq d},\quad b \leq e,\quad \hi{d \leq e}.
  \end{equation}

  The corresponding Hasse diagram includes only the highlighted edges. The rest of the edges can be recovered via transitivity. In this case, the Hasse diagram has edges
  \begin{equation}\label{eq:ex:def:hasse_diagram/hasse_graph}
    \set{ a \to c, a \to d, b \to d, d \to e }
  \end{equation}

  \begin{figure}[!ht]
    \centering
    \includegraphics[page=1]{output/ex__def__hasse_diagram}
    \caption{A drawing of the Hasse diagram \eqref{eq:ex:def:hasse_diagram/hasse_graph}}
    \label{fig:ex:def:hasse_diagram}
  \end{figure}
\end{example}

\begin{example}\label{ex:preorder_nonuniqueness}
  Consider the preordered set \( P \) in \cref{fig:ex:preorder_nonuniqueness} in which \( b \leq c \) and \( c \leq b \), but \( b \neq c \). We cannot properly draw a \hyperref[def:hasse_diagram]{Hasse diagram} because we have the restriction that \( c \) is drawn (strictly) higher than \( b \) if \( c > b \) and that \( c \) is drawn lower than \( b \) if \( c < b \). We face a similar problem formally, for example in the definition of \hyperref[def:lindenbaum_tarski_algebra]{Lindenbaum-Tarski algebras}, where the preorder \( \vdash \) allows \( \varphi \vdash \psi \) and \( \psi \vdash \varphi \), but still \( \varphi \neq \psi \). Thus, we have nonuniqueness --- every tautology is a largest element with respect to \( \vdash \), while it is natural to prefer a unique largest element.

  \begin{figure}[!ht]
    \hfill
    \includegraphics[page=1]{output/ex__preorder_nonuniqueness}
    \hfill
    \includegraphics[page=2]{output/ex__preorder_nonuniqueness}
    \hfill\hfill
    \caption{A preordered set and its induced partially ordered set.}
    \label{fig:ex:preorder_nonuniqueness}
  \end{figure}

  This leads to \fullref{def:antisymmetric_quotient}, where we factor \( P \) by an equivalence relation and obtain a partially ordered set. In the language of graph theory, if we have \hyperref[def:graph_cycle]{cycles}, we can contract each directed cycle into a single vertex, at which point the graph becomes acyclic. This corresponds to \hyperref[def:directed_graph_condensation]{graph condensation}.
\end{example}

\begin{definition}\label{def:antisymmetric_quotient}\mimprovised
  Let \( (P, \leq) \) be a preordered set. Define the relation \( x \sim y \) to hold if \( x \leq y \) and \( y \leq x \).

  On the \hyperref[def:equivalence_relation/quotient]{quotient set} \( Q \coloneqq P / {\sim} \), define the relation \( [x] \preceq [y] \) to hold if \( x \leq y \).

  The pair \( (Q, \preceq) \) is then a \hyperref[def:partially_ordered_set]{partially ordered set}, which we will call the \term{antisymmetric quotient} of \( (P, \leq) \).
\end{definition}
\begin{comments}
  \item The relation \( \sim \) is the intersection of the relation \( \leq \) with its \hyperref[def:binary_relation/inverse]{inverse}.
  \item We may prefer, as in the case of \fullref{thm:lindenbaum_tarski_theories}, to refer to the projection operator \( \pi: P \to Q \) rather than to quotient sets.
\end{comments}
\begin{defproof}
  First, we must show that the relation \( \preceq \) is well-defined. Indeed, let \( x \sim x' \) and \( y \sim y' \). If \( x \leq y \), we have \( x' \leq x \leq y \leq y' \), hence \( x' \leq y' \) because of transitivity.

  It is then clear that \( \preceq \) is a partial order because it inherits reflexivity and transitivity from \( \leq \) and antisymmetry is imposed by taking quotient sets --- equality in \( Q \) holds precisely when \( {\sim} \) holds in \( P \).

  Thus, \( (Q, \preceq) \) is indeed a partially ordered set.
\end{defproof}

\paragraph{Intervals}

\begin{definition}\label{def:order_interval}\mcite[31]{Harzheim2005OrderedSets}
  Fix a \hyperref[def:partially_ordered_set]{partially ordered set} \( (P, \leq) \). For any two elements \( a \leq b \) we define the following related (ordered) subsets:

  \begin{thmenum}
    \thmitem{def:order_interval/unbounded} First, we define the \term{open initial segment} and \term{open final segment} and as
    \begin{equation*}
      \begin{aligned}
        P_{>a} \coloneqq \set{ x \in P \given b > a },
        \\
        P_{<b} \coloneqq \set{ x \in P \given x < b }.
      \end{aligned}
    \end{equation*}

    Similarly, we define the corresponding \term{closed segments} as
    \begin{equation*}
      \begin{aligned}
        P_{\geq a} \coloneqq \set{ x \in P \given x \geq a },
        \\
        P_{\leq b} \coloneqq \set{ x \in P \given x \leq b }.
      \end{aligned}
    \end{equation*}

    \thmitem{def:order_interval/closed} Next, we define the \term[bg=затворен интервал (\cite[39]{Тагамлицки1971Диф}), ru=отрезок (\cite[82]{АлександровМаркушевичХинчинЭнциклопедия1951Том1}); отрезок/сегмент (\cite[def. 6]{Александров1977Топология})]{closed interval} with endpoints \( a \) and \( b \) as
    \begin{equation*}
      [a, b] \coloneqq \set{ x \in P \given a \leq x \leq b } = P_{\geq a} \cap P_{\leq b}.
    \end{equation*}

    \thmitem{def:order_interval/open} The \term[bg=отворен интервал (\cite[39]{Тагамлицки1971Диф}), ru=интервал (\cite[82]{АлександровМаркушевичХинчинЭнциклопедия1951Том1}); интервал/промежуток (\cite[def. 6]{Александров1977Топология})]{open interval} with endpoints \( a \) and \( b \) is
    \begin{equation*}
      (a, b) \coloneqq \set{ x \in P \given a < x < b } = P_{> a} \cap P_{< b}.
    \end{equation*}

    \thmitem{def:order_interval/half_open} The \term[bg=полузатворен интервал (\cite[39]{Тагамлицки1971Диф}), ru=полуинтервал (\cite[82]{АлександровМаркушевичХинчинЭнциклопедия1951Том1})]{half-open intervals} are
    \begin{equation*}
      \begin{aligned}
        (a, b] \coloneqq \set{ x \in P \given a < x \leq b } = P_{> a} \cap P_{\leq b},
        \\
        [a, b) \coloneqq \set{ x \in P \given a \leq x < b } = P_{\geq a} \cap P_{< b}.
      \end{aligned}
    \end{equation*}
  \end{thmenum}
\end{definition}
\begin{comments}
  \item We implicitly assume that \( a \leq b \), but this is not strictly necessary --- \( [a, b] \) is an empty set otherwise.

  \item In the context of \hyperref[def:extended_real_numbers]{extended real numbers}, we use \( -\infty \) and \( \infty \) as interval endpoints. If the interval is open at \( -\infty \) and \( \infty \), as in \( (-\infty, b] = \BbbR_{\leq b} \) or \( [a, \infty) = \BbbR_{\geq a} \), we obtain sets containing only real numbers.

  \item Another notation for open intervals is \enquote{\( ]a, b[ \)}. It is used by \incite[51]{Зорич2019АнализТом1}. We prefer \enquote{\( (a, b) \)}, which is in turn used by most of our sources --- \incite[9]{Lorentz1966Approximations}, \incite[31]{Rudin1976Principles}, \incite[10]{Engelking1989Topology}, \incite[269]{Jacobson1985AlgebraPart1}, \incite[157]{Gratzer2011Lattices}, \incite[31]{Harzheim2005OrderedSets}, \incite[15]{BrannanEsplenGray2011Geometry}, Jerrold Grossman in \cite[67]{Rosen1999DiscreteHandbook}, \incite[82]{АлександровМаркушевичХинчинЭнциклопедия1951Том1}, \incite[25]{МагарилИльяевТихомиров2002ВыпуклыйАнализ}, \incite[18]{ИльинСадовничийСендов1985АнализТом1}, \incite[\S 22.2]{Тыртышников2007ЛинАлгебра}, \incite[12]{Александров1977Топология}, \incite[112]{Винберг2014Алгебра}, \incite[10]{БелоусовТкачёв2004ДискретнаяМатематика} \incite[23]{Гуров2013Решётки}, \incite[7]{Боянов2008ЧислениМетоди}, \incite[225]{ГеновМиховскиМоллов1991Алгебра} and \incite[39]{Тагамлицки1971Диф}.

  \incite[36]{Knuth1997ArtVol1} uses \enquote{\( (a..b) \)}.
\end{comments}

\paragraph{Chains}

\begin{definition}\label{def:partial_order_chain}
  Fix a partially ordered set \( (P, \leq) \) and a subset \( C \) of \( P \).

  \begin{thmenum}
    \thmitem{def:partial_order_chain/chain}\mcite[4]{Gratzer2011Lattices} We say that \( C \) is a \term[bg=верига (\cite[10]{Проданов1982ФункАнализ}), ru=цепь (\cite[def. 3.4]{Гуров2013Решётки})]{chain} if every two elements of \( C \) are comparable.

    \thmitem{def:partial_order_chain/length}\mimprovised We define the \term[ru=длина (\cite[83]{Гуров2013Решётки}), en=length (\cite[4]{Gratzer2011Lattices})]{length} \( \len(C) \) of a \hi{nonempty} chain \( C \) as the unique \hyperref[def:cardinal]{cardinal} satisfying \( {\len(C) + 1 = \card(A)} \), that is, \( \len(C) = \card(C) - 1 \) for finite chains and \( \len(C) = \card(C) \) otherwise.

    \thmitem{def:partial_order_chain/height}\mimprovised We define the \term[ru=высота (\cite[83]{Гуров2013Решётки}), en=height (\cite[24]{Harzheim2005OrderedSets})]{height} of \( P \) as the supremum among the lengths of all nonempty chains in \( P \).

    If \( P \) has no nonempty chains, the supremum is the bottom element of the natural numbers, zero.

    \thmitem{def:partial_order_chain/antichain}\mcite[4]{Gratzer2011Lattices} We say that \( C \) is an \term[ru=антицепь (\cite[78]{Гуров2013Решётки})]{antichain} if no two elements of \( C \) are comparable.

    \thmitem{def:partial_order_chain/width}\mcite[57]{Harzheim2005OrderedSets} We define the \term[ru=ширина (\cite[83]{Гуров2013Решётки})]{width} of \( P \) as the supremum among the cardinalities of all antichains.
  \end{thmenum}
\end{definition}
\begin{comments}
  \item The definitions for length, width and height are defined for finite sets of elements in \cite[4]{Gratzer2011Lattices}. We extend the definition to infinite partially ordered sets.
  \item Unlike for the height, for the width we do not subtract \( 1 \) from the cardinality of antichains.
  \item \Fullref{thm:union_of_set_of_cardinals} implies that the height and width are well-defined cardinals.
  \item The concepts of width and height become apparent in the context of Hasse diagrams --- see \fullref{ex:def:partial_order_chain}.
  \item \incite[5]{Birkhoff1967Lattices} and \incite[4]{Gratzer2011Lattices} use \enquote{length} for what we call \enquote{height}, while \incite[24]{Harzheim2005OrderedSets} and \incite[83]{Гуров2013Решётки} prefer our terminology.
  \item \incite[19]{Harzheim2005OrderedSets} defines the length of a chain as the cardinality itself, while \incite[5]{Birkhoff1967Lattices}, \incite[4]{Gratzer2011Lattices}, \incite[def. 2.37]{DaveyPriestley2002Lattices} and \incite[83]{Гуров2013Решётки} subtract one. We prefer the latter convention.
\end{comments}

\begin{example}\label{ex:def:partial_order_chain}
  We list some examples of \hyperref[def:partial_order_chain]{chains and antichains}:
  \begin{thmenum}
    \thmitem{ex:def:partial_order_chain/natural_numbers} The set \( \BbbN \) of \hyperref[def:natural_numbers]{natural numbers}, as well as any subset of \( \BbbN \), is a chain.

    The length of the initial segment \( \BbbN_{\leq n} = \set{ 0, 1, \ldots, n } \) is \( n \) because it has \( n + 1 \) elements. The length of \( \BbbN \) itself is \( \aleph_0 \).

    Hence, the height of \( \BbbN \) is \( \aleph_0 \) and the height of any individual number \( n \) is \( n \) itself.

    The width of \( \BbbN \) (or, more generally, any \hyperref[def:totally_ordered_set]{totally ordered set}) is \( 1 \).

    \thmitem{ex:def:partial_order_chain/binary_power_set} Consider the \hyperref[def:basic_set_operations/power_set]{power set} of the two-element set \( \set{ a, b } \) depicted in \cref{fig:ex:def:partial_order_chain/binary_power_set}.

    \begin{figure}[!ht]
      \hfill
      \includegraphics[page=1]{output/ex__def__partial_order_chain__binary_power_set}
      \hfill\hfill
      \caption{A Hasse diagram of the power set of a two-element set.}
      \label{fig:ex:def:partial_order_chain/binary_power_set}
    \end{figure}

    Only the sets \( \set{ a } \) and \( \set{ b } \) are incomparable, thus the width of the power set is \( 2 \).

    On the other hand, the maximal chain \( \varnothing \subseteq \set{ a } \subseteq \set{ a, b } \) has three elements, hence is of length two. Thus, the height of the power set is \( 2 \).

    \thmitem{ex:def:partial_order_chain/ternary_power_set} Consider the \hyperref[def:basic_set_operations/power_set]{power set} of \( \set{ a, b, c } \) depicted in \cref{fig:ex:def:partial_order_chain/ternary_power_set}.

    \begin{figure}[!ht]
      \hfill
      \includegraphics[page=1]{output/ex__def__partial_order_chain__ternary_power_set}
      \hfill\hfill
      \caption{A Hasse diagram of the power set of a three-element set.}
      \label{fig:ex:def:partial_order_chain/ternary_power_set}
    \end{figure}

    Like in \fullref{ex:def:partial_order_chain/binary_power_set}, we can find several maximal antichains:
    \begin{itemize}
      \item The singleton sets \( \set{ a } \), \( \set{ b } \) and \( \set{ c } \).
      \item The binary sets \( \set{ a, b } \), \( \set{ a, c } \) and \( \set{ b, c } \).
    \end{itemize}

    Both have cardinality three, hence the width of the power set is \( 3 \).

    Maximal chains have four elements, so the height of the power set is \( 3 \).

    \thmitem{ex:def:partial_order_chain/quad_power_set} The pattern from \fullref{ex:def:partial_order_chain/binary_power_set} and \fullref{ex:def:partial_order_chain/ternary_power_set} breaks when we consider four elements.

    Indeed, for the power set of \( \set{ a, b, c, d } \), the following is an antichain of six elements:
    \begin{equation}\label{eq:ex:def:partial_order_chain/quad_power_set/antichain}
      \set{ a, b }, \set{ a, c }, \set{ a, d }, \set{ b, c }, \set{ b, d }, \set{ c, d }.
    \end{equation}

    This generalizes to \fullref{thm:sperners_theorem}.
  \end{thmenum}
\end{example}

\begin{proposition}\label{thm:def:partial_order_chain}
  \hyperref[thm:def:partial_order_chain]{Chains and antichains} have the following basic properties:
  \begin{thmenum}
    \thmitem{thm:def:partial_order_chain/image_of_chain} Let \( f: P \to Q \) be an order-preserving map between partially ordered sets. If \( A \) is a chain in \( P \), then its image \( f[A] \) is a chain in \( Q \).

    \thmitem{thm:def:partial_order_chain/preimage_of_antichain} Again, let \( f: P \to Q \) be an order-preserving map between partially ordered sets. If the image \( f[A] \) of a subset \( A \) of \( P \) is an antichain, then \( A \) itself is an antichain.

    \thmitem{thm:def:partial_order_chain/unbounded} A chain is \hyperref[def:extremal_points/bounds]{unbounded from above} (resp. below) if, for every element, it contains a strictly larger (resp. smaller) element.

  \end{thmenum}
\end{proposition}
\begin{proof}
  \SubProofOf{thm:def:partial_order_chain/image_of_chain} Fix two elements \( x \) and \( y \) in \( f[A] \). Let \( a \) and \( b \) be elements of \( P \) such that \( f(a) = x \) and \( f(b) = y \). Then \( a \leq b \) implies \( f(a) \leq f(b) \) and \( a > b \) implies \( f(a) \geq f(b) \). Hence, \( x \) and \( y \) are comparable, and since they were arbitrary, we conclude that \( f[A] \) is a chain.

  \SubProofOf{thm:def:partial_order_chain/preimage_of_antichain} Follows from \fullref{def:partially_ordered_set/comparables_reflect_inequalities}.

  \SubProofOf{thm:def:partial_order_chain/unbounded} Fix a chain \( C \) in a partially ordered set \( (P, \leq) \).

  Expanding the definition \fullref{def:extremal_points/bounds} of upper bound, we conclude that
  \begin{displayquote}
    \( C \) is unbounded if and only if there does not exist an element \( x \in C \) such that, for every \( y \in C \), we have \( y \leq x \).
  \end{displayquote}

  The consequent of the above statement has a negation in front of its outermost quantifier. \Fullref{thm:first_order_quantifiers_are_dual} justifies moving the negation inwards to obtain
  \begin{displayquote}
    \( C \) is unbounded if and only if, for every \( x \in C \), there exists some \( y \in C \) such that \( y \leq x \) does not hold.
  \end{displayquote}

  Taking into account that every pair of elements in \( C \) is comparable, \( y \leq x \) does not hold if and only if \( y > x \) holds. Then
  \begin{displayquote}
    \( C \) is unbounded if and only if, for every \( x \in C \), there exist \( y \in C \) such that \( y > x \).
  \end{displayquote}

  The case of lower bounds is analogous.
\end{proof}

\begin{theorem}[Sperner's theorem]\label{thm:sperners_theorem}\mcite[287]{Harzheim2005OrderedSets}
  Let \( S \) be a set with \( n \) elements and let \( \mscrU \) be an \hyperref[def:partial_order_chain/antichain]{antichain} in the power set of \( S \).

  Then \( \mscrU \) has at most \( \binom n {\quot(n, 2)} \) elements.
\end{theorem}

\begin{definition}\label{def:partial_order_element_height}\mimprovised
  Fix a \hyperref[def:partially_ordered_set]{partially ordered set} \( (P, \leq) \) with \hyperref[def:extremal_points/top_and_bottom]{bottom element} \( \bot \).

  We define the \term[ru=высота (\cite[83]{Гуров2013Решётки}), en=height (\cite[5]{Birkhoff1967Lattices})]{height} of an element \( x \) of \( P \) as the \hyperref[def:partial_order_chain/height]{height} of the \hyperref[def:order_interval/unbounded]{initial segment} \( P_{\leq x} \).

  If \( x \) has height \( 1 \), we call it an \term[ru=атом (\cite[89]{Гуров2013Решётки}), en=atom (\cite[5]{Birkhoff1967Lattices})]{atom}.
\end{definition}
\begin{comments}
  \item The condition for \( P \) to be bounded from below, which ensures well-definedness, is due to \incite[5]{Birkhoff1967Lattices}, while using the height of the initial segment is due to \incite[4]{Gratzer2011Lattices}.

  \item The definition of atom is also due to \incite[5]{Birkhoff1967Lattices}.
\end{comments}

\begin{proposition}\label{thm:atoms_are_incomparable}
  Distinct \hyperref[def:partial_order_element_height]{atoms} are incomparable.
\end{proposition}
\begin{proof}
  If \( x \leq y \) are atoms, then in order for \( \bot \leq x \leq y \) to be a chain of length \( 1 \), \( x \) and \( y \) must be equal.
\end{proof}

\paragraph{Lexicographic order}

\begin{definition}\label{def:lexicographic_order}\mimprovised
  Let \( (P, \leq_P) \) and \( (Q, \leq_Q) \) be partially ordered sets.

  We define the \term[en=lexicographic order (\cite[18]{DaveyPriestley2002Lattices}), ru=лексикографический (порядок) (\cite[99]{Гуров2013Решётки})]{lexicographic order} on \( P \times Q \) as
  \begin{equation}\label{eq:def:lexicographic_order}
    (a, b) \prec (c, d) \thickspace \T{if and only if} \thickspace \parens[\Big]{ a <_P c \T{or} \parens[\Big]{ a = c \T{and} b <_Q d } }
  \end{equation}
  and the \term{reverse lexicographic order} as
  \begin{equation}\label{eq:def:lexicographic_order/reverse}
    (a, b) \prec (c, d) \thickspace \T{if and only if} \thickspace \parens[\Big]{ b <_Q d \T{or} \parens[\Big]{ b = d \T{and} a <_P c } }.
  \end{equation}
\end{definition}
\begin{comments}
  \item We can use natural number recursion to extend this to arbitrary \( n \)-tuples.
\end{comments}

\begin{proposition}\label{thm:lexicographic_order_is_partial_order}
  The \hyperref[eq:def:lexicographic_order]{lexicographic} and \hyperref[eq:def:lexicographic_order/reverse]{reverse lexicographic} orders are \hyperref[def:partially_ordered_set]{strict partial order} relations.
\end{proposition}
\begin{comments}
  \item An analogous result holds for total orders (\fullref{thm:def:totally_ordered_set/lexicographic}) and well-ordered sets (\fullref{thm:def:well_ordered_set/lexicographic}).
\end{comments}
\begin{proof}
  \SubProofOf[def:binary_relation/irreflexive]{irreflexivity} Trivial.

  \SubProofOf[def:binary_relation/transitive]{transitivity} Let \( \prec \) be a lexicographic order on \( P \times Q \).

  If \( (a, b) \prec (c, d) \) and \( (c, d) \prec (e, f) \), then
  \begin{itemize}
    \item If \( a < c \), then \( a < c \leq e \) and thus \( (a, b) \prec (e, f) \).

    \item If \( a = c \) and \( b < d \), then \( a \leq e \) and \( b < d \leq f \) and thus \( (a, b) \prec (e, f) \).
  \end{itemize}

   The proof for the reverse lexicographic order is analogous.
\end{proof}

\begin{example}\label{ex:def:lexicographic_order}
  We list several examples of \hyperref[def:lexicographic_order]{lexicographic ordering}:
  \begin{thmenum}
    \thmitem{ex:def:lexicographic_order/words} In a real-world dictionary like a thesaurus, the word \enquote{bright} comes after \enquote{blight}.

    Denote the English alphabet by \( A \). Then the lexicographic ordering on the Cartesian power \( A^6 \) suggests that since the first letters of \enquote{bright} and \enquote{blight} are equal, we should compare the second letters, which would lead us to the usual thesaurus order.

    \thmitem{ex:def:lexicographic_order/heterogenous_words} Real-world dictionaries list words with different numbers of letters. It is not clear how to compare \enquote{car} to \enquote{carboy}.

    We can add a sentinel symbol \( \anon* \) to the English alphabet \( A \) to denote empty spaces. Then \enquote{car} in \( A^6 \) becomes \enquote{car\( \anon* \anon* \anon* \)}. Other situations like \fullref{ex:def:lexicographic_order/natural_numbers} may require prepending \( \anon* \) rather than appending it.

    Now, if we specify how \( \anon* \) should compare to the letters, we can use lexicographic ordering to compare the two words. For example, making \( \anon* \) the bottom of the alphabet will make \enquote{car} precede \enquote{carboy}, while making it the top would do the converse.

    Note that this is only a problem if one word is a prefix of another. We can easily conclude that \enquote{apple} is smaller than \enquote{apricot} because the third letter \( p \) is smaller than \( r \).

    \thmitem{ex:def:lexicographic_order/natural_numbers} Consider the language of natural numbers in decimal notation from \fullref{def:positional_number_system/decimal}. Ordering single-digit strings is trivial, but ordering multi-digit numeric strings requires using lexicographic ordering.

    Furthermore, even though it is clear that \( \syn1 \syn0 < \syn1 \syn1 \), handling numeric strings of differing lengths requires prepending a sentinel symbol \( \anon* \) that is less than any numeral. Then we have \( \anon* \syn9 < \syn1 \syn0 \), which corresponds to the natural number ordering discussed in \fullref{def:natural_numbers_ordering}.

    \thmitem{ex:def:lexicographic_order/rectangle} A slightly more relevant example for mathematics is the lexicographic ordering
    \begin{equation*}
      AB < AD < BC < CD
    \end{equation*}
    of the names of the edges of a \hyperref[def:parallelogram/rectangle]{rectangle}. The reverse lexicographic ordering is
    \begin{equation*}
      AB < BC < AD < CD.
    \end{equation*}

    \thmitem{ex:def:lexicographic_order/triangle} For the sides of a \hyperref[def:triangle]{triangle}, we have \( AB < AC < BC \) for both orderings.

    \thmitem{ex:def:lexicographic_order/antichain} The antichain \eqref{eq:ex:def:partial_order_chain/quad_power_set/antichain} is ordered with respect to the lexicographic ordering on the Cartesian square of \( \set{ a, b, c, d } \).

    \thmitem{ex:def:lexicographic_order/graph} The edges of the graph in \eqref{eq:fig:def:directed_multigraph} are numbered in lexicographic order, which also happens to be the reverse lexicographic order.

    \thmitem{ex:def:lexicographic_order/ordinals} \Fullref{thm:ordinal_addition_disjoin_union} and \fullref{thm:ordinal_multiplication_cartesian_product} contain more interesting applications of lexicographic orders.
  \end{thmenum}
\end{example}

\paragraph{Chain conditions}

\begin{remark}\label{rem:ascending_chains}
  As discussed in \fullref{rem:order_homomorphism_terminology}, some authors refer to order-preserving maps as \enquote{ascending}. When regarding a \hyperref[def:sequence]{sequence} as a map between ordered sets, an \enquote{ascending sequence} is then simply one where every next element is larger than or equal to the previous one.

  Such sequences are used in the context of the \enquote{ascending chain condition}, which states that no strictly ascending sequences exist. This condition is named so by many authors, usually in the context of lattices of ring ideals. The term can be found in English --- see \incite[244]{Aluffi2009Algebra}, \incite[417]{Knapp2016BasicAlgebra}, \incite[102]{Jacobson1985AlgebraPart1}, \incite[112]{Lang2002Algebra}, \incite[304]{Rotman2010Algebra}, \incite[69]{Golan1999Semirings} --- as well as in Russian (\enquote{возрастающая цепь}) --- see \incite[61]{Шафаревич1999Алгебра} and \incite[def. 9.4.1]{Винберг2014Алгебра} --- and Bulgarian (\enquote{растяща верига}) --- see \incite[248]{ГеновМиховскиМоллов1991Алгебра}.

  In the more general context of abstract ordered sets, the term is used by \incite[51]{DaveyPriestley2002Lattices}, \incite[181]{Birkhoff1967Lattices} and \incite[24]{Gratzer2011Lattices}.

  None of the aforementioned authors, however, gives a definition for an \enquote{ascending chain}, leaving open the possibility that an ascending chain is not merely an ascending sequence.

  Suppose that we define an \enquote{ascending chain} as a chain, in the sense of \fullref{def:partial_order_chain/chain}, that is not \hyperref[def:extremal_points/bounds]{bounded from above}. As we shall see in our proof equivalence in \fullref{def:chain_condition}, no ascending chains (in this sense) exist if and only if no strictly ascending sequences exist.

  Thus, if we decide to provide a definition for the term \enquote{ascending chain}, we would have to choose between at least two options --- ascending sequences, which is truer to their usage, and unbounded from above chains, which is truer to the concept of chains defined in \fullref{def:partial_order_chain}.

  We avoid defining the term \enquote{ascending chain} entirely and give several equivalent definitions for the \enquote{ascending chain conditions} in \fullref{def:chain_condition}.
\end{remark}

\begin{definition}\label{def:stabilizing_sequence}\mcite[244]{Aluffi2009Algebra}
  Consider the \hyperref[def:order_function/ascending]{ascending sequence}
  \begin{equation*}
    x_1 \leq x_2 \leq x_3 \leq \cdots
  \end{equation*}

  If there exists an index \( n \) such that \( x_k = x_n \) whenever \( k > n \), we say that the chain \term[bg=стабилизира (\cite[41]{КоцевСидеров2016КомАлгебра}), ru=стабилизируется (\cite[52]{Яблонский2003ДискретнаяМатематика})]{stabilizes} at \( n \).
\end{definition}
\begin{comments}
  \item Of course, descending sequences can also stabilize.
\end{comments}

\begin{definition}\label{def:chain_condition}
  We say that a \hyperref[def:partially_ordered_set]{partially ordered set} satisfies the \term{ascending chain condition} (resp. \term{descending chain condition}) if any of the following equivalent conditions hold:
  \begin{thmenum}
    \thmitem{def:chain_condition/maximal}\mcite[180]{Birkhoff1967Lattices} Every nonempty subset has a \hyperref[def:extremal_points/maximal_and_minimal_element]{maximal} (resp. minimal) element.

    \thmitem{def:chain_condition/stabilization}\mcite[51]{DaveyPriestley2002Lattices} Every nonstrict ascending (resp. descending) sequence \hyperref[def:stabilizing_sequence]{stabilizes}.

    \thmitem{def:chain_condition/infinite} There exists no strictly ascending (resp. descending) sequence.

    \thmitem{def:chain_condition/unbounded} There exists no \hyperref[def:extremal_points/bounds]{unbounded from above} (resp. from below) \hyperref[def:partial_order_chain]{chain}.
  \end{thmenum}
\end{definition}
\begin{defproof}
  We will restrict ourselves to ascending sequences, maximal elements and upper bounds.

  \ImplicationSubProof{def:chain_condition/maximal}{def:chain_condition/stabilization} Suppose that every subset has a maximal element.

  Suppose that there is a nonstrict ascending sequence
  \begin{equation*}
    x_1 \leq x_2 \leq x_3 \leq \cdots.
  \end{equation*}

  Then the set \( \set{ x_k \given k \geq 1 } \) has a maximal element, say \( x_{k_0} \). Then, since \( x_{k_0} \) is maximal, using \hyperref[con:induction/peano_arithmetic]{natural number induction} we can prove that \( x_{k_0} = x_{k_0 + i} \) for \( i \geq 0 \).

  Thus, the sequence stabilizes.

  \ImplicationSubProof{def:chain_condition/stabilization}{def:chain_condition/infinite} Suppose that every ascending sequence in \( P \) stabilizes.

  Every strictly ascending sequence is also nonstrictly ascending, hence it stabilizes, which contradicts our assumption that it is strictly ascending. The obtained contradiction shows that there are not strictly ascending sequences in \( P \).

  \ImplicationSubProof{def:chain_condition/infinite}{def:chain_condition/unbounded} Suppose that there exists no infinite strictly ascending sequence.

  Suppose also that there exists an unbounded from above chain \( C \). Let \( x_1 \) be an element of \( C \). Via \hyperref[rem:natural_number_recursion]{natural number recursion}, we can define a sequence by letting \( x_{n+1} \) be an element of \( C \) strictly larger than \( x_n \) --- \fullref{thm:def:partial_order_chain/unbounded} guarantees the existence of such an element. Note that the axiom of choice is needed here.

  We have thus constructed a strictly ascending sequence
  \begin{equation*}
    x_1 < x_2 < x_3 < \cdots
  \end{equation*}

  The existence of such a sequence contradicts our assumption, hence also the existence of \( C \).

  \ImplicationSubProof{def:chain_condition/unbounded}{def:chain_condition/maximal} Suppose that there exists no unbounded from above chain.

  Let \( A \) be some nonempty subset. Suppose that \( A \) has no maximal element. Then we can construct a strictly ascending sequence
  \begin{equation*}
    x_1 < x_2 < x_3 < \cdots
  \end{equation*}

  The set
  \begin{equation*}
    \set{ x_1, x_2, x_3, \cdots }
  \end{equation*}
  is an unbounded from above chain.

  The obtained contradiction shows that \( A \) has a maximal element.
\end{defproof}

\paragraph{Cofinal sets}

\begin{definition}\label{def:cofinal_set}\mcite[71]{Harzheim2005OrderedSets}
  We say that a subset \( A \) of a \hyperref[def:preordered_set]{preordered set} \( (P, \leq) \) is \term{cofinal} if, for every \( x \in P \), there exists some \( y \in A \) such that \( x \leq y \).
\end{definition}

\begin{example}\label{ex:def:cofinal_set}
  We list several examples of \hyperref[def:cofinal_set]{cofinal} and non-cofinal sets.

  \begin{thmenum}
    \thmitem{ex:def:cofinal_set/finite} In a finite set like \( \set{ 0, 1, 2 } \), the set \( \set{ 2 } \) containing the maximum is cofinal. This is generalized by \fullref{thm:def:cofinal_set/top} and \fullref{thm:def:cofinal_set/maximal}.

    \thmitem{ex:def:cofinal_set/integers} Consider the set \( \BbbZ \) of integers. Clearly the set \( 2\BbbZ \) of even integers is cofinal. This is generalized by \fullref{thm:def:totally_ordered_set/cofinal_iff_unbounded}.

    \thmitem{ex:def:cofinal_set/net_convergence} Cofinal sets are encountered in topology when discussing convergence of nets --- see \fullref{subsec:net_convergence}.

    \thmitem{ex:def:cofinal_set/regular_cardinals} \hyperref[def:regular_cardinal]{Regular cardinals} are equal to their own \hyperref[def:cofinality]{cofinality}.
  \end{thmenum}
\end{example}

\begin{proposition}\label{thm:def:cofinal_set}
  \hyperref[def:cofinal_set]{Cofinal sets} have the following basic properties:
  \begin{thmenum}
    \thmitem{thm:def:cofinal_set/top} If a preordered set is bounded from above, a subset is cofinal if and only if it contains a \hyperref[def:extremal_points/top_and_bottom]{top element}.

    \thmitem{thm:def:cofinal_set/maximal} A cofinal subset of a \hi{partially ordered set} contains all \hyperref[def:extremal_points/maximal_and_minimal_element]{maximal elements}.

    \thmitem{thm:def:cofinal_set/acc} In a partially ordered set satisfying the \hyperref[def:chain_condition]{ascending chain condition}, if a set contains all \hyperref[def:extremal_points/maximal_and_minimal_element]{maximal elements}, it is cofinal.

    \thmitem{thm:def:cofinal_set/transitive}\mcite[72]{Harzheim2005OrderedSets} For any preordered set \( P \), if \( A \) is cofinal in \( P \) and \( B \) --- in \( A \), then \( B \) is cofinal in \( P \).

    This is a form of \hyperref[def:binary_relation/transitive]{transitivity}.
  \end{thmenum}
\end{proposition}
\begin{proof}
  \SubProofOf{thm:def:cofinal_set/top} Let \( (P, \leq) \) be a preordered set with top element \( \top \).

  \SufficiencySubProof* Let \( A \) be a cofinal subset of \( P \). Then it must contain some element \( x \) such that \( \top \leq x \). By transitivity, for any element \( y \) of \( P \), we have \( y \leq \top \) and \( \top \leq x \), hence \( y \leq x \).

  Then \( x \) is a top element of \( P \) that belongs to \( A \).

  \NecessitySubProof* Suppose that \( \top \in A \). Fix an arbitrary element \( x \) of \( P \). Since \( \top \) is an upper bound of \( P \), it follows that \( x \leq \top \). But \( \top \) belongs to \( A \) and \( x \) is arbitrary, thus \( A \) is cofinal.

  \SubProofOf{thm:def:cofinal_set/maximal} Suppose that \( A \) is cofinal in \( (P, \leq) \), where \( \leq \) is a partial order. Let \( m \) be a maximal element of \( P \).

  There must exist some \( x \) in \( A \) such that \( m \leq x \). But \( m \) is maximal, thus \( x = m \). Therefore, \( m \) belongs to \( A \).

  Generalizing on \( m \), we conclude that \( A \) contains all maximal elements of \( P \).

  \SubProofOf{thm:def:cofinal_set/acc} Suppose that \( A \) contains all maximal elements.

  Let \( x \) be an arbitrary element of \( P \). Let \( B \) be the set of all elements strictly larger than \( x \).
  \begin{itemize}
    \item If \( B \) is empty, then \( x \) is itself maximal, and by assumption belongs to \( A \).
    \item Otherwise, since \( P \) satisfies the ascending chain condition, \( B \) has a maximal element \( m \). If \( y \) is any element of \( P \) such that \( m \leq y \), by transitivity \( y \) belongs to \( B \) and, since \( m \) is maximal, \( m = y \). Hence, \( m \) is also maximal for \( P \), and thus belongs to \( A \).

    Therefore, \( A \) has an element \( m \) larger than \( x \).
  \end{itemize}

  Generalizing on \( x \), we conclude that \( A \) is cofinal.

  \SubProofOf{thm:def:cofinal_set/transitive} Fix any \( p \in P \). Since \( A \) is cofinal in \( P \), there exists some \( a \in A \) such that \( p \leq a \), and similarly some \( b \in B \) such that \( a \leq b \). Transitivity of \( \leq \) implies that \( p \leq b \).

  Hence, \( B \) is transitive in \( P \).
\end{proof}

\paragraph{Moore closure operators}

\begin{definition}\label{def:extensive_function}\mcite[40]{Harzheim2005OrderedSets}
  We say that an \hyperref[def:function/endofunction]{endofunction} \( f \) on a \hyperref[def:partially_ordered_set]{partially ordered set} \( (P, \leq) \) is \term[en=extensive \cite[111]{Birkhoff1967Lattices}]{extensive} if \( x \leq f(x) \) for every \( x \) in \( P \).
\end{definition}
\begin{comments}
  \item \incite[40]{Harzheim2005OrderedSets} prefers the term \enquote{extensional}, while \incite[def. 2.12]{Гуров2013Решётки} uses \enquote{reflexive} (\enquote{рефлексивный (оператор)}).
\end{comments}

\begin{definition}\label{def:idempotent_function}\mcite[40]{Harzheim2005OrderedSets}
  We say that an \hyperref[def:function/endofunction]{endofunction} \( f \) on an arbitrary \hyperref[def:set]{set} \( A \) is \term{idempotent} if \( f(f(x)) = f(x) \) for every \( x \) in \( A \).
\end{definition}
\begin{comments}
  \item Idempotent functions are \hyperref[def:monoid_idempotent]{idempotent elements} in \hyperref[def:endomorphism_monoid]{endomorphism monoids}.
\end{comments}

\begin{definition}\label{def:moore_closure_operator}\mcite[40]{Harzheim2005OrderedSets}
  Let \( (P, \leq) \) be a \hyperref[def:partially_ordered_set]{partially ordered set}. We say that the function \( \cl: P \to P \) is a \term[ru=оператор замыкания (\cite[def. 4.12]{Гуров2013Решётки})]{Moore closure operator} in \( P \) if it is \hyperref[def:extensive_function]{extensive}, \hyperref[def:idempotent_function]{idempotent} and \hyperref[def:order_function/preserving]{order-preserving}.

  We say that \( x \) is \term[ru=замкнутый (\cite[def. 4.12]{Гуров2013Решётки})]{closed} with respect to \( \cl \) if \( x = \cl(x) \).
\end{definition}
\begin{comments}
  \item \Fullref{thm:closure_operator_minimality} gives an equivalent condition for an element to be closed while \fullref{thm:closure_operator_from_set_semilattice} simplifies defining closure operators on power sets.

  \item \incite[40]{Harzheim2005OrderedSets} and \incite[146]{DaveyPriestley2002Lattices} call such an operator simply a \enquote{closure operator}, \incite[def. 26]{Gratzer2011Lattices} uses \enquote{closure system}, \incite[111]{Birkhoff1967Lattices} uses \enquote{closure operation} but restricts the definition to lattices of subsets.

  We add the prefix \enquote{Moore} because of the related Moore families discussed in \fullref{def:moore_family}.
\end{comments}

\begin{proposition}\label{thm:closure_operator_minimality}
  For a given \hyperref[def:moore_closure_operator]{Moore closure operator}, we have \( \cl(x) = c \) if and only if \( c \) is the \hyperref[def:extremal_points/greatest_and_least]{least} of all closed elements greater than or equal to \( x \).
\end{proposition}
\begin{proof}
  Fix some element \( x \) and consider the set
  \begin{equation*}
    D \coloneqq \set{ d \geq x \given \cl(d) = d }.
  \end{equation*}

  Clearly \( \cl(x) \) belongs to \( D \) because \( \cl \) is extensive.

  \SufficiencySubProof Consider the closure \( \cl(x) \). Let \( d \geq x \) be an arbitrary closed element. We have \( x \leq d \), and since \( \cl \) preserves order, \( \cl(x) \leq \cl(d) = d \).

  Therefore, \( \cl(x) \) is a lower bound of \( D \) that belongs to \( D \), that is, the least element of \( D \).

  \NecessitySubProof Suppose that \( c \) is the least element of \( D \). Then \( x \leq c \leq \cl(x) \). But
  \begin{equation*}
    \cl(x) \leq \underbrace{\cl(c)}_{c} \leq \underbrace{\cl(\cl(x))}_{\cl(x)},
  \end{equation*}
  hence
  \begin{equation*}
    \cl(x) = c.
  \end{equation*}
\end{proof}

\begin{definition}\label{def:moore_family}\mcite[111]{Birkhoff1967Lattices}
  We say that a family of subsets of an arbitrary \hyperref[def:set]{set} is a \term{Moore family} if it is closed under arbitrary (including empty) intersections.
\end{definition}

\begin{proposition}\label{thm:closure_operator_from_set_semilattice}
  Let \( X \) be some set and \( \mscrL \) be a \hyperref[def:moore_family]{Moore family} in \( X \). Then the following function is a \hyperref[def:moore_closure_operator]{Moore closure operator} on \( X \):
  \begin{equation*}
    \begin{aligned}
      &\cl: \pow(X) \to \mscrL, \\
      &\cl(A) \coloneqq \bigcap \set{ L \in \mscrL \given A \subseteq L }. \\
    \end{aligned}
  \end{equation*}
\end{proposition}
\begin{comments}
  \item This proposition implies that \( \cl(A) \) is the intersection of all closed sets containing \( A \).
  \item \Fullref{thm:closure_operator_minimality} implies that \( \cl(A) \) is the smallest closed set containing \( A \).
  \item This allows us to introduce a closure operator on arbitrary families that are closed under intersection --- including \hyperref[def:topological_space]{topological closed sets}, \hyperref[def:affine_hull]{affine hulls}, \hyperref[def:convex_hull]{convex hulls} and \hyperref[def:first_order_generated_substructure]{generated first-order substructures} (groups, rings, \( R \)-modules and lattices, among others --- see \fullref{ex:def:category_of_small_first_order_models}).
\end{comments}
\begin{proof}
  \SubProofOf[def:extensive_function]{extensiveness} The intersection \( \cl(A) \) of \( \set{ L \in \mscrL \given A \subseteq L } \) obviously contains \( A \).

  \SubProofOf[def:idempotent_function]{idempotence} Note that \( \cl(A) \) itself belongs to \( \mscrL \), thus
  \begin{equation*}
    \cl(\cl(A)) = \bigcap \set{ L \in \mscrL \given \cl(A) \subseteq L } = \cl(A).
  \end{equation*}

  \SubProofOf[def:order_function/preserving]{monotonicity} If \( A \subseteq B \), then every set from \( \mscrL \) containing \( B \) also contains \( A \), hence \( \cl(A) \subseteq \cl(B) \).
\end{proof}

\paragraph{Zorn's lemma}

\begin{lemma}[Zorn's lemma]\label{thm:zorns_lemma}
  If every \hyperref[def:partial_order_chain]{chain} in a nonempty \hyperref[def:partially_ordered_set]{partially ordered set} has an \hyperref[def:extremal_points/bounds]{upper bound}, then the entire set has a \hyperref[def:extremal_points/maximal_and_minimal_element]{maximal element}.
\end{lemma}
\begin{comments}
  \item Within \hyperref[def:zfc]{\logic{ZF}}, this theorem is equivalent to the \hyperref[def:zfc/choice]{axiom of choice} --- see \fullref{thm:axiom_of_choice_equivalences/zorns_lemma}.

  \item Zorn's lemma is sometimes stated and used only in a \hyperref[thm:boolean_algebra_of_subsets]{lattice of sets} --- for example by \incite[117]{Gratzer2011Lattices}, \incite[317]{Rotman2010Algebra}, \incite[thm. 6.4.34]{Hinman2005Logic}, \incite[151]{Enderton1977Sets} and \incite[16]{КанторовичАкилов1984ФункАнализ}. A more general statement similar to this one is used by \incite[50]{Harzheim2005OrderedSets}, \incite[33]{Kelley1975Topology}, \incite[8]{Engelking1989Topology}, \incite[lemma V.3.1]{Aluffi2009Algebra}, \incite[880]{Lang2002Algebra}, \incite[117]{Гуров2013Решётки}, \incite[31]{Мальцев1970ОбщаяАлгебра} and \incite[13]{Проданов1982ФункАнализ}. \incite[2]{Jacobson1989AlgebraPart2} states both.
\end{comments}
\begin{proof}
  \ImplicationSubProof[def:zfc/choice]{the axiom of choice}[thm:zorns_lemma]{Zorn's lemma} Let \( (P, \leq) \) be a partially ordered set in which every chain has an upper bound. Aiming at a contradiction, suppose that \( P \) has no maximal elements.

  Let \( \mscrC \) be the set of all chains in \( P \). Define the set-valued map \( F: \mscrC \multto P \) that assigns to each set in \( P \) the family of all its strict upper bounds.

  Since every chain \( C \) has an upper bound, say \( u \), and since \( P \) has no maximal element, there exists some element strictly larger than \( u \). Hence, \( F \) is a total set-valued map. By \fullref{thm:existence_of_single_valued_selections}, there exists a single-valued selection \( f: \pow(P) \to P \) of \( F \).

  By \fullref{thm:hartogs_lemma}, there exists a smallest \hyperref[def:ordinal]{ordinal} \( \alpha \) such that no function from \( \alpha \) to \( P \) is injective. Using \fullref{thm:bounded_transfinite_recursion}, we can define
  \begin{equation*}
    \begin{aligned}
      &g: \alpha \to P, \\
      &g(\beta) \coloneqq f(\set{ g(\gamma) \given \gamma < \beta }) = f(g[\beta]). \\
    \end{aligned}
  \end{equation*}

  If \( \beta_1 < \beta_2 \), then
  \begin{equation*}
    \set{ g(\gamma) \given \gamma < \beta_1 }
    \subsetneq
    \set{ g(\gamma) \given \gamma < \beta_2 },
  \end{equation*}
  and thus
  \begin{equation*}
    g(\beta_1) < g(\beta_2).
  \end{equation*}

  Then the images under \( g \) of different elements of \( \alpha \) are different, i.e. \( g \) is injective. But this contradicts our choice of \( \alpha \).

  The obtained contradiction shows that \( P \) has a maximal element.

  \ImplicationSubProof[thm:zorns_lemma]{Zorn's lemma}[def:zfc/choice]{axiom of choice} Let \( \mscrA \) be a family of nonempty sets. Let \( \mscrF \) be the set of all \hyperref[def:set_valued_map/partial]{partial single-valued functions} from \( \mscrA \) to \( \bigcup \mscrA \) with the subset ordering. That is, \( f \leq g \) if \( \dom(f) \subseteq \dom(g) \) for \( f, g \in \mscrF \).

  Clearly every chain has a greatest element - a total single-valued function. Then \( \mscrF \) itself has a maximal element by Zorn's lemma. This maximal element is necessarily a total function because otherwise it would not be maximal.

  Then this is the desired choice function for the family \( \mscrA \).
\end{proof}
