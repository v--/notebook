\subsection{Partially ordered sets}\label{subsec:partially_ordered_sets}

\paragraph{Partially ordered sets}\hfill

\hyperref[def:preordered_set]{Preordered sets} are simple to define and arise naturally, for example \fullref{def:lindenbaum_tarski_algebra} or \fullref{thm:semiring_divisibility_order}. Unfortunately, they require certain uniqueness considerations, as discussed in \fullref{ex:preorder_nonuniqueness}. If we want to overcome them, we arrive at the concept of a partially ordered set.

\begin{definition}\label{def:partially_ordered_set}\mcite[1]{Birkhoff1967}
  We define a \term[ru=частично упорядоченное множество (\cite[72]{Гуров2013})]{partially ordered set} to be a \hyperref[def:preordered_set]{preordered set} that also satisfies the antisymmetry condition \eqref{eq:def:binary_relation/antisymmetric}.

  The metamathematical properties are inherited from \fullref{def:preordered_set} with this additional axiom added. We denote the corresponding category by \( \cat{Pos} \).
\end{definition}
\begin{comments}
  \item Peter Hinman in \incite[def. 2.1.1(ii)]{Hinman2005} uses our terminology, however some authors give slightly different definitions. \incite[62]{Enderton1977Sets} defines partial orders as what we call \enquote{strict partial orders}. \incite[4]{Engelking1989} and \incite[def. I.6]{Birkhoff1948} define partial orders as nonstrict (Engelking defines total orders as strict). \incite[13]{Kelley1975} defines partial orders as simply \hyperref[def:binary_relation/transitive]{transitive} relations.

  \item The category \( \cat{Pos} \) is isomorphic to that of small skeletal preorder categories --- see \fullref{thm:order_category_isomorphism}.
\end{comments}

\begin{definition}\label{def:strict_partial_order}\mcite[13]{Harzheim2005}
  We call an \hyperref[def:binary_relation/irreflexive]{irreflexive} and \hyperref[def:binary_relation/transitive]{transitive} \hyperref[def:binary_relation]{binary relation} a \term{strict partial order}.
\end{definition}

\begin{proposition}\label{thm:strict_partial_order}
  For an arbitrary set \( P \) and binary relation \( \leq \) on \( P \), the pair \( (P, \leq) \) is a \hyperref[def:partially_ordered_set]{partially ordered set} if and only the relation \( < \), defined via \eqref{eq:def:preordered_set/compatibility_strict}, is a \hyperref[def:strict_partial_order]{strict partial order}.
\end{proposition}
\begin{proof}
  \SufficiencySubProof Let \( \leq \) be a partial order. We will show that \( < \) is a strict partial order.

  \SubProofOf*[def:binary_relation/transitive]{transitivity} The relation \( < \) is \hyperref[def:binary_relation/transitive]{transitive}. To see this, let \( x < y \) and \( y < z \). In particular, \( x \leq y \) and \( y \leq z \). From transitivity, we have \( x \leq z \).

  Additionally, \( x \neq y \) and \( y \neq z \). Assume that \( x = z \). From reflexivity of \( \leq \) we have \( z \leq x \) and, since \( y \leq z \), from transitivity we obtain \( y \leq x \). But since \( x \leq y \), from the antisymmetry of \( \leq \), we have \( x = y \), which contradicts the assumption that \( x < y \).

  Therefore, \( x < z \).

  \SubProofOf*[def:binary_relation/irreflexive]{irreflexivity}. Follows directly from reflexivity of \( \leq \) and the compatibility condition.

  Since the right side is false, the left side \( x < x \) is also false.

  \NecessitySubProof Let \( < \) be a strict partial order. We will show that \( \leq \) is a partial order.

  \SubProofOf*[def:binary_relation/reflexive]{reflexivity} Fix \( x \in P \) and assume that \( x \not\leq x \). Then \( x \neq x \) which contradicts the reflexivity of equality. Hence, \( x \leq x \).

  \SubProofOf*[def:binary_relation/antisymmetric]{antisymmetry} Let \( x \leq y \) and \( y \leq x \), that is, either \( x = y \) or both \( x < y \) and \( y < x \) hold. Assume the latter. By the transitivity of \( \leq \), we have \( x < x \), which contradicts the irreflexivity of \( < \). Hence, \( x = y \).

  \SubProofOf*[def:binary_relation/transitive]{transitivity} Let \( x \leq y \) and \( y \leq z \). Then we have four cases depending on which of \( x \), \( y \) and \( z \) are equal. Since both relations \( < \) and \( = \) are transitive, it follows that in all four cases \( x \leq z \).
\end{proof}

\begin{proposition}\label{thm:def:partially_ordered_set}
  \hyperref[def:partially_ordered_set]{Partially ordered sets} have the following basic properties:
  \begin{thmenum}
    \thmitem{def:partially_ordered_set/comparables_reflect_inequalities} For any \hyperref[def:order_function/preserving]{order-preserving function} \( f: P \to Q \) between partially ordered sets, if \( x \) and \( y \) are comparable elements, \( f(x) < f(y) \) implies \( x < y \).
  \end{thmenum}
\end{proposition}
\begin{proof}
  \SubProofOf{def:partially_ordered_set/comparables_reflect_inequalities} Let \( f(x) <_Q f(y) \) and suppose that \( x \geq y \). Since \( f \) is a monotone map, we have \( f(x) \geq_Q f(y) \), which is a contradiction.
\end{proof}

\begin{definition}\label{def:hasse_diagram}\mcite[11]{DaveyPriestley2002}
  It can be easier to define small finite partially ordered sets by drawing graphs than by enumerating all relation pairs. Let \( (P, \leq) \) be a finite partially ordered set. The relation \( \leq \) may also be regarded as the set of edges of a \hyperref[def:directed_graph]{directed graph}. We will instead consider the \hyperref[def:transitive_reduction]{transitive reduction} \( \red^T(\leq) \).

  We call the graph \( (P, \red^T(\leq)) \) the \term[ru=диаграмма Хассе (\cite[78]{Гуров2013})]{Hasse diagram} of \( (P, \leq) \).
\end{definition}
\begin{comments}
  \item When drawing Hasse diagrams, we draw no arrowheads; instead, arcs point upwards. See \fullref{ex:def:hasse_diagram}.
\end{comments}

\begin{example}\label{ex:def:hasse_diagram}
  Consider the partial order over \( \set{ a, b, c, d, e } \) defined via
  \begin{equation}\label{eq:ex:def:hasse_diagram/partially ordered set}
    \hi{a \leq c},\quad \hi{a \leq d},\quad a \leq e,\quad \hi{b \leq d},\quad b \leq e,\quad \hi{d \leq e}.
  \end{equation}

  The corresponding Hasse diagram includes only the highlighted edges. The rest of the edges can be recovered via transitivity. In this case, the Hasse diagram has edges
  \begin{equation}\label{eq:ex:def:hasse_diagram/hasse_graph}
    \set{ a \to c, a \to d, b \to d, d \to e }
  \end{equation}

  \begin{figure}[!ht]
    \centering
    \includegraphics[page=1]{output/ex__def__hasse_diagram}
    \caption{A drawing of the Hasse diagram \eqref{eq:ex:def:hasse_diagram/hasse_graph}}
    \label{fig:ex:def:hasse_diagram}
  \end{figure}
\end{example}

\begin{example}\label{ex:preorder_nonuniqueness}
  Consider the preordered set \( P \) in \cref{fig:ex:preorder_nonuniqueness} in which \( b \leq c \) and \( c \leq b \), but \( b \neq c \). We cannot properly draw a \hyperref[def:hasse_diagram]{Hasse diagram} because we have the restriction that \( c \) is drawn (strictly) higher than \( b \) if \( c > b \) and that \( c \) is drawn lower than \( b \) if \( c < b \). We face a similar problem formally, for example in the definition of a \hyperref[def:lindenbaum_tarski_algebra]{Lindenbaum-Tarski algebra} of a \hyperref[def:first_order_theory]{logical theory}, where the preorder \( \vdash \) allows \( \varphi \vdash \psi \) and \( \psi \vdash \varphi \), but still \( \varphi \neq \psi \). Thus, we have nonuniqueness --- every tautology is a largest element with respect to \( \vdash \), while we want to have a single largest element for the sake of building a tidier theory.

  If we are only interested in members of \( P \) up to the equivalence relation from \fullref{thm:preorder_to_partial_order}, it is easy to factor \( P \) by the equivalence relation and obtain a partially ordered set. In the language of graph theory, if we have \hyperref[def:graph_cycle]{cycles} that we may wish to avoid, we can contract each directed cycle into a single vertex, at which point the graph becomes acyclic. This corresponds to \hyperref[def:directed_graph_condensation]{graph condensation}.

  The formulation and proof of correctness of this process can be found in \fullref{thm:preorder_to_partial_order} and an example can be found in \cref{fig:ex:preorder_nonuniqueness}.

  \begin{figure}[!ht]
    \hfill
    \includegraphics[page=1]{output/ex__preorder_nonuniqueness}
    \hfill
    \includegraphics[page=2]{output/ex__preorder_nonuniqueness}
    \hfill\hfill
    \caption{A preordered set and its induced partially ordered set.}
    \label{fig:ex:preorder_nonuniqueness}
  \end{figure}
\end{example}

\begin{proposition}\label{thm:preorder_to_partial_order}
  Let \( (P, \leq) \) be a preordered set. Define the relation \( x \sim y \) to hold if \( x \leq y \) and \( y \leq x \).

  Define the relation \( [x] \preceq [y] \) on the \hyperref[def:equivalence_relation/quotient]{quotient set} \( P / \sim \) to hold if \( x \leq y \).

  The pair \( (P / {\sim}, \sim) \) is then a \hyperref[def:partially_ordered_set]{partially ordered set}.
\end{proposition}
\begin{comments}
  \item The relation \( \sim \) is the intersection of the relation \( \leq \) with its \hyperref[def:binary_relation/inverse]{inverse}.
\end{comments}
\begin{proof}
  First, we must show that the relation \( \preceq \) is well-defined. Indeed, let \( x \sim x' \) and \( y \sim y' \). If \( x \leq y \), we have \( x' \leq x \leq y \leq y' \), hence \( x' \leq y' \) because of transitivity.

  It is then clear that \( \preceq \) is a partial order because it inherits reflexivity and transitivity from \( \leq \) and antisymmetry is imposed by taking quotient sets --- equality in \( P / \sim \) holds precisely when \( {\sim} \) holds in \( P \).

  Thus, \( (P / \sim, \preceq) \) is indeed a partially ordered set.
\end{proof}

\paragraph{Intervals}

\begin{definition}\label{def:order_interval}\mcite[31]{Harzheim2005}
  Fix a \hyperref[def:partially_ordered_set]{partially ordered set} \( (P, \leq) \). For any two elements \( a \leq b \) we define the following related (ordered) subsets:

  \begin{thmenum}
    \thmitem{def:order_interval/unbounded} First, we define the \term{open initial segment} and \term{open final segment} and as
    \begin{equation*}
      \begin{aligned}
        P_{>a} \coloneqq \set{ x \in P \given b > a },
        \\
        P_{<b} \coloneqq \set{ x \in P \given x < b }.
      \end{aligned}
    \end{equation*}

    Similarly, we define the corresponding \term{closed segments} as
    \begin{equation*}
      \begin{aligned}
        P_{\geq a} \coloneqq \set{ x \in P \given x \geq a },
        \\
        P_{\leq b} \coloneqq \set{ x \in P \given x \leq b }.
      \end{aligned}
    \end{equation*}

    \thmitem{def:order_interval/closed} Next, we define the \term[bg=затворен интервал (\cite[39]{Тагамлицки1971Диф}), ru=отрезок (\cite[51]{Зорич2019Том1}); отрезок/сегмент (\cite[def. 6]{Александров1977Введение})]{closed interval} with endpoints \( a \) and \( b \) as
    \begin{equation*}
      [a, b] \coloneqq \set{ x \in P \given a \leq x \leq b } = P_{\geq a} \cap P_{\leq b}.
    \end{equation*}

    \thmitem{def:order_interval/open} The \term[bg=отворен интервал (\cite[39]{Тагамлицки1971Диф}), ru=интервал (\cite[51]{Зорич2019Том1}); интервал/промежуток (\cite[def. 6]{Александров1977Введение})]{open interval} with endpoints \( a \) and \( b \) is
    \begin{equation*}
      (a, b) \coloneqq \set{ x \in P \given a < x < b } = P_{> a} \cap P_{< b}.
    \end{equation*}

    \thmitem{def:order_interval/half_open} The \term[bg=полузатворен интервал (\cite[39]{Тагамлицки1971Диф}), ru=полуинтервал (\cite[63]{Зорич2019Том1})]{half-open intervals} are
    \begin{equation*}
      \begin{aligned}
        (a, b] \coloneqq \set{ x \in P \given a < x \leq b } = P_{> a} \cap P_{\leq b},
        \\
        [a, b) \coloneqq \set{ x \in P \given a \leq x < b } = P_{\geq a} \cap P_{< b}.
      \end{aligned}
    \end{equation*}
  \end{thmenum}
\end{definition}
\begin{comments}
  \item We implicitly assume that \( a \leq b \), but this is not strictly necessary --- \( [a, b] \) is an empty set otherwise.

  \item In the context of \hyperref[def:extended_real_numbers]{extended real numbers}, we use \( -\infty \) and \( \infty \) as interval endpoints. If the interval is open at \( -\infty \) and \( \infty \), as in \( (-\infty, b] = \BbbR_{\leq b} \) or \( [a, \infty) = \BbbR_{\geq a} \), we obtain sets containing only real numbers.
\end{comments}

\paragraph{Chains}

\begin{definition}\label{def:partial_order_chain}
  Fix a partially ordered set \( (P, \leq) \) and a subset \( A \) of \( P \).

  \begin{thmenum}
    \thmitem{def:partial_order_chain/chain}\mcite[4]{Gratzer2011} We say that \( A \) is a \term[bg=верига (\cite[10]{Проданов1982}), ru=цепь (\cite[78]{Гуров2013})]{chain} if every two elements of \( A \) are comparable.

    \thmitem{def:partial_order_chain/length}\mimprovised We define the \term[ru=длина (\cite[83]{Гуров2013}), en=length (\cite[4]{Gratzer2011})]{length} \( \len(A) \) of a \hi{nonempty} chain \( A \) as the unique \hyperref[def:cardinal]{cardinal} satisfying \( {\len(A) + 1 = \card(A)} \), that is, \( \len(A) = \card(A) - 1 \) for finite chains and \( \len(A) = \card(A) \) otherwise.

    \thmitem{def:partial_order_chain/height}\mimprovised If \( P \) is nonempty, we define the \term[ru=высота (\cite[83]{Гуров2013}), en=height (\cite[24]{Harzheim2005})]{height} of \( P \) as the supremum among the lengths of all nonempty chains in \( P \).

    \thmitem{def:partial_order_chain/antichain}\mcite[4]{Gratzer2011} We say that \( A \) is an \term[ru=антицепь (\cite[78]{Гуров2013})]{antichain} if no two elements of \( A \) are comparable.

    \thmitem{def:partial_order_chain/width}\mimprovised We define the \term[ru=ширина (\cite[83]{Гуров2013}), en=width (\cite[4]{Gratzer2011})]{width} of an element \( x \) as the height of the \hyperref[def:order_interval/unbounded]{initial segment} \( P_{\leq x} \).
  \end{thmenum}
\end{definition}
\begin{comments}
  \item The definitions for length, width and height are defined for finite sets of elements in \cite[4]{Gratzer2011}. We extend the definition to infinite partially ordered sets.
  \item Unlike for the height, for the width we do not subtract \( 1 \) from the cardinality of antichains.
  \item \Fullref{thm:union_of_set_of_cardinals} implies that the height and width are well-defined cardinals.
  \item The concepts of width and height become apparent in the context of Hasse diagrams --- see \fullref{ex:def:partial_order_chain}.
  \item \incite[5]{Birkhoff1967} and \incite[4]{Gratzer2011} use \enquote{length} for what we call \enquote{height}, while \incite[24]{Harzheim2005} and \incite[83]{Гуров2013} prefer our terminology.
  \item \incite[19]{Harzheim2005} defines the length of a chain as the cardinality itself, while \incite[5]{Birkhoff1967}, \incite[4]{Gratzer2011}, \incite[def. 2.37]{DaveyPriestley2002} and \incite[83]{Гуров2013} subtract one. We prefer the latter convention.
\end{comments}

\begin{example}\label{ex:def:partial_order_chain}
  We list some examples of \hyperref[def:partial_order_chain]{chains and antichains}:
  \begin{thmenum}
    \thmitem{ex:def:partial_order_chain/natural_numbers} The set \( \BbbN \) of \hyperref[def:natural_numbers]{natural numbers}, as well as any subset of \( \BbbN \), is a chain.

    The length of the initial segment \( \BbbN_{\leq n} = \set{ 0, 1, \ldots, n } \) is \( n \) because it has \( n + 1 \) elements. The length of \( \BbbN \) itself is \( \aleph_0 \).

    Hence, the height of \( \BbbN \) is \( \aleph_0 \) and the height of any individual number \( n \) is \( n \) itself.

    The width of \( \BbbN \) (or, more generally, any \hyperref[def:totally_ordered_set]{totally ordered set}) is \( 1 \).

    \thmitem{ex:def:partial_order_chain/binary_power_set} Consider the \hyperref[def:basic_set_operations/power_set]{power set} of the two-element set \( \set{ a, b } \) depicted in \cref{fig:ex:def:partial_order_chain/binary_power_set}.

    \begin{figure}[!ht]
      \hfill
      \includegraphics[page=1]{output/ex__def__partial_order_chain__binary_power_set}
      \hfill\hfill
      \caption{A Hasse diagram of the power set of a two-element set.}
      \label{fig:ex:def:partial_order_chain/binary_power_set}
    \end{figure}

    Only the sets \( \set{ a } \) and \( \set{ b } \) are incomparable, thus the width of the power set is \( 2 \).

    On the other hand, the maximal chain \( \varnothing \subseteq \set{ a } \subseteq \set{ a, b } \) has three elements, hence is of length two. Thus, the height of the power set is \( 2 \).

    \thmitem{ex:def:partial_order_chain/ternary_power_set} Consider the \hyperref[def:basic_set_operations/power_set]{power set} of \( \set{ a, b, c } \) depicted in \cref{fig:ex:def:partial_order_chain/ternary_power_set}.

    \begin{figure}[!ht]
      \hfill
      \includegraphics[page=1]{output/ex__def__partial_order_chain__ternary_power_set}
      \hfill\hfill
      \caption{A Hasse diagram of the power set of a three-element set.}
      \label{fig:ex:def:partial_order_chain/ternary_power_set}
    \end{figure}

    Like in \fullref{ex:def:partial_order_chain/binary_power_set}, we can find several maximal antichains:
    \begin{itemize}
      \item The singleton sets \( \set{ a } \), \( \set{ b } \) and \( \set{ c } \).
      \item The binary sets \( \set{ a, b } \), \( \set{ a, c } \) and \( \set{ b, c } \).
    \end{itemize}

    Both have cardinality three, hence the width of the power set is \( 3 \).

    Maximal chains have four elements, so the height of the power set is \( 3 \).

    \thmitem{ex:def:partial_order_chain/quad_power_set} The pattern from \fullref{ex:def:partial_order_chain/binary_power_set} and \fullref{ex:def:partial_order_chain/ternary_power_set} breaks when we consider four elements.

    Indeed, for the power set of \( \set{ a, b, c, d } \), the following is an antichain of six elements:
    \begin{equation}\label{eq:ex:def:partial_order_chain/quad_power_set/antichain}
      \set{ a, b }, \set{ a, c }, \set{ a, d }, \set{ b, c }, \set{ b, d }, \set{ c, d }.
    \end{equation}

    This generalizes to \fullref{thm:sperners_theorem}.
  \end{thmenum}
\end{example}

\begin{theorem}[Sperner's theorem]\label{thm:sperners_theorem}\mcite[287]{Harzheim2005}
  Let \( S \) be a set with \( n \) elements and let \( \mscrU \) be an \hyperref[def:partial_order_chain/antichain]{antichain} in the power set of \( S \).

  Then \( \mscrU \) has at most \( \binom n {\quot(n, 2)} \) elements.
\end{theorem}

\begin{definition}\label{def:partial_order_element_height}\mimprovised
  FIx a \hyperref[def:partially_ordered_set]{partially ordered set} \( (P, \leq) \) with \hyperref[def:extremal_points/top_and_bottom]{bottom element} by \( \bot \).

  We define the \term[ru=высота (\cite[83]{Гуров2013}), en=height (\cite[5]{Birkhoff1967})]{height} of an element \( x \) of \( P \) as the \hyperref[def:partial_order_chain/height]{height} of the \hyperref[def:order_interval/unbounded]{initial segment} \( P_{\leq x} \).

  If \( x \) has height \( 1 \), we call it an \term[ru=атом (\cite[89]{Гуров2013}), en=atom (\cite[5]{Birkhoff1967})]{atom}.
\end{definition}
\begin{comments}
  \item The condition for \( P \) to be bounded from below, which ensures well-definedness, is due to \incite[5]{Birkhoff1967}, while using the height of the initial segment is due to \incite[4]{Gratzer2011}.

  \item The definition of atom is also due to \incite[5]{Birkhoff1967}.
\end{comments}

\begin{proposition}\label{thm:atoms_are_incomparable}
  Distinct \hyperref[def:partial_order_element_height]{atoms} are incomparable.
\end{proposition}
\begin{proof}
  If \( x \leq y \) are atoms, then in order for \( \bot \leq x \leq y \) to be a chain of length \( 1 \), \( x \) and \( y \) must be equal.
\end{proof}

\paragraph{Lexicographic order}

\begin{definition}\label{def:lexicographic_order}\mcite[18]{DaveyPriestley2002}
  Let \( (P, \leq_P) \) and \( (Q, \leq_Q) \) be partially ordered sets.

  We define the \term[ru=лексикографический (порядок) (\cite[99]{Гуров2013})]{lexicographic order} on \( P \times Q \) as
  \begin{equation}\label{eq:def:lexicographic_order}
    (a, b) \prec (c, d) \thickspace \T{if and only if} \thickspace \parens[\Big]{ a <_P c \T{or} \parens[\Big]{ a = c \T{and} b <_Q d } }
  \end{equation}
  and the \term{reverse lexicographic order} as
  \begin{equation}\label{eq:def:lexicographic_order/reverse}
    (a, b) \prec (c, d) \thickspace \T{if and only if} \thickspace \parens[\Big]{ b <_Q d \T{or} \parens[\Big]{ b = d \T{and} a <_P c } }.
  \end{equation}
\end{definition}
\begin{comments}
  \item We can use natural number recursion to extend this to arbitrary \( n \)-tuples.
  \item Lexicographic orders are also known as \term{dictionary orders}.
\end{comments}

\begin{proposition}\label{thm:lexicographic_order_is_partial_order}
  The \hyperref[eq:def:lexicographic_order]{lexicographic} and \hyperref[eq:def:lexicographic_order/reverse]{reverse lexicographic} orders are \hyperref[def:partially_ordered_set]{strict partial order} relations.
\end{proposition}
\begin{comments}
  \item An analogous result holds for total orders (\fullref{thm:total_lexicographic_order_is_total_order}) and well-ordered sets (\fullref{thm:well_ordered_lexicographic_order_is_well_ordered}).
\end{comments}
\begin{proof}
  \SubProofOf[def:binary_relation/irreflexive]{irreflexivity} Trivial.

  \SubProofOf[def:binary_relation/transitive]{transitivity} Let \( \prec \) be a lexicographic order on \( P \times Q \).

  If \( (a, b) \prec (c, d) \) and \( (c, d) \prec (e, f) \), then
  \begin{itemize}
    \item If \( a < c \), then \( a < c \leq e \) and thus \( (a, b) \prec (e, f) \).

    \item If \( a = c \) and \( b < d \), then \( a \leq e \) and \( b < d \leq f \) and thus \( (a, b) \prec (e, f) \).
  \end{itemize}

   The proof for the reverse lexicographic order is analogous.
\end{proof}

\begin{example}\label{ex:def:lexicographic_order}
  We list several examples of \hyperref[def:lexicographic_order]{lexicographic ordering}:
  \begin{thmenum}
    \thmitem{ex:def:lexicographic_order/words} In a real-world dictionary like a thesaurus, the word \enquote{bright} comes after \enquote{blight}.

    Denote the English alphabet by \( A \). Then the lexicographic ordering on the Cartesian power \( A^6 \) suggests that since the first letters of \enquote{bright} and \enquote{blight} are equal, we should compare the second letters, which would lead us to the usual thesaurus order.

    \thmitem{ex:def:lexicographic_order/heterogenous_words} Real-world dictionaries list words with different numbers of letters. It is not clear how to compare \enquote{car} to \enquote{carboy}.

    We can add a sentinel symbol \( \anon* \) to the English alphabet \( A \) to denote empty spaces. That is, \enquote{car} in \( A^6 \) becomes \enquote{car\( \anon* \anon* \anon* \)}.

    Now, if we specify how \( \anon* \) should compare to the letters, we can use lexicographic ordering to compare the two words. For example, making \( \anon* \) the bottom of the alphabet will make \enquote{car} precede \enquote{carboy}, while making it the top would do the converse.

    Note that this is only a problem if one word is a prefix of another. We can easily conclude that \enquote{apple} is smaller than \enquote{apricot} because the third letter \( p \) is smaller than \( r \).

    \thmitem{ex:def:lexicographic_order/rectangle} A slightly more relevant example for mathematics is the lexicographic ordering
    \begin{equation*}
      AB < AD < BC < CD
    \end{equation*}
    of the names of the edges of a \hyperref[def:parallelogram/rectangle]{rectangle}. The reverse lexicographic ordering is
    \begin{equation*}
      AB < BC < AD < CD.
    \end{equation*}

    \thmitem{ex:def:lexicographic_order/triangle} For the sides of a \hyperref[def:triangle]{triangle}, we have \( AB < AC < BC \) for both orderings.

    \thmitem{ex:def:lexicographic_order/antichain} The antichain \eqref{eq:ex:def:partial_order_chain/quad_power_set/antichain} is ordered with respect to the lexicographic ordering on the Cartesian square of \( \set{ a, b, c, d } \).

    \thmitem{ex:def:lexicographic_order/graph} The edges of the graph in \eqref{eq:fig:def:directed_multigraph} are numbered in lexicographic order, which also happens to be the reverse lexicographic order.

    \thmitem{ex:def:lexicographic_order/ordinals} \Fullref{thm:ordinal_addition_disjoin_union} and \fullref{thm:ordinal_multiplication_cartesian_product} contain more interesting applications of lexicographic orders.
  \end{thmenum}
\end{example}

\paragraph{Zorn's lemma}

\begin{lemma}[Zorn's lemma]\label{thm:zorns_lemma}
  If every \hyperref[def:partial_order_chain]{chain} in a nonempty \hyperref[def:partially_ordered_set]{partially ordered set} has an \hyperref[def:extremal_points/upper_and_lower_bounds]{upper bound}, then the entire set has a \hyperref[def:extremal_points/maximal_and_minimal_element]{maximal element}.
\end{lemma}
\begin{comments}
  \item Within \hyperref[def:zfc]{\logic{ZF}}, this theorem is equivalent to the \hyperref[def:zfc/choice]{axiom of choice} --- see \fullref{thm:axiom_of_choice_equivalences/zorns_lemma}.

  \item Zorn's lemma is sometimes stated and used only in a \hyperref[thm:boolean_algebra_of_subsets]{lattice of sets} --- for example by \incite[117]{Gratzer2011}, \incite[317]{Rotman2010}, \incite[thm. 6.4.34]{Hinman2005}, \incite[151]{Enderton1977Sets} and \incite[16]{КанторовичАкилов1984}. A more general statement similar to this one is used by \incite[50]{Harzheim2005}, \incite[33]{Kelley1975}, \incite[8]{Engelking1989}, \incite[lemma V.3.1]{Aluffi2009}, \incite[880]{Lang2002}, \incite[117]{Гуров2013}, \incite[31]{Мальцев1970} and \incite[13]{Проданов1982}. \incite[2]{Jacobson1985Vol2} states both.
\end{comments}
\begin{proof}
  \ImplicationSubProof[def:zfc/choice]{the axiom of choice}[thm:zorns_lemma]{Zorn's lemma} Let \( (P, \leq) \) be a partially ordered set in which every chain has an upper bound. Aiming at a contradiction, suppose that \( P \) has no maximal elements.

  Let \( \mscrC \) be the set of all chains in \( P \). Define the set-valued map \( F: \mscrC \multto P \) that assigns to each set in \( P \) the family of all its strict upper bounds.

  Since every chain \( C \) has an upper bound, say \( u \), and since \( P \) has no maximal element, there exists some element strictly larger than \( u \). Hence, \( F \) is a total set-valued map. By \fullref{thm:existence_of_single_valued_selections}, there exists a single-valued selection \( f: \pow(P) \to P \) of \( F \).

  By \fullref{thm:hartogs_lemma}, there exists a smallest \hyperref[def:ordinal]{ordinal} \( \alpha \) such that no function from \( \alpha \) to \( P \) is injective. Using \fullref{thm:bounded_transfinite_recursion}, we can define
  \begin{equation*}
    \begin{aligned}
      &g: \alpha \to P, \\
      &g(\beta) \coloneqq f(\set{ g(\gamma) \given \gamma < \beta }) = f(g[\beta]). \\
    \end{aligned}
  \end{equation*}

  If \( \beta_1 < \beta_2 \), then
  \begin{equation*}
    \set{ g(\gamma) \given \gamma < \beta_1 }
    \subsetneq
    \set{ g(\gamma) \given \gamma < \beta_2 },
  \end{equation*}
  and thus
  \begin{equation*}
    g(\beta_1) < g(\beta_2).
  \end{equation*}

  Then the images under \( g \) of different elements of \( \alpha \) are different, i.e. \( g \) is injective. But this contradicts our choice of \( \alpha \).

  The obtained contradiction shows that \( P \) has a maximal element.

  \ImplicationSubProof[thm:zorns_lemma]{Zorn's lemma}[def:zfc/choice]{axiom of choice} Let \( \mscrA \) be a family of nonempty sets. Let \( \mscrF \) be the set of all \hyperref[def:set_valued_map/partial]{partial single-valued functions} from \( \mscrA \) to \( \bigcup \mscrA \) with the subset ordering. That is, \( f \leq g \) if \( \dom(f) \subseteq \dom(g) \) for \( f, g \in \mscrF \).

  Clearly every chain has a maximum - a total single-valued function. Then \( \mscrF \) itself has a maximal element by Zorn's lemma. This maximal element is necessarily a total function because otherwise it would not be maximal.

  Then this is the desired choice function for the family \( \mscrA \).
\end{proof}
