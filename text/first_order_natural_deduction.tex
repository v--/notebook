\section{First-order natural deduction}\label{sec:first_order_natural_deduction}

\paragraph{Simultaneous substitution}

\begin{algorithm}[First-order formula substitution]\label{alg:fol_formula_substitution}\mimprovised
  We can extend an \hyperref[def:atomic_lambda_term_substitution]{atomic simultaneous substitution} \( (\Bbbs, \sharp) \) to arbitrary \hyperref[def:fol_formula]{first-order formulas} as follows:
  \begin{empheq}[left={\varphi[\Bbbs]} \coloneqq \empheqlbrace]{align}
    &\varphi,                                                    &&\varphi \in \op*{PConst},                                                                      \label{eq:alg:fol_formula_substitution/const}         \\
    &\tau[\Bbbs] \syneq \sigma[\Bbbs],                           &&\varphi = (\tau \syneq \sigma),                                                                \label{eq:alg:fol_formula_substitution/eq} \\
    &p\parens[\big]{ \sigma_1[\Bbbs], \ldots, \sigma_n[\Bbbs] }, &&\varphi = p(\sigma_1, \ldots, \sigma_n),                                                       \label{eq:alg:fol_formula_substitution/application} \\
    &\synneg \psi[\Bbbs],                                        &&\varphi = \synneg \psi,                                                                        \label{eq:alg:fol_formula_substitution/neg} \\
    &\psi[\Bbbs] \syncirc \theta[\Bbbs],                         &&\varphi = (\psi \syncirc \theta), {\syncirc} \in \op*{Conn},                                   \label{eq:alg:fol_formula_substitution/conn} \\
    &\quantifier Q x \psi[\Bbbs_{x \mapsto x}],                  &&\varphi = \quantifier Q x \psi, Q \in \op*{Quant} \T{and} x \not\in \op*{Free}_\Bbbs(\varphi), \label{eq:alg:fol_formula_substitution/quant/direct} \\
    &\quantifier Q x \psi[\Bbbs_{x \mapsto n}],                  &&\varphi = \quantifier Q x \psi, Q \in \op*{Quant} \T{and} x \in \op*{Free}_\Bbbs(\varphi)      \label{eq:alg:fol_formula_substitution/quant/renaming}
  \end{empheq}
  where \( n = \sharp(\op*{Free}_\Bbbs(\varphi)) \).
\end{algorithm}
\begin{comments}
  \item This algorithm can be found as \identifier{math.logic.substitution.apply_substitution_to_formula} in \cite{notebook:code}.
  \item This substitution is defined as to have the properties listed in \cref{rem:variable_binding_properties}; we elaborate on this in \fullref{thm:alg:fol_formula_substitution}.
\end{comments}

\begin{proposition}\label{thm:alg:fol_formula_substitution}
  \Fullref{alg:fol_formula_substitution} has the following basic properties:
  \begin{thmenum}
    \thmitem{thm:alg:fol_formula_substitution/free} For any formula \( \varphi \) and any atomic substitution \( \Bbbs \), we have
    \begin{equation}\label{eq:thm:alg:fol_formula_substitution/free}
      \op*{Free}( \varphi[\Bbbs] ) = \overbrace{\bigcup_{\mathclap{v \in \op*{Free}(\varphi)}} \op*{Free}(\Bbbs(v))}^{\op*{Free}_\Bbbs(\varphi)}.
    \end{equation}

    \thmitem{thm:alg:fol_formula_substitution/free_single} For any terms \( \varphi \) and \( \psi \) and any variable \( x \), we have
    \begin{equation}\label{eq:thm:alg:fol_formula_substitution/free_single}
      \op*{Free}( \varphi[x \mapsto \psi] ) = \begin{cases}
        (\op*{Free}(\varphi) \setminus \set{ x }) \cup \op*{Free}(\psi), &x \in \op*{Free}(\varphi) \\
        \op*{Free}(\varphi),                                             &\T{otherwise.}
      \end{cases}
    \end{equation}

    In both cases,
    \begin{equation}\label{eq:thm:alg:fol_formula_substitution/free_single/subset}
      \op*{Free}( \varphi[x \mapsto \psi] ) \subseteq (\op*{Free}(\varphi) \setminus \set{ x }) \cup \op*{Free}(\psi).
    \end{equation}

    \thmitem{thm:alg:fol_formula_substitution/quant_single_rule} For any quantifier formula \( \varphi = \quantifier Q x \psi \) and any substitution \( \Bbbs \), there exists a variable \( v \not\in \op*{Free}_\Bbbs(\varphi) \) such that
    \begin{equation}\label{eq:thm:alg:fol_formula_substitution/quant_single_rule}
      \varphi[\Bbbs] = \quantifier Q v \psi[\Bbbs_{x \mapsto v}].
    \end{equation}

    \thmitem{thm:alg:fol_formula_substitution/substitutions_agree} If the substitutions \( \Bbbs \) and \( \Bbbt \) agree on the free variables of \( \varphi \), then \( M[\Bbbs] = M[\Bbbt] \).
    \thmitem{thm:alg:fol_formula_substitution/noop} We have \( \varphi[\Bbbs] = \varphi \) if and only if the free variables of \( \varphi \) are fixed by the substitution \( \Bbbs \).
    \thmitem{thm:alg:fol_formula_substitution/identity} For any formula \( \varphi \), we have \( \varphi[\id] = \varphi \) for the identity substitution.
    \thmitem{thm:alg:fol_formula_substitution/closed} For any closed formula \( \varphi \) and any atomic substitution \( \Bbbs \), we have \( \varphi[\Bbbs] = \varphi \).
  \end{thmenum}
\end{proposition}
\begin{proof}
  \SubProofOf{thm:alg:fol_formula_substitution/free} Can be proven as in \cref{thm:lambda_substitution_free_variables}.
  \SubProofOf{thm:alg:fol_formula_substitution/free_single} Follows from \cref{thm:lambda_substitution_free_variables}, similarly to \cref{thm:lambda_substitution_free_variables_single}.
  \SubProofOf{thm:alg:fol_formula_substitution/quant_single_rule} Can be proven as in \cref{thm:lambda_substitution_single_rule}.
  \SubProofOf{thm:alg:fol_formula_substitution/substitutions_agree} Can be proven as in \cref{thm:lambda_substitutions_agree}.
  \SubProofOf{thm:alg:fol_formula_substitution/noop} Can be proven as in \cref{thm:lambda_substitution_noop}.
  \SubProofOf{thm:alg:fol_formula_substitution/identity} Follows from \cref{thm:alg:fol_formula_substitution/noop}.
  \SubProofOf{thm:alg:fol_formula_substitution/closed} Vacuously follows from \cref{thm:lambda_substitution_identity} since combinators simply have no free variables.
\end{proof}

\paragraph{\( \alpha \)-equivalence}

\begin{remark}\label{def:abstract_sequent_calculus_system_eigenvariables}
  % \hyperref[con:predicate_logic]{predicate logic}, the entries must also account for \hyperref[def:fol_schema/formula]{eigenvariables}
\end{remark}
