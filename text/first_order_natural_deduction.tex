\section{First-order natural deduction}\label{sec:first_order_natural_deduction}

\paragraph{Simultaneous substitution}

\begin{definition}\label{def:atomic_fol_substitution}\mimprovised
  We will define \hyperref[con:syntactic_substitution]{simultaneous substitution} for \hyperref[def:fol_term]{first-order terms} and \hyperref[def:fol_formula]{formulas} based on how we have defined it for \hyperref[def:lambda_term]{\( \muplambda \)-term} in \cref{def:atomic_lambda_term_substitution}.

  Namely, for a fixed \hyperref[def:fol_signature]{signature} \( \Sigma \), we will need a pair \( (\Bbbs, \sharp) \), where
  \begin{thmenum}[series=def:atomic_lambda_term_substitution]
    \thmitem{def:atomic_fol_substitution/atomic} \( \Bbbs: \op*{Var} \to \op*{Term}_\Sigma \) is a function specifying how variables need to be replaced with terms. We allow only finitely many variables to not be fixed by \( \Bbbs \). We call \( \Bbbs \) an \term{atomic substitution}.

    \thmitem{def:atomic_fol_substitution/sharp} \( \sharp \) is, as in \cref{def:atomic_lambda_term_substitution/sharp}, a function providing us with fresh variables, i.e. \( \sharp(V) \) is a new variable not in \( V \).

    By default, we suppose that \( \sharp(V) \) is the smallest, with respect to the \hyperref[def:variable_identifier]{identifier order}, variable not in \( V \).
  \end{thmenum}
\end{definition}

\begin{remark}\label{def:abstract_sequent_calculus_system_eigenvariables}
  % \hyperref[con:predicate_logic]{predicate logic}, the entries must also account for \hyperref[def:fol_schema/formula]{eigenvariables}
\end{remark}
