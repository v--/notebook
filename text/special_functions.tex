\section{Special functions}\label{sec:special_functions}

\begin{definition}\label{def:gamma_function}
  The \term[bg=гама (функция) (\cite[280]{Тагамлицки1978Инт}), ru=гамма функция (\cite[406]{Зорич2019АнализТом2})]{Gamma function} \( \Gamma: \BbbR_{>0} \to \BbbR \) can be defined via the following equivalent definitions:
  \begin{thmenum}
    \thmitem{def:gamma_function/direct}\mcite[\S 8.17]{Rudin1976AnalysisPrinciples}
    \begin{equation}\label{eq:def:gamma_function/direct}
      \Gamma(x) \coloneqq \int_0^\infty t^{x-1} e^{-t} \dl t
    \end{equation}

    \thmitem{def:gamma_function/characterization}\mcite[\S 8.19]{Rudin1976AnalysisPrinciples} \( \Gamma \) is the unique function satisfying the following conditions:
    \begin{thmenum}
      \thmitem{def:gamma_function/characterization/recurrent} \( \Gamma(x + 1) = x \Gamma(x) \).
      \thmitem{def:gamma_function/characterization/starting} \( \Gamma(1) = 1 \).
      \thmitem{def:gamma_function/characterization/log_convex} \( \log \Gamma \) is \hyperref[def:convex_functions]{convex}.
    \end{thmenum}
  \end{thmenum}
\end{definition}

\begin{theorem}[Stirling's gamma approximation]\label{thm:stirlings_gamma_approximation}\mcite[24]{Артин2009Гамма}
  For the \( \Gamma \) function we have
  \begin{equation}\label{eq:thm:stirlings_gamma_approximation}
    \Gamma(x) = \sqrt{2 \pi} \cdot x^{x - 1 / 2} \cdot e^{-x + \mu(x)},
  \end{equation}
  where
  \begin{equation}\label{eq:thm:stirlings_gamma_approximation/mu}
    \mu(x) \coloneqq \sum_{k=0}^\infty \parens[\Big]{ \parens[\Big]{ x + k + \frac 1 2 } \ln\parens[\Big]{ 1 + \frac 1 {x + k} } - 1 }.
  \end{equation}

  Furthermore,
  \begin{equation}\label{eq:thm:stirlings_gamma_approximation/mu_inequality}
    0 < \mu(x) < \frac 1 {12x}.
  \end{equation}
\end{theorem}
\begin{comments}
  \item This approximation can be found as \identifier{numeric.gamma.stirling} in \cite{notebook:code}.
\end{comments}
