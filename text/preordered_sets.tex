\subsection{Preordered sets}\label{subsec:preordered_sets}

\paragraph{Preorders}

\begin{definition}\label{def:preordered_set}\mcite[13]{Harzheim2005}
  We say that a \hyperref[def:binary_relation]{binary relation} \( \leq \) on \( P \) is a \term[bg=преднаредба (\cite[9]{Проданов1982}), ru=предпорядок (\cite[def 3.1]{Гуров2013})]{preorder} if it is \hyperref[def:binary_relation/reflexive]{reflexive} and \hyperref[def:binary_relation/transitive]{transitive}. The pair \( (P, \leq) \) is then called a \term{preordered set}.

  It is conventional to use the same symbol \( \leq \) as for \hyperref[def:partially_ordered_set]{partial orders}, however the lack of \hyperref[def:binary_relation/antisymmetric]{antisymmetry} may be confusing --- see \fullref{ex:preorder_nonuniqueness}.

  Preordered sets have the following metamathematical properties:
  \begin{thmenum}[series=def:preordered_set]
    \thmitem{def:preordered_set/inverse}\mcite[14]{Harzheim2005} We define \( \geq \) as the \hyperref[def:binary_relation/inverse]{inverse relation} of \( \leq \).

    \thmitem{def:preordered_set/strict}\mcite[14]{Harzheim2005} We also define the relation \( < \) as \( \leq \) minus the \hyperref[def:binary_relation/diagonal]{diagonal relation} \( \Delta \). This condition corresponds to the following axiom:
    \begin{equation}\label{eq:def:preordered_set/compatibility_nonstrict}
      (\xi \leq \eta) \syniff \parens[\Big]{(\xi < \eta) \synvee (\xi \syneq \eta)}.
    \end{equation}

    By adding \( \anon \synwedge \synneg (x = y) \) to both sides of \eqref{eq:def:preordered_set/compatibility_nonstrict}, using \fullref{thm:de_morgans_laws} and taking irreflexivity of \( < \) into account, we obtain
    \begin{equation}\label{eq:def:preordered_set/compatibility_strict}
      (\xi < \eta) \syniff \parens[\Big]{(\xi \leq \eta) \synwedge \synneg (\xi = \eta)}.
    \end{equation}

    We call \( < \) the \term{strict preorder} associated with \( \leq \), in which context we call \( \leq \) \term{nonstrict preorder}. We also define \( > \) as the converse relation of \( < \).

    \thmitem{def:preordered_set/comparable}\mcite[29]{Harzheim2005} We call the elements \( x \) and \( y \) are \term1{comparable} if either \( x \leq y \) or \( y \leq x \) and \term{incomparable} otherwise.
  \end{thmenum}

  Preordered sets have the following metamathematical properties:
  \begin{thmenum}[resume=def:preordered_set]
    \thmitem{def:preordered_set/theory} Consider a \hyperref[def:first_order_language]{first-order language} \( \mscrL \) with two \hyperref[rem:first_order_formula_conventions/infix]{infix} binary predicate symbols --- \( \leq \) and \( \geq \).

    The theory of preordered sets is a \hyperref[def:first_order_theory]{first-order theory} in \( \mscrL \) consisting of the axioms \eqref{eq:def:binary_relation/reflexive} and \eqref{eq:def:binary_relation/transitive} for \( \leq \) and the compatibility axiom
    \begin{equation}\label{eq:def:preordered_set/theory}
      (\xi \leq \eta) \syniff (\eta \geq \xi).
    \end{equation}

    We purposely avoid adding \( < \) and \( > \) to this language because that would change the behavior of \hyperref[def:first_order_homomorphism]{first-order homomorphisms}. If needed, we can regard them as \hyperref[con:predicate_formula]{predicate formulas}.

    If we want to add strict orders to the language, we should also add one of the compatibility axioms \eqref{eq:def:preordered_set/compatibility_nonstrict} or \eqref{eq:def:preordered_set/compatibility_strict} to the resulting theory (it is unnecessary to add both).

    \thmitem{def:preordered_set/homomorphism} The \hyperref[def:first_order_homomorphism]{first-order homomorphisms} between two preordered sets are the \hyperref[def:order_function/preserving]{nonstrictly order-preserving maps}, which we will discuss shortly.

    \thmitem{def:preordered_set/submodel} Since the theory contains only positive formulas over a language with no functional symbols, any subset \( A \) of the domain of a preordered set \( P \) becomes a preordered set with the induced preorder \( \leq_A \) defined as the restriction of \( \leq_P \) to only elements of \( A \).

    \thmitem{def:preordered_set/opposite}\mcite[316]{PicadoPultr2012} We define the \term{dual} or \term{opposite} preordered set of \( (P, \leq) \) as \( (P, \geq) \).

    \thmitem{def:preordered_set/category}  We denote the \hyperref[def:category_of_small_first_order_models]{category of \( \mscrU \)-small models} for the theory of preordered sets by \( \ucat{PreOrd} \).

    This category is isomorphic to that of \( \mscrU \)-small preorder categories --- see \fullref{thm:order_category_isomorphism}.
  \end{thmenum}
\end{definition}
\begin{comments}
  \item Following \incite[3]{Gratzer2011} and \incite[def 3.1]{Гуров2013}, we prefer the term \enquote{preorder} to \enquote{quasi-order}. Other authors like \incite[20]{Birkhoff1967} and \incite[13]{Harzheim2005} instead prefer \enquote{quasi-order}.
\end{comments}

\begin{theorem}[Principle of duality for preorders]\label{thm:preorder_duality}\mcite[5]{Gratzer2011}
  Consider the \hyperref[def:preordered_set/theory]{first-order theory of preordered sets}. Within it, consider the opposite formula \( \varphi^\oppos \) of the \hyperref[def:first_order_syntax/closed_formula]{closed formula} \( \varphi \), in which we swap all instances of \( \leq \) and \( \geq \).

  If every preordered set \hyperref[def:first_order_model]{satisfies} \( \varphi \), then every preordered set also satisfies \( \varphi^\oppos \).

  More generally, \( (P, \leq) \) satisfies \( \varphi \) if and only if its \hyperref[def:preordered_set/opposite]{dual} \( (P, \geq) \) satisfies \( \varphi^\oppos \).
\end{theorem}
\begin{comments}
  \item The corresponding syntactic statement is that the formula \( \varphi \) is \hyperref[def:logical_framework]{derivable} in the \hyperref[def:preordered_set/theory]{theory of preordered sets} if and only if the opposite formula \( \varphi^\oppos \) is also derivable.

  \item The opposite of the opposite formula of \( \varphi \) is obviously \( \varphi \).

  \item Another form of this duality is formalized in \fullref{thm:order_category_isomorphism}.

  \item Similar statements hold in special cases --- see \fullref{thm:lattice_duality} and \fullref{thm:boolean_algebra_duality}.

  \item The actual replacement can be formalized by performing the \hyperref[def:first_order_substitution/term_in_formula]{simultaneous substitution}
  \begin{equation*}
    \begin{aligned}
      \varphi^\oppos \coloneqq \varphi[
        &\tau_1 \leq \sigma_1 \mapsto \tau_1 \geq \sigma_1, &&\tau_1 \geq \sigma_1 \mapsto \tau_1 \leq \sigma_1, \\
        &\vdots                                       &&\vdots \\
        &\tau_n \leq \sigma_n \mapsto \tau_n \geq \sigma_n, &&\tau_n \geq \sigma_n \mapsto \tau_n \leq \sigma_n]
    \end{aligned}
  \end{equation*}
  for all pairs \( (\tau_k, \sigma_k) \) of terms in \( \varphi \).
\end{comments}
\begin{proof}
  Note that the semantics of \( \leq \) and \( \geq \) are swapped along with the swapping of the corresponding symbols in the formulas. Therefore, if \( (P, \leq) \) satisfies \( \varphi \), then \( (P, \geq) \) must satisfy \( \varphi^\oppos \).

  Now suppose that every preordered set satisfies \( \varphi \). For a given preordered set \( (P, \leq) \), its dual \( (P, \geq) \) satisfies \( \varphi \), and hence \( (P, \leq) \) satisfies \( \varphi^\oppos \). Since \( (P, \leq) \) was chosen arbitrarily, we conclude that every preordered sets satisfies \( \varphi^\oppos \).
\end{proof}

\paragraph{Inequalities}

\begin{definition}\label{def:inequality}\mimprovised
  When regarding some \hyperref[def:preordered_set]{preorder relation} \( \leq \) as a \hyperref[def:first_order_language/pred]{first-order predicate}, we call the corresponding \hyperref[def:first_order_syntax/atomic_formula]{atomic formulas} \term{inequalities}. More simply put, an inequality is a formula of the form \( \tau \leq \sigma \).

  We distinguish between \term{strict inequalities} \( \tau < \sigma \) and \term{nonstrict inequalities} \( \tau \leq \sigma \).

  As in the case of \hyperref[def:first_order_equation]{equations}, we call the set \hyperref[def:first_order_definability]{defined} by an inequality a \term{solution set}.
\end{definition}

\paragraph{Extremal points}

\begin{definition}\label{def:extremal_points}
  We introduce the following terminology for extremal elements of a preordered set. These definitions are often given only for \hyperref[def:partially_ordered_set]{partially ordered sets}, but we prefer a more general setting that allows us to cover some cases like the \hyperref[thm:semiring_divisibility_order]{divisibility preorder}. The downside of this generality is that, without \hyperref[def:binary_relation/antisymmetric]{antisymmetry}, some common notions like maxima and suprema lack uniqueness.

  The notions on the left and on the right are \hyperref[thm:preorder_duality]{dual}, but we discuss both nonetheless.

  \begin{thmenum}
    \thmitem{def:extremal_points/bounds}\mcite[21]{Harzheim2005}
    \begin{TwoColumns*}
      An \term[bg=горна граница (\cite[18]{Тагамлицки1971Диф}) / мажоранта (\cite[10]{Проданов1982}), ru=верхняя грань/мажоранта (\cite[def. 3.8]{Гуров2013})]{upper bound} for the set \( A \) is an element \( m \) of the ambient preordered set such that \( x \leq m \) for every \( x \in A \).

      We say that the upper bound is \term{strict} if it does not itself belong to \( A \).

      If \( A \) has at least one upper bound, we say that \( A \) is \term{bounded from above}.
    \BeginSecondColumn
      Dually, \( m \) is a \term[bg=долна граница (\cite[18]{Тагамлицки1971Диф}) / миноранта (\cite[10]{Проданов1982}), ru=нижняя грань/миноранта (\cite[def. 3.8]{Гуров2013})]{lower bound} of \( A \) if \( m \leq x \) for every \( x \in A \).

      If \( A \) has a lower bound, we say that \( A \) is \term{bounded from below}.

      We say that \( A \) is \term[bg=ограничено (множество) (\cite[19]{Тагамлицки1971Диф}), ru=ограниченное (множество) (\cite[39]{Зорич2019Том1})]{bounded} if it is both bounded from above and from below.
    \end{TwoColumns*}

    \thmitem{def:extremal_points/maximal_and_minimal_element}\mimprovised
    \begin{TwoColumns*}
      A \term[ru=максимальный (елемент) (\cite[def. 3.6]{Гуров2013}), en=maximal (element) (\cite[33]{Harzheim2005})]{maximal element} for the set \( A \subseteq P \) is a member \( m \) of \( A \) such that, whenever \( x \geq m \) for some \( x \in A \), we have \( x \leq m \).

      If antisymmetry holds, the last \enquote{\( x \leq m \)} can be simplified to \enquote{\( x = m \)}.
    \BeginSecondColumn
      Dually, \( m \in A \) is \term[ru=минимальный (елемент) (\cite[def. 3.6]{Гуров2013}), en=minimal (element) (\cite[33]{Harzheim2005})]{minimal} if, whenever \( x \leq m \) for some \( x \in A \), we have \( x \geq m \).

      If antisymmetry holds, the last \enquote{\( x \geq m \)} can be simplified to \enquote{\( x = m \)}.
    \end{TwoColumns*}

    \thmitem{def:extremal_points/greatest_and_least}\mcite[22]{Harzheim2005}
    \begin{TwoColumns*}
      If \( m \) is an upper bound of \( A \) that belongs to \( A \), we call it a \term[ru=наибольший (елемент) (\cite[def. 3.6]{Гуров2013})]{greatest element}.

      If antisymmetry holds, greatest elements are unique.
    \BeginSecondColumn
      Dually, if \( m \) is a lower bound of \( A \) that belongs to \( A \), we call it a \term[ru=наименьший (елемент) (\cite[def. 3.6]{Гуров2013})]{least element}.

      If antisymmetry holds, least elements are unique
    \end{TwoColumns*}

    \thmitem{def:extremal_points/maximum_and_minimum}\mimprovised
    \begin{TwoColumns*}
      In totally ordered sets, due to \fullref{thm:def:totally_ordered_set/maximal_iff_greatest} an element of \( A \) is maximal if and only if it is the greatest element of \( A \). We call such an element the \term{maximum} of \( A \) and denote it by \( \max A \).
    \BeginSecondColumn
      Dually, in a totally ordered set an element of \( A \) is minimal if and only if it is the least element of \( A \). We call such an element the \term{minimum} of \( A \) and denote it by \( \min A \).
    \end{TwoColumns*}

    \thmitem{def:extremal_points/supremum_and_infimum}\mcite[22]{Harzheim2005}
    \begin{TwoColumns*}
      If \( m \) is a minimum among all upper bounds of \( A \), we call it a \term[bg=точна горна граница (\cite[19]{Тагамлицки1971Диф}), ru=точная верхняя граница (\cite[\S 5.12]{Ляпин1960})]{least upper bound} or \term[bg=супремум (\cite[10]{Проданов1982}), ru=супремум (\cite[\S 5.12]{Ляпин1960})]{supremum} of \( A \).

      If antisymmetry holds, suprema are unique, and we use the notation \( \sup A \).
    \BeginSecondColumn
      Dually, if \( m \) is a maximum among all lower bounds of \( A \), we call it a \term[bg=точна долна граница (\cite[19]{Тагамлицки1971Диф}), ru=точная нижняя граница (\cite[\S 5.12]{Ляпин1960})]{greatest lower bound} or \term[bg=инфимум (\cite[10]{Проданов1982}), ru=супремум (\cite[\S 5.12]{Ляпин1960})]{infimum} of \( A \).

      If antisymmetry holds, infima are unique, and we use the notation \( \inf A \).
    \end{TwoColumns*}

    \thmitem{def:extremal_points/top_and_bottom}\mcite[15]{DaveyPriestley2002}
    \begin{TwoColumns*}
      If the ambient space itself has a maximum, we call it a \term{top element}.

      If antisymmetry holds, the top element is unique, and we denote it via \( \top \).
    \BeginSecondColumn
      Dually, if the ambient space has a minimum, we call it a \term{bottom element}.

      If antisymmetry holds, the bottom element is unique, and we use \( \bot \).
    \end{TwoColumns*}
  \end{thmenum}
\end{definition}

\begin{proposition}\label{thm:def:extremal_points}
  Fix a \hyperref[def:preordered_set]{preordered set} \( P \) and some nonempty subset \( A \) of \( P \). The extremal point notions from \fullref{def:extremal_points} have the following basic properties:
  \begin{thmenum}
    \thmitem{thm:def:extremal_points/empty_exact_bounds} If \( P \) has a bottom element, it is a supremum of the empty set.

    Dually, if \( P \) has a top element, it is an infimum of the empty set.

    \thmitem{thm:def:extremal_points/greatest_is_maximal} The greatest element of \( A \), if it exists, is also a maximal element of \( A \).

    Dually, the least element of \( A \) is a minimal element of \( A \).

    The converse holds in \hyperref[def:totally_ordered_set]{totally ordered sets} --- see \fullref{thm:def:totally_ordered_set/maximal_iff_greatest}.

    \thmitem{tthm:def:extremal_points/greatest_is_supremum} The greatest element of \( A \), if it exists, is the supremum of \( A \).

    Dually, the least element of \( A \) is the infimum of \( A \).

    \thmitem{thm:def:extremal_points/finite_set_has_maximal_element} If \( A \) is \hi{finite}, it has a maximal element and a minimal element.

    \thmitem{thm:def:extremal_points/unique_maximal_element} If \( m \) is a unique maximal element of \( A \) and if every nonempty subset of \( A \) has at least one maximal element, then \( m \) is a greatest element of \( A \).

    Dually, if \( m \) is a unique minimal element and every nonempty subset of \( A \) has a minimal element, then \( m \) is a minimum.
  \end{thmenum}
\end{proposition}
\begin{proof}
  \SubProofOf{thm:def:extremal_points/empty_exact_bounds} A supremum is a minimum among all upper bounds. All elements of \( P \) are vacuously upper bounds of \( \varnothing \) since there is nothing to compare them to, so a minimum of \( P \) is a supremum of \( \varnothing \).

  \SubProofOf{thm:def:extremal_points/greatest_is_maximal} Let \( m \) be a greatest element of \( A \).

  Let \( a \) be a member of \( A \) such that \( m \leq a \). But \( m \leq a \) since \( m \) is an upper bound of \( A \). Generalizing on \( a \), we conclude that \( m \) is maximal.

  \SubProofOf{tthm:def:extremal_points/greatest_is_supremum} Let \( m \) be a greatest element of \( A \).

  Let \( u \) be any upper bound of \( A \). Then, in particular, \( m \leq u \) because \( m \) belongs to \( A \). Generalizing on \( u \), we conclude that \( m \) is a least upper bound of \( A \).

  \SubProofOf{thm:def:extremal_points/finite_set_has_maximal_element} Suppose that \( A \) is finite. Label the elements of \( A \) as follows:
  \begin{equation*}
    a_1, a_2, \ldots, a_n.
  \end{equation*}

  We will use induction on \( n \) to show that \( A \) has a maximal element.
  \begin{itemize}
    \item The base case \( n = 1 \) is trivial --- \( A \) has one element, which is maximal.
    \item Suppose that every set of size \( n - 1 \) has a maximal element. Let \( m \) be a maximal element of \( A \setminus \set{ a_n } \) . Then we have the following possibilities:
    \begin{itemize}
      \item If \( m \) is not comparable to \( a_n \), then both \( m \) and \( a_n \) are maximal for \( A \).

      \item If \( m \leq a_n \), then \( a_n \) is maximal for \( A \). Indeed, if \( a_k \geq a_n \) for \( k < n \), then \( a_k \leq m \) and, by transitivity, \( a_k \leq a_n \).

      \item If \( m \geq a_n \), the \( m \) is maximal for \( A \). Indeed, if \( a_k \geq m \) for \( k < n \), then \( a_k \leq m \) because \( m \) is maximal for \( A \setminus \set{ a_n } \), and if \( a_n \geq m \), we already have \( m \geq a_n \) by assumption.
    \end{itemize}

    In all cases, we have shown that \( A \) has at least one maximal element.
  \end{itemize}

  \SubProofOf{thm:def:extremal_points/unique_maximal_element} Suppose that every nonempty subset of \( A \) has a maximal element. Suppose also that \( m \) is a unique maximal element of \( A \) itself.

  Consider the set \( B \) of elements incomparable to \( m \). Suppose that \( M \) is nonempty. Let \( m' \) be a maximal element of \( B \). Since \( m \) and \( m' \) are incomparable, \( m' \) is then a maximal element of \( A \).

  But we have assumed that \( m \) is unique, hence \( m = m' \), which contradicts our choice of \( m' \).

  The obtained contradiction shows that \( B \) is empty. Then every element of \( A \) is comparable to \( m \). Fix an arbitrary element \( a \) of \( A \).
  \begin{itemize}
    \item If \( a \geq m \), then, since \( m \) is maximal, we have \( a \leq m \).
    \item The remaining option is \( a \leq m \).
  \end{itemize}

  Therefore, \( m \) is a greatest element of \( A \).
\end{proof}

\begin{example}\label{ex:thm:def:extremal_points}
  We list examples of for the extremal point notions from \fullref{def:extremal_points}:
  \begin{thmenum}
    \thmitem{ex:thm:def:extremal_points/empty} Fix some arbitrary preordered set \( P \) and consider the empty subset \( \varnothing \).

    Clearly every member of \( P \) is both an upper and a lower bound of \( \varnothing \). If \( P \) has a bottom element, it is a supremum of \( \varnothing \) because it is a minimum among all upper bounds of \( \varnothing \), that is, among all elements of \( P \). Similarly, if \( P \) has a top element, it is an infimum of \( \varnothing \).

    Since \( \varnothing \) is empty, it cannot have a greatest element, nor a maximal element.

    \thmitem{ex:thm:def:extremal_points/one} Consider the one-element set \( \set{ a } \). The only possible reflexive relation entails \( a \leq a \), so \( a \) is the top and bottom of \( \set{ a } \).

    \thmitem{ex:thm:def:extremal_points/two_isolated} Consider the two-element set \( \set{ a, b } \) with the \hyperref[def:binary_relation/diagonal]{diagonal relation} (every element is related only to itself).

    Then \( a \) is the only upper bound of \( \set{ a } \), hence also a maximal element, greatest element and supremum. It is similarly a lower bound, minimal element, minimum and infimum.

    But for \( \set{ a, b } \), it is neither an upper nor lower bound.

    On the other hand, both \( a \) and \( b \) are maximal elements of \( \set{ a, b } \), because there exist no elements greater than them. Similarly, they are both minimal.

    \thmitem{ex:thm:def:extremal_points/two_connected} Consider again the two-element set \( \set{ a, b } \), but this time with \( a \leq b \). \Fullref{thm:two_element_lattice} implies that all such preordered sets are isomorphic.

    The \( a \) is a bottom element and \( b \) is a top element.

    \thmitem{ex:thm:def:extremal_points/two_equivalent} Consider yet again the two-element set \( \set{ a, b } \), but this time with both \( a \leq b \) and \( b \leq a \).

    Both of them are a top element and a bottom element. Because we lack antisymmetry in general, we cannot conclude that \( a = b \), so we have two distinct top elements.

    \thmitem{ex:thm:def:extremal_points/three_incomparable} Consider the three-element set \( \set{ a, b, c } \) such that \( a \leq c \) and \( b \leq c \).

    Clearly \( c \) is a top element. But there is no bottom. Both \( a \) and \( b \) are minimal elements of \( \set{ a, b, c } \), but neither of them is a minimum because they are not comparable.

    \thmitem{ex:thm:def:extremal_points/unique_maximal_element_not_greatest}\mcite{MathSE:unique_maximal_element_that_is_not_greatest} Consider the set \( \BbbZ \) of integers with the standard ordering from \fullref{def:integer_ordering}.

    Adjoin some symbol \( u \) and define a relation to \( \BbbZ \cup \set{ u } \) extending the ordering on \( \BbbZ \) with \( u \leq u \).

    Then \( u \) is incomparable with any integer. It is the unique maximal element of \( \BbbZ \cup \set{ u } \) because no integer is greater than \( u \). But, since it is incomparable with integers, \( u \) is not a greatest element.

    \thmitem{ex:thm:def:extremal_points/nonunique_maxima} Consider again the integers \( \BbbZ \), but this time adjoin two incomparable elements --- \( \infty \) and \( \tieinfty \) --- and let both of them be greater than any integer.

    Then both \( \infty \) and \( \tieinfty \) are upper bounds of \( \BbbZ \) in \( \BbbZ \cup \set{ \infty, \tieinfty } \). Furthermore, both of them are minimal in the set \( \set{ \infty, \tieinfty } \) of upper bounds, but, since they are not comparable, neither is a least upper bound.
  \end{thmenum}
\end{example}

\paragraph{Functions between ordered sets}

\begin{definition}\label{def:order_function}
  Fix \hyperref[def:preordered_set]{preordered sets} \( (P, \leq_P) \) and \( (Q, \leq_Q) \) and consider an arbitrary function \( f: P \to Q \). \Fullref{rem:order_homomorphism_terminology} demonstrates how widely the terminology for such functions varies. We try to choose the most unambiguous terminology.

  \begin{thmenum}
    \thmitem{def:order_function/preserving}\mcite[35]{Harzheim2005} We call \( f \) \term{order-preserving} or an \term{order homomorphism} if
    \begin{equation}\label{eq:def:order_function/preserving}
      x \leq_P y \T{implies} f(x) \leq_Q f(y).
    \end{equation}

    If the inequalities are strict, i.e. if
    \begin{equation}\label{eq:def:order_function/preserving/strict}
      x <_P y \T{implies} f(x) <_Q f(y),
    \end{equation}
    we call \( f \) \term{strictly order-preserving}.

    To disambiguate, we sometimes call function satisfying \eqref{eq:def:order_function/preserving} \term{nonstrictly order-preserving}.

    While we avoid the term \enquote{monotone} for non-real-valued functions, we refer to this property as \term{monotonicity}.

    \thmitem{def:order_function/reflecting}\mcite[35]{Harzheim2005} We call \( f \) \term{order-reflecting} if it satisfies \hyperref[def:conditional_formula/converse]{converse} of \eqref{eq:def:order_function/preserving}:
    \begin{equation}\label{eq:def:order_function/reflecting}
      f(x) \leq_Q f(y) \T{implies} x \leq_Q y.
    \end{equation}

    \thmitem{def:order_function/reversing}\mcite[35]{Harzheim2005} \hyperref[thm:preorder_duality]{Dually}, we call \( f \) \term{order-reversing} if
    \begin{equation}\label{eq:def:order_function/reversing}
      x \leq_P y \T{implies} f(x) \geq_Q f(y).
    \end{equation}
    and \term{strictly decreasing} or \term{strictly order-reversing} if
    \begin{equation}\label{eq:def:order_function/reversing/strict}
      x <_P y \T{implies} f(x) >_Q f(y).
    \end{equation}

    \thmitem{def:order_function/increasing}\mcite[35]{Harzheim2005} In the case of \hyperref[def:totally_ordered_set]{totally ordered sets}, we prefer the term (strictly) \term[bg=растяща (функция) (\cite[198]{Тагамлицки1971Диф}), ru=возрастающая (функция) (\cite[132]{ИльинСадовничийСендов1985Том1})]{increasing} (resp. \term[bg=намаляваща (функция) (\cite[198]{Тагамлицки1971Диф}), ru=убывающая (функция) (\cite[132]{ИльинСадовничийСендов1985Том1})]{decreasing}) to (strictly) \enquote{order-preserving} (resp. \enquote{order-reversing}).

    \thmitem{def:order_function/monotone}\mcite[95]{Rudin1976Principles} In the context of real-valued functions, we use the \term[bg=монотонна (функция) (\cite[199]{Тагамлицки1971Диф}), ru=монотонная (функция) (\cite[\S 47]{ФихтенгольцОсновыТом1})]{monotone} as a collective term for nondecreasing and nonincreasing functions. Similarly, we use \term{strictly monotone} for increasing and decreasing functions, in which context we refer to usual monotone functions as \term{nonstrictly monotone}.

    Outside of analysis, we generally avoid the term \enquote{monotone} due to ambiguity.

    \thmitem{def:order_function/ascending}\incite[35]{Harzheim2005} In the context of \hyperref[def:chain_condition]{chain conditions}, we refer to increasing and decreasing sequences as \term{ascending} and \term{descending}.
  \end{thmenum}
\end{definition}
\begin{comments}
  \item Nonstrict order-preserving maps are used extensively in the theory of \hyperref[subsec:partially_ordered_sets]{partially ordered sets}, in particular in \hyperref[subsec:lattices]{lattice theory}, while strict order-preserving maps are used in the theory of \hyperref[subsec:partially_ordered_sets]{totally ordered sets}, in particular for \hyperref[subsec:ordinals]{ordinals}.
\end{comments}

\begin{remark}\label{rem:order_homomorphism_terminology}
  There is widely varying terminology for the maps defined in \fullref{def:order_function}.

  \begin{itemize}
    \item \enquote{Monotone} is used by
    \begin{itemize}
      \item \incite[30]{Gratzer2011}, \incite[23]{DaveyPriestley2002}, \incite[317]{PicadoPultr2012}, \incite[3]{Kelley1975} and \incite[92]{MacLane1998} for what we call a nonstrictly order-preserving map.

      \item \incite[132]{ИльинСадовничийСендов1985Том1}, \incite[\S 47]{ФихтенгольцОсновыТом1} and \incite[def. 4.28]{Rudin1976Principles} for what we call either a nonincreasing or nondecreasing real-valued function.
    \end{itemize}

    \item \enquote{Weakly monotone} is used \incite[175]{MacLane1998} for what we call a nondecreasing function.

    \item \enquote{Strictly monotone} is used \incite[133]{ИльинСадовничийСендов1985Том1} and \incite[def. 4.28]{Rudin1976Principles} for what we call a strictly order-preserving real-valued function.

    \item \enquote{Isotone} is used \incite[2]{Birkhoff1967}, \incite[30]{Gratzer2011}, \incite[35]{Harzheim2005}, \incite[23]{DaveyPriestley2002}, \incite[3]{Kelley1975} and \incite[def. 3.9]{Гуров2013} for what we call a nonstrictly order-preserving function.

    \item \enquote{Reverse isotone} is used \incite[def. 3.9]{Гуров2013} for what we call a nonstrictly order-reflecting map.

    \item \enquote{Antitone} is used by
    \begin{itemize}
      \item \incite[30]{Gratzer2011} for what we call an order-reversing map.
      \item \incite[3]{Birkhoff1967} for when the order-reversing condition \eqref{eq:def:order_function/reversing} is an equivalence rather than an implication
    \end{itemize}

    \item \enquote{Anti-isotone} is used \incite[35]{Harzheim2005} and \incite[def. 3.9]{Гуров2013} for what we call an order-reversing map.

    \item \enquote{Order-preserving} is used by
    \begin{itemize}
       \item \incite[2]{Birkhoff1967}, \incite[30]{Gratzer2011}, \incite{Harzheim2005}, \incite[3]{Kelley1975} and \incite[95]{MacLane1998} for what we call a nondecreasing function.

       \item \incite[4]{Engelking1989} for what we call a strictly order-preserving function, however only for totally ordered sets.
    \end{itemize}

    \item \enquote{Order-reversing} is used by \incite[35]{Harzheim2005} and \incite[95]{MacLane1998} for what we call analogously.

    \item \enquote{Order-reflecting} is used by \incite[35]{Harzheim2005} for what we call analogously.

    \item \enquote{Nondecreasing} is used by
    \begin{itemize}
      \item \incite[35]{Harzheim2005} for what we call analogously.
      \item \incite[8]{Engelking1989} for what we call analogously, however only for totally ordered sets.
    \end{itemize}

    The term \enquote{nonincreasing} is used similarly.

    \item \enquote{Increasing} and \enquote{decreasing} is used by \incite[def. 4.28]{Rudin1976Principles} for what we call analogously, however only for the real numbers.

    \item (Strictly) \enquote{ascending} (resp. \enquote{descending}) is used by \incite[35]{Harzheim2005} for what we call a (strictly) order-preserving (resp. order-reflecting) function.

    \item \incite[132]{ИльинСадовничийСендов1985Том1}, \incite[\S 47]{ФихтенгольцОсновыТом1} and \incite[def. 17]{Зорич2019Том1} use \enquote{возрастающая функция} (resp. \enquote{убывающая функция}) for what we call a strictly order-preserving (resp. order-reversing) map. The terms can be translated as \enquote{growing} and \enquote{diminishing}, respectively. For non-strict maps, the authors suggest adding a \enquote{не} prefix, similarly to how \enquote{non} is added in English.
  \end{itemize}
\end{remark}

\begin{proposition}\label{thm:def:order_function}
  The functions between ordered sets discussed in \fullref{def:order_function} have the following basic properties:
  \begin{thmenum}
    \thmitem{thm:def:order_function/injective_implies_strict} Every \hyperref[def:function_invertibility/injective]{injective} \hyperref[def:order_function/preserving]{order-preserving map} is \hyperref[def:order_function/preserving]{strict}.

    The converse holds for totally ordered sets --- see \fullref{thm:def:totally_ordered_set/embedding_iff_strict}.

    \thmitem{thm:def:preordered_set/homomorphism_is_reflecting} An injective function between preordered sets is a \hyperref[def:first_order_embedding]{first-order embedding} if and only if it is both \hyperref[def:order_function/preserving]{order-preserving} and \hyperref[def:order_function/preserving]{order-reflecting}.
  \end{thmenum}
\end{proposition}
\begin{defproof}
  \SubProofOf{thm:def:order_function/injective_implies_strict} Let \( f: P \to Q \) be an injective order-preserving map.

  Let \( x <_P y \) for some members \( x \) and \( y \) of \( P \). Since \( f \) is order-preserving, we have \( f(x) \leq_Q f(y) \). Since it is also injective, \( f(x) = f(y) \) implies \( x = y \), hence it remains for \( f(x) <_Q f(y) \) to hold.

  Generalizing on \( x \) and \( y \), we conclude that \( f \) is strict.

  \SubProofOf{thm:def:preordered_set/homomorphism_is_reflecting} Let \( f: P \to Q \) be an injective function between preordered sets.

  \SufficiencySubProof* Suppose that \( f \) is an embedding, that is, an injective order-preserving map whose inverse is order-preserving.

  If \( f(x) \leq_Q f(y) \), then
  \begin{equation*}
    x = f^{-1}(f(x)) \leq_P f^{-1}(f(y)) = y.
  \end{equation*}

  We conclude that \( f \) is order-reflecting.

  \NecessitySubProof* Suppose that \( f \) is both order-preserving and order-reflecting.

  If \( f(x) \leq_Q f(y) \), then
  \begin{equation*}
    f^{-1}(f(x)) = x \leq_P y = f^{-1}(f(y)).
  \end{equation*}

  We conclude that \( f^{-1} \) is order-preserving.
\end{defproof}
