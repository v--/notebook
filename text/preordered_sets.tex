\subsection{Preordered sets}\label{subsec:preordered_sets}

\begin{definition}\label{def:preordered_set}\mcite[4]{Birkhoff1948}
  We say that a \hyperref[def:binary_relation]{binary relation} \( \leq \) on \( P \) is a \term{preorder} if it is \hyperref[def:binary_relation/reflexive]{reflexive} and \hyperref[def:binary_relation/transitive]{transitive}. The pair \( (P, \leq) \) is then called a \term{preordered set}.

  It is conventional to use the same symbol \( \leq \) as for \hyperref[def:partially_ordered_set]{partial orders}, however the lack of \hyperref[def:binary_relation/antisymmetric]{antisymmetry} may be confusing --- see \fullref{ex:preorder_nonuniqueness}.

  We define \( \geq \) as the \hyperref[def:binary_relation/inverse]{inverse relation} of \( \leq \).

  We also define the relation \( < \) as \( \leq \) without the \hyperref[def:binary_relation/diagonal]{diagonal relation} \( \Delta \). This condition corresponds to the following axiom:
  \begin{equation}\label{eq:def:preordered_set/compatibility_nonstrict}
    (\xi \leq \eta) \leftrightarrow \parens[\Big]{(\xi < \eta) \vee (\xi \doteq \eta)}.
  \end{equation}

  By adding \( \anon \wedge \neg (x = y) \) to both sides of \eqref{eq:def:preordered_set/compatibility_nonstrict}, using \fullref{thm:de_morgans_laws} and taking irreflexivity of \( < \) into account, we obtain
  \begin{equation}\label{eq:def:preordered_set/compatibility_strict}
    (\xi < \eta) \leftrightarrow \parens[\Big]{(\xi \leq \eta) \wedge \neg (\xi = \eta)}.
  \end{equation}

  We call \( < \) the \term{strict preorder} associated with \( \leq \), in which context we call \( \leq \) \term{nonstrict preorder}. We also define \( > \) as the converse relation of \( < \).

  The elements \( x, y \in P \) are called \term{comparable} if either \( x \leq y \) or \( y \leq x \). That is, they are comparable if they are related by \( \leq \).

  Preordered sets have the following metamathematical properties:
  \begin{thmenum}
    \thmitem{def:preordered_set/theory} Consider a \hyperref[def:first_order_language]{first-order language} \( \mscrL \) with two \hyperref[rem:first_order_formula_conventions/infix]{infix} binary predicate symbols --- \( \leq \) and \( \geq \).

    The theory of preordered sets is a \hyperref[def:first_order_theory]{first-order theory} in \( \mscrL \) consisting of the axioms \eqref{eq:def:binary_relation/reflexive} and \eqref{eq:def:binary_relation/transitive} for \( \leq \) and the compatibility axiom
    \begin{equation}\label{eq:def:preordered_set/theory}
      (\xi \leq \eta) \leftrightarrow (\eta \geq \xi).
    \end{equation}

    We purposely avoid adding \( < \) and \( > \) to this language because that would change the behavior of \hyperref[def:first_order_homomorphism]{first-order homomorphisms}. If needed, we can regard them as \hyperref[rem:predicate_formula]{predicate formulas}.

    If we want to add strict orders to the language, we should also add one of the compatibility axioms \eqref{eq:def:preordered_set/compatibility_nonstrict} or \eqref{eq:def:preordered_set/compatibility_strict} to the resulting theory (it is unnecessary to add both).

    \thmitem{def:preordered_set/homomorphism} The \hyperref[def:first_order_homomorphism]{first-order homomorphisms} between two preordered sets are the \hyperref[def:order_homomorphism/increasing]{nonstrictly order-preserving maps}, which we will discuss shortly.

    \thmitem{def:preordered_set/submodel} Since the theory contains only positive formulas over a language with no functional symbols, any subset \( A \) of the domain of a preordered set \( P \) becomes a preordered set with the induced preorder \( \leq_A \) defined as the restriction of \( \leq_P \) to only elements of \( A \).

    \thmitem{def:preordered_set/trivial} The \hyperref[rem:trivial_structure]{trivial} preordered set is the empty set. It is also the \hyperref[thm:substructures_form_complete_lattice/bottom]{initial substructure} of any preordered set.

    See \fullref{rem:empty_first_order_structures} regarding allowing empty sets as first-order structures.

    \thmitem{def:preordered_set/initial} The \hyperref[thm:substructures_form_complete_lattice/bottom]{initial substructure} of any preordered set is also empty.

    \thmitem{def:preordered_set/category}  We denote the \hyperref[def:category_of_small_first_order_models]{category of \( \mscrU \)-small models} for the theory of preordered sets by \( \ucat{PreOrd} \).

    This category is isomorphic to that of \( \mscrU \)-small preorder categories --- see \fullref{thm:order_category_isomorphism}.
  \end{thmenum}
\end{definition}

\begin{theorem}[Principle of duality for preorders]\label{thm:preorder_duality}\mcite[thm. I.3.2]{Birkhoff1948}
  We define the \term{opposite preordered set} of \( (P, \leq) \) as \( (P, \geq) \) and the \term{opposite formula} \( \varphi^{\opcat} \)as \( \varphi \) in which we swap all instances of \( \leq \) and \( \geq \).

  Then the structure \( (P, \leq) \) \hyperref[def:first_order_model]{satisfies} \( \varphi \) if and only if \( (P, \geq) \) satisfies \( \varphi^{\opcat} \).
\end{theorem}
\begin{comments}
  \item The weaker syntactic statement is that the formula \( \varphi \) is \hyperref[def:proof_derivability]{derivable} in the \hyperref[def:preordered_set/theory]{theory of preordered sets} if and only if the opposite formula \( \varphi^{\opcat} \) is also derivable.

  \item The opposite of the opposite formula of \( \varphi \) is obviously \( \varphi \).

  \item Another form of this duality is formalized in \fullref{thm:order_category_isomorphism}.

  \item The actual replacement can be formalized by performing the \hyperref[def:first_order_substitution/term_in_formula]{simultaneous substitution}
  \begin{equation*}
    \begin{aligned}
      \varphi^{\opcat} \coloneqq \varphi[
        &\tau_1 \leq \sigma_1 \mapsto \tau_1 \geq \sigma_1, &&\tau_1 \geq \sigma_1 \mapsto \tau_1 \leq \sigma_1, \\
        &\vdots                                       &&\vdots \\
        &\tau_n \leq \sigma_n \mapsto \tau_n \geq \sigma_n, &&\tau_n \geq \sigma_n \mapsto \tau_n \leq \sigma_n]
    \end{aligned}
  \end{equation*}
  for all pairs \( (\tau_k, \sigma_k) \) of terms in \( \varphi \).
\end{comments}
\begin{proof}
  Note that the semantics of \( \leq \) and \( \geq \) are swapped along with the swapping of the corresponding symbols in the formulas.
\end{proof}

\begin{definition}\label{def:inequality}\mimprovised
  When regarding some \hyperref[def:preordered_set]{preorder relation} \( \leq \) as a \hyperref[def:first_order_language/pred]{first-order predicate}, we call the corresponding \hyperref[def:first_order_syntax/atomic_formula]{atomic formulas} \term{inequalities}. That is, an inequality is a formula of the form \( \tau \leq \sigma \).

  We distinguish between \term{strict inequalities} \( \tau < \sigma \) and \term{nonstrict inequalities} \( \tau \leq \sigma \).

  As in the case of \hyperref[def:first_order_equation]{equations}, we call the set \hyperref[def:first_order_definability]{defined} by an inequality a \term{solution set}.
\end{definition}

\begin{definition}\label{def:cofinal_set}\mcite[8]{Engelking1989}
  A subset \( A \) of a preordered set \( (P, \leq) \) is called \term{cofinal} if for every \( x \in P \) there exists some \( y \in A \) such that \( x \leq y \).
\end{definition}

\begin{example}\label{ex:def:cofinal_set}
  We list several examples of \hyperref[def:cofinal_set]{cofinal} and non-cofinal sets.

  \begin{itemize}
    \thmitem{ex:def:cofinal_set/finite} In a finite set like \( \set{ 0, 1, 2 } \), the set \( \set{ 2 } \) containing the maximum is cofinal. This is generalized by \fullref{thm:partially_ordered_cofinal_equivalences}.

    \thmitem{ex:def:cofinal_set/integers} Consider the set \( \BbbZ \) of integers. Clearly the set \( 2\BbbZ \) of even integers is cofinal. This is generalized by \fullref{thm:totally_ordered_cofinal_equivalences}.

    \thmitem{ex:def:cofinal_set/net_convergence} Cofinal sets are important in topology because it is used to define \hyperref[def:net_convergence]{convergence of nets}.

    \thmitem{ex:def:cofinal_set/regular_cardinals} \hyperref[def:regular_cardinal]{Regular cardinals} are equal to their own \hyperref[def:cofinality]{cofinality}.
  \end{itemize}
\end{example}

\begin{definition}\label{def:order_homomorphism}
  Fix \hyperref[def:preordered_set]{preordered sets} \( (P, \leq_P) \) and \( (Q, \leq_Q) \) and consider an arbitrary function \( f: P \to Q \). \Fullref{rem:order_homomorphism_terminology} demonstrates how widely the terminology for such functions varies. We try to choose the most unambiguous terminology.

  \begin{thmenum}
    \thmitem{def:order_homomorphism/increasing}\mcite[3]{Birkhoff1948} We call \( f \) \term{order-preserving} or an \term{order homomorphism} if
    \begin{equation}\label{eq:def:order_homomorphism/increasing}
      x \leq_P y \T{implies} f(x) \leq_Q f(y).
    \end{equation}

    If the inequalities are strict, i.e. if
    \begin{equation}\label{eq:def:order_homomorphism/increasing/strict}
      x <_P y \T{implies} f(x) <_Q f(y),
    \end{equation}
    we call \( f \) \term{strictly order-preserving}.

    To disambiguate, we sometimes call function satisfying \eqref{eq:def:order_homomorphism/increasing} \term{nonstrictly order-preserving}.

    In the case of a \hyperref[def:totally_ordered_set]{totally ordered set}, we usually prefer the term (strictly) \term{increasing}.

    While we avoid the term \term{monotone} for non-real-valued functions, we refer to this property as \term{monotonicity}.

    \thmitem{def:order_homomorphism/decreasing} \hyperref[thm:preorder_duality]{Dually}, we call \( f \) \term{order-reversing} or \term{decreasing} if
    \begin{equation}\label{eq:def:order_homomorphism/decreasing}
      x \leq_P y \T{implies} f(x) \geq_Q f(y).
    \end{equation}
    and \term{strictly decreasing} or \term{strictly order-reversing} if
    \begin{equation}\label{eq:def:order_homomorphism/decreasing/strict}
      x <_P y \T{implies} f(x) >_Q f(y).
    \end{equation}

    In the case of a \hyperref[def:totally_ordered_set]{totally ordered set}, we usually prefer the term (strictly) \term{decreasing}.

    \thmitem{def:order_homomorphism/monotone}\mcite[№47]{ФихтенгольцОсновыТом1} In the context of real-valued functions, we use the \term{monotone}  as a collective term for nondecreasing and nonincreasing functions. Similarly, we use \term{strictly monotone} for increasing and decreasing functions, in which context we refer to usual monotone functions as \term{nonstrictly monotone}.

    Outside of analysis, we generally avoid the term \enquote{monotone} due to ambiguity.

    \thmitem{def:order_homomorphism/reflecting}\mcite[example 1.5.9]{Perrone2019} We call \( f \) \term{order-reflecting} if it satisfies \hyperref[def:material_implication/converse]{converse} of \eqref{eq:def:order_homomorphism/increasing}:
    \begin{equation}\label{eq:def:order_homomorphism/reflecting}
      f(x) \leq_Q f(y) \T{implies} x \leq_Q y.
    \end{equation}

    \thmitem{def:order_homomorphism/embedding}\mcite[3]{Birkhoff1948} If \( f \) is an \hyperref[def:function_invertibility/injective]{injective function} that is both order-preserving and order-reflecting, we say that it is an \term{order embedding}.

    \thmitem{def:order_homomorphism/isomorphism}\mcite[3]{Birkhoff1948} If \( f \) is a \hyperref[def:function_invertibility/bijective]{bijective} embedding, we say that it is an \term{order isomorphism}.
  \end{thmenum}
\end{definition}
\begin{comments}
  \item Nonstrict order-preserving maps are used extensively in the theory of \hyperref[subsec:partially_ordered_sets]{partially ordered sets}, in particular in \hyperref[subsec:lattices]{lattice theory}.

  \item Strict order-preserving maps are used in the theory of \hyperref[subsec:partially_ordered_sets]{totally ordered sets}, in particular for \hyperref[subsec:well_ordered_sets]{well-ordered sets} and \hyperref[subsec:ordinals]{ordinals}.

  \item Embeddings and isomorphisms are simpler for totally ordered sets, where \fullref{thm:total_order_embedding_iff_strict} and \fullref{thm:totally_ordered_strict_isomorphisms} hold.
\end{comments}

\begin{remark}\label{rem:order_homomorphism_terminology}
  There is widely varying terminology for the maps in \fullref{def:order_homomorphism}.

  \begin{itemize}
    \item \term{Monotone} is used by
    \begin{itemize}
      \item John Kelley in \cite[3]{Kelley1975}, Paolo Perrone in \cite[example 1.5.9]{Perrone2019} and Saunders Mac Lane in \cite[92]{MacLane1998} for what we call a nonstrictly order-preserving map.
      \item Grigoriy Fichtenholz in \cite[№47]{ФихтенгольцОсновыТом1} and Walter Rudin in \cite[def. 4.28]{Rudin1976Principles} for what we call either a nonincreasing or nondecreasing real-valued function.
    \end{itemize}

    \item \term{Weakly monotone} is used by Saunders Mac Lane in \cite[175]{MacLane1998} for what we call a nondecreasing function.

    \item \term{Strictly monotone} is used Walter Rudin in \cite[def. 4.28]{Rudin1976Principles} for what we call an increasing real-valued function.

    \item \term{Isotone} is used by Garret Birkhoff in \cite[3]{Birkhoff1948} and John Kelley in \cite[3]{Kelley1975} for what we call a nondecreasing function.

    \item \term{Order-preserving} is used by
    \begin{itemize}
       \item Garret Birkhoff in \cite[3]{Birkhoff1948}, John Kelley in \cite[3]{Kelley1975} and Saunders Mac Lane in \cite[95]{MacLane1998} for what we call a nondecreasing function.

       \item Ryszard Engelking in \cite[4]{Engelking1989} for what we call a strictly order-preserving function, however only for totally ordered sets.
    \end{itemize}

    \item \term{Order-reversing} is used by Saunders Mac Lane in \cite[95]{MacLane1998} for what we call analogously.

    \item \term{Order-reflecting} is used by Paolo Perrone in \cite[example 1.5.9]{Perrone2019} for what we call analogously.

    \item \term{Nondecreasing} is used by
    \begin{itemize}
      \item Garret Birkhoff in \cite[208]{Birkhoff1948} for what we call analogously, in the context of real-valued functions.
      \item Ryszard Engelking in \cite[8]{Engelking1989} for what we call analogously, however only for totally ordered sets.
      \item Grigoriy Fichtenholz in \cite[№47]{ФихтенгольцОсновыТом1} for what we call analogously, however only for the real numbers.
    \end{itemize}

    Both use \term{nonincreasing} similarly.

    \item \term{Increasing} is used by Grigoriy Fichtenholz in \cite[№47]{ФихтенгольцОсновыТом1} and Walter Rudin in \cite[def. 4.28]{Rudin1976Principles} for what we call analogously, however only for the real numbers (both use \term{decreasing} in the same context).
  \end{itemize}
\end{remark}

\begin{proposition}\label{thm:order_embedding_is_strict}
  Every \hyperref[def:function_invertibility/injective]{injective} \hyperref[def:order_homomorphism/increasing]{order homomorphism} is \hyperref[def:order_homomorphism/increasing]{strict}.
\end{proposition}
\begin{comments}
  \item For totally ordered sets, the converse holds --- see \fullref{thm:total_order_embedding_iff_strict}.
\end{comments}
\begin{proof}
  Let \( f: P \to Q \) be an injective homomorphism.

  Let \( x <_P y \) for some members \( x \) and \( y \) of \( P \). Since \( f \) is an order homomorphism, we have \( f(x) \leq_Q f(y) \). Since it is also injective, \( f(x) = f(y) \) implies \( x = y \), which contradicts our previous assumption.

  Therefore, \( f(x) <_Q f(y) \). Generalizing on \( x \) and \( y \), we conclude that \( f \) is a strict order homomorphism.
\end{proof}
