\section{Preordered sets}\label{sec:preordered_sets}

\paragraph{Preorders}

\begin{definition}\label{def:preordered_set}\mcite[13]{Harzheim2005OrderedSets}
  We say that a \hyperref[def:binary_relation]{binary relation} \( \leq \) on \( P \) is a \term[bg=преднаредба (\cite[9]{Проданов1982ФункционаленАнализЧаст1}), ru=предпорядок (\cite[def 3.1]{Гуров2013ТеорияРешёток})]{preorder} if it is \hyperref[def:binary_relation/reflexive]{reflexive} and \hyperref[def:binary_relation/transitive]{transitive}. The pair \( (P, \leq) \) is then called a \term{preordered set}.

  It is conventional to use the same symbol \( \leq \) as for \hyperref[def:partially_ordered_set]{partial orders}, however the lack of \hyperref[def:binary_relation/antisymmetric]{antisymmetry} may be confusing --- see \cref{ex:preorder_nonuniqueness}.

  Preordered sets have the following metamathematical properties:
  \begin{thmenum}[series=def:preordered_set]
    \thmitem{def:preordered_set/inverse}\mcite[14]{Harzheim2005OrderedSets} We define \( \geq \) as the \hyperref[def:binary_relation/converse]{inverse relation} of \( \leq \).

    \thmitem{def:preordered_set/strict}\mcite[14]{Harzheim2005OrderedSets} We also define the relation \( < \) as \( \leq \) minus the \hyperref[def:binary_relation/diagonal]{diagonal relation} \( \Delta \). This condition corresponds to the following axiom:
    \begin{equation}\label{eq:def:preordered_set/compatibility_nonstrict}
      (\xi \leq \eta) \syniff \parens[\big]{(\xi < \eta) \synvee (\xi \syneq \eta)}.
    \end{equation}

    By adding \( \anon \synwedge \synneg (x = y) \) to both sides of \eqref{eq:def:preordered_set/compatibility_nonstrict}, using \fullref{thm:de_morgans_laws} and taking irreflexivity of \( < \) into account, we obtain
    \begin{equation}\label{eq:def:preordered_set/compatibility_strict}
      (\xi < \eta) \syniff \parens[\big]{(\xi \leq \eta) \synwedge \synneg (\xi = \eta)}.
    \end{equation}

    We call \( < \) the \term{strict preorder} associated with \( \leq \), in which context we call \( \leq \) \term{nonstrict preorder}. We also define \( > \) as the converse relation of \( < \).

    \thmitem{def:preordered_set/comparable}\mcite[29]{Harzheim2005OrderedSets} We call the elements \( x \) and \( y \) are \term1{comparable} if either \( x \leq y \) or \( y \leq x \) and \term{incomparable} otherwise.
  \end{thmenum}

  Preordered sets have the following metamathematical properties:
  \begin{thmenum}[resume=def:preordered_set]
    \thmitem{def:preordered_set/theory} Consider a \hyperref[def:first_order_language]{first-order language} \( \mscrL \) with two \hyperref[def:fol_signature/notation]{infix} binary predicate symbols --- \( \leq \) and \( \geq \).

    The theory of preordered sets is a \hyperref[def:first_order_theory]{first-order theory} in \( \mscrL \) consisting of the axioms \eqref{eq:def:binary_relation/reflexive} and \eqref{eq:def:binary_relation/transitive} for \( \leq \) and the compatibility axiom
    \begin{equation}\label{eq:def:preordered_set/theory}
      (\xi \leq \eta) \syniff (\eta \geq \xi).
    \end{equation}

    We purposely avoid adding \( < \) and \( > \) to this language because that would change the behavior of \hyperref[def:fol_homomorphism]{first-order homomorphisms}. If needed, we can regard them as \hyperref[con:formula_defined_predicate]{formula-defined predicates}.

    If we want to add strict orders to the language, we should also add one of the compatibility axioms \eqref{eq:def:preordered_set/compatibility_nonstrict} or \eqref{eq:def:preordered_set/compatibility_strict} to the resulting theory (it is unnecessary to add both).

    \thmitem{def:preordered_set/homomorphism} The \hyperref[def:fol_homomorphism]{first-order homomorphisms} between two preordered sets are the \hyperref[def:order_function/preserving]{nonstrictly order-preserving maps}, which we will discuss shortly.

    \thmitem{def:preordered_set/submodel} Since the theory contains only positive formulas over a language with no functional symbols, any subset \( A \) of the domain of a preordered set \( P \) becomes a preordered set with the induced preorder \( \leq_A \) defined as the restriction of \( \leq_P \) to only elements of \( A \).

    \thmitem{def:preordered_set/category}  We denote the \hyperref[def:category_of_first_order_models]{category of \( \mscrU \)-small models} for the theory of preordered sets by \( \ucat{PreOrd} \).

    This category is isomorphic to that of \( \mscrU \)-small preorder categories --- see \fullref{thm:order_category_isomorphism}.

    \thmitem{def:preordered_set/opposite}\mcite[3]{Birkhoff1967LatticeTheory} The categories of preordered sets have a natural \hyperref[con:opposite_object]{opposite object functor}. We define \( (P, \leq)^{\oppos} \) as \( (P, \geq) \). For the relation symbols themselves, we define \( \geq^{\oppos} \) as \( \leq \) and vice versa.

    Because of \fullref{thm:order_duality}, opposite preordered sets are called \term{dual}.
  \end{thmenum}
\end{definition}
\begin{comments}
  \item The term \enquote{preorder} is also used in
  \cite[133]{Bourbaki2004TheoryOfSets},
  \cite[3]{Grätzer2011LatticeTheory},
  \cite[\S 5.4.2]{Mimram2020ProgramEqualsProof},
  \cite[example 9.1.14]{UnivalentFoundationsProgram2013HoTT},
  \cite[10]{ЦаленкоШульгейфер1974ОсновыТеорииКатегорий} (as \enquote{предпорядок}),
  \cite[def. 3.1]{Гуров2013ТеорияРешёток} (as \enquote{предпорядок}) and
  \cite[66]{БелоусовТкачёв2004ДискретнаяМатематика} (as \enquote{предпорядок}). Another possibility is \enquote{quasi-order}, used in
  \cite[20]{Birkhoff1967LatticeTheory},
  \cite[\S 3.1]{Harzheim2005OrderedSets} and
  \cite[445]{Rotman2015AdvancedModernAlgebraPart1}.

  \item This concept is unrelated to pre-order enumeration of \hyperref[def:ordered_tree]{ordered trees} defined in \cref{def:ordered_tree_enumeration}.
\end{comments}

\paragraph{Duality}

\begin{concept}\label{con:duality}
  A quintessential notion in mathematics is that of \term[en=duality (\cite[188]{GowersEtAl2008PrincetonCompanion})]{duality}. \incite[5]{ConwayGuy1998BookOfNumbers} describe it as follows:
  \begin{displayquote}
    A concept exhibits \enquote{duality} when it has two aspects or has a \enquote{dual} purpose.
  \end{displayquote}

  The authors only mention duality while going through the etymology of English words related to the number \( 2 \). A discussion focused on abstract mathematics can be found in \cite[\S III.19]{GowersEtAl2008PrincetonCompanion}, which starts as follows:
  \begin{displayquote}
    Duality is an important general theme that has manifestations in almost every area of mathematics. Over and over again, it turns out that one can associate with a given mathematical object a related, \enquote{dual} object that helps one to understand the properties of the object one started with. Despite the importance of duality in mathematics, there is no single definition that covers all instances of the phenomenon.
  \end{displayquote}

  More formally, we may regard any \hyperref[def:morphism_invertibility/involution]{involution} as a form of duality. Various \hyperref[con:opposite_object]{opposite objects} are described by involutions that invert the operands of a binary operation or relation.

  A duality may fail to be an involution like in the case of a \hyperref[rem:double_dual]{double dual} vector spaces, where \( V \) is isomorphic to a subspace of \( V^{**} \).

  In some cases such as \fullref{thm:order_duality}, \fullref{thm:lattice_duality} and \fullref{thm:boolean_algebra_duality}, the syntax of the object language allows encoding the duality.
\end{concept}

\begin{theorem}[Principle of duality for ordered sets]\label{thm:order_duality}\mcite[thm. 1.20]{DaveyPriestley2002LatticeTheory}
  A generalization of \hyperref[def:elementary_equivalence]{elementary equivalence} holds for a \hyperref[def:preordered_set]{preordered set} and its \hyperref[def:preordered_set/opposite]{opposite}.

  Consider the \hyperref[def:preordered_set/theory]{first-order theory of preordered sets}. Denote by \( \varphi^{\oppos} \) the transformation of the formula \( \varphi \) in which we swap all instances of \( \synleq \) and \( \syngeq \) via \fullref{alg:fol_formula_signature_translation}. In the extended theory with strict inequalities we must also swap \( \synless \) and \( \syngreater \).

  \begin{thmenum}
    \thmitem{thm:order_duality/assignment} For every preordered set \( (P, \leq) \), every \hyperref[def:fol_variable_assignment]{variable assignment} \( v \) into \( P \) and every formula \( \varphi \), we have
    \begin{equation*}
      \Bracks{\varphi}_{(P, \leq)}^v = \Bracks{\varphi^{\oppos}}_{(P, \geq)}^v.
    \end{equation*}

    \thmitem{thm:order_duality/equisatisfiable} The set \( \Gamma \) of formulas is \hyperref[def:equisatisfiability]{equisatisfiable} with \( \Gamma^{\oppos} \).

    \thmitem{thm:order_duality/tautology} The formula \( \varphi \) is a \hyperref[def:fol_semantics/tautology]{tautology} if and only if \( \varphi^{\oppos} \) is.
  \end{thmenum}
\end{theorem}
\begin{comments}
  \item Another form of this principle is formalized in \fullref{thm:order_category_isomorphism}.

  \item Variations of this principle include \fullref{thm:lattice_duality} and \fullref{thm:boolean_algebra_duality}.
\end{comments}
\begin{proof}
  The denotations of \( \leq \) and \( \geq \) are swapped along with the corresponding symbols.
\end{proof}

\begin{remark}\label{rem:duality}
  We list various manifestations of \hyperref[con:duality]{duality}:
  \begin{itemize}
    \item The multitude of \hyperref[con:opposite_object]{opposite objects} listed in \cref{rem:con:opposite_object}.

    \item \hyperref[def:dual_vector_space]{Dual} and \hyperref[rem:double_dual]{double dual} \hyperref[def:vector_space]{vectors spaces}, as well as \hyperref[def:continuous_dual_space]{continuous dual spaces}.
  \end{itemize}
\end{remark}

\paragraph{Inequalities}

\begin{definition}\label{def:inequality}\mimprovised
  The predicate application formulas in the \hyperref[def:preordered_set/theory]{first-order theory of preordered sets} are either \( \tau \synleq \sigma \) or \( \tau \syngeq \sigma \) for some terms \( \tau \) and \( \sigma \). We call these formulas \term{inequalities}.

  In the extended theory with symbols \( \synless \) and \( \syngreater \), we distinguish between the \term{strict inequality} \( \tau \synless \sigma \) and its corresponding \term{nonstrict inequality} \( \tau \syngeq \sigma \).

  As in the case of \hyperref[def:fol_equation]{equations}, we call the set \hyperref[def:first_order_definability]{defined} by an inequality a \term{solution set}.
\end{definition}

\paragraph{Extremal points}

\begin{definition}\label{def:extremal_points}
  We introduce the following terminology for extremal elements of a preordered set. These definitions are often given only for \hyperref[def:partially_ordered_set]{partially ordered sets}, but we prefer a more general setting that allows us to cover some cases like the \hyperref[thm:semiring_divisibility_order]{divisibility preorder}. The downside of this generality is that, without \hyperref[def:binary_relation/antisymmetric]{antisymmetry}, some common notions like maxima and suprema lack uniqueness.

  The notions on the left and on the right are \hyperref[thm:order_duality]{dual}, but we discuss both nonetheless.

  \begin{thmenum}
    \thmitem{def:extremal_points/bounds}
    \begin{paracol}{2}
      \begin{leftcolumn}\mcite[21]{Harzheim2005OrderedSets}
        An \term[bg=горна граница (\cite[18]{Тагамлицки1971ДиференциалноСмятане}) / мажоранта (\cite[10]{Проданов1982ФункционаленАнализЧаст1}), ru=верхняя грань/мажоранта (\cite[def. 3.8]{Гуров2013ТеорияРешёток})]{upper bound} for the set \( A \) is an element \( m \) of the ambient preordered set such that \( x \leq m \) for every \( x \in A \).

        We say that the upper bound is \term{strict} if it does not itself belong to \( A \).

        If \( A \) has at least one upper bound, we say that \( A \) is \term{bounded from above}.
      \end{leftcolumn}

      \begin{rightcolumn}
        Dually, \( m \) is a \term[bg=долна граница (\cite[18]{Тагамлицки1971ДиференциалноСмятане}) / миноранта (\cite[10]{Проданов1982ФункционаленАнализЧаст1}), ru=нижняя грань/миноранта (\cite[def. 3.8]{Гуров2013ТеорияРешёток})]{lower bound} of \( A \) if \( m \leq x \) for every \( x \in A \).

        If \( A \) has a lower bound, we say that \( A \) is \term{bounded from below}.

        We say that \( A \) is \term[bg=ограничено (множество) (\cite[19]{Тагамлицки1971ДиференциалноСмятане}), ru=ограниченное (множество) (\cite[39]{Зорич2019АнализЧасть1})]{bounded} if it is both bounded from above and from below.
      \end{rightcolumn}
    \end{paracol}

    \thmitem{def:extremal_points/maximal_and_minimal_element}
    \begin{paracol}{2}
      \begin{leftcolumn}\mimprovised
        A \term[ru=максимальный (елемент) (\cite[def. 3.6]{Гуров2013ТеорияРешёток}), en=maximal (element) (\cite[33]{Harzheim2005OrderedSets})]{maximal element} for the set \( A \subseteq P \) is a member \( m \) of \( A \) such that, whenever \( x \geq m \) for some \( x \in A \), we have \( x \leq m \).

        If antisymmetry holds, the last \enquote{\( x \leq m \)} can be simplified to \enquote{\( x = m \)}.
      \end{leftcolumn}

      \begin{rightcolumn}
        Dually, \( m \in A \) is \term[ru=минимальный (елемент) (\cite[def. 3.6]{Гуров2013ТеорияРешёток}), en=minimal (element) (\cite[33]{Harzheim2005OrderedSets})]{minimal} if, whenever \( x \leq m \) for some \( x \in A \), we have \( x \geq m \).

        If antisymmetry holds, the last \enquote{\( x \geq m \)} can be simplified to \enquote{\( x = m \)}.
      \end{rightcolumn}
    \end{paracol}

    \thmitem{def:extremal_points/greatest_and_least}
    \begin{paracol}{2}
      \begin{leftcolumn}\mcite[22]{Harzheim2005OrderedSets}
        If \( m \) is an upper bound of \( A \) that belongs to \( A \), we call it a \term[ru=наибольший (елемент) (\cite[def. 3.6]{Гуров2013ТеорияРешёток})]{greatest element}.

        If antisymmetry holds, greatest elements are unique.
      \end{leftcolumn}

      \begin{rightcolumn}
        Dually, if \( m \) is a lower bound of \( A \) that belongs to \( A \), we call it a \term[ru=наименьший (елемент) (\cite[def. 3.6]{Гуров2013ТеорияРешёток})]{least element}.

        If antisymmetry holds, least elements are unique
      \end{rightcolumn}
    \end{paracol}

    \thmitem{def:extremal_points/maximum_and_minimum}
    \begin{paracol}{2}
      \begin{leftcolumn}\mimprovised
        In totally ordered sets, due to \cref{thm:def:totally_ordered_set/maximal_iff_greatest} an element of \( A \) is maximal if and only if it is the greatest element of \( A \). We call such an element the \term{maximum} of \( A \) and denote it by \( \max A \).
      \end{leftcolumn}

      \begin{rightcolumn}
        Dually, in a totally ordered set an element of \( A \) is minimal if and only if it is the least element of \( A \). We call such an element the \term{minimum} of \( A \) and denote it by \( \min A \).
      \end{rightcolumn}
    \end{paracol}

    \thmitem{def:extremal_points/supremum_and_infimum}
    \begin{paracol}{2}
      \begin{leftcolumn}\mcite[22]{Harzheim2005OrderedSets}
        If \( m \) is a minimum among all upper bounds of \( A \), we call it a \term[bg=точна горна граница (\cite[19]{Тагамлицки1971ДиференциалноСмятане}), ru=точная верхняя граница (\cite[\S 5.12]{Ляпин1960Полугруппы})]{least upper bound} or \term[bg=супремум (\cite[10]{Проданов1982ФункционаленАнализЧаст1}), ru=супремум (\cite[\S 5.12]{Ляпин1960Полугруппы})]{supremum} of \( A \).

        If antisymmetry holds, suprema are unique, and we use the notation \( \sup A \).
      \end{leftcolumn}

      \begin{rightcolumn}
        Dually, if \( m \) is a maximum among all lower bounds of \( A \), we call it a \term[bg=точна долна граница (\cite[19]{Тагамлицки1971ДиференциалноСмятане}), ru=точная нижняя граница (\cite[\S 5.12]{Ляпин1960Полугруппы})]{greatest lower bound} or \term[bg=инфимум (\cite[10]{Проданов1982ФункционаленАнализЧаст1}), ru=супремум (\cite[\S 5.12]{Ляпин1960Полугруппы})]{infimum} of \( A \).

        If antisymmetry holds, infima are unique, and we use the notation \( \inf A \).
      \end{rightcolumn}
    \end{paracol}

    \thmitem{def:extremal_points/top_and_bottom}
    \begin{paracol}{2}
      \begin{leftcolumn}\mcite[15]{DaveyPriestley2002LatticeTheory}
        If the ambient space itself has a maximum, we call it a \term{top element}.

        If antisymmetry holds, the top element is unique, and we denote it via \( \top \).
      \end{leftcolumn}

      \begin{rightcolumn}
        Dually, if the ambient space has a minimum, we call it a \term{bottom element}.

        If antisymmetry holds, the bottom element is unique, and we use \( \bot \).
      \end{rightcolumn}
    \end{paracol}
  \end{thmenum}
\end{definition}

\begin{proposition}\label{thm:def:extremal_points}
  Fix a \hyperref[def:preordered_set]{preordered set} \( P \) and some nonempty subset \( A \) of \( P \). The extremal point notions from \cref{def:extremal_points} have the following basic properties:
  \begin{thmenum}
    \thmitem{thm:def:extremal_points/empty_exact_bounds} If \( P \) has a bottom element, it is a supremum of the empty set.

    Dually, if \( P \) has a top element, it is an infimum of the empty set.

    \thmitem{thm:def:extremal_points/greatest_is_maximal} The greatest element of \( A \), if it exists, is also a maximal element of \( A \).

    Dually, the least element of \( A \) is a minimal element of \( A \).

    The converse holds in \hyperref[def:totally_ordered_set]{totally ordered sets} --- see \cref{thm:def:totally_ordered_set/maximal_iff_greatest}.

    \thmitem{thm:def:extremal_points/greatest_is_supremum} The greatest element of \( A \), if it exists, is the supremum of \( A \).

    Dually, the least element of \( A \) is the infimum of \( A \).

    \thmitem{thm:def:extremal_points/finite_set_has_maximal_element} If \( A \) is \hi{finite}, it has a maximal element and a minimal element.

    \thmitem{thm:def:extremal_points/unique_maximal_element} If \( m \) is a unique maximal element of \( A \) and if every nonempty subset of \( A \) has at least one maximal element, then \( m \) is a greatest element of \( A \).

    Dually, if \( m \) is a unique minimal element and every nonempty subset of \( A \) has a minimal element, then \( m \) is a minimum.
  \end{thmenum}
\end{proposition}
\begin{proof}
  \SubProofOf{thm:def:extremal_points/empty_exact_bounds} A supremum is a minimum among all upper bounds. All elements of \( P \) are vacuously upper bounds of \( \varnothing \) since there is nothing to compare them to, so a minimum of \( P \) is a supremum of \( \varnothing \).

  \SubProofOf{thm:def:extremal_points/greatest_is_maximal} Let \( m \) be a greatest element of \( A \).

  Let \( a \) be a member of \( A \) such that \( m \leq a \). But \( m \leq a \) since \( m \) is an upper bound of \( A \). Generalizing on \( a \), we conclude that \( m \) is maximal.

  \SubProofOf{thm:def:extremal_points/greatest_is_supremum} Let \( m \) be a greatest element of \( A \).

  Let \( u \) be any upper bound of \( A \). Then, in particular, \( m \leq u \) because \( m \) belongs to \( A \). Generalizing on \( u \), we conclude that \( m \) is a least upper bound of \( A \).

  \SubProofOf{thm:def:extremal_points/finite_set_has_maximal_element} Suppose that \( A \) is finite. Label the elements of \( A \) as follows:
  \begin{equation*}
    a_1, a_2, \ldots, a_n.
  \end{equation*}

  We will use induction on \( n \) to show that \( A \) has a maximal element.
  \begin{itemize}
    \item The base case \( n = 1 \) is trivial --- \( A \) has one element, which is maximal.
    \item Suppose that every set of size \( n - 1 \) has a maximal element. Let \( m \) be a maximal element of \( A \setminus \set{ a_n } \) . Then we have the following possibilities:
    \begin{itemize}
      \item If \( m \) is not comparable to \( a_n \), then both \( m \) and \( a_n \) are maximal for \( A \).

      \item If \( m \leq a_n \), then \( a_n \) is maximal for \( A \). Indeed, if \( a_k \geq a_n \) for \( k < n \), then \( a_k \leq m \) and, by transitivity, \( a_k \leq a_n \).

      \item If \( m \geq a_n \), the \( m \) is maximal for \( A \). Indeed, if \( a_k \geq m \) for \( k < n \), then \( a_k \leq m \) because \( m \) is maximal for \( A \setminus \set{ a_n } \), and if \( a_n \geq m \), we already have \( m \geq a_n \) by assumption.
    \end{itemize}

    In all cases, we have shown that \( A \) has at least one maximal element.
  \end{itemize}

  \SubProofOf{thm:def:extremal_points/unique_maximal_element} Suppose that every nonempty subset of \( A \) has a maximal element. Suppose also that \( m \) is a unique maximal element of \( A \) itself.

  Consider the set \( B \) of elements incomparable to \( m \). Suppose that \( M \) is nonempty. Let \( m' \) be a maximal element of \( B \). Since \( m \) and \( m' \) are incomparable, \( m' \) is then a maximal element of \( A \).

  But we have assumed that \( m \) is unique, hence \( m = m' \), which contradicts our choice of \( m' \).

  The obtained contradiction shows that \( B \) is empty. Then every element of \( A \) is comparable to \( m \). Fix an arbitrary element \( a \) of \( A \).
  \begin{itemize}
    \item If \( a \geq m \), then, since \( m \) is maximal, we have \( a \leq m \).
    \item The remaining option is \( a \leq m \).
  \end{itemize}

  Therefore, \( m \) is a greatest element of \( A \).
\end{proof}

\begin{example}\label{ex:thm:def:extremal_points}
  We list examples of for the extremal point notions from \cref{def:extremal_points}:
  \begin{thmenum}
    \thmitem{ex:thm:def:extremal_points/empty} Fix some arbitrary preordered set \( P \) and consider the empty subset \( \varnothing \).

    Clearly every member of \( P \) is both an upper and a lower bound of \( \varnothing \). If \( P \) has a bottom element, it is a supremum of \( \varnothing \) because it is a minimum among all upper bounds of \( \varnothing \), that is, among all elements of \( P \). Similarly, if \( P \) has a top element, it is an infimum of \( \varnothing \).

    Since \( \varnothing \) is empty, it cannot have a greatest element, nor a maximal element.

    \thmitem{ex:thm:def:extremal_points/one} Consider the one-element set \( \set{ a } \). The only possible reflexive relation entails \( a \leq a \), so \( a \) is the top and bottom of \( \set{ a } \).

    \thmitem{ex:thm:def:extremal_points/two_isolated} Consider the two-element set \( \set{ a, b } \) with the \hyperref[def:binary_relation/diagonal]{diagonal relation} (every element is related only to itself).

    Then \( a \) is the only upper bound of \( \set{ a } \), hence also a maximal element, greatest element and supremum. It is similarly a lower bound, minimal element, minimum and infimum.

    But for \( \set{ a, b } \), it is neither an upper nor lower bound.

    On the other hand, both \( a \) and \( b \) are maximal elements of \( \set{ a, b } \), because there exist no elements greater than them. Similarly, they are both minimal.

    \thmitem{ex:thm:def:extremal_points/two_connected} Consider again the two-element set \( \set{ a, b } \), but this time with \( a \leq b \). \Cref{thm:two_element_lattice} implies that all such preordered sets are isomorphic.

    The \( a \) is a bottom element and \( b \) is a top element.

    \thmitem{ex:thm:def:extremal_points/two_equivalent} Consider yet again the two-element set \( \set{ a, b } \), but this time with both \( a \leq b \) and \( b \leq a \).

    Both of them are a top element and a bottom element. Because we lack antisymmetry in general, we cannot conclude that \( a = b \), so we have two distinct top elements.

    \thmitem{ex:thm:def:extremal_points/three_incomparable} Consider the three-element set \( \set{ a, b, c } \) such that \( a \leq c \) and \( b \leq c \).

    Clearly \( c \) is a top element. But there is no bottom. Both \( a \) and \( b \) are minimal elements of \( \set{ a, b, c } \), but neither of them is a minimum because they are not comparable.

    \thmitem{ex:thm:def:extremal_points/unique_maximal_element_not_greatest}\mcite{MathSE:unique_maximal_element_that_is_not_greatest} Consider the set \( \BbbZ \) of integers with the standard ordering from \cref{def:integer_ordering}.

    Adjoin some symbol \( u \) and define a relation to \( \BbbZ \cup \set{ u } \) extending the ordering on \( \BbbZ \) with \( u \leq u \).

    Then \( u \) is incomparable with any integer. It is the unique maximal element of \( \BbbZ \cup \set{ u } \) because no integer is greater than \( u \). But, since it is incomparable with integers, \( u \) is not a greatest element.

    \thmitem{ex:thm:def:extremal_points/nonunique_maxima} Consider again the integers \( \BbbZ \), but this time adjoin two incomparable elements --- \( \infty \) and \( \tieinfty \) --- and let both of them be greater than any integer.

    Then both \( \infty \) and \( \tieinfty \) are upper bounds of \( \BbbZ \) in \( \BbbZ \cup \set{ \infty, \tieinfty } \). Furthermore, both of them are minimal in the set \( \set{ \infty, \tieinfty } \) of upper bounds, but, since they are not comparable, neither is a least upper bound.
  \end{thmenum}
\end{example}

\paragraph{Functions between ordered sets}

\begin{definition}\label{def:order_function}\mimprovised
  Fix \hyperref[def:preordered_set]{preordered sets} \( (P, \leq_P) \) and \( (Q, \leq_Q) \).

  \begin{thmenum}
    \thmitem{def:order_function/preserving} We call the function \( f: P \to Q \) \term[en=order-preserving (mapping) (\cite[def. 9.2]{Harzheim2005OrderedSets})]{order-preserving} if
    \begin{equation}\label{eq:def:order_function/preserving}
      x \leq_P y \T{implies} f(x) \leq_Q f(y).
    \end{equation}

    If the inequalities are strict, i.e. if
    \begin{equation}\label{eq:def:order_function/preserving/strict}
      x <_P y \T{implies} f(x) <_Q f(y),
    \end{equation}
    we call \( f \) \term{strictly order-preserving}.

    To disambiguate, we sometimes call function satisfying \eqref{eq:def:order_function/preserving} \term{nonstrictly order-preserving}.

    In \hyperref[def:totally_ordered_set]{totally ordered sets}, we will also call order-preserving maps \term[ru=возрастающая (функция) (\cite[\S 47]{Фихтенгольц1968ОсновыАнализаТом1}), en=ascending (mapping) (\cite[def. 9.2]{Harzheim2005OrderedSets})]{ascending} (resp. \term{nondescending}) or \term[en=increasing (mapping) (\cite[def. 9.2]{Harzheim2005OrderedSets})]{increasing} (resp. \term[ru=неубывающая (функция) (\cite[\S 47]{Фихтенгольц1968ОсновыАнализаТом1})]{nondecreasing}).

    \thmitem{def:order_function/reversing} Dually, we call \( f \) \term[en=order-reversing map (\cite[\S 3.17]{Schechter1997AnalysisHandbook})]{order-reversing} if
    \begin{equation}\label{eq:def:order_function/reversing}
      x \leq_P y \T{implies} f(x) \geq_Q f(y).
    \end{equation}
    and or \term{strictly order-reversing} if
    \begin{equation}\label{eq:def:order_function/reversing/strict}
      x <_P y \T{implies} f(x) >_Q f(y).
    \end{equation}

    In \hyperref[def:totally_ordered_set]{totally ordered sets}, we will also call order-reversing maps \term[ru=убывающая (функция) (\cite[\S 47]{Фихтенгольц1968ОсновыАнализаТом1}), en=descending (mapping) (\cite[def. 9.2]{Harzheim2005OrderedSets})]{descending} (resp. \term{nonascending}) or \term[en=decreasing (mapping) (\cite[def. 9.2]{Harzheim2005OrderedSets})]{decreasing} (resp. \term[ru=невозрастающая (функция) (\cite[\S 47]{Фихтенгольц1968ОсновыАнализаТом1})]{nonincreasing}).

    \thmitem{def:order_function/monotone} We may collectively refer to either order-preserving or order-reversing maps as \term{monotone}, and to either property as \term{monotonicity}.

    We avoid calling functions \enquote{monotonic} where this may be ambiguous; see \cref{rem:monotone_function_terminology}.
  \end{thmenum}
\end{definition}
\begin{comments}
  \item Nonstrict order-preserving maps are used extensively in the theory of \hyperref[sec:partially_ordered_sets]{partially ordered sets}, in particular in \hyperref[sec:lattices]{lattice theory}, while strict order-preserving maps are used in the theory of \hyperref[sec:partially_ordered_sets]{totally ordered sets}, in particular for \hyperref[sec:ordinals]{ordinals}.

  \item An alternative term for order-preserving maps is \enquote{isotone}; it is used in
  \cite[2]{Birkhoff1967LatticeTheory},
  \cite[30]{Grätzer2011LatticeTheory},
  \cite[def. 1.34(i)]{DaveyPriestley2002LatticeTheory},
  \cite[def. 9.2]{Harzheim2005OrderedSets},
  \cite[\S 3.17]{Schechter1997AnalysisHandbook} and
  \cite[def. 3.9]{Гуров2013ТеорияРешёток} (as \enquote{изотонное изображение}).

  For order-reversing maps, the equivalent is \enquote{antitone}; it is used in
  \cite[30]{Grätzer2011LatticeTheory},
  \cite[def. 1.34(i)]{DaveyPriestley2002LatticeTheory},
  \cite[\S 3.17]{Schechter1997AnalysisHandbook}. In \cite[2]{Birkhoff1967LatticeTheory} \enquote{antitone} is instead used for dual isomorphisms.
\end{comments}

\begin{remark}\label{rem:monotone_function_terminology}
  \hyperref[def:order_function/preserving]{Order-preserving functions} are also called \enquote{monotone}, for example in
  \cite[30]{Grätzer2011LatticeTheory},
  \cite[def. 1.34(i)]{DaveyPriestley2002LatticeTheory},
  \cite[6]{Aigner1997CombinatorialTheory}

  This is unfortunately ambiguous because, in elementary real analysis, \enquote{monotone function} may refer to either an increasing or decreasing real-valued function; see
  \cite[12]{Folland1999RealAnalysis},
  \cite[119]{Tao2022AnalysisI},
  \cite[\S 47]{Фихтенгольц1968ОсновыАнализаТом1},
  \cite[192]{Натансон1974ВещественныйАнализ} and
  \cite[126]{Зорич2019АнализЧасть1}.

  In more abstract real analysis, monotonically increasing functions are generalized to \hyperref[def:banach_space]{Banach spaces} and called simply \enquote{monotone operators}; see \cref{def:banach_space_monotone_operator}.
\end{remark}

\begin{proposition}\label{thm:def:order_function}
  The functions between ordered sets discussed in \cref{def:order_function} have the following basic properties:
  \begin{thmenum}
    \thmitem{thm:def:order_function/injective_implies_strict} Every \hyperref[def:function_invertibility/injective]{injective} \hyperref[def:order_function/preserving]{order-preserving map} is \hyperref[def:order_function/preserving]{strict}.

    The converse holds for totally ordered sets --- see \cref{thm:def:totally_ordered_set/embedding_iff_strict}.

    \thmitem{thm:def:preordered_set/homomorphism_is_reflecting} An injective function between preordered sets is a \hyperref[def:first_order_embedding]{first-order embedding} if and only if it is both \hyperref[def:order_function/preserving]{order-preserving} and \hyperref[def:order_function/preserving]{order-reflecting}.
  \end{thmenum}
\end{proposition}
\begin{defproof}
  \SubProofOf{thm:def:order_function/injective_implies_strict} Let \( f: P \to Q \) be an injective order-preserving map.

  Let \( x <_P y \) for some members \( x \) and \( y \) of \( P \). Since \( f \) is order-preserving, we have \( f(x) \leq_Q f(y) \). Since it is also injective, \( f(x) = f(y) \) implies \( x = y \), hence it remains for \( f(x) <_Q f(y) \) to hold.

  Generalizing on \( x \) and \( y \), we conclude that \( f \) is strict.

  \SubProofOf{thm:def:preordered_set/homomorphism_is_reflecting} Let \( f: P \to Q \) be an injective function between preordered sets.

  \SufficiencySubProof* Suppose that \( f \) is an embedding, that is, an injective order-preserving map whose inverse is order-preserving.

  If \( f(x) \leq_Q f(y) \), then
  \begin{equation*}
    x = f^{-1}(f(x)) \leq_P f^{-1}(f(y)) = y.
  \end{equation*}

  We conclude that \( f \) is order-reflecting.

  \NecessitySubProof* Suppose that \( f \) is both order-preserving and order-reflecting.

  If \( f(x) \leq_Q f(y) \), then
  \begin{equation*}
    f^{-1}(f(x)) = x \leq_P y = f^{-1}(f(y)).
  \end{equation*}

  We conclude that \( f^{-1} \) is order-preserving.
\end{defproof}
