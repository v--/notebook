\section{Propositional completeness}\label{sec:propositional_completeness}

\begin{definition}\label{def:propositional_theory}\mimprovised
  We restate here some definitions related to \hyperref[def:logical_theory]{logical theories}, adapted to \hyperref[def:propositional_logic]{propositional logic}.

  First and foremost, a (syntactic or semantic) \term[en=theory (\cite[def. 1.4.5]{Hinman2005Logic})]{theory} is, as in \cref{def:general_logic_theory}, a set of sentences closed under logical consequence. For any set of sentences \( \Gamma \), we denote its consequence closure by \( \op*{Th}(\Gamma) \).

  \begin{thmenum}
    \thmitem{def:propositional_theory/morphism} There are no nontrivial \hyperref[def:institution/signatures]{signature morphisms} in the \hyperref[def:propositional_institution]{propositional institutions}, so the only possible theory morphisms are inclusion maps.

    \thmitem{def:propositional_theory/extension} We say that \( \Gamma^+ \) is an \term{extension} of \( \Gamma \) if it is a superset of \( \Gamma \).

    \thmitem{def:propositional_theory/category} Based on propositional theories and their extensions, as per \cref{def:category_of_theories}, we have a (syntactic or semantic) category of theories \( \cat{Th} \).

    \thmitem{def:propositional_theory/model}\mcite[def. 1.4.10]{Hinman2005Logic} As in \cref{def:theory_of_institutional_model}, we define the \term{theory} \( \op*{Th}(I) \) \hi{of} a propositional interpretation \( I \) as the set of all sentences valid in \( I \).
  \end{thmenum}
\end{definition}

\paragraph{Consistent theories}

\begin{definition}\label{def:consistent_set_of_sentences}\mcite[def. 3]{Tarski1983FundamentalConceptsOfMetamathematics}
  Fix a \hyperref[def:consequence_operator]{consequence operator} \( \op{Cn} \) on a set of sentences \( S \).

  We say that the set of sentences \( \Gamma \) is \term[ru=противоречивое (множество формул) (\cite[def. 1.3.15]{Герасимов2014Вычислимость})]{inconsistent} with respect to this operator if \( \op{Cn}(\Gamma) \) coincides with \( S \). When \( \Gamma \) is not inconsistent, we say that it is \term{consistent}.
\end{definition}
\begin{comments}
  \item For \hyperref[def:propositional_logic]{propositional logic}, inconsistent sets are characterized in \cref{thm:propositional_natural_deduction_inconsistency} and \cref{thm:propositional_semantic_inconsistency}.

  \item We choose inconsistency as the primitive notion because it is easier to express; it also allows us to rely less on double negation elimination in our proofs.

  \item Within a given \hyperref[def:general_logic]{general logic}, we will by default refer to syntactic consistency, as a counterpart to \hyperref[def:propositional_semantics/satisfaction]{satisfiable theories}.
\end{comments}

\begin{proposition}\label{thm:def:consistent_set_of_sentences}
  \hyperref[def:consistent_set_of_sentences]{Consistent sets of sentences} have the following basic properties:
  \begin{thmenum}
    \thmitem{thm:def:consistent_set_of_sentences/superset} Every superset of an inconsistent set is inconsistent.
    \thmitem{thm:def:consistent_set_of_sentences/subset} Every subset of a consistent set is consistent.
    \thmitem{thm:def:consistent_set_of_sentences/entailment} If \( \Gamma \vdash \varphi \), then \( \Gamma \) is consistent if and only if \( \Gamma \cup \set{ \varphi } \) is.
  \end{thmenum}
\end{proposition}
\begin{proof}
  \SubProofOf{thm:def:consistent_set_of_sentences/superset} Let \( \Gamma \) be an inconsistent set and let \( \Delta \) be a superset.

  Then \eqref{eq:def:consequence_relation/monotonicity} implies that every sentence derivable from \( \Gamma \) can be derived from \( \Delta \). But \( \Gamma \) can derive any sentence, thus \( \Delta \) also can. Then \( \Delta \) is inconsistent.

  \SubProofOf{thm:def:consistent_set_of_sentences/subset} Let \( \Delta \) be a consistent set and let \( \Gamma \) be a subset.

  If we suppose that \( \Gamma \) is inconsistent, \cref{thm:def:consistent_set_of_sentences/superset} would imply that \( \Delta \) also is, which is a contradiction. Then \( \Gamma \) is not inconsistent, i.e. it is consistent.

  \SubProofOf{thm:def:consistent_set_of_sentences/entailment} Let \( \Gamma \vdash \varphi \). Then every sentence derivable from \( \Gamma \cup \set{ \varphi } \) can be derived from \( \Gamma \), and thus the two are equivalent. Then either both are inconsistent or consistent.
\end{proof}

\begin{proposition}\label{thm:propositional_natural_deduction_inconsistency}
  With respect to the \hyperref[def:propositional_natural_deduction]{intuitionistic natural deduction system}, the following conditions are equivalent for a set \( \Gamma \) of \hyperref[def:propositional_formula]{propositional sentences}:
  \begin{thmenum}
    \thmitem{thm:propositional_natural_deduction_inconsistency/abstract} The set \( \Gamma \) is \hyperref[def:consistent_set_of_sentences]{inconsistent}, i.e. every formula is derivable from \( \Gamma \).

    \thmitem{thm:propositional_natural_deduction_inconsistency/contradiction} For every formula \( \varphi \), both \( \varphi \) and \( \synneg \varphi \) are derivable from \( \Gamma \).

    \thmitem{thm:propositional_natural_deduction_inconsistency/bot} The falsum \( \synbot \) is derivable from \( \Gamma \).
  \end{thmenum}
\end{proposition}
\begin{comments}
  \item This is the syntactic counterpart to \cref{thm:propositional_semantic_inconsistency}.
\end{comments}
\begin{proof}
  \ImplicationSubProof{thm:propositional_natural_deduction_inconsistency/abstract}{thm:propositional_natural_deduction_inconsistency/contradiction} Trivial.

  \ImplicationSubProof{thm:propositional_natural_deduction_inconsistency/contradiction}{thm:propositional_natural_deduction_inconsistency/bot} Let \( P \) and \( N \) be \hyperref[def:propositional_natural_deduction_proof_tree]{proof trees} of \( \varphi \) and \( \synneg \varphi \) with assumptions in \( \Gamma \). Then the following is a proof of \( \synbot \) with the same assumptions:
  \begin{equation*}
    \begin{prooftree}
      \hypo{ \Gamma }
      \ellipsis { \( N \) } { \synneg \varphi }

      \hypo{ \Gamma }
      \ellipsis { \( P \) } { \varphi }

      \infer2[\ref{inf:def:propositional_natural_deduction/neg/elim}]{ \synbot }.
    \end{prooftree}
  \end{equation*}

  \ImplicationSubProof{thm:propositional_natural_deduction_inconsistency/bot}{thm:propositional_natural_deduction_inconsistency/abstract} Let \( P \) be a proof tree of \( \synbot \) with assumptions from \( \Gamma \). Then the following is a proof of an arbitrary formula \( \varphi \) with the same assumptions:
  \begin{equation*}
    \begin{prooftree}
      \hypo{}
      \ellipsis { \( P \) } { \synbot }
      \infer1[\ref{inf:def:propositional_natural_deduction/bot/efq}]{ \varphi }
    \end{prooftree}
  \end{equation*}
\end{proof}

\begin{proposition}\label{thm:propositional_semantic_inconsistency}
  With respect to the \hyperref[def:truth_value_algebra/intuitionistic]{intuitionistic} \hyperref[def:propositional_semantics/entailment]{propositional entailment relation}, the following conditions are equivalent for a set \( \Gamma \) of \hyperref[def:propositional_formula]{propositional sentences}:
  \begin{thmenum}
    \thmitem{thm:propositional_semantic_inconsistency/unsatisfiable} Either the semantics is \hyperref[def:truth_value_algebra/degenerate]{degenerate}\fnote{Under degenerate semantics, every set of formulas is inconsistent, so this case is not particularly illuminating.} or otherwise the set \( \Gamma \) is \hyperref[thm:propositional_semantic_inconsistency]{unsatisfiable}, i.e. has no model.

    \thmitem{thm:propositional_semantic_inconsistency/abstract} The set \( \Gamma \) is \hyperref[def:consistent_set_of_sentences]{inconsistent}, i.e. \( \Gamma \) entails every formula.

    \thmitem{thm:propositional_semantic_inconsistency/contradiction} For every formula \( \varphi \), \( \Gamma \) entails both \( \varphi \) and \( \synneg \varphi \).

    \thmitem{thm:propositional_semantic_inconsistency/bot} The set \( \Gamma \) entails the falsum \( \synbot \).
  \end{thmenum}
\end{proposition}
\begin{comments}
  \item This is the semantic counterpart to \cref{thm:propositional_semantic_inconsistency}.
\end{comments}
\begin{proof}
  \ImplicationSubProof{thm:propositional_semantic_inconsistency/unsatisfiable}{thm:propositional_semantic_inconsistency/abstract} Under degenerate semantics, every interpretation satisfies every formula.

  Under non-degenerate semantics, if \( \Gamma \) is unsatisfiable, then, for any formula \( \varphi \), vacuously \( I \vDash \varphi \) whenever \( I \vDash \Gamma \) (since no interpretation satisfies \( \Gamma \)).

  \ImplicationSubProof{thm:propositional_semantic_inconsistency/abstract}{thm:propositional_semantic_inconsistency/contradiction} Trivial.

  \ImplicationSubProof{thm:propositional_semantic_inconsistency/contradiction}{thm:propositional_semantic_inconsistency/bot} Suppose that \( \Gamma \vDash \varphi \) and \( \Gamma \vDash \synneg \varphi \).

  \Cref{thm:intuitionistic_equivalences/negation_bottom} implies that \( \Gamma \vDash (\varphi \synimplies \synbot) \). \Fullref{thm:propositional_semantic_deduction_theorem} then implies that \( \Gamma, \varphi \vDash \synbot \). It remains to eliminate \( \varphi \) from the left --- this is justified by \cref{def:consequence_relation/transitivity} because \( \Gamma \vDash \varphi \).

  \ImplicationSubProof{thm:propositional_semantic_inconsistency/bot}{thm:propositional_semantic_inconsistency/unsatisfiable} Suppose that \( \Gamma \vDash \synbot \).

  Under non-degenerate semantics, \( \Gamma \) has no model. Indeed, if \( I \vDash \Gamma \), then \( I \vDash \synbot \), i.e. \( \Bracks{\synbot}_I = \semtop \), which is impossible since \( \Bracks{\synbot}_I = \sembot \) by definition.
\end{proof}

\paragraph{Lindenbaum-Tarski algebras}

\begin{definition}\label{def:lindenbaum_tarski_algebra}\mcite[27]{Эдельман1975Логика}
  Fix a \hyperref[def:consequence_relation]{consequence relation} \( {\vdash} \) on a set of sentences \( S \).

  For some fixed subset \( \Gamma \) of sentences, consider the \hyperref[def:binary_relation]{binary relation}
  \begin{equation*}
    \varphi \leq_\Gamma \psi \T{if} \Gamma, \varphi \vdash \psi.
  \end{equation*}

  Then \( S \) is a \hyperref[def:preordered_set]{preordered set} with respect to \( \leq_\Gamma \) due to \eqref{eq:def:consequence_relation/reflexivity} and \eqref{eq:def:consequence_relation/transitivity}. We call the corresponding \hyperref[def:antisymmetric_quotient]{antisymmetric quotient} the \term[ru=алгебра Линденбаума-Тарского, en=Lindenbaum-Tarski algebra (\cite[def. 6.3.1]{CitkinMuravitsky2022ConsequenceRelations})]{Lindenbaum-Tarski algebra} of \( \Gamma \). We denote it by \( \BbbL_\Gamma \), and the corresponding projection by \( \pi_\Gamma: S \to \BbbL_\Gamma \).
\end{definition}
\begin{comments}
  \item We will mostly be interested in the Lindenbaum-Tarski algebra \( \BbbL \) of an empty set of premises, where \( \varphi \leq \psi \) if and only if \( \varphi \vdash \psi \).
\end{comments}

\begin{proposition}\label{thm:inconsistent_lindenbaum_tarski_algebra}
  A \hi{nonempty} set of sentences \( \Gamma \) is \hyperref[def:consistent_set_of_sentences]{inconsistent} if and only if its \hyperref[def:lindenbaum_tarski_algebra]{Lindenbaum-Tarski algebra} \( \BbbL_\Gamma \) is trivial, i.e. contains only one equivalence class.
\end{proposition}
\begin{proof}
  \SufficiencySubProof Suppose that \( \Gamma \) is inconsistent. Then \( \Gamma \vdash \psi \) for every formula \( \psi \), and we can extend the premise with \( \varphi \) so that \( \Gamma, \varphi \vdash \psi \), i.e. we have \( \varphi \leq_\Gamma \psi \) for any pair of formulas \( \varphi \) and \( \psi \).

  \NecessitySubProof Suppose that \( \BbbL_\Gamma \) is trivial. Then, for any pair of formulas \( \varphi \) and \( \psi \), we have \( \varphi \leq_\Gamma \psi \), i.e. \( \Gamma, \varphi \vdash \psi \).

  Since \( \Gamma \) is nonempty, we can simply take \( \varphi \) to be any formula in \( \Gamma \), thus obtaining \( \Gamma \vdash \psi \) for any \( \psi \). This shows that \( \Gamma \) is inconsistent.
\end{proof}

\begin{proposition}\label{thm:lindenbaum_tarski_algebras}
  We will now study the \hyperref[def:lindenbaum_tarski_algebra]{Lindenbaum-Tarski algebra} \( \BbbL_\Gamma \) of the set \( \Gamma \) of \hyperref[def:propositional_formula]{propositional sentences} in the different natural deduction systems from \cref{def:propositional_natural_deduction}.

  \begin{thmenum}
    \thmitem{thm:lindenbaum_tarski_algebras/minimal} For minimal logic, \( \BbbL_\Gamma \) is a \hyperref[def:extremal_points/bounds]{bounded from above} \hyperref[def:distributive_lattice]{distributive lattice}.

    We will show this in several steps:
    \begin{thmenum}
      \thmitem{thm:lindenbaum_tarski_algebras/minimal/verum} The coset \( \pi_\Gamma(\syntop) \) is the \hyperref[def:extremal_points/top_and_bottom]{top element}, and this coset contains every formula of \( \Gamma \).

      \thmitem{thm:lindenbaum_tarski_algebras/minimal/disjunction} The \hyperref[def:extremal_points/supremum_and_infimum]{supremum} of \( \pi_\Gamma(\varphi) \) and \( \pi_\Gamma(\psi) \) is \( \pi_\Gamma(\varphi \synvee \psi) \).

      \thmitem{thm:lindenbaum_tarski_algebras/minimal/conjunction} The \hyperref[def:extremal_points/supremum_and_infimum]{infimum} of \( \pi_\Gamma(\varphi) \) and \( \pi_\Gamma(\psi) \) is \( \pi_\Gamma(\varphi \synwedge \psi) \).

      \thmitem{thm:lindenbaum_tarski_algebras/minimal/distributivity} The supremum and infimum distribute over each other.
    \end{thmenum}

    \thmitem{thm:lindenbaum_tarski_algebras/intuitionistic} For intuitionistic logic, \( \BbbL_\Gamma \) is a \hyperref[def:heyting_algebra]{Heyting algebra}.

    We will again show this in several steps:
    \begin{thmenum}
      \thmitem{thm:lindenbaum_tarski_algebras/intuitionistic/falsum} The coset \( \pi_\Gamma(\synbot) \) is the \hyperref[def:extremal_points/top_and_bottom]{bottom element}.

      With this, \( \BbbL_\Gamma \) becomes a bounded distributive lattice, and it remains to define a relative pseudocomplement operation.

      \thmitem{thm:lindenbaum_tarski_algebras/intuitionistic/conditional} The \hyperref[def:heyting_algebra]{relative pseudocomplement} of \( \pi_\Gamma(\varphi) \) and \( \pi_\Gamma(\psi) \) is \( \pi_\Gamma(\varphi \synimplies \psi) \).

      This shows that \( \BbbL_\Gamma \) is a Heyting algebra; the other steps simply characterize some derived operations.

      \thmitem{thm:lindenbaum_tarski_algebras/intuitionistic/negation} The \hyperref[eq:def:heyting_algebra/pseudocomplement]{pseudocomplement} of \( \pi_\Gamma(\psi) \) is \( \pi_\Gamma(\synneg \psi) \).

      \thmitem{thm:lindenbaum_tarski_algebras/intuitionistic/biconditional} The \hyperref[def:heyting_algebra/biconditional]{biconditional} of \( \pi_\Gamma(\varphi) \) and \( \pi_\Gamma(\psi) \) is \( \pi_\Gamma(\varphi \syniff \psi) \).
    \end{thmenum}

    \thmitem{thm:lindenbaum_tarski_algebras/classical} For intuitionistic logic, \( \BbbL_\Gamma \) is a \hyperref[def:boolean_algebra]{Boolean algebra}.
  \end{thmenum}
\end{proposition}
\begin{comments}
  \item This also holds for \hyperref[sec:first_order_logic]{first-order logic} since the same natural deduction rules hold there.
  \item See \cref{ex:con:curry_howard_correspondence/algebraic_types} for how this statement relates to \hyperref[def:simple_algebraic_type]{simple algebraic types} via the \hyperref[con:curry_howard_correspondence]{Curry-Howard correspondence}.
\end{comments}
\begin{proof}
  It will be sufficient to consider the case \( \Gamma = \varnothing \). Indeed, if \( \varphi \vdash \psi \), then we can weaken the hypothesis so that \( \Gamma, \varphi \vdash \psi \), i.e. \( \varphi \leq_\Gamma \psi \).

  \SubProofOf{thm:lindenbaum_tarski_algebras/minimal} Suppose first we are working in minimal logic.

  \SubProofOf{thm:lindenbaum_tarski_algebras/minimal/verum} The rule \ref{inf:def:propositional_natural_deduction/top/intro} requires no premises, hence \( \syntop \) can be derived from no hypotheses, and hence from any sentence \( \varphi \) as a vacuous hypothesis. Thus, \( \varphi \leq \syntop \) for every sentence \( \varphi \), i.e. \( \BbbL \) is bounded from above by \( \pi(\syntop) \).

  Assuming for the moment that \( \Gamma \) is nonempty, we will show that every sentence \( \psi \) from \( \Gamma \) belongs to \( \pi_\Gamma(\syntop) \). We have already shown that \( \psi \leq_\Gamma \syntop \). The converse inequality holds because \( \Gamma \vdash \psi \) for every \( \psi \) in \( \Gamma \), and we can add \( \syntop \) as a hypothesis so that \( \Gamma, \syntop \vdash \psi \), i.e. \( \syntop \leq_\Gamma \psi \).

  \SubProofOf{thm:lindenbaum_tarski_algebras/minimal/disjunction} Consider the cosets \( \pi_\Gamma(\varphi) \) and \( \pi_\Gamma(\psi) \).

  The rule \ref{inf:def:propositional_natural_deduction/or/intro_left} can derive \( \varphi \synvee \psi \) from \( \varphi \) and \ref{inf:def:propositional_natural_deduction/or/intro_right} can derive \( \varphi \synvee \psi \) from \( \psi \), hence \( \varphi \synwedge \psi \) is an upper bound of \( \varphi \) and \( \psi \).

  We must show that it is the least upper bound. Let \( \theta \) also be an upper bound. Then there exists a proof tree \( P \) deriving \( \theta \) from \( \varphi \) and a tree \( R \) deriving \( \theta \) from \( \psi \). Then the following tree derives \( \theta \) from \( \varphi \synvee \psi \):
  \begin{equation*}
    \begin{prooftree}
      \hypo{ [\varphi \synvee \psi]^u }

      \hypo{ \varphi }
      \ellipsis { \( P \) } { \theta }

      \hypo{ \psi }
      \ellipsis { \( R \) } { \theta }

      \infer3[\ref{inf:def:propositional_natural_deduction/or/elim}]{ \theta }
    \end{prooftree}
  \end{equation*}

  \SubProofOf{thm:lindenbaum_tarski_algebras/minimal/conjunction} Consider again the cosets \( \pi_\Gamma(\varphi) \) and \( \pi_\Gamma(\psi) \).

  The rule \ref{inf:def:propositional_natural_deduction/and/elim_left} can derive \( \varphi \) from \( \varphi \synwedge \psi \) and \ref{inf:def:propositional_natural_deduction/and/elim_right} can derive \( \psi \) from \( \varphi \synvee \psi \), hence \( \varphi \synwedge \psi \) is a lower bound of \( \varphi \) and \( \psi \).

  We must show that it is the greatest lower bound. Let \( \theta \) also be a lower bound. Then there exists a proof tree \( P \) deriving \( \varphi \) from \( \theta \) and a tree \( R \) deriving \( \psi \) from \( \theta \). Then the following tree derives \( \varphi \synvee \psi \) from \( \theta \):
  \begin{equation*}
    \begin{prooftree}
      \hypo{ \theta }
      \ellipsis { \( P \) } { \varphi }

      \hypo{ \theta }
      \ellipsis { \( R \) } { \psi }

      \infer2[\ref{inf:def:propositional_natural_deduction/and/intro}]{ \varphi \synvee \psi }
    \end{prooftree}
  \end{equation*}

  \SubProofOf{thm:lindenbaum_tarski_algebras/minimal/distributivity} We show distributivity for \hyperref[def:simple_algebraic_type]{simple algebraic types} via explicit \hyperref[def:type_derivation_tree]{type derivation trees} in \cref{thm:simple_algebraic_type_arithmetic/addition_distributes} and \cref{thm:simple_algebraic_type_arithmetic/multiplication_distributes}.

  Via \fullref{alg:type_derivation_to_proof_tree}, we can convert those derivation trees to proof trees appropriate for proving distributivity in \( \BbbL \).

  \SubProofOf{thm:lindenbaum_tarski_algebras/intuitionistic} Suppose we are working in intuitionistic logic, where the rule \ref{inf:def:propositional_natural_deduction/bot/efq} is available.

  \SubProofOf{thm:lindenbaum_tarski_algebras/intuitionistic/falsum} The rule \ref{inf:def:propositional_natural_deduction/bot/efq} allows deriving any formula \( \varphi \) from \( \synbot \), hence \( \synbot \leq \varphi \), i.e. \( \BbbL \) is bounded from below by \( \pi(\synbot) \).

  \SubProofOf{thm:lindenbaum_tarski_algebras/intuitionistic/conditional} Consider again the cosets \( \pi(\varphi) \) and \( \pi(\psi) \).

  First, we must show that \( \varphi \synwedge (\varphi \synimplies \psi) \leq \psi \); the following proof tree suffices:
  \begin{equation*}
    \begin{prooftree}
      \hypo{ [\varphi \synwedge (\varphi \synimplies \psi)]^u }
      \infer1[\ref{inf:def:propositional_natural_deduction/and/elim_right}]{ \varphi \synimplies \psi }

      \hypo{ [\varphi \synwedge (\varphi \synimplies \psi)]^u }
      \infer1[\ref{inf:def:propositional_natural_deduction/and/elim_left}]{ \varphi }

      \infer2[\ref{inf:def:propositional_natural_deduction/imp/elim}]{ \psi }
    \end{prooftree}
  \end{equation*}

  It remains to show that \( \varphi \synimplies \psi \) is the greatest such element, i.e. we must show that \( \theta \leq \varphi \synimplies \psi \) whenever \( \varphi \synwedge \theta \leq \psi \). Indeed, let \( P \) be a proof tree deriving \( \psi \) from \( \varphi \synwedge \theta \). Then we have the tree
  \begin{equation*}
    \begin{prooftree}
      \hypo{ [\varphi]^u }
      \hypo{ [\theta]^v }
      \infer2[\ref{inf:def:propositional_natural_deduction/and/intro}]{ \varphi \synwedge \theta }
      \ellipsis {\( P \)} { \psi }

      \infer[left label=\( u \)]1[\ref{inf:def:propositional_natural_deduction/imp/intro}]{ \varphi \synimplies \psi }
    \end{prooftree}
  \end{equation*}

  \SubProofOf{thm:lindenbaum_tarski_algebras/intuitionistic/negation} The pseudocomplement of \( \pi(\varphi) \) is by definition \( \pi(\varphi) \rightarrow \pi(\synbot) \). \Cref{thm:lindenbaum_tarski_algebras/conditional} implies that
  \begin{equation*}
    \pi(\varphi) \rightarrow \pi(\synbot) = \pi(\varphi \synimplies \synbot).
  \end{equation*}

  It remains to show that \( \pi(\varphi \synimplies \synbot) = \pi(\synneg \varphi) \). In one direction, we have the proof tree
  \begin{equation*}
    \begin{prooftree}
      \hypo{ [\varphi \synimplies \synbot]^u }
      \hypo{ [\varphi]^v }
      \infer2[\ref{inf:def:propositional_natural_deduction/imp/elim}]{ \synbot }
      \infer[left label=\( v \)]1[\ref{inf:def:propositional_natural_deduction/neg/intro}]{ \synneg \varphi }
    \end{prooftree}
  \end{equation*}

  In the other direction, we have the tree
  \begin{equation*}
    \begin{prooftree}
      \hypo{ [\synneg \varphi]^u }
      \hypo{ [\varphi]^v }
      \infer2[\ref{inf:def:propositional_natural_deduction/neg/elim}]{ \synbot }
      \infer[left label=\( v \)]1[\ref{inf:def:propositional_natural_deduction/imp/intro}]{ \varphi \synimplies \synbot }
    \end{prooftree}
  \end{equation*}

  \SubProofOf{thm:lindenbaum_tarski_algebras/intuitionistic/biconditional} Note that
  \begin{equation*}
    \pi\parens[\big]{ \varphi \syniff \psi } = \pi\parens[\big]{ (\varphi \synimplies \psi) \synwedge (\psi \synimplies \varphi) }.
  \end{equation*}

  This can be shown using the derived rule \ref{inf:thm:propositional_admissible_rules/biconditional_as_conjunction} in one direction and \ref{inf:thm:propositional_admissible_rules/conjunction_of_conditionals} in the other.

  By what we have already shown,
  \begin{align*}
    \pi\parens[\big]{ \varphi \syniff \psi }
    &=
    \pi\parens[\big]{ (\varphi \synimplies \psi) \synwedge (\psi \synimplies \varphi) }
    \reloset {\ref{thm:lindenbaum_tarski_algebras/minimal/conjunction}} = \\ &=
    \pi(\varphi \synimplies \psi) \wedge \pi(\psi \synimplies \varphi)
    \reloset {\ref{thm:lindenbaum_tarski_algebras/intuitionistic/conditional}} = \\ &=
    \parens[\big]{ \pi(\varphi) \rightarrow \pi(\psi) } \wedge \parens[\big]{ \pi(\psi) \rightarrow \pi(\varphi) },
  \end{align*}
  which by definition equals \( \pi(\varphi) \leftrightarrow \pi(\psi) \).

  \SubProofOf{thm:lindenbaum_tarski_algebras/classical} Finally, now that we have shown that \( \BbbL \) is a Heyting algebra, we must show that it is also a Boolean algebra when we consider the additional rule \ref{inf:def:propositional_natural_deduction/bot/raa}.

  \Cref{thm:boolean_algebra_heyting_characterization} implies that \( \pi(\varphi) \leq \pi(\synneg \synneg \varphi) \). We must show the converse inequality; but this is precisely what \ref{inf:def:propositional_natural_deduction/bot/raa} allows us to do:
  \begin{equation*}
    \begin{prooftree}
      \hypo{ [\synneg \synneg \varphi]^u }
      \hypo{ [\synneg \varphi]^v }
      \infer2[\ref{inf:def:propositional_natural_deduction/neg/elim}]{ \synbot }
      \infer[left label=\( v \)]1[\ref{inf:def:propositional_natural_deduction/bot/raa}]{ \varphi }
    \end{prooftree}
  \end{equation*}
\end{proof}

\begin{theorem}[Intuitionistic propositional completeness]\label{thm:intuitionistic_propositional_completeness}
  Fix a set \( \Gamma \) of \hyperref[def:propositional_formula]{propositional sentences} and consider the \hyperref[def:lindenbaum_tarski_algebra]{Lindenbaum-Tarski algebra} \( \BbbL_\Gamma \) of \( \Gamma \).

  If \( \Gamma \vDash \varphi \) with \( \BbbL_\Gamma \) as a \hyperref[def:truth_value_algebra]{truth value algebra}, then \( \Gamma \vdash \varphi \) in the \hyperref[def:propositional_natural_deduction]{intuitionistic natural deduction system}.
\end{theorem}
\begin{comments}
  \item A statement that is often more convenient is given in \cref{thm:intuitionistic_propositional_soundness_and_completeness}.
\end{comments}
\begin{proof}
  Consider the interpretation \( I: \op*{Prop} \to \BbbL_\Gamma \) obtained as a restriction of \( \pi \) to propositional variables. As a consequence of \cref{thm:lindenbaum_tarski_algebras}, the corresponding \hyperref[alg:propositional_denotation]{denotation} is then simply the Lindenbaum-Tarski projection \( \pi_\Gamma: \op*{Form} \to \BbbL_\Gamma \).

  Note that every propositional interpretation satisfies \( \Gamma \) because, for every \( \psi \) in \( \Gamma \), \cref{thm:lindenbaum_tarski_algebras/minimal/verum} implies that
  \begin{equation*}
    \underbrace{\pi_\Gamma(\psi)}_{\Bracks{\psi}_I} = \underbrace{\pi_\Gamma(\syntop)}_{\semtop}.
  \end{equation*}

  Suppose that \( \Gamma \vDash \varphi \). Then every propositional interpretation satisfying \( \Gamma \) also satisfies \( \varphi \). We have already shown that every interpretation satisfies \( \Gamma \). Thus, every interpretation also satisfies \( \varphi \); i.e. for every interpretation \( I \), we have
  \begin{equation*}
    \underbrace{\Bracks{\varphi}_I}_{\pi_\Gamma(\varphi)} = \underbrace{\semtop}_{\pi_\Gamma(\syntop)}.
  \end{equation*}

  It follows that \( \syntop \leq_\Gamma \varphi \), i.e. \( \Gamma, \syntop \vdash \varphi \). There must thus exist a proof tree \( P \) deriving \( \varphi \) from \( \Gamma \cup \set{ \syntop } \).

  If \( P \) contains an assumption for \( \syntop \), we can replace it with an application of the rule \ref{inf:def:propositional_natural_deduction/top/intro}, thus obtaining a proof tree deriving \( \varphi \) from \( \Gamma \).

  Therefore, as desired, we have \( \Gamma \vdash \varphi \).
\end{proof}

\begin{corollary}\label{thm:intuitionistic_propositional_soundness_and_completeness}
  For \hyperref[def:propositional_formula]{propositional sentences}, we have \( \Gamma \vdash \varphi \) in the \hyperref[def:propositional_natural_deduction]{intuitionistic natural deduction system} if and only if \( \Gamma \vDash \varphi \) with respect to \hi{every} \hyperref[def:truth_value_algebra]{truth value algebra}.
\end{corollary}
\begin{proof}
  \SufficiencySubProof If \( \Gamma \vdash \varphi \), \fullref{thm:propositional_natural_deduction_soundness} implies that \( \Gamma \vDash \varphi \) with respect to an arbitrary Heyting algebra.

  \NecessitySubProof Suppose that \( \Gamma \vDash \varphi \) with respect to every Heyting algebra, and in particular for the \hyperref[def:lindenbaum_tarski_algebra]{Lindenbaum-Tarski algebra} \( \BbbL_\Gamma \). \Fullref{thm:intuitionistic_propositional_completeness} then implies that \( \Gamma \vdash \varphi \).
\end{proof}

\paragraph{Paraconsistent logic}

\begin{definition}\label{def:paraconsistent_consequence_operator}\mcite{StanfordPlato:paraconsistent_logic}
  We say that the \hyperref[def:consequence_operator]{consequence operator} \( \vdash \) for \hyperref[def:propositional_formula]{propositional formulas} is \term{explosive} if \fullref{thm:propositional_semantic_efq} holds, i.e. if \( \synbot \vdash \varphi \) for an arbitrary formula \( \varphi \).

  If \( \vdash \) is not explosive, we say that it is \term{paraconsistent}.
\end{definition}
\begin{comments}
  \item We have chosen to use \fullref{thm:propositional_semantic_efq} for this definition because we find it simple to state. We could have instead used \fullref{thm:propositional_semantic_ecq} or even \fullref{thm:propositional_semantic_lnc}.

  \item Our definition is based on the following excerpt from \cite{StanfordPlato:paraconsistent_logic}:
  \begin{displayquote}
    A standard contemporary logical view has it that, from contradictory premises, anything follows. A logical consequence relation is \textit{explosive} if according to it any arbitrary conclusion \( B \) is entailed by any arbitrary contradiction \( A \), \( \neg A \) (\textit{ex contradictione quodlibet} (ECQ)). Classical logic, and most standard \enquote*{non-classical} logics too such as intuitionist logic, are explosive. Inconsistency, according to received wisdom, cannot be coherently reasoned about.

    Thus, if a consequence relation is paraconsistent, then even in circumstances where the available information is inconsistent, the consequence relation does not explode into \textit{triviality}. Thus, paraconsistent logic accommodates inconsistency in a controlled way that treats inconsistent information as potentially informative.
  \end{displayquote}
\end{comments}

\begin{proposition}\label{thm:intuitionistic_natural_deduction_is_explosive}
  \hyperref[def:propositional_natural_deduction]{Intuitionistic propositional natural deduction} is \hyperref[def:paraconsistent_consequence_operator]{explosive}.
\end{proposition}
\begin{proof}
  The rule \ref{inf:def:propositional_natural_deduction/bot/efq} can derive an arbitrary formula from \( \synbot \).
\end{proof}

\begin{proposition}\label{thm:minimal_natural_deduction_is_paraconsistent}
  \hyperref[def:propositional_natural_deduction]{Minimal propositional natural deduction} is \hyperref[def:paraconsistent_consequence_operator]{paraconsistent}.
\end{proposition}
\begin{proof}
  Without \ref{inf:def:propositional_natural_deduction/bot/efq} or \ref{inf:def:propositional_natural_deduction/bot/raa}, no rule allows introducing an arbitrary formula (e.g. a propositional variable) given only \( \synbot \).
\end{proof}

\begin{remark}\label{rem:minimal_propositional_semantics}
  Minimal logic was introduced by Ingebrigt Johansson as an attempt to further refine intuitionistic logic. An analysis of his works and his interactions with Gerhard Gentzen is given by \incite{VanDerMolen2016MinimalLogic}.

  We have presented a natural deduction system for minimal logic in \cref{def:propositional_natural_deduction} based on \cite[35]{TroelstraSchwichtenberg2000BasicProofTheory}. In \cref{thm:intuitionistic_natural_deduction_is_explosive}, we have shown that this system is \hyperref[def:paraconsistent_consequence_operator]{paraconsistent}. We can use the latter as a guide if we want to define semantics corresponding to this deduction system.

  We have defined propositional denotations in \fullref{alg:propositional_denotation} so that, for any \hyperref[def:propositional_interpretation]{propositional interpretation} in any \hyperref[def:truth_value_algebra]{truth value algebra} \( \BbbT \), the denotation \( \Bracks{\synbot}_I \) of \( \synbot \) is the \hyperref[def:extremal_points/top_and_bottom]{bottom element} \( \sembot \) of \( \BbbT \).

  Whatever formula \( \varphi \) we take, we will have \( \Bracks{\synbot}_I \leq \Bracks{\varphi}_I \), which via \cref{thm:intuitionistic_deduction_consequences/leq} implies that \( \synbot \vDash \varphi \). Thus, with the standard denotation, semantic entailment is explosive.

  Therefore, we must modify our denotation so that \( \synbot \) is not necessarily \( \sembot \). A simple way to do this is to extend the interpretation \( I \) to \( \synbot \), and let
  \begin{empheq}[left={\Bracks{\varphi}_I} \coloneqq \empheqlbrace]{align}
    &I(\synbot),                             &&\varphi = \synbot,      \label{eq:rem:minimal_propositional_semantics/bot} \\
    &\Bracks{\psi}_I \rightarrow I(\synbot), &&\varphi = \synneg \psi, \label{eq:rem:minimal_propositional_semantics/neg} \\
    &\vdots,                                 && \notag
  \end{empheq}
  the other cases being the same as in \fullref{alg:propositional_denotation}.

  It follows from \fullref{thm:propositional_natural_deduction_soundness} that this semantics is sound with respect to the minimal natural deduction system. It is also possible to generalize \fullref{thm:intuitionistic_propositional_completeness} so encompass minimal logic.

  Still, we find minimal semantics inconvenient, so we only sketch it briefly here.
\end{remark}
\begin{comments}
  \item For \hyperref[def:first_order_logic]{first-order logic}, we must instead require \hyperref[def:fol_structure]{first-order structures} to interpret \( \synbot \).
\end{comments}

\paragraph{Complete theories}

\begin{definition}\label{def:complete_set_of_sentences}
  Fix a \hyperref[def:consequence_operator]{consequence operator} \( \op{Cn} \) on a set of sentences \( S \). Let \( {\vdash} \) be the corresponding \hyperref[def:consequence_relation]{consequence relation}. We say that the set of sentences \( \Gamma \) is \term[ru=полное (множество формул) (\cite[def. 1.3.16]{Герасимов2014Вычислимость})]{complete} with respect to them if any of the following equivalent conditions hold:
  \begin{thmenum}
    \thmitem{def:complete_set_of_sentences/operator}\mcite[def. 4]{Tarski1983FundamentalConceptsOfMetamathematics} Every \hyperref[def:consistent_set_of_sentences]{consistent} superset of \( \Gamma \) is \hyperref[def:equivalent_sets_of_sentences]{equivalent} to \( \Gamma \).

    \thmitem{def:complete_set_of_sentences/relation}\mimprovised For any formula \( \varphi \) for which \( \Gamma \cup \set{ \varphi } \) is consistent, we have \( \Gamma \vdash \varphi \).
  \end{thmenum}
\end{definition}
\begin{comments}
  \item For \hyperref[con:classical_logic]{classical} \hyperref[def:propositional_logic]{propositional logic}, complete sets are characterized in \cref{thm:propositional_complete_set}.

  \item We have not required \( \Gamma \) to be consistent; in fact, all inconsistent sets are complete --- see \cref{thm:def:complete_set_of_sentences/inconsistent}.
\end{comments}
\begin{defproof}
  \ImplicationSubProof{def:complete_set_of_sentences/operator}{def:complete_set_of_sentences/relation} Suppose that every consistent superset of \( \Gamma \) is equivalent to \( \Gamma \). Fix a formula \( \varphi \) for which \( \Gamma \cup \set{ \varphi } \) is consistent.

  Since \( {\vdash} \) satisfies \eqref{def:consequence_relation/reflexivity}, it follows that \( \Gamma, \varphi \vdash \varphi \). By assumption, \( \Gamma \cup \set{ \varphi } \) is equivalent to \( \Gamma \), i.e. every formula derivable from \( \Gamma \cup \set{ \varphi } \) is also derivable from \( \Gamma \).

  Therefore, \( \Gamma \vdash \varphi \), as desired.

  \ImplicationSubProof{def:complete_set_of_sentences/relation}{def:complete_set_of_sentences/operator} Conversely, suppose that, whenever \( \Gamma \cup \set{ \varphi } \) is consistent, we have \( \Gamma \vdash \varphi \).

  Let \( \Delta \) is a consistent extension of \( \Gamma \). For every formula \( \varphi \) in \( \Delta \), by \cref{thm:def:consistent_set_of_sentences/subset}, \( \Gamma \cup \set{ \varphi } \) is consistent, and thus \( \Gamma \vdash \varphi \).

  Therefore, every formula of \( \Delta \) is derivable from \( \Gamma \); it follows that \( \Gamma \) and \( \Delta \) are equivalent.
\end{defproof}

\begin{proposition}\label{thm:def:complete_set_of_sentences}
  \hyperref[def:complete_set_of_sentences]{Complete sets of sentences} have the following basic properties:
  \begin{thmenum}
    \thmitem{thm:def:complete_set_of_sentences/inconsistent} An \hyperref[def:consistent_set_of_sentences]{inconsistent set} is vacuously complete.
    \thmitem{thm:def:complete_set_of_sentences/superset} Every superset of a complete set is complete.
    \thmitem{thm:def:complete_set_of_sentences/subset} Every subset of an incomplete set is incomplete.
  \end{thmenum}
\end{proposition}
\begin{proof}
  \SubProofOf{thm:def:complete_set_of_sentences/inconsistent} Fix an inconsistent set \( \Gamma \). \Cref{thm:def:consistent_set_of_sentences/superset} implies that, for every formula \( \varphi \), the set \( \Gamma \cup \set{ \varphi } \) is also inconsistent. Then \cref{def:complete_set_of_sentences/relation} holds vacuously.

  \SubProofOf{thm:def:complete_set_of_sentences/superset} Let \( \Gamma \) be a complete set and fix a superset \( \Delta \) of \( \Gamma \).

  Suppose that \( \Delta \cup \set{ \varphi } \) is consistent for some formula \( \varphi \). Then \( \Gamma \cup \set{ \varphi } \) is also consistent as a consequence of \cref{thm:def:consistent_set_of_sentences/subset}. Since \( \Gamma \) is complete, we have \( \Gamma \vdash \varphi \), and, since \( {\vdash} \) satisfies \cref{def:consequence_relation/monotonicity}, we also have \( \Delta \vdash \varphi \).

  Generalizing on \( \varphi \), we conclude that \( \Delta \) is complete.

  \SubProofOf{thm:def:complete_set_of_sentences/subset} Let \( \Gamma \) be an incomplete set and fix a subset \( \Delta \) of \( \Gamma \). If \( \Delta \) was complete, by \cref{thm:def:complete_set_of_sentences/superset}, \( \Gamma \) would be complete, which is a contradiction. It remains for \( \Delta \) to be incomplete.
\end{proof}

\begin{proposition}\label{thm:propositional_complete_set}
  With respect to the \hyperref[def:propositional_natural_deduction]{classical natural deduction system}, the set \( \Gamma \) of \hyperref[def:propositional_formula]{propositional sentences} is \hyperref[def:complete_set_of_sentences]{complete} if and only if, for any formula \( \varphi \), either \( \Gamma \vdash \varphi \) or \( \Gamma \vdash \synneg \varphi \) (or possibly both, if \( \Gamma \) is \hyperref[def:consistent_set_of_sentences]{inconsistent}).
\end{proposition}
\begin{comments}
  \item In intuitionistic logic, without double negation elimination, we can only conclude \( \Gamma \vdash \synneg \synneg \varphi \) from \( \Gamma \not\vdash \neg \varphi \) and \( \Gamma \not\vdash \varphi \).
\end{comments}
\begin{proof}
  \SufficiencySubProof Suppose that \( \Gamma \) is complete in the sense of \cref{def:complete_set_of_sentences/relation}, i.e. \( \Gamma \vdash \varphi \) whenever \( \Gamma \cup \set{ \varphi } \) is consistent. Fix an arbitrary formula \( \varphi \).

  \begin{itemize}
    \item Suppose that \( \Gamma \cup \set{ \varphi } \) is inconsistent. \Cref{thm:propositional_natural_deduction_inconsistency/bot} implies that there exists a proof \( P \) deriving \( \synbot \) from \( \Gamma \cup \set{ \varphi } \).

    \begin{itemize}
      \item If \( \varphi \) is an open assumption in \( P \) with marker \( u \), the following is a proof tree deriving \( \synneg \varphi \) from \( \Gamma \):
      \begin{equation*}
        \begin{prooftree}
          \hypo{ [\varphi]^u }
          \ellipsis { \( P \) } { \synbot }
          \infer1[\ref{inf:def:propositional_natural_deduction/neg/intro}]{ \synneg \varphi }.
        \end{prooftree}
      \end{equation*}

      \item Otherwise, \( P \) is a proof of \( \synbot \) from \( \Gamma \), and \cref{thm:propositional_natural_deduction_inconsistency/bot} implies that \( \Gamma \) is inconsistent. \Cref{thm:propositional_natural_deduction_inconsistency/contradiction} then implies that both \( \varphi \) and \( \synneg \varphi \) are derivable from \( \Gamma \).
    \end{itemize}

    \item Otherwise, \( \Gamma \cup \set{ \varphi } \) is consistent, and, since \( \Gamma \) is complete, \( \Gamma \vdash \varphi \).
  \end{itemize}

  \NecessitySubProof Suppose that, for any formula \( \varphi \), either \( \Gamma \vdash \varphi \) or \( \Gamma \vdash \synneg \varphi \) or both.

  Fix a formula \( \varphi \) for which \( \Gamma \cup \set{ \varphi } \) is consistent. \Cref{thm:def:consistent_set_of_sentences/subset} then implies that \( \Gamma \) is consistent.

  Clearly \( \Gamma, \varphi \vdash \varphi \). Aiming at a contradiction, suppose that \( \Gamma \not\vdash \varphi \). Then, by assumption, we have \( \Gamma \vdash \synneg \varphi \), and, after adding \( \varphi \) as a hypothesis, we have \( \Gamma, \varphi \vdash \synneg \varphi \). Then \( \Gamma \cup \set{ \varphi } \) satisfies \cref{thm:propositional_natural_deduction_inconsistency/bot} and is thus inconsistent, contradicting our assumption.

  The obtained contradiction implies that \( \Gamma \vdash \varphi \).

  Generalizing on \( \varphi \), we conclude that \( \Gamma \) satisfies \cref{def:complete_set_of_sentences/relation} and is thus complete.
\end{proof}

\begin{proposition}\label{thm:lindenbaum_tarski_theories}
  With respect to the \hyperref[def:propositional_natural_deduction]{minimal natural deduction system}, consider the \hyperref[def:lindenbaum_tarski_algebra]{Lindenbaum-Tarski algebra} \( \BbbL \) of an empty set of premises.

  \begin{thmenum}
    \thmitem{thm:lindenbaum_tarski_theories/filter} The set \( \Delta \) of propositional sentences is a \hyperref[def:propositional_theory]{theory} (i.e. is closed under logical consequence) if and only if the projection \( \pi[\Delta] \) is a \hyperref[def:lattice_ideal]{lattice filter} in \( \BbbL \).

    \thmitem{thm:lindenbaum_tarski_theories/consistent} The theory \( \Delta \) is \hyperref[def:consistent_set_of_sentences]{inconsistent} if and only if \( \pi[\Delta] = \BbbL \).

    \thmitem{thm:lindenbaum_tarski_theories/complete} In classical logic, the consistent theory \( \Delta \) is \hyperref[def:complete_set_of_sentences]{complete} if and only if \( \pi[\Delta] \) is an \hyperref[def:ultrafilter]{ultrafilter}.
  \end{thmenum}
\end{proposition}
\begin{proof}
  \SubProofOf{thm:lindenbaum_tarski_theories/filter} We will use the characterization \cref{thm:def:lattice_ideal/directed_and_closed} of lattice filters.

  \SufficiencySubProof* Suppose that \( \Delta \) is a propositional theory.

  The family \( \pi[\Delta] \) is clearly \hyperref[def:closed_ordered_subset]{upward closed}. Indeed, if \( \varphi \) is in \( \Delta \) and if \( \varphi \vdash \psi \), since \( \Delta \) is a theory, \( \psi \) also belongs to \( \Delta \). Thus, if \( \pi(\varphi) \leq \pi(\psi) \) and \( \pi(\varphi) \) belongs to \( \pi[\Delta] \), so does \( \pi(\psi) \).

  We must also show that \( \pi[\Delta] \) is \hyperref[def:directed_set]{downward directed}. Fix two formulas \( \varphi \) and \( \psi \) in \( \Delta \). Then \( \varphi \synwedge \psi \) is also in \( \Delta \). \Cref{thm:lindenbaum_tarski_algebras/minimal/conjunction} implies that the coset \( \pi(\varphi \synwedge \psi) \) is the infimum of \( \pi(\varphi) \) and \( \pi(\psi) \). Then every two elements of \( \pi[\Delta] \) have a lower bound, i.e. \( \pi[\Delta] \) is downward directed.

  Therefore, \( \pi[\Delta] \) is a lattice filter.

  \NecessitySubProof* Suppose that \( \pi[\Delta] \) is a lattice filter. Fix some sentence \( \varphi \) such that \( \Delta \vdash \varphi \).

  By the compactness shown in \cref{thm:propositional_natural_deduction_derivation_compact}, there exists a finite subset \( \Delta_0 \) such that \( \Delta_0 \vdash \varphi \). Let \( \delta \) be a conjunction of all formulas in \( \Delta_0 \) (in some order). \Cref{thm:syntactic_propositional_conjunction_of_premises} implies that \( \delta \vdash \varphi \).

  By \cref{thm:lindenbaum_tarski_algebras/minimal/conjunction}, \( \delta \) is a lower bound for the formulas from \( \Delta_0 \); the latter belong to \( \Delta \), hence, by \eqref{eq:def:lattice_ideal/filter_meet}, \( \pi(\delta) \) belongs to \( \pi[\Delta] \).

  Now, since \( \pi[\Delta] \) is upward closed, it follows that \( \pi(\varphi) \) also belongs to \( \pi[\Delta] \).

  Generalizing on \( \varphi \), we conclude that \( \Delta \) is closed under logical consequence.

  \SubProofOf{thm:lindenbaum_tarski_theories/consistent}

  \SufficiencySubProof* Suppose that \( \Delta \) is inconsistent. Then it can derive any propositional sentence, and, since it is a theory, it also contains every propositional sentence. Naturally, \( \pi[\Delta] = \BbbL \).

  \NecessitySubProof* Suppose that \( \pi[\Delta] = \BbbL \). Fix any formula \( \varphi \). The projection \( \pi_\Gamma(\varphi) \) is in \( \pi[\Delta] \), implying that \( \Delta \vdash \varphi \).

  Generalizing on \( \varphi \), we conclude that \( \Delta \) is inconsistent.

  \SubProofOf{thm:lindenbaum_tarski_theories/complete} \Cref{thm:propositional_complete_set} corresponds precisely to \cref{def:ultrafilter/direct}.
\end{proof}

\paragraph{Propositional completeness}

\begin{lemma}[Lindenbaum's lemma]\label{thm:extension_to_complete_consistent_set}
  For the \hyperref[def:propositional_natural_deduction_system]{classical propositional natural deduction system}, every \hyperref[def:consistent_set_of_sentences]{consistent} set of sentences can be extended to a \hyperref[def:complete_set_of_sentences]{complete} consistent set.
\end{lemma}
\begin{comments}
  \item The name of the lemma is taken from \incite[lemma 1.3.21]{Герасимов2014Вычислимость}, but the proof is simplified.
\end{comments}
\begin{proof}
  Denote by \( \BbbL \) the \hyperref[thm:lindenbaum_tarski_algebras]{classical Lindenbaum-Tarski algebra} with no premises.

  Let \( \Gamma \) be a consistent set of formulas. Consider the \hyperref[def:propositional_theory]{theory} \( \op*{Th}(\Gamma) \) axiomatized by \( \Gamma \). \Cref{thm:lindenbaum_tarski_theories} implies that \( \pi[\op*{Th}(\Gamma)] \) is a proper filter of \( \BbbL \). \Cref{thm:ultrafilter_lemma} implies that there exists an ultrafilter \( M \) containing \( \pi[\op*{Th}(\Gamma)] \).

  Let \( \Epsilon \) be the union of \( M \), that is, the set of all formulas \( \varphi \) for which \( \pi_\Gamma(\varphi) \) is in \( M \). We have \( \pi[\Epsilon] = M \). \Cref{thm:lindenbaum_tarski_theories/filter} implies that \( \Epsilon \) is a theory.

  Since \( M \) is a proper filter, \cref{thm:lindenbaum_tarski_theories/consistent} implies that \( \Epsilon \) is consistent. Since it is an ultrafilter, \cref{thm:lindenbaum_tarski_theories/complete} implies that \( \Epsilon \) is complete.
\end{proof}

\begin{proposition}\label{thm:propositional_consistent_implies_satisfiable}\mcite[thm. 1.3.19]{Герасимов2014Вычислимость}
  With respect to the \hyperref[def:propositional_natural_deduction_system]{classical propositional natural deduction system}, for every (syntactically) \hyperref[def:complete_set_of_sentences]{complete} \hyperref[def:consistent_set_of_sentences]{consistent} set of formulas \( \Gamma \), there exists a \hyperref[def:truth_value_algebra/classical]{classical} \hyperref[def:propositional_interpretation]{interpretation} \( I \) such that \( \Gamma \vdash \varphi \) if and only if \( I \vDash \varphi \).
\end{proposition}
\begin{proof}
  Define the interpretation \( I(p) \) as \( \semtop \) if \( \Gamma \vdash p \) and \( \sembot \) otherwise.

  We use \fullref{thm:induction_on_abstract_syntax} on \( \varphi \) to prove that \( \Gamma \vdash \varphi \) if and only if \( I \vDash \varphi \).

  \SubProof{Proof when \( \varphi = \syntop \)} We have \( \Bracks{\syntop}_I = \semtop \) by definition, and the rule \ref{inf:def:propositional_natural_deduction/top/intro} allows us to derive \( \syntop \) from any set of premises.

  In particular, \( \Gamma \vdash \syntop \) if and only if \( I \vDash \syntop \).

  \SubProof{Proof when \( \varphi = \synbot \)} In this case the statement is vacuous since both assumptions \( \Gamma \vdash \syntop \) and \( I \vDash \syntop \) lead to contradictions, i.e. are canonically false.

  \SufficiencySubProof* The assumption \( \Gamma \vdash \synbot \) leads to a contradiction because by \cref{thm:propositional_natural_deduction_inconsistency/bot}, \( \Gamma \) must be inconsistent. But we have assumed that \( \Gamma \) is consistent.

  \SufficiencySubProof* The assumption \( \Bracks{\sembot}_I = \semtop \) also leads to a contradiction, hence no interpretation satisfies \( \varphi = \synbot \).

  \SubProof{Proof when \( \varphi = p \in \op*{Prop} \)} By definition of \( I \), we have \( \Gamma \vdash p \) if and only if \( \Bracks{p}_I = I(p) = \semtop \).

  \SubProof{Proof when \( \varphi = \synneg \psi \)} Suppose that the inductive hypothesis holds for \( \psi \).

  \SufficiencySubProof* If \( \Gamma \vdash \synneg \psi \), the sentence \( \psi \) is not derivable from \( \Gamma \) because the latter is consistent. By the inductive hypothesis, \( \Bracks{\psi}_I = \sembot \), and
  \begin{equation*}
    \Bracks{\synneg \psi}_I = \oline{\Bracks{\psi}} = \semtop.
  \end{equation*}

  \NecessitySubProof* Conversely, if \( \Bracks{\synneg \psi}_I = \semtop \), then \( \Bracks{\psi}_I = \sembot \) and, by the inductive hypothesis, \( \Gamma \not\vdash \psi \). Since \( \Gamma \) is complete, it follows that \( \Gamma \vdash \synneg \psi \).

  \SubProof{Proof when \( \varphi = \psi \synwedge \theta \)} Suppose that the inductive hypothesis holds for both \( \psi \) and \( \theta \).

  \SufficiencySubProof* Suppose that \( \Gamma \vdash (\psi \synwedge \theta) \). The rule \ref{inf:def:propositional_natural_deduction/and/elim_left} allows us to extend a proof of \( \psi \synwedge \theta \) from \( \Gamma \) to a proof of \( \psi \) from \( \Gamma \). We can analogously build a proof of \( \theta \).

  By the inductive hypothesis, we have \( \Bracks{\psi}_I = \Bracks{\theta}_I = \semtop \), hence
  \begin{equation*}
    \Bracks{\psi \synwedge \theta}_I
    =
    \semtop \wedge \semtop
    =
    \semtop
  \end{equation*}

  \NecessitySubProof* Conversely, if \( \Bracks{\psi \synwedge \theta}_I = \semtop \), then \( \Bracks{\psi}_I = \Bracks{\theta}_I = \semtop \) and, by the inductive hypothesis, there exist proof trees deriving \( \psi \) and \( \theta \) from \( \Gamma \). We can combine them via \ref{inf:def:propositional_natural_deduction/and/intro}.

  \SubProof{Proof when \( \varphi = \psi \synvee \theta \)} Suppose that the inductive hypothesis holds for both \( \psi \) and \( \theta \).

  \SufficiencySubProof* Suppose that \( \Gamma \vdash (\psi \synvee \theta) \).

  Aiming at a contradiction, suppose that \( \Bracks{\psi \synvee \theta}_I = \sembot \), i.e. \( \Bracks{\psi}_I = \Bracks{\theta}_I = \sembot \). Then \( \Bracks{\synneg \psi}_I = \Bracks{\synneg \theta}_I = \semtop \) and, by the inductive hypothesis, we conclude that \( \Gamma \vdash \synneg \psi \) and \( \Gamma \vdash \synneg \psi \), i.e. there exists a proof tree \( P \) deriving \( \synneg \psi \) from \( \Gamma \) and a tree \( R \) deriving \( \synneg \theta \) from \( \Gamma \).

  Consider the derivation
  \begin{equation*}
    \begin{prooftree}
      \hypo{ [\psi \synvee \theta]^u }

      \hypo{ \Gamma }
      \ellipsis {\( P \)} { \synneg \psi }
      \hypo{ [\psi]^v }
      \infer2[\ref{inf:def:propositional_natural_deduction/neg/elim}]{ \synbot }

      \hypo{ \Gamma }
      \ellipsis {\( R \)} { \synneg \theta }
      \hypo{ [\theta]^w }
      \infer2[\ref{inf:def:propositional_natural_deduction/neg/elim}]{ \synbot }

      \infer[left label={\( v, w \)}]3[\ref{inf:def:propositional_natural_deduction/or/elim}]{ \synbot }
      \infer[left label=\( u \)]1[\ref{inf:def:propositional_natural_deduction/neg/intro}]{ \synneg (\psi \synvee \theta) }
    \end{prooftree}
  \end{equation*}

  We have \( \Gamma \vdash \synneg (\psi \synvee \theta) \), which contradicts the consistency of \( \Gamma \) since \( \Gamma \vdash \psi \synvee \theta \).

  It remains for \( \Bracks{\psi \synvee \theta}_I \) to be \( \semtop \).

  \NecessitySubProof* Suppose that \( \Bracks{\psi \synvee \theta}_I = \semtop \). \Cref{thm:propositional_satisfaction_characterization/disjunction} implies that \( \Bracks{\psi}_I = \semtop \) or \( \Bracks{\theta}_I = \semtop \).

  If \( \Bracks{\psi}_I = \semtop \), the inductive hypothesis implies that \( \Gamma \vdash \psi \), and we can use \ref{inf:def:propositional_natural_deduction/or/intro_left} to conclude that \( \Gamma \vdash (\psi \synvee \theta) \).

  Otherwise, we have \( \Bracks{\theta}_I = \semtop \), and we proceed similarly.

  \SubProof{Proof when \( \varphi = \psi \synimplies \theta \)} Suppose that the inductive hypothesis holds for both \( \psi \) and \( \theta \).

  \SufficiencySubProof* Suppose that \( \Gamma \vdash (\psi \synimplies \theta) \).

  We will utilize \cref{thm:propositional_satisfaction_characterization/conditional}, by which it is sufficient to show that \( \Bracks{\psi}_I = \semtop \) implies \( \Bracks{\theta}_I = \semtop \) in order to deduce \( \Bracks{\psi \synimplies \theta}_I = \semtop \).

  By the inductive hypothesis on \( \psi \), if \( \Bracks{\psi}_I = \semtop \), we have \( \Gamma \vdash \psi \). Since \( \Gamma \) is a theory, \( \psi \) belongs to \( \Gamma \).

  \Cref{thm:propositional_syntactic_deduction_theorem} implies that \( \Gamma, \psi \vdash \theta \), i.e. \( \Gamma \vdash \theta \). By the inductive hypothesis on \( \theta \), we have \( \Bracks{\theta}_I = \semtop \).

  It follows from \cref{thm:propositional_satisfaction_characterization/conditional} that \( \Bracks{\psi \synimplies \theta}_I = \semtop \).

  \NecessitySubProof* Conversely, suppose that \( \Bracks{\psi \synimplies \theta}_I = \semtop \).

  We have two possibilities:
  \begin{itemize}
    \item If \( \Gamma \vdash \psi \), the inductive hypothesis on \( \psi \) implies that \( \Bracks{\psi}_I = \semtop \).

    \Cref{thm:propositional_satisfaction_characterization/conditional} implies that \( \Bracks{\theta}_I = \semtop \), and the inductive hypothesis on \( \theta \) implies that \( \Gamma \vdash \theta \).

    The rule \ref{inf:def:propositional_natural_deduction/imp/intro} then allows us to construct a proof tree deriving \( \psi \synimplies \theta \) from \( \Gamma \).

    \item If \( \Gamma \nvdash \psi \), since \( \Gamma \) is complete, \cref{thm:propositional_complete_set} implies that \( \Gamma \vdash \synneg \psi \).

    Let \( P \) be a proof tree deriving \( \synneg \psi \) from \( \Gamma \). We can construct the following proof tree:
    \begin{equation*}
      \begin{prooftree}
        \hypo{ \Gamma }
        \ellipsis {\( P \)} { \synneg \psi }
        \infer1[\ref{inf:def:propositional_natural_deduction/imp/intro}]{ \synneg \theta \synimplies \synneg \psi }
        \infer1[\ref{thm:propositional_admissible_rules/contraposition_elim}]{ \psi \synimplies \theta }
      \end{prooftree}
    \end{equation*}
  \end{itemize}

  \SubProof{Proof when \( \varphi = \psi \syniff \theta \)} Suppose that the inductive hypothesis holds for both \( \psi \) and \( \theta \).

  \SufficiencySubProof* Suppose that \( \Gamma \vdash (\psi \syniff \theta) \).

  Due to \cref{thm:propositional_satisfaction_characterization/biconditional}, it is sufficient to show that \( \Bracks{\psi}_I = \semtop \) if and only if \( \Bracks{\theta}_I = \semtop \) in order to deduce \( \Bracks{\psi \syniff \theta}_I = \semtop \).

  If \( \Bracks{\psi} = \semtop \), the inductive hypothesis on \( \psi \) implies that \( \Gamma \vdash \psi \). Then \ref{inf:def:propositional_natural_deduction/iff/elim_right} allows deriving \( \theta \) from \( \Gamma \), which by the inductive hypothesis on \( \theta \) implies that \( \Bracks{\theta} = \semtop \).

  In the other direction, if \( \Bracks{\theta} = \semtop \), we simply use \ref{inf:def:propositional_natural_deduction/iff/elim_left} instead.

  \NecessitySubProof* Suppose that \( \Bracks{\psi \syniff \theta}_I = \semtop \). \Cref{thm:propositional_satisfaction_characterization/biconditional} implies that \( \Bracks{\psi}_I = \semtop \) if and only if \( \Bracks{\theta}_I = \semtop \).

  \begin{itemize}
    \item If \( \Bracks{\psi}_I = \Bracks{\theta}_I = \semtop \), by the inductive hypothesis, we conclude that \( \Gamma \vdash \psi \) and \( \Gamma \vdash \theta \). The rule \ref{inf:def:propositional_natural_deduction/iff/intro} allows constructing a proof tree deriving \( \psi \syniff \theta \) from \( \Gamma \).

    \item If \( \Bracks{\psi}_I = \Bracks{\theta}_I = \sembot \), by the inductive hypothesis, we conclude that \( \Gamma \nvdash \psi \) and \( \Gamma \nvdash \theta \).

    Since \( \Gamma \) is complete, \cref{thm:propositional_complete_set} implies that \( \Gamma \vdash \synneg \psi \) and \( \Gamma \vdash \synneg \theta \). Let \( P \) be a proof tree deriving \( \psi \) from \( \Gamma \), and let \( R \) be a tree deriving \( \theta \).

    We can construct the following:
    \begin{equation*}
      \begin{prooftree}
        \hypo{ \Gamma }
        \ellipsis {\( R \)} { \synneg \theta }

        \hypo{ [\theta]^u }
        \infer2[\ref{inf:def:propositional_natural_deduction/neg/elim}]{ \synbot }
        \infer1[\ref{inf:def:propositional_natural_deduction/bot/efq}]{ \psi }

        \hypo{ \Gamma }
        \ellipsis {\( P \)} { \synneg \psi }

        \hypo{ [\psi]^v }
        \infer2[\ref{inf:def:propositional_natural_deduction/neg/elim}]{ \synbot }
        \infer1[\ref{inf:def:propositional_natural_deduction/bot/efq}]{ \theta }

        \infer[left label={\( u, v \)}]2[\ref{inf:def:propositional_natural_deduction/iff/intro}]{ \psi \syniff \theta }
      \end{prooftree}
    \end{equation*}
  \end{itemize}
\end{proof}

\begin{corollary}\label{thm:consistent_implies_satisfiable}
  If a set of sentences is \hyperref[def:consistent_set_of_sentences]{consistent} with respect to the \hyperref[def:propositional_natural_deduction_system]{classical propositional natural deduction system}, it is \hyperref[def:propositional_semantics/satisfaction]{satisfiable} with respect to \hyperref[def:truth_value_algebra/classical]{classical semantics}.
\end{corollary}
\begin{proof}
  Let \( \Gamma \) be a \hyperref[def:consistent_set_of_sentences]{consistent} set of formulas. Denote by \( \Delta \) the \hyperref[def:complete_set_of_sentences]{complete} consistent extension of the \hyperref[def:propositional_theory]{theory} \( \op*{Th}(\Gamma) \) given by \cref{thm:extension_to_complete_consistent_set}.

  \Cref{thm:propositional_consistent_implies_satisfiable} implies that there exists an interpretation \( I \) such that \( \Delta \vdash \varphi \) if and only if \( I \vDash \varphi \).

  Then \( I \) is a model of \( \Delta \), and hence also of \( \Gamma \).
\end{proof}

\begin{theorem}[Classical propositional completeness]\label{thm:classical_propositional_completeness}\mcite[thm. 1.3.24]{Герасимов2014Вычислимость}
  The \hyperref[def:propositional_natural_deduction]{classical propositional natural deduction system} is \hyperref[def:general_logic/completeness]{complete} with respect to \hyperref[def:truth_value_algebra/classical]{classical semantics}.
\end{theorem}
\begin{comments}
  \item Incidentally, due to \cref{thm:propositional_sequent_calculus_and_natural_deduction}, \hyperref[def:classical_propositional_sequent_calculus]{classical propositional sequent calculus} is also complete.
\end{comments}
\begin{proof}
  Suppose that \( \Gamma \vDash \varphi \). Then the set \( \Gamma \cup \set{ \neg \varphi } \) is \hyperref[def:propositional_semantics/satisfaction]{unsatisfiable} because \( \Bracks{\neg \varphi}_I = \sembot \) for every model \( I \) of \( \Gamma \).

  The set \( \Gamma \cup \set{ \neg \varphi } \) must be \hyperref[def:consistent_set_of_sentences]{inconsistent} since otherwise \cref{thm:consistent_implies_satisfiable} would imply that it is satisfiable.

  Since it is inconsistent, there exists a proof tree deriving \( \bot \) from \( \Gamma \cup \set{ \neg \varphi } \). Then we can apply \ref{inf:def:propositional_natural_deduction/bot/raa} and obtain a proof of \( \varphi \) from \( \Gamma \).
\end{proof}

\paragraph{Propositional compactness}

\begin{definition}\label{def:finitely_consistent_set_of_sentences}\mimprovised
  For a given \hyperref[def:consequence_relation]{consequence relation} (or \hyperref[def:consequence_operator]{operator}), we say that the set \( \Gamma \) of sentences is \term[en=finitely consistent (set) (\cite[def. 1.4.21]{Hinman2005Logic})]{finitely consistent} if every finite subset of \( \Gamma \) is \hyperref[def:consistent_set_of_sentences]{consistent}.
\end{definition}
\begin{comments}
  \item We generalize Hinman's definition for propositional entailment from \cite[def. 1.4.21]{Hinman2005Logic}.
\end{comments}

\begin{definition}\label{def:finitely_satisfiable_set_of_sentences}\mimprovised
  For a given \hyperref[def:institutional_satisfaction]{institutional satisfaction relation}, we say that the set \( \Gamma \) of sentences is \term[en=finitely satisfiable (set) (\cite[117]{CitkinMuravitsky2022ConsequenceRelations})]{finitely satisfiable} if every finite subset of \( \Gamma \) is satisfiable.
\end{definition}
\begin{comments}
  \item We generalize Citkin and Muravitsky's definition for propositional entailment from \cite[def. 1.4.21]{Hinman2005Logic}.
\end{comments}

\begin{theorem}[Classical propositional compactness]\label{thm:classical_propositional_semantic_compactness}
  All of the following equivalent statements hold for the \hyperref[def:truth_value_algebra/classical]{classical} \hyperref[def:propositional_semantics/entailment]{propositional entailment relation}:
  \begin{thmenum}
    \thmitem{thm:classical_propositional_semantic_compactness/relation} The relation is \hyperref[def:consequence_relation/compactness]{compact}: if \( \Gamma \vDash \varphi \), there exists a finite subset \( \Gamma_0 \) of \( \Gamma \) such that \( \Gamma_0 \vDash \varphi \).

    \thmitem{thm:classical_propositional_semantic_compactness/satisfiable} A set of sentences is \hyperref[def:propositional_semantics/model]{satisfiable} (has a model) if and only if it is \hyperref[def:finitely_satisfiable_set_of_sentences]{finitely satisfiable} (every finite subset has a model).

    \thmitem{thm:classical_propositional_semantic_compactness/consistent} A set of sentences is \hyperref[def:consistent_set_of_sentences]{consistent} (does not entail \( \synbot \)) if and only if it is \hyperref[def:finitely_consistent_set_of_sentences]{finitely consistent} (no finite subset entails \( \synbot \)).
  \end{thmenum}
\end{theorem}
\begin{comments}
  \item \Cref{thm:propositional_semantic_inconsistency} gives several equivalent conditions for a set of propositional sentences to be semantically consistent. Satisfiability is among them.

  \item \incite*[47]{Hinman2005Logic} calls \cref{thm:classical_propositional_semantic_compactness/relation} the \enquote{propositional compactness theorem} and \cref{thm:classical_propositional_semantic_compactness/satisfiable} the \enquote{alternative propositional compactness theorem}.

  The latter is called \enquote{the} compactness theorem by \cite[exerc. 1.5.8]{VanDalen2004LogicAndStructure}, \cite[thm 1.3.23]{Герасимов2014Вычислимость} and \cite[thm. 21]{ШеньВерещагин2017ЯзыкиИИсчисления}.

  The third formulation \cref{thm:classical_propositional_semantic_compactness/consistent} is there to highlight that, in our formulation, consistency and satisfiability are distinct (although equivalent) concepts.

  \item This is one of several compactness theorems presented here --- see \cref{rem:logical_compactness_theorems}.
\end{comments}
\begin{proof}
  \SubProofOf{thm:classical_propositional_semantic_compactness/relation} Follows from \fullref{thm:classical_propositional_completeness} via \cref{thm:completeness_implies_compactness}.

  \ImplicationSubProof{thm:classical_propositional_semantic_compactness/relation}{thm:classical_propositional_semantic_compactness/consistent} Suppose that \( {\vDash} \) is compact and fix a set of sentences \( \Gamma \).

  \SufficiencySubProof* Suppose that \( \Gamma \) is consistent. \Cref{thm:def:consistent_set_of_sentences/subset} implies that any subset of \( \Gamma \) is also consistent. Hence, \( \Gamma \) is finitely consistent.

  \NecessitySubProof* Suppose that \( \Gamma \) is finitely consistent.

  Aiming at a contradiction, suppose that \( \Gamma \) is inconsistent. \Cref{thm:propositional_natural_deduction_inconsistency/bot} implies that \( \Gamma \vDash \synbot \). Then, by compactness, there exists a finite subset \( \Gamma_0 \) such that \( \Gamma_0 \vDash \synbot \). Again by \cref{thm:propositional_natural_deduction_inconsistency/bot}, \( \Gamma_0 \) is inconsistent, contradicting the assumptions that \( \Gamma \) is finitely consistent.

  It remains for \( \Gamma \) to be consistent.

  \EquivalenceSubProof{thm:classical_propositional_semantic_compactness/consistent}{thm:classical_propositional_semantic_compactness/satisfiable} \Cref{thm:propositional_semantic_inconsistency} provides equivalence between semantic consistency and satisfiability.

  \EquivalenceSubProof{thm:classical_propositional_semantic_compactness/satisfiable}{thm:classical_propositional_semantic_compactness/relation} Suppose that every satisfiable set is finitely satisfiable. Suppose also that \( \Gamma \vDash \varphi \).

  If \( I \vDash \Gamma \), then \( I \vDash \varphi \), i.e. \( \Bracks{\varphi}_I = \semtop \), hence \( \Bracks{\synneg \varphi}_I = \sembot \).

  We have shown that \( \Gamma \cup \set{ \synneg \varphi } \) is unsatisfiable. Then there must exist a finite subset \( \Gamma_0 \) of \( \Gamma \) such that \( \Gamma_0 \cup \set{ \synneg \varphi } \) is also unsatisfiable.

  It is possible that \( \Gamma_0 \) is satisfiable, in which case it has a model, say \( I \). In this case \( \Bracks{\synneg \varphi}_I = \semtop \), hence \( \Bracks{\varphi}_I = \semtop \). Thus, \( \Gamma_0 \vDash \varphi \).
\end{proof}
