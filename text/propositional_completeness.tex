\subsection{Propositional completeness}\label{subsec:propositional_completeness}

\paragraph{Consistent theories}

\begin{definition}\label{def:consistent_set_of_sentences}\mcite[def. III.3]{Woodger1983Tarski}
  Fix a \hyperref[def:consequence_operator]{consequence operator} \( \op{Cn} \) on a set of sentences \( S \).

  We say that the set of sentences \( \Gamma \) is \term[ru=противоречивое (множество формул) (\cite[def. 1.3.15]{Герасимов2011})]{inconsistent} with respect to this operator if \( \op{Cn}(\Gamma) \) coincides with \( S \). When \( \Gamma \) is not inconsistent, we say that it is \term{consistent}.
\end{definition}
\begin{comments}
  \item We choose inconsistency as the base notion in order to avoid relying on double negation elimination in our proofs.
  \item Within a given \hyperref[def:logical_framework]{logical framework}, we will by default refer to syntactic consistency, as a counterpart to \hyperref[def:satisfiable_set_of_sentences]{satisfiable theories}.
\end{comments}

\begin{proposition}\label{thm:def:consistent_set_of_sentences}
  \hyperref[def:consistent_set_of_sentences]{Consistent sets of sentences} have the following basic properties:
  \begin{thmenum}
    \thmitem{thm:def:consistent_set_of_sentences/subset} Every subset of a consistent set is consistent.
    \thmitem{thm:def:consistent_set_of_sentences/superset} Every superset of an inconsistent set is inconsistent.
    \thmitem{thm:def:consistent_set_of_sentences/entailment} If \( \Gamma \vdash \varphi \), then \( \Gamma \) is consistent if and only if \( \Gamma \cup \set{ \varphi } \) is.
  \end{thmenum}
\end{proposition}
\begin{proof}
  \SubProofOf{thm:def:consistent_set_of_sentences/subset} Straightforward since consequence operators preserve order.

  \SubProofOf{thm:def:consistent_set_of_sentences/superset} Straightforward since consequence operators preserve order.

  \SubProofOf{thm:def:consistent_set_of_sentences/entailment} Let \( \Gamma \vdash \varphi \). Then the theory axiomatized by \( \Gamma \cup \set{ \varphi } \) coincides with that of \( \Gamma \), and thus the two are equivalent. Then either both are inconsistent or consistent.
\end{proof}

\begin{proposition}\label{thm:propositional_consistent_set}
  Within \hyperref[def:intuitionistic_logic]{intuitionistic logic}, the following conditions are equivalent for a set \( \Gamma \) of \hyperref[def:propositional_syntax/formula]{propositional formulas} to be \hyperref[def:consistent_set_of_sentences]{inconsistent}:
  \begin{thmenum}
    \thmitem{thm:propositional_consistent_set/bot} The falsum \( \synbot \) is derivable from \( \Gamma \).

    \thmitem{thm:propositional_consistent_set/contradiction} For every formula \( \varphi \), both \( \varphi \) and \( \neg \varphi \) are derivable from \( \Gamma \).
  \end{thmenum}
\end{proposition}
\begin{proof}
  \SubProofOf{thm:propositional_consistent_set/bot}

  \SufficiencySubProof* If \( \Gamma \) is inconsistent in the sense of \fullref{def:consistent_set_of_sentences}, then, as a propositional formula, \( \synbot \) is derivable from \( \Gamma \).

  \NecessitySubProof* Let \( P \) be a \hyperref[def:natural_deduction_proof_tree]{proof tree} of \( \synbot \) from \( \Gamma \). Then, for any formula \( \varphi \), the following is a proof of \( \varphi \) from \( \Gamma \):
  \begin{equation*}
    \begin{prooftree}
      \hypo{}
      \ellipsis { \( P \) } { \synbot }
      \infer1[\eqref{eq:def:propositional_natural_deduction_systems/bot/efq}]{ \varphi }.
    \end{prooftree}
  \end{equation*}

  \SubProofOf{thm:propositional_consistent_set/contradiction}

  \SufficiencySubProof* If \( \Gamma \) is inconsistent in the sense of \fullref{def:consistent_set_of_sentences}, then, for any formula \( \varphi \), both \( \varphi \) and \( \neg \varphi \) are derivable from \( \Gamma \).

  \NecessitySubProof* Let \( P \) and \( N \) be proof trees of \( \varphi \) and \( \neg \varphi \) from \( \Gamma \), respectively. Let \( \psi \) be any formula. Then the following is a proof tree of \( \psi \) from \( \Gamma \):
  \begin{equation*}
    \begin{prooftree}
      \hypo{}
      \ellipsis { \( N \) } { \neg \varphi }

      \hypo{}
      \ellipsis { \( P \) } { \varphi }

      \infer2[\eqref{eq:def:propositional_natural_deduction_systems/neg/elim}]{ \bot }.

      \infer1[\eqref{eq:def:propositional_natural_deduction_systems/bot/efq}]{ \psi }.
    \end{prooftree}
  \end{equation*}
\end{proof}

\paragraph{Lindenbaum-Tarski algebras}

\begin{definition}\label{def:lindenbaum_tarski_algebra}\mcite[27]{Эдельман1975}
  Fix a \hyperref[def:consequence_relation]{consequence relation} \( {\vdash} \) on a set of sentences \( S \).

  For a set \( \Gamma \) of sentences, consider the \hyperref[def:binary_relation]{binary relation}
  \begin{equation*}
    \varphi \leq_\Gamma \psi \T{if} \Gamma, \varphi \vdash \psi.
  \end{equation*}

  Then \( S \) is a \hyperref[def:preordered_set]{preordered set} with respect to \( \leq_\Gamma \) due to \eqref{eq:def:consequence_relation/reflexivity} and \eqref{eq:def:consequence_relation/transitivity}. The corresponding \hyperref[def:antisymmetric_quotient]{antisymmetric quotient} \( Q \) is then a \hyperref[def:partially_ordered_set]{partially ordered set}, which we call the \term[ru=алгебра Линденбаума-Тарского, en=Lindenbaum-Tarski algebra (\cite[def. 6.3.1]{CitkinMuravitsky2021})]{Lindenbaum-Tarski algebra} of \( \Gamma \).
\end{definition}
\begin{comments}
  \item We will find useful the projection \( \pi_\Gamma: S \to Q \).
\end{comments}

\begin{proposition}\label{thm:lindenbaum_tarski_algebra_soundness}
  Let \( L \) be the \hyperref[def:lindenbaum_tarski_algebra]{Lindenbaum-Tarski algebra} of \( \Gamma \) for the \hyperref[def:propositional_natural_deduction_systems]{minimal natural deduction system}. Finally, fix some \hyperref[def:truth_value_algebra]{truth value Heyting algebra} and let \( I \) be an interpretation satisfying \( \Gamma \).

  Then \( \pi_\Gamma(\varphi) = \pi_\Gamma(\psi) \) implies \( \Bracks{\varphi}_I = \Bracks{\psi}_I \).
\end{proposition}
\begin{proof}
  Follows from \fullref{thm:propositional_natural_deduction_soundness}.
\end{proof}

\begin{proposition}\label{thm:lindenbaum_tarski_algebras}
  The \hyperref[def:lindenbaum_tarski_algebra]{Lindenbaum-Tarski algebras} of the \hyperref[def:propositional_natural_deduction_systems]{minimal natural deduction system} are \hyperref[def:extremal_points/bounds]{bounded from above} \hyperref[def:distributive_lattice]{distributive lattices}.

  \begin{thmenum}[series=thm:lindenbaum_tarski_algebras]
    \thmitem{thm:lindenbaum_tarski_algebras/intuitionistic} For the intuitionistic natural deduction system, the Lindenbaum-Tarski algebras are \hyperref[def:heyting_algebra]{Heyting algebras}.

    \thmitem{thm:lindenbaum_tarski_algebras/classical} For the classical natural deduction system, they are instead \hyperref[def:boolean_algebra]{Boolean algebras}.
  \end{thmenum}

  More precisely, in every such lattice:
  \begin{thmenum}[resume=thm:lindenbaum_tarski_algebras]
    \thmitem{thm:lindenbaum_tarski_algebras/verum} The class of \hyperref[def:propositional_tautology]{tautologies} \( \pi(\syntop) \) is the \hyperref[def:extremal_points/top_and_bottom]{top element}.

    \thmitem{thm:lindenbaum_tarski_algebras/falsum} For intuitionistic logic, the class of \hyperref[def:propositional_contradiction]{contradictions} \( \pi(\synbot) \) is the \hyperref[def:extremal_points/top_and_bottom]{bottom element}.

    \thmitem{thm:lindenbaum_tarski_algebras/negation} For intuitionistic logic, the \hyperref[eq:def:heyting_algebra/pseudocomplement]{pseudocomplement} \( \widetilde{\pi(\psi)} \) of the equivalence class \( \pi(\psi) \) is the class \( \pi(\synneg \psi) \) of their \hyperref[def:propositional_alphabet/negation]{negation}.

    \thmitem{thm:lindenbaum_tarski_algebras/disjunction} The \hyperref[def:lattice/join]{join} of \( \pi(\varphi) \) and \( \pi(\psi) \) is the class \( \pi(\varphi \synvee \psi) \) of their \hyperref[def:propositional_alphabet/connectives/disjunction]{disjunction}.

    \thmitem{thm:lindenbaum_tarski_algebras/conjunction} The \hyperref[def:lattice/meet]{meet} of \( \pi(\varphi) \) and \( \pi(\psi) \) is the equivalence class \( \pi(\varphi \synwedge \psi) \) of their \hyperref[def:propositional_alphabet/connectives/conjunction]{conjunction}.

    \thmitem{thm:lindenbaum_tarski_algebras/conditional} The \hyperref[def:heyting_algebra]{relative pseudocomplement} of \( \pi(\varphi) \) and \( \pi(\psi) \) is the equivalence class \( \pi(\varphi \synimplies \psi) \) of their \hyperref[def:propositional_alphabet/connectives/conditional]{conditional formula}.

    \thmitem{thm:lindenbaum_tarski_algebras/biconditional} The \hyperref[def:heyting_algebra/biconditional]{biconditional} of \( \pi(\varphi) \) and \( \pi(\psi) \) is the equivalence class \( \pi(\varphi \syniff \psi) \) of their \hyperref[def:propositional_alphabet/connectives/biconditional]{biconditional formula}.
  \end{thmenum}
\end{proposition}
\begin{proof}
  Fix a set of formulas \( \Gamma \) and denote by \( L \) its Lindenbaum-Tarski algebra for the intuitionistic natural deduction system.

  We will prove \fullref{thm:lindenbaum_tarski_algebras/verum} through \fullref{thm:lindenbaum_tarski_algebras/biconditional}. From \fullref{thm:lindenbaum_tarski_algebras/conjunction} and \fullref{thm:lindenbaum_tarski_algebras/disjunction} it will follow that \( L \) is a lattice; boundedness will follow from \fullref{thm:lindenbaum_tarski_algebras/verum} and \fullref{thm:lindenbaum_tarski_algebras/falsum}. \Fullref{thm:lindenbaum_tarski_algebras/conditional} will show that \( L \) is a Heyting algebra.

  If we instead consider minimal logic, \( L \) may not be a Heyting algebra and may no longer be bounded from below, but we can still show that \( L \) is a distributive lattice by replicating the proof of \fullref{thm:def:heyting_algebra/distributive}, which uses the characterization from \fullref{thm:distributive_lattice_characterization} with concrete counterexamples given by \fullref{ex:heyting_conditional_ideal/pentagon} and \fullref{ex:heyting_conditional_ideal/diamond}.

  After all of that, we will consider classical logic in the proof of \fullref{thm:lindenbaum_tarski_algebras/classical}.

  \SubProofOf{thm:lindenbaum_tarski_algebras/verum} The rule \eqref{eq:def:propositional_natural_deduction_systems/top/intro} allows us to construct a proof of \( \syntop \) from \( \Gamma \cup \set{ \varphi } \) from any formula \( \varphi \), hence \( \varphi \leq_\Gamma \top \). Then \( \pi(\syntop) \) is the unique top element.

  \SubProofOf{thm:lindenbaum_tarski_algebras/falsum} The rule \eqref{eq:def:propositional_natural_deduction_systems/bot/efq} allows us to construct a proof of any formula \( \varphi \) from \( \synbot \). Then \( \pi(\synbot) \) is the unique bottom element.

  \SubProofOf{thm:lindenbaum_tarski_algebras/negation} The rule \eqref{eq:def:propositional_natural_deduction_systems/neg/intro} allows us to construct a proof of \( \synneg \varphi \) from \( \varphi \) for any formula \( \varphi \).

  \SubProofOf{thm:lindenbaum_tarski_algebras/disjunction} Suppose that \( \theta \) is an upper bound of \( \varphi \) and \( \psi \). Then \( \Gamma, \varphi \vdash \theta \) and \( \Gamma, \varphi \vdash \theta \), and \ref{eq:def:propositional_natural_deduction_systems/or/elim} allows us to construct a proof of \( \theta \) from \( \Gamma \cup \set{ \varphi \synvee \psi } \), hence \( \varphi \synvee \psi \) is the least upper bound of \( \varphi \) and \( \psi \).

  \SubProofOf{thm:lindenbaum_tarski_algebras/conjunction} Suppose that \( \theta \) is a lower bound of \( \varphi \) and \( \psi \). Then \( \Gamma, \theta \vdash \varphi \) and \( \Gamma, \theta \vdash \psi \), and \ref{eq:def:propositional_natural_deduction_systems/and/intro} allows us to construct a proof of \( \varphi \synwedge \psi \) from \( \Gamma \cup \set{ \theta } \), hence \( \varphi \synwedge \psi \) is the greatest lower bound of \( \varphi \) and \( \psi \).

  \SubProofOf{thm:lindenbaum_tarski_algebras/conditional}

  \SubProof*{Proof that \( \varphi \synwedge (\varphi \synimplies \psi) \leq_\Gamma \psi \)}

  We have the following proof tree, which we will denote via \( P \):
  \begin{equation*}
    \begin{prooftree}
      \hypo{ [\varphi \synwedge (\varphi \synimplies \psi)]^u }
      \infer1[\ref{eq:def:propositional_natural_deduction_systems/and/elim_left}]{ \varphi \synimplies \psi }

      \hypo{ [\varphi \synwedge (\varphi \synimplies \psi)]^u }
      \infer1[\ref{eq:def:propositional_natural_deduction_systems/and/elim_right}]{ \varphi }

      \infer2[\ref{eq:def:propositional_natural_deduction_systems/imp/elim}]{ \psi }
    \end{prooftree}
  \end{equation*}

  \SubProof*{Proof that \( \theta \leq_\Gamma \varphi \synimplies \psi \) if and only if \( \varphi \synwedge \theta \leq_\Gamma \psi \)}

  First, let \( F \) be a proof tree deriving \( \varphi \synimplies \psi \) from \( \Gamma \cup \set{ \theta } \). The following tree derives \( \psi \) from \( \Gamma \cup \set{ \varphi \synwedge \theta } \):
  \begin{equation*}
    \begin{prooftree}
      \hypo{ [\varphi \synwedge \theta]^v }
      \infer1[\ref{eq:def:propositional_natural_deduction_systems/and/elim_left}]{ \theta }

      \ellipsis {\( F \)} {\varphi \synimplies \psi}

      \hypo{ [\varphi \synwedge \theta]^v }
      \infer1[\ref{eq:def:propositional_natural_deduction_systems/and/elim_right}]{ \varphi }

      \infer2[\ref{eq:def:propositional_natural_deduction_systems/imp/elim}]{ \psi }
    \end{prooftree}
  \end{equation*}

  Conversely, let \( G \) be a tree that derives \( \psi \) from \( \Gamma \cup \set{ \varphi \synwedge \theta } \). The following tree derives \( \varphi \synimplies \psi \) from \( \theta \):
  \begin{equation*}
    \begin{prooftree}
      \hypo{ [\varphi]^w }
      \hypo{ [\theta]^x }
      \infer1[\ref{eq:def:propositional_natural_deduction_systems/and/intro}]{ \varphi \synwedge \theta }

      \ellipsis {\( G \)} {\psi}

      \infer[left label=\( w \)]2[\ref{eq:def:propositional_natural_deduction_systems/imp/intro}]{ \varphi \synimplies \psi }
    \end{prooftree}
  \end{equation*}

  \SubProofOf{thm:lindenbaum_tarski_algebras/biconditional} We will first show that
  \begin{equation*}
    \pi\parens[\Big]{ \varphi \syniff \psi } = \pi\parens[\Big]{ (\varphi \synimplies \psi) \synwedge (\psi \synimplies \varphi) }.
  \end{equation*}

  Indeed, we have
  \begin{equation*}
    \begin{prooftree}
      \hypo{ [\varphi \syniff \psi]^u }
      \hypo{ [\varphi]^v }
      \infer2[\ref{eq:def:propositional_natural_deduction_systems/iff/elim_left}]{ \psi }
      \infer[left label=\( v \)]1[\ref{eq:def:propositional_natural_deduction_systems/imp/intro}]{ \varphi \synimplies \psi }

      \hypo{ [\varphi \syniff \psi]^u }
      \hypo{ [\psi]^w }
      \infer2[\ref{eq:def:propositional_natural_deduction_systems/iff/elim_right}]{ \varphi }
      \infer[left label=\( w \)]1[\ref{eq:def:propositional_natural_deduction_systems/imp/intro}]{ \psi \synimplies \varphi }

      \infer2[\ref{eq:def:propositional_natural_deduction_systems/and/intro}]{ (\varphi \synimplies \psi) \synwedge (\psi \synimplies \varphi) }
    \end{prooftree}
  \end{equation*}
  and
  \begin{equation*}
    \begin{prooftree}
      \hypo{ [(\varphi \synimplies \psi) \synwedge (\psi \synimplies \varphi)]^u }
      \infer1[\ref{eq:def:propositional_natural_deduction_systems/and/elim_right}]{ \varphi \synimplies \psi }
      \hypo{ [\varphi]^v }
      \infer2[\ref{eq:def:propositional_natural_deduction_systems/imp/elim}]{ \psi }

      \hypo{ [(\varphi \synimplies \psi) \synwedge (\psi \synimplies \varphi)]^u }
      \infer1[\ref{eq:def:propositional_natural_deduction_systems/and/elim_left}]{ \psi \synimplies \varphi }
      \hypo{ [\psi]^w }
      \infer2[\ref{eq:def:propositional_natural_deduction_systems/imp/elim}]{ \varphi }

      \infer[left label={\( v, w \)}]2[\ref{eq:def:propositional_natural_deduction_systems/iff/intro}]{ \varphi \syniff \psi }
    \end{prooftree}
  \end{equation*}

  \Fullref{thm:lindenbaum_tarski_algebras/conjunction} and \fullref{thm:lindenbaum_tarski_algebras/conditional} then imply that
  \begin{align*}
    \pi\parens[\Big]{ \varphi \syniff \psi }
    &=
    \pi\parens[\Big]{ (\varphi \synimplies \psi) \synwedge (\psi \synimplies \varphi) }
    \reloset {\ref{thm:lindenbaum_tarski_algebras/conjunction}} = \\ &=
    \pi(\varphi \synimplies \psi) \wedge \pi(\psi \synimplies \varphi)
    \reloset {\ref{thm:lindenbaum_tarski_algebras/conditional}} = \\ &=
    \parens[\Big]{ \pi(\varphi) \rightarrow \pi(\psi) } \wedge \parens[\Big]{ \pi(\psi) \rightarrow \pi(\varphi) },
  \end{align*}
  which by definition equals \( \pi(\varphi) \leftrightarrow \pi(\psi) \).

  \SubProofOf{thm:lindenbaum_tarski_algebras/classical} Finally, now that we have shown that \( L \) is a Heyting algebra, we must show that it is also a Boolean algebra when we consider the additional rule \eqref{eq:def:propositional_natural_deduction_systems/bot/dne}.

  According to \fullref{thm:boolean_algebra_heyting_characterization}, we only need to derive \( \varphi \) from \( \neg \neg \varphi \); but this is precisely what \eqref{eq:def:propositional_natural_deduction_systems/bot/dne} allows us to do.
\end{proof}

\begin{theorem}[Intuitionistic propositional completeness]\label{thm:intuitionistic_propositional_completeness}\mcite[705]{TroelstraVanDalen1988Vol2}
  For any finite set \( \Gamma \) of \hyperref[def:propositional_syntax/formula]{propositional formulas}, there exists a \hyperref[def:heyting_algebra]{Heyting algebra} with respect to which \( \Gamma \vDash \varphi \) whenever \( \Gamma \vdash \varphi \) in the \hyperref[def:propositional_natural_deduction_systems]{intuitionistic natural deduction system}.
\end{theorem}
\begin{proof}
  \Fullref{thm:lindenbaum_tarski_algebras} implies that the \hyperref[def:lindenbaum_tarski_algebra]{Lindenbaum-Tarski algebra} \( \BbbH \) of \( \Gamma \) is a Heyting algebra.

  Consider the \hyperref[def:propositional_valuation/interpretation]{interpretation} \( I: \op*{Var} \to \BbbH \) obtained as a restriction of \( \BbbH \) to propositional variables. As a consequence of \fullref{thm:lindenbaum_tarski_algebras}, the corresponding \hyperref[def:propositional_valuation/formula_valuation]{valuation} is then simply the Lindenbaum-Tarski projection \( \pi: \op*{Prop} / {\sim_\Gamma} \to \BbbH \).

  Suppose that \( \Gamma \vdash \varphi \). Let \( \psi_1, \ldots, \psi_n \) be an enumeration of \( \Gamma \). \Fullref{thm:syntactic_propositional_conjunction_of_premises} implies that \( \psi_1 \wedge \cdots \wedge \psi_n \vdash \varphi \).

  \Fullref{thm:lindenbaum_tarski_algebras/conjunction} implies that
  \begin{equation*}
    \pi(\psi_1) \wedge \cdots \wedge \pi(\psi_n)
    =
    \pi(\psi_1 \wedge \cdots \wedge \psi_n)
    \leq
    \pi(\varphi).
  \end{equation*}

  \Fullref{thm:def:heyting_algebra/leq} implies that \( \psi_1 \wedge \cdots \wedge \psi_n \vDash \varphi \). \Fullref{thm:semantic_propositional_conjunction_of_premises} then implies that \( \Gamma \vDash \varphi \).
\end{proof}

\paragraph{Complete theories}

\begin{definition}\label{def:complete_set_of_sentences}
  Fix a \hyperref[def:consequence_operator]{consequence operator} \( \op{Cn} \) on a set of sentences \( S \), and let \( {\vdash} \) be the corresponding \hyperref[def:consequence_relation]{consequence relation}. We say that the set of sentences \( \Gamma \) is \term[ru=полное (множество формул) (\cite[def. 1.3.16]{Герасимов2011})]{complete} with respect to them if any of the following equivalent conditions hold:
  \begin{thmenum}
    \thmitem{def:complete_set_of_sentences/operator}\mcite[def. III.4]{Woodger1983Tarski} Whenever \( \Delta \) is a \hyperref[def:consistent_set_of_sentences]{consistent} set of sentences extending \( \Gamma \), it is \hyperref[def:logical_theory/equivalent]{equivalent} to \( \Gamma \).

    \thmitem{def:complete_set_of_sentences/relation} For any formula \( \varphi \) for which \( \Gamma \cup \set{ \varphi } \) is consistent, we have \( \Gamma \vdash \varphi \).
  \end{thmenum}
\end{definition}
\begin{proof}
  \ImplicationSubProof{def:complete_set_of_sentences/operator}{def:complete_set_of_sentences/relation} Suppose that \( \Gamma \) is complete in the sense of \fullref{def:complete_set_of_sentences}. Fix a formula \( \varphi \) for which \( \Gamma \cup \set{ \varphi } \) is consistent. From \eqref{eq:def:consequence_relation/reflexivity} it follows that \( \Gamma, \varphi \vdash \varphi \) and, by assumption, \( \Gamma \vdash \varphi \).

  \ImplicationSubProof{def:complete_set_of_sentences/relation}{def:complete_set_of_sentences/operator} Conversely, suppose that, whenever \( \Gamma \cup \set{ \varphi } \) is consistent, we have \( \Gamma \vdash \varphi \).

  Let \( \Delta \) is a consistent theory extending \( \Gamma \). For every formula \( \varphi \) in \( \Delta \), by \fullref{thm:def:consistent_set_of_sentences/subset}, the theory \( \Gamma \cup \set{ \varphi } \) is consistent, and thus \( \Gamma \vdash \varphi \). Then \( \Delta \) is equivalent to \( \Gamma \).
\end{proof}

\begin{proposition}\label{thm:def:complete_set_of_sentences}
  \hyperref[def:complete_set_of_sentences]{Complete sets of sentences} have the following basic properties:
  \begin{thmenum}
    \thmitem{thm:def:complete_set_of_sentences/inconsistent} An \hyperref[def:consistent_set_of_sentences]{inconsistent set} is vacuously complete.
    \thmitem{thm:def:complete_set_of_sentences/superset} Every superset of a complete set is complete.
    \thmitem{thm:def:complete_set_of_sentences/subset} Every subset of an incomplete set is incomplete.
  \end{thmenum}
\end{proposition}
\begin{proof}
  \SubProofOf{thm:def:complete_set_of_sentences/inconsistent} Fix an inconsistent set \( \Gamma \). \Fullref{thm:def:consistent_set_of_sentences/superset} implies that, for every formula \( \varphi \), the set \( \Gamma \cup \set{ \varphi } \) is also inconsistent. Then \fullref{def:complete_set_of_sentences/relation} holds vacuously.

  \SubProofOf{thm:def:complete_set_of_sentences/superset} Let \( \Gamma \) be a complete set and fix a superset \( \Delta \) of \( \Gamma \).

  Suppose that \( \Delta \cup \set{ \varphi } \) is consistent for some formula \( \varphi \). Then \( \Gamma \cup \set{ \varphi } \) is also consistent as a consequence of \fullref{thm:def:consistent_set_of_sentences/subset}. Since \( \Gamma \) is complete, we have \( \Gamma \vdash \varphi \), and, by \eqref{eq:def:consequence_relation/monotonicity}, \( \Delta \vdash \varphi \).

  Generalizing on \( \varphi \), we conclude that \( \Delta \) is consistent.

  \SubProofOf{thm:def:complete_set_of_sentences/subset} Let \( \Gamma \) be an incomplete set and fix a subset \( \Delta \) of \( \Gamma \). Suppose that \( \Gamma \) is complete. By \fullref{thm:def:consistent_set_of_sentences/superset}, \( \Gamma \) is complete, which is a contradiction. Then \( \Delta \) is incomplete.
\end{proof}

\begin{proposition}\label{thm:propositional_complete_set}
  With respect to the \hyperref[def:propositional_natural_deduction_systems]{minimal natural deduction system}, the set \( \Gamma \) of \hyperref[def:propositional_syntax/formula]{propositional formulas} is \hyperref[def:complete_set_of_sentences]{complete} if and only if, for any formula \( \varphi \), either \( \Gamma \vdash \varphi \) or \( \Gamma \vdash \neg \varphi \) (or possibly both, if \( \Gamma \) is \hyperref[def:consistent_set_of_sentences]{inconsistent}).
\end{proposition}
\begin{proof}
  \SufficiencySubProof Suppose that \( \Gamma \) is complete in the sense of \fullref{def:complete_set_of_sentences/relation}. Fix an arbitrary formula \( \varphi \).

  \begin{itemize}
    \item Suppose that \( \Gamma \cup \set{ \varphi } \) is inconsistent. \Fullref{thm:propositional_consistent_set/bot} implies that there exists a proof \( P \) deriving \( \bot \) from \( \Gamma \cup \set{ \varphi } \).

    \begin{itemize}
      \item If \( \varphi \) is an open assumption in \( P \) with marker \( u \), the following is a proof tree of \( \neg \varphi \) from \( \Gamma \):
      \begin{equation*}
        \begin{prooftree}
          \hypo{ [\varphi]^u }
          \ellipsis { \( P \) } { \bot }
          \infer1[\eqref{eq:def:propositional_natural_deduction_systems/neg/intro}]{ \neg \varphi }.
        \end{prooftree}
      \end{equation*}

      \item Otherwise, \( P \) is a proof of \( \bot \) from \( \Gamma \), and \fullref{thm:propositional_consistent_set/bot} implies that \( \Gamma \) is inconsistent. \Fullref{thm:propositional_consistent_set/contradiction} then implies that both \( \varphi \) and \( \neg \varphi \) are derivable from \( \Gamma \).
    \end{itemize}

    \item If \( \Gamma \cup \set{ \varphi } \) is not inconsistent, by definition it is consistent, and, since \( \Gamma \) is complete, \( \Gamma \vdash \varphi \).
  \end{itemize}

  \NecessitySubProof Suppose that, for any formula \( \varphi \), we have \( \Gamma \vdash \varphi \) or \( \Gamma \vdash \neg \varphi \) (or both).

  Fix a formula \( \varphi \) for which \( \Gamma \cup \set{ \varphi } \) is consistent. \Fullref{thm:def:consistent_set_of_sentences/subset} then implies that \( \Gamma \) is consistent.

  \begin{itemize}
    \item If \( \Gamma \vdash \neg \varphi \), \fullref{thm:propositional_consistent_set/contradiction} implies that \( \Gamma \) is inconsistent, which is a contradiction.
    \item It remains for \( \Gamma \vdash \varphi \) to hold.
  \end{itemize}

  Generalizing on \( \varphi \), we conclude that \( \Gamma \) satisfies \fullref{def:complete_set_of_sentences/relation}.
\end{proof}

\begin{proposition}\label{thm:lindenbaum_tarski_theories}
  With respect to the \hyperref[def:propositional_natural_deduction_systems]{minimal natural deduction system}, consider the \hyperref[def:lindenbaum_tarski_algebra]{Lindenbaum-Tarski algebra} \( L \) of an empty set of premises.

  \begin{thmenum}
    \thmitem{thm:lindenbaum_tarski_theories/filter} For every \hyperref[def:logical_theory]{theory} \( \Delta \), the projection \( \pi[\Delta] \) is a \hyperref[def:lattice_ideal]{filter} in \( L \).

    \thmitem{thm:lindenbaum_tarski_theories/consistent} The theory \( \Delta \) of formulas is \hyperref[def:consistent_set_of_sentences]{inconsistent} if and only if \( \pi[\Delta] = L \).

    \thmitem{thm:lindenbaum_tarski_theories/complete} In the natural deduction system is instead classical, the theory \( \Delta \) is \hyperref[def:complete_set_of_sentences]{complete} if and only if \( \pi[\Delta] \) is an \hyperref[def:ultrafilter]{ultrafilter}.
  \end{thmenum}
\end{proposition}
\begin{proof}
  \SubProofOf{thm:lindenbaum_tarski_theories/filter} We will use the characterization \fullref{thm:def:lattice_ideal/directed_and_closed}.

  \SubProofOf*[def:directed_set]{downward directedness} Fix two formula \( \varphi \) and \( \psi \) in \( \Delta \). Since \( \Delta \) is a theory, it is closed under entailment, hence \( \varphi \synwedge \psi \) is in \( \Delta \). \Fullref{thm:lindenbaum_tarski_algebras/conjunction} implies that the equivalence class \( \pi(\varphi \synwedge \psi) \) is the meet of \( \pi(\varphi) \) and \( \pi(\psi) \), hence \( \pi(\varphi \synwedge \psi) \) is in \( \pi[\Delta] \).

  Then \( \pi[\Delta] \) is closed with respect to meets. In particular, every two elements of \( \pi[\Delta] \) have a lower bound.

  \SubProofOf*[def:directed_set]{upward closure} Let \( \varphi \) be a formula in \( \Delta \), and suppose that \( \varphi \vdash \psi \). Then \eqref{eq:def:consequence_relation/monotonicity} implies that \( \Delta \vdash \psi \), and, since it is a theory, we have \( \psi \in \Gamma \). Hence, \( \pi(\psi) \) is in \( \pi[\Delta] \).

  Then \( \pi[\Delta] \) is upward closed.

  \SubProofOf{thm:lindenbaum_tarski_theories/consistent}

  \SufficiencySubProof* Suppose that \( \Delta \) is inconsistent. Then it can derive any propositional formula, and, since it is a theory, it also contains every propositional formula. Naturally, \( \pi[\Delta] = L \).

  \NecessitySubProof* Suppose that \( \pi[\Delta] = L \). Fix any formula \( \varphi \). The projection \( \pi(\varphi) \) is in \( \pi[\Delta] \), implying that there is a formula \( \psi \) derivable from \( \Delta \) in the preimage of \( \pi(\varphi) \). Then \eqref{eq:def:consequence_relation/transitivity} implies that \( \varphi \) is also derivable from \( \Delta \).

  \SubProofOf{thm:lindenbaum_tarski_theories/complete} \Fullref{thm:propositional_complete_set} corresponds precisely to \fullref{def:ultrafilter/direct}.
\end{proof}

\begin{lemma}[Lindenbaum's lemma]\label{thm:extension_to_complete_consistent_set}
  Within the \hyperref[def:propositional_natural_deduction_system]{classical propositional natural deduction system}, every \hyperref[def:consistent_set_of_sentences]{consistent} \hyperref[def:logical_theory]{theory} can be extended to a \hyperref[def:complete_set_of_sentences]{complete} consistent theory.
\end{lemma}
\begin{comments}
  \item The name of the lemma is taken from \incite[lemma 1.3.12]{Герасимов2011}, but the proof is simplified.
\end{comments}
\begin{proof}
  Denote by \( L \) the \hyperref[thm:lindenbaum_tarski_algebras]{classical Lindenbaum-Tarski algebra} with no premises.

  Let \( \Gamma \) be a consistent set of formulas, and let \( \Delta \) be the theory axiomatized by \( \Gamma \). \Fullref{thm:lindenbaum_tarski_theories/consistent} implies that \( \pi[\Delta] \) is a proper filter of \( L \). \Fullref{thm:ultrafilter_lemma} implies that there exists an ultrafilter \( M \) containing \( \pi[\Delta] \).

  Let \( \Epsilon \) be the union of \( M \), that is, the set of all formulas \( \varphi \) for which \( \pi(\varphi) \) is in \( M \). We have \( \pi[\Epsilon] = M \).

  Then \( \Epsilon \) is a consistent due to \fullref{thm:lindenbaum_tarski_theories/consistent}, and complete due to \fullref{thm:lindenbaum_tarski_theories/complete}.
\end{proof}
