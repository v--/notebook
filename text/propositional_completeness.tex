\section{Propositional completeness}\label{sec:propositional_completeness}

\begin{remark}\label{rem:classical_metatheoretic_proofs}
  In the proofs here we tend to avoid the metatheoretic counterparts of \ref{inf:def:propositional_natural_deduction_systems/bot/efq} and \ref{inf:def:propositional_natural_deduction_systems/bot/dne}, but the underlying notions are built in classical logic; thus, the proofs are still technically classical.
\end{remark}

\paragraph{Consistent theories}

\begin{definition}\label{def:consistent_set_of_sentences}\mcite[def. III.3]{Woodger1983TarskiPapers}
  Fix a \hyperref[def:consequence_operator]{consequence operator} \( \op{Cn} \) on a set of sentences \( S \).

  We say that the set of sentences \( \Gamma \) is \term[ru=противоречивое (множество формул) (\cite[def. 1.3.15]{Герасимов2011Вычислимость})]{inconsistent} with respect to this operator if \( \op{Cn}(\Gamma) \) coincides with \( S \). When \( \Gamma \) is not inconsistent, we say that it is \term{consistent}.
\end{definition}
\begin{comments}
  \item We choose inconsistency as the base notion in order to avoid relying on double negation elimination in our proofs.
  \item Within a given \hyperref[def:logical_framework]{logical framework}, we will by default refer to syntactic consistency, as a counterpart to \hyperref[def:satisfiable_set_of_sentences]{satisfiable theories}.
\end{comments}

\begin{proposition}\label{thm:def:consistent_set_of_sentences}
  \hyperref[def:consistent_set_of_sentences]{Consistent sets of sentences} have the following basic properties:
  \begin{thmenum}
    \thmitem{thm:def:consistent_set_of_sentences/superset} Every superset of an inconsistent set is inconsistent.
    \thmitem{thm:def:consistent_set_of_sentences/subset} Every subset of a consistent set is consistent.
    \thmitem{thm:def:consistent_set_of_sentences/entailment} If \( \Gamma \vdash \varphi \), then \( \Gamma \) is consistent if and only if \( \Gamma \cup \set{ \varphi } \) is.
  \end{thmenum}
\end{proposition}
\begin{proof}
  \SubProofOf{thm:def:consistent_set_of_sentences/superset} Let \( \Gamma \) be an inconsistent set and let \( \Delta \) be a superset.

  Then \eqref{eq:def:consequence_relation/monotonicity} implies that every sentence derivable from \( \Gamma \) can be derived from \( \Delta \). But \( \Gamma \) can derive any sentence, thus \( \Delta \) also can. Then \( \Delta \) is inconsistent.

  \SubProofOf{thm:def:consistent_set_of_sentences/subset} Let \( \Delta \) be a consistent set and let \( \Gamma \) be a subset.

  If we suppose that \( \Gamma \) is inconsistent, \fullref{thm:def:consistent_set_of_sentences/superset} would imply that \( \Delta \) also is, which is a contradiction. Then \( \Gamma \) is not inconsistent, i.e. it is consistent.

  \SubProofOf{thm:def:consistent_set_of_sentences/entailment} Let \( \Gamma \vdash \varphi \). Then every sentence derivable from \( \Gamma \cup \set{ \varphi } \) can be derived from \( \Gamma \), and thus the two are equivalent. Then either both are inconsistent or consistent.
\end{proof}

\begin{proposition}\label{thm:propositional_consistent_set}
  With respect to the \hyperref[def:propositional_natural_deduction_systems]{intuitionistic natural deduction system}, following conditions are equivalent for a set \( \Gamma \) of \hyperref[def:propositional_syntax/formula]{propositional formulas} to be \hyperref[def:consistent_set_of_sentences]{inconsistent}:
  \begin{thmenum}
    \thmitem{thm:propositional_consistent_set/bot} The falsum \( \synbot \) is derivable from \( \Gamma \).

    \thmitem{thm:propositional_consistent_set/contradiction} For every formula \( \varphi \), both \( \varphi \) and \( \synneg \varphi \) are derivable from \( \Gamma \).
  \end{thmenum}
\end{proposition}
\begin{proof}
  \SubProofOf{thm:propositional_consistent_set/bot}

  \SufficiencySubProof* If \( \Gamma \) is inconsistent in the sense of \fullref{def:consistent_set_of_sentences}, then, as a propositional formula, \( \synbot \) is derivable from \( \Gamma \).

  \NecessitySubProof* Let \( P \) be a \hyperref[def:natural_deduction_proof_tree]{proof tree} of \( \synbot \) from \( \Gamma \). Then, for any formula \( \varphi \), the following is a proof of \( \varphi \) from \( \Gamma \):
  \begin{equation*}
    \begin{prooftree}
      \hypo{}
      \ellipsis { \( P \) } { \synbot }
      \infer1[\ref{inf:def:propositional_natural_deduction_systems/bot/efq}]{ \varphi }.
    \end{prooftree}
  \end{equation*}

  \SubProofOf{thm:propositional_consistent_set/contradiction}

  \SufficiencySubProof* If \( \Gamma \) is inconsistent in the sense of \fullref{def:consistent_set_of_sentences}, then, for any formula \( \varphi \), both \( \varphi \) and \( \synneg \varphi \) are derivable from \( \Gamma \).

  \NecessitySubProof* Let \( P \) and \( N \) be proof trees of \( \varphi \) and \( \synneg \varphi \) from \( \Gamma \), respectively. Let \( \psi \) be any formula. Then the following is a proof tree deriving \( \psi \) from \( \Gamma \):
  \begin{equation*}
    \begin{prooftree}
      \hypo{ \Gamma }
      \ellipsis { \( N \) } { \synneg \varphi }

      \hypo{ \Gamma }
      \ellipsis { \( P \) } { \varphi }

      \infer2[\ref{inf:def:propositional_natural_deduction_systems/neg/elim}]{ \synbot }.

      \infer1[\ref{inf:def:propositional_natural_deduction_systems/bot/efq}]{ \psi }.
    \end{prooftree}
  \end{equation*}
\end{proof}

\paragraph{Lindenbaum-Tarski algebras}

\begin{definition}\label{def:lindenbaum_tarski_algebra}\mcite[27]{Эдельман1975Логика}
  Fix a \hyperref[def:consequence_relation]{consequence relation} \( {\vdash} \) on a set of sentences \( S \).

  For some fixed subset \( \Gamma \) of sentences, consider the \hyperref[def:binary_relation]{binary relation}
  \begin{equation*}
    \varphi \leq_\Gamma \psi \T{if} \Gamma, \varphi \vdash \psi.
  \end{equation*}

  Then \( S \) is a \hyperref[def:preordered_set]{preordered set} with respect to \( \leq_\Gamma \) due to \eqref{eq:def:consequence_relation/reflexivity} and \eqref{eq:def:consequence_relation/transitivity}. The corresponding \hyperref[def:antisymmetric_quotient]{antisymmetric quotient} \( Q \) is then a \hyperref[def:partially_ordered_set]{partially ordered set}, which we call the \term[ru=алгебра Линденбаума-Тарского, en=Lindenbaum-Tarski algebra (\cite[def. 6.3.1]{CitkinMuravitsky2022ConsequenceRelations})]{Lindenbaum-Tarski algebra} of \( \Gamma \).
\end{definition}
\begin{comments}
  \item We will find useful the projection \( \pi_\Gamma: S \to Q \).
\end{comments}

\begin{proposition}\label{thm:inconsistent_lindenbaum_tarski_algebra}
  A set of sentences is \hyperref[def:consistent_set_of_sentences]{inconsistent} if and only if its \hyperref[def:lindenbaum_tarski_algebra]{Lindenbaum-Tarski algebra} is trivial, i.e. \( T = F \).
\end{proposition}
\begin{proof}
  Any formula is derivable in an inconsistent theory.
\end{proof}

\begin{proposition}\label{thm:lindenbaum_tarski_algebras}
  The \hyperref[def:lindenbaum_tarski_algebra]{Lindenbaum-Tarski algebras} of the \hyperref[def:propositional_natural_deduction_systems]{minimal natural deduction system} are \hyperref[def:extremal_points/bounds]{bounded from above} \hyperref[def:distributive_lattice]{distributive lattices}.

  \begin{thmenum}[series=thm:lindenbaum_tarski_algebras]
    \thmitem{thm:lindenbaum_tarski_algebras/intuitionistic} For the intuitionistic natural deduction system, the Lindenbaum-Tarski algebras are \hyperref[def:heyting_algebra]{Heyting algebras}.

    \thmitem{thm:lindenbaum_tarski_algebras/classical} For the classical natural deduction system, they are instead \hyperref[def:boolean_algebra]{Boolean algebras}.
  \end{thmenum}

  More precisely, in every such lattice:
  \begin{thmenum}[resume=thm:lindenbaum_tarski_algebras]
    \thmitem{thm:lindenbaum_tarski_algebras/verum} The class of \hyperref[def:propositional_tautology]{tautologies} \( \pi(\syntop) \) is the \hyperref[def:extremal_points/top_and_bottom]{top element}.

    \thmitem{thm:lindenbaum_tarski_algebras/falsum} For intuitionistic logic, the class of \hyperref[def:propositional_contradiction]{contradictions} \( \pi(\synbot) \) is the \hyperref[def:extremal_points/top_and_bottom]{bottom element}.

    \thmitem{thm:lindenbaum_tarski_algebras/negation} For intuitionistic logic, the \hyperref[eq:def:heyting_algebra/pseudocomplement]{pseudocomplement} \( \oline{\pi(\psi)} \) of the equivalence class \( \pi(\psi) \) is the class \( \pi(\synneg \psi) \) of their \hyperref[def:propositional_alphabet/negation]{negation}.

    \thmitem{thm:lindenbaum_tarski_algebras/disjunction} The \hyperref[def:lattice/join]{join} of \( \pi(\varphi) \) and \( \pi(\psi) \) is the class \( \pi(\varphi \synvee \psi) \) of their \hyperref[def:propositional_alphabet/connectives/disjunction]{disjunction}.

    \thmitem{thm:lindenbaum_tarski_algebras/conjunction} The \hyperref[def:lattice/meet]{meet} of \( \pi(\varphi) \) and \( \pi(\psi) \) is the equivalence class \( \pi(\varphi \synwedge \psi) \) of their \hyperref[def:propositional_alphabet/connectives/conjunction]{conjunction}.

    \thmitem{thm:lindenbaum_tarski_algebras/conditional} The \hyperref[def:heyting_algebra]{relative pseudocomplement} of \( \pi(\varphi) \) and \( \pi(\psi) \) is the equivalence class \( \pi(\varphi \synimplies \psi) \) of their \hyperref[def:propositional_alphabet/connectives/conditional]{conditional formula}.

    \thmitem{thm:lindenbaum_tarski_algebras/biconditional} The \hyperref[def:heyting_algebra/biconditional]{biconditional} of \( \pi(\varphi) \) and \( \pi(\psi) \) is the equivalence class \( \pi(\varphi \syniff \psi) \) of their \hyperref[def:propositional_alphabet/connectives/biconditional]{biconditional formula}.
  \end{thmenum}
\end{proposition}
\begin{proof}
  Fix a set of formulas \( \Gamma \) and denote by \( L \) its Lindenbaum-Tarski algebra for the intuitionistic natural deduction system.

  We will prove \fullref{thm:lindenbaum_tarski_algebras/verum} through \fullref{thm:lindenbaum_tarski_algebras/biconditional}. From \fullref{thm:lindenbaum_tarski_algebras/conjunction} and \fullref{thm:lindenbaum_tarski_algebras/disjunction} it will follow that \( L \) is a lattice; boundedness will follow from \fullref{thm:lindenbaum_tarski_algebras/verum} and \fullref{thm:lindenbaum_tarski_algebras/falsum}. \Fullref{thm:lindenbaum_tarski_algebras/conditional} will show that \( L \) is a Heyting algebra.

  If we instead consider minimal logic, \( L \) may not be a Heyting algebra and may no longer be bounded from below, but we can still show that \( L \) is a distributive lattice by replicating the proof of \fullref{thm:def:heyting_algebra/distributive}, which uses the characterization from \fullref{thm:distributive_lattice_characterization} with concrete counterexamples given by \fullref{ex:heyting_conditional_ideal/pentagon} and \fullref{ex:heyting_conditional_ideal/diamond}.

  After all of that, we will consider classical logic in the proof of \fullref{thm:lindenbaum_tarski_algebras/classical}.

  \SubProofOf{thm:lindenbaum_tarski_algebras/verum} The rule \ref{inf:def:propositional_natural_deduction_systems/top/intro} allows us to construct a proof of \( \syntop \) from \( \Gamma \cup \set{ \varphi } \) from any formula \( \varphi \), hence \( \varphi \leq_\Gamma \top \). Then \( \pi(\syntop) \) is the unique top element.

  \SubProofOf{thm:lindenbaum_tarski_algebras/falsum} The rule \ref{inf:def:propositional_natural_deduction_systems/bot/efq} allows us to construct a proof of any formula \( \varphi \) from \( \synbot \). Then \( \pi(\synbot) \) is the unique bottom element.

  \SubProofOf{thm:lindenbaum_tarski_algebras/negation} The rule \ref{inf:def:propositional_natural_deduction_systems/neg/intro} allows us to construct a proof of \( \synneg \varphi \) from \( \varphi \) for any formula \( \varphi \).

  \SubProofOf{thm:lindenbaum_tarski_algebras/disjunction} Suppose that \( \theta \) is an upper bound of \( \varphi \) and \( \psi \). Then \( \Gamma, \varphi \vdash \theta \) and \( \Gamma, \varphi \vdash \theta \), and \ref{inf:def:propositional_natural_deduction_systems/or/elim} allows us to construct a proof of \( \theta \) from \( \Gamma \cup \set{ \varphi \synvee \psi } \), hence \( \varphi \synvee \psi \) is the least upper bound of \( \varphi \) and \( \psi \).

  \SubProofOf{thm:lindenbaum_tarski_algebras/conjunction} Suppose that \( \theta \) is a lower bound of \( \varphi \) and \( \psi \). Then \( \Gamma, \theta \vdash \varphi \) and \( \Gamma, \theta \vdash \psi \), and \ref{inf:def:propositional_natural_deduction_systems/and/intro} allows us to construct a proof of \( \varphi \synwedge \psi \) from \( \Gamma \cup \set{ \theta } \), hence \( \varphi \synwedge \psi \) is the greatest lower bound of \( \varphi \) and \( \psi \).

  \SubProofOf{thm:lindenbaum_tarski_algebras/conditional}

  \SubProof*{Proof that \( \varphi \synwedge (\varphi \synimplies \psi) \leq_\Gamma \psi \)}

  We have the following proof tree, which we will denote via \( P \):
  \begin{equation*}
    \begin{prooftree}
      \hypo{ [\varphi \synwedge (\varphi \synimplies \psi)]^u }
      \infer1[\ref{inf:def:propositional_natural_deduction_systems/and/elim_left}]{ \varphi \synimplies \psi }

      \hypo{ [\varphi \synwedge (\varphi \synimplies \psi)]^u }
      \infer1[\ref{inf:def:propositional_natural_deduction_systems/and/elim_right}]{ \varphi }

      \infer2[\ref{inf:def:propositional_natural_deduction_systems/imp/elim}]{ \psi }
    \end{prooftree}
  \end{equation*}

  \SubProof*{Proof that \( \theta \leq_\Gamma \varphi \synimplies \psi \) if and only if \( \varphi \synwedge \theta \leq_\Gamma \psi \)}

  First, let \( T \) be a proof tree deriving \( \varphi \synimplies \psi \) from \( \Gamma \cup \set{ \theta } \). The following tree derives \( \psi \) from \( \Gamma \cup \set{ \varphi \synwedge \theta } \):
  \begin{equation*}
    \begin{prooftree}
      \hypo{ [\varphi \synwedge \theta]^v }
      \infer1[\ref{inf:def:propositional_natural_deduction_systems/and/elim_left}]{ \theta }

      \ellipsis {\( T \)} {\varphi \synimplies \psi}

      \hypo{ [\varphi \synwedge \theta]^v }
      \infer1[\ref{inf:def:propositional_natural_deduction_systems/and/elim_right}]{ \varphi }

      \infer2[\ref{inf:def:propositional_natural_deduction_systems/imp/elim}]{ \psi }
    \end{prooftree}
  \end{equation*}

  Conversely, let \( G \) be a tree that derives \( \psi \) from \( \Gamma \cup \set{ \varphi \synwedge \theta } \). The following tree derives \( \varphi \synimplies \psi \) from \( \theta \):
  \begin{equation*}
    \begin{prooftree}
      \hypo{ [\varphi]^w }
      \hypo{ [\theta]^x }
      \infer1[\ref{inf:def:propositional_natural_deduction_systems/and/intro}]{ \varphi \synwedge \theta }

      \ellipsis {\( G \)} {\psi}

      \infer[left label=\( w \)]2[\ref{inf:def:propositional_natural_deduction_systems/imp/intro}]{ \varphi \synimplies \psi }
    \end{prooftree}
  \end{equation*}

  \SubProofOf{thm:lindenbaum_tarski_algebras/biconditional} We will first show that
  \begin{equation*}
    \pi\parens[\Big]{ \varphi \syniff \psi } = \pi\parens[\Big]{ (\varphi \synimplies \psi) \synwedge (\psi \synimplies \varphi) }.
  \end{equation*}

  Indeed, we have \eqref{eq:thm:propositional_derivations/biconditional_as_conjunction} in one direction and \eqref{eq:thm:propositional_derivations/conjunction_of_conditionals} in the other.

  \Fullref{thm:lindenbaum_tarski_algebras/conjunction} and \fullref{thm:lindenbaum_tarski_algebras/conditional} then imply that
  \begin{align*}
    \pi\parens[\Big]{ \varphi \syniff \psi }
    &=
    \pi\parens[\Big]{ (\varphi \synimplies \psi) \synwedge (\psi \synimplies \varphi) }
    \reloset {\ref{thm:lindenbaum_tarski_algebras/conjunction}} = \\ &=
    \pi(\varphi \synimplies \psi) \wedge \pi(\psi \synimplies \varphi)
    \reloset {\ref{thm:lindenbaum_tarski_algebras/conditional}} = \\ &=
    \parens[\Big]{ \pi(\varphi) \rightarrow \pi(\psi) } \wedge \parens[\Big]{ \pi(\psi) \rightarrow \pi(\varphi) },
  \end{align*}
  which by definition equals \( \pi(\varphi) \leftrightarrow \pi(\psi) \).

  \SubProofOf{thm:lindenbaum_tarski_algebras/classical} Finally, now that we have shown that \( L \) is a Heyting algebra, we must show that it is also a Boolean algebra when we consider the additional rule \ref{inf:def:propositional_natural_deduction_systems/bot/dne}.

  According to \fullref{thm:boolean_algebra_heyting_characterization}, we only need to derive \( \varphi \) from \( \synneg \synneg \varphi \); but this is precisely what \ref{inf:def:propositional_natural_deduction_systems/bot/dne} allows us to do.
\end{proof}

\begin{theorem}[Intuitionistic propositional completeness]\label{thm:intuitionistic_propositional_completeness}
  For any set \( \Gamma \) of \hyperref[def:propositional_syntax/formula]{propositional formulas}, there exists a \hyperref[def:heyting_algebra]{Heyting algebra} with respect to which \( \Gamma \vDash \varphi \) whenever \( \Gamma \vdash \varphi \) in the \hyperref[def:propositional_natural_deduction_systems]{intuitionistic natural deduction system}.
\end{theorem}
\begin{comments}
  \item \Fullref{thm:inconsistent_lindenbaum_tarski_algebra} implies that we cannot get a much nicer result at this level of generality.
\end{comments}
\begin{proof}
  Denote by \( L \) the \hyperref[def:lindenbaum_tarski_algebra]{Lindenbaum-Tarski algebra} of \( \Gamma \). \Fullref{thm:lindenbaum_tarski_algebras} implies that \( L \) is a Heyting algebra.

  Consider the \hyperref[def:propositional_valuation/interpretation]{interpretation} \( I: \op*{Prop} \to L \) obtained as a restriction of \( \pi \) to propositional variables. As a consequence of \fullref{thm:lindenbaum_tarski_algebras}, the corresponding \hyperref[def:propositional_valuation/formula_valuation]{valuation} is then simply the Lindenbaum-Tarski projection \( \pi: \op*{Form} \to L \).

  Suppose that there exists a proof tree \( P \) deriving \( \varphi \) from \( \Gamma \). Let \( \psi_1, \ldots, \psi_n \) be an enumeration of the open premises of \( P \) (\fullref{thm:propositional_natural_deduction_entailment_compact} implies they are finitely many).

  \Fullref{thm:syntactic_propositional_conjunction_of_premises} implies that \( \psi_1 \wedge \cdots \wedge \psi_n \vdash \varphi \). \Fullref{thm:lindenbaum_tarski_algebras/conjunction} implies that
  \begin{equation*}
    \Bracks{\psi_1}_I \wedge \cdots \wedge \Bracks{\psi_n}_I
    =
    \pi(\psi_1) \wedge \cdots \wedge \pi(\psi_n)
    =
    \pi(\psi_1 \synwedge \cdots \synwedge \psi_n)
    \leq_\Gamma
    \pi(\varphi)
    =
    \Bracks{\varphi}_I.
  \end{equation*}

  \Fullref{thm:def:heyting_algebra/leq} implies that \( \psi_1 \synwedge \cdots \synwedge \psi_n \vDash \varphi \). \Fullref{thm:semantic_propositional_conjunction_of_premises} and \eqref{eq:def:consequence_relation/monotonicity} then imply that \( \Gamma \vDash \varphi \).
\end{proof}

\paragraph{Complete theories}

\begin{definition}\label{def:complete_set_of_sentences}
  Fix a \hyperref[def:consequence_operator]{consequence operator} \( \op{Cn} \) on a set of sentences \( S \), and let \( {\vdash} \) be the corresponding \hyperref[def:consequence_relation]{consequence relation}. We say that the set of sentences \( \Gamma \) is \term[ru=полное (множество формул) (\cite[def. 1.3.16]{Герасимов2011Вычислимость})]{complete} with respect to them if any of the following equivalent conditions hold:
  \begin{thmenum}
    \thmitem{def:complete_set_of_sentences/operator}\mcite[def. III.4]{Woodger1983TarskiPapers} Every \hyperref[def:consistent_set_of_sentences]{consistent} superset of \( \Gamma \) is \hyperref[def:logical_theory/equivalent]{equivalent} to \( \Gamma \).

    \thmitem{def:complete_set_of_sentences/relation} For any formula \( \varphi \) for which \( \Gamma \cup \set{ \varphi } \) is consistent, we have \( \Gamma \vdash \varphi \).
  \end{thmenum}
\end{definition}
\begin{proof}
  \ImplicationSubProof{def:complete_set_of_sentences/operator}{def:complete_set_of_sentences/relation} Suppose that every consistent superset of \( \Gamma \) is equivalent to \( \Gamma \). Fix a formula \( \varphi \) for which \( \Gamma \cup \set{ \varphi } \) is consistent. From \eqref{eq:def:consequence_relation/reflexivity} it follows that \( \Gamma, \varphi \vdash \varphi \) and, since \( \Gamma \cup \set{ \varphi } \) is a consistent superset of \( \Gamma \), we have \( \Gamma \vdash \varphi \).

  \ImplicationSubProof{def:complete_set_of_sentences/relation}{def:complete_set_of_sentences/operator} Conversely, suppose that, whenever \( \Gamma \cup \set{ \varphi } \) is consistent, we have \( \Gamma \vdash \varphi \).

  Let \( \Delta \) is a consistent theory extending \( \Gamma \). For every formula \( \varphi \) in \( \Delta \), by \fullref{thm:def:consistent_set_of_sentences/subset}, the theory \( \Gamma \cup \set{ \varphi } \) is consistent, and thus \( \Gamma \vdash \varphi \).

  Due to \eqref{eq:def:consequence_relation/transitivity}, every formula derivable from \( \Delta \) is also derivable from \( \Gamma \), and thus \( \Gamma \) and \( \Delta \) are equivalent.
\end{proof}

\begin{proposition}\label{thm:def:complete_set_of_sentences}
  \hyperref[def:complete_set_of_sentences]{Complete sets of sentences} have the following basic properties:
  \begin{thmenum}
    \thmitem{thm:def:complete_set_of_sentences/inconsistent} An \hyperref[def:consistent_set_of_sentences]{inconsistent set} is vacuously complete.
    \thmitem{thm:def:complete_set_of_sentences/superset} Every superset of a complete set is complete.
    \thmitem{thm:def:complete_set_of_sentences/subset} Every subset of an incomplete set is incomplete.
  \end{thmenum}
\end{proposition}
\begin{proof}
  \SubProofOf{thm:def:complete_set_of_sentences/inconsistent} Fix an inconsistent set \( \Gamma \). \Fullref{thm:def:consistent_set_of_sentences/superset} implies that, for every formula \( \varphi \), the set \( \Gamma \cup \set{ \varphi } \) is also inconsistent. Then \fullref{def:complete_set_of_sentences/relation} holds vacuously.

  \SubProofOf{thm:def:complete_set_of_sentences/superset} Let \( \Gamma \) be a complete set and fix a superset \( \Delta \) of \( \Gamma \).

  Suppose that \( \Delta \cup \set{ \varphi } \) is consistent for some formula \( \varphi \). Then \( \Gamma \cup \set{ \varphi } \) is also consistent as a consequence of \fullref{thm:def:consistent_set_of_sentences/subset}. Since \( \Gamma \) is complete, we have \( \Gamma \vdash \varphi \), and, by \eqref{eq:def:consequence_relation/monotonicity}, \( \Delta \vdash \varphi \).

  Generalizing on \( \varphi \), we conclude that \( \Delta \) is complete.

  \SubProofOf{thm:def:complete_set_of_sentences/subset} Let \( \Gamma \) be an incomplete set and fix a subset \( \Delta \) of \( \Gamma \). If \( \Delta \) was complete, by \fullref{thm:def:complete_set_of_sentences/superset}, \( \Gamma \) would be complete, which is a contradiction. It remains for \( \Delta \) to be incomplete.
\end{proof}

\begin{proposition}\label{thm:propositional_complete_set}
  With respect to the \hyperref[def:propositional_natural_deduction_systems]{classical natural deduction system}, the set \( \Gamma \) of \hyperref[def:propositional_syntax/formula]{propositional formulas} is \hyperref[def:complete_set_of_sentences]{complete} if and only if, for any formula \( \varphi \), either \( \Gamma \vdash \varphi \) or \( \Gamma \vdash \synneg \varphi \) (or possibly both, if \( \Gamma \) is \hyperref[def:consistent_set_of_sentences]{inconsistent}).
\end{proposition}
\begin{comments}
  \item In intuitionistic logic, without double negation elimination, we can instead only conclude \( \Gamma \vdash \synneg \synneg \varphi \) from \( \Gamma \not\vdash \neg \varphi \) and \( \Gamma \not\vdash \varphi \).
\end{comments}
\begin{proof}
  \SufficiencySubProof Suppose that \( \Gamma \) is complete in the sense of \fullref{def:complete_set_of_sentences/relation}. Fix an arbitrary formula \( \varphi \).

  \begin{itemize}
    \item Suppose that \( \Gamma \cup \set{ \varphi } \) is inconsistent. \Fullref{thm:propositional_consistent_set/bot} implies that there exists a proof \( P \) deriving \( \synbot \) from \( \Gamma \cup \set{ \varphi } \).

    \begin{itemize}
      \item If \( \varphi \) is an open assumption in \( P \) with marker \( u \), the following is a proof tree deriving \( \synneg \varphi \) from \( \Gamma \):
      \begin{equation*}
        \begin{prooftree}
          \hypo{ [\varphi]^u }
          \ellipsis { \( P \) } { \synbot }
          \infer1[\ref{inf:def:propositional_natural_deduction_systems/neg/intro}]{ \synneg \varphi }.
        \end{prooftree}
      \end{equation*}

      \item Otherwise, \( P \) is a proof of \( \synbot \) from \( \Gamma \), and \fullref{thm:propositional_consistent_set/bot} implies that \( \Gamma \) is inconsistent. \Fullref{thm:propositional_consistent_set/contradiction} then implies that both \( \varphi \) and \( \synneg \varphi \) are derivable from \( \Gamma \).
    \end{itemize}

    \item Otherwise, \( \Gamma \cup \set{ \varphi } \) is consistent, and, since \( \Gamma \) is complete, \( \Gamma \vdash \varphi \).
  \end{itemize}

  \NecessitySubProof Suppose that, for any formula \( \varphi \), either \( \Gamma \vdash \varphi \) or \( \Gamma \vdash \synneg \varphi \) or both.

  Fix a formula \( \varphi \) for which \( \Gamma \cup \set{ \varphi } \) is consistent. \Fullref{thm:def:consistent_set_of_sentences/subset} then implies that \( \Gamma \) is consistent.

  Clearly \eqref{eq:def:consequence_relation/reflexivity} implies that \( \Gamma, \varphi \vdash \varphi \).

  Aiming at a contradiction, suppose that \( \Gamma \not\vdash \varphi \). Then, by assumption, we have \( \Gamma \vdash \synneg \varphi \), and \eqref{eq:def:consequence_relation/monotonicity} implies that \( \Gamma, \varphi \vdash \synneg \varphi \). Then \( \Gamma \cup \set{ \varphi } \) must be inconsistent, but we have concluded that \( \Gamma \) is consistent.

  The obtained contradiction implies that \( \Gamma \vdash \varphi \).

  Generalizing on \( \varphi \), we conclude that \( \Gamma \) satisfies \fullref{def:complete_set_of_sentences/relation}.
\end{proof}

\begin{proposition}\label{thm:lindenbaum_tarski_theories}
  With respect to the \hyperref[def:propositional_natural_deduction_systems]{minimal natural deduction system}, consider the \hyperref[def:lindenbaum_tarski_algebra]{Lindenbaum-Tarski algebra} \( L \) of an empty set of premises.

  \begin{thmenum}
    \thmitem{thm:lindenbaum_tarski_theories/filter} For every \hyperref[def:logical_theory]{theory} \( \Delta \), the projection \( \pi[\Delta] \) is a \hyperref[def:lattice_ideal]{filter} in \( L \).

    \thmitem{thm:lindenbaum_tarski_theories/consistent} The theory \( \Delta \) of formulas is \hyperref[def:consistent_set_of_sentences]{inconsistent} if and only if \( \pi[\Delta] = L \).

    \thmitem{thm:lindenbaum_tarski_theories/complete} In the natural deduction system is instead classical, the theory \( \Delta \) is \hyperref[def:complete_set_of_sentences]{complete} if and only if \( \pi[\Delta] \) is an \hyperref[def:ultrafilter]{ultrafilter}.
  \end{thmenum}
\end{proposition}
\begin{proof}
  \SubProofOf{thm:lindenbaum_tarski_theories/filter} We will use the characterization \fullref{thm:def:lattice_ideal/directed_and_closed}.

  \SubProofOf*[def:directed_set]{directed from belowness} Fix two formula \( \varphi \) and \( \psi \) in \( \Delta \). Since \( \Delta \) is a theory, it is closed under entailment, hence \( \varphi \synwedge \psi \) is in \( \Delta \). \Fullref{thm:lindenbaum_tarski_algebras/conjunction} implies that the equivalence class \( \pi(\varphi \synwedge \psi) \) is the meet of \( \pi(\varphi) \) and \( \pi(\psi) \), hence \( \pi(\varphi \synwedge \psi) \) is in \( \pi[\Delta] \).

  Then \( \pi[\Delta] \) is closed with respect to meets. In particular, every two elements of \( \pi[\Delta] \) have a lower bound.

  \SubProofOf*[def:directed_set]{upward closure} Let \( \varphi \) be a formula in \( \Delta \), and suppose that \( \varphi \vdash \psi \). Then \eqref{eq:def:consequence_relation/monotonicity} implies that \( \Delta \vdash \psi \), and, since it is a theory, we have \( \psi \in \Gamma \). Hence, \( \pi(\psi) \) is in \( \pi[\Delta] \).

  Then \( \pi[\Delta] \) is upward closed.

  \SubProofOf{thm:lindenbaum_tarski_theories/consistent}

  \SufficiencySubProof* Suppose that \( \Delta \) is inconsistent. Then it can derive any propositional formula, and, since it is a theory, it also contains every propositional formula. Naturally, \( \pi[\Delta] = L \).

  \NecessitySubProof* Suppose that \( \pi[\Delta] = L \). Fix any formula \( \varphi \). The projection \( \pi(\varphi) \) is in \( \pi[\Delta] \), implying that there is a formula \( \psi \) derivable from \( \Delta \) in the preimage of \( \pi(\varphi) \). Then \eqref{eq:def:consequence_relation/transitivity} implies that \( \varphi \) is also derivable from \( \Delta \).

  \SubProofOf{thm:lindenbaum_tarski_theories/complete} \Fullref{thm:propositional_complete_set} corresponds precisely to \fullref{def:ultrafilter/direct}.
\end{proof}

\begin{lemma}[Lindenbaum's lemma]\label{thm:extension_to_complete_consistent_set}
  Within the \hyperref[def:abstract_natural_deduction_system]{classical propositional natural deduction system}, every \hyperref[def:consistent_set_of_sentences]{consistent} set of sentences can be extended to a \hyperref[def:complete_set_of_sentences]{complete} consistent set.
\end{lemma}
\begin{comments}
  \item The name of the lemma is taken from \incite[lemma 1.3.12]{Герасимов2011Вычислимость}, but the proof is simplified.
\end{comments}
\begin{proof}
  Denote by \( L \) the \hyperref[thm:lindenbaum_tarski_algebras]{classical Lindenbaum-Tarski algebra} with no premises.

  Let \( \Gamma \) be a consistent set of formulas, and let \( \Delta \) be the theory axiomatized by \( \Gamma \). \Fullref{thm:lindenbaum_tarski_theories/consistent} implies that \( \pi[\Delta] \) is a proper filter of \( L \). \Fullref{thm:ultrafilter_lemma} implies that there exists an ultrafilter \( M \) containing \( \pi[\Delta] \).

  Let \( \Epsilon \) be the union of \( M \), that is, the set of all formulas \( \varphi \) for which \( \pi(\varphi) \) is in \( M \). We have \( \pi[\Epsilon] = M \).

  Then \( \Epsilon \) is a consistent due to \fullref{thm:lindenbaum_tarski_theories/consistent}, and complete due to \fullref{thm:lindenbaum_tarski_theories/complete}.
\end{proof}

\begin{proposition}\label{thm:consistent_implies_satisfiable_interpretation}\mcite[thm. 1.3.19]{Герасимов2011Вычислимость}
  With respect to the \hyperref[def:abstract_natural_deduction_system]{classical propositional natural deduction system}, for every \hyperref[def:complete_set_of_sentences]{complete} \hyperref[def:consistent_set_of_sentences]{consistent} set of formulas \( \Gamma \), there exists a \hyperref[def:propositional_semantics/classical]{classical} \hyperref[def:propositional_valuation/interpretation]{interpretation} \( I \) such that \( \Gamma \vdash \varphi \) if and only if \( I \vDash \varphi \).
\end{proposition}
\begin{proof}
  Define the interpretation \( I(\upsilon) \) as \( T \) if \( \Gamma \vdash \upsilon \) and \( F \) otherwise.

  We proceed via \fullref{thm:induction_on_rooted_trees} on \( \varphi \).

  \SubProof{Proof when \( \varphi = \top \)} We always have \( \Bracks{\varphi} = T \)
  \SubProof{Proof when \( \varphi = \synbot \)} Then \( \varphi \) is not derivable from \( \Gamma \) because the latter is consistent.
  \SubProof{Proof when \( \varphi \) is a variable} By definition of \( I \), we have \( \Gamma \vdash \varphi \) if and only if \( \Bracks{\varphi}_I = T \).

  \SubProof{Proof when \( \varphi = \synneg \psi \)} Suppose that the inductive hypothesis holds for \( \psi \).

  \SufficiencySubProof* If \( \Gamma \vdash \varphi \), the formula \( \psi \) is not derivable from \( \Gamma \) because the latter is consistent. By the inductive hypothesis, \( \Bracks{\psi}_I = F \), and
  \begin{equation*}
    \Bracks{\synneg \psi}_I = \oline{\Bracks{\psi}} = T.
  \end{equation*}

  \NecessitySubProof* Conversely, if \( \Bracks{\synneg \psi}_I = T \), then \( \Bracks{\psi}_I = F \) and, by the inductive hypothesis, \( \Gamma \not\vdash \psi \). Since \( \Gamma \) is complete, it follows that \( \Gamma \vdash \synneg \psi \).

  \SubProof{Proof when \( \varphi = \psi \synvee \theta \)} Suppose that the inductive hypothesis holds for both \( \psi \) and \( \theta \).

  Note that we can derive \( \synneg (\psi \synvee \theta) \) from \( \synneg \psi \) and \( \synneg \theta \):
  \begin{equation*}
    \begin{prooftree}
      \hypo{ [\psi \synvee \theta]^u }

      \hypo{ [\synneg \psi]^{nv} }
      \hypo{ [\psi]^v }
      \infer2[\ref{inf:def:propositional_natural_deduction_systems/neg/elim}]{ \synbot }

      \hypo{ [\synneg \theta]^{nw} }
      \hypo{ [\theta]^w }
      \infer2[\ref{inf:def:propositional_natural_deduction_systems/neg/elim}]{ \synbot }

      \infer[left label={\( v, w \)}]3[\ref{inf:def:propositional_natural_deduction_systems/or/elim}]{ \synbot }
      \infer[left label=\( u \)]1[\ref{inf:def:propositional_natural_deduction_systems/neg/intro}]{ \synneg (\varphi \synvee \psi) }
    \end{prooftree}
  \end{equation*}

  Since \( \Gamma \) is consistent, we cannot derive both \( \psi \synvee \theta \) and \( \synneg (\psi \synvee \theta) \). Therefore, \( \Gamma \vdash \psi \synvee \theta \) if and only if either \( \Gamma \vdash \psi \) or \( \Gamma \vdash \theta \) or both.

  By the inductive hypothesis, \( \Gamma \vdash \psi \synvee \theta \) if and only if \( \Bracks{\psi}_I = T \) or \( \Bracks{\theta}_I = T \), that is, if
  \begin{equation*}
    T
    =
    \Bracks{\psi}_I \synvee \Bracks{\theta}_I
    =
    \Bracks{\psi \synvee \theta}_I.
  \end{equation*}

  \SubProof{Proof when \( \varphi = \psi \synwedge \theta \)} Suppose that the inductive hypothesis holds for both \( \psi \) and \( \theta \).

  \SufficiencySubProof* If there exists a proof tree \( P \) deriving \( \psi \synwedge \theta \) from \( \Gamma \), then the following is a proof of \( \psi \) from \( \Gamma \):
  \begin{equation*}
    \begin{prooftree}
      \hypo{ \Gamma }
      \ellipsis {\( P \)} { \psi \synwedge \theta }
      \infer1[\ref{inf:def:propositional_natural_deduction_systems/and/elim_right}]{ \psi }
    \end{prooftree}
  \end{equation*}

  We can analogously build a proof of \( \theta \). By the inductive hypothesis, we have \( \Bracks{\psi}_I = \Bracks{\theta}_I = T \), therefore
  \begin{equation*}
    \Bracks{\psi \synwedge \theta}_I
    =
    T \wedge T
    =
    T
  \end{equation*}

  \NecessitySubProof* Conversely, if \( \Bracks{\psi \synwedge \theta}_I = T \), then \( \Bracks{\psi}_I = \Bracks{\theta}_I = T \) and, by the inductive hypothesis, there exist proof trees \( P \) deriving \( \psi \) from \( \Gamma \) and \( R \) deriving \( \theta \) from \( \Gamma \). The following is then a proof of \( \psi \synwedge \theta \) from \( \Gamma \):
  \begin{equation*}
    \begin{prooftree}
      \hypo{ \Gamma }
      \ellipsis {\( P \)} { \psi }

      \hypo{ \Gamma }
      \ellipsis {\( R \)} { \theta }

      \infer2[\ref{inf:def:propositional_natural_deduction_systems/and/intro}]{ \psi \synwedge \theta }
    \end{prooftree}
  \end{equation*}

  \SubProof{Proof when \( \varphi = \psi \synimplies \theta \)} Suppose that the inductive hypothesis holds for both \( \psi \) and \( \theta \).

  \SufficiencySubProof* If \( \Gamma \vdash \psi \synimplies \theta \), \eqref{thm:propositional_derivations/conditional_as_conjunction} implies that \( \Gamma \vdash \synneg(\psi \synwedge \synneg \theta) \).

  We have already proved the inductive hypothesis for negations and conjunctions and, since by assumption it hold for \( \psi \) and \( \theta \), we can conclude that it holds for \( \synneg(\psi \synwedge \synneg \theta) \). Then
  \begin{equation*}
    T
    =
    \Bracks{\synneg(\psi \synwedge \synneg \theta)}_I
    =
    \oline{\Bracks{\psi}_I \wedge \oline{\Bracks{\theta}_I}},
  \end{equation*}
  hence
  \begin{equation*}
    F = \Bracks{\psi}_I \wedge \oline{\Bracks{\theta}_I},
  \end{equation*}
  hence either \( \Bracks{\psi}_I = F \) or \( \Bracks{\theta}_I = T \) or both. Thus,
  \begin{equation*}
    \Bracks{\psi \synimplies \theta}_I
    \reloset {\eqref{eq:thm:classical_equivalences/conditional_as_disjunction}} =
    \Bracks{\synneg \psi \synvee \theta}_I
    =
    \Bracks{\synneg \psi}_I \vee \Bracks{\theta}_I
    =
    T.
  \end{equation*}

  \NecessitySubProof* Suppose that \( \Bracks{\psi \synimplies \theta}_I = T \). From \eqref{eq:thm:classical_equivalences/conditional_as_disjunction} it follows that \( \Bracks{\synneg \psi}_I \vee \Bracks{\theta}_I = T \), thus either \( \Bracks{\psi}_I = F \) or \( \Bracks{\theta}_I = T \) or both.

  \begin{itemize}
    \item If \( \Bracks{\psi}_I = F \), by the inductive hypothesis we have \( \Gamma \not\vdash \psi \). Since \( \Gamma \) is complete, we conclude that \( \Gamma \vdash \synneg \psi \). Then \ref{inf:def:propositional_natural_deduction_systems/or/intro_left} allows us to derive \( \synneg \psi \synvee \theta \) from \( \Gamma \), and \eqref{eq:thm:propositional_derivations/disjunction_as_conditional} implies that \( \Gamma \vdash \psi \synimplies \theta \).

    \item Otherwise, if \( \Bracks{\psi}_I = T \), we must have \( \Bracks{\theta}_I = T \). By the inductive hypothesis, we have \( \Gamma \vdash \theta \). Via \ref{inf:def:propositional_natural_deduction_systems/or/intro_right} we can derive \( \synneg \psi \synvee \theta \) from \( \Gamma \). Then, again, \eqref{eq:thm:propositional_derivations/disjunction_as_conditional} implies that \( \Gamma \vdash \psi \synimplies \theta \).
  \end{itemize}

  \SubProof{Proof when \( \varphi = \psi \syniff \theta \)} Via \eqref{eq:thm:propositional_derivations/biconditional_as_conjunction} and \eqref{eq:thm:propositional_derivations/conjunction_of_conditionals} we can reduce this case to the already proven cases for conjunctions and conditionals.
\end{proof}

\begin{corollary}\label{thm:consistent_implies_satisfiable}
  If a set of sentences is \hyperref[def:consistent_set_of_sentences]{consistent} with respect to the \hyperref[def:abstract_natural_deduction_system]{classical propositional natural deduction system}, it is \hyperref[def:satisfiable_set_of_sentences]{satisfiable} with respect to \hyperref[def:propositional_semantics/classical]{classical semantics}.
\end{corollary}
\begin{proof}
  Let \( \Gamma \) be a \hyperref[def:consistent_set_of_sentences]{consistent} set of formulas. Denote by \( \Delta \) the \hyperref[def:complete_set_of_sentences]{complete} consistent extension obtained from \fullref{thm:extension_to_complete_consistent_set}.

  \Fullref{thm:consistent_implies_satisfiable_interpretation} implies that there exists an interpretation \( I \) such that \( \Delta \vdash \varphi \) if and only if \( I \vDash \varphi \).

  Then \( I \) is a model of \( \Delta \), and hence also of \( \Gamma \).
\end{proof}

\begin{theorem}[Classical propositional completeness]\label{thm:classical_propositional_completeness}\mcite[thm. 1.3.24]{Герасимов2011Вычислимость}
  The \hyperref[def:propositional_natural_deduction_systems]{classical propositional natural deduction system} is \hyperref[def:logical_framework/completeness]{complete} with respect to \hyperref[def:propositional_semantics/classical]{classical semantics}.
\end{theorem}
\begin{proof}
  Suppose that \( \Gamma \vDash \varphi \). Then the set \( \Gamma \cup \set{ \neg \varphi } \) is \hyperref[def:satisfiable_set_of_sentences]{unsatisfiable} because \( \Bracks{\neg \varphi}_I = F \) for every model \( I \) of \( \Gamma \).

  Then \( \Gamma \cup \set{ \neg \varphi } \) must be \hyperref[def:consistent_set_of_sentences]{inconsistent} since otherwise \fullref{thm:consistent_implies_satisfiable} would imply that it is satisfiable.

  Since it is inconsistent, there exists a proof tree deriving \( \bot \) from \( \Gamma \cup \set{ \neg \varphi } \). Then we can apply \ref{inf:def:propositional_natural_deduction_systems/bot/dne} and obtain a proof of \( \varphi \) from \( \Gamma \).
\end{proof}
