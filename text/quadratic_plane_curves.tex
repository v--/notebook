\section{Quadratic plane curves}\label{sec:quadratic_plane_curves}

This subsection is dedicated to \hyperref[def:parametric_curve]{curves} in \( \BbbR^2 \) described by \hyperref[def:polynomial_degree]{quadratic polynomials}. We first formalize the meaning of \hyperref[def:affine_algebraic_set/variety]{algebraic variety} and \hyperref[def:affine_algebraic_set/curve]{algebraic curve}, but do not go into algebraic geometry beyond that. After that, we restrict ourselves to the classical theory of quadratic curves.

\begin{definition}\label{def:quadratic_plane_curve}\mimprovised
  A \term{quadratic plane curve} is the \hyperref[def:root_of_polynomial]{set of zeros} of a bivariate quadratic polynomial in the \hyperref[def:euclidean_plane]{Euclidean plane} \( \BbbR^2 \).

  Those that are \hyperref[def:affine_algebraic_set/curve]{algebraic curves} are precisely the \hyperref[def:ellipse]{ellipses}, \hyperref[def:hyperbola]{hyperbolas} and \hyperref[def:parabola]{parabolas} with their standard equations, as well as the imaginary ellipses described in \cref{ex:imaginary_ellipse}. We call the other quadratic plane curves \term{degenerate}.

  \Fullref{alg:canonization_of_quadratic_plane_curves} describes a procedure for finding an \hyperref[def:affine_coordinate_system]{affine coordinate system} in \( \BbbR^2 \) in which we can easily classify the type of algebraic curve. \Cref{thm:change_of_polynomial_basis} ensures that irreducible polynomials are invariant under affine changes of coordinates.

  \begin{figure}[!ht]
    \hfill
    \hfill
    \includegraphics[align=c]{output/def__conic_section__ellipse}
    \hfill
    \includegraphics[align=c]{output/def__conic_section__hyperbola}
    \hfill
    \includegraphics[align=c]{output/def__conic_section__parabola}
    \hfill
    \caption{An \hyperref[def:ellipse]{ellipse}, \hyperref[def:hyperbola]{hyperbola} and \hyperref[def:parabola]{parabola} defined via their parametric equations. The starting point is highlighted and the direction of the parametric curves is shown.}\label{fig:def:quadratic_plane_curve}
  \end{figure}
\end{definition}

\begin{proposition}\label{thm:quadratic_polynomial_irreducibility}
  For a quadratic polynomial \( f(X, Y) \) in two indeterminates over an \hyperref[def:algebraically_closed_field]{algebraically closed} \hyperref[def:field]{field}, the \hyperref[def:affine_algebraic_set]{affine algebraic set} \( \mscrV(\braket{ f(X, Y) }) \) is an \hyperref[def:affine_algebraic_set/variety]{algebraic curve} if and only if \( f(X, Y) \) is an \hyperref[def:domain_divisibility/irreducible]{irreducible polynomial}.
\end{proposition}
\begin{proof}
  For a field \( \BbbK \), by \cref{thm:polynomial_ring_over_gcd_domain}, \( \BbbK[X, Y] \) is also a GCD domain, and, by \cref{thm:def:gcd_domain/irreducible_is_prime}, every \hyperref[def:domain_divisibility/irreducible]{irreducible polynomial} is \hyperref[def:domain_divisibility/prime]{prime}.

  Thus, if \( f(X, Y) \) is a nonconstant \hyperref[def:domain_divisibility/irreducible]{irreducible polynomial}, then the following \hyperref[def:hasse_diagram]{Hasse diagram} shows how the principal ideal of \( \braket{ f(X, Y) } \) relates to other prime ideals
  \begin{equation*}
    \begin{aligned}
      \includegraphics[page=1]{output/ex__quadratic_curves}
    \end{aligned}
  \end{equation*}

  By \cref{thm:def:krull_dimension/polynomial_ring}, the \hyperref[def:krull_dimension]{Krull dimension} of \( \BbbK[X, Y] \) is \( 2 \). Hence, by \fullref{thm:lattice_theorem_for_ideals}, the ascending sequence of \hyperref[def:ring/quotient]{quotients}
  \begin{equation*}
    \frac {\braket{ f(X, Y) }} {\braket{ f(X, Y) }} \subsetneq \frac {\braket{ X, Y }} {\braket{ f(X, Y) }} \subsetneq \frac {\BbbK[X, Y]} {\braket{ f(X, Y) }}
  \end{equation*}
  is a maximal ascending sequence of prime ideals.

  Therefore, the coordinate ring of \( \mscrV(\braket{ f(X, Y) }) \) is unidimensional, and hence the affine algebraic set itself is an \hyperref[def:affine_algebraic_set/curve]{algebraic curve}.
\end{proof}

\begin{example}\label{ex:real_algebraic_curves}
  Suppose that affine varieties are defined for fields that are not algebraically closed.

  The polynomial \( X^2 + Y^2 \) is irreducible \( \BbbR[X, Y] \) as shown in \cref{thm:axn_byn_irreducible}. \Cref{thm:quadratic_polynomial_irreducibility} implies that the set
  \begin{equation*}
    \mscrV(\braket{ X^2 + Y^2 }) = \set{ (0, 0) }
  \end{equation*}
  is an algebraic curve. It is not a curve intuitively, but it fits the definition.

  To avoid degenerate cases like this one, we restrict the definition of affine variety to algebraically closed fields. In this case, \( X^2 + Y^2 = (X - iY) (X + iY) \) is a reducible polynomial in \( \BbbC[X, Y] \), and hence does not induce an algebraic curve.

  We have described in \cref{rem:real_affine_varieties} our approach to algebraic curves in \( \BbbR^2 \). It still has pathologies --- see \cref{ex:imaginary_ellipse} --- but much less of them.
\end{example}

\begin{definition}\label{def:lame_curve}\mimprovised
  A \term{Lam\'e curve} of degree \( n \) over the \hyperref[def:field]{field} \( \BbbK \) is the \hyperref[def:affine_algebraic_set/curve]{algebraic curve} given by the \hyperref[def:algebraic_equation]{algebraic equation}
  \begin{equation}\label{eq:def:lame_curve}
    \frac {X^n} {a^n} + \frac {Y^n} {b^n} = 1.
  \end{equation}
\end{definition}
\begin{comments}
  \item Our definition is based on \cite[179]{Савелов1960ПлоскиеКривые}, however we generalize the definition from the real numbers to arbitrary fields at the cost of restricting \( n \) an integer. Gabriel Lam\'e himself used these curves to prove a special case of \fullref{thm:fermats_last_theorem}.
\end{comments}
\begin{defproof}
  \Cref{thm:axn_byn_czn_irreducible} implies that the trivariate polynomial
  \begin{equation*}
    \frac {X^n} {a^n} + \frac {Y^n} {b^n} - Z^n
  \end{equation*}
  is irreducible, and \cref{thm:homogeneous_polynomial_constant} implies that the bivariate polynomial from \eqref{eq:def:lame_curve} is irreducible.

  Then \cref{thm:quadratic_polynomial_irreducibility} implies that the corresponding set of solutions is an algebraic curve.
\end{defproof}

\begin{definition}\label{def:fermat_curve}\mimprovised
  A \term{Fermat curve} of degree \( n > 1 \) over the \hyperref[def:field]{field} \( \BbbK \) is the \hyperref[def:lame_curve]{Lam\'e curve}
  \begin{equation}\label{eq:def:fermat_curve}
    X^n + Y^n = 1.
  \end{equation}
\end{definition}
\begin{comments}
  \item For this definition, we generalize the conventional equation of a circle. Fermat curves relate to \fullref{thm:fermats_last_theorem} via \cref{thm:fermat_curve_rational_points_via_fermat_triples}.
\end{comments}

\begin{definition}\label{def:ellipse}
  An \term{ellipse} is, up to a choice of \hyperref[def:affine_coordinate_system]{affine coordinate system}, the \hyperref[def:root_of_polynomial]{set of zeros} of the polynomial
  \begin{equation}\label{eq:def:ellipse/polynomial}
    \frac {X^2} {a^2} + \frac {Y^2} {b^2} - 1
  \end{equation}
  for some real numbers \( a > b > 0 \), which we call the big and small \term{radii}.

  The polynomial is unique in the sense that no two polynomials of the form \eqref{eq:def:hyperbola/polynomial} have the same set of zeros. In particular, the radii are well-defined.

  It is an \hyperref[def:affine_algebraic_set/curve]{algebraic curve} because, by \cref{thm:axn_byn_czn_irreducible} and \cref{thm:homogeneous_polynomial_constant}, the polynomial is irreducible over \( \BbbC \).

  \begin{thmenum}
    \thmitem{def:ellipse/standard_equation} The set of zeros of \eqref{eq:def:ellipse/polynomial} consists of all pairs \( (x, y) \) of real numbers satisfying
    \begin{equation}\label{eq:def:ellipse/standard_equation}
      \frac {x^2} {a^2} + \frac {y^2} {b^2} = 1.
    \end{equation}

    We call \eqref{eq:def:ellipse/standard_equation} the \term{standard equation} of the ellipse.

    \thmitem{def:ellipse/parametric_equation} Ellipses can also be described via the \hyperref[def:parametric_curve]{parametric curve}
    \begin{equation}\label{eq:def:ellipse/parametric_equation}
      \begin{cases}
        x = a \cdot \cos \varphi, \\
        y = b \cdot \sin \varphi,
      \end{cases}
    \end{equation}
    where \( \varphi \in [0, 2\pi) \).
  \end{thmenum}

  \thmitem{def:ellipse/eccentricity} The \term{linear eccentricity} of the ellipse is
  \begin{equation*}
    c \coloneq \sqrt{ a^2 - b^2 },
  \end{equation*}
  and the \term{eccentricity} is
  \begin{equation*}
    e \coloneqq \frac c a.
  \end{equation*}

  The eccentricity of an ellipse is always in the interval \( [0, 1) \).

  \thmitem{def:ellipse/foci} The \term{foci} of the ellipse are the points \( F_1 \coloneqq (-c, 0) \) and \( F_2 \coloneqq (c, 0) \).

  \thmitem{def:ellipse/focal_equation} We can also describe an ellipse as a set of points \( P \) such that
  \begin{equation}\label{eq:def:ellipse/focal_equation}
    \norm{\vect{F_1 P}} + \norm{\vect{F_2 P}} = 2a.
  \end{equation}

  We call \eqref{eq:def:ellipse/focal_equation} the \term{focal equation} of the ellipse. Focal equations have the benefit of not depending on the coordinate system.
\end{definition}
\begin{defproof}
  \ImplicationSubProof{def:ellipse/standard_equation}{def:ellipse/parametric_equation} Suppose that \( (x, y) \) satisfies the standard equation \eqref{eq:def:ellipse/standard_equation}.

  Since \( \abs{x} \leq a \), the ratio \( x / a \) is always between \( -1 \) and \( 1 \). Thus, \( \arccos(x / a) \) is defined, and is a number between \( 0 \) and \( \pi \).

  \begin{itemize}
    \item Define
    \begin{equation*}
      \varphi \coloneqq \begin{cases}
        \arccos(x / a),  &y \geq 0 \\
        -\arccos(x / a), &y < 0.
      \end{cases}
    \end{equation*}

    In both cases,
    \begin{equation*}
      a \cos\varphi = a \cdot x / a = x.
    \end{equation*}

    If \( y \geq 0 \), then, for some \( \varepsilon \in \set{ -1, 1 } \),
    \begin{equation*}
      b \sin\varphi
      =
      b \sin\arccos(x / a)
      \reloset {\eqref{eq:thm:trigonometric_identities/pythagorean_identity}} =
      \varepsilon b \sqrt{ 1 - \cos(x / a)^2 }
      =
      \varepsilon b \sqrt{ 1 - \frac {x^2} {a^2} }
      =
      \varepsilon b \cdot y / b
      =
      \varepsilon y.
    \end{equation*}

    Since \( y \geq 0 \), \( b > 0 \) and \( \sin\varphi > 0 \), it follows that \( \varepsilon = 1 \).

    \item Otherwise,
    \begin{equation*}
      b \sin\varphi
      =
      -b \sin\arccos(x / a)
      =
      \cdots
      =
      \varepsilon y.
    \end{equation*}

    Since \( y < 0 \) and \( \sin\varphi < 0 \), again it follows that \( \varepsilon = 1 \).
  \end{itemize}

  \ImplicationSubProof{def:ellipse/parametric_equation}{def:ellipse/standard_equation} Conversely, if \( (x, y) \) satisfies \eqref{eq:def:ellipse/parametric_equation}, then
  \begin{equation*}
    \frac {x^2} {a^2} + \frac {y^2} {b^2}
    =
    (\cos \varphi)^2 + (\sin \varphi)^2
    \reloset {\eqref{eq:thm:trigonometric_identities/pythagorean_identity}} =
    1.
  \end{equation*}

  \EquivalenceSubProof{def:ellipse/standard_equation}{def:ellipse/focal_equation} On the line through \( F_1 \) and \( F_2 \) with origin their midpoint, \( F_1 \) and \( F_2 \) have coordinates \( (\pm c, 0) \), where \( c \) is half of the distance between \( F_1 \) and \( F_2 \).

  We have an affine coordinate system on some line. Add an orthogonal basis vector so that this extends to a coordinate system for the entire plane. For any point \( P \) with coordinates \( (x, y) \), we then have
  \begin{equation*}
    \norm{\vect{F_1 P}} = \sqrt{(x + c)^2 + y^2}
  \end{equation*}
  and
  \begin{equation*}
    \norm{\vect{F_2 P}} = \sqrt{(x - c)^2 + y^2}.
  \end{equation*}

  \begin{equation*}
    p \coloneqq \frac {\norm{\vect{F_1 P}} + \norm{\vect{F_2 P}}} 2.
  \end{equation*}

  Then
  \begin{equation*}
    \norm{\vect{F_1 P}} = 2p - \norm{\vect{F_2 P}},
  \end{equation*}
  hence
  \begin{equation*}
    \norm{\vect{F_1 P}}^2 = 4p^2 - 4p\norm{\vect{F_2 P}} + \norm{\vect{F_2 P}}^2
  \end{equation*}
  and
  \begin{equation*}
    \norm{\vect{F_2 P}}
    =
    p + \frac {\norm{\vect{F_2 P}}^2 - \norm{\vect{F_1 P}}^2} {4p}
    =
    p + \frac {(x - c)^2 + y^2 - (x + c)^2 - y^2} {4p}
    =
    p + \frac {-2xc - 2xc} {4p}.
  \end{equation*}

  Thus,
  \begin{equation*}
    \sqrt{(x - c)^2 + y^2} = \norm{\vect{F_2 P}} = p - \frac a p x.
  \end{equation*}

  We have
  \begin{equation*}
    \norm{\vect{F_2 P}}^2
    =
    (x - c)^2 + y^2
    =
    x^2 - 2xc + c^2 + y^2.
  \end{equation*}

  Then
  \begin{equation*}
    x^2 - 2xc + c^2 + y^2
    =
    \norm{\vect{F_2 P}}^2
    =
    p^2 - 2 x c + \frac {c^2} {p^2} x^2,
  \end{equation*}
  hence
  \begin{equation*}
    x^2 + y^2 + c^2 = p^2 - \frac {c^2} {p^2} x^2.
  \end{equation*}

  This can be rewritten as
  \begin{equation*}
    \parens*{ 1 - \frac {c^2} {p^2} } x^2 + y^2 = p^2 - c^2
  \end{equation*}
  and, finally,
  \begin{equation*}
    \parens*{ \frac {\cancel{p^2 - c^2}} {p^2} \cdot \frac 1 {\cancel{p^2 - c^2}} } x^2 + \frac {y^2} {p^2 - c^2} = 1.
  \end{equation*}

  Then, in the given coordinate system, this becomes the standard equation of an ellipse \eqref{eq:def:ellipse/standard_equation} with big radii \( p \) and small radii \( \sqrt{p^2 - c^2} \).
\end{defproof}

\begin{proposition}\label{thm:ellipse_is_closed_simple_curve}
  An \hyperref[def:ellipse]{ellipse}, defined via the parametric equations \eqref{eq:def:ellipse/parametric_equation}, is a \hyperref[def:simple_curve]{closed simple curve} (technically, we must allow \( \varphi \) to be \( 2\pi \) for this to be true).
\end{proposition}
\begin{proof}
  We will show that the ellipse is a \hyperref[def:simple_curve]{closed simple curve}. It is obviously closed, hence we must show that the function
  \begin{equation*}
    \varphi \mapsto \parens[\Big]{ a \cos \varphi, b \sin \varphi }
  \end{equation*}
  is injective.

  If
  \begin{equation*}
    \cos \varphi - \cos \psi = 0,
  \end{equation*}
  then \cref{thm:trigonometric_identities/sums} implies that
  \begin{equation*}
    0
    =
    \cos \varphi - \cos \psi
    =
    -2 \sin\parens*{ \frac{\varphi - \psi} 2 } \sin\parens*{ \frac{\varphi + \psi} 2 }.
  \end{equation*}

  Hence, either \( (\varphi - \psi) / 2 \) or \( (\varphi + \psi) / 2 \) is a multiple of \( \pi \).

  Similarly, if
  \begin{equation*}
    \sin \varphi - \sin \psi = 0,
  \end{equation*}
  then \cref{thm:trigonometric_identities/sums} implies that
  \begin{equation*}
    0
    =
    \sin \varphi - \sin \psi
    =
    -2 \sin\parens*{ \frac{\varphi - \psi} 2 } \cos\parens*{ \frac{\varphi + \psi} 2 }.
  \end{equation*}

  Hence, either \( (\varphi - \psi) / 2 \) or \( (\varphi + \psi + \pi) / 2 \) is a multiple of \( \pi \).

  Both \( \varphi + \psi \) and \( \varphi + \psi + \pi \) cannot be multiples of \( 2\pi \),  thus it remains for \( \varphi - \psi \) to be. Since both \( \varphi \) and \( \psi \) are taken from the interval \( [0, 2\pi) \), it follows that \( \varphi = \psi \).
\end{proof}

\begin{example}\label{ex:imaginary_ellipse}
  The only pathological example of a nondegenerate quadratic curve in \( \BbbR^2 \) is the \term{imaginary ellipse} induced by the polynomial
  \begin{equation}\label{eq:ex:imaginary_ellipse}
    \frac {X^2} {a^2} + \frac {Y^2} {b^2} + 1.
  \end{equation}

  It has no real roots, yet it is irreducible in \( \BbbC[X, Y] \) as a consequence of \cref{thm:axn_byn_czn_irreducible} and \cref{thm:homogeneous_polynomial_constant}. Imaginary ellipses naturally occur during canonization of quadratic curves --- see \fullref{alg:canonization_of_quadratic_plane_curves/oval/imaginary_ellipse}.
\end{example}

\begin{definition}\label{def:hyperbola}
  A \term{hyperbola} is, up to a choice of \hyperref[def:affine_coordinate_system]{affine coordinate system}, the \hyperref[def:root_of_polynomial]{set of zeros} of the polynomial
  \begin{equation}\label{eq:def:hyperbola/polynomial}
    \frac {X^2} {a^2} - \frac {Y^2} {b^2} - 1
  \end{equation}
  for some positive real numbers \( a \) and \( b \), which we call the real and imaginary \term{radii}.

  The polynomial is unique in the sense that no two polynomials of the form \eqref{eq:def:hyperbola/polynomial} have the same set of zeros. In particular, the radii are well-defined.

  It is an \hyperref[def:affine_algebraic_set/curve]{algebraic curve} because, by \cref{thm:axn_byn_czn_irreducible} and \cref{thm:homogeneous_polynomial_constant}, the polynomial is irreducible over \( \BbbC \).

  \begin{thmenum}
    \thmitem{def:hyperbola/standard_equation} The set of zeros of \eqref{eq:def:hyperbola/polynomial} consists of all pairs \( (x, y) \) of real numbers satisfying
    \begin{equation}\label{eq:def:hyperbola/standard_equation}
      \frac {x^2} {a^2} - \frac {y^2} {b^2} = 1.
    \end{equation}

    We call \eqref{eq:def:hyperbola/standard_equation} the \term{standard equation} of the hyperbola.

    \thmitem{def:hyperbola/parametric_equation} Hyperbolas can also be described via the pair of \hyperref[def:parametric_curve]{parametric curves}
    \begin{equation}\label{eq:def:hyperbola/parametric_equation}
      \begin{cases}
        x = \pm a \cdot \cosh t, \\
        y = b \cdot \sinh t,
      \end{cases}
    \end{equation}
    where \( t \in \BbbR \).

    \thmitem{def:hyperbola/eccentricity} The \term{linear eccentricity} of the hyperbola is
    \begin{equation*}
      c \coloneq \sqrt{ a^2 + b^2 },
    \end{equation*}
    and the \term{eccentricity} is
    \begin{equation*}
      e \coloneqq \frac c a.
    \end{equation*}

    The eccentricity of a hyperbola is always strictly greater than \( 1 \).

    \thmitem{def:hyperbola/foci} The \term{foci} of the hyperbola are the points \( F_1 \coloneqq (-c, 0) \) and \( F_2 \coloneqq (c, 0) \).

    \thmitem{def:hyperbola/focal_equation} We can also describe a hyperbola as a set of points \( P \) such that
    \begin{equation}\label{eq:def:hyperbola/focal_equation}
      \abs[\Big]{ \norm{\vect{F_1 P}} - \norm{\vect{F_2 P}} } = 2a.
    \end{equation}

    We call \eqref{eq:def:hyperbola/focal_equation} the \term{focal equation} of the hyperbola.
  \end{thmenum}
\end{definition}
\begin{defproof}
  \ImplicationSubProof{def:hyperbola/standard_equation}{def:hyperbola/parametric_equation} Suppose that \( (x, y) \) satisfies the standard equation \eqref{eq:def:hyperbola/standard_equation}.

  Define
  \begin{equation*}
    t \coloneqq \begin{cases}
      \hyperref[eq:thm:hyperbolic_identities/inverse/sinh]{\sinh^{-1}}(y / b), &x \geq 0, \\
      -\sinh^{-1}(y / b),                                                      &x < 0.
    \end{cases}
  \end{equation*}

  In both cases,
  \begin{equation*}
    b \sinh t = b \cdot y / b = y.
  \end{equation*}

  \begin{itemize}
    \item If \( x \geq 0 \), then, for some \( \varepsilon \in \set{ -1, 1 } \),
    \begin{equation*}
      a \cosh\varphi
      =
      a \cosh\sinh^{-1}(y / b)
      \reloset {\eqref{eq:thm:trigonometric_identities/pythagorean_identity}} =
      =
      \varepsilon a \sqrt{ \frac {y^2} {b^2} + 1 }
      =
      \varepsilon a \cdot x / a
      =
      \varepsilon x.
    \end{equation*}

    Since \( x \geq 0 \), \( a > 0 \) and \( \cosh\varphi > 0 \), it follows that \( \varepsilon = 1 \).

    \item Otherwise,
    \begin{equation*}
      a \cosh\varphi
      =
      a \cosh\parens*{ -\sinh^{-1}(y / b) }
      =
      \cdots
      =
      \varepsilon x.
    \end{equation*}

    Since \( x < 0 \), it follows that \( \varepsilon = -1 \).
  \end{itemize}

  \ImplicationSubProof{def:hyperbola/parametric_equation}{def:hyperbola/standard_equation} Conversely, if \( (x, y) \) satisfies \eqref{eq:def:hyperbola/parametric_equation}, then
  \begin{equation*}
    \frac {x^2} {a^2} - \frac {y^2} {b^2}
    =
    \parens[\Big]{ \pm \cosh(t) }^2 - \sinh(t)^2
    \reloset {\eqref{eq:thm:hyperbolic_identities/pythagorean_identity}} =
    1.
  \end{equation*}

  \EquivalenceSubProof{def:hyperbola/standard_equation}{def:hyperbola/focal_equation} The proof is analogous to the proof for ellipses but with
  \begin{equation*}
    p \coloneqq \frac {\norm{\vect{F_1 P}} - \norm{\vect{F_2 P}}} 2.
  \end{equation*}
  rather than
  \begin{equation*}
    p \coloneqq \frac {\norm{\vect{F_1 P}} + \norm{\vect{F_2 P}}} 2.
  \end{equation*}
\end{defproof}

\begin{proposition}\label{thm:hyperbola_is_closed_simple_curve}
  A \hyperref[def:hyperbola]{hyperbola}, defined via the parametric equations \eqref{eq:def:hyperbola/parametric_equation}, is a pair of non-intersecting \hyperref[def:simple_curve]{simple curve}.
\end{proposition}
\begin{proof}
  The inverse of \( \sinh \) is defined for all real numbers; hence the parametric functions are injective and the curves are \hyperref[def:simple_curve]{simple curves}. Furthermore, since \( \cosh \) is strictly positive, the two curves do not intersect.
\end{proof}

\begin{definition}\label{def:parabola}
  A \term{parabola} is, up to a choice of \hyperref[def:affine_coordinate_system]{affine coordinate system}, the \hyperref[def:root_of_polynomial]{set of zeros} of the polynomial
  \begin{equation}\label{eq:def:parabola/polynomial}
    Y^2 - 2p X
  \end{equation}
  for some nonzero real number \( p \), called the \term{parameter} of the parabola.

  The polynomial is unique in the sense that no two polynomials of the form \eqref{eq:def:hyperbola/polynomial} have the same set of zeros. In particular, the parameter is well-defined.

  It is an \hyperref[def:affine_algebraic_set/curve]{algebraic curve} because, by \cref{thm:axz_byy_irreducible} and \cref{thm:homogeneous_polynomial_constant}, the polynomial is irreducible over \( \BbbC \).

  \begin{thmenum}
    \thmitem{def:parabola/standard_equation} The set of zeros of \eqref{eq:def:parabola/polynomial} consists of all pairs \( (x, y) \) of real numbers satisfying
    \begin{equation}\label{eq:def:parabola/standard_equation}
      y^2 = 2px.
    \end{equation}

    We call \eqref{eq:def:parabola/standard_equation} the \term{standard equation} of the parabola.

    \thmitem{def:parabola/parametric_equation} Parabolas can also be described via the pair of \hyperref[def:parametric_curve]{parametric curves}
    \begin{equation}\label{eq:def:parabola/parametric_equation}
      \begin{cases}
        x = t, \\
        y = \pm \sqrt{ 2pt },
      \end{cases}
    \end{equation}
    where \( t \geq 0 \).

    \thmitem{def:parabola/eccentricity} The \term{linear eccentricity} and \term{eccentricity} of parabolas are \( c = e = 1 \).

    \thmitem{def:parabola/focus} The \term{focus} of the parabola is the point \( F \coloneqq (p / 2, 0) \).

    \thmitem{def:parabola/directrix} The \term{directrix} \( d \) of the parabola is the line with \hyperref[def:plane_line_equations/general]{general equation} \( x = -p / 2 \).

    \thmitem{def:parabola/focal_equation} We can also describe a parabola as a set of points \( P \) such that
    \begin{equation}\label{eq:def:parabola/focal_equation}
      \norm{\vect{FP}} = \op{dist}(P, d),
    \end{equation}
    where \( \op{dist}(P, d) \) is the distance from \( P \) to \( d \) in the sense of \cref{def:distance_to_subspace}.

    We call \eqref{eq:def:parabola/focal_equation} the \term{focal equation} of the parabola.
  \end{thmenum}
\end{definition}
\begin{proof}
  \EquivalenceSubProof{def:parabola/standard_equation}{def:parabola/focal_equation} The vector \( v \) with coordinates \( (0, 1) \) is directional for \( d \) as a consequence of \cref{thm:coordinates_of_directional_vector}, and point \( O \) with coordinates \( (-p / 2, 0) \) lies on \( d \).

  The orthogonal projection of the point \( P \) with coordinates \( (x, y) \) onto \( d \) is then
  \begin{align*}
    \pi_d(P)
    &=
    O + \inprod v {\vect{OP}} \cdot v
    = \\ &=
    \parens*{ -\frac p 2, 0 } + \inprod*{ \parens*{ 0, 1 } } { \parens*{ x + \frac p 2, y } } \cdot \parens*{ 0, 1 }
    = \\ &=
    \parens*{ -\frac p 2, 0 } + y \cdot \parens*{ 0, 1 }
    = \\ &=
    \parens*{ -\frac p 2, y }.
  \end{align*}

  Then
  \begin{equation*}
    \op{dist}(P, d)
    =
    \norm{P - \pi_d(P)}
    =
    \norm*{\parens*{ x + \frac p 2, 0 }}
    =
    x + \frac p 2.
  \end{equation*}

  Then
  \begin{equation*}
    \norm{P - F}^2 - \norm{P - \pi_d(P)}^2
    =
    \parens*{ x - \frac p 2 }^2 + y^2 - \parens*{ x + \frac p 2 }^2
    =
    y^2 - 2px.
  \end{equation*}
\end{proof}

\begin{proposition}\label{thm:parabola_is_closed_simple_curve}
  A \hyperref[def:parabola]{parabola}, defined via the parametric equations \eqref{eq:def:parabola/parametric_equation}, is a pair of \hyperref[def:simple_curve]{simple curves} that intersect only at the origin of the coordinate system.
\end{proposition}
\begin{proof}
  Since \( \sqrt {\anon} \) is injective on the nonnegative real numbers, the parametric curves \eqref{eq:def:hyperbola/parametric_equation} are \hyperref[def:simple_curve]{simple curve}. Since \( \sqrt {\anon} \) is also positive on positive real numbers, it follows that the two curves only intersect at the origin.
\end{proof}

\begin{algorithm}\label{alg:canonization_of_quadratic_plane_curves}
  We can use \hyperref[def:rigid_motion]{rigid motions} to convert \hyperref[def:quadratic_plane_curve]{quadratic plane curve} into what we will call its \term{canonical form}, whose set of zeros is either
  \begin{itemize}
    \item an ellipse \eqref{eq:def:ellipse/polynomial} --- case \fullref{alg:canonization_of_quadratic_plane_curves/oval/ellipse}.
    \item a hyperbola \eqref{eq:def:hyperbola/polynomial} --- case \fullref{alg:canonization_of_quadratic_plane_curves/oval/hyperbola}.
    \item a parabola \eqref{eq:def:parabola/polynomial} --- cases \fullref{alg:canonization_of_quadratic_plane_curves/y_parabola} and \fullref{alg:canonization_of_quadratic_plane_curves/x_parabola}.
    \item an imaginary ellipse \eqref{eq:ex:imaginary_ellipse} --- case \fullref{alg:canonization_of_quadratic_plane_curves/oval/imaginary_ellipse}.
    \item a degenerate quadratic curve --- cases \fullref{alg:canonization_of_quadratic_plane_curves/y_parallel_lines},
  \fullref{alg:canonization_of_quadratic_plane_curves/x_parallel_lines} and \fullref{alg:canonization_of_quadratic_plane_curves/oval/lines}.
  \end{itemize}

  Consider the polynomial
  \begin{equation*}
    f(X, Y)
    \coloneqq
    a X^2 + b XY + c Y^2 + d X + e Y + f
    =
    \begin{pmatrix}
      X \\ Y
    \end{pmatrix}^T
    \begin{pmatrix}
      a          & b / 2 \\
      b / 2 & c
    \end{pmatrix}
    \begin{pmatrix}
      X \\ Y
    \end{pmatrix}
    +
    \begin{pmatrix}
      d \\ e
    \end{pmatrix}^T
    \begin{pmatrix}
      X \\ Y
    \end{pmatrix}
    +
    f.
  \end{equation*}

  The matrix
  \begin{equation*}
    \begin{pmatrix}
      a          & b / 2 \\
      b / 2 & c
    \end{pmatrix}
  \end{equation*}
  is symmetric, and \fullref{thm:finite_dimensional_spectral_theorem} ensures that it can be diagonalized via some \hyperref[def:unitary_matrix]{orthogonal matrix} \( P \). Let \( \alpha \) and \( \gamma \) be the eigenvalues corresponding to \( P \). Then
  \begin{equation*}
    f(X, Y)
    =
    \bracks*
      {
        P
        \begin{pmatrix}
          X \\ Y
        \end{pmatrix}
      }^T
    \begin{pmatrix}
      \alpha &       \\
             & \gamma
    \end{pmatrix}
    \bracks*
      {
        P
        \begin{pmatrix}
          X \\ Y
        \end{pmatrix}
      }
    +
    \bracks*
      {
        P
        \begin{pmatrix}
          d \\ e
        \end{pmatrix}
      }
    \bracks*
      {
        P
        \begin{pmatrix}
          X \\ Y
        \end{pmatrix}
      }
    +
    f.
  \end{equation*}

  The matrix \( P \) thus induces a \hyperref[con:change_of_basis]{change of basis}. Denote by \( \xi \) and \( \eta \) the new indeterminates, and by \( \delta \) and \( \varepsilon \) the corresponding images of \( d \) and \( e \). Then
  \begin{equation*}
    f(\xi, \eta)
    =
    \begin{pmatrix}
      \xi \\ \eta
    \end{pmatrix}^T
    \begin{pmatrix}
      \alpha &       \\
             & \gamma
    \end{pmatrix}
    \begin{pmatrix}
      \xi \\ \eta
    \end{pmatrix}
    +
    \begin{pmatrix}
      \delta \\ \varepsilon
    \end{pmatrix}^T
    \begin{pmatrix}
      \xi \\ \eta
    \end{pmatrix}
    +
    f.
  \end{equation*}

  Want to further simplify the expression via \hyperref[def:rigid_motion/translation]{translations}. There are several cases to consider. Note that \( \alpha \) and \( \gamma \) cannot simultaneously be zero.

  \begin{thmenum}
    \thmitem{alg:canonization_of_quadratic_plane_curves/y_parallel_lines} If \( \alpha = \delta = 0 \) and \( \gamma \neq 0 \), we use the translation
    \begin{balign*}
      p\parens*{ \xi, \eta - \frac \varepsilon {2 \gamma} }
      &=
      \gamma \parens*{ \eta - \frac \varepsilon {2 \gamma} }^2 + \varepsilon \parens*{ \eta - \frac \varepsilon {2 \gamma} } + f
      = \\ &=
      \gamma \eta^2 + \parens[\Big]{ 2 \gamma \cdot \frac {-\varepsilon} {2 \gamma} + \varepsilon } \eta + \parens[\Big]{ \frac {\varepsilon^2} {4 \gamma^2} + f }
      =
      \gamma \eta^2 + \parens[\Big]{ \frac {\varepsilon^2} {4 \gamma^2} + f }.
    \end{balign*}

    This polynomial is reducible over \( \BbbC \), hence its set of zeros is not an algebraic curve, but rather a pair of parallel lines in \( \BbbC^2 \) (but possibly no real roots).

    \thmitem{alg:canonization_of_quadratic_plane_curves/x_parallel_lines} If \( \alpha \neq 0 \) and \( \gamma = \varepsilon = 0 \), we reduce this to \fullref{alg:canonization_of_quadratic_plane_curves/y_parallel_lines} by swapping \( \xi \) and \( \eta \).

    \thmitem{alg:canonization_of_quadratic_plane_curves/y_parabola} If \( \alpha = 0 \), and \( \gamma \neq 0 \) and \( \delta \neq 0 \), we use the translation
    \begin{balign*}
      p\parens*{ \xi - \frac {\varepsilon^2} {4 \delta \gamma^2} - \frac f \delta, \eta - \frac \varepsilon {2 \gamma} }
      &=
      \gamma \parens*{ \eta - \frac \varepsilon {2 \gamma} }^2 + \delta \parens[\Big]{ \xi - \frac {\varepsilon^2} {4 \delta \gamma^2} - \frac f \delta } + \varepsilon \parens*{ \eta - \frac \varepsilon {2 \gamma} } + f
      = \\ &=
      \gamma \eta^2 + \delta \parens[\Big]{ \xi - \frac {\varepsilon^2} {4 \delta \gamma^2} - \frac f \delta } + \parens[\Big]{ 2 \gamma \cdot \frac {-\varepsilon} {2 \gamma} + \varepsilon } \eta + \parens[\Big]{ \frac {\varepsilon^2} {4 \gamma^2} + f }
      = \\ &=
      \gamma \eta^2 + \delta \xi.
    \end{balign*}

    We conclude that \( f(X, Y) \) induces a \hyperref[def:parabola]{parabola}.

    \thmitem{alg:canonization_of_quadratic_plane_curves/x_parabola} If \( \alpha \neq 0 \), \( \varepsilon \neq 0 \) and \( \gamma = 0 \), we reduce this to \fullref{alg:canonization_of_quadratic_plane_curves/y_parabola} by swapping \( \xi \) and \( \eta \).

    \thmitem{alg:canonization_of_quadratic_plane_curves/oval} If \( \alpha \neq 0 \) and \( \gamma \neq 0 \), we proceed as follows:
    \begin{balign*}
      &\phantom{{}={}}
      p\parens*{ \xi - \frac \delta {2 \alpha}, \eta - \frac \varepsilon {2 \gamma} }
      = \\ &=
      \alpha \parens*{ \xi - \frac \delta {2 \alpha} }^2 + \gamma \parens*{ \eta - \frac \varepsilon {2 \gamma} }^2 + \delta \parens*{ \xi - \frac \delta {2 \alpha} } + \varepsilon \parens*{ \eta - \frac \varepsilon {2 \gamma} } + f
      = \\ &=
      \alpha \xi^2 + \gamma \eta^2 + \parens[\Big]{ 2 \alpha \cdot \frac {-\delta} {2 \alpha} + \delta } \xi + \parens[\Big]{ 2 \gamma \cdot \frac {-\varepsilon} {2 \gamma} + \varepsilon } \eta + \parens[\Big]{ \frac {\delta^2} {4 \alpha^2} + \frac {\varepsilon^2} {4 \gamma^2} + f }
      = \\ &=
      \alpha \xi^2 + \gamma \eta^2 + \parens[\Big]{ \frac {\delta^2} {4 \alpha^2} + \frac {\varepsilon^2} {4 \gamma^2} + f }.
    \end{balign*}

    Define
    \begin{equation*}
      \varphi \coloneqq \frac {\delta^2} {4 \alpha^2} + \frac {\varepsilon^2} {4 \gamma^2} + f.
    \end{equation*}

    Finally,
    \begin{equation*}
      \alpha \xi^2 + \gamma \eta^2 + \varphi.
    \end{equation*}

    \begin{thmenum}
      \thmitem{alg:canonization_of_quadratic_plane_curves/oval/lines} If \( \varphi = 0 \), then \( f(X, Y) \) is reducible over \( \BbbC \), and its set of zeros is a pair of orthogonal lines in \( \BbbC^2 \) (but possibly no real roots).

      \thmitem{alg:canonization_of_quadratic_plane_curves/oval/imaginary_ellipse} If all coefficients have the same sign, then \( f(X, Y) \) is irreducible, but has no roots --- it is an \hyperref[ex:imaginary_ellipse]{imaginary ellipse}.

      \thmitem{alg:canonization_of_quadratic_plane_curves/oval/hyperbola} If \( \varphi \neq 0 \) and \( \alpha \gamma < 0 \), then \( f(X, Y) \) induces a \hyperref[def:hyperbola]{hyperbola}.

      \thmitem{alg:canonization_of_quadratic_plane_curves/oval/ellipse} If \( \alpha \gamma > 0 \) and \( \alpha \varphi < 0 \), then \( f(X, Y) \) induces an \hyperref[def:ellipse]{ellipse}.
    \end{thmenum}
  \end{thmenum}
\end{algorithm}

\begin{definition}\label{def:disk}
  A \term[bg=кръг,ru=круг]{disk} with \term{center} \( O = (x_0, y_0) \) and \term{radius} \( r > 0 \) is a geometric figure that can be defined equivalently as:

  \begin{thmenum}
    \thmitem{def:disk/sphere} The \hyperref[def:metric_space/ball]{closed ball} \( \cl B(O, r) \) in the metric space \( \BbbR^2 \).

    Consistently with the terminology for balls in metric spaces, the term \enquote{unit disk} refers to the case \( r = 1 \).

    \thmitem{def:disk/standard_equation} The set of points \( (x, y) \) satisfying
    \begin{equation}\label{eq:def:disk/standard_equation}
      (x - x_0)^2 + (y - y_0)^2 \leq r^2.
    \end{equation}
  \end{thmenum}
\end{definition}
\begin{proof}
  Note that the left side of \eqref{eq:def:disk/standard_equation} is the squared norm in \( \BbbR^2 \).
\end{proof}

\begin{definition}\label{def:circle}\mimprovised
  A \term[bg=окръжност,ru=окружность]{circle} with \term{center} \( O = (x_0, y_0) \) and \term{radius} \( r > 0 \) is a geometric figure that can be defined equivalently as:

  \begin{thmenum}[series=def:circle]
    \thmitem{def:circle/disk} The \hyperref[def:topological_boundary_operator]{topological boundary} of the \hyperref[def:disk]{disk} with center \( O \) and radius \( r \).

    \thmitem{def:circle/sphere} The \hyperref[def:metric_space/sphere]{sphere} \( S(O, r) \) in the metric space \( \BbbR^2 \).

    Consistently with the terminology for spheres in metric spaces, the term \enquote{unit circle} refers to the case \( r = 1 \).

    \thmitem{def:circle/ellipse} The \hyperref[def:ellipse]{ellipse} with \hyperref[def:ellipse]{radii} \( a = b = r \) and \hyperref[def:ellipse/foci]{foci} \( F_1 = F_2 = O \).

    The standard equation, also called the \term{central equation}, is conventionally written as
    \begin{equation}\label{eq:def:circle/ellipse/central_equation}
      (x - x_0)^2 + (y - y_0)^2 = r^2.
    \end{equation}

    As a consequence, ellipses have eccentricity \( 0 \).
  \end{thmenum}
\end{definition}
\begin{defproof}
  \EquivalenceSubProof{def:circle/disk}{def:circle/sphere} Spheres are the boundaries of the corresponding balls.
  \EquivalenceSubProof{def:circle/sphere}{def:circle/ellipse} The equivalence of definitions follows from the equivalence of the \hyperref[def:ellipse/focal_equation]{standard equation} and the \hyperref[def:ellipse/focal_equation]{focal equation} of an ellipse.
\end{defproof}

\begin{proposition}\label{thm:circle_diameter}
  The \hyperref[def:metric_space/diameter]{diameter} of a \hyperref[def:circle]{circle} or \hyperref[def:disk]{disk} with radius \( r \) is \( 2r \).
\end{proposition}
\begin{proof}
  The \hyperref[def:metric_space/diameter]{diameter} is twice the radius of the smallest closed ball containing the disk. This ball is the disk itself, hence the diameter of is \( 2r \).
\end{proof}

\begin{definition}\label{def:circumference}
  The term \term{circumference} is used for the perimeter of a disk or the \hyperref[def:arc_length]{arc length} of a circle.
\end{definition}
\begin{defproof}
  The two values are equivalent since circles are boundaries of disks.
\end{defproof}

\begin{proposition}\label{thm:circumference}
  The \hyperref[def:circumference]{circumference} of a \hyperref[def:circle]{circle} or \hyperref[def:disk]{disk} with radius \( r \) is \( 2\pi r \).
\end{proposition}
\begin{proof}
  Follows from \cref{thm:circle_diameter} by noting that \( \pi \) is defined in \cref{def:pi/circle} as the ratio of the circumference by the diameter.
\end{proof}

\begin{proposition}\label{thm:area_of_circle}
  The \hyperref[def:figure_area]{area} of a \hyperref[def:disk]{disk} with radius \( r \) is \( \pi r^2 \).
\end{proposition}
\begin{proof}
  We want to determine the Lebesgue measure of the disk \( D \). This requires integrating
  \begin{equation*}
    \int_D \dl x \dl y.
  \end{equation*}

  The parametric equation \eqref{eq:def:ellipse/parametric_equation} suggests expressing \( D \) via \hyperref[con:polar_coordinate_system]{polar coordinates}. Let \( O = (x_0, y_0) \) be the center of \( D \). Then the Cartesian coordinates of \( D \) are
  \begin{equation*}
    \begin{cases}
      x = x_0 + \rho \cdot \cos \varphi, \\
      y = y_0 + \rho \cdot \sin \varphi,
    \end{cases}
  \end{equation*}
  where \( 0 \leq \rho \leq r \) and \( 0 \leq \varphi < 2\pi \). The Jacobian of this change of variables is
  \begin{equation*}
    \det
    \begin{pmatrix}
      \cos \varphi & -\rho \sin \varphi \\
      \sin \varphi & \rho \cos \varphi
    \end{pmatrix}
    =
    \rho (\cos \varphi)^2 + \rho (\sin \varphi)^2
    =
    \rho.
  \end{equation*}

  Then
  \begin{equation*}
    \int_D \dl x \dl y
    =
    \int_0^{2\pi} \int_0^r \rho \dl \rho \dl \varphi
    =
    \frac {r^2} 2 \int_0^{2\pi} \dl \varphi
    =
    \pi r^2.
  \end{equation*}
\end{proof}
