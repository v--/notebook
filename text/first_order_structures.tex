\section{First-order structures}\label{sec:first_order_structures}

\paragraph{Substructures}

\begin{definition}\label{def:fol_substructure}\mcite[def. 2.3.12]{Hinman2005Logic}
  Consider two \hyperref[def:fol_structure]{first-order structures} \( \mscrX = (X, I) \) and \( \mscrY = (Y, J) \) over the same \hyperref[def:fol_signature]{signature} \( \Sigma \).

  We say that \( \mscrY \) is a \term[ru=подструктура (\cite[114]{ШеньВерещагин2017ЯзыкиИИсчисления})]{substructure} of \( \mscrX \) if, For every \( n \)-ary symbol \( s \) in \( \Sigma \), the interpretation \( J(s) \) is a \hyperref[def:set_valued_map/restriction]{restriction} of \( I(s) \) to \( Y^n \).

  By abuse of notation, we may write \( J = I\restr_Y \) for the interpretation and \( \mscrY = \mscrX\restr_Y \) for the substructure itself.

  \begin{thmenum}
    \thmitem{def:fol_substructure/induces}\mimprovised If \( Y \) is the universe of a substructure of \( \mscrX \), its interpretation is uniquely defined. In this case we say that \( Y \) \term{induces} a substructure of \( \mscrX \).

    A useful condition for induced substructures is \cref{thm:fol_substructure_characterization}.

    \thmitem{def:fol_substructure/inclusion}\mimprovised The inclusion \( \iota: X \to Y \) is a \hyperref[def:fol_homomorphism]{homomorphism}. Furthermore, we show in \cref{thm:fol_inclusion_is_embedding} that it is an \hyperref[def:fol_embedding]{embedding}.

    \thmitem{def:fol_substructure/model}\mimprovised In case both \( \mscrX \) and \( \mscrY \) are both \hyperref[def:fol_semantics/model]{models} of the same set of sentences, we say that \( \mscrY \) is a \term[ru=подмодель (\cite[53]{Мальцев1970АлгебраическиеСистемы})]{submodel} of \( \mscrX \).

    As shown in \cref{thm:fol_equational_theory_models/substructure}, if \( T \) is an \hyperref[def:fol_equational_theory]{equational theory}, a substructure of a model of \( T \) is also a model of \( T \).
  \end{thmenum}
\end{definition}
\begin{comments}
  \item Substructures correspond to \hyperref[def:categorical_subobject]{categorical subobjects}; see \cref{thm:fol_categorical_subobjects}.
\end{comments}

\begin{example}\label{ex:replacing_functional_symbols_via_relations}
  Consider the \hyperref[def:semigroup/theory]{theory of semigroups}. We have a functional symbol \( \cdot \), which we can also represent via the ternary predicate \( p(x, y, z) \), which holds for \( (a, b, c) \) in some structure \( \mscrX = (X, I) \) if and only if \( a \cdot b = c \).

  If we choose to work only with the relation, the signature would not have any function symbols, and every subset of \( X \) would induce a substructure.

  This also introduces a complication, however. We must ensure that the relation represents a function, and this can be done via the axiom
  \begin{equation*}
    \qforall x \qforall y \qExists z p(x, y, z),
  \end{equation*}
  where we have used the unique existence shorthand from \cref{rem:fol_exists_unique_abbreviation}.

  In this setting, a model of the theory of semigroups must satisfy this axiom, and thus it is possible for a substructure not to be a model.

  For example, the negative real numbers are not a semigroup under multiplication because the product of two negative numbers is positive. A functional symbol encodes this requirement into the definition of a substructure. But otherwise we must encode this via formulas, which makes the definition of substructure trivial, but now it is possible for a substructure not to be a model.
\end{example}

\begin{proposition}\label{thm:fol_substructure_characterization}
  Given a structure \( \mscrX = (X, I) \), the subset \( A \subseteq X \) \hyperref[def:fol_substructure/induces]{induces} a substructure of \( \mscrX \) if and only if it is closed under \( I(f) \) for every function symbol \( f \).
\end{proposition}
\begin{proof}
  \SufficiencySubProof By definition.
  \NecessitySubProof Suppose that, for every function symbol \( f \) of arity \( n \), \( I(f)(a_1, \ldots, a_n) \) belongs to \( A \) whenever \( a_1, \ldots, a_n \) do.

  We can define an interpretation \( J(f) \) as \( I(f) \) restricted to \( A^n \). The structure \( (A, J) \) then vacuously satisfies \cref{def:fol_substructure}, i.e. is a substructure of \( \mscrX \).
\end{proof}

\begin{proposition}\label{thm:def:fol_substructure}
  \hyperref[def:fol_substructure]{First-order substructures} have the following basic properties:
  \begin{thmenum}
    \thmitem{thm:def:fol_substructure/homomorphism_image} Fix structures \( \mscrX = (X, I) \) and \( \mscrY = (Y, J) \).

    For every \hyperref[def:fol_homomorphism]{homomorphism} \( h: \mscrX \to \mscrY \) and every substructure \( \mscrX' = (X', I') \) of \( \mscrX \), the set-theoretic image \( h[X'] \) induces a substructure of \( \mscrY \).

    As shown in \cref{thm:fol_equational_theory_models/image}, if \( \mscrX' \) is a model of an \hyperref[def:fol_equational_theory]{equational theory} \( T \), the image \( h[X'] \) is also a model.

    \thmitem{thm:def:fol_substructure/homomorphism_preimage} Similarly, for every homomorphism \( h: \mscrX \to \mscrY \) and every substructure \( \mscrY' = (Y, J') \) \hi{of \( \mscrY \)}, the set-theoretic preimage \( h^{-1}(Y') \) induces a substructure of \( \mscrX \).
  \end{thmenum}
\end{proposition}
\begin{proof}
  \SubProofOf{thm:def:fol_substructure/homomorphism_image} Fix a function symbol \( f \) of arity \( n \).

  For every tuple \( b_1, \ldots, b_n \) in \( h[X'] \), there exists a tuple \( a_1, \ldots, a_n \) of preimages. Thus,
  \begin{equation*}
    I(f)(b_1, \ldots, b_n)
    =
    J(f)(h(a_1), \ldots, h(a_n))
    =
    h(I(f)(a_1, \ldots, a_n)),
  \end{equation*}
  which also belongs to \( h[X'] \).

  Therefore, \( h[X'] \) is closed under function application. \Cref{thm:fol_substructure_characterization} implies that \( h[X'] \) induces a substructure of \( \mscrY \).

  \SubProofOf{thm:def:fol_substructure/homomorphism_preimage} Fix a function symbol \( f \) of arity \( n \).

  For every tuple \( a_1, \ldots, a_n \) in \( h^{-1}(Y') \), we have
  \begin{equation*}
    J(f)(h(a_1), \ldots, h(a_n))
    =
    h(I(f)(a_1, \ldots, a_n)),
  \end{equation*}
  so \( I(f)(a_1, \ldots, a_n) \) also belongs to \( h^{-1}(Y') \).
\end{proof}

\begin{definition}\label{def:fol_isomorphism}
  We say that a \hyperref[def:fol_homomorphism]{first-order homomorphism} \( h: \mscrX \to \mscrY \) is an \term[ru=изоморфизм (\cite[170]{Герасимов2014Вычислимость}), en=isomorphism (\cite[def. 2.3.1]{Hinman2005Logic})]{isomorphism} if any of the following equivalent conditions hold:
  \begin{thmenum}
    \thmitem{def:fol_isomorphism/categorical}\mimprovised \( h \) is a \hyperref[def:morphism_invertibility/isomorphism]{categorical isomorphism} in \( \cat{Mod}(\Sigma) \): there exists an inverse homomorphism \( g \) such that \( g \bincirc h \) is the identity in \( \mscrX \) and \( h \bincirc g \) is the identity in \( \mscrY \).

    \thmitem{def:fol_isomorphism/inverse}\mimprovised \( h \) is a \hyperref[def:function_invertibility/bijective]{bijective} homomorphism and its set-theoretic inverse \( h^{-1} \) is a homomorphism.

    \thmitem{def:fol_isomorphism/direct}\mcite[170]{Герасимов2014Вычислимость} \( h \) is a \hyperref[def:fol_homomorphism/predicates]{strong} \hyperref[def:function_invertibility/bijective]{bijective} homomorphism.

    If the signature has no predicate symbols, all homomorphisms are strong, so it is sufficient for the homomorphism to be bijective.
  \end{thmenum}
\end{definition}
\begin{comments}
  \item \Cref{ex:bijective_order_homomorphism_not_isomorphism} demonstrates how a bijective homomorphism may fail to be an isomorphism.
\end{comments}
\begin{defproof}
  \EquivalenceSubProof{def:fol_isomorphism/categorical}{def:fol_isomorphism/inverse} Follows from \cref{thm:concrete_category_function_invertibility/bijective}.

  \EquivalenceSubProof{def:fol_isomorphism/inverse}{def:fol_isomorphism/direct} If \( h \) is bijective, the inequality in \eqref{eq:def:fol_homomorphism/predicates} for \( h^{-1} \) coincides with the converse inequality for \( h \).
\end{defproof}

\begin{example}\label{ex:bijective_order_homomorphism_not_isomorphism}
  Consider the \hyperref[def:integers]{set of integers} \( \BbbZ \) endowed with two different \hyperref[def:partially_ordered_set]{partial orders}:
  \begin{itemize}
    \item The standard total order \( \leq \) where \( n \leq m \) if there exists a nonnegative integer \( k \) such that \( n + k = m \).
    \item The equality relation \( = \).
  \end{itemize}

  The identity \( \id(x) = x \) is an \hyperref[def:order_function]{order homomorphisms} from \( (\BbbZ, =) \) to \( (\BbbZ, \leq) \). Indeed, for any integers \( n \) and \( m \), \( n = m \) implies \( n \leq m \).

  Furthermore, the identity function is bijective. The inverse of \( \id \), which is again \( \id \), is not however a homomorphism from \( (\BbbZ, \leq) \) to \( (\BbbZ, =) \) since, for example, \( 1 \leq 2 \), but \( 1 \neq 2 \).

  Hence, \( \id: (\BbbZ, =) \to (\BbbZ, \leq) \) is a bijective homomorphism, but not an isomorphism.
\end{example}

\begin{definition}\label{def:fol_embedding}
  We say that the homomorphism \( h: \mscrX \to \mscrY \) is an \term[en=embedding (\cite[def. 2.3.26(ii)]{Hinman2005Logic})]{isomorphic embedding} if it satisfies any of the following equivalent conditions:
  \begin{thmenum}
    \thmitem{def:fol_embedding/direct}\mcite[def. 2.3.26(ii)]{Hinman2005Logic} \( h \) is a strong injective homomorphism.

    If the signature has no predicate symbols, all homomorphisms are strong, so it is sufficient for the homomorphism to be injective.

    \thmitem{def:fol_embedding/substructure}\mcite[120]{VanDalen2004LogicAndStructure} The \hyperref[def:set_valued_map/restriction]{corestriction} of \( h \) to its image \( \mscrY\restr_{h[X]} \) is an \hyperref[def:fol_isomorphism]{isomorphism}.
  \end{thmenum}
\end{definition}
\begin{comments}
  \item \Cref{thm:def:elementary_embedding/isomorphism} implies that every isomorphism is an \hyperref[def:elementary_embedding]{elementary embeddings}, so \( \mscrY\restr_{h[X]} \) is guaranteed to be a model of the same \hyperref[def:fol_theory]{theories} as \( \mscrX \).
\end{comments}
\begin{defproof}
  \ImplicationSubProof{def:fol_embedding/direct}{def:fol_embedding/substructure} Suppose that \( h \) is a strong injective homomorphism.

  Its corestriction \( h' \) to its image \( \mscrY\restr_{h[X]} \) is then a strong bijective homomorphism. It thus satisfies \cref{def:fol_isomorphism/direct}, making \( h' \) an isomorphism.

  \ImplicationSubProof{def:fol_embedding/substructure}{def:fol_embedding/direct} Suppose that the corestriction \( h' \) of \( h \) to \( \mscrY_{h[X]} \) is an isomorphism.

  Then \( h' \) satisfies the equivalent conditions in \cref{def:fol_isomorphism}; in particular, it is a strong bijective homomorphism. Then \( h \) is a strong injective homomorphism.
\end{defproof}

\begin{proposition}\label{thm:fol_inclusion_is_embedding}
  The inclusion \( \iota: Y \to X \) of the \hyperref[def:fol_substructure]{substructure} \( \mscrY = (Y, J) \) into \( \mscrX = (X, I) \) is an \hyperref[def:fol_embedding]{first-order embeddings}.
\end{proposition}
\begin{proof}
  Clearly the inclusion \( \iota: Y \to X \) is injective. Furthermore, it is a strong homomorphism because it vacuously satisfies \eqref{eq:def:fol_homomorphism/functions} and \eqref{eq:def:fol_homomorphism/predicates}.
\end{proof}

\begin{definition}\label{def:fol_fixed_point_substructure}\mimprovised
  We associate with every \hyperref[def:fol_homomorphism]{first-order endomorphism} \( h: \mscrX \to \mscrX \) the \hyperref[def:fol_substructure]{substructure} \( \fix(h) \) of \hyperref[def:function_fixed_point]{fixed points} of \( h \).
\end{definition}
\begin{comments}
  \item As shown in \cref{thm:fol_equational_theory_models/substructure}, if \( \mscrX \) is a model of an \hyperref[def:fol_equational_theory]{equational theory} \( T \), all substructures, including \( \fix(h) \), is also a model of \( T \).
\end{comments}
\begin{defproof}
  Fix a function symbol \( f \) of arity \( n \). If \( a_1, \ldots, a_n \) are fixed points of \( h \), then
  \begin{equation*}
    I(f)(a_1, \ldots, a_n)
    =
    I(f)(h(a_1), \ldots, h(a_n))
    =
    h(I(f)(a_1, \ldots, a_n)).
  \end{equation*}

  Thus, \( \fix(h) \) it closed under applications of \( f \), and \cref{thm:fol_substructure_characterization} implies that it induces a substructure of \( \mscrX \).
\end{defproof}

\paragraph{Generated substructures}

\begin{definition}\label{def:fol_intersection_substructure}\mimprovised
  Fix a \hyperref[def:fol_structure]{first-order structure} \( \mscrX = (X, I) \) and a family \( \seq{ (X_k, I_k) }_{k \in \mscrK} \) of \hyperref[def:fol_substructure]{substructures}.

  If the intersection \( \bigcap_{k \in \mscrK} X_k \) is nonempty, restricting the interpretations gives us a substructure of \( \mscrX \), which we call the \term{intersection substructure}.
\end{definition}
\begin{comments}
  \item In case the substructures are disjoint, their intersection is empty, and we have disallowed structures with empty universes. If we allow empty universes, as per \cref{rem:fol_empty_universe/semantics}, then the condition for the substructures not to be disjoint is unnecessary.

  \item As shown in \cref{thm:fol_equational_theory_models/substructure}, if \( \mscrX \) is a model of an \hyperref[def:fol_equational_theory]{equational theory} \( T \), all substructures, including the intersection, is also a model of \( T \).
\end{comments}
\begin{defproof}
  We must show that \( \bigcap_{k \in \mscrK} X_k \) is indeed the universe of a substructure, which amounts to showing that it is closed under applications the signature functions.

  Let \( f \) be a function symbol of arity \( n \) and let \( a_1, \ldots, a_n \) be members of the intersection. For every index \( k \), the elements \( a_1, \ldots, a_n \) belong to \( X_k \), and hence \( I(f)(a_1, \ldots, a_n) \) belongs to \( X_k \). Generalizing \( k \), we conclude that \( I(f)(a_1, \ldots, a_n) \) belongs to the intersection.
\end{defproof}

\begin{definition}\label{def:fol_generated_substructure}\mcite[prop. 2.3.16]{Hinman2005Logic}
  Let \( \mscrX = (X, I) \) be a \hyperref[def:fol_structure]{first-order structure}. Given a subset \( A \) of \( X \), we define the substructure \( \braket{ A } \) \term[ru=порожденная (\cite[55]{Мальцев1970АлгебраическиеСистемы})]{generated} by \( A \) as the \hyperref[def:fol_intersection_substructure]{intersection substructure} of all substructures whose universes contain \( A \).

  We denote this generated substructure by \( \braket{ A } \).
\end{definition}
\begin{comments}
  \item \Cref{thm:moore_family_closure_operator} implies that the generated substructure is a \hyperref[def:moore_closure_operator]{Moore closure operator} on \( \pow(X) \).

  In particular, as a consequence of \cref{thm:closure_operator_minimality}, \( \braket{ A } \) is the least among all substructure universes containing \( A \).

  \item \Cref{thm:fol_generated_substructure_iteration} provides a useful characterization.

  \item As shown in \cref{thm:fol_equational_theory_models/substructure}, if \( \mscrX \) is a model of an \hyperref[def:fol_equational_theory]{equational theory} \( T \), every generated substructure is also a model of \( T \).
\end{comments}

\begin{proposition}\label{thm:fol_generated_substructure_iteration}
  Fix a \hyperref[def:fol_structure]{structure} \( \mscrX = (X, I) \) over \( \Sigma \). The \hyperref[def:fol_substructure]{substructure} of \( \mscrX \) \hyperref[def:fol_generated_substructure]{generated} by the set \( A \) is the smallest \hyperref[def:function_fixed_point]{fixed point} of the following operator:
  \begin{equation}\label{eq:thm:fol_generated_substructure_iteration/operator}
    \begin{aligned}
      &T: \pow(X) \to \pow(X), \\
      &T(S) \coloneqq S \cup \set{ I(f)(a_1, \ldots, a_{\# f}) \given f \in \op*{Fun}_\Sigma \T{and} a_1, \ldots, a_{\# f} \in S }.
    \end{aligned}
  \end{equation}

  Furthermore, \( \braket{ A } \) is the union of the sequence
  \begin{equation}\label{eq:thm:fol_generated_substructure_iteration/sequence}
    A_k \coloneqq \begin{cases}
      A,          &k = 0, \\
      T(A_{k-1}), &k > 0.
    \end{cases}
  \end{equation}
\end{proposition}
\begin{proof}
  Denote by \( F \) the union of \( A_0, A_1, \ldots \). It is a fixed point of \( T \) by \cref{thm:knaster_tarski_iteration/continuous}. The latter requires verifying that \( T \) is \hyperref[def:scott_continuity]{Scott-continuous}, which can be shown analogously to how it is done in \fullref{thm:recursively_defined_abstract_syntax}.

  It remains to show that \( \braket{ A } = F \). Since \( F \) is closed under \( I(f) \) for any function symbol \( f \), \cref{thm:fol_substructure_characterization} implies that it is a substructure of \( \mscrX \). By definition then \( \braket{ A } \subseteq F \).

  Conversely, an element \( a \) of \( F \) must belong to some \( A_k \). Without loss of generality, suppose that \( k \) is minimal. We will use induction on \( k \) to show that \( a \) belongs to every substructure of \( \mscrX \) containing \( A \).
  \begin{itemize}
    \item If \( k = 0 \), then \( A_0 = A \), so \( a \) clearly also belongs to every substructure containing \( A \).

    \item Otherwise, for some function symbol \( f \) of arity \( n \), \( a = I(f)(a_1, \ldots, a_n) \) for some elements \( a_1, \ldots, a_n \) from \( A_{k-1} \). By the inductive hypothesis, \( a_1, \ldots, a_n \) belong to every substructure containing \( A \); then so does \( a \).
  \end{itemize}

  It follows that \( F \subseteq \braket{ A } \).
\end{proof}

\begin{corollary}\label{thm:generated_substructure_coincides}
  Given a structure \( \mscrX = (X, I) \), the subset \( A \subseteq X \) induces a substructure of \( \mscrX \) if and only if \( \braket{ A } = A \).
\end{corollary}
\begin{proof}
  Restatement of \cref{thm:fol_substructure_characterization} considering that the sequence \eqref{eq:thm:fol_generated_substructure_iteration/sequence} stabilizes immediately.
\end{proof}

\begin{theorem}[Induction on generated substructures]\label{thm:induction_on_generated_substructures}
  Over the \hyperref[def:fol_signature]{signature} \( \Sigma \), fix a \hyperref[def:fol_structure]{first-order structure} \( \mscrX = (X, I) \). Consider a subset \( A \) of \( X \), a formula \( \varphi \) over \( \Sigma \), a variable \( x \) and a variable assignment \( v \) into \( X \).

  Then, in order for the modified assignment \( v_{x \mapsto a} \) to satisfy \( \varphi \) for every \( a \) in the substructure \( \braket{ A } \) \hyperref[def:fol_generated_substructure]{generated} by \( A \), the following are sufficient:
  \begin{thmenum}
    \thmitem{thm:induction_on_generated_substructures/base} The \hyperref[def:fol_semantics/satisfaction]{assignment} \( v_{x \mapsto a} \) satisfies \( \varphi \) for every \( a \) in \( A \).

    \thmitem{thm:induction_on_generated_substructures/step} For every function symbol \( f \) in \( \Sigma \) of arity \( n \), if there exists a tuple \( a_1, \ldots, a_n \) such that \( v_{x \mapsto a_k} \) satisfies \( \varphi \) for every \( k = 1, \ldots, n \), then \( v_{x \mapsto I(f)(x_1, \ldots, x_n)} \) also satisfies \( \varphi \).
  \end{thmenum}
\end{theorem}
\begin{proof}
  Suppose that both \cref{thm:induction_on_generated_substructures/base} and \cref{thm:induction_on_generated_substructures/step} hold.

  Let \( a \) be an arbitrary member of \( \braket{ A } \). We will show that \( v_{x \mapsto a} \) satisfies \( \varphi \).

  We will use the characterization \eqref{eq:thm:fol_generated_substructure_iteration/sequence} of \( \braket{ A } \) from \cref{thm:fol_generated_substructure_iteration}. Let \( i \) be the smallest index such that \( a \) belongs to \( A_i \). We will use induction on \( i \).

  \begin{itemize}
    \item If \( i = 0 \), then \cref{thm:induction_on_generated_substructures/base} ensures that \( v_{x \mapsto a} \) satisfies \( \varphi \).

    \item Otherwise, \( a = I(f)(a_1, \ldots, a_n) \), where the inductive hypothesis holds for \( A_{i-1} \), then \cref{thm:induction_on_generated_substructures/step} is satisfied, and it implies that \( v_{x \mapsto a} \) satisfies \( \varphi \).
  \end{itemize}
\end{proof}

\paragraph{Direct products}

\begin{definition}\label{def:fol_direct_product}\mcite[def. 2.3.54]{Hinman2005Logic}
  Fix a family \( \seq{ \mscrX_k }_{k \in \mscrK} \) of \hyperref[def:fol_structure]{first-order structures} over a common signature \( \Sigma \), where \( \mscrX_k = (X_k, I_k) \) for every \( k \) in \( \mscrK \).

  We call the \term{direct product} of this family the structure \( (X, I) \) denoted by \( \prod_{k \in \mscrK} \mscrX_k \) (or \( \mscrX^\mscrK \) if all structures are equal to \( \mscrX \)), defined as follows:
  \begin{thmenum}
    \thmitem{def:fol_direct_product/universe} The universe \( X \) is the \hyperref[def:cartesian_product]{Cartesian product} \( \prod_{k \in \mscrK} X_k \) of the corresponding universes.

    \thmitem{def:fol_direct_product/functions} For every \( n \)-ary functional symbol \( f \) in \( \Sigma \), we let
    \begin{equation*}
      \mathllap{I(f)\parens[\big]{ \seq{ a_{1,k} }_{k \in \mscrK}, \ldots, \seq{ a_{n,k} }_{k \in \mscrK} }} \coloneqq \mathrlap{\seq[\big]{ I_k(f)(a_{1,k}, \ldots, a_{n,k}) }_{k \in \mscrK}.}
    \end{equation*}

    \thmitem{def:fol_direct_product/predicates} For every \( n \)-ary predicate symbol \( p \) in \( \Sigma \), we define its interpretation componentwise:
    \begin{equation*}
      \mathllap{I(p)\parens[\big]{ \seq{ a_{1,k} }_{k \in \mscrK}, \ldots, \seq{ a_{n,k} }_{k \in \mscrK} }} \coloneqq \mathrlap{\bigwedge_{k \in \mscrK} I_k(p)(a_{1,k}, \ldots, a_{n,k}).}
    \end{equation*}
  \end{thmenum}
\end{definition}
\begin{comments}
  \item As shown in \cref{thm:fol_equational_theory_models/direct_product}, if \( T \) is an \hyperref[def:fol_equational_theory]{equational theory}, the direct product of models of \( T \) is also a model of \( T \).
\end{comments}

\begin{proposition}\label{thm:fol_direct_product_projections}
  The \hyperref[def:cartesian_product_projection]{projections} of the \hyperref[def:fol_direct_product]{direct product} of \hyperref[def:fol_structure]{first-order structures} are \hyperref[def:fol_homomorphism]{homomorphisms}.
\end{proposition}
\begin{proof}
  Consider the product \( \prod_{k \in \mscrK} \mscrX_k \) in the setting of \cref{def:fol_direct_product}.

  \SubProofOf{def:fol_homomorphism/functions} For every \( n \)-ary functional symbol \( f \) in \( \Sigma \), we have
  \begin{align*}
    &\phantom{{}={}}
    \pi_m\parens[\big]{ I(f)\parens[\big]{ \seq{ a_{1,k} }_{k \in \mscrK}, \ldots, \seq{ a_{n,k} }_{k \in \mscrK} } }
    = \\ &=
    \pi_m\parens[\big]{ \seq[\big]{ I_k(f)\parens{ a_{1,k}, \ldots, a_{n,k} } }_{k \in \mscrK} }
    = \\ &=
    I_m(f)\parens{ a_{1,m}, \ldots, a_{n,m} }
    = \\ &=
    I_m(f)\parens[\big]{ \pi_m\parens[\big]{ \seq{ a_{1,k} }_{k \in \mscrK} }, \ldots, \pi_m\parens[\big]{ \seq{ a_{n,k} }_{k \in \mscrK} } }.
  \end{align*}

  \SubProofOf{def:fol_homomorphism/predicates} Now fix an \( n \)-ary predicate symbol \( p \) in \( \Sigma \) and suppose that
  \begin{equation*}
    I(p)\parens[\big]{ \seq{ a_{1,k} }_{k \in \mscrK}, \ldots, \seq{ a_{n,k} }_{k \in \mscrK} } = \semtop.
  \end{equation*}

  Then
  \begin{equation*}
    I_m(p)(a_{1,m}, \ldots, a_{n,m}) = \semtop,
  \end{equation*}
  so
  \begin{equation*}
    I_m(p)\parens[\big]{ \pi_m\parens[\big]{ \seq{ a_{1,k} }_{k \in \mscrK} }, \ldots, \pi_m\parens[\big]{ \seq{ a_{n,k} }_{k \in \mscrK} } }
    =
    I_m(p)(a_{1,m}, \ldots, a_{n,m})
    =
    \semtop.
  \end{equation*}
\end{proof}

\begin{theorem}[Direct product universal property]\label{thm:direct_product_universal_property}
  Consider the \hyperref[def:fol_institution/models/obj]{category of models} \( \cat{Mod}(\Sigma) \) of some signature.

  For a family \( \seq{ \mscrX_k }_{k \in \mscrK} \), the \hyperref[def:fol_direct_product]{direct product} \(  \prod_{k \in \mscrK} \mscrX_k \) is a \hyperref[def:discrete_category_limits]{categorical product} in \( \cat{Mod}(\Sigma) \) --- the unique up to a unique \hyperref[def:fol_isomorphism]{isomorphism} structure that satisfies the following \hyperref[rem:universal_mapping_property]{universal mapping property}:
  \begin{displayquote}
    If we are given a structure \( \mscrY \) and, for every index \( k \) from \( \mscrK \), a homomorphism \( \beta_k: \mscrY \to \mscrX_k \), then there exists a homomorphism \( l_\mscrY: \mscrY \to \prod_{k \in \mscrK} \mscrX_k \) such that the following diagram commutes:
    \begin{equation}\label{eq:thm:direct_product_universal_property}
      \begin{aligned}
        \includegraphics[page=1]{output/thm__direct_product_universal_property}
      \end{aligned}
    \end{equation}
  \end{displayquote}
\end{theorem}
\begin{comments}
  \item As shown in \cref{thm:fol_equational_theory_models/direct_product}, if \( T \) is an \hyperref[def:fol_equational_theory]{equational theory}, the direct product of models of \( T \) is also a model of \( T \).
\end{comments}
\begin{proof}
  Let \( \mscrX_k = (X_k, I_k) \) and \( \mscrY = (Y, J) \). Denote by \( I \) the interpretation in \( \prod_{k \in \mscrK} \mscrX_k \). \Fullref{thm:cartesian_product_universal_property} shows that
  \begin{equation*}
    \begin{aligned}
      &l_\mscrY: Y \to \prod_{k \in \mscrK} X_k \\
      &l_\mscrY(b) \coloneqq \seq{ \beta_k(b) }_{k \in \mscrK}
    \end{aligned}
  \end{equation*}
  is the unique function satisfying the universal property for Cartesian products.

  We must only show that it is a homomorphism.

  \SubProofOf{def:fol_homomorphism/functions} For every \( n \)-ary functional symbol \( f \), we have
  \begin{align*}
    l_\mscrY\parens[\big]{ J(f)(b_1, \ldots, b_n) }
    &=
    \seq[\big]{ \beta_k(J(f)(b_1, \ldots, b_n)) }_{k \in \mscrK}
    = \\ &=
    \seq[\big]{ I_k(f)(\beta_k(b_1), \ldots, \beta_k(b_n)) }_{k \in \mscrK}
    = \\ &=
    I(f)\parens[\big]{ \seq{ \beta_k(b_1) }_{k \in \mscrK}, \ldots, \seq{ \beta_k(b_n) }_{k \in \mscrK} }
    = \\ &=
    I(f)\parens[\big]{ l_\mscrY(b_1), \ldots, l_\mscrY(b_n) }.
  \end{align*}

  \SubProofOf{def:fol_homomorphism/predicates} Fix an \( n \)-ary predicate symbol \( p \). Suppose that
  \begin{equation*}
    J(p)(b_1, \ldots, b_n) = \semtop.
  \end{equation*}

  Then
  \begin{equation*}
    I(p)\parens[\big]{ l_\mscrY(b_1), \ldots, l_\mscrY(b_n) }
    =
    \bigwedge_{k \in \mscrK} I_k(p)\parens[\big]{ \beta_k(b_1), \ldots, \beta_k(b_n) }
    \geq
    J(p)(b_1, \ldots, b_n)
    =
    \semtop.
  \end{equation*}
\end{proof}

\begin{proposition}\label{thm:fol_homomorphism_as_substructure}
  Let \( \mscrX = (X, I) \) and \( \mscrY = (Y, J) \) be \hyperref[def:fol_structure]{structures} over a common \hyperref[def:fol_signature]{signature} \hi{without predicate symbols}. The \hyperref[def:function]{set-theoretic function} \( h: X \to Y \) is a \hyperref[def:fol_homomorphism]{homomorphism} if and only if it induces a \hyperref[def:fol_substructure]{substructure} of the \hyperref[def:fol_direct_product]{direct product} \( \mscrX \times \mscrY \).
\end{proposition}
\begin{proof}
  \SufficiencySubProof Suppose that \( h: X \to Y \) is a homomorphism. We must show that \( h \) is closed under function application in \( \mscrX \times \mscrY \). Denote by \( K \) the interpretation in the structure \( \mscrX \times \mscrY \).

  Let \( f \) be an \( n \)-ary function symbol. Then, for any \( n \)-tuple of pairs \( (a_1, b_1), \ldots, (a_n, b_n) \) from \( h \), we have
  \begin{equation*}
    K(f)\parens[\big]{ (a_1, b_1), \ldots, (a_n, b_n) }
    =
    \parens[\big]{ I(f)\parens[\big]{ a_1, \ldots, a_n }, J(f)\parens[\big]{ \underbrace{b_1}_{h(a_1)}, \ldots, \underbrace{b_n}_{h(a_n)} } }.
  \end{equation*}

  Since \( h \) satisfies \cref{def:fol_homomorphism/functions}, it follows that
  \begin{equation*}
    h\parens[\big]{ I(f)(a_1, \ldots, a_n) } = J(f)\parens[\big]{ h(a_1), \ldots, h(a_n) },
  \end{equation*}
  and thus
  \begin{equation*}
    K(f)\parens[\big]{ (a_1, b_1), \ldots, (a_n, b_n) }
  \end{equation*}
  belongs to \( h \).

  Therefore, due to \cref{thm:fol_substructure_characterization}, \( h \) is a substructure of \( \mscrX \times \mscrY \).

  \NecessitySubProof Conversely, suppose that \( h \) is the universe of a substructure of \( \mscrX \times \mscrY \). Again, denote by \( K \) the interpretation in \( \mscrX \times \mscrY \).

  We must show that \( h \) is a homomorphism. There are no predicate symbols, so we must only verify \cref{def:fol_homomorphism/functions}.

  Fix an \( n \)-ary functional symbol \( f \) and an \( n \)-tuple \( a_1, \ldots, a_n \) from \( X \). The pair \( (a_k, h(a_k)) \) obviously belongs to \( h \) for every \( k = 1, \ldots, n \). Since \( h \) is closed under application of \( I(f) \), it must contain
  \begin{equation*}
    K(f)\parens[\big]{ (a_1, h(a_1)), \ldots, (a_n, h(h_n)) }
    =
    \parens[\big]{ I(f)\parens[\big]{ a_1, \ldots, a_n }, J(f)\parens[\big]{ h(a_1), \ldots, h(a_n) } }.
  \end{equation*}

  Therefore,
  \begin{equation*}
    h\parens[\big]{ I(f)(a_1, \ldots, a_n) } = J(f)\parens[\big]{ h(a_1), \ldots, h(a_n) }.
  \end{equation*}
\end{proof}

\paragraph{Congruences}

\begin{definition}\label{def:fol_congruence}\mimprovised
  Fix a \hyperref[def:fol_structure]{first-order structure} \( \mscrX = (X, I) \) and an \hyperref[def:equivalence_relation]{equivalence relation} \( {\cong} \) on \( X \). We say that \( {\cong} \) is a \term[ru=конгруэнция (\cite[46]{Мальцев1970АлгебраическиеСистемы})]{congruence} on \( \mscrX \) if the following hold:

  \begin{thmenum}
    \thmitem{def:fol_congruence/functions} For any \( n \)-ary functional symbol \( f \), from \( (a_1 \cong b_1), \ldots, (a_n \cong b_n) \) it follows that
    \begin{equation}\label{eq:def:fol_congruence/functions}
      I(f)(a_1, \ldots, a_n) \cong I(f)(b_1, \ldots, b_n).
    \end{equation}

    \thmitem{def:fol_congruence/predicates} For any \( n \)-ary predicate symbol \( p \), from \( (a_1 \cong b_1), \ldots, (a_n \cong b_n) \) it follows that
    \begin{equation}\label{eq:def:fol_congruence/predicates}
      I(p)(a_1, \ldots, a_n) \leq I(p)(b_1, \ldots, b_n).
    \end{equation}
  \end{thmenum}
\end{definition}
\begin{comments}
  \item We base this definition on \cite[46]{Мальцев1970АлгебраическиеСистемы}, where predicate symbols are not considered. The additional condition for predicates in based on our discussion of intuitionistic equality from \cref{rem:intuitionistic_equality}.
\end{comments}

\begin{proposition}\label{thm:fol_congruence_as_substructure}
  Fix a \hyperref[def:fol_structure]{structure} \( \mscrX = (X, I) \) over a \hyperref[def:fol_signature]{signature} \hi{without predicate symbols}. The \hyperref[def:equivalence_relation]{equivalence relation} \( {\cong} \) on \( X \) is a \hyperref[def:fol_congruence]{congruence} in \( \mscrX \) if and only if it is the universe of a \hyperref[def:fol_substructure]{substructure} of the \hyperref[def:fol_direct_product]{direct product} \( \mscrX^2 \).
\end{proposition}
\begin{proof}
  Denote by \( J \) the interpretation in \( \mscrX^2 \). Note that
  \begin{equation*}
    J(f)\parens[\big]{ (a_1, b_1), \ldots, (a_n, b_n) }
    =
    \parens[\big]{ I(f)(a_1, \ldots, a_n), I(f)(b_1, \ldots, b_n) }.
  \end{equation*}

  Therefore, if \( a_1 \cong b_1, \ldots, a_n \cong b_n \), then
  \begin{equation*}
    J(f)\parens[\big]{ (a_1, b_1), \ldots, (a_n, b_n) } \in {\cong}
  \end{equation*}
  if and only if
  \begin{equation*}
    I(f)(a_1, \ldots, a_n) \cong I(f)(b_1, \ldots, b_n).
  \end{equation*}
\end{proof}

\begin{proposition}\label{thm:homomorphism_induces_congruence}
  Every \hyperref[def:fol_homomorphism]{strong first-order homomorphism} \( h: \mscrX \to \mscrY \) induces a \hyperref[def:fol_congruence]{congruence} \( {\cong} \) on its universe \( \mscrX \) defined by putting \( a \cong b \) if \( h(a) = h(b) \).
\end{proposition}
\begin{proof}
  Let \( \mscrX = (X, I) \) and \( \mscrY = (Y, I) \). The relation defined is obviously an equivalence relation on \( X \).

  \SubProofOf{def:fol_congruence/functions} Fix an \( n \)-ary functional symbol \( f \) and \( n \) pairs of congruent elements \( (a_1 \cong b_1), \ldots, (a_n \cong b_n) \). Then
  \begin{equation*}
    \varphi\parens[\big]{ I(f)(a_1, \ldots, a_n) }
    =
    J(f)\parens[\big]{ \underbrace{\varphi(a_1)}_{=\varphi(b_1)}, \ldots, \underbrace{\varphi(a_n)}_{=\varphi(b_n)} }
    =
    \varphi\parens[\big]{ I(f)(b_1, \ldots, b_n) }.
  \end{equation*}

  \SubProofOf{def:fol_congruence/predicates} Similarly, fix an \( n \)-ary predicate symbol \( p \) and \( n \) pairs of congruent elements \( (a_1 \cong b_1), \ldots, (a_n \cong b_n) \). Then
  \begin{equation*}
    I(p)(a_1, \ldots, a_n)
    =
    J(p)\parens[\big]{ \underbrace{\varphi(a_1)}_{=\varphi(b_1)}, \ldots, \underbrace{\varphi(a_n)}_{=\varphi(b_n)} }
    =
    I(p)(b_1, \ldots, b_n).
  \end{equation*}
\end{proof}

\begin{definition}\label{def:fol_generated_congruence}\mimprovised
  Let \( \mscrX = (X, I) \) be a \hyperref[def:fol_structure]{first-order structure} over \( \Sigma \) and let \( {\sim} \) be a binary relation on \( X \).

  The intersection of all \hyperref[def:fol_congruence]{congruences} on \( X \) containing \( {\sim} \) is again congruence. We call it the congruence \term{generated} by \( {\sim} \).
\end{definition}
\begin{comments}
  \item \Cref{thm:moore_family_closure_operator} implies that this is a \hyperref[def:moore_closure_operator]{Moore closure operator} on \( X^2 \).

  In particular, as a consequence of \cref{thm:closure_operator_minimality}, the congruence generated by \( {\sim} \) is the least among all congruences containing \( {\sim} \).
\end{comments}

\begin{proposition}\label{thm:fol_generated_congruence_characterization}
  Let \( \mscrX = (X, I) \) be a \hyperref[def:fol_structure]{first-order structure} over \( \Sigma \) and let \( {\sim} \) be a binary relation on \( X \).

  We can characterize the \hyperref[def:fol_congruence]{congruence} \( {\cong} \) \hyperref[def:fol_generated_congruence]{generated} by \( {\sim} \) via \fullref{thm:recursively_defined_relations}, by extending the equivalence relation rules from \cref{thm:generated_equivalence_relation} with the following:
  \begin{equation*}\taglabel[\ensuremath{ \logic{Cong}_f }]{inf:def:fol_generated_congruence/fun}
    \begin{prooftree}
      \hypo{ f \in \op*{Fun}_\Sigma }
      \hypo{ a_1 \cong b_1 }
      \hypo{ \cdots }
      \hypo{ a_{\#f} \cong b_{\#f} }
      \infer4[\ref{inf:def:fol_generated_congruence/fun}]{ I(f)(a_1, \ldots, a_{\#f}) \cong I(f)(b_1, \ldots, b_{\#f}) }
    \end{prooftree}
  \end{equation*}

  \begin{equation*}\taglabel[\ensuremath{ \logic{Cong}_f }]{inf:def:fol_generated_congruence/pred}
    \begin{prooftree}
      \hypo{ p \in \op*{Pred}_\Sigma }
      \hypo{ a_1 \cong b_1 }
      \hypo{ \cdots }
      \hypo{ a_{\#p} \cong b_{\#p} }
      \hypo{ I(p)(a_1, \ldots, a_{\#p}) = \semtop }
      \infer5[\ref{inf:def:fol_generated_congruence/pred}]{ I(p)(b_1, \ldots, b_{\#p}) = \semtop }
    \end{prooftree}
  \end{equation*}
\end{proposition}
\begin{proof}
  Let \( {\cong} \) be the relation obtained using these rules. Let \( {\approx} \) be any congruence on \( \mscrX \).

  Let \( a \cong b \). We will use \fullref{thm:induction_on_recursively_defined_relations} to show that \( a \approx b \). It will follow that \( {\cong} \) is a subset of every congruence that contains \( {\sim} \), and hence coincides with their intersection, the generated congruence.

  \Cref{thm:generated_equivalence_relation} demonstrates this for the rules presented there, so it remains to show it only for the two rules here.
  \begin{itemize}
    \item If \( a \cong b \) due to \ref{inf:def:fol_generated_congruence/fun}, then there exists an \( n \)-ary function symbol \( f \) and congruent pairs \( (a_1 \cong b_1), \ldots, (a_n \cong b_n) \), for which the inductive hypothesis holds, such that \( a = I(f)(a_1, \ldots, a_n) \) and \( b = I(f)(b_1, \ldots, b_n) \).

    By the inductive hypothesis, \( a_k \approx b_k \) for \( k = 1, \ldots, n \), so \( a \approx b \) because \( {\approx} \) satisfies \cref{def:fol_congruence/functions}.

    \item If \( a \cong b \) due to \ref{inf:def:fol_generated_congruence/pred}, we proceed similarly.
  \end{itemize}
\end{proof}

\paragraph{Quotient structures}

\begin{definition}\label{def:fol_quotient_structure}\mcite[62]{Мальцев1970АлгебраическиеСистемы}
  Consider a \hyperref[def:fol_congruence]{first-order congruence} \( \cong \) on the \hyperref[def:fol_structure]{structure} \( \mscrX = (X, I) \) over the \hyperref[def:fol_signature]{signature} \( \Sigma \).

  As an equivalence relation, \( {\cong} \) allows forming the \hyperref[def:equivalence_relation/quotient]{quotient set} \( X / {\cong} \) with a corresponding \hyperref[def:equivalence_relation/projection]{quotient map} \( \pi: X \to X / {\cong} \).

  We can define an interpretation \( I_\cong \) on \( X / {\cong} \) as follows:
  \begin{thmenum}[series=def:fol_quotient_structure]
    \thmitem{def:fol_quotient_structure/functions} For every \( n \)-ary functional symbol \( f \) in \( \Sigma \), let
    \begin{equation}\label{eq:def:fol_quotient_structure/functions}
      I_\cong(f)\parens[\big]{ \pi(a_1), \ldots, \pi(a_n) } \coloneqq \pi\parens[\big]{ I(f)(a_1, \ldots, a_n) }.
    \end{equation}

    \thmitem{def:fol_quotient_structure/predicates} For every \( n \)-ary predicate symbol \( p \) in \( \Sigma \), let
    \small
    \begin{equation}\label{eq:def:fol_quotient_structure/predicates}
      I_\cong(p)\parens[\big]{ \pi(a_1), \ldots, \pi(a_n) } \coloneqq \bigvee\set[\big]{ I(p)(b_1, \ldots, b_n) \given* b_1 \in \pi(a_1), \ldots, b_n \in \pi(a_n) }
    \end{equation}
    \normalsize
  \end{thmenum}

  Denote the structure \( (X / {\cong}, I_\cong) \) by \( \mscrX / {\cong} \).

  \begin{thmenum}[resume=def:fol_quotient_structure]
    \thmitem{def:fol_quotient_structure/projection}\mimprovised The projection \( \pi: X \to Y \) is a \hyperref[def:fol_homomorphism]{homomorphism}. Furthermore, we show in \cref{thm:fol_quotient_map_is_epimorphism} that it is an \hyperref[def:fol_epimorphism]{epimorphism}.

    \thmitem{def:fol_quotient_structure/model}\mimprovised In case both \( \mscrX \) and \( \mscrX / {\cong} \) are both \hyperref[def:fol_semantics/model]{models} of the same set of sentences, we say that \( \mscrX / {\cong} \) is a \term{quotient model} of \( \mscrX \).

    If \( T \) is an \hyperref[def:fol_equational_theory]{equational theory}, a quotient of a model of \( T \) is also a model of \( T \).
  \end{thmenum}
\end{definition}
\begin{comments}
  \item Quotient structures correspond to \hyperref[def:categorical_quotient_object]{categorical quotient objects}; see \cref{thm:fol_categorical_quotient_objects}.

  \item As shown in \cref{thm:fol_equational_theory_models/quotient}, if \( \mscrX \) is a model of an \hyperref[def:fol_equational_theory]{equational theory} \( T \), so is \( \mscrX / {\cong} \).
\end{comments}

\begin{theorem}[Quotient structure universal property]\label{thm:quotient_structure_universal_property}
  For every first-order structure \( \mscrX \) and every \hyperref[def:fol_congruence]{congruence} \( \cong \) on \( \mscrX \), the \hyperref[def:fol_quotient_structure]{quotient structure} \( \mscrX / {\cong} \) has the following \hyperref[rem:universal_mapping_property]{universal mapping property}:
  \begin{displayquote}
    Every homomorphism \( h: \mscrX \to \mscrY \) for which \( x \cong x' \) implies \( h(x) = h(x') \) \hyperref[def:factors_through]{uniquely factors through} \( \mscrX / {\cong} \).

    More precisely, there exists a unique homomorphism \( \widetilde{h}: \mscrX / {\cong} \to \mscrY \), such that the following diagram commutes:
    \begin{equation}\label{eq:thm:quotient_structure_universal_property/diagram}
      \begin{aligned}
        \includegraphics[page=1]{output/thm__quotient_structure_universal_property}
      \end{aligned}
    \end{equation}
  \end{displayquote}
\end{theorem}
\begin{comments}
  \item As shown in \cref{thm:fol_equational_theory_models/quotient}, the quotient of a model of an \hyperref[def:fol_equational_theory]{equational theory} \( T \) is also a model of \( T \).

  \item For \hyperref[def:group]{groups}, this theorem can be restated via \hyperref[def:group/kernel]{kernels} and \hyperref[def:normal_subgroup]{normal subgroups}:
  \begin{displayquote}
    Given a normal subgroup \( N \) of \( G \), every homomorphism \( h: G \to H \) satisfying \( N \subseteq \ker h \) uniquely factors through \( G / N \).
  \end{displayquote}

  Similarly, when stated for \hyperref[def:ring]{rings} and \hyperref[def:algebra_over_ring]{algebras over rings}, the theorem uses \hyperref[def:ring/kernel]{kernels} and \hyperref[def:semiring_ideal]{ideals}:
  \begin{displayquote}
    Given an ideal \( I \) of \( R \), every homomorphism \( h: R \to T \) satisfying \( I \subseteq \ker h \) uniquely factors through \( R / I \).
  \end{displayquote}

  For \hyperref[def:module]{modules over rings}, it becomes particularly simple:
  \begin{displayquote}
    Given an \( R \)-submodule \( N \) of \( M \), every homomorphism \( h: M \to K \) satisfying \( N \subseteq \ker h \) uniquely factors through \( M / N \).
  \end{displayquote}
\end{comments}
\begin{proof}
  For any element \( x \) of \( \mscrX \), we want
  \begin{equation*}
    \widetilde{h}(\pi(x)) = h(x).
  \end{equation*}

  This condition can be used as a definition, but only in the case where \( h \) only depends on the equivalence class \( \pi(x) \), but not the representative. This is the reason we have the additional restriction that \( x \cong x' \) must imply \( h(x) = h(x') \).

  Uniqueness follows by construction.
\end{proof}

\paragraph{Structures of functions}

\begin{definition}\label{def:fol_structure_of_functions}\mimprovised
  Consider a \hyperref[def:fol_structure]{first-order structure} \( \mscrX = (X, I) \) over a \hyperref[def:fol_signature]{signature} \( \Sigma \). Fix a set \( S \), possibly unrelated to \( X \).

  Define the interpretation \( J \) on the \hyperref[def:set_of_all_functions]{set of all functions} \( \fun(S, X) \) as follows:
  \begin{thmenum}
    \thmitem{def:fol_structure_of_functions/functions} For every function symbol \( f \) of arity \( n \), let
    \begin{equation*}
      \mathllap{J(f)(a_1, \ldots, a_n)} \coloneqq \mathrlap{\parens[\big]{ s \mapsto I(f)(a_1(s), \ldots, a_n(s)) }.}
    \end{equation*}

    \thmitem{def:fol_structure_of_functions/predicates} For every predicate symbol \( f \) of arity \( n \), let
    \begin{equation*}
      \mathllap{J(p)(a_1, \ldots, a_n)} \coloneqq \mathrlap{\bigwedge_{s \in S} I(p)\parens[\big]{ a_1(s), \ldots, a_n(s) }.}
    \end{equation*}
  \end{thmenum}

  We denote this structure by \( \fun(S, \mscrX) \).
\end{definition}
\begin{comments}
  \item As shown in \cref{thm:fol_equational_theora_models/functions}, if \( \mscrX \) is a model of an \hyperref[def:fol_equational_theory]{equational theory} \( T \), then \( \fun(S, \mscrX) \) is also a model of \( T \).
\end{comments}

\paragraph{Lattice of substructures}

\begin{definition}\label{def:fol_lattice_of_substructures}
  Fix a \hyperref[def:fol_structure]{first-order structure} \( \mscrX = (X, I) \). We associate with \( \mscrX \) the set \( \op*{Sub}(\mscrX) \) of all \hyperref[def:fol_substructure]{substructures} of \( \mscrX \).

  We will show in \cref{thm:fol_lattice_of_substructures} that \( \op*{Sub}(\mscrX) \) a \hyperref[def:fol_lattice]{lattice} with respect to the substructure relation. For this reason, we will call \( \op*{Sub}(\mscrX) \) the \term{lattice of substructures} of \( \mscrX \).

  \begin{thmenum}
    \thmitem{def:fol_lattice_of_substructures/model} In case \( \mscrX \) and all its substructures are \hyperref[def:fol_semantics/model]{models} of the same set of sentences, we also call \( \op*{Sub}(\mscrX) \) a \term{lattice of submodels}.

    As shown \cref{thm:fol_equational_theory_models/lattice}, this is the case for \hyperref[def:fol_equational_theory]{equational theories}. It also holds for \hyperref[def:preordered_set]{ordered sets}.
  \end{thmenum}
\end{definition}

\begin{proposition}\label{thm:fol_lattice_of_substructures}
  The \hyperref[def:fol_lattice_of_substructures]{lattice of substructures} \( \op*{Sub}(\mscrX) \) of \( \mscrX = (X, I) \) is a \hyperref[def:complete_lattice]{complete lattice} with respect to the substructure relation.

  \begin{thmenum}
    \thmitem{thm:fol_lattice_of_substructures/top} The \hyperref[def:extremal_points/top_and_bottom]{top element} is \( \mscrX \) itself.

    \thmitem{thm:fol_lattice_of_substructures/bottom} The \hyperref[def:extremal_points/top_and_bottom]{bottom element} is the \hyperref[def:fol_intersection_structure]{intersection} of all substructures of \( \mscrX \).

    \thmitem{thm:fol_lattice_of_substructures/join} The \hyperref[def:lattice/join]{join} of the family of substructures \( \seq{ (Y_k, I\restr_{Y_k}) }_{k \in \mscrK} \) is the \hyperref[def:fol_generated_substructure]{generated substructure}
    \begin{equation*}
      \mscrY \coloneqq \braket*{ \bigcup_{k \in \mscrK} Y_k }
    \end{equation*}

    \thmitem{thm:fol_lattice_of_substructures/meet} The \hyperref[def:lattice/meet]{meet} of a family of substructures is their \hyperref[def:fol_intersection_structure]{intersection}.
  \end{thmenum}
\end{proposition}
\begin{proof}
  \SubProofOf{thm:fol_lattice_of_substructures/top} Trivial.

  \SubProofOf{thm:fol_lattice_of_substructures/bottom} Any substructure of \( \mscrX \) is a superstructure of the intersection of \hi{all} substructures. Thus, this intersection is a substructure of all others.

  \SubProofOf{thm:fol_lattice_of_substructures/join} Let \( \mscrZ = (Z, I\restr_Z) \) be a least upper bound of the family \( \seq{ \mscrY_k }_{k \in \mscrK} \), i.e. a substructure of \( \mscrX \) which is a superstructure of \( \mscrY_k = (Y_k, I\restr_{Y_k}) \) for every \( k \) in \( \mscrK \).

  We must show that \( \mscrZ \) is also a superstructure of \( \mscrY \).

  We will use \fullref{thm:induction_on_generated_substructures} on any element \( a \) from (the universe of) \( \mscrY \) to show that it belongs to \( Z \).
  \begin{itemize}
    \item If \( a \) belongs to \( \bigcup_{k \in \mscrK} Y_k \), then it obviously also belongs to \( Z \) since the latter is a superstructure of \( \mscrY_k \) for every \( k \) in \( \mscrK \).

    \item Otherwise, \( a = I(f)(a_1, \ldots, a_n) \), where \( a_1, \ldots, a_n \) belong to \( Z \). But \( Z \) is closed under \( I(f) \) since it induces a structure, so \( a \) also belongs to \( Z \).
  \end{itemize}

  It follows that \( \mscrY \) is a substructure of \( \mscrZ \), i.e. the least upper bound of the family \( \seq{ \mscrY_k }_{k \in \mscrK} \).

  \SubProofOf{thm:fol_lattice_of_substructures/meet} Again consider the family \( \seq{ \mscrY_k }_{k \in \mscrK} \) of substructures of \( \mscrX \), where \( \mscrY = (Y_k, I\restr_{Y_k}) \) for every \( k \) in \( \mscrK \).

  \Cref{thm:boolean_algebra_of_subsets/meet} implies that the meet of the family \( \seq{ Y_k }_{k \in \mscrK} \) of sets is their intersection \( \bigcap_{k \in \mscrK} Y_k \). This intersection happens to be the universe of the intersection structure \( \bigcap_{k \in \mscrK} \mscrY_k \), which must thus be the meet of the original family \( \seq{ \mscrY_k }_{k \in \mscrK} \).
\end{proof}

\begin{theorem}[Lattice theorem for substructures]\label{thm:lattice_theorem_for_substructures}
  Fix a \hyperref[def:fol_structure]{first-order structure} \( \mscrX = (X, I) \) and a \hyperref[def:fol_congruence]{congruence} \( {\cong} \) on \( \mscrX \).

  We will consider only \hyperref[def:fol_substructure]{substructures} \( \mscrY = (Y, I\restr_Y) \) of \( \mscrX \) satisfying the following compatibility condition:
  \begin{equation}\label{eq:thm:lattice_theorem_for_substructures/compatibility}
    a \in Y \T{and} a \cong a' \T{together imply} a' \in Y.
  \end{equation}

  This condition ensures that any equivalence class \( \pi(a) \) of the quotient set \( Y / {\cong} \) is also an element of \( X / {\cong} \).

  We will give a verbose formulation as a buildup for \cref{thm:lattice_theorem_for_substructures/isomorphism}; the theorem is summarized in \cref{fig:thm:lattice_theorem_for_substructures}.

  \begin{figure}[!ht]
    \centering
    \includegraphics[page=1]{output/thm__lattice_theorem_for_substructures}
    \caption{The lattices from \fullref{thm:lattice_theorem_for_substructures}.}
    \label{fig:thm:lattice_theorem_for_substructures}
  \end{figure}

  \begin{thmenum}
    \thmitem{thm:lattice_theorem_for_substructures/sublattice} The family of substructures of \( \mscrX \) compatible with \( {\cong} \) is a \hyperref[def:complete_lattice]{complete lattice}.

    It is however generally not a \hyperref[def:lattice/submodel]{sublattice} of the \hyperref[def:fol_lattice_of_substructures]{lattice of all substructures} \( \op*{Sub}(\mscrX) \) --- as an example, take any nontrivial equivalence relation in the \hyperref[def:pure_equality]{theory of pure equality}.

    \thmitem{thm:lattice_theorem_for_substructures/direct} For every substructure \( \mscrY = (Y, I\restr_Y) \) of \( \mscrX \) compatible with \( {\cong} \), the quotient structure \( \mscrY / {\cong} \) is a substructure of \( \mscrX / {\cong} \).

    Furthermore, \( \mscrY / {\cong} \) coincides with the induced substructure of \( Y / {\cong} \) in \( \mscrX / {\cong} \).

    \thmitem{thm:lattice_theorem_for_substructures/reverse} For every substructure \( \mscrQ = (Q, I_\cong\restr_Q) \) of \( \mscrX / {\cong} \), the union \( \bigcup Q \) induces a substructure of \( \mscrX \) compatible with \( {\cong} \).

    \thmitem{thm:lattice_theorem_for_substructures/isomorphism} The map \( \Phi(Y) \coloneqq Y / {\cong} \) is an \hyperref[def:complete_latice/homomorphism]{isomorphism} between the lattice of substructures of \( \mscrX \) compatible with \( {\cong} \) and the lattice of all substructures of \( \mscrX / {\cong} \).
  \end{thmenum}
\end{theorem}
\begin{comments}
  \item Simpler forms of this theorem hold in some special cases --- see \fullref{thm:lattice_theorem_for_subgroups}, \fullref{thm:lattice_theorem_for_ideals} and especially \fullref{thm:lattice_theorem_for_submodules}.
\end{comments}
\begin{proof}
  Let \( \Sigma \) be the signature of \( \mscrX \).

  \SubProofOf{thm:lattice_theorem_for_substructures/sublattice} Denote by \( L \) the family of substructures of \( \mscrX \) compatible with \( {\cong} \).

  This family is closed under intersections, so by the same principle as \cref{thm:fol_lattice_of_substructures}, \( L \) is a complete lattice.

  \SubProofOf{thm:lattice_theorem_for_substructures/direct} Let \( \mscrY = (Y, I\restr_Y) \) be a substructure of \( \mscrX = (X, I) \).

  The set \( Y / {\cong} \) is the universe of the quotient \( \mscrY / {\cong} \). Due to the compatibility condition for \( {\cong} \), it is a subset of \( X / {\cong} \), and it thus induces a substructure of \( \mscrX / {\cong} \). We must show that the two interpretations coincide.

  For every \( n \)-ary function symbol \( f \) over \( \Sigma \), if \( \pi(a_1), \ldots, \pi(a_n) \) are in \( Y / {\cong} \), we have
  \begin{balign*}
    (I\restr_Y)_\cong(f)\parens[\big]{ \pi(a_1), \ldots, \pi(a_n) }
    &=
    \pi\parens[\big]{ I\restr_Y(f)(a_1, \ldots, a_n) }
    = \\ &=
    \pi\parens[\big]{ I(f)(a_1, \ldots, a_n) }
    = \\ &=
    I_\cong(f)\parens[\big]{ \pi(a_1), \ldots, \pi(a_n) }
    = \\ &=
    \parens[\big]{ I_\cong\restr_{Y / {\cong}}(f) }\parens[\big]{ \pi(a_1), \ldots, \pi(a_n) }.
  \end{balign*}

  This shows that the interpretations of \( f \) coincide in \( \mscrY / {\cong} \) and in the induced substructure of \( Y / {\cong} \). Generalizing on \( f \), we conclude that the two interpretations coincide.

  \SubProofOf{thm:lattice_theorem_for_substructures/reverse} Let \( \mscrQ = (Q, I_\cong\restr_Q) \) be a substructure of \( \mscrX / {\cong} \) and let \( Y \coloneqq \bigcup Q \).

  We must show that \( Y \) is closed with respect to the interpretation \( I(f) \) of every \( n \)-ary function symbol \( f \) over \( \Sigma \). If \( a_1, \ldots, a_n \) are members of \( Y = \bigcup Q \), the cosets \( \pi(a_1), \ldots, \pi(a_n) \) belong to \( Q \). Since \( Q \) is closed under application of \( I_\cong(f) \), it contains
  \begin{equation*}
    I_\cong(f)\parens[\big]{ \pi(a_1), \ldots, \pi(a_n) }
    =
    \pi\parens[\big]{ I(f)(a_1, \ldots, a_n) }.
  \end{equation*}

  Due to the compatibility condition, \( Y \) contains all members of \( \pi\parens[\big]{ I(f)(a_1, \ldots, a_n) } \) including \( I(f)(a_1, \ldots, a_n) \) itself.

  Generalizing on \( f \), we conclude that \( Y \) induces a substructure of \( \mscrX \). It is compatible with \( {\cong} \) by construction.

  \SubProofOf{thm:lattice_theorem_for_substructures/isomorphism} Clearly \( \Phi \) is injective, and it preserves and reflects order. \Cref{thm:def:preordered_set/homomorphism_is_reflecting} implies that it is an order embedding. \Cref{thm:def:complete_lattice/embedding} implies that it is an embedding of complete lattices.

  \Cref{thm:lattice_theorem_for_substructures/reverse} shows that \( \Phi \) is surjective. A surjective embedding is simply an isomorphism. Therefore, \( \Phi \) is an isomorphism of complete lattices.
\end{proof}
