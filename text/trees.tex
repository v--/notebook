\subsection{Trees}\label{subsec:trees}

In this section, we will regard the edges of simple undirected graphs as sets of unordered tuples and edges of simple directed graphs as sets of ordered tuples.

\begin{definition}\label{def:tree}
  The possibly infinite \hyperref[def:undirected_multigraph]{simple undirected graph} \( T= (E, V, \mscrE) \) is called a \term{tree} if any of the following equivalent conditions hold:
  \begin{thmenum}
    \thmitem{def:tree/connected_acyclic} \( T \) is \hyperref[def:undirected_multigraph_connectedness]{connected} and \hyperref[def:undirected_multigraph_path/cycle]{acyclic}.
    \thmitem{def:tree/maximally_acyclic} \( T \) is \term{maximally acyclic}, meaning that adding an edge between existing vertices would create a \hyperref[def:undirected_multigraph_path/cycle]{cycle}.
    \thmitem{def:tree/minimally_connected} \( T \) is \term{minimally connected}, meaning that removing an edge will make the graph \hyperref[def:directed_multigraph_connectedness]{disconnected}.
    \thmitem{def:tree/single_path} For every pair of vertices in \( T \), there exists a unique \hyperref[def:undirected_multigraph_path]{path} connecting them.

    This condition motivates the definition of \hyperref[def:arborescence]{arborescences}.
  \end{thmenum}

  It is conventional to use the term \term{node} for the vertices of a tree.
\end{definition}
\begin{proof}
  \ImplicationSubProof{def:tree/connected_acyclic}{def:tree/maximally_acyclic} Suppose that \( T \) is connected and acyclic. There is always an edge not in \( T \) because the \hyperref[ex:complete_graph]{complete graph} \( K_V \) contains cycles and \( T \) does not. Let \( \set{ u, v } \) be an edge not in \( T \) and let \( T' \) be the supergraph that adjoins only this new edge.

  Since \( T \) is connected, there exists a path \( p \) connecting \( u \) and \( v \). Appending the edge \( \set{ u, v } \) to this path creates a cycle in \( T' \).

  Therefore, \( T' \) is not acyclic.

  \ImplicationSubProof{def:tree/maximally_acyclic}{def:tree/connected_acyclic} Suppose that \( T \) is maximally acyclic. We will show that it is connected.

  Suppose that \( T \) is not connected. Then there exist vertices \( u \) and \( v \) with no path between them. Let \( T' \) be the supergraph that adjoins only the edge \( \set{ u, v } \).

  Since there is no path between \( u \) and \( v \) in \( T \), \( T' \) is also acyclic. But this contradicts the maximality of \( T \).

  Therefore, \( T \) is connected.

  \ImplicationSubProof{def:tree/connected_acyclic}{def:tree/minimally_connected} Suppose that \( T \) is connected and acyclic. Let \( \set{ u, v } \) be any edge of \( T \) and let \( T' \) be the subgraph that does not contain this edge.

  Suppose that \( T' \) is connected. Let \( w \) be any vertex and let \( p \) be a path in \( T' \) from \( u \) to \( w \). Then adding \( \set{ u, v } \) creates a cycle in \( T \), which contradicts our assumption that \( T \) is acyclic.

  Therefore, \( T' \) is not connected.

  \ImplicationSubProof{def:tree/minimally_connected}{def:tree/connected_acyclic} Suppose that \( T \) is minimally connected. We will show that it is acyclic.

  Suppose that \( T \) has a cycle \( p \). We can thus remove any edge of \( p \) from \( T \) and the resulting subgraph will also be connected. This contradicts the minimality of \( T \).

  Therefore, \( T \) is acyclic.

  \ImplicationSubProof{def:tree/connected_acyclic}{def:tree/single_path} Suppose that \( T \) is connected and acyclic. Since \( T \) is connected, there exists at least one path between any two vertices. Since it is acyclic, this path must be unique because otherwise we could easily create a cycle by joining two such paths.

  \ImplicationSubProof{def:tree/single_path}{def:tree/connected_acyclic} Suppose that, for every pair of vertices in \( T \), there exists a unique path connecting them. It is clear that \( T \) is connected.

  Suppose that \( T \) has a cycle \( p \) from \( u \) to \( u \) passing through \( v \). We thus obtain two paths from \( u \) to \( v \) --- one not containing the first edge in \( p \) and one not containing the last edge. But this contradicts our assumption that any two paths from \( u \) to \( v \) are equal.

  Therefore, \( T \) is acyclic.
\end{proof}

\begin{example}\label{ex:def:tree}
  The reduced infinite integer graphs \eqref{eq:ex:infinite_integer_graphs/positive}, \eqref{eq:ex:infinite_integer_graphs/negative} and \eqref{eq:ex:infinite_integer_graphs/two_sided} from \fullref{ex:infinite_integer_graphs} are simple examples of infinite trees.

  Examples of finite trees are \hyperref[def:proof_tree]{proof trees} from \fullref{subsec:deductive_systems} and \hyperref[def:parse_tree]{concrete} and \hyperref[rem:abstract_syntax_tree]{abstract syntax trees} from \hyperref[def:formal_grammar]{formal grammar}.
\end{example}

\begin{definition}\label{def:arborescence}\mcite[ch. 4, sec. 3.1]{GondranMinoux1984Graphs}
  Let \( T = (V, A) \) be a \hyperref[def:directed_multigraph/simple]{simple directed graph}. Suppose that there exists a vertex \( r \in V \) such that for every other vertex \( v \in V \), there exists a unique \hyperref[def:directed_multigraph_path/directed]{directed path} from \( r \) to \( v \). The pair \( (G, r) \) is called a (directed) \term{arborescence}. It is often more convenient to consider the triple \( T = (V, A, r) \) instead. The vertex \( r \) is called the \term{root} of the arborescence.

  \begin{thmenum}
    \thmitem{def:arborescence/undirected} A \hyperref[def:tree]{tree} with a distinguished vertex \( r \) is called a \term{rooted tree}.

    For every tree \( T \), as a consequence of \fullref{def:tree/single_path}, every vertex \( r \) of \( T \) induces an \hyperref[def:multigraph_orientation]{orientation} \( O(T) \) of \( T \) such that \( O(T) \) is an arborescence.

    For this reason, we identify rooted trees with their induced arborescences.

    \thmitem{def:arborescence/depth} The \term{depth} of a node \( v \) is the length of the path from \( r \) to \( v \). The term \term{level} is also sometimes used but with a slightly different connotation --- the greater the depth, the lower the level.

    \thmitem{def:arborescence/ancestry} If \( v \) has a strictly greater depth than \( u \), we say that \( u \) is an \term{ancestor} of \( v \) and that \( v \) is a \term{descendant} of \( u \).

    The ancestor of \( v \) at the lowest possible level is called the \term{parent} of \( v \). If \( u \) is a parent of \( v \), \( u \) is a \term{child} of \( v \). If a node has no children, we say that it is a \term{leaf node}.

    Finally, if \( u \) and \( v \) are on the same level, we call them \term{siblings}.

    \thmitem{def:arborescence/height} The \term{height} or \term{depth} of the entire tree \( T \), if it exists, is the supremum of the depths of all nodes.

    \thmitem{def:arborescence/width} The \term{width} or \term{breadth} of the entire tree \( T \), if it exists, is the supremum of the number of siblings among all vertices. It is equal to the degree of \( T \) minus \( 1 \).

    We use terminology similar to \fullref{def:relation/arity}, e.g. \enquote{binary tree}, \enquote{ternary tree}, \ldots.

    \thmitem{def:arborescence/sub-arborescence} A \hyperref[def:directed_multigraph/submodel]{subgraph} of an arborescence that is itself an arborescence is called a \term{sub-arborescence}. The same holds for rooted subtrees.

    For every node \( v \) of \( T \), we define the \term{induced sub-arborescence} of \( T \) with root \( v \) as the subgraph \hyperref[def:directed_multigraph/submodel]{induced} by \( V \setminus \dom(p) \), where \( p \) is the unique path from \( r \) to \( v \).
  \end{thmenum}
\end{definition}

\begin{definition}\label{def:ordered_tree}\mimprovised
  An \term{ordered tree} is a \hyperref[def:arborescence/undirected]{tree} with a \hyperref[def:partially_ordered_set]{partial order} such that every set of \hyperref[def:arborescence/ancestry]{siblings} is a \hyperref[def:partial_order_chain]{chain}.
\end{definition}

\begin{definition}\label{def:traversal_ordering}\mimprovised
  We will define two \term{traversal orderings} on a given finite \hyperref[def:ordered_tree]{ordered tree} \( T = (V, A, r, \leq) \). These are \hyperref[def:well_ordered_set]{well-orderings} \( \leq_W \) of the vertices of \( V \). A \term{pre-order traversal}\footnote{This concept is not related to \hyperref[def:preordered_set]{preordered sets}.} (see \ref{def:traversal_ordering/step/pre}) visits each root before the successors, while a \term{post-order traversal} (see \ref{def:traversal_ordering/step/post}) visits the root last. Equivalently, we can regard a traversal as a \hyperref[def:sequence]{finite sequence} of elements of \( V \) rather than an ordering of \( V \).

  The relation \( \leq_W \) can be defined on \( \card(V) \) as follows:
  \begin{thmenum}
    \thmitem{def:traversal_ordering/init} If \( \card(V) = 1 \), then the ordering is trivial --- let \( r \leq_W r \).
    \thmitem{def:traversal_ordering/step} Suppose that \( \card(V) = n \) and that we have already built a well-order for all smaller trees. Let \( T_k = (V_k, A_k, r_k) \), \( k = 1, \ldots, n \) be rooted subtrees, whose roots \( r_1 < \ldots < r_n \) are successors of \( r \), that is,
    \begin{equation*}
      \begin{aligned}
        \includegraphics[page=1]{output/def__traversal_ordering.pdf}
      \end{aligned}
    \end{equation*}

    Let \( \leq_{W_k} \) be the recursively defined traversal ordering on \( T_k \).

    Define the relation \( x \leq_W y \) on \( V \) to hold:
    \begin{thmenum}
      \thmitem{def:traversal_ordering/step/same} If both \( x \) and \( y \) belong to \( V_i \) and \( x \leq_{W_i} y \).
      \thmitem{def:traversal_ordering/step/different} If \( x \in V_i \) and \( y \in V_j \) with \( i < j \).
      \thmitem{def:traversal_ordering/step/root} If \( x = y = r \).
      \thmitem{def:traversal_ordering/step/pre} If \( x = r \) in case we want a pre-order.
      \thmitem{def:traversal_ordering/step/post} If \( y = r \) in case we want a post-order.
    \end{thmenum}
  \end{thmenum}

  \begin{figure}[!ht]
    \centering
    \includegraphics[page=2]{output/def__traversal_ordering.pdf}
    \caption{An ordered tree whose \hyperref[def:traversal_ordering]{pre-order traversal} is \( abcde \) and whose \hyperref[def:traversal_ordering]{post-order traversal} is \( cdbea \).}
    \label{fig:def:traversal_ordering}
  \end{figure}
\end{definition}
\begin{defproof}
  We will show via induction on \( \card(V) \) that \( \leq_W \) is a well-order. Since the trivial case \( \card(V) = 1 \) is vacuous, suppose that \( \card(V) > 1 \).

  \SubProofOf[def:binary_relation/reflexive]{reflexivity} Follows from \fullref{def:traversal_ordering/step/root}.

  \SubProofOf[def:binary_relation/antisymmetric]{antisymmetry} Suppose that \( x \leq_W y \) and \( y \leq_W x \) with \( x \neq y \).
  \begin{itemize}
    \item If both \( x \) and \( y \) belong to \( V_i \), antisymmetry follows from the inductive hypothesis.
    \item If \( x \in V_i \) and \( y \in V_j \) with \( i < j \), then we cannot possibly have \( y \leq_W x \).
    \item If \( x = r \), then \( r \leq y \); this is pre-order traversal, hence we cannot have \( r \geq y \).
    \item We handle post-order traversal similarly.
  \end{itemize}

  \SubProofOf[def:binary_relation/transitive]{transitivity} Suppose that \( x \leq_W y \) and \( y \leq_W z \) with all three elements distinct.
  \begin{itemize}
    \item If all three belong to \( V_i \), antisymmetry follows from the inductive hypothesis.
    \item If \( x = r \), then \( r \leq y \); this is pre-order traversal, hence by definition \( x \leq_W z \).
    \item If \( z = r \), then \( y \leq z \); this is post-order traversal, hence by definition \( x \leq_W z \).
    \item We cannot have \( y = r \).
    \item If \( x, y \in V_i \) and \( z \in V_j \) with \( i < j \), then \( x \leq_W z \) by \fullref{def:traversal_ordering/step/different}.
    \item If \( x \in V_i \) and \( y, z \in V_j \) with \( i < j \), then also \( x \leq_W z \) by \fullref{def:traversal_ordering/step/different}.
  \end{itemize}

  \SubProofOf[def:well_founded_relation]{well-foundedness} Follows from \fullref{thm:finite_totally_ordered_set_is_well_ordered}.
\end{defproof}

\begin{lemma}[K\"onig's lemma]\label{thm:konigs_lemma}
  Every \hyperref[def:hypergraph/degree]{locally finite} \hyperref[def:arborescence]{arborescence} of \hyperref[def:hypergraph/order]{infinite} \hyperref[def:hypergraph/order]{order} contains a \hyperref[def:undirected_multigraph_path/simple]{simple path} of infinite length.
\end{lemma}
\begin{proof}
  Let \( T = (V, A, r) \) be a locally finite infinite arborescence. We will build an infinite simple path
  \begin{equation*}
    p = (p_1, p_2, \cdots)
  \end{equation*}
  using \hyperref[rem:natural_number_recursion]{natural number recursion} starting at one. Let \( c \) be a \hyperref[def:choice_function]{choice function} on the family
  \begin{equation*}
    \pow(V) \setminus \set{ \varnothing }.
  \end{equation*}

  Such a choice function exists by the \hyperref[def:zfc/choice]{axiom of choice}.

  \begin{itemize}
    \item Since \( T \) is locally finite, there are finitely many children of \( r \). If we suppose that the \hyperref[def:arborescence/sub-arborescence]{subarborescence} induced by \( v \) is finite for every child \( v \) of \( r \), then we would obtain that \( T \) itself is finite, which is a contradiction. Therefore, for at least one child, the induced sub-arborescence is infinite. Using the choice function \( c \), pick one such child and denote it by \( v_1 \).

    Define \( p_1 \) to be the arc \( r \to v_1 \).

    \item Fix \( k > 1 \). Using the same argument on the children of \( t(e_{k-1}) \), we can pick a child \( v_k \) that has an infinite induced sub-arborescence.

    Define \( e_k \) to be the arc \( t(e_{k-1}) \to v_k \). The path
    \begin{equation*}
      (e_1, e_2, \ldots, e_{k-1}, e_k)
    \end{equation*}
    is simple by construction because \( T \) is an arborescence and thus contains no directed cycles.
  \end{itemize}

  Thus, we have constructed an infinite simple path.
\end{proof}
