\subsection{Trees}\label{subsec:trees}

\begin{definition}\label{def:tree}\mcite[thm. 1.5.1]{Diestel2005}
  We call the (possibly infinite) \hyperref[def:undirected_graph]{simple undirected graph} \( T = (E, V) \) a \term[bg=дърво (\cite[21]{Мирчев2001}), ru=дерево (\cite[53]{Емеличев1990})]{tree} if any of the following equivalent conditions hold:
  \begin{thmenum}[series=def:tree]
    \thmitem{def:tree/connected_acyclic} \( T \) is \hyperref[def:graph_connectedness/undirected]{connected} and \hyperref[def:acyclic_graph]{acyclic}.

    \thmitem{def:tree/minimally_connected} \( T \) is \term{minimally connected}, meaning that removing an edge will make the graph \hyperref[def:graph_connectedness/undirected]{disconnected}.

    \thmitem{def:tree/maximally_acyclic} \( T \) is \term{maximally acyclic}, meaning that adding an edge between existing vertices would create a \hyperref[def:graph_cycle]{cycle}.

    \thmitem{def:tree/single_path} There exists a unique \hyperref[def:graph_walk/path]{path} connecting every pair of distinct vertices in \( T \).
  \end{thmenum}

  Trees inherit metamathematical properties from undirected graphs:
  \begin{thmenum}[resume=def:tree]
    \thmitem{def:tree/homomorphism}\mimprovised A \term{homomorphism} between trees is simply an undirected graph homomorphisms.

    \thmitem{def:tree/category}\mimprovised Correspondingly, for a \hyperref[def:grothendieck_universe]{Grothendieck universe} \( \mscrU \), the category of trees is a \hyperref[def:subcategory/full]{full subcategory} of the category of undirected graphs.

    \thmitem{def:tree/subtree}\mcite[649]{Rosen1999} We say that the tree \( R \) is a \term{subtree} of \( T \) if it is a \hyperref[def:undirected_graph/subgraph]{subgraph} of \( T \).
  \end{thmenum}
\end{definition}
\begin{comments}
  \item We will refer to vertices in \hyperref[def:tree]{trees} as \term[en=node (\cite[190]{Erickson2019})]{nodes}.
\end{comments}
\begin{defproof}
  \ImplicationSubProof{def:tree/connected_acyclic}{def:tree/minimally_connected} Let \( T \) be a connected acyclic graph.

  Let \( T' \) be the graph obtained from \( T \) by removing the edge \( \set{ u, v } \). Then \( T' \) is not connected because, if there is a path from \( u \) to \( v \) in \( T' \), it would extend to a cycle in \( T \), contradicting the acyclicity of \( T \).

  Therefore, \( T \) is minimally connected.

  \ImplicationSubProof{def:tree/minimally_connected}{def:tree/maximally_acyclic} Let \( T \) be a minimally connected graph.

  \SubProof*{Proof that \( T \) is acyclic} Suppose that there exists a cycle
  \begin{equation*}
    v_0 \to v_1 \to \cdots \to v_n.
  \end{equation*}

  \Fullref{thm:connected_graph_cycles} implies that the graph obtained from \( T \) by removing the edge \( \set{ v_0, v_1 } \) is still connected, contradicting the assumption that \( T \) is minimally connected.

  The obtained contradiction shows that no cycles can exist in \( T \).

  \SubProof*{Proof that \( T \) is minimally acyclic} Let \( T' \) be the graph obtained from \( T \) by adding an edge between the vertices \( u \) and \( v \). Since \( T \) is connected, there exists a path from \( u \) to \( v \), which in \( T' \) can be extended to a cycle on \( u \). Thus, \( T' \) is not acyclic.

  Therefore, \( T \) is maximally acyclic.

  \ImplicationSubProof{def:tree/maximally_acyclic}{def:tree/single_path} Let \( T \) be a maximally acyclic graph.

  Fix two vertices \( u \) and \( v \) and suppose that no walk exists between them. Let \( T' \) be the graph obtained by adding the edge \( \set{ u, v } \) to \( T \).

  Suppose additionally that \( T' \) has a cycle, say
  \begin{equation*}
    w_0 \to w_1 \to \cdots \to w_n.
  \end{equation*}

  Since \( T \) is acyclic, \( \set{ u, v } \) must be an edge in this cycle.
  \begin{itemize}
    \item If, for some \( k = 0, \ldots, n - 1 \), we have \( u = w_k \) and \( v = w_{k+1} \), then the following is a walk in \( T \):
    \begin{equation*}
      u = w_k \to w_{k-1} \to \cdots \to w_0 \to w_{n-1} \to \cdots \to w_{k+1} = v.
    \end{equation*}

    \item Otherwise, for some \( k = 0, \ldots, n - 1 \), we must have \( u = w_{k+1} \) and \( v = w_k \). Then the following is a walk in \( T \):
    \begin{equation*}
      u = w_{k+1} \to w_{k+2} \to \cdots \to w_n \to w_1 \to \cdots \to w_k.
    \end{equation*}
  \end{itemize}

  The existence of these walks contradict our assumption that there is no walk from \( u \) to \( v \) in \( T \). Thus, \( T' \) must also be acyclic, which in turn contradicts our main assumption --- that \( T \) is maximally acyclic.

  Therefore, it remains for \( T \) to be connected. Furthermore, by \fullref{thm:acyclic_graph_paths}, there exists at most one path, and hence exactly one path between any set of vertices.

  \ImplicationSubProof{def:tree/single_path}{def:tree/connected_acyclic} Let \( T \) be a graph in which every pair of vertices is connected by a unique path.

  Then \( T \) is clearly connected. Furthermore, if \( T \) has some cycle
  \begin{equation*}
    v_0 \to v_1 \to \cdots \to v_{n-1} \to v_n,
  \end{equation*}
  then
  \begin{equation*}
    v_0 \to v_1 \to \cdots \to v_{n-1}
  \end{equation*}
  is a path connecting \( v_0 \) to \( v_{n-1} \), and so is the single-edge path
  \begin{equation*}
    v_0 \to v_{n-1}.
  \end{equation*}

  We thus obtain two distinct paths from \( v_0 \) to \( v_{n-1} \), contradicting out initial assumption.

  Therefore, \( T \) must be acyclic.
\end{defproof}

\begin{lemma}\label{thm:removing_node_from_tree}
  If we remove any node from a \hyperref[def:tree]{tree} along with its incident edges, the \hyperref[def:graph_connected_component]{connected components} of the resulting graph are trees.

  \begin{figure}[!ht]
    \centering
    \includegraphics[page=1]{output/thm__removing_node_from_tree}
    \caption{An illustration of \fullref{thm:removing_node_from_tree}.}\label{fig:thm:removing_node_from_tree}
  \end{figure}
\end{lemma}
\begin{proof}
  Since the ambient graph is a tree, there exists at most one path between any pair of points. Within a connected component, there exists at least one path between any pair of points, hence exactly one path. Thus, every component is a tree.
\end{proof}

\begin{lemma}[K\"{o}nig's infinity lemma]\label{thm:konigs_infinity_lemma}
  Every \hyperref[def:graph_cardinality/local]{locally finite} \hyperref[def:tree]{tree} of \hyperref[def:graph_cardinality/order]{infinite order} has an \hyperref[def:graph_walk/path]{infinite path}.
\end{lemma}
\begin{comments}
  \item We use an explicit choice function here given by the \hyperref[def:zfc/choice]{axiom of choice}, but in special cases like \hyperref[def:ordered_tree]{ordered trees}, there are natural choice functions to consider.
\end{comments}
\begin{proof}
  Let \( T = (V, E) \) be a locally finite infinite tree. Fix a \hyperref[def:choice_function]{choice function} \( c \) on all families of nodes of \( T \) and let \( v_0 \coloneqq c(V) \).

  We will build in parallel an infinite path
  \begin{equation*}
    v_0 \to v_1 \to v_2 \cdots
  \end{equation*}
  and a sequence of trees \( T_0, T_1, \ldots \), each also locally finite of infinite order, such that \( T_{k+1} \) is a connected component of \( T_k \) with \( v_k \) removed (as in \fullref{thm:removing_node_from_tree}).

  \begin{itemize}
    \item For the base case, let \( T_0 \) be \( T \).

    \item For the inductive case, suppose that we have already built \( T_k \) and
    \begin{equation*}
      v_0 \to v_1 \to \cdots \to v_k.
    \end{equation*}

    Denote by \( W \) the set of nodes adjacent to \( v_k \) in \( T_k \) and by \( \mscrR \) the family of trees obtained by removing \( v_k \) from \( T_k \).

    Since \( \mscrR \) partitions the infinitely many nodes of \( T_k \) (excluding \( v_k \)), \fullref{thm:pigeonhole_principle} implies that at least one of the trees in \( \mscrR \) has infinite order. Denote by \( \mscrR_\infty \) be the set of all such trees.

    Finally, let
    \begin{equation*}
      v_{k+1} \coloneqq c\parens[\Big]{ \set[\Big]{ w \in W \given \qexists* {R \in \mscrR_\infty} w \in R } }
    \end{equation*}
    and let \( T_{k+1} \) be the tree in \( \mscrR_\infty \) containing \( v_{k+1} \).
  \end{itemize}

  We have thus built an infinite path in \( T \) with the aid of a predetermined choice function.
\end{proof}

\paragraph{Rooted trees}

\begin{definition}\label{def:rooted_tree}\mcite[15]{Diestel2005}
  A \term[ru=ориентированное дерево (\cite[323]{Емеличев1990}); корневое дерево (\cite[\S 9.2.1]{Новиков2013})]{rooted tree} is a triple \( T = (V, E, r) \), where \( (V, E) \) is a \hyperref[def:tree]{tree} and \( r \) is a distinguished node, called the \term{root}.

  As per \fullref{def:tree/single_path}, every node \( v \) has a unique path from \( r \).

  \begin{thmenum}
    \thmitem{def:rooted_tree/order} We define the (nonstrict) \hyperref[def:partially_ordered_set]{partial order} \( u \leq v \) to hold if \( u \) lies on the path from \( r \) to \( v \). We call \( \leq \) the \term{ancestor order} or \term{vertical order} of \( T \) in order to distinguish it from the horizontal order in \hyperref[def:ordered_tree]{ordered trees}.

    \thmitem{def:rooted_tree/height} The \term{height} of a node \( v \) is the length of the unique path from \( r \). This is precisely the height of \( v \) in the sense of \fullref{def:partial_order_element_height}.

    \thmitem{def:rooted_tree/ancestor_descendant}\mcite[637]{Rosen1999} If \( u \leq v \), we say that \( u \) is an \term[ru=предок (\cite[298]{БелоусовТкачёв2004})]{ancestor} of \( v \) and that \( v \) is a \term[ru=потомок (\cite[298]{БелоусовТкачёв2004})]{descendant} of \( u \).

    \thmitem{def:rooted_tree/parent_child}\mcite[637; 639]{Rosen1999} If \( u \) is the penultimate element in the path from \( r \) to \( v \), we say that \( u \) is a \term[ru=отец (\cite[298]{БелоусовТкачёв2004})]{parent} of \( v \) and that \( v \) is a \term[ru=сын (\cite[298]{БелоусовТкачёв2004})]{child} of \( u \).

    \thmitem{def:rooted_tree/leaf}\mcite[638]{Rosen1999} If a node has no children, we call it a \term[ru=лист (\cite[298]{БелоусовТкачёв2004})]{leaf node}.

    \thmitem{def:rooted_tree/siblings}\mcite[639]{Rosen1999} If \( u \) and \( v \) has the same parent, we call them \term{siblings}.

    \thmitem{def:rooted_tree/immediate_subtree}\mimprovised An \term{immediate subtree} of \( T \) is a rooted tree that is a subtree of \( T \) and whose root is a child of \( r \).
  \end{thmenum}

  Rooted trees have the following metamathematical properties:
  \begin{thmenum}[resume=def:rooted_tree]
    \thmitem{def:rooted_tree/homomorphism}\mimprovised A \term{homomorphism} between rooted trees is a \hyperref[def:tree/homomorphism]{tree homomorphism} that preserves roots.

    \thmitem{def:rooted_tree/category}\mimprovised For a \hyperref[def:grothendieck_universe]{Grothendieck universe} \( \mscrU \), we can consider the category of \( \mscrU \)-small rooted trees with rooted tree homomorphisms. It is \hyperref[def:concrete_category]{concrete} over the category of trees.
  \end{thmenum}
\end{definition}
\begin{comments}
  \item \incite[\S 9.2.1]{Новиков2013} calls a \enquote{rooted tree} (\enquote{корневое дерево}) what we call an \hyperref[def:oriented_tree]{oriented tree}. The difference is inessential due to \fullref{thm:oriented_and_rooted_trees}.
\end{comments}

\begin{definition}\label{def:oriented_tree}\mimprovised
  An \term[bg=ориентирано дърво (\cite[21]{Мирчев2001}), ru=ориентированное дерево (\cite[def. 5.6]{БелоусовТкачёв2004}), en=oriented tree (\cite[373]{Knuth1997Vol1})]{oriented tree} or \term[en=arborescence (\cite[\S 3.1]{GondranMinoux1984Graphs})]{arborescence} is an \hyperref[def:well_founded_graph]{well-founded} \hyperref[def:directed_graph]{simple directed graph} \( T = (V, A) \) in which one distinguished vertex, called the \term{root}, has \hyperref[def:graph_cardinality/directed_degree]{in-degree} \( 0 \), and all other vertices have in-degree \( 1 \).
\end{definition}
\begin{comments}
  \item The terminology from \fullref{def:rooted_tree} applies to oriented trees because of \fullref{thm:oriented_and_rooted_trees/orientation}.

  \item \incite[def. 5.6]{БелоусовТкачёв2004} require oriented trees to be acyclic rather than well-founded, however that definition leads to pathological infinite oriented trees --- see \fullref{ex:infinite_oriented_tree}.

  \item Another possible convention for oriented trees is to place the same restrictions on out-degrees rather than in-degrees, which essentially reverses the directions of the arcs. This is the approach taken by \incite[373]{Knuth1997Vol1}.
\end{comments}

\begin{example}\label{ex:infinite_oriented_tree}
  The requirement for well-foundedness in \fullref{def:oriented_tree} is essential. Consider the graph
  \begin{equation}\label{eq:ex:infinite_oriented_tree}
    \begin{aligned}
      \includegraphics[page=1]{output/ex__infinite_oriented_tree}
    \end{aligned}
  \end{equation}

  If we were to follow \incite[def. 5.6]{БелоусовТкачёв2004} and require oriented trees to be merely acyclic, then \eqref{eq:ex:infinite_oriented_tree} would satisfy the definition. For finite graphs acyclicity is sufficient due to \fullref{thm:def:well_founded_graph/finite_acyclic}.

  Another option would be to require oriented trees to be acyclic, but forbid infinite paths.
\end{example}

\begin{lemma}\label{thm:tree_order_well_founded}
  The \hyperref[def:tree/order]{tree order} of a rooted tree is \hyperref[def:well_founded_relation]{well-founded} as a binary relation.
\end{lemma}
\begin{proof}
  Consider any descending sequence of nodes
  \begin{equation*}
    \cdots \leq v_n \leq v_{n-1} \leq \cdots \leq v_1 \leq v_0.
  \end{equation*}

  By definition, \( v_k \) lies on the path from \( r \) to \( v_0 \) for every \( k \geq 0 \). But the path is finite, thus, by \fullref{thm:pigeonhole_principle}, the sequence \hyperref[def:stabilizing_sequence]{stabilizes}.
\end{proof}

\begin{proposition}\label{thm:oriented_and_rooted_trees}
  \hfill
  \begin{thmenum}
    \thmitem{thm:oriented_and_rooted_trees/forgetful} For every \hyperref[def:oriented_tree]{oriented tree} \( T \), its underlying undirected graph \hyperref[def:graph_functors/simple_forgetful]{\( U_S \)}\( (T) \) is a \hyperref[def:tree]{tree}, and the root of \( T \) makes it a \hyperref[def:rooted_tree]{rooted tree}.

    \thmitem{thm:oriented_and_rooted_trees/orientation} Every rooted tree \( T \) is an oriented tree with respect to the \hyperref[def:multigraph_orientation]{orientation} in which \( u \to v \) if \( u \leq v \) with respect to the \hyperref[def:rooted_tree/order]{tree order}.
  \end{thmenum}
\end{proposition}
\begin{proof}
  \SubProofOf{thm:oriented_and_rooted_trees/forgetful} Let \( T = (V, A) \) be an oriented tree with root \( r \).

  \SubProof*{Proof of connectedness} Fix any vertex \( v \).

  Let \( v_0 \coloneqq v \). We will build a path
  \begin{equation*}
    r = v_n \to \cdots \to v_1 \to v_0 = v.
  \end{equation*}

  At step \( k \), if \( v_k = r \), we are done. Otherwise, the in-degree of \( v_k \) is \( 1 \), and thus there exists exactly one vertex \( v_{k+1} \) such that \( (v_{k+1}, v_k) \) is an arc. Furthermore, \( v_{k+1} \) cannot be among \( v_k, \ldots, v_0 \) because \( T \) is acyclic, thus at every step
  \begin{equation*}
    v_{k+1} \to v_k \to \cdots \to v_1 \to v_0 = v
  \end{equation*}
  is a path.

  Since \( T \) is well-founded, there exists an index \( n \) such that \( v_n \) has in-degree zero. But there is only one such vertex --- precisely \( r \).

  \SubProof*{Proof of acyclicity} Since \( T \) is well-founded, \fullref{thm:def:well_founded_graph/acyclic} implies that it is also acyclic.

  Suppose that the underlying undirected graph \( U_S(T) \) has an (undirected) cycle
  \begin{equation*}
    v_0, e_1, \ldots, e_n, v_n.
  \end{equation*}

  Then, since \( T \) is acyclic, at least one of the arcs \( e_1, \ldots, e_n \) in this (undirected) cycle must be negatively oriented. Furthermore, at least one of the arcs is positively oriented --- otherwise
  \begin{equation*}
    v_n, e_n, \cdots, e_1, v_0
  \end{equation*}
  would be a directed cycle in \( T \).

  Let \( e_k \) be the last positively oriented arc.
  \begin{itemize}
    \item If \( k < n \), then \( e_{k+1} \) is negatively oriented. Thus, \( e_k \) connects \( v_{k-1} \) to \( v_k \), while \( e_{k+1} \) connects \( v_{k+1} \) to \( v_k \). Then \( v_k \) has in-degree at least \( 2 \), which contradicts our assumption that \( T \) is an oriented tree.

    \item If \( k = n \), we can instead consider the last negatively oriented arc and similarly reach a contradiction.
  \end{itemize}

  From the assumption that \( U_S(T) \) has an undirected cycle, we have reached a contradiction, thus it remains for \( U_S(T) \) to be acyclic.

  \SubProofOf{thm:oriented_and_rooted_trees/orientation} Let \( T = (V, E, r) \) be a rooted tree and let \( D \) be its orientation in which \( u \to v \) if \( u \leq v \).

  \SubProof*{Proof of well-foundedness} By \fullref{thm:tree_order_well_founded}, the tree order is well-founded. \Fullref{thm:def:well_founded_graph/simple} implies that \( D \) it well-founded as a graph.

  \SubProof*{Proof of degree conditions} The arcs ending at \( v \) are precisely those connecting the parent of \( v \) to \( v \) itself. There is one parent for every node except for the root, which has none. Thus, the root has in-degree \( 0 \), while the other vertices have in-degree \( 1 \).
\end{proof}

\begin{theorem}[Induction on rooted trees]\label{thm:induction_on_rooted_trees}\mimprovised
  In order to prove a statement for every finite \hyperref[def:rooted_tree]{rooted tree}, it is sufficient to do the following:
  \begin{displayquote}
    Given a finite rooted tree \( T \), suppose that the statement holds for all its \hyperref[def:rooted_tree/immediate_subtree]{immediate subtrees} and prove the statement for \( T \) itself.
  \end{displayquote}

  Generalizing on \( T \), we can conclude that the statement holds for all finite rooted trees.
\end{theorem}
\begin{comments}
  \item Unlike for most induction principles discussed in \fullref{con:induction}, we did not formulate this one via logical formulas since that would make the statement unnecessarily convoluted with little gain.

  \item We will mostly use this for \hyperref[con:abstract_syntax_tree]{abstract syntax trees} or even rely on \hyperref[con:evaluation]{pattern matching} to avoid working with trees directly.

  \item We will usually deal with only certain kinds of rooted trees, for example \hyperref[def:propositional_formula_ast]{propositional formula ASTs}. In this case the statement should be conditional on the tree \( T \) being a propositional AST.
\end{comments}
\begin{proof}
  By \fullref{thm:tree_order_well_founded}, the \hyperref[def:rooted_tree]{tree order} is well-founded. This allows us to reduce the theorem to \fullref{thm:well_founded_induction}.
\end{proof}

\paragraph{Ordered trees}

\begin{definition}\label{def:ordered_tree}\mimprovised
  An \term[ru=упорядоченное дерево (\cite[\S 9.3.5]{Новиков2013}), en=ordered tree (\cite[639]{Rosen1999})]{ordered tree} is a quadruple \( T = (V, E, r, \leq) \), where \( (V, E, r) \) is a \hyperref[def:rooted_tree]{rooted tree} and \( {\leq} \) is a \hyperref[def:partially_ordered_set]{partial order}, called the \term{sibling order} or \term{horizontal order}, such that the children of every node are \hyperref[def:totally_order_chain]{totally ordered}, while non-siblings are pairwise-incomparable.

  \begin{thmenum}[series=def:ordered_tree]
    \thmitem{def:ordered_tree/leftmost_rightmost} We say that the child \( v \) of \( u \) is \term{leftmost} (resp. \term{rightmost}) if it is minimal (resp. maximal) with respect to \( {\leq} \). If \( u \) has only two children, we call them \term{left} and \term{right}.
  \end{thmenum}

  Ordered trees have the following metamathematical properties:
  \begin{thmenum}
    \thmitem{def:ordered_tree/homomorphism} A \term{homomorphism} between ordered trees is a \hyperref[def:order_function/preserving]{strictly order-preserving} \hyperref[def:rooted_tree/homomorphism]{rooted tree homomorphism}.

    \thmitem{def:ordered_tree/category} For a \hyperref[def:grothendieck_universe]{Grothendieck universe} \( \mscrU \), we consider the category of \( \mscrU \)-small ordered trees with ordered tree homomorphisms. It is \hyperref[def:concrete_category]{concrete} over both the category of rooted trees and the \hyperref[def:partially_ordered_set]{category of partially ordered sets}.
  \end{thmenum}
\end{definition}
\begin{comments}
  \item \incite[639]{Rosen1999} provides a similar definition, however he doesn't explicitly deal with non-sibling elements. We want to be explicit about this or order to strictly differentiate between vertical and horizontal ordering.

  \item \incite[\S 2.3]{Holtkamp2011} uses \enquote{planar rooted tree} as a synonym for \enquote{ordered tree}.
\end{comments}

\begin{definition}\label{def:ordered_tree_grafting}\mimprovised
  Fix an \hyperref[def:ordered_tree]{ordered tree} \( T \) and a leaf node \( v \) of \( T \). Fix also a family \( \seq{ T_k }_{k \in \mscrK} \) of ordered trees, indexed by a \hyperref[def:totally_ordered_set]{totally ordered set} \( \mscrK \). We will define a tree \( B_+^v(\seq{ T_k }_{k \in \mscrK}) \) as follows:
  \begin{thmenum}
    \thmitem{def:ordered_tree_grafting/union} Consider the \hyperref[def:graph_disjoint_union]{graph disjoint union} \( \parens[\Big]{ \coprod_{k \in \mscrK} T_k } \amalg T \), with the \hyperref[thm:order_category_isomorphism_properties/coproduct]{disjoint union partial order} on its nodes.

    \thmitem{def:ordered_tree_grafting/root} For every \( k \in \mscrK \), add an edge from \( \iota(v) \) to \( \iota_k(r_k) \), where \( r_k \) is the root of \( T_k \) and \( \iota \) and \( \iota_k \) are inclusions of \( T \) and of \( T_k \) into the disjoint union.

    \thmitem{def:ordered_tree_grafting/order} Extend the sibling order so that \( \iota_k(r_k) < \iota_m(r_m) \) whenever \( k < m \).
  \end{thmenum}

  We say that \( B_+^v(\seq{ T_k }_{k \in \mscrK}) \) is obtained by \term{grafting} \( \seq{ T_k }_{k \in \mscrK} \) onto \( v \).
\end{definition}
\begin{comments}
  \item In case the trees are \hyperref[def:labeled_tree]{labeled}, we must label the vertex \( \iota_k(u) \) in \( B_+^v(\seq{ T_k }_{k \in \mscrK}) \) based on its label in \( T_k \), and similarly for the labels of \( T \) itself.

  \item We base our terminology and notation on \enquote{grafting products} defined in \fullref{def:ordered_tree_grafting_product}.
\end{comments}

\begin{remark}\label{rem:tree_grafting_nodes}
  When building \hyperref[def:ordered_tree]{ordered trees} recursively, for example in \fullref{def:propositional_formula_ast}, \fullref{def:lambda_term_ast} or \fullref{def:natural_deduction_proof_tree}, we start with singleton trees and then graft existing trees onto new roots.

  We choose \( 0 = \varnothing \) as a \enquote{canonical} node for singleton trees. It is a natural candidate and, since we will mostly graft lists of trees onto it, due to how define disjoint unions, each node in such a tree will be tuple of natural numbers.

  The actual nodes do not matter when the nodes are \hyperref[def:labeled_set]{labeled}, which motivates \fullref{def:canonical_singleton_tree}.

  Thus, the canonical abstract syntax tree for the propositional formula \( ((\synp \synvee \synq) \rightarrow \syn r) \) is
  \begin{equation*}
    \begin{aligned}
      \includegraphics[page=1]{output/rem__tree_grafting_nodes}
    \end{aligned}
  \end{equation*}
  which without labels becomes
  \begin{equation*}
    \begin{aligned}
      \includegraphics[page=2]{output/rem__tree_grafting_nodes}
    \end{aligned}
  \end{equation*}

  This leads to \fullref{def:ordered_tree_grafting_product}.
\end{remark}

\begin{definition}\label{def:ordered_tree_grafting_product}\mimprovised
  For any family \( \seq{ T_k }_{k \in \mscrK} \) of \hyperref[def:ordered_tree]{ordered trees} indexed by a \hyperref[def:totally_ordered_set]{totally ordered set} \( \mscrK \), we will define their \term{grafting product} \( B_+(\seq{ T_k }_{k \in \mscrK}) \) as the tree obtained by \hyperref[def:labeled_tree_grafting]{grafting} \( \seq{ T_k }_{k \in \mscrK} \) onto a singleton tree with root \( 0 \).
\end{definition}
\begin{comments}
  \item Our choice of \( 0 \) for the new root is discussed in \fullref{rem:tree_grafting_nodes}.

  \item We base our terminology on \cite[\S 3.6]{Holtkamp2011} and our notation on the earlier \cite[9]{Hoffman2002}.
\end{comments}

\begin{definition}\label{def:n_ary_tree}\mcite[642]{Rosen1999}
  We say that the \hyperref[def:ordered_tree]{ordered tree} is \term{\( n \)-ary} if each node has \hi{at most} \( n \) children and \term{full \( n \)-ary} if each non-leaf node has \hi{exactly} \( n \) children.

  For some small values of \( n \), we use the adjectives from \fullref{def:operation_arity} (e.g. unary trees, binary trees).
\end{definition}

\begin{definition}\label{def:ordered_tree_enumeration}
  Fix a finite \hyperref[def:ordered_tree]{ordered tree} \( T = (V, E, r, \leq) \). Let \( R_1, \ldots, R_n \) be the immediate subtrees of \( T \). We will define two \hyperref[def:enumeration]{enumeration} of \( V \) --- the \term{pre-order enumeration}\fnote{This concept is not related to \hyperref[def:preordered_set]{preordered sets}.}
  \begin{equation}\label{eq:def:ordered_tree_enumeration/pre}
    \op{pre}(T) \coloneqq \parens[\Big]{ r, \op{pre}(R_1), \cdots, \op{pre}(R_n) }.
  \end{equation}
  visits each root before the children, while the \term{post-order enumeration}
  \begin{equation}\label{eq:def:ordered_tree_enumeration/post}
    \op{post}(T) \coloneqq \parens[\Big]{ \op{post}(R_1), \cdots, \op{post}(R_n), r }.
  \end{equation}
  visits the root after them.

  Additionally, if \( T \) is a \hyperref[def:n_ary_tree]{full binary tree}, we can also define the \term{in-order enumeration}
  \begin{equation}\label{eq:def:ordered_tree_enumeration/in}
    \op{in}(T) \coloneqq \parens[\Big]{ \op{in}(R_1), r, \op{in}(R_2) }.
  \end{equation}

  \begin{figure}[!ht]
    \centering
    \includegraphics[page=1]{output/def__ordered_tree_enumeration}
    \caption{A full binary tree whose \hyperref[eq:def:ordered_tree_enumeration/pre]{pre-order enumeration} is \( abcde \), whose \hyperref[eq:def:ordered_tree_enumeration/post]{post-order enumeration} is \( cdbea \) and whose \hyperref[eq:def:ordered_tree_enumeration/in]{in-order enumeration} is \( cbdae \).}
    \label{fig:def:ordered_tree_enumeration}
  \end{figure}
\end{definition}
\begin{comments}
  \item We base our definition on \cite[228]{Erickson2019}, however instead of \enquote{pre-order traversal}, which has temporal connotations, we use \enquote{pre-order enumeration}.
\end{comments}

\begin{algorithm}[Finite ordered tree to binary tree]\label{alg:finite_ordered_tree_to_binary_tree}
  For every finite \hyperref[def:ordered_tree]{ordered tree} \( T \), we will build a \hyperref[def:n_ary_tree]{binary tree} \( B \) that has the same \hyperref[eq:def:ordered_tree_enumeration/pre]{pre-order enumeration}.

  Let \( v_1, \ldots, v_n \) be the pre-order enumeration of \( T \). We will build via recursion on \( k = 1, 2, \ldots \) a binary tree \( B_k \) enumerating \( v_1, \ldots, v_{2k} \) if \( n = 2k \) and \( v_1, \ldots, v_{2k} v_{2k+1} \) otherwise.

  \begin{figure}[!ht]
    \hfill
    \includegraphics[page=1]{output/alg__finite_ordered_tree_to_binary_tree}
    \hfill
    \includegraphics[page=2]{output/alg__finite_ordered_tree_to_binary_tree}
    \hfill
    \hfill
    \caption{The two possibilities in \fullref{alg:finite_ordered_tree_to_binary_tree/step}.}
    \label{fig:alg:finite_ordered_tree_to_binary_tree}
  \end{figure}

  \begin{thmenum}
    \thmitem{alg:finite_ordered_tree_to_binary_tree/base} Define \( B_1 \) to be the singleton tree \( v_1 \). If \( n = 1 \), yield \( B_1 \) as the result of the algorithm.

    \thmitem{alg:finite_ordered_tree_to_binary_tree/step} At step \( k > 1 \), having already constructed \( B_{k-1} \), we have two possibilities:
    \begin{thmenum}
      \thmitem{alg:finite_ordered_tree_to_binary_tree/even} If \( n = 2k \), define \( B_k \) by \hyperref[def:ordered_tree_grafting]{grafting} \( v_n \) onto \( v_{2k-1} \) and yield \( B_k \) as the result of the algorithm.

      \thmitem{alg:finite_ordered_tree_to_binary_tree/odd} Otherwise, \( n < 2k \), define \( B_k \) by grafting \( v_{2k} \) and \( v_{2k+1} \) onto \( v_{2k-1} \).

      If \( n = 2k+1 \), yield \( B_k \) as the result of the algorithm.
    \end{thmenum}
  \end{thmenum}
\end{algorithm}

\paragraph{Labeled trees}

\begin{definition}\label{def:labeled_tree}\mimprovised
  An \( L \)-\term{labeled tree} is a quintuple \( T = (V, E, r, \leq, l) \), where \( (V, E, r, \leq) \) is an \hyperref[def:ordered_tree]{ordered tree} and \( l: V \to L \) is a \hyperref[def:labeled_set]{labeling} of the vertices.

  Labeled trees have the following metamathematical properties:
  \begin{thmenum}
    \thmitem{def:labeled_tree/homomorphism} A \term{homomorphism} between \( L \)-labeled trees is an \hyperref[def:ordered_tree/homomorphism]{ordered tree homomorphism} that also preserves labels.

    \thmitem{def:labeled_tree/category} For a \hyperref[def:grothendieck_universe]{Grothendieck universe} \( \mscrU \), we can consider the category of \( \mscrU \)-small \( L \)-labeled trees with \( L \)-labeled tree homomorphisms. It is \hyperref[def:concrete_category]{concrete} over the category of ordered trees.
  \end{thmenum}
\end{definition}
\begin{comments}
  \item We introduce this definition mostly for having an explicit concept for isomorphic \hyperref[con:abstract_syntax_tree]{abstract syntax tree}.

  \item For \hyperref[rem:arbitrary_kind_graph]{general graphs} and even \hyperref[def:tree]{general trees} we allow both vertices and arcs/edges to be labeled, however in the case of ordered trees we are mostly interested in the vertices.

  We have chosen the term \enquote{labeled tree} for brevity; \incite[471]{Mimram2020} and \incite[655]{Rosen1999} use the same term for generic trees with labeled nodes, while \incite[\S 2.8]{Holtkamp2011} uses it for rooted trees with labeled nodes.
\end{comments}

\begin{definition}\label{def:canonical_singleton_tree}\mimprovised
  For every possible \hyperref[def:labeled_set]{label} \( l \), there corresponds a singleton tree with node \( 0 \) and label \( l \). We call this the \term{canonical singleton tree} labeled by \( l \).

  The choice of node \( 0 \) is discussed in \fullref{rem:tree_grafting_nodes}.
\end{definition}
