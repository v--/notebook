\subsection{Trees}\label{subsec:trees}

\begin{definition}\label{def:tree}
  We call the possibly infinite \hyperref[def:undirected_graph]{simple undirected graph} \( T = (E, V) \) a \term{tree} if any of the following equivalent conditions hold:
  \begin{thmenum}
    \thmitem{def:tree/connected_acyclic} \( T \) is \hyperref[def:graph_connectedness/undirected]{connected} and \hyperref[def:acyclic_graph]{acyclic}.

    \thmitem{def:tree/maximally_acyclic} \( T \) is \term{maximally acyclic}, meaning that adding an edge between existing vertices would create a \hyperref[def:graph_cycle]{cycle}.

    \thmitem{def:tree/minimally_connected} \( T \) is \term{minimally connected}, meaning that removing an edge will make the graph \hyperref[def:graph_connectedness/undirected]{disconnected}.

    \thmitem{def:tree/single_path} There exists a unique \hyperref[def:graph_walk/path]{path} connecting every pair of distinct vertices in \( T \).
  \end{thmenum}
\end{definition}
\begin{comments}
  \item We will refer to vertices in \hyperref[def:tree]{trees} as \term{nodes}.
\end{comments}
\begin{proof}
  \ImplicationSubProof{def:tree/connected_acyclic}{def:tree/maximally_acyclic} Suppose that \( T \) is connected and acyclic.

  \begin{itemize}
    \item If \( T \) has a single node, it has no edges, there is nothing to prove.
    \item If \( T \) has two nodes, it has an edge between them, and there is again nothing to prove.
    \item Suppose that \( T \) has at least three nodes. Since \( T \) is connected, there must exist some path
    \begin{equation*}
      a \to b \to c
    \end{equation*}
    of length three. Since \( T \) is acyclic, there cannot be an edge between \( a \) and \( c \). Thus, it is always possible to add an edge to \( T \).

    Let \( T' \) be an arbitrary graph obtained by adjoining a single edge \( \set{ u, v } \) to \( T \). Since \( T \) is connected, there must exist a path from \( u \) to \( v \), and we can concatenate this new edge to obtain a cycle in \( T' \). Hence, \( T' \) is not acyclic.

    Since \( \set{ u, v } \) was arbitrary, we conclude that \( T \) is maximally acyclic.
  \end{itemize}

  \ImplicationSubProof{def:tree/maximally_acyclic}{def:tree/minimally_connected} Suppose that \( T \) is maximally acyclic. We will show that it is minimally connected.

  Suppose that \( T \) is not connected. Then there exist vertices \( a \) and \( b \) with no path between them. Let \( T' \) be the graph obtained by adjoining the edge \( \set{ a, b } \) to \( T \). Since \( T \) is acyclic, \( T' \) must also be acyclic. But this contradicts our assumption that \( T \) is maximally acyclic. Hence, \( T \) is connected.

  Now let \( \set{ u, v } \) be an edge in \( T \) and let \( T^\dprime \) be the subgraph obtained by removing this edge. Suppose that \( T^\dprime \) is connected, i.e. there exists a path from \( u \) to \( v \). But then we can add \( \set{ u, v } \) to this path, hence \( T \) has a cycle, contradicting the assumption that \( T \) is acyclic. Hence, no proper subgraph of \( T \) is connected.

  \ImplicationSubProof{def:tree/minimally_connected}{def:tree/single_path} Suppose that \( T \) is minimally connected.

  Since \( T \) is connected, there exists at least one path between any pair of distinct vertices. If there is more than one path, then we can remove an edge without affecting connectivity. But \( T \) is minimally connected.

  Hence, there exists exactly one path between any pair of distinct vertices

  \ImplicationSubProof{def:tree/single_path}{def:tree/connected_acyclic} Suppose that there exists a unique path connecting every pair of distinct vertices in \( T \). It is clear that \( T \) is connected.

  Aiming at a contradiction, suppose that \( T \) has a cycle
  \begin{equation*}
    v_0 \to v_1 \to \cdots \to v_{n-1} \to v_0.
  \end{equation*}

  Then
  \begin{equation*}
    v_0 \to v_1 \to \cdots \to v_{n-1}
  \end{equation*}
  is a path connecting \( v_0 \) to \( v_{n-1} \), and so is the single-edge path
  \begin{equation*}
    v_0 \to v_{n-1}.
  \end{equation*}

  We thus obtain two distinct paths from \( v_0 \) to \( v_{n-1} \), contradicting out initial assumption.

  Therefore, \( T \) is also acyclic in addition to being connected.
\end{proof}

\begin{definition}\label{def:rooted_tree}\mcite[15]{Diestel2005}
  A \term[ru=корневое дерево (\cite[37]{Карпов2017})]{rooted tree} is a triple \( T = (V, E, r) \), where \( (V, E) \) is a \hyperref[def:tree]{tree} and \( r \) is a distinguished node, called the \term{root}.

  As per \fullref{def:tree/single_path}, every node \( v \) has a unique path from \( r \).

  \begin{thmenum}
    \thmitem{def:rooted_tree/order} We define the (nonstrict) \hyperref[def:partially_ordered_set]{partial order} \( u \leq v \) to hold if \( u \) lies on the path from \( r \) to \( v \). We call \( \leq \) the \term{tree order} of \( T \).

    \thmitem{def:rooted_tree/height} The \term{height} of a node \( v \) is the length of the unique path from \( r \). This is precisely the height of \( v \) in the sense of \fullref{def:partial_order_element_height}.

    \thmitem{def:rooted_tree/ancestor_descendant}\mcite[637]{Rosen1999} If \( u \leq v \), we say that \( u \) is an \term{ancestor} of \( v \) and that \( v \) is a \term{descendant} of \( u \).

    \thmitem{def:rooted_tree/parent_child}\mcite[637; 639]{Rosen1999} If \( u \) is the penultimate element in the path from \( r \) to \( v \), we say that \( u \) is a \term{parent} of \( v \) and that \( v \) is a \term{child} of \( u \).

    \thmitem{def:rooted_tree/leaf}\mcite[638]{Rosen1999} If a node has no children, we call it a \term[ru=лист (\cite[25]{Карпов2017})]{leaf node}.

    \thmitem{def:rooted_tree/siblings}\mcite[639]{Rosen1999} If \( u \) and \( v \) has the same parent, we call them \term{siblings}.
  \end{thmenum}
\end{definition}

\paragraph{Multiway trees}

\begin{remark}\label{rem:multiway_trees}
  The term \enquote{tree} refers to slightly different concepts in graph theory, where \fullref{def:tree} is ubiquitous, and in informatics, where the approach we will present is more useful.

  Consider the \hyperref[con:abstract_syntax_tree]{abstract syntax tree} of the axiom schema \eqref{eq:def:minimal_implication_logic/dist}:
  \begin{equation}\label{eq:rem:multiway_trees/dist}
    \begin{aligned}
      \includegraphics[page=1]{output/rem__multiway_trees}
    \end{aligned}
  \end{equation}

  Let us analyze the features that \fullref{def:tree} lacks in order to represent it:
  \begin{itemize}
    \item The above tree has a distinguished root and leaves marked with formula schemas. Graph-theoretic trees have no distinguished root, although this problem is easily solved by considering \hyperref[def:rooted_tree]{rooted trees} instead.
    \item The nodes of every child are ordered by their horizontal position in the drawing. We can mitigate this for graph-theoretic trees by introducing an ordering that is total on the children of any node.
    \item The same node cannot be a child of two distinct nodes because that would introduce a \hyperref[def:graph_cycle]{cycle}. This can be mitigated by instead considering trees whose nodes are natural numbers, and the actual values we are interested in representing are \hyperref[def:labeled_set]{labels}.
  \end{itemize}

  The first two points require some adjustments to the definition, but the third has an important consequence that is not easy to mitigate --- we lose uniqueness because multiple distinct trees may have the same labels, and we rely on such trees being identical.

  Furthermore, the kinds of trees we are interested in representing in this way are built recursively, with a focus on inductive reasoning.

  This leads to \fullref{def:multiway_tree}, which provides a very different outlook on trees, but also provides a simple reduction to graph-theoretic trees.
\end{remark}

\begin{definition}\label{def:multiway_tree}\mimprovised
  Suppose that we are given a \hyperref[def:grothendieck_universe]{Grothendieck universe} \( \mscrU \), which is safe to assume to be the smallest suitable one as explained in \fullref{def:large_and_small_sets}.

  Fix a set \( X \) in \( \mscrU \), whose elements we will call \term{values}, and consider the following operator:
  \begin{equation*}
    \begin{aligned}
      &R: \mscrU \to \mscrU \\
      &R(A) \coloneqq X \times A^*,
    \end{aligned}
  \end{equation*}
  where \( A^* \) is the \hyperref[def:formal_language/kleene_star]{Kleene star} of \( A \).

  A \term{multiway tree} over \( X \) is a member of the least fixed point, in the sense of \fullref{thm:knaster_tarski_iteration/continuous}, of \( R \). More concretely, a multiway tree is an \( (n + 1) \)-tuple \( T = (x, T_1, \ldots, T_n) \), where \( x \) is a value and \( T_1, \ldots, T_n \) are other trees. We will refer to these trees as the \term{immediate subtrees} of \( T \).

  \begin{thmenum}
    \thmitem{def:multiway_tree/subtree} We will say that \( T' \) is a \term{subtree} of \( T \) if it is either \( T \) itself, an immediate subtree (i.e. a non-value tuple member) of \( T \), or, recursively, a subtree of an immediate subtree of \( T \).

    \thmitem{def:multiway_tree/rooted} We can regard a multiway tree \( T \) as a graph-theoretic \hyperref[def:rooted_tree]{rooted tree} as follows:
    \begin{itemize}
      \item The set of nodes (vertices) is the set of subtrees of \( T \), the root being \( T \) itself.
      \item Two nodes are connected by an edge if either is an immediate subtree of the other.
    \end{itemize}

    \thmitem{def:multiway_tree/occurrence} An \term{occurrence} of the value \( a \) in \( T \) is simply a node (subtree) of \( T \) whose value is \( v \).
  \end{thmenum}
\end{definition}
\begin{comments}
  \item We use the adjective \enquote{multiway} to contrast with trees with an unrestricted number of immediate subtrees, similarly to \incite[561]{Savage1998}. We needed a specific adjective to distinguish multiway trees from graph-theoretic trees.

  \item Although multiway trees have a very distinct definition from graph-theoretic trees, \fullref{def:multiway_tree/rooted} allows us to use the machinery of \fullref{subsec:trees} where appropriate. Of course, the definitions \fullref{def:multiway_tree/leaf} and \fullref{def:rooted_tree/leaf} for a leaf coincide.
\end{comments}
\begin{defproof}
  \SubProof{Proof that \( R \) is \hyperref[def:scott_continuity]{Scott-continuous}} We need this in order to justify our usage of \fullref{thm:knaster_tarski_iteration/continuous}.

  Fix an upward-directed family \( \mscrA \) of sets in \( \mscrU \). If \( (r, T_1, \ldots, T_n) \in R(\bigcup \mscrA) \), then \( T_1, \ldots, T_n \) are elements of \( \bigcup \mscrA \). Denote by \( A_i \) the set towards which \( T_i \) belongs. Then, since \( \mscrA \) is upward-directed, there exists a set \( A \) containing \( A_1, \ldots, A_n \), and \( (r, T_1, \ldots, T_n) \in R(A) \).

  \SubProof{Proof that the subtrees of \( T \) form a rooted tree} It is easiest to show \fullref{def:tree/single_path} --- there is indeed a unique path from the root to any node.
\end{defproof}

\begin{definition}\label{def:traversal_ordering}\mcite[228]{Erickson2019}
  Fix a finite \hyperref[def:multiway_tree]{multiway tree} \( T \) and let \( V \) be its family of nodes. We will define two \hyperref[def:well_ordered_set]{well-orderings} on \( V \) obtained by \enquote{traversing} \( T \).

  A \term{pre-order traversal}\fnote{This concept is not related to \hyperref[def:preordered_set]{preordered sets}.} (see \ref{def:traversal_ordering/step/pre}) visits each root before the children, while a \term{post-order traversal} (see \ref{def:traversal_ordering/step/post}) visits the root after them.

  The relation \( \leq \) can be defined via recursion on the cardinality \( n \) of \( V \) as follows:
  \begin{thmenum}
    \thmitem{def:traversal_ordering/init} If \( n = 1 \), the ordering is trivial --- let \( v \leq v \).
    \thmitem{def:traversal_ordering/step} Suppose that we have already built a well-ordering for trees with fewer than \( n \) nodes, and that \( T \) has exactly \( n \) nodes. Let \( r \) be the root and \( T_1, \ldots, T_m \) be the immediate descendants of \( T \), that is,
    \begin{equation*}
      \includegraphics[page=1]{output/def__traversal_ordering}
    \end{equation*}

    For \( k = 1, \ldots, m \) denote by \( \leq_i \) the recursively defined ordering on \( T_i \), and by \( V_i \) --- the set of nodes.

    Define the relation \( v \leq w \) on \( V \) to hold in the following cases:
    \begin{thmenum}
      \thmitem{def:traversal_ordering/step/same} If both \( v \) and \( w \) belong to \( V_i \) and \( x \leq_i y \).
      \thmitem{def:traversal_ordering/step/different} If \( v \in V_i \) and \( w \in V_j \) with \( i < j \).
      \thmitem{def:traversal_ordering/step/root} If \( v = w = r \).
      \thmitem{def:traversal_ordering/step/pre} If \( v = w \) in case we want a pre-order.
      \thmitem{def:traversal_ordering/step/post} If \( v = r \) in case we want a post-order.
    \end{thmenum}
  \end{thmenum}

  \begin{figure}[!ht]
    \centering
    \includegraphics[page=2]{output/def__traversal_ordering}
    \caption{An ordered tree whose \hyperref[def:traversal_ordering]{pre-order traversal} is \( abcde \) and whose \hyperref[def:traversal_ordering]{post-order traversal} is \( cdbea \).}
    \label{fig:def:traversal_ordering}
  \end{figure}
\end{definition}
\begin{defproof}
  We will show via induction on \( n \) that \( \leq \) is a well-order. Since the trivial case \( n = 1 \) is vacuous, suppose that \( n > 1 \).

  \SubProofOf[def:binary_relation/reflexive]{reflexivity} Follows from \fullref{def:traversal_ordering/step/root}.

  \SubProofOf[def:binary_relation/antisymmetric]{antisymmetry} Suppose that \( v \leq w \) and \( w \leq v \) with \( v \neq w \).
  \begin{itemize}
    \item If both \( v \) and \( w \) belong to \( V_i \), antisymmetry follows from the inductive hypothesis.
    \item If \( v \in V_i \) and \( w \in V_j \) with \( i < j \), then we cannot possibly have \( v \geq_W w \).
    \item If \( v = r \), then \( r \leq w \); this is pre-order traversal, hence we cannot have \( r \geq w \).
    \item We handle post-order traversal similarly.
  \end{itemize}

  \SubProofOf[def:binary_relation/transitive]{transitivity} Suppose that \( u \leq v \) and \( v \leq w \) with all three elements distinct.
  \begin{itemize}
    \item If all three belong to \( V_i \), antisymmetry follows from the inductive hypothesis.
    \item If \( u = r \), then \( r \leq v \); this is pre-order traversal, hence by definition \( u \leq w \).
    \item If \( w = r \), then \( v \leq w \); this is post-order traversal, hence by definition \( u \leq w \).
    \item We cannot have \( v = r \).
    \item If \( x, v \in V_i \) and \( w \in V_j \) with \( i < j \), then \( u \leq w \) by \ref{def:traversal_ordering/step/different}.
    \item If \( u \in V_i \) and \( y, w \in V_j \) with \( i < j \), then also \( u \leq w \) by \ref{def:traversal_ordering/step/different}.
  \end{itemize}

  \SubProofOf[def:well_founded_relation]{well-foundedness} Follows from \fullref{thm:def:well_ordered_set/finite_chain}.
\end{defproof}
