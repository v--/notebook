\subsection{Trees}\label{subsec:trees}

\begin{remark}\label{rem:vertices_called_nodes}
  It is conventional to refer to vertices in \hyperref[def:tree]{trees} as \term{nodes}.
\end{remark}

\begin{definition}\label{def:tree}
  We call the possibly infinite \hyperref[def:undirected_graph]{simple undirected graph} \( T = (E, V) \) a \term{tree} if any of the following equivalent conditions hold:
  \begin{thmenum}
    \thmitem{def:tree/connected_acyclic} \( T \) is \hyperref[def:graph_connectedness/undirected]{connected} and \hyperref[def:acyclic_graph]{acyclic}.

    \thmitem{def:tree/maximally_acyclic} \( T \) is \term{maximally acyclic}, meaning that adding an edge between existing vertices would create a \hyperref[def:graph_cycle]{cycle}.

    \thmitem{def:tree/minimally_connected} \( T \) is \term{minimally connected}, meaning that removing an edge will make the graph \hyperref[def:graph_connectedness/undirected]{disconnected}.

    \thmitem{def:tree/single_path} There exists a unique \hyperref[def:graph_walk/path]{path} connecting every pair of distinct vertices in \( T \).
  \end{thmenum}
\end{definition}
\begin{proof}
  \ImplicationSubProof{def:tree/connected_acyclic}{def:tree/maximally_acyclic} Suppose that \( T \) is connected and acyclic.

  \begin{itemize}
    \item If \( T \) has a single node, it has no edges, there is nothing to prove.
    \item If \( T \) has two nodes, it has an edge between them, and there is again nothing to prove.
    \item Suppose that \( T \) has at least three nodes. Since \( T \) is connected, there must exist some path
    \begin{equation*}
      a \to b \to c
    \end{equation*}
    of length three. Since \( T \) is acyclic, there cannot be an edge between \( a \) and \( c \). Thus, it is always possible to add an edge to \( T \).

    Let \( T' \) be an arbitrary graph obtained by adjoining a single edge \( \set{ u, v } \) to \( T \). Since \( T \) is connected, there must exist a path from \( u \) to \( v \), and we can concatenate this new edge to obtain a cycle in \( T' \). Hence, \( T' \) is not acyclic.

    Since \( \set{ u, v } \) was arbitrary, we conclude that \( T \) is maximally acyclic.
  \end{itemize}

  \ImplicationSubProof{def:tree/maximally_acyclic}{def:tree/minimally_connected} Suppose that \( T \) is maximally acyclic. We will show that it is minimally connected.

  Suppose that \( T \) is not connected. Then there exist vertices \( a \) and \( b \) with no path between them. Let \( T' \) be the graph obtained by adjoining the edge \( \set{ a, b } \) to \( T \). Since \( T \) is acyclic, \( T' \) must also be acyclic. But this contradicts our assumption that \( T \) is maximally acyclic. Hence, \( T \) is connected.

  Now let \( \set{ u, v } \) be an edge in \( T \) and let \( T^\dprime \) be the subgraph obtained by removing this edge. Suppose that \( T^\dprime \) is connected, i.e. there exists a path from \( u \) to \( v \). But then we can add \( \set{ u, v } \) to this path, hence \( T \) has a cycle, contradicting the assumption that \( T \) is acyclic. Hence, no proper subgraph of \( T \) is connected.

  \ImplicationSubProof{def:tree/minimally_connected}{def:tree/single_path} Suppose that \( T \) is minimally connected.

  Since \( T \) is connected, there exists at least one path between any pair of distinct vertices. If there is more than one path, then we can remove an edge without affecting connectivity. But \( T \) is minimally connected.

  Hence, there exists exactly one path between any pair of distinct vertices

  \ImplicationSubProof{def:tree/single_path}{def:tree/connected_acyclic} Suppose that there exists a unique path connecting every pair of distinct vertices in \( T \). It is clear that \( T \) is connected.

  Aiming at a contradiction, suppose that \( T \) has a cycle
  \begin{equation*}
    v_0 \to v_1 \to \cdots \to v_{n-1} \to v_0.
  \end{equation*}

  Then
  \begin{equation*}
    v_0 \to v_1 \to \cdots \to v_{n-1}
  \end{equation*}
  is a path connecting \( v_0 \) to \( v_{n-1} \), and so is the single-edge path
  \begin{equation*}
    v_0 \to v_{n-1}.
  \end{equation*}

  We thus obtain two distinct paths from \( v_0 \) to \( v_{n-1} \), contradicting out initial assumption.

  Therefore, \( T \) is also acyclic in addition to being connected.
\end{proof}

\begin{example}\label{ex:def:tree}
  We list several examples of \hyperref[def:tree]{trees}:
  \begin{thmenum}
    \thmitem{ex:def:tree/integers} The reduced infinite integer graphs \eqref{eq:ex:infinite_integer_graphs/positive}, \eqref{eq:ex:infinite_integer_graphs/negative} and \eqref{eq:ex:infinite_integer_graphs/two_sided} from \fullref{ex:infinite_integer_graphs} are simple examples of infinite trees.

    \thmitem{ex:def:tree/proofs} We use \hyperref[def:proof_tree]{proof trees} in \fullref{subsec:deductive_systems}. Proof trees can have \hyperref[def:labeled_set]{labels} on their nodes and edges.

    \thmitem{ex:def:tree/syntax} In order to formalize logical formulas, we use \hyperref[def:parse_tree]{concrete} and \hyperref[rem:abstract_syntax_tree]{abstract syntax trees} from \hyperref[def:formal_grammar]{formal grammar}. Except for labels, they also have additional structure --- they are \hyperref[def:ordered_tree]{ordered}.
  \end{thmenum}
\end{example}

\begin{definition}\label{def:rooted_tree}\mcite[15]{Diestel2005}
  A \term[ru=корневое дерево (\cite[37]{Карпов2017})]{rooted tree} is a triple \( T = (V, E, r) \), where \( (V, E) \) is a \hyperref[def:tree]{tree} and \( r \) is a distinguished node, called the \term{root}.

  As per \fullref{def:tree/single_path}, every node \( v \) has a unique path from \( r \).

  \begin{thmenum}
    \thmitem{def:rooted_tree/order} We define the (nonstrict) \hyperref[def:partially_ordered_set]{partial order} \( u \leq v \) to hold if \( u \) lies on the path from \( r \) to \( v \). We call \( \leq \) the \term{tree order} of \( T \).

    \thmitem{def:rooted_tree/height} The \term{height} of a node \( v \) is the length of the unique path from \( r \). This is precisely the height of \( v \) in the sense of \fullref{def:partial_order_element_height}.

    \thmitem{def:rooted_tree/ancestor_descendant}\mcite[637]{Rosen1999} If \( u \leq v \), we say that \( u \) is an \term{ancestor} of \( v \) and that \( v \) is a \term{descendant} of \( u \).

    \thmitem{def:rooted_tree/parent_child}\mcite[637; 639]{Rosen1999} If \( u \) is the penultimate element in the path from \( r \) to \( v \), we say that \( u \) is a \term{parent} of \( v \) and that \( v \) is a \term{child} of \( u \).

    \thmitem{def:rooted_tree/leaf}\mcite[638]{Rosen1999} If a node has no children, we call it a \term[ru=лист (\cite[25]{Карпов2017})]{leaf node}.

    \thmitem{def:rooted_tree/siblings}\mcite[639]{Rosen1999} If \( u \) and \( v \) has the same parent, we call them \term{siblings}.
  \end{thmenum}
\end{definition}

\begin{definition}\label{def:ordered_tree}\mcite[exer. 3.52(b)]{Erickson2019}
  An \term{ordered tree} is a \hyperref[def:rooted_tree]{rooted tree} with a \hyperref[def:partially_ordered_set]{partial order} such that every set of \hyperref[def:rooted_tree/siblings]{siblings} is a \hyperref[def:partial_order_chain]{chain}.
\end{definition}

\begin{definition}\label{def:traversal_ordering}\mcite[228]{Erickson2019}
  We will define two \term{traversal orderings} on a given \hyperref[def:set_finiteness]{finiteness} \hyperref[def:ordered_tree]{ordered tree} \( T = (V, A, r, \leq) \). These are \hyperref[def:well_ordered_set]{well-orderings} \( \leq_W \) of the vertices of \( V \). A \term{pre-order traversal}\fnote{This concept is not related to \hyperref[def:preordered_set]{preordered sets}.} (see \ref{def:traversal_ordering/step/pre}) visits each root before the successors, while a \term{post-order traversal} (see \ref{def:traversal_ordering/step/post}) visits the root last. Equivalently, we can regard a traversal as a \hyperref[def:sequence]{finite sequence} of vertices rather than an ordering of \( V \).

  The relation \( \leq_W \) can be defined on \( \card(V) \) as follows:
  \begin{thmenum}
    \thmitem{def:traversal_ordering/init} If \( \card(V) = 1 \), then the ordering is trivial --- let \( r \leq_W r \).
    \thmitem{def:traversal_ordering/step} Suppose that \( \card(V) = n \) and that we have already built a well-order for all smaller trees. Let \( T_k = (V_k, A_k, r_k) \), \( k = 1, \ldots, n \) be rooted subtrees, whose roots \( r_1 < \ldots < r_n \) are successors of \( r \), that is,
    \begin{equation*}
      \begin{aligned}
        \includegraphics[page=1]{output/def__traversal_ordering}
      \end{aligned}
    \end{equation*}

    Let \( \leq_{W_k} \) be the recursively defined traversal ordering on \( T_k \).

    Define the relation \( x \leq_W y \) on \( V \) to hold:
    \begin{thmenum}
      \thmitem{def:traversal_ordering/step/same} If both \( x \) and \( y \) belong to \( V_i \) and \( x \leq_{W_i} y \).
      \thmitem{def:traversal_ordering/step/different} If \( x \in V_i \) and \( y \in V_j \) with \( i < j \).
      \thmitem{def:traversal_ordering/step/root} If \( x = y = r \).
      \thmitem{def:traversal_ordering/step/pre} If \( x = r \) in case we want a pre-order.
      \thmitem{def:traversal_ordering/step/post} If \( y = r \) in case we want a post-order.
    \end{thmenum}
  \end{thmenum}

  \begin{figure}[!ht]
    \centering
    \includegraphics[page=2]{output/def__traversal_ordering}
    \caption{An ordered tree whose \hyperref[def:traversal_ordering]{pre-order traversal} is \( abcde \) and whose \hyperref[def:traversal_ordering]{post-order traversal} is \( cdbea \).}
    \label{fig:def:traversal_ordering}
  \end{figure}
\end{definition}
\begin{defproof}
  We will show via induction on \( \card(V) \) that \( \leq_W \) is a well-order. Since the trivial case \( \card(V) = 1 \) is vacuous, suppose that \( \card(V) > 1 \).

  \SubProofOf[def:binary_relation/reflexive]{reflexivity} Follows from \fullref{def:traversal_ordering/step/root}.

  \SubProofOf[def:binary_relation/antisymmetric]{antisymmetry} Suppose that \( x \leq_W y \) and \( y \leq_W x \) with \( x \neq y \).
  \begin{itemize}
    \item If both \( x \) and \( y \) belong to \( V_i \), antisymmetry follows from the inductive hypothesis.
    \item If \( x \in V_i \) and \( y \in V_j \) with \( i < j \), then we cannot possibly have \( x \geq_W y \).
    \item If \( x = r \), then \( r \leq y \); this is pre-order traversal, hence we cannot have \( r \geq y \).
    \item We handle post-order traversal similarly.
  \end{itemize}

  \SubProofOf[def:binary_relation/transitive]{transitivity} Suppose that \( x \leq_W y \) and \( y \leq_W z \) with all three elements distinct.
  \begin{itemize}
    \item If all three belong to \( V_i \), antisymmetry follows from the inductive hypothesis.
    \item If \( x = r \), then \( r \leq y \); this is pre-order traversal, hence by definition \( x \leq_W z \).
    \item If \( z = r \), then \( y \leq z \); this is post-order traversal, hence by definition \( x \leq_W z \).
    \item We cannot have \( y = r \).
    \item If \( x, y \in V_i \) and \( z \in V_j \) with \( i < j \), then \( x \leq_W z \) by \ref{def:traversal_ordering/step/different}.
    \item If \( x \in V_i \) and \( y, z \in V_j \) with \( i < j \), then also \( x \leq_W z \) by \ref{def:traversal_ordering/step/different}.
  \end{itemize}

  \SubProofOf[rem:well_founded_relation]{well-foundedness} Follows from \fullref{thm:def:well_ordered_set/finite_chain}.
\end{defproof}
