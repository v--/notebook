\section{Graph matrices}\label{sec:graph_matrices}

\begin{remark}\label{rem:graphs_linear_algebra_and_topology}
  As we shall see, the \hyperref[def:graph_adjacency]{adjacency} and \hyperref[def:graph_incidence]{incidence} of a directed multigraph can be studied using linear algebra via \hyperref[def:graph_adjacency_matrix]{adjacency} and \hyperref[def:directed_incidence_matrix]{incidence matrices}, while \hyperref[def:graph_connectedness]{connectedness} can be studied using \hyperref[def:graph_connectedness]{topology} via \hyperref[def:graph_geometric_realization/embedding]{graph embeddings}.
\end{remark}

\begin{remark}\label{rem:graph_network_terminology}
  The term \enquote{network} is used for certain kinds of graphs describing electrical networks, transportation networks, social networks, etc.

  \begin{itemize}
    \item \incite[655]{Knuth1997ArtVol1} defines a \enquote{network} as a simple graph, directed or undirected, where either the vertices or arcs/edges are weighted.

    \item \incite[13]{Мирчев2001Графи} defines a \enquote{network} (\enquote{мрежа}) similarly, but with weights assigned only to vertices.

    \item \incite[62]{Harary1969Graphs} defines a \enquote{network} similarly, but with weights assigned only to arcs/edges, and with weights restricted to positive real numbers.

    \item \incite[54]{Bollobás1998Graphs} defines an \enquote{electric network} as a simple undirected graph with weighted edges.

    \item \incite[\S 7.3.3]{Новиков2013ДискретнаяМатематика} defines a \enquote{network} (\enquote{сеть}) as a simple directed graph with two distinguished vertices --- an initial and terminal vertex.

    \item \incite[151]{Diestel2017Graphs} defines a \enquote{network} as a directed multigraph with weighted arcs and two distinguished vertices, with the intention of studying \enquote{flows} between these vertices.

    \item \incite[def. 6.8.1]{Knauer2019AlgGraphTheory} defines a \enquote{transportation network} as a simple directed weakly connected graph with two distinguished vertices and weighted arcs.

    \item \incite[655]{Knuth1997ArtVol1} defines a \enquote{network} as a graph of arbitrary kind where either the vertices or arcs/edges are weighted.

    \item \incite[def. 5.17]{БелоусовТкачёв2004ДискретнаяМатематика} define a \enquote{network} (\enquote{сеть}) as directed acyclic graphs and study distances from vertices with in-degree \( 0 \).
  \end{itemize}

  Due to the inherent ambiguity, we will avoid this terminology and instead prefer being more concrete.
\end{remark}

\paragraph{Adjacency matrices}

\begin{definition}\label{def:graph_adjacency_matrix}\mcite[343; 500]{Stanley2012EnumCombinatoricsVol1}
  Let \( G = (V, A) \) be a finite \hyperref[def:directed_graph]{simple directed graph}, possibly with loops, and let \( v_1, \ldots, v_n \) be an enumeration of its vertices. We define the \term{unweighted} \term[bg=матрица на съседство (\cite[52]{Мирчев2001Графи}), ru=матрица смежности (\cite[27]{ЕмеличевИПр1990Графы}), en=adjacency matrix (\incite[27]{Diestel2017Graphs})]{adjacency matrix} (resp. \term{weighted} with respect to the \hyperref[def:weighted_set]{weight function} \( w: A \to R \)) as the matrix
  \begin{equation*}
    M = \seq{ m_{i,j} }_{i,j=1}^n
  \end{equation*}
  with elements
  \begin{paracol}{2}
    \begin{leftcolumn}
      \begin{equation*}
        m_{i,j} \coloneqq \begin{cases}
          w(v_i, v_j), &(v_i, v_j) \in A \\
          0_R,         &\T{otherwise.}
        \end{cases}
      \end{equation*}
      \begin{center}
        (weighted)
      \end{center}
    \end{leftcolumn}

    \begin{rightcolumn}
      \begin{equation*}
        m_{i,j} \coloneqq \begin{cases}
          1, &(v_i, v_j) \in A \\
          0, &\T{otherwise.}
        \end{cases}
      \end{equation*}
      \begin{center}
        (unweighted)
      \end{center}
    \end{rightcolumn}
  \end{paracol}
\end{definition}
\begin{comments}
  \item The adjacency matrix is thus also defined for \hyperref[def:undirected_graph]{simple undirected graphs} due to the doubling functor \hyperref[def:graph_functors/simple_doubling]{\( D_S \)}. The matrix is \hyperref[def:transpose_matrix]{symmetric} in this case.

  \item Only unweighted adjacency matrices are usually considered, for example by \incite[28]{Diestel2017Graphs}, \incite[151]{Harary1969Graphs}, \incite[54]{Bollobás1998Graphs}, \incite[195]{Erickson2019Algorithms}, \incite[32]{Зыков2004Графы}, \incite[27]{ЕмеличевИПр1990Графы}, \incite[\S 7.4.2]{Новиков2013ДискретнаяМатематика} and \incite[52]{Мирчев2001Графи}.

  \item Rather than considering explicit weights, \incite[def. 2.1.1]{Knauer2019AlgGraphTheory} define adjacency matrices for directed multigraphs as matrices of the underlying directed graph \hyperref[def:graph_functors/directed_forgetful]{\( U_D(G) \)}, weighted by the number of parallel arcs. We provide an example in \fullref{ex:def:graph_adjacency_matrix/multi}.
\end{comments}

\begin{proposition}\label{thm:adjacency_matrix_degree}\mcite[prop. 2.1.4]{Knauer2019AlgGraphTheory}
  Let \( M = \seq{ m_{i,j} }_{i,j=1}^n \) be the unweighted \hyperref[def:graph_adjacency_matrix]{adjacency matrix} of \( G \). Then
  \begin{align*}
    \deg_{\op{in}}(v_j)  &= \sum_{k=1}^n m_{k,j}, \\
    \deg_{\op{out}}(v_i) &= \sum_{k=1}^n m_{i,k}.
  \end{align*}
\end{proposition}
\begin{proof}
  Trivial.
\end{proof}

\begin{proposition}\label{thm:adjacency_matrix_power}
  Let \( M \) be the unweighted \hyperref[def:graph_adjacency_matrix]{adjacency matrix} of \( G \). Then the \( (i, j) \)-th entry of \( M^l \), where \( l \geq 1 \), is the number of \hyperref[def:graph_walk]{walks} of length \( l \) from \( i \) to \( j \).
\end{proposition}
\begin{proof}
  Let \( n \) be the order of \( G \). We will use induction on \( l \).
  \begin{itemize}
    \item In the base case \( l = 1 \) the matrix \( M \) has \( 1 \) as its \( (i,j) \)-th entry whenever there is an arc (i.e. a walk of length \( 1 \)) from \( v_i \) to \( v_j \).

    \item Suppose that the statement holds for \( l \). The \( (i, j) \)-th entry of \( M^{l+1} \) is
    \begin{equation*}
      m^{(l+1)}_{i,j} = \sum_{k=1}^n m^{(l)}_{i,k} \cdot m_{k,j}.
    \end{equation*}

    If there are \( m \) walks of length \( l \) from \( v_i \) to \( v_k \) and if there is an arc from \( v_k \) to \( v_j \), these walks can be extended to walks of length \( l + 1 \) from \( v_i \) to \( v_j \). Summing by all \( k \), we obtain \( m^{(l+1)}_{i,j} \).
  \end{itemize}
\end{proof}

\begin{proposition}\label{thm:adjacency_matrix_tropical_power}
  Suppose that the arcs of \( G = (V, A) \) are weighted by \( w: A \to \BbbR \). Let \( M \) be the weighted adjacency matrix of \( G \).

  Denote by \( R \) the \hyperref[def:tropical_semiring]{\( \min \)-plus semiring} over \( \BbbR \) and consider the power \( M^l \) over \( R \). Then the \( (i, j) \)-th entry of \( M^l \) is the length of the walk from \( v_i \) to \( v_j \) of length \( l \) with least weight.
\end{proposition}
\begin{proof}
  Analogous to \fullref{thm:adjacency_matrix_power}.
\end{proof}

\begin{proposition}\label{thm:nilpotent_adjacency_matrix}
  Let \( M \) be the unweighted \hyperref[def:graph_adjacency_matrix]{adjacency matrix} of \( G \). Then \( G \) is \hyperref[def:acyclic_graph]{acyclic} if and only if \( M \) is \hyperref[def:nilradical]{nilpotent}.
\end{proposition}
\begin{proof}
  \SufficiencySubProof* Suppose that \( G \) is acyclic. Denote by \( n \) the order of \( G \). By \fullref{thm:acyclic_graph_walk_length}, it has no paths of length \( n \). \Fullref{thm:adjacency_matrix_power} then implies that \( M^n = 0 \).

  \NecessitySubProof* Suppose that \( M^k = 0 \) for some positive integer \( k \). By \fullref{thm:acyclic_graph_walk_length}, \( G \) has no paths of length \( k \). \Fullref{thm:acyclic_graph_walk_length} then implies that \( G \) is acyclic.
\end{proof}

\begin{example}\label{ex:def:graph_adjacency_matrix}
  We list some examples of \hyperref[def:graph_adjacency_matrix]{adjacency matrices}:
  \begin{thmenum}
    \thmitem{ex:def:graph_adjacency_matrix/triangle} The \hyperref[def:graph_adjacency_matrix]{adjacency matrix} of the graph
    \begin{equation}\label{eq:ex:def:graph_adjacency_matrix/triangle/graph}
      \begin{aligned}
        \includegraphics[page=1]{output/ex__def__graph_adjacency_matrix__triangle}
      \end{aligned}
    \end{equation}
    is
    \begin{equation}\label{eq:ex:def:graph_adjacency_matrix/triangle/graph/matrix}
      M =
      \begin{blockarray}{*{3}{c} c}
        a & b & c &   \\
      \begin{block}{(*{3}{c}) c}
        1 & 1 & 1 & a \\
          &   & 1 & b \\
          &   &   & c \\
      \end{block}
      \end{blockarray}
    \end{equation}

    \Fullref{thm:adjacency_matrix_degree} implies that
    \begin{itemize}
      \item \( \deg_{\op{in}}(a) = \deg_{\op{in}}(b) = 1 \) and \( \deg_{\op{in}}(c) = 2 \).
      \item \( \deg_{\op{out}}(a) = 3 \), \( \deg_{\op{in}}(b) = 1 \) and \( \deg_{\op{in}}(c) = 0 \).
    \end{itemize}

    To demonstrate \fullref{thm:adjacency_matrix_power}, consider
    \begin{equation*}
      M^2 =
      \begin{blockarray}{*{3}{c} c}
        a & b & c &   \\
      \begin{block}{(*{3}{c}) c}
        1 & 1 & 2 & a \\
          &   &   & b \\
          &   &   & c \\
      \end{block}
      \end{blockarray}
    \end{equation*}
    which describes the following walks of length \( 2 \):
    \begin{itemize}
      \item \( a \to a \to a \),
      \item \( a \to a \to b \),
      \item \( a \to a \to c \),
      \item \( a \to b \to c \).
    \end{itemize}

    Furthermore, we have \( M^3 = M^2 \), which corresponds to the following walks:
    \begin{itemize}
      \item \( a \to a \to a \) extends to
      \begin{itemize}
        \item \( a \to a \to a \to a \).
        \item \( a \to a \to a \to b \).
        \item \( a \to a \to a \to c \).
      \end{itemize}
      \item \( a \to a \to b \) extends to \( a \to a \to b \to c \).
    \end{itemize}

    For \( l > 3 \), the walks are analogous.

    \thmitem{ex:def:graph_adjacency_matrix/square} The \hyperref[def:graph_adjacency_matrix]{weighted adjacency matrix} of the graph
    \begin{equation}\label{eq:ex:def:graph_adjacency_matrix/square/graph}
      \begin{aligned}
        \includegraphics[page=1]{output/ex__def__graph_adjacency_matrix__square}
      \end{aligned}
    \end{equation}
    is
    \begin{equation}\label{eq:ex:def:graph_adjacency_matrix/square/graph/matrix}
      M =
      \begin{blockarray}{*{4}{c} c}
        a & b & c & d &  \\
      \begin{block}{(*{4}{c}) c}
        a & b & c & d &   \\
          & 3 & 2 & 4 & a \\
          &   &   & 1 & b \\
          &   &   & 1 & c \\
          &   &   &   & c \\
      \end{block}
      \end{blockarray}
    \end{equation}

    In the \hyperref[def:tropical_semiring]{\( \min \)-plus} semiring, the matrix above has \( \infty \) rather than \( 0 \). The \( (i, j) \)-th entry of \( M^2 \) in the \( \min \)-plus semiring is
    \begin{equation*}
      \min_{k=1}^n \parens[\Big]{ w(i,k) + w(k,j) }
    \end{equation*}
    and the matrix itself has values
    \begin{equation*}
      M^2 =
      \begin{blockarray}{*{4}{c} c}
        a & b & c & d &  \\
      \begin{block}{(*{4}{c}) c}
        a & b & c & d &   \\
          &   &   & 3 & a \\
          &   &   &   & b \\
          &   &   &   & c \\
          &   &   &   & c \\
      \end{block}
      \end{blockarray}
    \end{equation*}

    Indeed, there are only two walks from \( a \) to \( d \) of length \( 2 \) and the shorter one has weight \( 3 \).

    \thmitem{ex:def:graph_adjacency_matrix/multi} Following \incite[def. 2.1.1]{Knauer2019AlgGraphTheory}, we can define the \hyperref[def:graph_adjacency_matrix]{adjacency matrix} of a \hyperref[def:directed_multigraph]{directed multigraph} \( G \) as the adjacency matrix of \hyperref[def:graph_functors/directed_forgetful]{\( U_D \)}\( (G) \) where the weights of an arc \( (u, v) \) in \( U_D(G) \) is the number of parallel arcs from \( u \) to \( v \) in \( G \).

    For example, the adjacency matrix of \eqref{eq:fig:def:directed_multigraph} would become
    \begin{equation}\label{eq:ex:def:graph_adjacency_matrix/multi}
      \begin{blockarray}{*{6}{c} c}
        a & b & c & d & e & f &   \\
      \begin{block}{(*{6}{c}) c}
          & 1 & 1 &   &   &   & a \\
          &   &   & 1 &   &   & b \\
          &   &   & 1 & 1 &   & c \\
          &   &   &   &   & 1 & d \\
          &   &   &   &   & 1 & e \\
          &   &   &   &   &   & f \\
      \end{block}
      \end{blockarray}
    \end{equation}
  \end{thmenum}
\end{example}

\begin{proposition}\label{thm:adjacency_matrix_components}\mcite[thm. 2.1.8; thm. 2.1.9]{Knauer2019AlgGraphTheory}
  Let \( G = (V, A) \) be a finite \hyperref[def:directed_graph]{simple directed graph}, possibly with loops. Let \( G_1, \ldots, G_s \) be an enumeration of its \hyperref[def:graph_connectedness/strong]{strongly connected components} such that there are no arcs from \( G_{k+1} \) to \( G_k \).

  Let \( c_0 \coloneqq 1 \) and, for each \( k = 1, \ldots, s \), let
  \begin{equation*}
    v_{c_{k-1}}, \ldots, v_{c_k}
  \end{equation*}
  be an enumeration of the vertices of \( G_k \). Then
  \begin{equation*}
    1 = c_0 < c_1 < \ldots < c_s = n
  \end{equation*}
  and \( v_1, \ldots, v_n \) is an enumeration of \( V \).

  Finally, let \( M \) be the (weighted or unweighted) \hyperref[def:graph_adjacency_matrix]{adjacency matrix} of \( G \) and let \( M_k \) be the adjacency matrix of \( G_k \). Then \( M \) is \hyperref[def:block_matrix]{block}-\hyperref[def:triangular_matrix]{upper triangular} with
  \begin{equation}\label{eq:thm:adjacency_matrix_components}
    M = \begin{pmatrix}
      M_1    & R_{1,2} & \cdots & R_{1,s} & R_{1,s}   \\
      0      & M_2     & \cdots & R_{2,s} & R_{2,s}   \\
      \vdots & \vdots  & \ddots & \vdots  & \vdots    \\
      0      & 0       & \cdots & M_{s-1} & R_{s-1,s} \\
      0      & 0       & \cdots & 0       & M_s       \\
    \end{pmatrix}
  \end{equation}

  If we consider the \hyperref[def:graph_connectedness/weak]{weak connected components}, each of the matrices \( R_{i,j} \) becomes zero.
\end{proposition}
\begin{proof}
  We will use induction on the number \( s \) of strongly connected components.
  \begin{itemize}
    \item If \( s = 1 \), then \( M = M_1 \), which vacuously satisfies \eqref{eq:thm:adjacency_matrix_components}.
    \item Suppose that the proposition holds for all matrices with \( s \) strongly connected components and let \( G \) be a graph with \( s + 1 \) strongly connected components.

    Let \( G' \) be the graph with vertices \( v_1, \ldots, v_{c_s} \) and let \( M' \) be the matrix \( G' \). We must show that
    \begin{equation*}
      M = \begin{pmatrix}
        M' & R_{1,s+1} \\
        0  & M_{s+1}
      \end{pmatrix},
    \end{equation*}
    that is, that \( M_{i,j} = 0 \) when \( i > c_s \) and \( j < c_s \).

    In this case \( v_i \) is in \( G_{s+1} \), while \( v_j \) is in \( G_{k_0} \) for some \( k_0 \leq s \). We have assumed that there are no arcs from \( G_{s+1} \) to \( G_s \), hence also no arcs from \( G_{s+1} \) to \( G_{k_0} \). Thus, \( M_{i,j} = 0 \).
  \end{itemize}
\end{proof}

\paragraph{Incidence matrices}

\begin{definition}\label{def:directed_incidence_matrix}\mcite[def. 2.2.1]{Knauer2019AlgGraphTheory}
  Let \( G = (V, A, h, t) \) be a finite \hyperref[def:directed_multigraph]{directed multigraph}. Let \( v_1, \ldots, v_m \) and \( a_1, \ldots, a_n \) be enumerations of \( V \) and of \( A \), correspondingly.

  We define the \term[bg=матрица на инцидентност (\cite[47]{Мирчев2001Графи}), ru=матрица инцидентности (\cite[289]{БелоусовТкачёв2004ДискретнаяМатематика})]{incidence matrix}
  \begin{equation*}
    M = \seq{ m_{i,j} }_{i,j=1}^{m,n}
  \end{equation*}
  of \( G \) with elements
  \begin{equation*}
    m_{i,j} \coloneqq \begin{cases}
      1,  &v_i = h(e_j), \\
      -1, &\T{otherwise if} v_i = t(e_j), \\
      0,  &\T{otherwise.}
    \end{cases}
  \end{equation*}

  In case \( v_i \) is a loop, \( m_{i,j} = 1 \).
\end{definition}
\begin{comments}
  \item Unlike for \hyperref[def:graph_adjacency_matrix]{adjacency matrices}, it is not clear how to handle weights in incidence matrices, so we avoid doing it altogether.
  \item For undirected (multi)graphs we present a different definition; see \fullref{def:undirected_incidence_matrix}.
\end{comments}

\begin{definition}\label{def:undirected_incidence_matrix}\mcite[def. 2.2.1]{Knauer2019AlgGraphTheory}
  Let \( G = (V, E, \mscrE) \) be a finite \hyperref[def:hypergraph]{hypergraph}. Let \( v_1, \ldots, v_m \) and \( e_1, \ldots, e_n \) be enumerations of \( V \) and of \( E \), correspondingly.

  We define the \term[bg=матрица на инцидентност (\cite[47]{Мирчев2001Графи}), ru=матрица инцидентности (\cite[288]{БелоусовТкачёв2004ДискретнаяМатематика})]{incidence matrix}
  \begin{equation*}
    M = \seq{ m_{i,j} }_{i,j=1}^{m,n}
  \end{equation*}
  of \( G \) with elements
  \begin{equation*}
    m_{i,j} \coloneqq \begin{cases}
      1,  &v_i \in \mscrE(e_j), \\
      0,  &\T{otherwise.}
    \end{cases}
  \end{equation*}
\end{definition}
\begin{comments}
  \item For directed (multi)graphs we present a different definition; see \fullref{def:undirected_incidence_matrix}.
\end{comments}

\begin{example}\label{ex:incidence_matrices}
  The \hyperref[def:undirected_incidence_matrix]{hypergraph incidence matrix} of \eqref{eq:fig:def:hypergraph} is
  \begin{equation}\label{eq:ex:incidence_matrices}
    \begin{blockarray}{*{9}{c} c}
      e_1      & e_2      & e_3      & e_4      & e_5      & e_6      & e_7      & e_8 & e_9 &   \\
    \begin{block}{(*{9}{c}) c}
      1        & 1        &          &          &          &          &          &     &     & a \\
      \fbox{1} &          & 1        &          &          &          &          &     &     & b \\
               & \fbox{1} &          & 1        & 1        &          &          &     & 1   & c \\
               &          & \fbox{1} & \fbox{1} &          & 1        &          &     & 1   & d \\
               &          &          &          & \fbox{1} &          & 1        &     & 1   & e \\
               &          &          &          &          & \fbox{1} & \fbox{1} & 1   &     & f \\
    \end{block}
    \end{blockarray}
  \end{equation}

  To obtain the \hyperref[def:directed_incidence_matrix]{directed incidence matrix} for the directed multigraph \eqref{eq:fig:def:directed_multigraph}, we need to remove the hyperedge column \( e_9 \) and flip the sign of all boxed entries.
\end{example}
