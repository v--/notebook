\section{Combinatorics}\label{sec:combinatorics}

Combinatorics originated as the study of problems related to counting. This study is now referred to as \enquote{enumerative combinatorics} to distinguish it from more abstract subfields. A canonical example of a counting problem is given in \fullref{ex:fibonacci_rabbits}. On the other hand, \fullref{thm:gamma_function_interpolates_factorial} provides an example of how mathematical analysis can answer questions related to counting. The latter subfield is called \enquote{analytic combinatorics}.

Results in combinatorics are traditionally obtained for \hyperref[def:set_finiteness]{finite sets} and \hyperref[def:integer_signum]{positive integers}, however many of them can be easily generalized. For example, \fullref{thm:pigeonhole_principle} is stated in several abstract forms.

\begin{example}\label{ex:fibonacci_rabbits}
  Leonardo Pisano, known as Fibonacci, published a book on arithmetic named \enquote{Liber Abaci} in 1202. In \cite[54]{АлександровМаркушевичХинчин1951ЭнциклопедияТом1}, Bashmakova and Yushkevich claim that books is among the first books featuring decimal numerals. Despite being influential, the book lacks the clarity of exposition that we can afford now after centuries of polishing existing concepts.

  The solution to one of the problems stated there later generalized to what is now known as the Fibonacci sequence. In an English translation of the book, \cite{Sigler2002LiberAbaci}, the problem is formulated as follows:
  \begin{displayquote}
    A certain man had one pair of rabbits together in a certain enclosed place, and one wishes to know how many are created from the pair in one year when it is the nature of them in a single month to bear another pair, and in the second month those born to bear also.
  \end{displayquote}

  More generally, we are interested in the number \( F_n \) of pairs of rabbits at month \( n \). If we set aside all physiological questions that may introduce ambiguity, there appears to be a canonical solution. We can conclude the following:
  \begin{itemize}
    \item At the zeroth month, there is only the initial pair, thus \( F_0 = 1 \).

    \item At the first month, we also have the firstborn pair, thus \( F_1 = F_0 + 1 = 2 \).

    \item At month \( n \geq 2 \), we have \( F_{n-1} \) pairs, \( F_{n-2} \) out of which are mature and produce offspring, thus \( F_n = F_{n-1} + F_{n-2} \).
  \end{itemize}

  The entire sequence can thus be built using the following recursive definition:
  \begin{equation*}
    F_k \coloneqq \begin{cases}
      1,                &k = 1, \\
      2,                &k = 2, \\
      F_{k-1} + F_{k-2} &k > 2.
    \end{cases}
  \end{equation*}

  For the first \( 12 \) months, the number of rabbits is given in the following table:
  \begin{equation*}
    \begin{array}{r *{13}{c}}
      \textbf{Month} & 0 & 1 & 2 & 3 & 4 & 5  & 6  & 7  & 8  & 9  & 10  & 11  & 12  \\
      \textbf{Pairs} & 1 & 2 & 3 & 5 & 8 & 13 & 21 & 34 & 55 & 89 & 144 & 233 & 377 \\
    \end{array}
  \end{equation*}

  It should also be noted that the Fibonacci sequence named after this problem is defined to start with \( F_0 = 0 \) and \( F_1 = 1 \) --- see \fullref{def:fibonacci_sequence}.
\end{example}
