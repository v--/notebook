\section{Lebesgue measures}\label{sec:lebesgue_measures}

\begin{definition}\label{def:subset_characteristic_function}\mimprovised
  Fix some set \( X \), which we regard as some ambient space that is clear from the context. We define the \term[bg=характеристична функция (\cite[exerc. 4.1]{Драганов2022ТеорияНаМярката}), ru=характеристическая функция (\cite[131]{Богачёв2003ОсновыТеорииМерыТом1}), en=characteristic function (\cite[15]{Halmos1976MeasureTheory})]{characteristic function} of the subset \( A \) of \( X \) with respect to the \hyperref[def:semiring]{semiring} \( R \) as
  \begin{equation}\label{eq:def:dirichlet_function}
    \begin{aligned}
      &\varphi_A: X \to R \\
      &\varphi_A(x) \coloneqq \begin{cases}
        1_R, &x \in A \\
        0_R, &x \not\in A.
      \end{cases}
    \end{aligned}
  \end{equation}

  The definition is mostly useful when \( R \) is the \hyperref[def:integers]{ring of integers}, but it naturally extends to \hyperref[def:truth_value_algebra]{truth value algebras}.
\end{definition}
\begin{comments}
  \item We simply generalize the standard definition from e.g. \cite[15]{Halmos1976MeasureTheory} to semirings.

  \item \hyperref[def:fuzzy_set]{Fuzzy sets} generalize characteristic functions by allowing values between \( 0 \) and \( 1 \).
\end{comments}

\begin{definition}\label{def:dirichlet_function}\mcite[88]{Müller2022HandbookOfDynamicsAndProbability}
  We call the \hyperref[def:subset_characteristic_function]{characteristic function} of \( \BbbQ \) in \( \BbbR \) \term{Dirichlet's function}.
\end{definition}

\begin{definition}\label{def:lebesgue_measure}
  \todo{Define Lebesgue measures}.
\end{definition}
