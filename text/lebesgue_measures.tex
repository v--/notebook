\section{Lebesgue measures}\label{sec:lebesgue_measures}

\begin{definition}\label{def:subset_characteristic_function}\mcite[15]{Halmos1976MeasureTheory}
  Fix some set \( X \), which we regard as some ambient space that is clear from the context. We define the \term[bg=характеристична функция (\cite[exerc. 4.1]{Драганов2022ТеорияНаМярката}), ru=характеристическая функция (\cite[131]{Богачёв2003ОсновыТеорииМерыТом1})]{characteristic function} of the subset \( A \) of \( X \) as
  \begin{equation}\label{eq:def:dirichlet_function}
    \begin{aligned}
      &I_B: X \to \set{ 0, 1 } \\
      &I_B(x) \coloneqq \begin{cases}
        1, &x \in A \\
        0, &x \not\in A.
      \end{cases}
    \end{aligned}
  \end{equation}
\end{definition}
\begin{comments}
  \item In logic, it makes sense to regard \( 1 \) and \( 0 \) as general \hyperref[con:boolean_value]{Boolean values}. Generally, however, characteristic functions are used in measure theory, where their numerical values are important.

  \item \hyperref[def:fuzzy_set]{Fuzzy sets} generalize characteristic functions by allowing values between \( 0 \) and \( 1 \).
\end{comments}

\begin{definition}\label{def:dirichlet_function}\mcite[88]{Müller2022HandbookOfDynamicsAndProbability}
  We call the \hyperref[def:subset_characteristic_function]{characteristic function} of \( \BbbQ \) in \( \BbbR \) \term{Dirichlet's function}.
\end{definition}

\begin{definition}\label{def:lebesgue_measure}
  \todo{Define Lebesgue measures}.
\end{definition}
