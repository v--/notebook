\section{Mathematical logic}\label{sec:mathematical_logic}

Mathematical logic uses mathematics to study logic and vice versa.

We start with purely syntactic objects --- strings of symbols called formulas. Definitions for formulas are given here using \hyperref[def:formal_grammar]{formal grammars}, which in turn depend on \hyperref[def:formal_language]{formal languages}. These formulas are then given \hyperref[def:first_order_valuation/formula_valuation]{valuations} via \hyperref[def:boolean_algebra]{Boolean algebras}.

These concepts help us define the theory necessary to study following two important intertwined topics:
\begin{itemize}
  \item We are interested in establishing whether the formula \( \varphi \) logically entails the formula \( \psi \). This is done using \hyperref[def:deductive_system]{deductive systems} which specify precisely how we can manipulate strings of symbols. This aspect is called \enquote{syntactic} or \enquote{logical} and is the basis or \hyperref[def:proof_derivability]{proof theory}. Formulas allow us to express statements about mathematics and proof theory allows us to systematically study the relationships between them. In practice, we manipulate \hyperref[rem:abstract_syntax_tree]{syntax trees} rather than strings.

  \item Given a formula, we are interested in assigning a meaning to it. Different logical systems provide different syntax that is useful for different purposes - \hyperref[def:propositional_grammar_schema/formula]{propositional formulas} allow us to express complex relationships between propositions via \hyperref[subsec:boolean_functions]{Boolean functions} while \hyperref[def:first_order_syntax/formula]{first-order logic formulas} allow us to go one level lower and give a precise meaning to these propositions via \hyperref[def:first_order_structure]{structures}. This aspect of logic is called \enquote{semantical} and is the basis of \hyperref[subsec:first_order_models]{model theory}. Model theory allows us to study logical formulas using pre-existing mathematics.
\end{itemize}

There are two aspects in which logical systems are categorized:
\begin{itemize}
  \item \hyperref[subsec:propositional_logic]{Propositional} and \hyperref[subsec:first_order_logic]{first-order logic} (among others) differ in what their syntax allows us to express. This also means that they differ in what their semantics can express, but, just as the syntax of first-order logic is a superset of the syntax of propositional logic, \hyperref[subsec:boolean_functions]{Boolean functions} can express relations between quantifierless atomic formulas in any \hyperref[def:first_order_structure]{structure}. In other words, semantical properties are identical in places where the syntax is the same.

  \item \hyperref[rem:classical_logic]{Classical} and \hyperref[rem:intuitionistic_logic]{intuitionistic logic} (among others) differ in their semantics and their logical \hyperref[def:judgment/inference_rule]{inference rules}. This has several implications:
  \begin{itemize}
     \item Boolean functions describe \hyperref[rem:classical_logic]{classical logic}, however they fail to describe \hyperref[rem:intuitionistic_logic]{intuitionistic logic} because double negation elimination \eqref{eq:thm:minimal_propositional_negation_laws/dne} no longer holds and neither do other similar statements. So, while retaining the same syntax, we must resort to much more complicated semantical frameworks like \hyperref[def:propositional_heyting_algebra_semantics]{Heyting} or \hyperref[def:propositional_topological_semantics]{topological semantics}.

     \item The proof theory that describes classical logic no longer matches the semantics, hence we must resort to other proof systems. This turns out not to be trivial because we need a clear understanding of which logical axioms imply the others. \Fullref{subsec:deductive_systems} lists different proof systems and their corresponding semantics.
  \end{itemize}
\end{itemize}

\begin{remark}\label{rem:metalogic}
  The statements of mathematical logic can themselves be studied logically. We distinguish between the \enquote{object logic} which we study and the \enquote{metalogic} which we use to study it. It is possible, for example, to study intuitionistic propositional logic using classical first-order logic. The metalogic is usually less formal and its statements are written in prose for the sake of easier understanding.

  It is an exercise in futility to try and completely formalize the language, syntax and theory of the metalogic --- or, as they are sometimes called, the \enquote{metalanguage}, \enquote{metasyntax} and \enquote{metatheory}. We must take a given metalogical framework for granted and then study a certain object logical framework. This is not to say that the principles and rules that hold in the metalogic are immaterial --- see for example the discussion of the differences between \hyperref[rem:intuitionistic_logic]{intuitionistic logic} and \hyperref[rem:classical_logic]{classical logic}. Still, it makes little sense to attempt to study the metalogic because at that point it becomes the object logic and the still more abstract conceptual framework in which we reason about the metalogic now becomes the new metalogic. We can thus form a hierarchy that is unbounded in both directions --- we can study a more concrete object logic within our object logic, and we can jump from one metalogical level to the next.

  An important connection between the logic and metalogic is given in \fullref{rem:set_definition_recursion}.
\end{remark}

\begin{remark}\label{rem:classical_logic}
  Classical logic is a term used to describe, among others:
  \begin{itemize}
    \item A semantical framework for propositional logic defined in \fullref{def:propositional_entailment} and \fullref{def:propositional_model}.
    \item A matching \hyperref[def:deductive_system]{deductive system}, defined in \fullref{def:classical_propositional_deductive_systems}.
    \item A semantical framework for \hyperref[subsec:first_order_logic]{first-order logic} defined in \fullref{def:first_order_semantics}.
    \item A matching deductive system, defined in \fullref{def:first_order_natural_deduction_system}.
  \end{itemize}

  Classical logic is characterized by the ability to use the law of double negation elimination \eqref{eq:thm:minimal_propositional_negation_laws/dne}. A more popular (but less accurate due to \fullref{thm:minimal_propositional_negation_laws}) characterization is that the law of the excluded middle \eqref{eq:thm:minimal_propositional_negation_laws/lem} holds.

  Within the metalogic, the law of the excluded middle is called the \enquote{principle of bivalence} and states that either a statement holds or its negation holds.
\end{remark}

\begin{remark}\label{rem:intuitionistic_logic}
  Intuitionistic logic is a generalization of \hyperref[rem:classical_logic]{classical logic}. It is also called \enquote{constructive logic} due to the \hyperref[rem:brouwer_heyting_kolmogorov_interpretation]{Brouwer-Heyting-Kolmogorov interpretation}. See \fullref{rem:brouwer_heyting_kolmogorov_interpretation_compatibility} for further discussion of the topic.

  Instead of the law of the excluded middle \eqref{eq:thm:minimal_propositional_negation_laws/lem}, we have the strictly weaker principle of explosion \eqref{eq:thm:minimal_propositional_negation_laws/efq} stating that everything can be proved given a contradiction.

  To these ideas there correspond \hyperref[def:propositional_heyting_algebra_semantics]{Heyting algebra semantics} and \hyperref[def:propositional_topological_semantics]{topological semantics} and a matching deductive system, \fullref{def:intuitionistic_propositional_deductive_systems}, for \hyperref[subsec:propositional_logic]{propositional logic}.
\end{remark}

\begin{remark}\label{rem:minimal_logic}
  Minimal logic is a further generalization of \hyperref[rem:intuitionistic_logic]{intuitionistic logic}.

  Instead of the law of the excluded middle \eqref{eq:thm:minimal_propositional_negation_laws/lem} or the strictly weaker principle of explosion \eqref{eq:thm:minimal_propositional_negation_laws/efq}, we have the even weaker law of non-contradiction \eqref{eq:thm:minimal_propositional_negation_laws/lnc}.

  Metalogically speaking, we can only conclude that there is no statement such that both the statement and its negation are true. If the statement instead does not hold, we cannot automatically conclude that its negation holds.

  \Fullref{def:minimal_propositional_hilbert_system} provides a deductive system for \hyperref[subsec:propositional_logic]{propositional logic}. We avoid studying semantics of minimal logic.
\end{remark}

\begin{remark}\label{rem:mathematical_logic_conventions}
  Several conventions related to logic are followed through the document.

  We only work within \hyperref[def:classical_propositional_deductive_systems]{classical metalogic}. Outside the section on logic, we use formulas and, more generally, use object logic only in dedicated places like \fullref{def:group/theory} describing the \hyperref[def:first_order_theory]{logical theory} of \hyperref[def:lattice]{lattices}. Other theories like that of \hyperref[def:complete_lattice]{complete lattices} cannot be elegantly presented via first-order formulas and are instead formulated entirely within the metalanguage under the assumption that we are working within a model of set theory --- \hyperref[def:axiom_of_universes]{\logic{ZFC+U}}, to be more precise. Using logical theories like \fullref{def:lattice/theory} allows us to implicitly reuse a plethora of definitions and theorems regarding first-order models from \fullref{subsec:first_order_models}. To keep a clear distinction between logical formulas and non-logical axioms and, more generally, to distinguishing between logic and metalogic, we use the following conventions:

  \begin{thmenum}
    \thmitem{rem:mathematical_logic_conventions/variable_symbols} Variables in the object language are denoted by the small Greek letters, usually \( \xi, \eta, \zeta \), while variables in the metalanguage are denoted by small Latin letters, usually \( x, y, z \). If needed, we add subscripts with indices.

    \thmitem{rem:mathematical_logic_conventions/greek_alphabet} Although the entire Greek alphabet is allocated for first-order variables in the object language, we only use a few letters, and allocate the rest towards other concepts in the metalogic:
    \begin{equation*}
      \aunderbrace{\alpha, \beta, \gamma, \delta, \varepsilon}_{\T{ordinals}},
      \aoverbrace[@{\downarrow\downarrow}]{\zeta, \eta}[U]^{\T{variables}},
      \aunderbrace[@{\mathrlap{\downarrow}}]{\theta}[D]_{\T{formulas}},
      \iota,
      \aunderbrace{\kappa, \lambda, \mu, \nu}_{\T{cardinals}},
      \aoverbrace[@{\mathrlap{\downarrow}}]{\xi}[U]^{\T{variables}},
      \aunderbrace{\omicron, \pi, \rho, \sigma, \tau}_{\T{terms}},
      \upsilon,
      \aunderbrace{\varphi, \chi, \psi, \omega}_{\T{formulas}}
    \end{equation*}

    Of course, whenever we need additional symbols, we may break the above convention.

    \thmitem{rem:mathematical_logic_conventions/connective_symbols} We prefer prose to symbolic quantifiers and connectives for statements in the metalanguage. For the object logic, we place dots over the base connectives from \fullref{def:propositional_alphabet}. This allows us to visually distinguish disjunctions from joins in \hyperref[def:lattice]{lattices}, for example in formulas such as
    \begin{equation*}
      (\xi \vee \eta) \synvee (\xi \vee \zeta).
    \end{equation*}

    We also place a dot over the equality, which allows us to distinguish equality of formulas from equality within formulas.

    A similar convention is used by Peter Hinman --- see \cite[rem. 2.1.3]{Hinman2005}.

    To avoid clutter, we do not put dots over predicate and functional symbols.

    \thmitem{rem:mathematical_logic_conventions/propositional_constants} We define the propositional constants for truth and falsity via \( \syntop \) and \( \synbot \) in the object language and by \( T \) and \( F \) in the metalanguage. We nevertheless use \( \top \) and \( \bot \) in general \hyperref[def:lattice]{lattices}.

    \thmitem{rem:mathematical_logic_conventions/structure_pairs} We often conflate structures in the metalogic (i.e. sets with functions and/or relations defined on them) with their domain --- see \fullref{rem:first_order_model_notation} for a discussion.

    \thmitem{rem:mathematical_logic_conventions/shorthands} We additionally use syntactic shorthands like \fullref{rem:propositional_formula_parentheses} and \fullref{rem:first_order_formula_conventions} when writing formulas.

    \thmitem{rem:mathematical_logic_conventions/quantification} We avoid writing excessive universal quantification and instead rely on implicit quantification as described in \fullref{thm:implicit_universal_quantification}. If we need the formulas to be closed, such as in the case of \hyperref[def:first_order_theory]{first-order theories} for example, we assume all formulas are closed and if they are not, we add explicit universal quantifiers in front of them.
  \end{thmenum}
\end{remark}
