\section{Mathematical logic}\label{sec:mathematical_logic}

Mathematical logic uses mathematics to study logic and vice versa.

We start with objects that are purely logical in nature --- formulas --- which are strings of symbols that represent truth values. Formal definitions for formulas are given here using \hyperref[def:formal_grammar]{grammars}, which in turn depend on \hyperref[def:formal_language]{languages}. Formal definitions for truth values are given using \hyperref[def:heyting_algebra]{Heyting} and \hyperref[def:boolean_algebra]{Boolean algebras}

These definitions help us define the theory necessary to study following two important intertwined topics:
\begin{itemize}
  \item We are interested in establishing whether the formula \( \varphi \) logically entails the formula \( \psi \). This is done using \hyperref[def:deductive_system]{deductive systems} which specify precisely how we can manipulate strings of symbols. This aspect is called \term{syntactic} or \term{logical} and is the basis or \hyperref[def:proof_derivability]{proof theory}. Formulas allow us to express statements about mathematics and proof theory allows us to systematically study the relationships between them.

  \item Given a formula, we are interested in assigning a meaning to it. Different logical systems provide different syntax that is useful for different purposes - \hyperref[def:propositional_syntax/formula]{propositional formulas} allow us to express complex relationships between propositions via \hyperref[subsec:boolean_functions]{Boolean functions} while \hyperref[def:first_order_syntax/formula]{first-order logic formulas} allow us to go one level lower and give a precise meaning to these propositions via \hyperref[def:first_order_structure]{structures}. This aspect of logic is called \term{semantical} and is the basis of \hyperref[subsec:first_order_models]{model theory}. Model theory allows us to study logical formulas using pre-existing mathematics.
\end{itemize}

There are two aspects in which logical systems are categorized:
\begin{itemize}
  \item \hyperref[subsec:propositional_logic]{Propositional} and \hyperref[subsec:first_order_logic]{first-order logic} (among others) differ in what their syntax allows us to express. This also means that they differ in what their semantics can express, but, just as the syntax of first-order logic is a superset of the syntax of propositional logic, \hyperref[subsec:boolean_functions]{Boolean functions} can express relations between quantifierless atomic formulas in any \hyperref[def:first_order_structure]{structures}. In other words, semantics are identical in places where the syntax is the same.

  \item \hyperref[def:classical_logic]{Classical} and \hyperref[def:intuitionistic_logic]{intuitionistic logic} (among others) differ in their semantics and their logical \hyperref[def:judgment/inference_rule]{inference rules}. This has two implications
  \begin{itemize}
     \item Boolean functions describe \hyperref[def:classical_logic]{classical logic}, however they fail to describe \hyperref[def:intuitionistic_logic]{intuitionistic logic} because double negation elimination \eqref{eq:thm:minimal_propositional_negation_laws/dne} no longer holds and neither do other similar statements. So, while retaining the same syntax, we must resort to much more complicated semantical frameworks like \hyperref[def:propositional_heyting_algebra_semantics]{Heyting} or \hyperref[def:propositional_topological_semantics]{topological semantics}.

     \item The proof theory that describes classical logic no longer matches the semantics, hence we must resort to other proof systems. This turns out not to be trivial because we need a clear understanding of which logical axioms imply the others. \Fullref{subsec:deductive_systems} lists different proof systems and their corresponding semantics.
  \end{itemize}
\end{itemize}

\begin{remark}\label{rem:metalogic}
  The statements of mathematical logic can themselves be studied logically. We distinguish between the \term{object logic} which we study and the \term{metalogic} which we use to study it. It is possible, for example, to study intuitionistic propositional logic using classical first-order logic. The metalogic is usually less formal and its statements are written in prose for the sake of easier understanding.

  It is an exercise in futility to try and completely formalize the language, syntax and theory of the metalogic --- the \term{metalanguage}, \term{metasyntax} and \term{metatheory}. We must take a given metalogical framework for granted and then study a certain object logical framework. This is not to say that the principles and rules that hold in the metalogic are immaterial --- see for example the discussion of the differences between \hyperref[def:intuitionistic_logic]{intuitionistic logic} and \hyperref[def:classical_logic]{classical logic}. This is to say that it makes little sense to attempt to study the metalogic because at that point it becomes the object logic and the still more abstract conceptual framework in which we reason about the metalogic now becomes the new metalogic. We can thus form a hierarchy that is unbounded in both directions --- we can study a more concrete object logic within the object logic, and we can jump from one metalogical level to the next.

  An important connection between the logic and metalogic is given in \fullref{rem:set_definition_recursion}.
\end{remark}

\begin{definition}\label{def:classical_logic}
  Classical logic is a vague term that we use to describe a semantic framework, \fullref{def:propositional_semantics}, and a matching \hyperref[def:deductive_system]{deductive system}, \fullref{def:classical_propositional_deductive_systems}, for \hyperref[subsec:propositional_logic]{propositional logic} and also a semantic framework, \fullref{def:first_order_semantics}, and matching proof deduction deduction, \fullref{def:first_order_natural_deduction_system}, for \hyperref[subsec:first_order_logic]{first-order logic}, among others.

  It is characterized by the ability to use the law of double negation elimination \eqref{eq:thm:minimal_propositional_negation_laws/dne}. A more popular (but less accurate due to \fullref{thm:minimal_propositional_negation_laws}) characterization is that the law of the excluded middle \eqref{eq:thm:minimal_propositional_negation_laws/lem} holds. Within the metalogic, this law is called the \term{principle of bivalence} and states that either a statement holds or its negation holds.
\end{definition}

\begin{definition}\label{def:intuitionistic_logic}
  Intuitionistic logic is a generalization of \hyperref[def:classical_logic]{classical logic}. It is also called \term{constructive logic} due to the \hyperref[def:brouwer_heyting_kolmogorov_interpretation]{Brouwer-Heyting-Kolmogorov interpretation}. See \fullref{rem:brouwer_heyting_kolmogorov_interpretation_compatibility} for further discussion of the topic.

  Instead of the law of the excluded middle \eqref{eq:thm:minimal_propositional_negation_laws/lem}, we have the strictly weaker principle of explosion \eqref{eq:thm:minimal_propositional_negation_laws/efq} stating that everything can be proved given a contradiction.

  To these ideas there correspond \hyperref[def:propositional_heyting_algebra_semantics]{Heyting algebra semantics} and \hyperref[def:propositional_topological_semantics]{topological semantics} and a matching deductive system, \fullref{def:intuitionistic_propositional_deductive_systems}, for \hyperref[subsec:propositional_logic]{propositional logic}.
\end{definition}

\begin{definition}\label{def:minimal_logic}
  Minimal logic is a generalization of \hyperref[def:intuitionistic_logic]{intuitionistic logic}.

  Instead of the law of the excluded middle \eqref{eq:thm:minimal_propositional_negation_laws/lem} and the strictly weaker principle of explosion \eqref{eq:thm:minimal_propositional_negation_laws/efq}, we have the even weaker law of non-contradiction \eqref{eq:thm:minimal_propositional_negation_laws/lnc}.

  Metalogically speaking, we can only conclude that there is no statement such that both the statement and its negation are true. If the statement instead does not hold, we cannot automatically conclude that its negation holds.

  \Fullref{def:minimal_propositional_axiomatic_deductive_system} provides a deductive system for \hyperref[subsec:propositional_logic]{propositional logic}, but we avoid studying semantics of minimal logic.
\end{definition}

\begin{remark}\label{rem:mathematical_logic_conventions}
  We will only work in \hyperref[def:classical_propositional_deductive_systems]{classical metalogic}. Outside the section on logic, we will use formulas and, more generally, use object logic only in dedicated places like \fullref{def:group/theory} describing the \hyperref[def:first_order_theory]{logical theory} of groups. Most axioms like \ref{def:norm/N1}-\ref{def:norm/N3} for norms are formulated entirely within the metalanguage under the assumption that we are working within a model of set theory. To keep a clear distinction between logical formulas and non-logical axioms and, more generally, to distinguishing between logic and metalogic, we use the following conventions:

  \begin{thmenum}
    \thmitem{rem:mathematical_logic_conventions/variable_symbols} Variables in the object language are denoted by the small Greek letters, usually \( \xi, \eta, \zeta \), while variables in the metalanguage are denoted by small Latin letters, usually \( x, y, z \). If needed, we add subscripts with indices.

    \thmitem{rem:mathematical_logic_conventions/formula_term_symbols} Formulas, which we only consider in the object language, are also denoted by small Greek letters --- \( \varphi, \psi, \theta, \chi \) --- and, so are terms --- \( \tau, \sigma, \rho, \kappa, \mu, \nu \).

    \thmitem{rem:mathematical_logic_conventions/propositional_constants} The propositional constants denoting truth and falsity are denoted by \( \top \) and \( \bot \) in the object language and by \( T \) and \( F \) in the metalanguage. This is only for the sake of following an established convention, and we still use \( \top \) and \( \bot \) in general \hyperref[def:semilattice/lattice]{lattices}.

    \thmitem{rem:mathematical_logic_conventions/connective_symbols} We usually prefer prose to symbolic quantifiers and connectives in the metalanguage. The longer double arrows \( \implies \) and \( \iff \) are sometimes used within the metalogic outside this section.

    \thmitem{rem:mathematical_logic_conventions/structure_pairs} We conflate structures in the metalogic (i.e. sets with functions and/or relations defined on them) with their domain --- see \fullref{rem:first_order_model_notation} for a discussion.

    \thmitem{rem:mathematical_logic_conventions/shorthands} We additionally use syntactic shorthands like \fullref{rem:propositional_formula_parentheses} and \fullref{rem:first_order_formula_conventions} when writing formulas.

    \thmitem{rem:mathematical_logic_conventions/quantification} We avoid writing excessive universal quantification and instead rely on implicit universal quantification as described in \fullref{thm:implicit_universal_quantification}. If we need the formulas to be closed, such as in the case of \hyperref[def:first_order_theory]{first-order theories} for example, we assume all formulas are closed and if they are not, we add explicit universal quantifiers in front.
  \end{thmenum}

  Some axioms like \eqref{eq:def:magma/idempotent} are formulated within the metalogic for convenience and clarity, but are used as formulas in the object language in theorems like \fullref{thm:positive_formulas_preserved_under_homomorphism}. In places like this, it is usually straightforward to translate axioms from the metalogic into logical formulas.
\end{remark}

\begin{remark}\label{rem:higher_order_logic}
  Since we describe first-order logic, it may be helpful to clarify why is it named so. It is merely a shorthand for \enquote{first-order predicate logic}. There are other predicate logical frameworks, namely second-order predicate logic (described in \cite[ch. VIII]{OpenLogicFull}) and higher-order predicate logic, also known as \enquote{simple type theory} (described in \cite[sec. 3]{Farmer2008}).

  Second-order logic allows us to quantify over relations between variables. In that case, we refer to the variables of first-order logic as \enquote{individuals} and to the relations as \enquote{relation variables}. This allows us, for example, to avoid axiom schemas like the \hyperref[def:zfc/specification]{axiom schema of specification} by instead replacing them with a single axiom that quantifies over unary relations. A downside of second-order logic is that it has worse properties --- it is incomplete in the sense that there exists no \hyperref[def:deductive_system]{deductive system} that is both sound and complete\footnote{refer to \cite[thm. 39.6]{OpenLogicFull}} and it is not compact in the sense that the analogue to \fullref{thm:first_order_compactness_theorem} does not hold\footnote{refer to \cite[thm. 39.7]{OpenLogicFull}}. This is attributed to the expressive power of second-order logic because a first-order axiom schema may have only a countable number of axioms while a second-order quantifier may range over uncountably many relations.

  Clearly anything that extends second-order logic must suffer from the same problems, however higher-order logic is still useful because it allows us to utilize some very powerful concepts. Rather than quantifying over relations over the relations over individuals that would happen in third-order logic, we instead consider the more abstract frameworks of type theory. Type theory itself comes in many flavors, but simple type theory can be viewed as a generalization of first-order logic --- see \cite[thm. 2]{Farmer2008}. The rough idea is that rather than having individual variables, relation variables, etc., we have \term{base types} and \term{type constructors}. The individual variables have a dedicated base type, and the types of functions and predicates are easily constructed using the basic type constructors, hence it is also easy to construct higher-order functions and predicates. The syntax of simple type theory is inspired by \( \lambda \)-calculus, which is a huge topic in itself and one of the frameworks for studying computability theory. The semantics of simple type theory are merely an extension of first-order semantics with different universes for different types. Like second-order logic, however, type theories have worse properties than those of first-order logic.

  Another benefit of type theories is that they allow for multiple base types. For example, in the definition of a \hyperref[def:vector_space]{vector space}, we have scalars and vectors, and we introduce an axiom schema parameterized by the scalars. In contrast, we could have a type for scalars and a type for vectors. This is also easily achievable in first-order logic via the so-called \enquote{many-sorted first-order logic}, where the types are called \enquote{sorts}. We lack type constructors, and thus we are restricted in how our functions and predicates are defined, however for simple cases many-sorted first-order logic is just as useful as simple type theory. As a matter of fact, both many-sorted first-order logic languages and simple type theory languages can be reformulated as first-order logic languages --- see \cite[ch. 8]{Farmer2008}.

  We circumvent the need for any of these higher-order logical frameworks by using set theory --- see \fullref{rem:first_order_theories_in_zfc}.
\end{remark}
