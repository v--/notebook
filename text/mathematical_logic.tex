\section{Mathematical logic}\label{sec:mathematical_logic}

Mathematical logic uses mathematics to study logic and vice versa. We try to give very general formal definitions in \fullref{subsec:logical_frameworks}. Nevertheless, in order to build a theory to study theories, if we are to use \hyperref[def:well_founded_relation]{well-founded} concepts, we must accept some notions as fundamental without being able to properly define them. We will call such an informal definition a \enquote{concept}.

\begin{concept}\label{con:judgment}
  Our first fundamental notion is that of a \term{judgment}. To quote \incite[9]{MartinLof1996Meanings}:
  \begin{displayquote}
    \enquote{What is a judgment?} is no small question, because the notion of judgment is just about the first of all the notions of logic, the one that has to be explained before all the others, before even the notions of proposition and truth, for instance. There is therefore an intimate relation between the answer to the question what a judgment is and the very question what logic itself is. I shall start by giving a very simple answer, which is essentially right: after some elaboration, at least, I hope that we shall have a sufficiently clear understanding of it. And the definition would simply be that, when understood as an act of judging, a judgment is nothing but an act of knowing, and, when understood as that which is judged, it is a piece or, more solemnly, an object of knowledge.
  \end{displayquote}
\end{concept}
\begin{comments}
  \item In more practical terms, within this monograph, what we call judgments are statements regarding objects of study --- every theorem, lemma, proposition or minor assertion listed here is a judgment. Mathematical logic allows us to systemize judgments and study them and the connections between them.

  \item In order to study judgments formally, we must encode them into some language that is convenient for us to study. We will call such an encoded judgment a \enquote{sentence}. \Fullref{def:consequence_operator} allows us to introduce important definitions independent of the encoding scheme.
\end{comments}

\begin{concept}\label{con:metalogic}
  When studying a \hyperref[def:logical_framework]{logical framework}, we call this framework the \term[en=object logic (\cite[3]{Kleene2002Logic})]{object logic}, and, for an implicitly fixed signature, we will call the corresponding collection of sentences the \term[ru=предметный язык (\cite[35]{Герасимов2011}), en=object language (\cite[3]{Kleene2002Logic})]{object language}. Correspondingly, when studying a concrete \hyperref[def:logical_theory]{logical theory}, we call it the \term{object theory}.

  The collection of all judgments also forms a language, which we call the \term[ru=метаязык (\cite[35]{Герасимов2011}), en=metalanguage (\cite[3]{Kleene2002Logic})]{metalanguage}. We reason about the metalanguage via some logical framework, the \term{metalogic}. The metalogic allows us to encode the concepts we work with, including the object logic, as a logical theory, which we call our \term[en=metatheory (\cite[199]{Kleene2002Logic})]{metatheory}.

  \Cref{con:metalogic} schematically represents the relationships between these concepts.

  \begin{figure}[!ht]
    \begin{equation*}
      \includegraphics[page=1]{output/rem__metalogic}
    \end{equation*}
    \caption{The relationship between the concepts in \fullref{con:metalogic}}\label{fig:con:metalogic}
  \end{figure}

  A human begins his journey of understanding mathematical logic by using the metalogical syntactic or semantic consequence operator to prove statements about the object logic. Whenever he improves his understanding of a given logical framework, this also improves his understanding of how to reason within the confines of this framework, that is, how to use it as a metalogic. This process can be roughly visualized as \enquote{moving} one level down in \cref{fig:con:metalogic}, so that our previous object theory becomes the new metatheory.
\end{concept}
\begin{comments}
  \item Set theory features an important connection between the object theory and the metatheory --- see \fullref{rem:set_definition_recursion}.
\end{comments}

\begin{remark}\label{rem:mathematical_logic_conventions}
  Several conventions related to logic are followed through the monograph.

  We only work within \hyperref[def:classical_logic]{classical} \hyperref[con:metalogic]{metalogic}. Outside this section, we will only adhere to formal logic in dedicated places like \fullref{def:lattice/theory}, which describes the \hyperref[def:logical_theory]{logical theory} of \hyperref[def:lattice]{lattices}. Other theories like that of \hyperref[def:complete_lattice]{complete lattices} cannot be elegantly presented via first-order sentences and are instead formulated entirely within the metalanguage under the assumption that we are working within a model of set theory --- \hyperref[def:axiom_of_universes]{\logic{ZFC+U}}, to be more precise. Using logical theories like \fullref{def:lattice/theory} allows us to implicitly reuse a plethora of definitions and theorems regarding first-order models from \fullref{subsec:first_order_models}. To keep a clear distinction between logical formulas and non-logical conditions and, more generally, to distinguishing between the object logic and the metalogic, we use the following conventions:

  \begin{thmenum}
    \thmitem{rem:mathematical_logic_conventions/variable_symbols} Variables in the object language are denoted by the small Greek letters, usually \( \xi, \eta, \zeta \), while variables in the metalanguage are denoted by small Latin letters, usually \( x, y, z \). If needed, we add subscripts with indices.

    \thmitem{rem:mathematical_logic_conventions/greek_alphabet} Although the entire Greek alphabet is allocated for first-order variables in the \hyperref[def:formal_grammar]{formal grammar} in \fullref{def:first_order_syntax/formula}, we only use a few letters, and allocate the rest towards other concepts in the metatheory:
    \begin{equation*}
      \underbrace{\alpha, \beta, \gamma, \delta, \varepsilon}_{\T{ordinals}},
      \overset{ \mathclap{ \substack{ \T{variables} \mathstrut \\ \mathstrut \downarrow \downarrow } } }{\zeta, \eta},
      \underset{ \mathclap{ \substack{ \uparrow \\ \T{formulas} } } }{\theta},
      \iota,
      \underbrace{\kappa, \lambda, \mu, \nu}_{\T{cardinals}},
      \overset{ \mathclap{ \substack{ \T{variables} \mathstrut \\ \mathstrut \downarrow } } }{\xi},
      \underbrace{\omicron, \pi, \rho, \sigma, \tau}_{\T{terms}},
      \upsilon,
      \underbrace{\varphi, \chi, \psi, \omega}_{\T{formulas}}
    \end{equation*}

    Of course, whenever we need additional symbols, we may break the above convention.

    \thmitem{rem:mathematical_logic_conventions/connective_symbols} We prefer prose to symbolic quantifiers and connectives for statements in the metalanguage. For the object logic, we place dots over the base connectives from \fullref{def:propositional_alphabet}. This allows us to visually distinguish disjunctions from joins in \hyperref[def:lattice]{lattices}, for example in formulas such as
    \begin{equation*}
      (\xi \vee \eta) \synvee (\xi \vee \zeta).
    \end{equation*}

    We also place a dot over the equality, which allows us to distinguish equality of formulas from equality within formulas.

    A similar convention is used by Peter Hinman --- see \cite[rem. 2.1.3]{Hinman2005}.

    To avoid clutter, we do not put dots over predicate and functional symbols.

    \thmitem{rem:mathematical_logic_conventions/propositional_constants} We define the propositional constants for truth and falsity via \( \syntop \) and \( \synbot \) in the object language and by \( T \) and \( F \) in the metalanguage. We use \( \top \) and \( \bot \) in general \hyperref[def:lattice]{lattices}.

    \thmitem{rem:mathematical_logic_conventions/structure_pairs} We often conflate structures in the metalogic (i.e. sets with functions and/or relations defined on them) with their domain --- see \fullref{rem:first_order_model_notation} for a further discussion.

    \thmitem{rem:mathematical_logic_conventions/shorthands} We additionally use syntactic shorthands like \fullref{rem:propositional_formula_parentheses} and \fullref{rem:first_order_formula_conventions} when writing formulas.

    \thmitem{rem:mathematical_logic_conventions/quantification} We avoid writing excessive universal quantification. If we need the formulas to be closed, such as in the case of \hyperref[def:first_order_theory]{first-order theories}, we rely on implicit quantification as described in \fullref{thm:implicit_universal_quantification}.

    \thmitem{rem:mathematical_logic_conventions/concepts} Some definitions are not completely formal --- beginning with fundamental concepts like the judgments discussed in \fullref{con:judgment} and all the way to the Backus normal form discussed in \fullref{con:backus_normal_form}. We name such definitions \enquote{concepts} to distinguish them from completely formal ones.
  \end{thmenum}
\end{remark}
