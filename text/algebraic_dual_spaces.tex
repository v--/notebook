\section{Algebraic dual spaces}\label{sec:algebraic_dual_spaces}

In this subsection, we restrict ourselves to fields rather than arbitrary ring.

\begin{definition}\label{def:dual_vector_space}\mcite[50]{Knapp2016BasicAlgebra}
  Let \( V \) be a \hyperref[def:vector_space]{vector space} over the \hyperref[def:field]{field} \( \BbbK \). By \fullref{thm:functions_over_algebra}, the set \( \hom(V, \BbbK) \) of all \hyperref[def:linear_function]{linear maps} from \( V \) to the underlying field \( \BbbK \) also form a vector space over \( \BbbK \).

  We will call this space the \term{algebraic dual space} of \( V \) and denote it by \( V^* \). We will call the functions in \( V^* \) \term{linear functionals}. The prefix \enquote{algebraic} is important when confusion is possible with \hyperref[def:continuous_dual_space]{continuous linear functionals}.
\end{definition}

\begin{remark}\label{rem:dual_space_bilinear_form}\mcite[16]{ИоффеТихомиров1974ЭкстремЗадачи}
  If \( l \) is a \hyperref[def:dual_vector_space]{linear functional} over \( V \), we often use the notation \( \inprod l x \) rather than the function notation \( l(x) \). This is an extension of the notation for \hyperref[def:inner_product_space]{inner product spaces}.

  Moreover, \( \inprod \anon \anon \) is a \hyperref[def:multilinear_function]{bilinear function} from the Cartesian product \( V^* \times V \) to \( \BbbK \). Hence, if \( V \) is isomorphic to \( V^* \), then this is precisely an inner product.
\end{remark}

\begin{concept}\label{con:functional}
  The term \enquote{functional} as a noun has no definite meaning.

  \begin{itemize}
    \item In the context of linear algebra, and in particular \fullref{def:dual_vector_space}, the term \enquote{functional} refers to \enquote{linear functional}, i.e. a \hyperref[def:linear_function]{linear map} from a \hyperref[def:vector_space]{vector space} to its base field.

    This terminology can be found, for example, in \cite[50]{Knapp2016BasicAlgebra} and \cite[sec. 26.1]{Тыртышников2007ЛинейнаяАлгебра}.

    \item In the context of functional analysis, \enquote{linear functional} may refer to either \hyperref[def:continuous_dual_space]{continuous linear functionals} from some \hyperref[def:topological_vector_space]{topological vector space} to its base field, or to arbitrary linear functionals.

    The former terminology can be found, for example, in \cite[def. 3.1]{Rudin1991FuncAnalysis} and \cite[sec. 1.3]{Clarke2013OptimalControl}.

    An arbitrary map from a topological vector space to its field may also be called a functional. For example, \cite[102]{KufnerFucik1980NonlinearDE} and \cite[223]{Deimling1985NonlinearFA} refer to \enquote{nonlinear functionals}. \hyperref[def:minkowski_functional]{Minkowski functionals} are notoriously nonlinear.

    \item In the context of recursive functions, for example in \cite{StanfordPlato:recursive_functions}, functionals are defined as \enquote{operations which map one or more functions of type \( \BbbN^k \to \BbbN \) (possibly of different arities) to other functions}.
  \end{itemize}

  The commonality between linear algebra and functional analysis is that \enquote{functional} refers to a map from a vector space to its base field. The commonality between functional analysis and logic is that \enquote{functional} refers to a map acting on a set of functions.
\end{concept}

\begin{remark}\label{rem:vector_space_and_dual_space}
  A vector space \( V \) over \( \BbbK \) with \hyperref[def:hamel_basis]{basis} \( E \) is, by definition, isomorphic to the \hyperref[def:free_semimodule]{free module} \( \BbbK^{\oplus E} \). We can thus regard \( V \) as the set of all \hyperref[def:set_finiteness]{finitely}-\hyperref[def:function_support]{supported} functions from \( E \) to \( \BbbK \).

  By \fullref{thm:free_semimodule_universal_property}, the linear functions from \( V \) to \( \BbbK \) are precisely the linear extensions of the functions from \( E \) to \( \BbbK \).

  It is now clear that \( V \) can be embedded in \( V^* \). This is explicitly given by the map \( e \mapsto \pi_e \), where \( \pi_e \) is the \hyperref[def:basis_decomposition]{projection} onto the basis vector \( e \).

  The space \( V \) is thus finite-dimensional if and only if \( V \) and \( V^* \) are isomorphic. We often restrict ourselves to \hyperref[def:continuous_dual_space]{continuous linear functionals}, in which case even infinite-dimensional vector spaces can be isomorphic to their duals --- see \fullref{sec:hilbert_spaces}.
\end{remark}

\begin{remark}\label{rem:finite_dimensional_dual_space_isomorphism}
  As discussed in \fullref{rem:vector_space_and_dual_space}, the vector space \( \BbbK^n \) is isomorphic to its dual.

  We discussed in \fullref{rem:matrices_as_functions} that vectors in \( \BbbK^n \) can be regarded as \hyperref[def:array/column_vector]{column vectors}. Depending on the situation, we regard linear functionals as either:
  \begin{itemize}
    \item Functions acting on vectors.
    \item Row vectors, which can be multiplied with column vectors from \( \BbbK^n \).
    \item Given the \hyperref[def:inner_product_space]{inner product} \( \inprod l x \coloneqq l^T x \), we can identify functionals with column vectors so that the functional \( l \) can be identified in \( x \mapsto \inprod l x \).
  \end{itemize}

  For example, given the real \hyperref[def:differentiability]{differentiable} function \( f(x, y) = xy \), we can regard its gradient at the point \( (x_0, y_0) \) as the row vector
  \begin{balign*}
    f'(x_0, y_0) =
    \begin{pmatrix}
      y_0 & x_0
    \end{pmatrix}.
  \end{balign*}

  This is a linear functional that acts on vectors from \( \BbbR^2 \) by multiplying them from the left.
\end{remark}

\begin{remark}\label{rem:complex_linear_functional}
  A complex-valued linear function \( l: V \to \BbbC \) is entirely determined by its real part \( \real l: V \to \BbbR \). More precisely,
  \begin{equation*}
    \imag \inprod l x = -\real i \inprod l x
  \end{equation*}
  because
  \begin{equation*}
    -\real i \inprod l x
    =
    -\real (i \real \inprod l x + i^2 \imag \inprod l x)
    =
    \imag \inprod l x.
  \end{equation*}
\end{remark}

\begin{definition}\label{def:vector_space_annihilator}\mcite[52]{Knapp2016BasicAlgebra}
  Fix a subset \( S \subseteq V \) of the vector space \( V \) over \( \BbbK \). We define the \term{annihilator} of \( S \) as the vector space of functionals
  \begin{equation*}
    S^\perp \coloneqq \set{ l \in V^* \given \qforall {x \in S} l(x) = 0_\BbbK }.
  \end{equation*}
\end{definition}

\begin{example}\label{ex:def:vector_space_annihilator}
  We list several examples of \hyperref[def:vector_space_annihilator]{vector space annihilators}:
  \begin{thmenum}
    \thmitem{ex:def:vector_space_annihilator/whole} The annihilator of the entire space \( V \) is the zero subspace
    \begin{equation*}
      V^\perp = \set{ 0_{V^*} }.
    \end{equation*}

    \thmitem{ex:def:vector_space_annihilator/zero} The annihilator of the zero subspace \( \set{ 0_V } \) is the entire space
    \begin{equation*}
      \set{ 0_V }^\perp = V^*.
    \end{equation*}

    \thmitem{ex:def:vector_space_annihilator/complement} Consider the space \( \BbbR^2 \) with basis \( \set{ x, y } \). The annihilator of the subspace
    \begin{equation*}
      \set{ tx \given t \in \BbbR }
    \end{equation*}
    is
    \begin{equation*}
      \set{ ty \given t \in \BbbR }.
    \end{equation*}
  \end{thmenum}
\end{example}

\begin{remark}\label{rem:double_dual}
  We discussed in \fullref{rem:vector_space_and_dual_space} that any vector space \( V \) can be embedded into its dual \( V^* \). The dual can, in turn, be embedded into the double dual \( V^{**} \).

  What is more remarkable is that \( V \) can be directly embedded into \( V^{**} \) via by identifying the vector \( x \) with the map \( l \mapsto l(x) \).

  This is an isomorphism if and only if \( V \) is finite dimensional. When restricted to only \hyperref[def:continuous_dual_space]{continuous functionals}, it is possible that \( V \) is isomorphic to \( V^{**} \) --- see \fullref{sec:reflexive_spaces}.
\end{remark}

\begin{theorem}\label{thm:linear_functionals_over_c}
  Let \( X \) be a \hyperref[def:vector_space]{vector space} over \( \BbbC \). There is a bijection between the real-valued and the complex-valued linear functionals on \( X \).
\end{theorem}
\begin{proof}
  Let \( c: X \to \BbbC \) be a complex-valued linear functional. Denote \( a(x) \coloneqq \real c(x) \) and \( b(x) \coloneqq \imag c(x) \). Then \( a: X \to \BbbR \) and \( b: X \to \BbbR \) are linear functionals. We will show that \( a(x) \) uniquely determines \( b(x) \) and hence \( c(x) \).

  Note that \( c(ix) = a(ix) + i b(ix) = i a(x) - b(x) \). Therefore, \( b(x) = a(ix) - c(ix) \) and
  \begin{equation*}
    c(x) = a(x) + i (a(ix) - c(ix)) = a(x) - a(x) + c(x) = c(x).
  \end{equation*}
\end{proof}

\begin{remark}\label{rem:linear_functionals_over_c}
  \Fullref{thm:linear_functionals_over_c} allows us to identify the dual space \( X* \) of a complex vector space \( X \) with \( \hom(X, \BbbR) \) in the case of an algebraic \hyperref[def:dual_vector_space]{dual} or with the corresponding subspace in the case of a \hyperref[def:continuous_dual_space]{continuous dual space}.

  This allows us to reuse some theory for real vector spaces, for example hyperplane \hyperref[def:hyperplane_separation]{separation}.
\end{remark}
