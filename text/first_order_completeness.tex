\section{First-order completeness}\label{sec:first_order_completeness}

\begin{remark}\label{rem:henkin_theories}
   We will present a proof of \fullref{thm:fol_completeness} based on \bycite{Henkin1949FOLCompleteness}. The first proof of this theorem is attributed to Kurt G\"odel, thus it is also called \enquote{Go\"del's completeness theorem}.

   Our presentation follows the modern adaptation of \incite[\S 3.1]{VanDalen2004LogicAndStructure}. Other adaptations are given in \cite[\S 4.2]{Shoenfield1967MathematicalLogic}, \cite[\S 4.5]{ШеньВерещагин2017ЯзыкиИИсчисления} and \cite[\S 2.6.3]{Герасимов2014Вычислимость}, and similar constructions for proving \fullref{thm:fol_semantic_compactness} are given in \cite[ch. 3]{Hinman2005Logic}.

   The differences are subtle but important. For example, Gerasimov imposes a weaker condition than in \cref{def:fol_henkin_completeness} --- if \( \qexists x \varphi \) belongs to \( T \), there must exist a closed term \( \tau \) such that \( \varphi[x \mapsto \tau] \) belongs to \( T \). This makes the interaction between \hyperref[def:fol_henkin_extension]{Henkin extensions} and other extensions more intricate, which in turn requires adapting \fullref{thm:fol_lindenbaums_lemma}.

   Furthermore, we prefer working with theories rather than arbitrary sets of formulas like in \fullref{sec:propositional_completeness}. If needed, we can relax this to sets of closed formulas, but there will be no gains in doing so.
\end{remark}

\paragraph{Henkin completion}

\begin{definition}\label{def:fol_henkin_completeness}\mcite[def. 3.1.4(iii)]{VanDalen2004LogicAndStructure}
  We say that the \hyperref[def:fol_theory]{first-order theory} \( T \) over \( \Sigma \) is \term[en=Henkin-complete (theory) (\cite[def. 3.1.7]{Hinman2005Logic}); Henkin theory (\cite[def. 3.1.4(iii)]{VanDalen2004LogicAndStructure})]{Henkin-complete} if, for every closed existential formula \( \qexists x \varphi \) over \( \Sigma \), there exists a constant \( c \) such that
  \begin{equation}\label{eq:def:fol_henkin_completeness/axiom}
    \qexists x \varphi \synimplies \varphi[x \mapsto c]
  \end{equation}
  belongs to \( T \). We call \( c \) a \term{witness} of \( \varphi \).
\end{definition}
\begin{comments}
  \item Note that we do not require neither \( \qexists x \varphi \) nor \( \varphi[x \mapsto c] \) to belong to \( T \).

  This allows us to define completions in \cref{thm:fol_henkin_completion_is_complete} and ensure that they are compatible with further extensions. This is one of several possible definitions of Henkin theories --- see \cref{rem:henkin_theories}.

  \item Since \( \qexists x \varphi \) is assumed to be closed, \( x \) is the only variable possibly free in \( \varphi \).
\end{comments}

\begin{definition}\label{def:fol_henkin_extension}\mcite[def. 3.1.6]{VanDalen2004LogicAndStructure}
  Consider some \hyperref[def:fol_theory]{first-order theory} \( T \) over the signature \( \Sigma \).

  Consider the \hyperref[def:fol_signature_extension]{extension} \( \Sigma^* \) of \( \Sigma \) where, for every existential formula \( \qexists x \varphi \) in \( T \), we add a new constant \( c_{x,\varphi} \). For the sake of determinism, suppose that \( c_{x,\varphi} \) is a symbol entirely determined by \( x \) and \( \varphi \).

  We define the (one-step) \term{Henkin extension} \( T^* \) as the consequence closure of
  \begin{equation}\label{eq:def:fol_henkin_extension}
    T \cup \set[\big]{ (\qexists x \varphi) \synimplies \varphi[x \mapsto c_{x,\varphi}] \given* \qexists x \varphi \T{belongs to} T }.
  \end{equation}
\end{definition}
\begin{comments}
  \item By iterating one-step Henkin extensions, we define Henkin completions in \cref{def:fol_henkin_completion}.

  \item Note that \( \qexists x \varphi \) is assumed to be closed, so \( x \) is the only variable possibly free in \( \varphi \).

  \item The determinism condition, although not present in \cite[def. 3.1.6]{VanDalen2004LogicAndStructure}, is important for some intermediate results like \cref{thm:def:fol_henkin_extension/scott_continuous}.
\end{comments}

\begin{theorem}[Theorem on constants]\label{thm:theorem_on_constants}\mcite[33]{Shoenfield1967MathematicalLogic}
  Over some \hyperref[def:fol_signature]{signature} \( \Sigma \), fix some set \( \Gamma \) of \hyperref[def:closed_fol_formula]{closed formulas} and a formula \( \varphi \) with free variables \( x_1, \ldots, x_n \).

  Consider the \hyperref[def:fol_signature_extension]{extension} \( \Sigma^+ \) of \( \Sigma \) with some constants \( c_1, \ldots, c_n \).

  Then
  \begin{equation*}
    \Gamma \vdash_\Sigma \qforall {x_1} \ldots \qforall {x_n} \varphi \T{if and only if} \Gamma \vdash_{\Sigma^+} \varphi[x_1 \mapsto c_1, \ldots, x_n \mapsto c_n].
  \end{equation*}
\end{theorem}
\begin{comments}
  \item The name of this theorem is based on \cite{Shoenfield1967MathematicalLogic}. It is useful for proving properties of \hyperref[def:fol_henkin_extension]{Henkin extensions}.
\end{comments}
\begin{proof}
  \SufficiencySubProof Let \( P \) be a proof tree over \( \Sigma \), deriving \( \qforall {x_1} \ldots \qforall {x_n} \varphi \) from \( \Gamma \). Let \( P' \) be the following extension:
  \begin{equation*}
    \begin{prooftree}
      \hypo{}
      \ellipsis {\( P \)} { \qforall {x_1} \qforall {x_2} \ldots \qforall {x_n} \varphi }
      \infer1[\ref{inf:def:fol_natural_deduction/forall/elim}]{ \qforall {x_2} \ldots \qforall {x_n} \varphi[x_1 \mapsto c_1] }
      \infer1[\ref{inf:def:fol_natural_deduction/forall/elim}]{ }
      \ellipsis {} {}
      \infer1[\ref{inf:def:fol_natural_deduction/forall/elim}]{ \varphi[x_1 \mapsto c_1] \ldots [x_n \mapsto c_n] }
    \end{prooftree}
  \end{equation*}

  Since \( c_k \) has no free variables, \cref{thm:alg:fol_formula_substitution/composition} implies that
  \begin{equation*}
    \varphi[x_1 \mapsto c_1, \ldots, x_n \mapsto c_n] = \varphi[x_1 \mapsto c_1] \ldots [x_n \mapsto c_n].
  \end{equation*}

  Therefore, \( P' \) derives \( \varphi[x_1 \mapsto c_1, \ldots, x_n \mapsto c_n] \) from \( \Gamma \) over \( \Sigma^+ \).

  \NecessitySubProof Let \( P \) be a proof tree over \( \Sigma^+ \), deriving \( \varphi[x_1 \mapsto c_1, \ldots, x_n \mapsto c_n] \) from \( \Gamma \).

  Let \( y_1, \ldots, y_n \) be distinct variables not present in \( P \), and let \( P' \) be the tree obtained from \( P \) by replacing \( c_k \) with \( y_k \). All rule applications remain, so \( P' \) is a well-formed proof tree over \( \Sigma \).

  Since \( y_1, \ldots, y_n \) are not present in \( \Gamma \), they can be used as eigenvariables; Thus, \( P' \) derives \( \varphi[x_1 \mapsto y_1, \ldots, x_n \mapsto y_n] \) from \( \Gamma \) over \( \Sigma \).

  \Cref{thm:alg:fol_formula_substitution/composition} implies that
  \begin{equation*}
    \varphi[x_1 \mapsto y_1, \ldots, x_n \mapsto y_n] = \varphi[x_1 \mapsto y_1] \ldots [x_n \mapsto y_n],
  \end{equation*}
  thus we can build the following tree:
  \begin{equation*}
    \begin{prooftree}
      \hypo{}
      \ellipsis {\( P' \)} { \varphi[x_1 \mapsto y_1] \ldots [x_{n-1} \mapsto y_{n-1}][x_n \mapsto y_n] }
      \infer1[\ref{inf:def:fol_natural_deduction/forall/intro}]{ \qforall {x_n} \varphi[x_1 \mapsto y_1] \ldots [x_{n-1} \mapsto y_{n-1}] }
      \infer1[\ref{inf:def:fol_natural_deduction/forall/intro}]{ }
      \ellipsis {} {}
      \infer1[\ref{inf:def:fol_natural_deduction/forall/intro}]{ \qforall {x_1} \ldots \qforall {x_n} \varphi }
    \end{prooftree}
  \end{equation*}
\end{proof}

\begin{lemma}\label{thm:natural_deduction_pulling_existential_quantifier}
  For \hyperref[def:fol_natural_deduction]{classical first-order natural deduction}, the following rule is \hyperref[con:inference_rule_admissibility]{admissible}:
  \begin{equation*}\taglabel[\ensuremath{ \exists_{\uparrow} }]{inf:thm:natural_deduction_pulling_existential_quantifier}
    \begin{prooftree}
      \hypo{ \varphi \synimplies \qexists x \psi }
      \infer1[\ref{inf:thm:natural_deduction_pulling_existential_quantifier}]{ \qexists x (\varphi \synimplies \psi) }
    \end{prooftree}
  \end{equation*}
  where \( x \) is not free in \( \varphi \).
\end{lemma}
\begin{comments}
  \item The proof tree is based on \cite{MathSE:natural_deduction_moving_quantifiers}.
\end{comments}
\begin{proof}
  \begin{equation*}
    \begin{prooftree}
      \hypo{ [\synneg \qexists x (\varphi \synimplies \psi)]^t }

      \hypo{ [\synneg \qexists x (\varphi \synimplies \psi)]^t }

      \hypo{ [\varphi \synimplies \qexists x \psi]^u }
      \hypo{ [\varphi]^v }
      \infer2[\ref{inf:def:propositional_natural_deduction/imp/elim}]{ \qexists x \psi }

      \hypo{ [\psi]^w }
      \infer1[\ref{inf:def:propositional_natural_deduction/imp/intro}]{ \varphi \synimplies \psi }
      \infer1[\ref{inf:def:fol_natural_deduction/exists/intro}]{ \qexists x (\varphi \synimplies \psi) }

      \infer[left label={\( v \)}]2[\ref{inf:def:fol_natural_deduction/exists/elim}]{ \qexists x \psi }
      \infer2[\ref{inf:def:propositional_natural_deduction/neg/elim}]{ \synbot }
      \infer1[\ref{inf:def:propositional_natural_deduction/bot/raa}]{ \psi }

      \infer[left label={\( w \)}]1[\ref{inf:def:propositional_natural_deduction/imp/intro}]{ \varphi \synimplies \psi }
      \infer1[\ref{inf:def:fol_natural_deduction/exists/intro}]{ \qexists x (\varphi \synimplies \psi) }

      \infer2[\ref{inf:def:propositional_natural_deduction/neg/elim}]{ \synbot }
      \infer[left label={\( t \)}]1[\ref{inf:def:propositional_natural_deduction/bot/raa}]{ \qexists x (\varphi \synimplies \psi) }
    \end{prooftree}
  \end{equation*}
\end{proof}

\begin{proposition}\label{thm:fol_henkin_extension_is_conservative}\mcite[lemma 3.1.7]{VanDalen2004LogicAndStructure}
  The (one-step) \hyperref[def:fol_henkin_extension]{Henkin extension} of a natural deduction theory is \hyperref[def:fol_theory/conservative]{conservative}.
\end{proposition}
\begin{proof}
  Let \( T \) be a theory over \( \Sigma \). Let \( T^* \) be the Henkin extension of \( T \) with signature \( \Sigma^+ \). Denote by \( \Gamma \) the set of additional axioms from \( T^* \), so that \( T^* = \op*{Th}(T \cup \Gamma) \).

  Fix a formula \( \varphi \) over \( \Sigma \) that belongs to \( T^* \). Since \( T^* \) is a consequence closure of \( T \cup \Gamma \), there exists a \hyperref[def:fol_natural_deduction_proof_tree]{proof tree} \( P \) for \( \varphi \) whose open assumptions are from \( T \cup \Gamma \).

  Let \( \Gamma_0 \) be the subset of formulas of \( \Gamma \) that are open assumptions in \( P \). We will recursively build a sequence \( \ldots \subseteq \Gamma_2 \subseteq \Gamma_1 \subseteq \Gamma_0 \) that eventually stabilizes at the empty set, and a corresponding sequence \( P_1, P_2, \ldots \) of proof trees, where \( P_k \) derives \( \varphi \) from \( T \cup \Gamma_k \) and does not contain new constants except those from \( \Gamma_k \).

  We start with \( \Gamma_0 \) and \( P_0 \coloneqq P \). At step \( k + 1 \), suppose we have already constructed \( \Gamma_k \) and \( P_k \). If \( \Gamma_k \) is empty, we are done with the proof. Otherwise, fix an axiom \( (\qexists x \psi) \synimplies \psi[x \mapsto c_{x,\psi}] \) from \( \Gamma_k \) and define \( \Gamma_{k+1} \) by removing it from \( \Gamma_k \).

  \Fullref{thm:fol_natural_deduction_deduction_theorem} gives us a proof tree \( P_k' \) deriving
  \begin{equation*}
    ((\qexists x \psi) \synimplies \psi[x \mapsto c_{x,\psi}]) \synimplies \varphi
  \end{equation*}
  from \( T \cup \Gamma_{k+1} \), and \fullref{thm:theorem_on_constants} gives us a tree \( P_k^\dprime \) not containing \( c_{x,\psi} \) and deriving
  \begin{equation*}
    \qforall y \parens[\big]{ ((\qexists x \psi) \synimplies \psi[x \mapsto y]) \synimplies \varphi }
  \end{equation*}
  from \( T \cup \Gamma_{k+1} \).

  Finally, let \( P_{k+1} \) be the following tree:
  \footnotesize
  \begin{equation*}
    \begin{prooftree}[separation=1em]
      \hypo{ [\qexists x \psi]^u }

      \hypo{ [\psi[x \mapsto y]]^v }
      \infer1[\ref{inf:def:fol_natural_deduction/exists/intro}]{ \qexists y \psi[x \mapsto y] }

      \infer[left label={\( v \)}]2[\ref{inf:def:fol_natural_deduction/exists/elim}]{ \qexists y \psi[x \mapsto y] }

      \infer[left label={\( u \)}]1[\ref{inf:def:propositional_natural_deduction/imp/intro}]{ (\qexists x \psi) \synimplies (\qexists y \psi[x \mapsto y]) }
      \infer1[\ref{inf:thm:natural_deduction_pulling_existential_quantifier}]{ \qexists y ((\qexists x \psi) \synimplies \psi[x \mapsto y]) }

      \hypo{}
      \ellipsis {\( P_k^\dprime \)} { \qforall y \parens[\big]{ ((\qexists x \psi) \synimplies \psi[x \mapsto y]) \synimplies \varphi } }
      \infer1[\ref{inf:def:fol_natural_deduction/forall/elim}]{ ((\qexists x \psi) \synimplies \psi[x \mapsto y]) \synimplies \varphi }

      \hypo{ [(\qexists x \psi) \synimplies \psi[x \mapsto y]]^w }
      \infer2[\ref{inf:def:propositional_natural_deduction/imp/elim}]{ \varphi }
      \infer[left label={\( w \)}]2[\ref{inf:def:fol_natural_deduction/exists/elim}]{ \varphi }
    \end{prooftree}
  \end{equation*}
  \normalsize

  Here \( P_{k+1} \) derives \( \varphi \) from \( T \cup \Gamma_{k+1} \), as desired.
\end{proof}

\begin{lemma}\label{thm:upward_union_of_theories}
  With respect to a \hyperref[def:consequence_relation/compactness]{compact} \hyperref[def:consequence_relation]{consequence relation}, the union of an \hyperref[def:directed_set]{upward-directed family} of \hyperref[def:fol_theory]{first-order theories} is again a theory.
\end{lemma}
\begin{comments}
  \item We take the union of signature symbols also.
\end{comments}
\begin{proof}
  Let \( \seq{ T_k }_{k \in \mscrK} \) be an upward-directed family of theories and let \( T \) be their union.

  If \( T \vdash \varphi \), by compactness there exists a finite subset \( T_0 \) of \( T \) such that \( T_0 \vdash \varphi \).

  For every formula \( \psi \) in \( T \), let \( T_{k_\psi} \) be the smallest theory to which \( \psi \) belongs, and let \( k_0 > \max\set{ k_\psi \given \psi \in T_0 } \). Then \( T_0 \) is a subset of \( T_{k_0} \).

  Since \( \varphi \) is a consequence of \( T_0 \), and hence of \( T_{k_0} \), it must belong to the latter, and hence to \( T \).

  Therefore, \( T \) is also closed under consequence, i.e. it is a theory.
\end{proof}

\begin{definition}\label{def:fol_henkin_completion}\mimprovised
  With respect to some \hyperref[def:consequence_relation/compactness]{compact} \hyperref[def:consequence_relation]{consequence relation}, consider some \hyperref[def:fol_theory]{first-order theory} \( T \) over \( \Sigma \) as follows.

  Define the \( \mu \)-indexed \hyperref[def:transfinite_sequence]{transfinite sequence} of \hyperref[def:fol_henkin_extension]{Henkin extensions}
  \begin{equation*}
    T^{*\alpha} \coloneqq \begin{cases}
      T,                                &\alpha = 0, \\
      (T^{*\beta})^*,                   &\alpha = \op{sc}(\beta), \\
      \bigcup_{\beta < \alpha} T^\beta, &\alpha \T{is a limit ordinal.}
    \end{cases}
  \end{equation*}

  For the signatures, we define an analogous sequence \( \seq{ \Sigma^{*\alpha} }_{\alpha < \mu} \).

  As we will show in \cref{thm:fol_henkin_completion_is_complete}, this sequence stabilizes at \( \alpha = \omega \). We thus call \( T^{*\omega} \) the \term{Henkin completion} of \( T \).
\end{definition}
\begin{comments}
  \item We extract the constructions from \cite[lemma 3.1.8]{VanDalen2004LogicAndStructure} and \cite[47]{Shoenfield1967MathematicalLogic} and take care of some formal details.

  \item As the name suggests, the completion is \hyperref[def:fol_henkin_completeness]{Henkin-complete} --- see \cref{thm:fol_henkin_completion_is_complete}.
\end{comments}
\begin{defproof}
  \Cref{thm:upward_union_of_theories} shows correctness of the limit step.
\end{defproof}

\begin{lemma}\label{thm:fol_henkin_extensions_scott_continuous}
  As an operator on \hyperref[def:fol_theory]{theories} over \hyperref[def:consequence_relation/compactness]{compact} \hyperref[def:consequence_relation]{consequence relations}, the \hyperref[def:fol_henkin_extension]{Henkin extension} operator \( (\anon)^* \) is \hyperref[def:scott_continuity]{Scott-continuous}.
\end{lemma}
\begin{proof}
  Let \( \seq{ T_k }_{k \in \mscrK} \) be an upward-directed family of theories and let \( T \) be their union (we take the union of signature symbols also). \Cref{thm:upward_union_of_theories} shows that \( T \) is also a theory.

  We must show that
  \begin{equation*}
    T^* = \bigcup_{k \in \mscrK} T_k^*.
  \end{equation*}

  Indeed, let \( \psi = (\qexists x \varphi) \synimplies \varphi[x \mapsto c_{x,\varphi}] \) be a new axiom introduced in \( T^* \). Then there exists some index \( T_{k_0} \) to which \( \qexists x \varphi \) belongs. Then \( \psi \) is also present in \( T_{k_0}^* \). It follows that
  \begin{equation*}
    T^* \subseteq \bigcup_{k \in \mscrK} T_k^*.
  \end{equation*}

  Conversely, let \( \psi = (\qexists x \varphi) \synimplies \varphi[x \mapsto c_{x,\varphi}] \) be a new axiom introduced in some \( T_{k_0}^* \) for some index \( k_0 \). Then \( \qexists x \varphi \) is in \( T_{k_0} \), so \( \psi \) belongs to \( T^* \).
\end{proof}

\begin{proposition}\label{thm:fol_henkin_completion_is_complete}
  The \hyperref[def:fol_henkin_completion]{Henkin completion} of a natural deduction theory is \hyperref[def:fol_theory/conservative]{conservative}, \hyperref[def:fol_henkin_completeness]{Henkin-complete} and invariant under \hyperref[def:fol_henkin_extension]{Henkin extensions}.
\end{proposition}
\begin{proof}
  Consider the theory \( T \) over the signature \( \Sigma \).

  \SubProof{Proof that \( T^{*\omega} \) is conservative} If \( \varphi \) is a formula over \( \Sigma \) that belongs to \( T^{*\omega} \), it also belongs to some finite \( T^{*k_0} \). Using \cref{thm:def:fol_henkin_extension/conservative}, by induction on \( k_0 \) we can show that \( T^{*k_0} \) is conservative over \( T \). Therefore, \( \varphi \) belongs to \( T \).

  \SubProof{Proof that \( T^{*\omega} \) is a fixed point} \Cref{thm:upward_union_of_theories} implies that the completion \( T^{*\omega} \) is a theory. Since the Henkin extension operator is \hyperref[def:scott_continuity]{Scott-continuous} by \cref{thm:fol_henkin_extensions_scott_continuous}, \fullref{thm:knaster_tarski_iteration/continuous} implies that \( T^{*\omega} \) is a fixed point.

  \SubProof{Proof that \( T^{*\omega} \) is a Henkin-complete} Every existential formula \( \qexists x \varphi \) over \( \Sigma^{*\omega} \) is a formula over some signature \( \Sigma^{*k_0} \) of finite index. In this case, \( T^{*k_0} \), and hence also \( T^{*\omega} \), contains the formula
  \begin{equation*}
    (\qexists x \varphi) \synimplies \varphi[x \mapsto c_{x,\varphi}].
  \end{equation*}

  It follows that \( c_{x,\varphi} \) is a witness of \( \qexists x \varphi \) in \( T^{*\omega} \). Generalizing, we conclude that \( T^{*\omega} \) is Henkin-complete.
\end{proof}

\paragraph{Term models}

\begin{definition}\label{def:fol_term_model}\mcite[107]{VanDalen2004LogicAndStructure}
  With respect to \hyperref[def:fol_natural_deduction_system]{first-order natural deduction}, fix a \hyperref[def:fol_henkin_completeness]{Henkin-complete}, \hyperref[def:fol_theory/complete]{complete} and \hyperref[def:fol_theory/consistent]{consistent} theory \( T \) over the \hyperref[def:fol_signature]{signature} \( \Sigma \).

  Using the \hyperref[def:fol_closed_term]{closed terms} over \( \Sigma \), we will construct a \hyperref[def:fol_semantics/model]{model} \( \mscrT \) of \( T \). We will call this the \term{term model} of \( T \).

  We only construct \( \mscrT \) here, and delegate to \fullref{thm:fol_model_existence_theorem} the check that it satisfies \( T \).

  \begin{thmenum}
    \thmitem{def:fol_term_model/intermediate} Let \( C \) be the set of all closed terms over \( T \). This set is nonempty because \( \Sigma \) contains a witness for \( \qexists x (x \syneq x) \).

    We define an interpretation \( I \) for \( \Sigma \) on \( C \) as follows:
    \begin{thmenum}
      \thmitem{def:fol_term_model/intermediate/functions} For every \( n \)-ary function symbol \( f \), let
      \begin{equation*}
        I(f)(\tau_1, \ldots, \tau_n) \coloneqq f(\tau_1, \ldots, \tau_n).
      \end{equation*}

      Thus, an interpretation of a term is the term itself. This interpretation is determined entirely by the signature \( \Sigma \).

      \thmitem{def:fol_term_model/intermediate/predicates} For every \( n \)-ary predicate symbol \( p \), let
      \begin{equation*}
        I(p)(\tau_1, \ldots, \tau_n) \coloneqq \begin{cases}
          \semtop, &p(\tau_1, \ldots, \tau_n) \in T, \\
          \sembot, &\T{otherwise.}
        \end{cases}
      \end{equation*}

      Here we rely on \( T \) to determine whether a predicate holds for some given terms.
    \end{thmenum}

    Denote the obtained structure by \( \mscrC = (C, I) \).

    \thmitem{def:fol_term_model/congruence} The above construction is not sufficient to validate \( T \). We additionally need the relation
    \begin{equation*}
      \tau \cong \sigma \T{if and only if} (\tau \syneq \sigma) \in T.
    \end{equation*}

    \Cref{thm:fol_natural_deduction_equality} implies that this is a \hyperref[def:fol_congruence]{congruence} on \( \mscrC \).

    \thmitem{def:fol_term_model/quotient} Finally, we define \( \mscrT \) as the \hyperref[def:fol_quotient_structure]{quotient structure} of \( \mscrC \) by \( {\cong} \).
  \end{thmenum}
\end{definition}

\begin{theorem}[First-order model existence theorem]\label{thm:fol_model_existence_theorem}\mcite[107]{VanDalen2004LogicAndStructure}
  As in \cref{def:fol_term_model}, fix a \hyperref[def:fol_henkin_completeness]{Henkin-complete}, \hyperref[def:fol_theory/complete]{complete} and \hyperref[def:fol_theory/consistent]{consistent} theory \( T \) (with respect to \hyperref[def:fol_natural_deduction_system]{natural deduction}).

  We claim that the term model \( \mscrT \) is indeed a model of \( T \).
\end{theorem}
\begin{proof}
  We proceed by induction on the \hyperref[def:graph_cardinality/order]{graph order} of the \hyperref[def:fol_formula_ast]{abstract syntax tree} of the \hi{closed} formula \( \varphi \) to show that \( T \vdash \varphi \) if and only if \( \mscrT \vDash \varphi \). For the purpose of this proof, we do not account for the syntax trees of terms, i.e. we will consider all terms as having graph order \( 1 \) so that substitution leaves the order invariant.

  The base case of logical constants and the inductive cases with connectives are analogous to \fullref{thm:propositional_model_existence_theorem}, and can formally be translated via \fullref{alg:fol_propositional_formula_translation}.

  \SubProof{Proof when \( \varphi = (\tau \syneq \sigma) \)} Since \( \varphi \) is closed, we do not need to consider variable assignments here.

  \SufficiencySubProof* Suppose that \( T \vdash (\tau \syneq \sigma) \).

  Denote the \hyperref[def:fol_quotient_structure/projection]{quotient map} in the term model construction by \( \pi: \mscrC \to \mscrT \). We have \( \Bracks{\tau}_\mscrT = \pi(\tau) \), which equals \( \Bracks{\sigma}_\mscrT = \pi(\sigma) \) because \( \tau \cong \sigma \). \Cref{thm:fol_equality_characterization} implies that \( \Bracks{\tau \syneq \sigma}_\mscrT = \semtop \).

  \NecessitySubProof* Suppose that \( \mscrT \vDash (\tau \syneq \sigma) \).

  \Cref{thm:fol_equality_characterization} implies that
  \begin{equation*}
    \pi(\tau) = \Bracks{\tau}_\mscrT = \Bracks{\sigma}_\mscrT = \pi(\sigma),
  \end{equation*}
  thus \( \tau \cong \sigma \), i.e. \( T \vdash (\tau \syneq \sigma) \).

  \SubProof{Proof when \( \varphi = p(\tau_1, \ldots, \tau_n) \)} We have defined the interpretation of the intermediate structure \( \mscrC \) so that \( \Bracks{p(\tau_1, \ldots, \tau_n)}_\mscrC = \semtop \) if and only if \( T \vdash p(\tau_1, \ldots, \tau_n) \).

  The quotient \( \mscrT \) preserves this, so \( T \vdash p(\tau_1, \ldots, \tau_n) \) if and only if \( \mscrT \vDash p(\tau_1, \ldots, \tau_n) \).

  \SubProof{Proof when \( \varphi = \qforall x \psi \)}

  \SufficiencySubProof* Suppose that \( T \vdash \qforall x \psi \). Fix any closed term \( \tau \) over \( \Sigma \).

  Using \ref{inf:def:fol_natural_deduction/forall/elim}, we can derive \( \psi[x \mapsto \tau] \) from \( T \). Then, by the inductive hypothesis, \( \mscrT \) satisfies \( \psi[x \mapsto \tau] \).

  In particular, for any variable assignment \( v \), \cref{thm:fol_formula_semantics_of_assignment_substitution} implies that the modified assignment \( v_{x \mapsto \tau} \) satisfies \( \psi \).

  Therefore, since \( \tau \) was arbitrary,
  \begin{equation*}
    \Bracks{ \varphi }_\mscrT^v
    =
    \bigwedge\set[\big]{ \Bracks{\psi}_\mscrX^{v_{x \mapsto \tau}} \given* \tau \T{is a closed term} }
    =
    \bigwedge\set{ \semtop \given \tau \T{is a closed term} }
    =
    \semtop.
  \end{equation*}

  \NecessitySubProof* Suppose that \( \mscrT \vDash \qforall x \psi \).

  Since \( T \) is a consistent and complete theory, it contains either \( \qforall x \psi \) or \( \synneg \qforall x \psi \). Aiming at a contradiction, suppose that it contains the latter. By a construction analogous to \cref{ex:def:fol_natural_deduction/quantifier_duality}, we conclude that \( \qexists x \synneg \psi \) belongs to \( T \).

  Since \( T \) is Henkin-closed, there exists a witness \( c \) and a formula
  \begin{equation*}
    \qexists x \synneg \psi \synimplies \synneg \psi[x \mapsto c]
  \end{equation*}
  that belongs to \( T \). We can thus apply \ref{inf:def:propositional_natural_deduction/imp/elim} to conclude that \( \synneg \psi[x \mapsto c] \) is in \( T \).

  However, for every variable assignment \( v \), the modified assignment \( v_{x \mapsto c} \) satisfies \( \psi \), and \cref{thm:fol_formula_semantics_of_assignment_substitution} implies that \( v \) satisfies \( \psi[x \mapsto c] \). By the inductive hypothesis, \( \psi[x \mapsto c] \) belongs to \( T \).

  But \( T \) is consistent and cannot contain both \( \synneg \psi[x \mapsto c] \) and \( \psi[x \mapsto c] \). The obtained contradiction shows that \( \synneg \qforall x \psi \) cannot belong to \( T \).

  Therefore, its (un)negation \( \varphi = \qforall x \psi \) must be in \( T \).

  \SubProof{Proof when \( \varphi = \qexists x \psi \)}

  \SufficiencySubProof* Suppose that \( T \vdash \qexists x \psi \).

  Since \( T \) is Henkin-complete, there exists a witness \( c \) and a formula
  \begin{equation*}
    \qexists x \psi \synimplies \psi[x \mapsto c]
  \end{equation*}
  that belongs to \( T \). We can thus apply \ref{inf:def:propositional_natural_deduction/imp/elim} to conclude that \( T \vdash \psi[x \mapsto c] \).

  By the inductive hypothesis, we conclude that \( T \vDash \psi[x \mapsto c] \). Therefore, for any variable assignment \( v \),
  \begin{equation*}
    \Bracks{ \varphi }_\mscrT^v
    =
    \bigvee\set[\big]{ \Bracks{\psi}_\mscrX^{v_{x \mapsto \tau}} \given* \tau \T{is a closed term} }
    =
    \underbrace{\Bracks{\psi[x \mapsto c]}_\mscrT^v}_{\semtop} \vee \bigvee \cdots
    =
    \semtop.
  \end{equation*}

  \NecessitySubProof* Suppose that \( \mscrT \vDash \qexists x \psi \).

  Then there exists some closed term \( \tau \) such that \( \Bracks{\psi}_\mscrT^{v_{x \mapsto \tau}} = \semtop \) for any variable assignment \( v \). \Cref{thm:fol_formula_semantics_of_assignment_substitution} implies that
  \begin{equation*}
    \Bracks{ \psi[x \mapsto \tau] }_\mscrT^v = \underbrace{\Bracks{ \psi }_\mscrT^{v_{x \mapsto \tau}}}_{\semtop}.
  \end{equation*}

  By the inductive hypothesis, we have \( T \vdash \psi[x \mapsto \tau] \). It remains to apply \ref{inf:def:fol_natural_deduction/exists/intro} to conclude that \( T \vdash \qexists x \psi \).
\end{proof}

\paragraph{Completeness}

\begin{lemma}[Lindenbaum's lemma for first-order logic]\label{thm:fol_lindenbaums_lemma}
  For the \hyperref[def:fol_natural_deduction_system]{classical first-order natural deduction system}, every \hyperref[def:fol_theory/consistent]{consistent} set of closed formulas can be extended to a \hyperref[def:fol_theory/complete]{complete} consistent set.
\end{lemma}
\begin{comments}
  \item This is an analogue of \fullref{thm:propositional_lindenbaums_lemma}.
\end{comments}
\begin{proof}
  Can be shown via \fullref{thm:ultrafilter_lemma} on the \hyperref[thm:lindenbaum_tarski_algebras]{classical Lindenbaum-Tarski algebra}, similarly to \fullref{thm:propositional_lindenbaums_lemma}.
\end{proof}

\begin{corollary}\label{thm:fol_consistent_implies_satisfiable}
  Consider a \hyperref[def:fol_signature]{first-order signature} \( \Sigma \). Let \( \kappa \) be the (cardinal) number of symbols in \( \Sigma \).

  If a set of closed formulas over \( \Sigma \) is \hyperref[def:fol_theory/consistent]{consistent} with respect to \hyperref[def:fol_natural_deduction_system]{classical natural deduction}, it has a \hyperref[def:fol_semantics/model]{model} of cardinality at most \( \max\set{ \aleph_0, \kappa } \) --- at most \( \aleph_0 \) when \( \kappa \) is finite and at most \( \kappa \) otherwise.
\end{corollary}
\begin{comments}
  \item This statement has an analog for \hyperref[def:propositional_logic]{propositional logic} --- see \cref{thm:propositional_consistent_implies_satisfiable}.
\end{comments}
\begin{proof}
  Let \( T \coloneqq \op*{Th}(\Gamma) \) be the theory generated by \( \Gamma \). Apply \fullref{thm:fol_lindenbaums_lemma} to the \hyperref[def:fol_henkin_completion]{Henkin completion} \( T^{*\omega} \) of \( T \) to obtain a theory \( T^+ \) that is consistent, complete and Henkin-complete.

  This allows us to construct a \hyperref[def:fol_term_model]{term model} \( \mscrT \) satisfying \( T^+ \).

  By construction, the universe of \( \mscrT \) is a quotient of the set \( C \) of all closed terms over \( \Sigma \). Its cardinality is thus at most that of \( C \).
  \begin{itemize}
    \item If \( \kappa \) is finite, by \cref{thm:def:formal_language/finite_alphabet_cardinality} there are countably many formulas over \( \Sigma \), so the Henkin extension \( \Sigma^* \) can add only \( \aleph_0 \) new constants. By induction, we conclude that \( \Sigma^{*k} \) is countably infinite for any positive integer \( k \), and \cref{thm:countably_infinite_union_of_countably_infinite_sets} implies that so is \( \Sigma^{*\omega} \).

    In this case, \cref{thm:def:formal_language/finite_alphabet_cardinality} implies that \( C \), as a language over \( \Sigma^{*\omega} \), is countably infinite.

    \item If \( \kappa \) is infinite, \cref{thm:def:formal_language/infinite_alphabet_cardinality} implies that each of the Henkin extensions \( \Sigma^{*k} \) have \( \kappa \) symbols, and hence so does \( \Sigma^{*\omega} \). Therefore, \( C \) also has cardinality \( \kappa \).
  \end{itemize}
\end{proof}

\begin{theorem}[First-order completeness]\label{thm:fol_completeness}\mcite[thm. 3.1.3]{VanDalen2004LogicAndStructure}
  The \hyperref[def:fol_natural_deduction]{classical first-order natural deduction system} is \hyperref[def:general_logic/completeness]{complete} with respect to \hyperref[def:fol_semantics]{first-order semantics}.
\end{theorem}
\begin{comments}
  \item This theorem is attributed to both Kurt G\"odel and to Leon Henkin; see \cref{rem:henkin_theories}.

  \item This is the first-order counterpart of \fullref{thm:classical_propositional_completeness}.
\end{comments}
\begin{proof}
  We can proceed similarly to \cref{thm:classical_propositional_completeness}, but we have to use \cref{thm:fol_consistent_implies_satisfiable} instead of \cref{thm:propositional_consistent_implies_satisfiable}.
\end{proof}

\paragraph{Compactness}

\begin{theorem}[First-order semantic compactness]\label{thm:fol_semantic_compactness}
  All of the following equivalent statements hold for the \hyperref[def:fol_semantics/entailment]{first-order entailment relation} over finite signatures:
  \begin{thmenum}
    \thmitem{thm:fol_propositional_semantic_compactness/relation} The relation is \hyperref[def:consequence_relation/compactness]{compact}: if \( \Gamma \vDash \varphi \), there exists a finite subset \( \Gamma_0 \) of \( \Gamma \) such that \( \Gamma_0 \vDash \varphi \).

    \thmitem{thm:classical_propositional_semantic_compactness/satisfiable} A set of sentences is \hyperref[def:fol_semantics/model]{satisfiable} (has a model) if and only if it is \hyperref[def:finitely_satisfiable_set_of_sentences]{finitely satisfiable} (every finite subset has a model).

    \thmitem{thm:classical_propositional_semantic_compactness/consistent} A set of sentences is \hyperref[def:fol_theory/consistent]{consistent} (does not entail \( \synbot \)) if and only if it is \hyperref[def:finitely_consistent_set_of_sentences]{finitely consistent} (no finite subset entails \( \synbot \)).
  \end{thmenum}
\end{theorem}
\begin{comments}
  \item \Cref{thm:propositional_semantic_inconsistency} gives several equivalent conditions for a set of closed formulas to be semantically consistent. Satisfiability is among them.

  \item This is one of several compactness theorems presented here --- see \cref{rem:logical_compactness_theorems}.
\end{comments}
\begin{proof}
  \Cref{thm:classical_propositional_semantic_compactness/relation} Follows from \fullref{thm:classical_propositional_semantic_compactness} via \cref{thm:completeness_implies_compactness}.

  The equivalence between the different conditions can be shown as in \fullref{thm:classical_propositional_semantic_compactness}.
\end{proof}
