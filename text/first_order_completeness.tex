\section{First-order completeness}\label{sec:first_order_completeness}

\begin{remark}\label{rem:henkin_theories}
   We will present a proof of \fullref{thm:fol_completeness} based on \bycite{Henkin1949FOLCompleteness}. The first proof of this theorem is attributed to Kurt G\"odel, thus it is also called \enquote{Go\"del's completeness theorem}.

   Our presentation follows the modern adaptation of \incite[\S 3.1]{VanDalen2004LogicAndStructure}. Other adaptations are given in \cite[\S 4.2]{Shoenfield1967MathematicalLogic}, \cite[\S 4.5]{ШеньВерещагин2017ЯзыкиИИсчисления} and \cite[\S 2.6.3]{Герасимов2014Вычислимость}, and similar constructions for proving \fullref{thm:fol_semantic_compactness} are given in \cite[ch. 3]{Hinman2005Logic}.

   The differences are subtle but important. For example, Gerasimov imposes a weaker condition than in \cref{def:fol_henkin_completeness} --- if \( \qexists x \varphi \) belongs to \( T \), there must exist a closed term \( \tau \) such that \( \varphi[x \mapsto \tau] \) belongs to \( T \). This requires modifying how \hyperref[def:fol_henkin_extension]{Henkin extensions} interact with other extensions, and requires also adapting \fullref{thm:fol_lindenbaums_lemma}.
\end{remark}

\paragraph{Henkin completion}

\begin{definition}\label{def:fol_henkin_completeness}\mcite[def. 3.1.4(iii)]{VanDalen2004LogicAndStructure}
  We say that the \hyperref[def:fol_theory]{first-order theory} \( T \) over \( \Sigma \) is \term[en=Henkin complete (theory) (\cite[def. 3.1.7]{Hinman2005Logic}); Henkin theory (\cite[def. 3.1.4(iii)]{VanDalen2004LogicAndStructure})]{Henkin complete} if, for every closed existential formula \( \qexists x \varphi \) over \( \Sigma \), there exists a constant \( c \) such that
  \begin{equation}\label{eq:def:fol_henkin_completeness/axiom}
    \qexists x \varphi \synimplies \varphi[x \mapsto c]
  \end{equation}
  belongs to \( T \). We call \( c \) a \term{witness} of \( \varphi \).
\end{definition}
\begin{comments}
  \item Note that we do not require neither \( \qexists x \varphi \) nor \( \varphi[x \mapsto c] \) to belong to \( T \).

  This allows us to define completions in \cref{thm:fol_henkin_completion_is_complete} and ensure that they are compatible with further extensions. This is one of several possible definitions of Henkin theories --- see \cref{rem:henkin_theories}.

  \item Since \( \qexists x \varphi \) is assumed to be closed, \( x \) is the only variable possibly free in \( \varphi \).
\end{comments}

\begin{definition}\label{def:fol_henkin_extension}\mcite[def. 3.1.6]{VanDalen2004LogicAndStructure}
  Consider some \hyperref[def:fol_theory]{first-order theory} \( T \) over the signature \( \Sigma \).

  Consider the \hyperref[def:fol_signature_extension]{extension} \( \Sigma^* \) of \( \Sigma \) where, for every existential formula \( \qexists x \varphi \) in \( T \), we add a new constant \( c_{x,\varphi} \). For the sake of determinism, suppose that \( c_{x,\varphi} \) is a symbol entirely determined by \( x \) and \( \varphi \).

  We define the (one-step) \term{Henkin extension} \( T^* \) as the consequence closure of
  \begin{equation}\label{eq:def:fol_henkin_extension}
    T \cup \set[\big]{ (\qexists x \varphi) \synimplies \varphi[x \mapsto c_{x,\varphi}] \given* \qexists x \varphi \T{belongs to} T }.
  \end{equation}
\end{definition}
\begin{comments}
  \item By iterating one-step Henkin extensions, we define Henkin completions in \cref{def:fol_henkin_completion}.

  \item Note that \( \qexists x \varphi \) is assumed to be closed, so \( x \) is the only variable possibly free in \( \varphi \).

  \item The determinism condition, although not present in \cite[def. 3.1.6]{VanDalen2004LogicAndStructure}, is important for some intermediate results like \cref{thm:def:fol_henkin_extension/scott_continuous}.
\end{comments}

\begin{theorem}[Theorem on constants]\label{thm:theorem_on_constants}\mcite[33]{Shoenfield1967MathematicalLogic}
  Over some \hyperref[def:fol_signature]{signature} \( \Sigma \), fix some set \( T \) of \hyperref[def:closed_fol_formula]{closed formulas} and a formula \( \varphi \) with free variables \( x_1, \ldots, x_n \).

  Consider the \hyperref[def:fol_signature_extension]{extension} \( \Sigma^+ \) of \( \Sigma \) with some constants \( c_1, \ldots, c_n \).

  Then
  \begin{equation*}
    T \vdash_\Sigma \qforall {x_1} \ldots \qforall {x_n} \varphi \T{if and only if} T \vdash_{\Sigma^+} \varphi[x_1 \mapsto c_1, \ldots, x_n \mapsto c_n].
  \end{equation*}
\end{theorem}
\begin{comments}
  \item The name of this theorem is based on \cite{Shoenfield1967MathematicalLogic}. It is useful for proving properties of \hyperref[def:fol_henkin_extension]{Henkin extensions}.
\end{comments}
\begin{proof}
  \SufficiencySubProof Let \( P \) be a proof tree over \( \Sigma \), deriving \( \qforall {x_1} \ldots \qforall {x_n} \varphi \) from \( T \). Let \( P' \) be the following extension:
  \begin{equation*}
    \begin{prooftree}
      \hypo{}
      \ellipsis {\( P \)} { \qforall {x_1} \qforall {x_2} \ldots \qforall {x_n} \varphi }
      \infer1[\ref{inf:def:fol_natural_deduction/forall/elim}]{ \qforall {x_2} \ldots \qforall {x_n} \varphi[x_1 \mapsto c_1] }
      \infer1[\ref{inf:def:fol_natural_deduction/forall/elim}]{ }
      \ellipsis {} {}
      \infer1[\ref{inf:def:fol_natural_deduction/forall/elim}]{ \varphi[x_1 \mapsto c_1] \ldots [x_n \mapsto c_n] }
    \end{prooftree}
  \end{equation*}

  Since \( c_k \) has no free variables, \cref{thm:alg:fol_formula_substitution/composition} implies that
  \begin{equation*}
    \varphi[x_1 \mapsto c_1, \ldots, x_n \mapsto c_n] = \varphi[x_1 \mapsto c_1] \ldots [x_n \mapsto c_n].
  \end{equation*}

  Therefore, \( P' \) derives \( \varphi[x_1 \mapsto c_1, \ldots, x_n \mapsto c_n] \) from \( T \) over \( \Sigma^+ \).

  \NecessitySubProof Let \( P \) be a proof tree over \( \Sigma^+ \), deriving \( \varphi[x_1 \mapsto c_1, \ldots, x_n \mapsto c_n] \) from \( T \).

  Let \( y_1, \ldots, y_n \) be distinct variables not present in \( P \), and let \( P' \) be the tree obtained from \( P \) by replacing \( c_k \) with \( y_k \). All rule applications remain, so \( P' \) is a well-formed proof tree over \( \Sigma \).

  Since \( y_1, \ldots, y_n \) are not present in \( T \), they can be used as eigenvariables; Thus, \( P' \) derives \( \varphi[x_1 \mapsto y_1, \ldots, x_n \mapsto y_n] \) from \( T \) over \( \Sigma \).

  \Cref{thm:alg:fol_formula_substitution/composition} implies that
  \begin{equation*}
    \varphi[x_1 \mapsto y_1, \ldots, x_n \mapsto y_n] = \varphi[x_1 \mapsto y_1] \ldots [x_n \mapsto y_n],
  \end{equation*}
  thus we can build the following tree:
  \begin{equation*}
    \begin{prooftree}
      \hypo{}
      \ellipsis {\( P' \)} { \varphi[x_1 \mapsto y_1] \ldots [x_{n-1} \mapsto y_{n-1}][x_n \mapsto y_n] }
      \infer1[\ref{inf:def:fol_natural_deduction/forall/intro}]{ \qforall {x_n} \varphi[x_1 \mapsto y_1] \ldots [x_{n-1} \mapsto y_{n-1}] }
      \infer1[\ref{inf:def:fol_natural_deduction/forall/intro}]{ }
      \ellipsis {} {}
      \infer1[\ref{inf:def:fol_natural_deduction/forall/intro}]{ \qforall {x_1} \ldots \qforall {x_n} \varphi }
    \end{prooftree}
  \end{equation*}
\end{proof}

\begin{lemma}\label{thm:natural_deduction_pulling_existential_quantifier}
  For \hyperref[def:fol_natural_deduction]{classical first-order natural deduction}, the following rule is \hyperref[con:inference_rule_admissibility]{admissible}:
  \begin{equation*}\taglabel[\ensuremath{ \exists_{\uparrow} }]{inf:thm:natural_deduction_pulling_existential_quantifier}
    \begin{prooftree}
      \hypo{ \varphi \synimplies \qexists x \psi }
      \infer1[\ref{inf:thm:natural_deduction_pulling_existential_quantifier}]{ \qexists x (\varphi \synimplies \psi) }
    \end{prooftree}
  \end{equation*}
  where \( x \) is not free in \( \varphi \).
\end{lemma}
\begin{comments}
  \item The proof tree is based on \cite{MathSE:natural_deduction_moving_quantifiers}.
\end{comments}
\begin{proof}
  \begin{equation*}
    \begin{prooftree}
      \hypo{ [\synneg \qexists x (\varphi \synimplies \psi)]^t }

      \hypo{ [\synneg \qexists x (\varphi \synimplies \psi)]^t }

      \hypo{ [\varphi \synimplies \qexists x \psi]^u }
      \hypo{ [\varphi]^v }
      \infer2[\ref{inf:def:propositional_natural_deduction/imp/elim}]{ \qexists x \psi }

      \hypo{ [\psi]^w }
      \infer1[\ref{inf:def:propositional_natural_deduction/imp/intro}]{ \varphi \synimplies \psi }
      \infer1[\ref{inf:def:fol_natural_deduction/exists/intro}]{ \qexists x (\varphi \synimplies \psi) }

      \infer[left label={\( v \)}]2[\ref{inf:def:fol_natural_deduction/exists/elim}]{ \qexists x \psi }
      \infer2[\ref{inf:def:propositional_natural_deduction/neg/elim}]{ \synbot }
      \infer1[\ref{inf:def:propositional_natural_deduction/bot/raa}]{ \psi }

      \infer[left label={\( w \)}]1[\ref{inf:def:propositional_natural_deduction/imp/intro}]{ \varphi \synimplies \psi }
      \infer1[\ref{inf:def:fol_natural_deduction/exists/intro}]{ \qexists x (\varphi \synimplies \psi) }

      \infer2[\ref{inf:def:propositional_natural_deduction/neg/elim}]{ \synbot }
      \infer[left label={\( t \)}]1[\ref{inf:def:propositional_natural_deduction/bot/raa}]{ \qexists x (\varphi \synimplies \psi) }
    \end{prooftree}
  \end{equation*}
\end{proof}

\begin{proposition}\label{thm:fol_henkin_extension_is_conservative}\mcite[lemma 3.1.7]{VanDalen2004LogicAndStructure}
  The \hyperref[def:fol_henkin_extension]{Henkin extension} of a natural deduction theory is \hyperref[def:fol_theory/conservative]{conservative}.
\end{proposition}
\begin{proof}
  Let \( T \) be a theory over \( \Sigma \). Let \( T^* \) be the Henkin extension of \( T \) with signature \( \Sigma^+ \). Denote by \( \Gamma \) the set of additional axioms from \( T^* \), so that \( T^* = \op*{Th}(T \cup \Gamma) \).

  Fix a formula \( \varphi \) over \( \Sigma \) that belongs to \( T^* \). Since \( T^* \) is a consequence closure of \( T \cup \Gamma \), there exists a \hyperref[def:fol_natural_deduction_proof_tree]{proof tree} \( P \) for \( \varphi \) whose open assumptions are from \( T \cup \Gamma \).

  Let \( \Gamma_0 \) be the subset of formulas of \( \Gamma \) that are open assumptions in \( P \). We will recursively build a sequence \( \ldots \subseteq \Gamma_2 \subseteq \Gamma_1 \subseteq \Gamma_0 \) that eventually stabilizes at the empty set, and a corresponding sequence \( P_1, P_2, \ldots \) of proof trees, where \( P_k \) derives \( \varphi \) from \( T \cup \Gamma_k \) and does not contain new constants except those from \( \Gamma_k \).

  We start with \( \Gamma_0 \) and \( P_0 \coloneqq P \). At step \( k + 1 \), suppose we have already constructed \( \Gamma_k \) and \( P_k \). If \( \Gamma_k \) is empty, we are done with the proof. Otherwise, fix an axiom \( (\qexists x \psi) \synimplies \psi[x \mapsto c_{x,\psi}] \) from \( \Gamma_k \) and define \( \Gamma_{k+1} \) by removing it from \( \Gamma_k \).

  \Fullref{thm:fol_natural_deduction_deduction_theorem} gives us a proof tree \( P_k' \) deriving
  \begin{equation*}
    ((\qexists x \psi) \synimplies \psi[x \mapsto c_{x,\psi}]) \synimplies \varphi
  \end{equation*}
  from \( T \cup \Gamma_{k+1} \), and \fullref{thm:theorem_on_constants} gives us a tree \( P_k^\dprime \) not containing \( c_{x,\psi} \) and deriving
  \begin{equation*}
    \qforall y \parens[\big]{ ((\qexists x \psi) \synimplies \psi[x \mapsto y]) \synimplies \varphi }
  \end{equation*}
  from \( T \cup \Gamma_{k+1} \).

  Finally, let \( P_{k+1} \) be the following tree:
  \footnotesize
  \begin{equation*}
    \begin{prooftree}[separation=1em]
      \hypo{ [\qexists x \psi]^u }

      \hypo{ [\psi[x \mapsto y]]^v }
      \infer1[\ref{inf:def:fol_natural_deduction/exists/intro}]{ \qexists y \psi[x \mapsto y] }

      \infer[left label={\( v \)}]2[\ref{inf:def:fol_natural_deduction/exists/elim}]{ \qexists y \psi[x \mapsto y] }

      \infer[left label={\( u \)}]1[\ref{inf:def:propositional_natural_deduction/imp/intro}]{ (\qexists x \psi) \synimplies (\qexists y \psi[x \mapsto y]) }
      \infer1[\ref{inf:thm:natural_deduction_pulling_existential_quantifier}]{ \qexists y ((\qexists x \psi) \synimplies \psi[x \mapsto y]) }

      \hypo{}
      \ellipsis {\( P_k^\dprime \)} { \qforall y \parens[\big]{ ((\qexists x \psi) \synimplies \psi[x \mapsto y]) \synimplies \varphi } }
      \infer1[\ref{inf:def:fol_natural_deduction/forall/elim}]{ ((\qexists x \psi) \synimplies \psi[x \mapsto y]) \synimplies \varphi }

      \hypo{ [(\qexists x \psi) \synimplies \psi[x \mapsto y]]^w }
      \infer2[\ref{inf:def:propositional_natural_deduction/imp/elim}]{ \varphi }
      \infer[left label={\( w \)}]2[\ref{inf:def:fol_natural_deduction/exists/elim}]{ \varphi }
    \end{prooftree}
  \end{equation*}
  \normalsize

  Here \( P_{k+1} \) derives \( \varphi \) from \( T \cup \Gamma_{k+1} \), as desired.
\end{proof}

\begin{lemma}\label{thm:upward_union_of_theories}
  With respect to a \hyperref[def:consequence_relation/compactness]{compact} \hyperref[def:consequence_relation]{consequence relation}, the union of an \hyperref[def:directed_set]{upward-directed family} of \hyperref[def:fol_theory]{first-order theories} is again a theory.
\end{lemma}
\begin{comments}
  \item We take the union of signature symbols also.
\end{comments}
\begin{proof}
  Let \( \seq{ T_k }_{k \in \mscrK} \) be an upward-directed family of theories and let \( T \) be their union.

  If \( T \vdash \varphi \), by compactness there exists a finite subset \( T_0 \) of \( T \) such that \( T_0 \vdash \varphi \).

  For every formula \( \psi \) in \( T \), let \( T_{k_\psi} \) be the smallest theory to which \( \psi \) belongs, and let \( k_0 > \max\set{ k_\psi \given \psi \in T_0 } \). Then \( T_0 \) is a subset of \( T_{k_0} \).

  Since \( \varphi \) is a consequence of \( T_0 \), and hence of \( T_{k_0} \), it must belong to the latter, and hence to \( T \).

  Therefore, \( T \) is also closed under consequence, i.e. it is a theory.
\end{proof}

\begin{definition}\label{def:fol_henkin_completion}\mimprovised
  With respect to some \hyperref[def:consequence_relation/compactness]{compact} \hyperref[def:consequence_relation]{consequence relation}, consider some \hyperref[def:fol_theory]{first-order theory} \( T \) over \( \Sigma \) as follows.

  Define the \( \mu \)-indexed \hyperref[def:transfinite_sequence]{transfinite sequence} of \hyperref[def:fol_henkin_extension]{Henkin extensions}
  \begin{equation*}
    T^{*\alpha} \coloneqq \begin{cases}
      T,                                &\alpha = 0, \\
      (T^{*\beta})^*,                   &\alpha = \op{sc}(\beta), \\
      \bigcup_{\beta < \alpha} T^\beta, &\alpha \T{is a limit ordinal.}
    \end{cases}
  \end{equation*}

  For the signatures, we define an analogous sequence \( \seq{ \Sigma^{*\alpha} }_{\alpha < \mu} \).

  As we will show in \cref{thm:fol_henkin_completion_is_complete}, this sequence stabilizes at \( \alpha = \omega \). We thus call \( T^{*\omega} \) the \term{Henkin completion} of \( T \).
\end{definition}
\begin{comments}
  \item We extract the constructions from \cite[lemma 3.1.8]{VanDalen2004LogicAndStructure} and \cite[47]{Shoenfield1967MathematicalLogic} and take care of some formal details.

  \item As the name suggests, the completion is \hyperref[def:fol_henkin_completeness]{Henkin complete} --- see \cref{thm:fol_henkin_completion_is_complete}.
\end{comments}
\begin{defproof}
  \Cref{thm:upward_union_of_theories} shows correctness of the limit step.
\end{defproof}

\begin{lemma}\label{thm:fol_henkin_extensions_scott_continuous}
  As an operator on \hyperref[def:fol_theory]{theories} over \hyperref[def:consequence_relation/compactness]{compact} \hyperref[def:consequence_relation]{consequence relations}, the \hyperref[def:fol_henkin_extension]{Henkin extension} operator \( (\anon)^* \) is \hyperref[def:scott_continuity]{Scott-continuous}.
\end{lemma}
\begin{proof}
  Let \( \seq{ T_k }_{k \in \mscrK} \) be an upward-directed family of theories and let \( T \) be their union (we take the union of signature symbols also). \Cref{thm:upward_union_of_theories} shows that \( T \) is also a theory.

  We must show that
  \begin{equation*}
    T^* = \bigcup_{k \in \mscrK} T_k^*.
  \end{equation*}

  Indeed, let \( \psi = (\qexists x \varphi) \synimplies \varphi[x \mapsto c_{x,\varphi}] \) be a new axiom introduced in \( T^* \). Then there exists some index \( T_{k_0} \) to which \( \qexists x \varphi \) belongs. Then \( \psi \) is also present in \( T_{k_0}^* \). It follows that
  \begin{equation*}
    T^* \subseteq \bigcup_{k \in \mscrK} T_k^*.
  \end{equation*}

  Conversely, let \( \psi = (\qexists x \varphi) \synimplies \varphi[x \mapsto c_{x,\varphi}] \) be a new axiom introduced in some \( T_{k_0}^* \) for some index \( k_0 \). Then \( \qexists x \varphi \) is in \( T_{k_0} \), so \( \psi \) belongs to \( T^* \).
\end{proof}

\begin{proposition}\label{thm:fol_henkin_completion_is_complete}
  The \hyperref[def:fol_henkin_completion]{Henkin completion} of a natural deduction theory is \hyperref[def:fol_theory/conservative]{conservative}, \hyperref[def:fol_henkin_completeness]{Henkin complete} and invariant under \hyperref[def:fol_henkin_extension]{Henkin extensions}.
\end{proposition}
\begin{proof}
  Consider the theory \( T \) over the signature \( \Sigma \).

  \SubProof{Proof that \( T^{*\omega} \) is conservative} If \( \varphi \) is a formula over \( \Sigma \) that belongs to \( T^{*\omega} \), it also belongs to some finite \( T^{*k_0} \). Using \cref{thm:def:fol_henkin_extension/conservative}, by induction on \( k_0 \) we can show that \( T^{*k_0} \) is conservative over \( T \). Therefore, \( \varphi \) belongs to \( T \).

  \SubProof{Proof that \( T^{*\omega} \) is a fixed point} \Cref{thm:upward_union_of_theories} implies that the completion \( T^{*\omega} \) is a theory. Since the Henkin extension operator is \hyperref[def:scott_continuity]{Scott-continuous} by \cref{thm:fol_henkin_extensions_scott_continuous}, \fullref{thm:knaster_tarski_iteration/continuous} implies that \( T^{*\omega} \) is a fixed point.

  \SubProof{Proof that \( T^{*\omega} \) is a Henkin complete} Every existential formula \( \qexists x \varphi \) over \( \Sigma^{*\omega} \) is a formula over some signature \( \Sigma^{*k_0} \) of finite index. In this case, \( T^{*k_0} \), and hence also \( T^{*\omega} \), contains the formula
  \begin{equation*}
    (\qexists x \varphi) \synimplies \varphi[x \mapsto c_{x,\varphi}].
  \end{equation*}

  It follows that \( c_{x,\varphi} \) is a witness of \( \qexists x \varphi \) in \( T^{*\omega} \). Generalizing, we conclude that \( T^{*\omega} \) is Henkin complete.
\end{proof}

\paragraph{Completeness}

\begin{theorem}[First-order completeness]\label{thm:fol_completeness}\mcite[thm. 3.1.3]{VanDalen2004LogicAndStructure}
  The \hyperref[def:fol_natural_deduction]{classical first-order natural deduction system} is \hyperref[def:general_logic/completeness]{complete} with respect to \hyperref[def:fol_semantics]{first-order semantics}.
\end{theorem}
\begin{comments}
  \item This theorem is attributed to both Kurt G\"odel and to Leon Henkin; see \cref{rem:henkin_theories}.
\end{comments}
