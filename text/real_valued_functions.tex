\section{Real-valued functions}\label{sec:real_valued_functions}

\paragraph{Definiteness}

\begin{definition}\label{def:real_function_definiteness}\mimprovised
  We say that a real-valued function \( f: X \to \BbbR \) on an \hyperref[con:additive_semigroup]{additive semigroup} \( X \) is \term[bg=положително положително дефинитна (квадратична форма) (\cite[86]{Обрешков1962ВисшаАлгебра}), ru=положительно определённая (квадратичная форма) (\cite[def. 1.4.7]{Кострикин2000АлгебраЧасть2}), en=positive definite (function) (\cite[17; 271]{Clarke2013OptimalControl})]{positive definite} if \( f(0_X) = 0 \) and \( f(x) > 0 \) for all nonzero \( x \).

  If we allow \( f(x) = 0 \) for nonzero \( x \), we instead say that \( f \) is \term[ru=положительно полуопрделённая (квадратичная форма) (\cite[def. 1.4.7]{Кострикин2000АлгебраЧасть2}), en=positive semidefinite (bilinear form) (\cite[107]{Knapp2016BasicAlgebra})]{positive semidefinite}.

  If the inequality is reversed, i.e. if \( f(x) < 0 \), we instead say that the function is \term[bg=отрицателно дефинитна (квадратична форма) (\cite[86]{Обрешков1962ВисшаАлгебра}), ru=отрицательно определённая (квадратичная форма) (\cite[def. 1.4.7]{Кострикин2000АлгебраЧасть2}), en=negative (semi)definite (bilinear form) (\cite[107]{Knapp2016BasicAlgebra})]{negative (semi)definite}.
\end{definition}

\begin{remark}\label{rem:real_function_definiteness_terminology}
  The terminology regarding \hyperref[def:real_function_definiteness]{function definiteness} is given in very different levels of generality across the literature:
  \begin{itemize}
    \item \incite[17; 271]{Clarke2013OptimalControl} gives inline definitions for positive definite real-valued functions (as is our case).

    \item \incite[def. 1.4.7]{Кострикин2000АлгебраЧасть2}, \incite[152]{Фаддеев1984Алгебра} and \incite[208]{Treil2017LinearAlgebra} gives an explicit definition for positive and negative (semi)definite finite dimensional \hyperref[thm:quadratic_forms]{quadratic forms}. Kostrikin require the forms to be nondegenerate.

    \item \incite[217]{Винберг2014Алгебра} and \incite[86]{Обрешков1962ВисшаАлгебра} give an explicit definition for both positive definite and negative definite real-valued quadratic forms.

    \item \incite[\S 25.2]{Тыртышников2007ЛинейнаяАлгебра} gives an explicit definition for positive definite complex-valued matrices. Later in, \cite[\S 33.6]{Тыртышников2007ЛинейнаяАлгебра}, he also gives a definition for positive semidefinite (complex-valued) matrices.

    \item \incite[107]{Knapp2016BasicAlgebra} gives explicit definitions for both positive and negative (semi)definite self-adjoint linear operators.

    \item \incite[374]{FriedbergInselSpence2018LinearAlgebra} give explicit definitions for positive definite and semidefinite self-adjoint linear operators and matrices.

    \item \incite[578]{Lang2002Algebra} gives an explicit definition for both positive definite and negative definite finite-dimensional symmetric bilinear forms.

    \item \incite[exerc. 6.4.1]{Jacobson1985AlgebraI} and \incite[def. 8.1.1]{Berger1987GeometryI} give inline definitions for positive definite symmetric bilinear forms. \incite*[378]{Berger1987GeometryI} also gives an inline definition for positive definite quadratic forms.

    \item \incite[230]{Roman2005LinearAlgebra} give an explicit definition for \enquote{positive} (what we call positive semidefinite) positive definite self-adjoint linear operators.

    \item \incite[295]{Тагамлицки1971Диф} gives an explicit definition for positive definite two-dimensional quadratic forms.

    \item \incite[def. 6.5.1]{Savage2008Computability} gives an explicit definition for positive definite real-valued symmetric \hyperref[def:array/matrix]{matrices}.

    \item \incite[246]{Strang2023LinearAlgebra} gives an explicit definition for positive semidefinite and semidefinite real-valued symmetric matrices in terms of their eigenvalues.

    \item \incite[59]{DontchevRockafellar2014SolutionMappings} give an inline definition for \hyperref[def:affine_operator]{affine maps}.
  \end{itemize}
\end{remark}

\paragraph{Homogeneous functions}

\begin{definition}\label{def:real_homogeneous_function}\mimprovised
  For real-valued functions we can extend the definition of homogeneous function from \fullref{def:homogeneous_function}. Fix \hyperref[def:topological_vector_space]{topological vector spaces} \( X \) and \( Y \).

  We say that a function \( f: X \to Y \) is \term{homogeneous} of degree \( \alpha > 0 \) if, for every scalar \( r \) and ever vector \( x \), we have
  \begin{equation}\label{eq:def:real_homogeneous_function}
    f(rx) = r^\alpha \cdot f(x).
  \end{equation}

  Compared to \fullref{def:homogeneous_function}, we allow the degree \( \alpha \) to be an arbitrary positive real number rather than only a positive integer.

  \begin{thmenum}
    \thmitem{def:real_homogeneous_function/positive}\mcite[rem. 2.10]{HugWeil2020ConvexGeometry} We say that \( f: X \to \BbbR \) is \term{positive homogeneous} (of degree \( d \)) if \eqref{eq:def:homogeneous_function} holds for \( r \geq 0 \) (but not necessarily when \( r < 0 \)).

    \thmitem{def:real_homogeneous_function/absolute} We say that \( f: X \to \BbbR \) is \term{absolutely homogeneous} (of degree \( d \)) if
    \begin{equation}\label{eq:def:real_homogeneous_function/absolute}
      f(rx) = \abs{r}^d \cdot f(x).
    \end{equation}
  \end{thmenum}
\end{definition}
\begin{comments}
  \item We generalize this definition from \incite[def. 2.3.1]{HillePhillips1996FunctionalAnalysis}, who define homogeneous functions between \hyperref[def:topological_vector_space]{topological vector spaces} without degrees, and \incite[416]{Зорич2019АнализЧасть1}, who defines homogeneous real-valued functions on \hyperref[def:euclidean_space]{Euclidean spaces} with positive real-valued degrees.
\end{comments}

\begin{remark}\label{rem:positive_homogeneous_function}
  Terminology regarding \hyperref[def:real_homogeneous_function/positive]{positive} and \hyperref[def:real_homogeneous_function/absolute]{absolutely} homogeneous functions differs across the literature.
  \begin{itemize}
    \item \incite[rem. 2.10]{HugWeil2020ConvexGeometry} use \enquote{positive homogeneous} like us, but restrict themselves to Euclidean spaces.

    \item \incite[def. 7.12.2]{HillePhillips1996FunctionalAnalysis} generalize positive homogeneous functions to cones rather than the interval \( [0, \infty) \), but only for \hyperref[def:additive_function/sub]{subadditive functions}.

    \item \incite[95]{DontchevRockafellar2014SolutionMappings} use \enquote{positive homogeneity} for functions between Euclidean spaces in the case \( r > 0 \).

    \item \incite[3]{Clarke2013OptimalControl} uses \enquote{positive homogeneity} for what we call \enquote{absolute homogeneity}.

    \item \incite[416]{Зорич2019АнализЧасть1} uses \enquote{положительно однородная функция} (\enquote{positive homogeneous function}) for what we call \enquote{absolutely homogeneous}.

    \item \incite[80]{КанторовичАкилов1984ФункАнализ} uses \enquote{однородная функция} (\enquote{homogeneous function}) for what we call \enquote{absolutely homogeneous}.

    \item \incite[12]{МагарилИльяевТихомиров2002ВыпуклыйАнализ} use \enquote{однородная функция} (\enquote{homogeneous function}) for what we call \enquote{positively homogeneous}.
  \end{itemize}
\end{remark}

\paragraph{...}

\begin{definition}\label{def:functions_vanish_nowhere}
  Let \( \mathcal{F} \) be a family of functions from a set \( S \) to a ring \( R \). We say that \( \mathcal{F} \) \term{vanishes nowhere} if for every \( x \in S \) there exists a function \( f \in \mathcal{F} \) such that \( f(x) \neq 0_R \).
\end{definition}

\begin{definition}\label{def:epigraph}
  Let \( X \) be an arbitrary set. The \term{epigraph} of the function \( f: X \to \BbbR \) is defined as
  \begin{equation*}
    \epi f \coloneqq \{ (x, r) \in X \times \BbbR \colon r \geq f(x) \},
  \end{equation*}
\end{definition}

\begin{definition}\label{def:vector_field}
  \todo{Define vector fields}
\end{definition}
