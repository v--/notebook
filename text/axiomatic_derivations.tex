\subsection{Axiomatic derivations}\label{subsec:axiomatic_derivations}

\paragraph{Axiomatic deduction systems}

\begin{definition}\label{def:propositional_formula_schema}\mimprovised
  It is sometimes convenient to refer to all \hyperref[def:propositional_syntax/formula]{propositional formulas} having the same abstract structure, as we have done in \fullref{thm:intuitionistic_tautologies} and \fullref{thm:classical_tautologies}. To refer to the abstract structure itself, we will use the following extension of the \hyperref[def:propositional_syntax]{propositional grammar schema}:
  \begin{bnf*}
    \bnfprod{atomic schema}  {\bnfpn{Greek identifier}}, \\
    \bnfprod{schema}         {\bnftsq{\( \syntop \)} \bnfor \bnftsq{\( \synbot \)} \bnfor} \\
    \bnfmore                      {\bnfpn{atomic schema} \bnfor} \\
    \bnfmore                      {\bnftsq{\( \synneg \)} \bnfsp \bnfpn{schema} \bnfor} \\
    \bnfmore                      {\bnftsq{(} \bnfsp \bnfpn{schema} \bnfsp \bnfpn{connective} \bnfsp \bnfpn{schema} \bnfsp \bnftsq{)}}
  \end{bnf*}
\end{definition}
\begin{comments}
  \item In the metalanguage we denote formulas via small Greek letters like \( \varphi \) and \( \psi \) (see \fullref{rem:mathematical_logic_conventions/greek_alphabet}), and we will use those same letters for denoting atomic schemas in the object language. For denoting schemas in the metalanguage, we will instead use the capital counterparts of those Greek letters, for example \( \Phi \) and \( \Psi \) rather than \( \varphi \) and \( \psi \).
  \item Schemas merely provide an abstraction for formula schemas from the metalanguage. The notation \( \synneg \synneg \varphi \synimplies \varphi \) can now have different meanings --- it may refer to both a formula schema where \( \varphi \) and \( \psi \) are yet to be given meaning, and it may refer to a concrete instance of \eqref{eq:thm:classical_tautologies/efq}.
  \item The grammar of schemas is unambiguous. Indeed, it almost coincides with the grammar for formulas, which is unambiguous as a consequence of \fullref{thm:propositional_formulas_are_unambiguous}.
  \item Schemas provide little benefit over formulas in propositional logic, but become useful in first-order logic --- see \fullref{def:first_order_formula_schema} --- because they help abstract away the concrete signature.
  \item In the metalanguage, we will denote schemas via capital Greek letters to distinguish them from formulas.
\end{comments}

\begin{definition}\label{def:uniform_schema_substitution}\mimprovised
  A \term{uniform schema substitution} is a map \( \sigma \) between \hyperref[def:formula_schema]{formula schemas} such that
  \begin{align*}
    \sigma(\neg \Phi) = \neg \sigma(\Phi) &&\T{and}&& \sigma(\Phi \syncirc \Psi) = \sigma(\Phi) \syncirc \sigma(\Psi).
  \end{align*}

  We call \( \sigma(\Phi) \) an \term{instance} of the schema \( \Phi \).
\end{definition}
\begin{comments}
  \item This definition adapted from \fullref{def:uniform_propositional_substitution}, which is in turn based on \cite[def. 3.1.3]{CitkinMuravitsky2021}.
  \item We use the notation from \fullref{rem:uniform_substitution_notation}.
\end{comments}

\begin{definition}\label{def:axiomatic_derivation_system}\mimprovised
  A \term[en=axiomatic system (\cite[6D1]{Hindley1997})]{propositional axiomatic derivation system}, also called a \term[ru=исчисление высказываний гильбертовского типа (\cite[35]{Герасимов2011}), en=Hilbert-type system (\cite[80]{Smullyan1995}); Hilbert-style system (\cite[6D1]{Hindley1997}); Hilbert system (\cite[33]{TroelstraSchwichtenberg2000}); Hilbert calculus (\cite[103]{Mimram2020})]{Hilbert-style system}, consists of a set of \hyperref[def:propositional_formula_schema]{propositional schemas}, whose elements we call \term[en=axiom schema (\cite[80]{Smullyan1995})]{logical axiom schemas}.

  For any schema \( \Phi \) and any \hyperref[def:uniform_schema_substitution]{substitution} \( \sigma \), we call the formula \( \Phi[\sigma] \) a \term{logical axiom}.
\end{definition}
\begin{comments}
  \item Our definition resembles those of \incite[80]{Smullyan1995} and \incite[6D1]{Hindley1997}, but with some important differences:
  \begin{itemize}
    \item We allow a distinction between formulas and sentences.
    \item We avoid any restrictions on formulas and sentences, while Hindley requires them to be \hyperref[def:propositional_tautology]{propositional tautologies}.
    \item We avoid adding any rules except for \eqref{eq:def:def:axiomatic_derivation_system/mp}.
  \end{itemize}

  \item The term \enquote{logical axiom} may have different meaning depending on the author. \incite[6D1]{Hindley1997} uses the term \enquote{axiom} for our notion of \enquote{logical axiom}. \incite[204]{Hinman2005} gives an ad-hoc definition of derivability from a set of \enquote{logical axioms}. \incite{TroelstraSchwichtenberg2000} uses both \enquote{logical axiom} and \enquote{nonlogical axiom} but does not define the terms, while \incite{Kleene2002Logic} only uses the latter term, again without defining it.

  \item An alternative approach is to encode the logical axioms as inference rules without premises, as it is done in \cite[103]{Mimram2020}.
\end{comments}

\begin{definition}\label{def:axiomatic_derivation}\mcite[51]{TroelstraSchwichtenberg2000}
  Fix an \hyperref[def:axiomatic_derivation_system]{axiomatic derivation system}.

  An \term[ru=вывод (\cite[35]{Герасимов2011})]{axiomatic derivation} of the single formula \( \varphi \) from the set \( \Gamma \) of formulas is a nonempty list \( \psi_1, \ldots, \psi_n \) of formulas such that \( \psi_n = \varphi \) and, for every \( k = 1, \ldots, n \), any of the following hold:
  \begin{thmenum}
    \thmitem{def:axiomatic_derivation/axiom} \( \psi_k \) is a logical axiom.
    \thmitem{def:axiomatic_derivation/premise} \( \psi_k \) is a premise, that is, it belongs to \( \Gamma \).
    \thmitem{def:axiomatic_derivation/mp} There exist indices \( i \) and \( j \), both smaller than \( k \), such that \( \psi_j = (\psi_i \synimplies \psi_k) \).
  \end{thmenum}
\end{definition}
\begin{comments}
  \item \incite[51]{TroelstraSchwichtenberg2000} use the term \enquote{axiomatic deduction}. We prefer \enquote{derivation} to \enquote{deduction} based on an analogy with \hyperref[def:formal_grammar/derivation]{derivations} on \hyperref[def:formal_grammar]{formal grammars}. \incite[117]{CitkinMuravitsky2021} also prefer the term \enquote{derivation}.
  \item \Fullref{def:axiomatic_derivation/mp} can be expressed via the inference rule \eqref{eq:thm:implicational_natural_deduction/mp} rule, as will be discussed in \fullref{thm:implicational_natural_deduction/mp}.
\end{comments}

\begin{definition}\label{def:axiomatic_derivation_entailment}\mcite[205]{Hinman2005}
  For a \hyperref[def:axiomatic_derivation_system]{axiomatic derivation system}, we define a \hyperref[def:consequence_relation]{consequence relation} \( \Gamma \vdash \varphi \) that holds when there is an \hyperref[def:axiomatic_derivation]{axiomatic derivation} of \( \varphi \) from \( \Gamma \).
\end{definition}
\begin{defproof}
  We must show that \( {\vdash} \) satisfies the conditions for being a consequence relation.

  \SubProofOf[def:consequence_relation/reflexivity]{reflexivity} \Fullref{def:axiomatic_derivation/premise} ensures that \( \Gamma \vdash \varphi \) whenever \( \varphi \in \Gamma \).

  \SubProofOf[def:consequence_relation/monotonicity]{monotonicity} If \( \Gamma \vdash \varphi \), then there exists a derivation of \( \varphi \) from \( \Gamma \). Such a derivation also proves \( \varphi \) from any superset of \( \Gamma \).

  \SubProofOf[def:consequence_relation/transitivity]{transitivity} Suppose that \( \Delta, \Epsilon \vdash \varphi \) and that, for every \( \psi \in \Delta \), we have \( \Gamma, \Epsilon \vdash \psi \).

  Let \( \theta_1, \ldots, \theta_n \) be a derivation of \( \varphi \) from \( \Delta \cup \Epsilon \) and let \( \chi_{1,k}, \ldots, \chi_{m_k,k} \) be a derivation of \( \theta_k \) from \( \Gamma \cup \Epsilon \). Then the following is a derivation of \( \varphi \) from \( \Gamma \cup \Epsilon \):
  \begin{equation*}
    \chi_{1,1}, \ldots, \chi_{m_1,1}, \ldots, \chi_{1,n}, \ldots, \chi_{m_n,n}.
  \end{equation*}
\end{defproof}

\paragraph{Implicational logic}\hfill

Implicational logic is both a toy derivation system we will explore and a useful device for the \hyperref[con:curry_howard_correspondence]{Curry-Howard correspondence}.

\begin{definition}\label{def:implicational_propositional_fragment}\mcite[49]{Mimram2020}
  The \term{implicational fragment} of propositional logic is the \hyperref[def:propositional_syntax/fragment]{fragment} obtained via the following grammar (based on the \hyperref[def:propositional_syntax]{usual schema}):
  \begin{bnf*}
    \bnfprod{impl. formula} {\bnfpn{variable} \bnfor \bnftsq{(} \bnfsp \bnfpn{impl. formula} \bnfsp \bnftsq{\( \synimplies \)} \bnfsp \bnfpn{impl. formula} \bnfsp \bnftsq{)}}
  \end{bnf*}
\end{definition}

\begin{definition}\label{def:minimal_implication_logic}\mcite[sec. 1.3.9]{TroelstraSchwichtenberg2000}
  The \term{minimal implicational logic} is an extraordinarily simple \hyperref[def:axiomatic_derivation_system]{axiomatic derivation system} for the \hyperref[def:implicational_propositional_fragment]{implicational fragment} of propositional logic.

  The system has the following logical axiom schemas:
  \begin{thmenum}
    \thmitem{def:minimal_implication_logic/intro} Every formula is the consequent of an implication whose antecedent can be any other formula:
    \begin{equation}\label{eq:def:minimal_implication_logic/intro}
      \varphi \synimplies (\psi \synimplies \varphi) \tag{\( \rightarrow_A^+ \)}.
    \end{equation}

    \thmitem{def:minimal_implication_logic/dist} Implication distributes over itself:
    \begin{equation}\label{eq:def:minimal_implication_logic/dist}
      \parens[\Big]{ \varphi \synimplies (\psi \synimplies \theta) } \synimplies \parens[\Big]{ (\varphi \synimplies \psi) \synimplies (\varphi \synimplies \theta)} \tag{\( \twoheadrightarrow_A \)}.
    \end{equation}
  \end{thmenum}
\end{definition}
\begin{comments}
  \item The subscript \enquote{A} in \eqref{eq:def:minimal_implication_logic/intro} highlights that it is an axiom schema. This convention allows us to distinguish the axiom schema \eqref{eq:thm:intuitionistic_tautologies/efq} from the inference rule \eqref{eq:def:intuitionistic_propositional_deduction_systems/rules/efq}.
\end{comments}

\begin{example}\label{ex:minimal_implication_logic_identity/derivations}
  Fix any \hyperref[def:minimal_implication_logic]{implicational formula} \( \varphi \) and consider
  \begin{equation}\label{eq:ex:minimal_implication_logic_identity/derivations/final}
    \varphi \synimplies \varphi.
  \end{equation}

  We will build an \hyperref[def:axiomatic_derivation]{axiomatic derivation} in \hyperref[def:minimal_implication_logic]{minimal implicational logic}. First, consider the following two instances of \eqref{eq:def:minimal_implication_logic/intro}:
  \begin{align}
    \varphi \synimplies (\varphi \synimplies \varphi), \label{eq:ex:minimal_implication_logic_identity/derivations/intro_1} \\
    \varphi \synimplies ((\varphi \synimplies \varphi) \synimplies \varphi). \label{eq:ex:minimal_implication_logic_identity/derivations/intro_2}
  \end{align}

  Our only intermediate step will be the following formula:
  \begin{equation}\label{eq:ex:minimal_implication_logic_identity/derivations/intermediate}
    (\varphi \synimplies (\varphi \synimplies \varphi)) \synimplies (\varphi \synimplies \varphi).
  \end{equation}

  Finally, we will need the following instance of \eqref{eq:def:minimal_implication_logic/dist}:
  \begin{equation}\label{eq:ex:minimal_implication_logic_identity/derivations/dist}
    \eqref{eq:ex:minimal_implication_logic_identity/derivations/intro_2} \synimplies \eqref{eq:ex:minimal_implication_logic_identity/derivations/intermediate}
  \end{equation}

  Our derivation is then
  \begin{equation*}
    \eqref{eq:ex:minimal_implication_logic_identity/derivations/intro_1}, \eqref{eq:ex:minimal_implication_logic_identity/derivations/intro_2}, \eqref{eq:ex:minimal_implication_logic_identity/derivations/intermediate}, \eqref{eq:ex:minimal_implication_logic_identity/derivations/dist}, \eqref{eq:ex:minimal_implication_logic_identity/derivations/final}.
  \end{equation*}

  Indeed, we can derive the intermediate step \eqref{eq:ex:minimal_implication_logic_identity/derivations/intermediate} via \fullref{def:axiomatic_derivation/mp} from \eqref{eq:ex:minimal_implication_logic_identity/derivations/dist} and \eqref{eq:ex:minimal_implication_logic_identity/derivations/intro_2} and the conclusion \eqref{eq:ex:minimal_implication_logic_identity/derivations/final} from the intermediate step \eqref{eq:ex:minimal_implication_logic_identity/derivations/intermediate} and \eqref{eq:ex:minimal_implication_logic_identity/derivations/intro_1}.
\end{example}
\begin{comments}
  \item We will present a more demonstrative proof in \eqref{ex:minimal_implication_logic_identity/tree}.
\end{comments}

\begin{algorithm}[Derivation conclusion hypothesis introduction]\label{alg:derivation_conclusion_hypothesis_introduction}
  We will present an algorithm that, given an \hyperref[def:axiomatic_derivation]{axiomatic derivation} of \( \varphi \) from \( \Gamma \cup \set{ \varphi } \), will provide a derivation of \( \varphi \synimplies \varphi \) from \( \Gamma \).

  We will refer to the formula \( \varphi \synimplies \varphi \) as our \term{goal}. We will use \hyperref[rem:natural_number_recursion]{natural number recursion} on the length of the derivation \( \theta_1, \ldots, \theta_n \), although we will only need this in \fullref{alg:derivation_conclusion_hypothesis_introduction/recursive}.

  \begin{thmenum}
    \thmitem{alg:derivation_conclusion_hypothesis_introduction/goal_premise} If our goal is either an axiom or a premise from \( \Gamma \), the desired derivation is simply the singleton list \( \varphi \synimplies \varphi \).

    \thmitem{alg:derivation_conclusion_hypothesis_introduction/identity} Otherwise, if \( \varphi = \psi \), then we must derive \( \varphi \synimplies \varphi \), which can be done via logical axioms alone as shown in \fullref{ex:minimal_implication_logic_identity/derivations}.

    \thmitem{alg:derivation_conclusion_hypothesis_introduction/conclusion_premise} Otherwise, if \( \psi \) is either an axiom or a premise from \( \Gamma \), then the following is a derivation of \( \varphi \synimplies \psi \):
    \begin{equation*}
      \underbrace{\psi \synimplies (\varphi \synimplies \psi)}_{\eqref{eq:def:minimal_implication_logic/intro}},
      \psi,
      \varphi \synimplies \psi.
    \end{equation*}

    \thmitem{alg:derivation_conclusion_hypothesis_introduction/recursive} Otherwise, there must exist indices \( i \) and \( j \) such that
    \begin{equation*}
      \theta_i = \theta_j \synimplies \theta_n.
    \end{equation*}

    Suppose that the inductive hypothesis holds for all formulas preceding \( \theta_n = \varphi \) in the derivation. Then we can obtain a derivation \( \chi_1, \ldots, \chi_m \) of \( \varphi \synimplies \theta_i \) and \( \omega_1, \ldots, \omega_k \) of \( \varphi \synimplies \theta_j \), and use them to build a derivation of our goal:
    \small
    \begin{equation*}
      \underbrace{\chi_1, \ldots, \chi_m}_{\varphi \synimplies \theta_i},
      \underbrace{(\varphi \synimplies \theta_i) \synimplies ((\varphi \synimplies \theta_j) \synimplies (\varphi \synimplies \theta_n))}_{\eqref{eq:def:minimal_implication_logic/intro}},
      \underbrace{((\varphi \synimplies \theta_j) \synimplies (\varphi \synimplies \theta_n))}_{\ref{def:axiomatic_derivation/mp}},
      \underbrace{\omega_1, \ldots, \omega_k}_{\varphi \synimplies \theta_j},
      \underbrace{\varphi \synimplies \theta_n}_{\ref{def:axiomatic_derivation/mp}}.
    \end{equation*}
    \normalsize
  \end{thmenum}
\end{algorithm}
\begin{comments}
  \item This algorithm can be found as \identifier{natural_deduction.derivation.introduce_conclusion_hypothesis} in \cite{code}.
\end{comments}

\begin{theorem}[Implicational syntactic deduction theorem]\label{thm:implicational_syntactic_deduction_theorem}
  With respect to the \hyperref[def:minimal_implication_logic]{minimal implicational logic}, for arbitrary propositional formulas, we have
  \begin{equation*}
    \Gamma, \varphi \vdash \psi \T{if and only if} \Gamma \vdash \varphi \synimplies \psi.
  \end{equation*}
\end{theorem}
\begin{comments}
  \item See \fullref{rem:deduction_theorem_list} for a list of similar theorems.
\end{comments}
\begin{proof}
  \SufficiencySubProof If \( \Gamma, \varphi \vdash \varphi \), \fullref{alg:derivation_conclusion_hypothesis_introduction} allows us to build a derivation of \( \varphi \synimplies \varphi \) from \( \Gamma \), and the existence of such a derivation implies \( \Gamma \vdash \varphi \).

  \NecessitySubProof If \( \Gamma \vdash \varphi \synimplies \psi \), we can add \( \varphi, \psi \) to any derivation to obtain a derivation of \( \psi \) from \( \Gamma \cup \set{ \varphi } \).
\end{proof}

\begin{theorem}[Minimal implicational logic soundness]\label{thm:minimal_implicational_logic_soundness}
  \hyperref[def:minimal_implication_logic]{Minimal implicational logic} is \hyperref[def:logical_framework/soundness]{sound} with respect to \hyperref[def:minimal_propositional_semantics]{minimal semantics}.
\end{theorem}
\begin{comments}
  \item By extension, it is also sound with respect to \hyperref[def:propositional_semantics]{intuitionistic and classical semantics}.
  \item See \fullref{rem:soundness_and_completeness_theorem_list} for a list of soundness and completeness theorems.
\end{comments}
\begin{proof}
  \SubProof{Proof for logical axioms} We must first prove that the logical axioms are \hyperref[def:propositional_tautology]{semantic tautologies}.

  \SubProof*{Proof for axiom schema \eqref{eq:def:minimal_implication_logic/intro}} Consider the formula \( \varphi = \psi \synimplies (\theta \synimplies \psi) \) and an arbitrary \hyperref[def:propositional_valuation]{propositional interpretation} \( I \).

  \Fullref{thm:def:heyting_algebra/leq_right} implies that
  \begin{equation*}
    \Bracks{\psi}_I \leq \parens[\Big]{ \Bracks{\theta}_I \rightarrow \Bracks{\psi}_I }
  \end{equation*}
  and \fullref{thm:def:heyting_algebra/leq_right} implies that
  \begin{equation*}
    \Bracks{\varphi}_I
    =
    \Bracks{\psi \synimplies (\theta \synimplies \psi)}_I
    =
    \Bracks{\psi}_I \rightarrow \parens[\Big]{ \Bracks{\theta}_I \rightarrow \Bracks{\psi}_I }.
    =
    T.
  \end{equation*}

  \SubProof*{Proof for axiom schema \eqref{eq:def:minimal_implication_logic/dist}} Consider
  \begin{equation*}
    \varphi = (\psi \rightarrow (\theta \rightarrow \chi)) \rightarrow ((\psi \rightarrow \theta) \rightarrow (\theta \rightarrow \chi)).
  \end{equation*}

  \Fullref{thm:def:heyting_algebra/dist} implies that
  \begin{equation*}
    \parens[\Big]{ \Bracks{\psi}_I \rightarrow \parens[\Big]{ \Bracks{\theta}_I \rightarrow \Bracks{\chi}_I } } \leq \parens[\Big]{ \parens[\Big]{ \Bracks{\psi}_I \rightarrow \Bracks{\theta}_I } \rightarrow \parens[\Big]{ \Bracks{\theta}_I \rightarrow \Bracks{\chi}_I } },
  \end{equation*}
  hence \( \Bracks{\varphi}_I = T \) follows via \fullref{thm:def:heyting_algebra/leq}

  \SubProof{Proof for premises} If \( \varphi \) is in \( \Gamma \), then \( \Gamma \vDash \varphi \) holds since \( \vDash \) is a \hyperref[def:consequence_relation]{consequence relation} and thus satisfies \fullref{def:consequence_relation/reflexivity}.

  \SubProof{Proof for general formulas} Now suppose that \( \Gamma \vdash \varphi \). Let \( I \) be an interpretation that satisfies every formula in \( \Gamma \). We will use induction on derivation length \( n \) to show that \( I \) satisfies \( \varphi \).

  \begin{itemize}
    \item In the base case \( n = 1 \), \( \varphi \) must either be a logical axiom or a premise from \( \Gamma \), and we have already shown that, in this case, \( \Gamma \vDash \varphi \).

    \item Suppose that \( \Gamma \vdash \varphi \) implies \( \Gamma \vDash \varphi \) when the derivation has length less than \( n \), and let \( \psi_1, \ldots, \psi_n = \varphi \) be a derivation of \( \varphi \) from \( \Gamma \).

    We must now perform case analysis on why \( \Gamma \vdash \varphi \).
    \begin{itemize}
      \item If \fullref{def:axiomatic_derivation/axiom} holds, then \( \varphi \) is a logical axiom, and if \fullref{def:axiomatic_derivation/premise} holds, then \( \varphi \) is a premise from \( \Gamma \), and we have already shown that \( \Gamma \vDash \varphi \) in these cases even without the assumption that \( \Gamma \vdash \varphi \).

      \item Otherwise, the case \fullref{def:axiomatic_derivation/mp} must hold, that is, there must exist indices \( i \) and \( j \), both strictly smaller than \( n \), such that \( \psi_j = (\psi_i \synimplies \psi_n) \).

      Then \( \psi_1, \ldots, \psi_i \) is a derivation of \( \psi_i \), and the inductive hypothesis implies that \( \Gamma \vDash \psi_i \). We can similarly conclude that \( \Gamma \vDash \psi_j \). Then
      \begin{equation*}
        \underbrace{ \Bracks{ \psi_j }_I }_T = \underbrace{ \Bracks{ \psi_i }_I }_T \rightarrow \Bracks{ \psi_n }_I.
      \end{equation*}

      Then \fullref{thm:def:heyting_algebra/leq} implies that \( T \leq \Bracks{ \psi_n }_I \), hence \( \Bracks{ \psi_n }_I = T \). By noting that \( \psi_n = \varphi \), we conclude that \( I \vDash \varphi \).

      Therefore, \( \Gamma \vDash \varphi \).
    \end{itemize}
  \end{itemize}
\end{proof}
