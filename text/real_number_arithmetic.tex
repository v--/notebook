\section{Real number arithmetic}\label{sec:real_number_arithmetic}

\paragraph{Irrational number arithmetic}

\begin{proposition}\label{thm:real_number_arithmetic}
  Arithmetic on \hyperref[def:real_numbers]{real numbers} has the following basic properties:
  \begin{thmenum}
    \thmitem{thm:real_number_arithmetic/rational_irrational_sum} If \( x \) is rational and \( y \) is irrational, then \( x + y \) is irrational.
  \end{thmenum}
\end{proposition}
\begin{proof}
  \SubProofOf{thm:real_number_arithmetic/rational_irrational_sum} If \( x + y \) is rational, then \( y = (x + y) + (-x) \) must also be rational, which contradicts the assumption that it is irrational.
\end{proof}

\paragraph{Extended real numbers}

\begin{remark}\label{rem:lemniscate_symbol}
  We use the \term{lemniscate} symbol \( \infty \) is used to denote \hyperref[con:transfinitum]{infinity} in its different manifestations. The symbol is attributed to John Wallis by \incite{MartinLöf1990Infinity}, where the latter also remarks that
  \begin{displayquote}
    \textellipsis we take great pains to explain that \( \infty \) makes no sense by itself, that is, is no detachable part of the notation for a limit, an infinite sum, product or the like \textellipsis
  \end{displayquote}

  \incite*[92]{Кантор1985Многообразия} argues that this usage refers to \hyperref[con:transfinitum]{potential infinity}, which explains why \( \infty \) has no well-defined meaning in general. For reasons of compatibility with existing notation, we use the symbol for the use cases Martin-L\"of described, for example
  \begin{equation*}
    \sum_{k=1}^\infty a_k.
  \end{equation*}

  As a meaningful standalone symbol, we will use \( \infty \) for positive and \( -\infty \) for negative infinity within the \hyperref[def:extended_real_numbers]{extended real numbers}. The latter numbers can be quite useful; see \cref{rem:adjoining_infinity}.
\end{remark}

\begin{definition}\label{def:extended_real_numbers}\mimprovised
  We are sometimes interested in \term[ru=расширенная вещественная прямая (\cite[23]{ИоффеТихомиров1974ЭкстремальныеЗадачи}), en=extended real number system (\cite[10]{Folland1999RealAnalysis})]{extended real numbers}, also called the \term[en=extended real line (\cite[\S 1.4]{Rudin1987RealAndComplexAnalysis})]{extended real line}.

  We obtain it by taking the \hyperref[def:lower_cut_completion]{\hi{unbounded} lower cut completion} of the \hyperref[def:rational_numbers]{rational numbers}. Due to \fullref{thm:unbounded_lower_cut_completion/dual_macneille}, we can equivalently take the \hyperref[def:dedekind_macnielle_completion]{Dedekind-MacNeille completion}, but we generally find it more convenient working with lower cuts.

  This extends \( \BbbR \) with a \hyperref[def:extremal_points/top_and_bottom]{bottom element} \( -\infty \) (corresponding to the empty subset of \( \BbbQ \)) and \hyperref[def:extremal_points/top_and_bottom]{top element} \( \infty \) (corresponding to \( \BbbQ \) itself). We call these elements \term{positive infinite} (or simply \term{infinity}) and \term{negative infinity}.

  We will extend the algebraic operations of \( \BbbR \) to some \hyperref[def:indeterminate_form]{indeterminate forms} based on \cite[\S 1.1.17]{Schechter1997AnalysisHandbook} and \cite[11]{Folland1999RealAnalysis}. We leave \( \infty - \infty \) undefined and take precautions as not to encounter this situation (see for example \cref{def:measure/positive}). The absorbing property of \( 0 \) when multiplied by either \( -\infty \) or \( \infty \) is controversial, however it is quite useful in \fullref{ch:measure_theory} and \fullref{ch:convex_analysis}. The rest of the conventions are motivated by \hyperref[rem:bourbaki]{Bourbaki}'s \cite[\S IV.4]{Bourbaki1995GeneralTopology1to4}.

  In the following, \( x \) is a (finite) positive real number and the operations are extended by commutativity:
  \begin{equation*}
    \begin{array}{c !{\qquad} *{5}{c}}
      \toprule
      \xi                 & -\infty       & -x      & 0      & x       & \infty        \\
      \midrule
      \xi - \infty        & -\infty       & -\infty & \infty & -\infty & \T{undefined} \\
      \xi + \infty        & \T{undefined} & \infty  & \infty & \infty  & \infty \\
      \xi \cdot (-\infty) & \infty        & \infty  & 0      & -\infty & -\infty \\
      \xi \cdot \infty    & -\infty       & -\infty & 0      & \infty  & \infty \\
    \end{array}
  \end{equation*}
\end{definition}
\begin{comments}
  \item See \cref{rem:lemniscate_symbol} for a general discussion of the lemniscate symbol \( \infty \) and \cref{con:transfinitum} for a discussion of infinity.
\end{comments}

\begin{remark}\label{rem:adjoining_infinity}
  Sometimes, \hyperref[def:natural_numbers]{natural} or \hyperref[def:real_numbers]{real numbers} are used, but the \hyperref[def:extended_real_numbers]{extended real number} \( \infty \) is added a possible value for convenience. In these cases \( \infty \) is useful because it greater than all conventional numbers and, furthermore, it is absorbing with respect to addition (i.e. \( x + \infty = \infty = \infty + x \)). We list some examples in \cref{ex:def:extended_real_numbers}.

  Rather than via a formal definition, \( \infty \) is often introduced as a complementary sentinel value satisfying the aforementioned properties. From this perspective, extended real numbers are not the only possibility of formally introducing \( \infty \):
  \begin{thmenum}
    \thmitem{rem:adjoining_infinity/aleph_0} The \hyperref[def:aleph_hierarchy]{smallest infinite cardinal} \( \aleph_0 \) is another value greater than all \hi{natural numbers} and is also absorbing with respect to addition, but it requires adjustments when working with more general real numbers.

    \thmitem{rem:adjoining_infinity/omega} The \hyperref[thm:omega_is_an_ordinal]{smallest infinite ordinal} \( \omega \) has the same downsides as \( \aleph_0 \), but has another disadvantage --- as shown in \cref{ex:ordinal_addition}, ordinal addition is not commutative, so \( 1 + \omega = \omega < \omega + 1 \).
  \end{thmenum}
\end{remark}

\begin{example}\label{ex:def:extended_real_numbers}
  We list examples where we use \hyperref[def:extended_real_numbers]{extended real numbers}:
  \begin{thmenum}
    \thmitem{ex:def:extended_real_numbers/positive_measure} Allowing \hyperref[def:measure/positive]{positive measures} to take infinite values allows us to avoid mundane checks of whether the measure is defined when working with sums.

    \thmitem{ex:def:extended_real_numbers/polynomial_degree} The \hyperref[def:polynomial_degree]{degree} of the zero polynomial is sometimes defined as \( -\infty \), even though we prefer to leave it undefined. See \cref{rem:zero_polynomial_degree}.

    \thmitem{ex:def:extended_real_numbers/walk_length} We allow the length of a \hyperref[def:graph_walk]{graph walk} to be infinite and denote this by \( \infty \).

    Since the walk is a defined as sequence, the \hyperref[thm:well_ordered_order_type_existence]{order type} of the walk can also serve as a definition --- in this case infinite walks will have length \( \omega \).

    As discussed in \cref{rem:adjoining_infinity}, we prefer working with extended real numbers.

    \thmitem{ex:def:extended_real_numbers/group_order} When defining the order of a group element in \cref{def:group_element_order}, we allow two equivalent conditions.

    If \( x \) has order \( n \), either \( n \) must be the minimal positive integer such that \( x^n = e \), or the generated subgroup \( \braket{ x } \) must have cardinality \( n \).

    If the subgroup has infinite cardinality, it can only be \( \omega \). So it makes sense to define the order as a member of \( \omega + 1 = \BbbN \cup \set{ \omega } \) rather than as an extended real number.

    We stick to the conventional notation \( \infty \) for elements of infinite order, which blurs the line with the corresponding extended real number.

    \thmitem{ex:def:extended_real_numbers/formal_order} When defining the minimal order of a \hyperref[def:hol_term/formula]{formula of higher-order logic} in \cref{def:hol_formula_ramification_order}, we are faced with the possibility of it being unramifiable. Then the order is undefined.

    We treat the order as either a nonnegative integer or \( \infty \), which allows us to easily handle the sums and maxima in \eqref{eq:def:hol_formula_ramification_order/direct} and \eqref{eq:def:quantifiable_type_ramification_rank}.
  \end{thmenum}
\end{example}

\begin{definition}\label{def:indeterminate_form}\mimprovised
  Some arithmetic expressions (\hyperref[def:first_order_syntax/term]{first-order terms} in the \hyperref[def:ring/theory]{theory of rings}) involving \hyperref[def:extended_real_numbers]{extended real numbers} are collectively called \term[bg=неопределености (\cite[228]{ИлинСадовничиСендов1984АнализТом1}), ru=неопределённые выражения (\cite[\S 41]{Фихтенгольц1968ОсновыАнализаТом1}) / неопределённость (\cite[235]{ИльинСадовничийСендов1985АнализТом1}), en=indeterminate forms (\cite[12]{Tao2011MeasureTheory})]{indeterminate forms}. As \incite[12]{Tao2011MeasureTheory} puts it,
  \begin{displayquote}
    \textellipsis one cannot assign a value to them without breaking many of the rules of algebra.
  \end{displayquote}

  These forms are the arithmetic expressions involving \( \infty \) and \( -\infty \), as well as \( 0^0 \) and \( x / 0 \) for any finite \( x \).
\end{definition}

\begin{remark}\label{rem:floating_point_indeterminate_forms}
  \hyperref[def:indeterminate_form]{Indeterminate forms} are supported by the standard for floating-point arithmetic, \cite[31]{IEEE:754:2019}, although their handling is somewhat controversial.

  The standard has a notion of a \term{signed zero}. The two zeros \( 0 \) and \( -0 \) are distinct, and, for \( x > 0 \),
  \begin{equation*}
    \frac x 0 = \infty = -\infty = \frac x {-0}.
  \end{equation*}

  Under the standard's notion of \enquote{equal}, which is not an \hyperref[def:equivalence_relation]{equivalence relation}, both zeros are considered equal, but \( \infty \) and \( -\infty \) are not.

  For other cases like \( 0 / 0 \), the standard introduces a new value, \( \op{NaN} \), abbreviation of \enquote{not a number}, which is not equal to anything, including itself.
\end{remark}

\begin{proposition}\label{thm:extended_real_numbers_are_not_ordered_field}
  Consider the \hyperref[def:extended_real_numbers]{extended real numbers} \( \BbbR \cup \set{ -\infty, \infty } \). If we extend \hyperref[def:real_number_arithmetic/addition]{addition} or \hyperref[def:real_number_arithmetic/multiplication]{multiplication} of real numbers to \( \infty \) and/or \( -\infty \) in any way, the obtained set is no longer an \hyperref[def:ordered_semiring]{ordered field}.
\end{proposition}
\begin{proof}
  Since \( 0 < 1 \), in an ordered field, \cref{thm:def:ordered_semiring/strict_sum} implies that \( \infty = 0 + \infty < 1 + \infty \). But \( \infty \) is the top element.

  Similarly, for multiplication, since \( 1 < 2 \), we have \( \infty < 2 \cdot \infty \), which is again a contradiction.

  The analogous result for \( -\infty \) follows via \fullref{thm:lattice_duality}.
\end{proof}

\begin{proposition}\label{thm:extended_real_semigroup}
  In the \hyperref[def:extended_real_numbers]{extended real numbers}, the following \hyperref[def:order_interval/closed]{intervals} are \hyperref[def:ordered_semigroup]{ordered} \hyperref[con:additive_semigroup]{additive semigroups}:
  \begin{align*}
    (-\infty, \infty) && (-\infty, \infty] && (x, \infty]   && [x, \infty)    && (x, \infty) \\
                      && [-\infty, \infty) && [-\infty, -x) && (-\infty, -x] && (-\infty, x)
  \end{align*}
  where \( 0 \leq x < \infty \).
\end{proposition}
\begin{proof}
  We have defined addition of \( -\infty \) and \( \infty \) so that it is associative whenever both are not involved simultaneously.

  To see that the interval is ordered, note that
  \begin{itemize}
    \item \( x \leq y < \infty \) implies \( x + \infty = y + \infty = \infty \).
    \item \( x \leq y = \infty \) implies \( y + z = \infty \) for any \( z \), thus \( x + z \leq y + z \).
    \item \( x = \infty \) and \( x \leq y \) imply \( x = y = \infty \), hence \( x + z = y + z = \infty \) for any \( z \).
  \end{itemize}
\end{proof}

\begin{definition}\label{def:effective_domain}\mimprovised
  We define the \term[ru=эффективное множество (\cite[50]{ПоловинкинБалашов2007ВыпуклыйАнализ}), en=effective domain (\cite[30]{Clarke2013OptimalControl})]{effective domain} of an extended real-valued function \( f: A \to [-\infty, \infty] \) as
  \begin{equation}\label{eq:def:effective_domain}
    \eff\dom f \coloneqq \set{ x \in A \given -\infty < f(x) < \infty }.
  \end{equation}

  We say that \( f \) is \term[ru=эффективное множество (\cite[50]{ПоловинкинБалашов2007ВыпуклыйАнализ}), en=effective domain (\cite[30]{Clarke2013OptimalControl})]{proper} if \( \eff\dom f \neq \varnothing \).
\end{definition}
\begin{comments}
  \item We base the definition on \cite[30]{Clarke2013OptimalControl} and \cite[50]{ПоловинкинБалашов2007ВыпуклыйАнализ}, but extend it to functions taking negative values (and possibly \( -\infty \)).
\end{comments}
