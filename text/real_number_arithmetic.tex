\section{Real number arithmetic}\label{sec:real_number_arithmetic}

\paragraph{Irrational number arithmetic}

\begin{proposition}\label{thm:real_number_arithmetic}
  Arithmetic on \hyperref[def:real_numbers]{real numbers} has the following basic properties:
  \begin{thmenum}
    \thmitem{thm:real_number_arithmetic/rational_irrational_sum} If \( x \) is rational and \( y \) is irrational, then \( x + y \) is irrational.
  \end{thmenum}
\end{proposition}
\begin{proof}
  \SubProofOf{thm:real_number_arithmetic/rational_irrational_sum} If \( x + y \) is rational, then \( y = (x + y) + (-x) \) must also be rational, which contradicts the assumption that it is irrational.
\end{proof}

\paragraph{Extended real numbers}

\begin{definition}\label{def:extended_real_numbers}\mimprovised
  We are sometimes interested in \term[ru=расширенная вещественная прямая (\cite[23]{ИоффеТихомиров1974ЭкстремальныеЗадачи}), en=extended real number system (\cite[10]{Folland1999RealAnalysis})]{extended real numbers}, also called the \term[en=extended real line (\cite[\S 1.4]{Rudin1987RealAndComplexAnalysis})]{extended real line}.

  We obtain it by taking the \hyperref[def:lower_cut_completion]{\hi{unbounded} lower cut completion} of the \hyperref[def:rational_numbers]{rational numbers}. Due to \fullref{thm:unbounded_lower_cut_completion/dual_macneille}, we can equivalently take the \hyperref[def:dedekind_macnielle_completion]{Dedekind-MacNeille completion}, but we generally find it more convenient working with lower cuts.

  This extends \( \BbbR \) with a \hyperref[def:extremal_points/top_and_bottom]{bottom element} \( -\infty \) (corresponding to the empty subset of \( \BbbQ \)) and \hyperref[def:extremal_points/top_and_bottom]{top element} \( \infty \) (corresponding to \( \BbbQ \) itself). We call these elements \term{positive infinite} (or simply \term{infinity}) and \term{negative infinity}.

  We will extend the algebraic operations of \( \BbbR \) to some \hyperref[def:indeterminate_form]{indeterminate forms} based on \cite[\S 1.1.17]{Schechter1997AnalysisHandbook} and \cite[11]{Folland1999RealAnalysis}. We leave \( \infty - \infty \) undefined and take precautions as not to encounter this situation (see for example \cref{def:measure/positive}). The absorbing property of \( 0 \) when multiplied by either \( -\infty \) or \( \infty \) is controversial, however it is quite useful in \fullref{ch:measure_theory} and \fullref{ch:convex_analysis}. The rest of the conventions are motivated by \hyperref[rem:bourbaki]{Bourbaki}'s \cite[\S IV.4]{Bourbaki1995GeneralTopology1to4}.

  In the following, \( x \) is a (finite) positive real number and the operations are extended by commutativity:
  \begin{equation*}
    \begin{array}{c !{\qquad} *{5}{c}}
      \toprule
      \xi                 & -\infty       & -x      & 0      & x       & \infty        \\
      \midrule
      \xi - \infty        & -\infty       & -\infty & \infty & -\infty & \T{undefined} \\
      \xi + \infty        & \T{undefined} & \infty  & \infty & \infty  & \infty \\
      \xi \cdot (-\infty) & \infty        & \infty  & 0      & -\infty & -\infty \\
      \xi \cdot \infty    & -\infty       & -\infty & 0      & \infty  & \infty \\
    \end{array}
  \end{equation*}
\end{definition}
\begin{comments}
  \item See \cref{rem:lemniscate_symbol} for a general discussion of the lemniscate symbol \( \infty \) and \cref{con:transfinitum} for a discussion of infinity.
\end{comments}

\begin{definition}\label{def:indeterminate_form}\mimprovised
  Some arithmetic expressions (\hyperref[def:first_order_syntax/term]{first-order terms} in the \hyperref[def:ring/theory]{theory of rings}) involving \hyperref[def:extended_real_numbers]{extended real numbers} are collectively called \term[bg=неопределености (\cite[228]{ИлинСадовничиСендов1984АнализТом1}), ru=неопределённые выражения (\cite[\S 41]{Фихтенгольц1968ОсновыАнализаТом1}) / неопределённость (\cite[235]{ИльинСадовничийСендов1985АнализТом1}), en=indeterminate forms (\cite[12]{Tao2011MeasureTheory})]{indeterminate forms}. As \incite[12]{Tao2011MeasureTheory} puts it,
  \begin{displayquote}
    \textellipsis one cannot assign a value to them without breaking many of the rules of algebra.
  \end{displayquote}

  These forms are the arithmetic expressions involving \( \infty \) and \( -\infty \), as well as \( 0^0 \) and \( x / 0 \) for any finite \( x \).
\end{definition}

\begin{remark}\label{rem:floating_point_indeterminate_forms}
  \hyperref[def:indeterminate_form]{Indeterminate forms} are supported by the standard for floating-point arithmetic, \cite[31]{IEEE:754:2019}, although their handling is somewhat controversial.

  The standard has a notion of a \term{signed zero}. The two zeros \( 0 \) and \( -0 \) are distinct, and, for \( x > 0 \),
  \begin{equation*}
    \frac x 0 = \infty = -\infty = \frac x {-0}.
  \end{equation*}

  Under the standard's notion of \enquote{equal}, which is not an \hyperref[def:equivalence_relation]{equivalence relation}, both zeros are considered equal, but \( \infty \) and \( -\infty \) are not.

  For other cases like \( 0 / 0 \), the standard introduces a new value, \( \op{NaN} \), abbreviation of \enquote{not a number}, which is not equal to anything, including itself.
\end{remark}

\begin{proposition}\label{thm:extended_real_numbers_are_not_ordered_field}
  Consider the \hyperref[def:extended_real_numbers]{extended real numbers} \( \BbbR \cup \set{ -\infty, \infty } \). If we extend \hyperref[def:real_number_arithmetic/addition]{addition} or \hyperref[def:real_number_arithmetic/multiplication]{multiplication} of real numbers to \( \infty \) and/or \( -\infty \) in any way, the obtained set is no longer an \hyperref[def:ordered_semiring]{ordered field}.
\end{proposition}
\begin{proof}
  Since \( 0 < 1 \), in an ordered field, \cref{thm:def:ordered_semiring/strict_sum} implies that \( \infty = 0 + \infty < 1 + \infty \). But \( \infty \) is the top element.

  Similarly, for multiplication, since \( 1 < 2 \), we have \( \infty < 2 \cdot \infty \), which is again a contradiction.

  The analogous result for \( -\infty \) follows via \fullref{thm:lattice_duality}.
\end{proof}

\begin{proposition}\label{thm:extended_real_semigroup}
  In the \hyperref[def:extended_real_numbers]{extended real numbers}, the following \hyperref[def:order_interval/closed]{intervals} are \hyperref[def:ordered_semigroup]{ordered} \hyperref[con:additive_semigroup]{additive semigroups}:
  \begin{align*}
    (-\infty, \infty) && (-\infty, \infty] && (x, \infty]   && [x, \infty)    && (x, \infty) \\
                      && [-\infty, \infty) && [-\infty, -x) && (-\infty, -x] && (-\infty, x)
  \end{align*}
  where \( 0 \leq x < \infty \).
\end{proposition}
\begin{proof}
  We have defined addition of \( -\infty \) and \( \infty \) so that it is associative whenever both are not involved simultaneously.

  To see that the interval is ordered, note that
  \begin{itemize}
    \item \( x \leq y < \infty \) implies \( x + \infty = y + \infty = \infty \).
    \item \( x \leq y = \infty \) implies \( y + z = \infty \) for any \( z \), thus \( x + z \leq y + z \).
    \item \( x = \infty \) and \( x \leq y \) imply \( x = y = \infty \), hence \( x + z = y + z = \infty \) for any \( z \).
  \end{itemize}
\end{proof}

\begin{definition}\label{def:effective_domain}\mimprovised
  We define the \term[ru=эффективное множество (\cite[50]{ПоловинкинБалашов2007ВыпуклыйАнализ}), en=effective domain (\cite[30]{Clarke2013OptimalControl})]{effective domain} of an extended real-valued function \( f: A \to [-\infty, \infty] \) as
  \begin{equation}\label{eq:def:effective_domain}
    \eff\dom f \coloneqq \set{ x \in A \given -\infty < f(x) < \infty }.
  \end{equation}

  We say that \( f \) is \term[ru=эффективное множество (\cite[50]{ПоловинкинБалашов2007ВыпуклыйАнализ}), en=effective domain (\cite[30]{Clarke2013OptimalControl})]{proper} if \( \eff\dom f \neq \varnothing \).
\end{definition}
\begin{comments}
  \item We base the definition on \cite[30]{Clarke2013OptimalControl} and \cite[50]{ПоловинкинБалашов2007ВыпуклыйАнализ}, but extend it to functions taking negative values (and possibly \( -\infty \)).
\end{comments}
