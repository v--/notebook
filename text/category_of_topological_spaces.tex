\section{Category of topological spaces}\label{sec:category_of_topological_spaces}

\begin{definition}\label{def:category_of_small_topological_spaces}
  Since the topology \( \mscrT \) of a \hyperref[def:topological_space]{topological space} \( (X, \mscrT) \) consists of subsets of \( X \), we cannot build a \hyperref[def:first_order_theory]{first-order theory} from \fullref{def:topological_space/O1}-\fullref{def:topological_space/O3}. We can, however, explicitly describe the \hyperref[def:category]{category} \( \cat{Top} \) of topological spaces as
  \begin{refenum}
    \refitem{def:category/objects} The \hyperref[def:set]{class} of objects is the class of all topological space.
    \refitem{def:category/morphisms} The morphisms between two topological spaces are the \hyperref[def:global_continuity]{continuous functions} between them.
    \refitem{def:category/composition} Composition of morphisms is the usual \hyperref[def:set_valued_map/composition]{function composition}.
  \end{refenum}
\end{definition}

\begin{theorem}\label{thm:top_complete_cocomplete}
  The category \( \cat{Top} \) of is both \hyperref[def:category_of_cones/limit]{complete} and \hyperref[def:category_of_cones/colimit]{cocomplete}.
\end{theorem}

\begin{definition}\label{def:initial_topology}\mcite{nLab:top}
  Let \( \{ (X_k, \mscrT_k) \}_{k \in \mscrK} \) be a \hyperref[def:indexed_family]{family} of topological spaces. Let \( X \) be a bare set and let
  \begin{equation*}
    \{ f_k: X \to X_k \}_{k \in \mscrK}
  \end{equation*}
  be a family of functions.

  The topology on \( X \) generated by the subbase
  \begin{equation*}
    \mathcal{P} \coloneqq \{ f_k^{-1}(U) \colon k \in \mscrK, U \in T_k \}
  \end{equation*}
  is called the \term{initial} (or \term{weak}) topology on \( X \) generated by the family \( \{ f_k \}_{k \in \mscrK} \).

  It is the weakest topology that makes all functions in the family \( \{ f_k \}_{k \in \mscrK} \) continuous.
\end{definition}

\begin{definition}\label{def:final_topology}\mcite{nLab:top}
  Dually, if the family of functions is of the type
  \begin{equation*}
    \{ f_k: X_k \to X \}_{k \in \mscrK},
  \end{equation*}
  then we define the \term{final} (or \term{strong}) topology on \( X \) generated by the family \( \{ f_k \}_{k \in \mscrK} \) as the topology
  \begin{equation*}
    \mscrT \coloneqq \{ U \subseteq X \colon \forall k \in \mscrK, f_k^{-1}(U) \in T_k \}.
  \end{equation*}

  It is the strongest topology that makes all functions in the family \( \{ f_k \}_{k \in \mscrK} \) continuous.
\end{definition}

\begin{proposition}\label{thm:initial_final_topology_limit}\mcite{nLab:top}
  Let \( D: \op* I \to \cat{Top} \) be a small \hyperref[def:categorical_diagram]{diagram}. For each space in the image \( D(\op* I) \), denote the set corresponding by \( X_k \) and the corresponding topology by \( \mscrT_k \).

  The limit (resp. colimit) \( (X, \mscrT) \) of \( D \) can then be described as
  \begin{thmenum}
    \item \( (X, \{ f_k \}_{k \in \cat{I}}) = \varprojlim UD \) (resp. \( \varinjlim UD \)) is the limit (resp. colimit) in \( \cat{Set} \) of \( U \circ D \), where \( U: \op*{Top} \to \cat{Set} \) is the forgetful functor.
    \item \( \mscrT \) is the \hyperref[def:initial_topology]{initial} (resp. \hyperref[def:final_topology]{final}) topology on \( X \) generated by the family of functions \( \{ f_k \}_{k \in \cat{I}} \).
  \end{thmenum}
\end{proposition}

\begin{definition}\label{def:topological_subspace}
  Let \( (X, \mscrT) \) be a topological space and let \( M \subseteq X \) be a subset of \( X \). The \term{topological subspace} \( (M, \mscrT_M) \) is obtained by endowing \( M \) with the topology
  \begin{equation*}
    \mscrT_M \coloneqq \{ U \cap M \colon U \in T \}.
  \end{equation*}

  The topology \( \mscrT_M \) is called the \term{subspace topology} or \term{induced topology}.

  It is the initial topology generated by the canonical embedding \( \iota: M \to X \).
\end{definition}

\begin{definition}\label{def:topological_product}
  The \term{topological product} or \term{Tychonoff product}
  \begin{equation*}
    \left( \prod_{k \in \mscrK} X_k, \prod_{k \in \mscrK} \mscrT_k \right)
  \end{equation*}
  of the family \( { (X_k, \mscrT_k) }_{k \in \mscrK} \) is simply the categorical product in the category \( \cat{Top} \) (see \fullref{def:discrete_category_limits}). The underlying set \( \prod_{k \in \mscrK} X_k \) is the \hyperref[thm:discrete_category_limits_in_set]{Cartesian product} and the topology \( \prod_{k \in \mscrK} \mscrT_k \) is called the \term{product topology}.

  Let \( { (X_k, \mscrT_k) }_{k \in \mscrK} \) and \( { (Y_k, \mathcal{O}_k) }_{k \in \mscrK} \) be two families of topological spaces and let
  \begin{equation*}
    \{ f_k: X_k \to Y_k \}_{k \in \mscrK}
  \end{equation*}
  be a family of arbitrary functions between them.

  We define the \term{product \( \prod_{k \in \mscrK} f_k \) of \( \{ f_k \}_{k \in \mscrK} \)} as the function
  \begin{balign*}
     & \left(\prod_{k \in \mscrK} f_k \right): \prod_{k \in \mscrK} X_k \to \prod_{k \in \mscrK} Y_k              \\
     & \left(\prod_{k \in \mscrK} f_k \right)(\seq{ x_k }_{k \in \mscrK}) \coloneqq \{ f_k (x_k) \}_{k \in \mscrK}.
  \end{balign*}

  If all of the spaces \( (X_k, \mscrT_k) \) are equal to some space \( (X, \mscrT) \), we call the product of \( \{ f_k \}_{k \in \mscrK} \) the \term{diagonal product} and denote it by
  \begin{equation*}
    \Delta_{k \in \mscrK} f_k: X \to \prod_{k \in \mscrK} Y_k.
  \end{equation*}
\end{definition}

\begin{definition}\label{def:topological_quotient}\mcite[90]{Engelking1989GeneralTopology}
  Let \( X \) be a topological space and let \( \cong \) be an \hyperref[def:equivalence_relation]{equivalence relation} on \( X \). The \term{quotient space} \( (X, \mscrT) / \sim \) is obtained by endowing the quotient set \( X / \cong \) with the final \hyperref[def:final_topology]{topology} given by the canonical projection map \( x \mapsto [x] \).
\end{definition}

\begin{definition}\label{def:topological_sum}\mcite[74]{Engelking1989GeneralTopology}
  The \term{topological direct sum}
  \begin{equation*}
    (\oplus_{k \in \mscrK} X_k, \oplus_{k \in \mscrK} \mscrT_k)
  \end{equation*}
  of the family \( { (X_k, \mscrT_k) }_{k \in \mscrK} \) is simply the categorical coproduct in the category \( \cat{Top} \) (see \fullref{def:discrete_category_limits}). The underlying set \( \oplus_{k \in \mscrK} X_k \) is the \hyperref[thm:discrete_category_limits_in_set]{disjoint union} and the topology \( \oplus_{k \in \mscrK} \mscrT_k \) is called the \term{direct sum topology}.

  Let \( { (X_k, \mscrT_k) }_{k \in \mscrK} \) and \( { (Y_k, \mathcal{O}_k) }_{k \in \mscrK} \) be two families of topological spaces and let
  \begin{equation*}
    \{ f_k: X_k \to Y_k \}_{k \in \mscrK}
  \end{equation*}
  be a family of arbitrary functions between them. Let \( \iota_{X_k}: X_k \to \oplus_{k \in \mscrK} X_k \) and \( \iota_{Y_k}: Y_k \to \oplus_{k \in \mscrK} Y_k \) be the corresponding canonical embeddings.

  We define the \term{direct sum \( \oplus_{k \in \mscrK} f_k \) of \( \{ f_k \}_{k \in \mscrK} \)} as the function
  \begin{balign*}
     & (\oplus_{k \in \mscrK} f_k): \oplus_{k \in \mscrK} X_k \to \oplus_{k \in \mscrK} Y_k   \\
     & (\oplus_{k \in \mscrK} f_k){\rvert}_{X_k} \coloneqq \iota_{Y_k} \circ f_k.
  \end{balign*}

  Obviously \( \oplus_{k \in \mscrK} f_k \) is continuous whenever all \( f_k \) are continuous.

  If all of the spaces \( (Y_k, \mathcal{O}_k) \) are equal to some space \( (Y, \mathcal{O}) \), we call the direct sum of \( \{ f_k \}_{k \in \mscrK} \) simply a \term{sum} and denote it by
  \begin{equation*}
    \sum_{k \in \mscrK} f_k: \oplus_{k \in \mscrK} X_k \to Y.
  \end{equation*}
\end{definition}

\begin{definition}\label{def:borel_hierarchy}
  \todo{Define}.
\end{definition}

\begin{definition}\label{def:category_of_small_frames}\mcite[43]{Johnstone1983PointlessTopology}
  Suppose that we are given a \hyperref[def:grothendieck_universe]{Grothendieck universe} \( \mscrU \), which is safe to assume to be the smallest suitable one as explained in \fullref{def:large_and_small_sets}. We describe the \term{category of \( \mscrU \)-small frames} as the following \hyperref[rem:concrete_categories]{concrete category}:

  \begin{itemize}
    \item The \hyperref[def:category/objects]{objects} are the \( \mscrU \)-small \hyperref[def:complete_lattice]{complete lattices} in which arbitrary meets distribute over finite joins. We call such lattices \term{frames}.

    \item The \hyperref[def:category/morphisms]{morphisms} between two frames are the functions between them preserving finite meets and arbitrary joins. We call such functions \term{frame homomorphisms}.
  \end{itemize}
\end{definition}

\begin{proposition}\label{thm:topological_spaces_are_frames}
  The topology of a topological space is a \hyperref[def:category_of_small_frames]{frame}.
\end{proposition}
\begin{proof}
  Trivial.
\end{proof}

\begin{remark}\label{rem:topology_frame_homomorphism}
  Consider the continuous function \( f: X \to Y \) between topological spaces.

  \Fullref{thm:function_preimage_properties/union} and \fullref{thm:function_preimage_properties/intersection} imply that the inverse \( f^{-1}: Y \to X \) preserves arbitrary unions and intersections and is hence a \hyperref[def:category_of_small_frames]{frame homomorphism} from \( \mscrT_Y \) to \( \mscrT_X \).

  This motivates defining \hyperref[def:category_of_small_locales]{locales}.
\end{remark}

\begin{definition}\label{def:category_of_small_locales}\mcite[43]{Johnstone1983PointlessTopology}
  We call the \hyperref[def:opposite_category]{opposite category} of the \hyperref[def:category_of_small_frames]{category of \( \mscrU \)-small frames} the \term{category of \( \mscrU \)-small locales}.
\end{definition}

\begin{remark}\label{rem:picking_a_point_from_a_locale}
  The topology of the one-element topological space is a \hyperref[thm:two_element_lattice]{two-element lattice}.

  The \hyperref[def:category_of_small_locales]{locale homomorphism} \( f^\oppos: \set{ \top, \bot } \to L \) then corresponds to a continuous function from the one-element space to some other space \( X \) whose topology is isomorphic to \( L \). This reduces to picking a point from \( X \).
\end{remark}

\begin{lemma}\label{thm:frame_homomorphism_kernel}\mcite[45]{Johnstone1983PointlessTopology}
  Fix a \hyperref[thm:two_element_lattice]{two-element lattice} \( \set{ \top, \bot } \). It is vacuously a \hyperref[def:category_of_small_locales]{frame}. Fix also an arbitrary locale \( L \).

  Then the \hyperref[def:lattice/homomorphism]{lattice homomorphisms} \( f: L \to \set{ \top, \bot } \) is a \hyperref[def:category_of_small_frames]{frame homomorphism} if and only if \( f^{-1}(\top) \) is a \hyperref[def:lattice_ideal/prime]{completely prime filter}.
\end{lemma}
\begin{proof}
  \SufficiencySubProof Suppose that \( f: L \to \set{ \top, \bot } \) is a frame homomorphism.

  We will first show that \( f^{-1}(\top) \) is a filter.
  \begin{itemize}
    \item It is closed under meets. Let \( f(a) = f(b) = \top \). Then, since \( f \) preserves meets,
    \begin{equation*}
      f(a \wedge b) = f(a) \wedge f(b) = \top \wedge \top = \top.
    \end{equation*}

    \item It is closed under joins with elements of \( L \). Let \( a \in L \) and \( f(b) = \top \). Then
    \begin{equation*}
      f(a \vee b) = f(a) \vee f(b) = f(a) \vee \top = \top.
    \end{equation*}
  \end{itemize}

  It remains to show that \( f^{-1}(\top) \) is completely prime. Let
  \begin{equation*}
    \top = f\parens*{ \bigvee_{k \in \mscrK} a_k } = \bigvee_{k \in \mscrK} f(a_k).
  \end{equation*}

  If \( f(a_k) = \bot \) for every \( k \in \mscrK \),
  \begin{equation*}
    \bigvee_{k \in \mscrK} f(a_k) = \bot.
  \end{equation*}

  Hence, there exists some \( k \in \mscrK \) such that \( f(a_k) = \top \).

  Therefore, \( f^{-1}(\top) \) is a completely prime filter.

  \NecessitySubProof Suppose that \( f \) is a lattice homomorphism and that \( f^{-1}(\top) \) is a completely prime filter. We will show that \( f \) preserves arbitrary joins.

  Let \( \seq{ a_k }_{k \in \mscrK} \) be some family of members of \( L \).
  \begin{itemize}
    \item Suppose that \( f(a_k) = \bot \) for each \( k \in \mscrK \).

    If \( f\parens*{ \bigvee_{k \in \mscrK} a_k } = \top \), since \( f^{-1}(\top) \) is completely prime filter, there exists some \( k_0 \in \mscrK \) such that \( f(a_{k_0}) = \top \). The obtained contradiction shows that
    \begin{equation*}
      f\parens*{ \bigvee_{k \in \mscrK} a_k } = \bot = \bigvee_{k \in \mscrK} f(a_k)
    \end{equation*}

    \item Suppose that \( f(a_{k_0}) = \top \) for some \( k_0 \in \mscrK \).

    Then, since \( f \) preserves finite joins,
    \begin{equation*}
      f\parens*{ \bigvee_{k \in \mscrK} a_k }
      =
      f\parens*{ \bigvee_{k \neq k_0} a_k } \vee f(a_{k_0})
      =
      f\parens*{ \bigvee_{k \neq k_0} a_k } \vee \top
      =
      \top
      =
      \bigvee_{k \in \mscrK} f(a_k).
    \end{equation*}
  \end{itemize}
\end{proof}

\begin{remark}\label{rem:prime_elements_of_locale}
  We discussed in \fullref{rem:picking_a_point_from_a_locale} how to \enquote{pick a point} from a locale via locale maps from the two-element locale. \Fullref{thm:frame_homomorphism_kernel} and \fullref{def:lattice_prime_element} imply that these functions correspond exactly to prime elements of the locale.
\end{remark}

\begin{proposition}\label{thm:locale_to_topology}\mcite[45]{Johnstone1983PointlessTopology}
  We define the set \( X \) of \term{points} of a \hyperref[def:category_of_small_locales]{locale} \( L \) as the set of all \hyperref[def:lattice_prime_element]{prime elements} of \( L \).

  Now the elements of \( L \) must become neighborhoods of these points. For each \( a \in L \), define the \enquote{closed} set
  \begin{equation*}
    F_a \coloneqq \set{ x \in X \given x \leq a }.
  \end{equation*}

  Finally, define
  \begin{equation*}
    \mscrF \coloneqq \set{ F_a \given a \in L }.
  \end{equation*}

  Then \( \mscrF \) is a family of closed sets for a topology on \( X \).
\end{proposition}
\begin{proof}
  The union and intersection of an arbitrary family of \enquote{closed} sets is again closed because \( L \) is complete. We only need to show \ref{thm:topology_from_closed_sets/C2}.

  First note that \( \bot \) is vacuously a prime element and that \( L \) has no elements strictly less than \( \bot \). Hence, \( F_\bot = \varnothing \) and thus \( \mscrF \) contains the empty set.

  The top \( \top \) is also prime, and \( F_\top = L \), implying that \( \mscrF \) also contains the set of all points.

  Therefore, \( \mscrF \) induces a topology on \( X \).
\end{proof}
