\section{First-order logic}\label{sec:first_order_logic}

The idea of first-order predicate logic (we will omit \enquote{predicate} and only refer to \enquote{first-order logic}) is to create a formal language whose semantics (given by structures) supports boolean operations and can quantify over all elements of an ambient universe. Unlike in \hyperref[sec:propositional_logic]{propositional logic}, there are different first-order logic languages.

\begin{definition}\label{def:first_order_language}\mcite[def. 2.1.2]{Hinman2005Logic}
  A \term{first-order language}\fnote{As in propositional logic, a first-order language is an \hyperref[def:formal_language]{alphabet} rather than a \hyperref[def:formal_language/language]{formal language}.} \( \mscrL \) extends the language of \hyperref[sec:propositional_logic]{propositional logic} and consists of two types of symbols.

  \begin{description}
    \item[Logical symbols]
    \hfill
    \begin{thmenum}[series=def:first_order_language]
      \thmitem{def:first_order_language/propositional} The entirety of the \hyperref[sec:propositional_logic]{propositional logic language}.

      \thmitem{def:first_order_language/quantifiers} The set \( \op*{Quan} \) consisting of the \term[ru=квантор общости (\cite[61]{Эдельман1975Логика})]{universal quantifier} \enquote{\( \synforall \)} and the \term[ru=квантор существования (\cite[61]{Эдельман1975Логика})]{existential quantifier} \enquote{\( \synexists \)}.

      \thmitem{def:first_order_language/dot} A dot \enquote{\( . \)} for separating a quantifier from its formula.

      \thmitem{def:first_order_language/equality} A symbol for \term{formal equality}\fnote{Equality is sometimes omitted by logicians, but examples of first-order languages without formal equality are obscure.} \enquote{\( \syneq \)}.
    \end{thmenum}

    \item[Non-logical symbols]
    \hfill
    \begin{thmenum}[resume=def:first_order_language]
      \thmitem{def:first_order_language/fun} A possibly empty \hi{finite} set \( \op*{Fun} \) of symbols for denoting functions.

      Note that, despite using notation like \( f_n \), we regard \( f_n \) as a single symbol, and it is usually a single symbol like \( + \) or \( / \).

      Each functional symbol has an associated natural number called its \term{arity}, which we denote by \( \# f \). Functional symbols of zero arity are called \term{constants}.

      Of course, none of the functional symbols are allowed to clash with the logical symbols.

      \thmitem{def:first_order_language/pred} A possibly empty \hi{finite} set \( \op*{Pred} \) of symbols for denoting predicates.

      Predicate symbols also have an associated arity. Predicate symbols of zero arity act as \hyperref[def:propositional_syntax/prop]{propositional variables}.

      None of the predicate symbols are allowed to clash with either the functional symbols or with the logical symbols.
    \end{thmenum}
  \end{description}

  The logical symbols are common for all first-order languages. Thus, first-order languages differ by their non-logical symbols. The collection of functional and predicate symbols of a language are sometimes called its \term{signature}.
\end{definition}
\begin{comments}
  \item The dot is not itself a quantifier and is not strictly necessary --- we use it only for readability.

  \item We can avoid functional symbols altogether in favor of predicate symbols because functions can be represented via relations. This however introduces certain complications --- see \fullref{ex:replacing_functional_symbols_via_relations}.
\end{comments}

\begin{remark}\label{rem:uncountable_first_order_language}
  We require the sets of \hyperref[def:first_order_language/fun]{functional symbols} and \hyperref[def:first_order_language/fun]{predicate symbols} to be finite. This restriction is natural in several ways:
  \begin{itemize}
    \item This allows us to stick to the conventional theory of \hyperref[def:formal_language/language]{formal languages} over finite alphabets and \hyperref[def:formal_grammar]{formal grammars} over finite alphabets with finitely many rules.

    \item The above carets to the philosophical implication of languages being at most countable.

    \item The above also allows us to implement these concepts using conventional programming languages, for example in \cite{notebook:code}.
  \end{itemize}

  Unfortunately, this decision also limits us in certain definitions like \fullref{def:semimodule/theory}, where we may need uncountably many functional symbols. This particular example may be considered pathological, because more general logical frameworks provide a more elegant solution.

  Thus, we restrict ourselves to what Peter Hinman in \incite[def. 2.1.27]{Hinman2005Logic} calls \term{finite languages}.

  An alternative is to use \fullref{thm:knaster_tarski_theorem} rather than formal grammars, however this has the downside of allowing languages which cannot be parsed.
\end{remark}

\begin{definition}\label{def:first_order_syntax}
  Similarly to the \hyperref[def:propositional_syntax]{syntax of propositional logic}, we define the \term{syntax} of a fixed \hyperref[def:first_order_language]{first-order language} \( \mscrL \).

  \begin{thmenum}
    \thmitem{def:first_order_syntax/grammar_schema} Consider the following \hyperref[def:formal_grammar/schema]{grammar schema}:
    \begin{bnf*}
      \bnfprod{connective}      {\bnftsq{\( \synvee \)} \bnfor \bnftsq{\( \synwedge \)} \bnfor \bnftsq{\( \synimplies \)} \bnfor \bnftsq{\( \syniff \)}} \\
      \bnfprod{quantifier}      {\bnftsq{\( \synforall \)} \bnfor \bnftsq{\( \synexists \)}} \\
      \bnfprod{variable}        {\bnfpn{Small Latin identifier}} \\
      \bnfprod{term}            {\bnfpn{variable} \bnfor} \\
      \bnfmore                  {f \thickspace \} \thickspace \T*{Standalone rule for each} f \in \op*{Fun} \T*{where} \# f = 0} \\
      \bnfmore                  {\underbrace{f \bnfsp \bnftsq{(} \bnfsp \overbrace{\bnfpn{term} \bnfsp \bnftsq{,} \bnfsp \bnfsk \bnfsp \bnftsq{,} \bnfsp \bnfpn{term}}^{\# f \T*{terms separated by commas}} \bnfsp \bnftsq{)}}_{\T*{Standalone rule for each} f \in \op*{Fun} \T*{where} \# f > 0}} \\
      \bnfprod{atomic formula}  {\bnftsq{\( \syntop \)} \bnfor \bnftsq{\( \synbot \)} \bnfor} \\
      \bnfmore                  {\bnftsq{(} \bnfsp \bnfpn{term} \bnfsp \bnftsq{\( \syneq \)} \bnfsp \bnfpn{term} \bnfsp \bnftsq{)} \bnfor} \\
      \bnfmore                  {\underbrace{p \bnfsp \bnftsq{(} \bnfsp \overbrace{\bnfpn{term} \bnfsp \bnftsq{,} \bnfsp \bnfsk \bnfsp \bnftsq{,} \bnfsp \bnfpn{term}}^{\# p \T*{terms separated by commas}} \bnfsp \bnftsq{)}}_{\T*{Standalone rule for each} p \in \op*{Prod}}} \\
      \bnfprod{formula}         {\bnfpn{atomic formula} \bnfor} \\
      \bnfmore                  {\bnftsq{\( \syntop \)} \bnfor \bnftsq{\( \synbot \)} \bnfor} \\
      \bnfmore                  {\bnftsq{\( \synneg \)} \bnfpn{formula} \bnfor} \\
      \bnfmore                  {\bnftsq{(} \bnfsp \bnfpn{formula} \bnfsp \bnfpn{connective} \bnfsp \bnfpn{formula} \bnfsp \bnftsq{)} \bnfor} \\
      \bnfmore                  {\bnfpn{quantifier} \bnfsp \bnfpn{variable} \bnfsp \bnftsq{.} \bnfsp \bnfpn{formula}}
    \end{bnf*}

    \thmitem{def:first_order_syntax/var} We denote by \( \op*{Var} \) the set of strings generated by the above grammar schema with starting nonterminal \( \bnfpn{variable} \).

    \thmitem{def:first_order_syntax/term} Similarly, we denote by \( \op*{Term} \) the set of strings generated with starting nonterminal \( \bnfpn{term} \).

    We implicitly associate with each first-order term a syntax tree. The grammar of terms is unambiguous as shown via \fullref{thm:propositional_formula_grammar/unambiguous}, which makes it possible to perform proofs via \fullref{thm:induction_on_rooted_trees}.

    \thmitem{def:first_order_syntax/subterm} If \( \tau \) and \( \kappa \) are terms and \( \kappa \) is a \hyperref[def:formal_language/substring]{substring} of \( \tau \), we say that \( \kappa \) is a \term{subterm} of \( \tau \).

    \thmitem{def:first_order_syntax/term_variables} For each term \( \tau \), we define the set of variables occurring in the term:
    \begin{equation}\label{eq:def:first_order_syntax/term_variables}
      \op*{Var}(\tau) \coloneqq \begin{cases}
        \synx,                                                            &\tau = \synx \in \op*{Var}, \\
        \op*{Var}(\kappa_1) \cup \ldots \cup \op*{Var}(\kappa_n), &\tau = f(\kappa_1, \ldots, \kappa_n).
      \end{cases}
    \end{equation}

    \thmitem{def:first_order_syntax/closed_term} A term \( \tau \) is called a \term{closed term} if \( \op*{Var}(\tau) = \varnothing \).

    Peter Hinman in \incite[def. 2.6.10]{Hinman2005Logic} calls them \enquote{constant terms}, but we prefer \enquote{closed term} because of similarity with closed formulas defined in \fullref{def:first_order_syntax/closed_formula}.

    \thmitem{def:first_order_syntax/atomic_formula} We denote by \( \op*{Var} \) the set of strings generated by the above grammar schema with starting nonterminal \( \bnfpn{atomic formula} \).

    \thmitem{def:first_order_syntax/formula} Similarly, we denote by \( \op*{Formula} \) the set of strings generated with starting nonterminal \( \bnfpn{formula} \).

    The grammar of first-order formulas is unambiguous as shown in \fullref{thm:first_order_terms_and_formulas_are_unambiguous}.

    \Fullref{ex:first_order_substitution} discussed the structure of some first-order formulas, while arbitrarily complicated formulas can be found in \fullref{ch:set_theory}, for example the axiom of universes \eqref{eq:def:axiom_of_universes}.

    \thmitem{def:first_order_syntax/subformula} If \( \varphi \) and \( \psi \) are formulas and \( \psi \) is a \hyperref[def:formal_language/substring]{substring} of \( \varphi \), we say that \( \psi \) is a \term{subformula} of \( \varphi \).

    \thmitem{def:first_order_syntax/formula_terms} If \( \varphi \) is a formula, if \( \tau \) is a term and if \( \tau \) is a \hyperref[def:formal_language/substring]{substring} of \( \varphi \), we say that \( \tau \) is a \term{term} of \( \varphi \).

    \thmitem{def:first_order_syntax/formula_bound_variables} For each formula \( \varphi \), we define its set of \term{bound variables} as
    \begin{equation*}
      \op*{Bound}(\varphi) \coloneqq \begin{cases}
        \varnothing,                                        &\varphi \T{is atomic,} \\
        \op*{Bound}(\psi),                               &\varphi = \synneg \psi, \\
        \op*{Bound}(\psi_1) \cup \op*{Bound}(\psi_2), &\varphi = \psi_1 \syncirc \psi_2, {\syncirc} \in \op*{Conn}, \\
        \op*{Bound}(\psi) \cup \set{ \synx },              &\varphi = \quantifier{q}{\synx} \psi, q \in \op*{Quan}.
      \end{cases}
    \end{equation*}

    \thmitem{def:first_order_syntax/formula_free_variables} We also define the set of \term{free variables} as
    \begin{equation*}
      \op*{Free}(\varphi) \coloneqq \begin{cases}
        \varnothing,                                                &\varphi \in \set{ \syntop, \synbot }, \\
        \op*{Var}(\tau_1) \cup \ldots \cup \op*{Var}(\tau_n), &\varphi = p(\tau_1, \ldots, \tau_n), \\
        \op*{Var}(\tau_1) \cup \op*{Var}(\tau_2),             &\varphi = \tau_1 \syneq \tau_2, \\
        \op*{Free}(\psi),                                        &\varphi = \synneg \psi, \\
        \op*{Free}(\psi_1) \cup \op*{Free}(\psi_2),           &\varphi = \psi_1 \syncirc \psi_2, {\syncirc} \in \op*{Conn}, \\
        \op*{Free}(\psi) \setminus \set{ \synx },                  &\varphi = \quantifier{q}{\synx} \psi, q \in \op*{Quan}
      \end{cases}
    \end{equation*}

    \thmitem{def:first_order_syntax/closed_formula}\mcite[def. 2.2.7]{Hinman2005Logic} A formula \( \varphi \) is called a \term{closed formula}.

    Peter Hinman in \incite[def. 2.2.7]{Hinman2005Logic} calls them \term{sentences}. This contrasts with propositional logic, where all formulas are called sentences --- see \fullref{def:propositional_syntax/formula}. We prefer \enquote{closed formula} because of similarity with closed terms defined in \fullref{def:first_order_syntax/closed_term}.

    \thmitem{def:first_order_syntax/formula_variables} Finally, the set of all variables of a formula \( \varphi \) is
    \begin{equation*}
      \op*{Var}(\varphi) \coloneqq \op*{Free}(\varphi) \cup \op*{Bound}(\varphi).
    \end{equation*}
  \end{thmenum}
\end{definition}
\begin{comments}
  \item Certain important theorems like \fullref{thm:semantic_deduction_theorem} and \fullref{thm:syntactic_deduction_theorem} require some formulas to be closed. In general, wherever we need closed formulas, we will rely on implicit quantification as mentioned in \fullref{rem:mathematical_logic_conventions/quantification}.
\end{comments}

\begin{proposition}\label{thm:first_order_terms_and_formulas_are_unambiguous}
  The grammars of \hyperref[def:first_order_syntax/term]{first-order terms} and of \hyperref[def:first_order_syntax/formula]{first-order formulas} are \hyperref[def:grammar_ambiguity]{unambiguous}.
\end{proposition}
\begin{proof}
  The proof is more complicated, but similar to \fullref{thm:propositional_formula_grammar/unambiguous}.
\end{proof}

\begin{remark}\label{rem:postfix_notation}
  \todo{Describe postfix notation}
\end{remark}

\begin{remark}\label{rem:first_order_formula_conventions}
  In order to simplify exposition, we use several conventions. As in the case of \fullref{rem:propositional_formula_parentheses}, both of these conventions exist only in the metalanguage and the formulas themselves are assumed to have the former form within the object language.
  \begin{thmenum}
    \thmitem{rem:first_order_formula_conventions/parentheses} We use the parentheses conventions from \fullref{rem:propositional_formula_parentheses}.

    \thmitem{rem:first_order_formula_conventions/infix} Binary functional symbols are often written using \term{infix notation}, i.e.
    \begin{equation*}
      \synz \syneq \synx + \syny
    \end{equation*}
    rather than the \term{prefix notation}
    \begin{equation*}
      \synz \syneq +(\synx, \syny).
    \end{equation*}

    This also applies to predicates --- we write \( \synx \sim \syny \) rather than \( \sim(\synx, \syny) \).

    \thmitem{rem:first_order_formula_conventions/negation} Negation of an infix binary predicate symbol \( \sim \) can be written more simply as
    \begin{equation*}
      \synx \not\sim \syny
    \end{equation*}
    rather than
    \begin{equation*}
      \synneg(\synx \sim \syny).
    \end{equation*}

    \thmitem{rem:first_order_formula_conventions/relativization}\mcite[def. 2.6.24]{Hinman2005Logic} For any predicate \( p \), to further shorten notation, we write
    \begin{equation*}
      \qforall {p(\synx, \cdots)} \varphi
    \end{equation*}
    as a shorthand for
    \begin{equation*}
      \mathbf{\qforall \synx (p(\synx, \cdots) \synimplies \varphi)}
    \end{equation*}
    \begin{equation*}
      \mathbf{\qforall x (p(x, \cdots) \synimplies \varphi)}
    \end{equation*}
    \begin{equation*}
      \qforall \synx (p(\synx, \cdots) \synimplies \varphi)
    \end{equation*}
    and
    \begin{equation*}
      \qexists {p(\synx, \cdots)} \varphi
    \end{equation*}
    as a shorthand for
    \begin{equation*}
      \qexists \synx (p(\synx, \cdots) \synwedge \varphi).
    \end{equation*}

    \Fullref{thm:relativized_first_order_quantifiers_are_dual} justifies this duality.

    This is called \term{relativization} of the quantifier and is useful when working with heterogeneous objects or even in \hyperref[ch:set_theory]{set theory}.

    \thmitem{rem:first_order_formula_conventions/exists_unique}\mcite[177]{Hinman2005Logic} We sometimes want to specify not only existence, but also uniqueness. This is the case in \eqref{eq:def:zfc/choice}, for example. It is conventional to write
    \begin{equation*}
      \qExists \synx \varphi
    \end{equation*}
    as a shorthand for
    \begin{equation*}
      \qexists \synx \parens[\Big]{ \varphi \synwedge \parens[\Big]{ \qforall \syny \varphi[\synx \mapsto \syny] \synimplies (\synx \syneq \syny) } },
    \end{equation*}
    where \( \varphi[\synx \mapsto \syny] \) is substitution defined in \fullref{def:first_order_substitution/term_in_formula}.

    \thmitem{rem:first_order_formula_conventions/necessary_signature} We only add to the language itself the functional and predicate symbols that are necessary for our desired axioms --- see \fullref{def:first_order_theory}. We can define additional functions and predicates in terms of these, but we avoid using them as much as possible when writing formulas in the object language. For example, we avoid adding the functional symbols \hyperref[thm:well_ordered_order_type_existence]{\( \ord(A) \)} and \hyperref[def:cardinal]{\( \card(A) \)} or even \hyperref[def:basic_set_operations/union]{\( \cup \)} and \hyperref[def:basic_set_operations/intersection]{\( \cap \)} to \hyperref[def:zfc]{\logic{ZFC}}.

    If needed, we can consider these new functions and predicates to be abbreviations for more verbose terms and formulas as described in \fullref{con:formula_defined_predicate}.
  \end{thmenum}
\end{remark}

\begin{definition}\label{def:first_order_structure}\mcite[def. 2.1.15]{Hinman2005Logic}
  Fix a first-order logic language \( \mscrL \). A \term{structure} for \( \mscrL \) is a pair \( \mscrX = (X, I) \), where
  \begin{thmenum}
    \thmitem{def:first_order_structure/set}\mcite[def. 2.1.15(i)]{Hinman2005Logic} \( X \) is a nonempty\fnote{See \fullref{rem:empty_first_order_structures} regarding special cases where we allow \( X \) to be empty.} set called the \term{domain} or \term{universe} of the structure \( \mscrX \).

    \thmitem{def:first_order_structure/interpretation} The \term{interpretation} \( I \) of the structure \( \mscrX \) is a \hyperref[def:function]{function} that is defined on the signature of \( \mscrL \) and satisfies the following conditions:
    \begin{thmenum}
      \thmitem{def:first_order_structure/interpretation/function}\mcite[def. 2.1.15(ii)]{Hinman2005Logic} For every \( n \)-ary function symbol \( f \), its interpretation is a \hyperref[def:function]{function} with signature \( I(f): X^n \to X \).

      \thmitem{def:first_order_structure/interpretation/predicate}\mcite[def. 2.1.15(iii)]{Hinman2005Logic} For every \( n \)-ary predicate \( p \), its interpretation is an n-ary \hyperref[def:boolean_function]{Boolean-valued function} with signature \( I(p): X^n \to \set{ T, F } \). A tuple \( (x_1, \ldots, x_n) \) satisfies \( p \) if \( p(x_1, \ldots, x_n) = T \).
    \end{thmenum}
  \end{thmenum}
\end{definition}
\begin{comments}
  \item Some authors like Peter Hinman in \incite[def. 2.1.15(iii)]{Hinman2005Logic} define \( I(p) \) to be a \hyperref[def:relation]{relation} \( I(p) \subseteq X^n \), however it is more convenient for us to work with Boolean-valued functions. The two approaches are equivalent as explained in \fullref{rem:boolean_valued_functions_and_predicates}.
  \item Unlike in the rest of this monograph, when dealing with first-order structures directly, it is important to distinguish between the structure \( \mscrX \) as a pair and its domain \( X \) as a set. See \fullref{rem:first_order_model_notation}.
\end{comments}

\begin{remark}\label{rem:empty_first_order_structures}
   If we allow the domain of a structure to be empty, we would have to reformulate a lot of important theorems (e.g. our proof of \fullref{thm:renaming_assignment_compatibility/formulas}), which would complicate compatibility between semantics and \hyperref[def:deduction_system]{deduction systems}.

   See \fullref{thm:intersection_substructure} and \fullref{thm:substructures_form_complete_lattice/bottom} for contexts where empty sets are justified as domains of first-order structures. We also use this in \fullref{rem:propositional_logic_as_first_order_logic}, although in the latter case the domain itself is irrelevant.
\end{remark}

\begin{definition}\label{def:first_order_valuation}
  Fix a structure \( \mscrX = (X, I) \) for a first-order logic language \( \mscrL \).

  \begin{thmenum}
    \thmitem{def:first_order_valuation/variable_assignment}\mcite[def. 2.1.16]{Hinman2005Logic} A \term[ru=оценка (\cite[77]{ШеньВерещагин2017Логика})]{variable assignment} for the variables of \( \mscrL \) is any function \( v: \op*{Var} \to X \).

    These are loosely similar to \hyperref[def:propositional_valuation/interpretation]{propositional interpretations}.

    \thmitem{def:first_order_valuation/modified_assignment}\mcite[92]{Hinman2005Logic} For every variable \( \synx \) and every domain element \( x \in X \) we also define the \term{modified assignment} at \( \synx \) with \( x \):
    \begin{equation*}
      v_{\synx \mapsto x}(\synz) \coloneqq \begin{cases}
        x,        &\synz = \synx, \\
        v(\synz), &\synz \neq \synx.
      \end{cases}
    \end{equation*}

    We can also modify the value at \( \synx \) with another variable, e.g.
    \begin{equation*}
      v_{\synx \mapsto \syny}(\synz) \coloneqq \begin{cases}
        v(\syny),  &\synz = \synx, \\
        v(\synz), &\synz \neq \synx.
      \end{cases}
    \end{equation*}

    Inductively,
    \begin{equation*}
      v_{\synx_1 \mapsto x_1, \ldots, \synx_n \mapsto x_n}(\syny) \coloneqq ((\ldots(v_{\synx_1 \mapsto x_1})\ldots)_{\synx_n \mapsto x_n})(\syny).
    \end{equation*}

    \thmitem{def:first_order_valuation/term_valuation}\mcite[def. 2.1.17]{Hinman2005Logic} The \term{valuation} of a term \( \tau \) is a value in the domain \( X \) given by
    \begin{equation}\label{eq:def:first_order_valuation/term_valuation}
      \Bracks{\tau}_v \coloneqq \begin{cases}
        v(\synx),                                                 &\tau = \synx \in \op*{Var}, \\
        I(f)(\Bracks{\kappa_1}_v, \ldots, \Bracks{\kappa_n}_v), &\tau = f(\kappa_1, \ldots, \kappa_n).
      \end{cases}
    \end{equation}

    \thmitem{def:propositional_valuation/term_valuation_function} Analogously to how we defined Boolean functions for propositional formulas in \fullref{def:propositional_valuation/valuation_function}, if \( \op*{Var}(\tau) \subseteq \set{ \synx_1, \ldots, \synx_n } \), the valuation \( \Bracks{\tau}_v \) only depends on the particular values \( v(\synx_1), \ldots, v(\synx_n) \). Hence, if the variables are clear from the context, we obtain a function
    \begin{equation*}
      \begin{aligned}
        &\Bracks{\tau}_\mscrX: X^n \to X, \\
        &\Bracks{\tau}_\mscrX(x_1, \ldots, x_n) \coloneqq \Bracks{\tau}_v,
      \end{aligned}
    \end{equation*}
    where \( v \) is any variable assignment such that \( v(\synx_k) = x_k \) for \( k = 1, \ldots, n \).

    \thmitem{def:first_order_valuation/formula_valuation}\mcite[def. 2.1.19, def. 2.1.22]{Hinman2005Logic} We extend the classical propositional valuations from \fullref{def:propositional_valuation}. The (classical) \term{valuation} of a formula \( \varphi \) is a \hyperref[con:boolean_value]{Boolean value} given by
    \begin{equation}\label{eq:def:first_order_valuation/formula_valuation}
      \Bracks{\varphi}_v \coloneqq \begin{cases}
        T,                                                                        &\varphi = \syntop, \\
        F,                                                                        &\varphi = \synbot, \\
        \Bracks{\tau_1}_v = \Bracks{\tau_2}_v,                                    &\varphi = \tau_1 \syneq \tau_2, \\
        I(p)(\Bracks{\tau_1}_v, \ldots, \Bracks{\tau_n}_v),                       &\varphi = p(\tau_1, \ldots, \tau_n), \\
        \oline{\Bracks{\psi}_v},                                               &\varphi = \synneg \psi, \\
        \Bracks{\psi_1}_v \relcirc \Bracks{\psi_2}_v,                    &\varphi = \psi_1 \syncirc \psi_2, {\syncirc} \in \op*{Conn}, \\
        \bigwedge\set[\Big]{ \Bracks{\psi}_{v_{\synx \mapsto x}} \given* x \in X }, &\varphi = \qforall \synx \psi, \\
        \bigvee\set[\Big]{ \Bracks{\psi}_{v_{\synx \mapsto x}} \given* x \in X },   &\varphi = \qexists \synx \psi.
      \end{cases}
    \end{equation}

    \thmitem{def:propositional_valuation/formula_valuation_function} We also define a function corresponding to a first-order formula, but we are only interested in the \hyperref[def:first_order_syntax/formula_free_variables]{free variables} rather than \hyperref[def:first_order_syntax/formula_variables]{all variables}:
    \begin{equation*}
      \begin{aligned}
        &\Bracks{\varphi}_\mscrX: X^n \to \set{ T, F }, \\
        &\Bracks{\varphi}_\mscrX(\synx_1, \ldots, \synx_n) \coloneqq \Bracks{\varphi}_v,
      \end{aligned}
    \end{equation*}
    where \( v \) is any variable assignment such that \( v(\synx_k) = x_k \) for \( k = 1, \ldots, n \).
  \end{thmenum}
\end{definition}
\begin{comments}
  \item The rules for evaluating constants, negations and connectives are a direct extension of the rules for propositional logic from \fullref{def:propositional_valuation/formula_valuation}.

  \item It is important that the domain is nonempty because \( \bigwedge\varnothing = T \), which directly contradicts our intent of defining \( \synexists \) as a quantifier for existence.

  \item Except for semantics of quantification, modified assignments are also used in other places like \fullref{thm:renaming_assignment_compatibility}.
\end{comments}

\begin{remark}\label{rem:propositional_logic_as_first_order_logic}
  We will show in details how, with certain limitations on the number of variables, propositional logic can be embedded into first-order logic.

  Fix a set \( \Gamma \) of \hyperref[def:propositional_syntax/formula]{propositional formulas}. Suppose that there is only a finite set of variables in \( \Gamma \); otherwise we would have to generalize first-order syntax as discussed in \fullref{rem:uncountable_first_order_language}. Consider the \hyperref[def:first_order_language]{first-order language} \( \mscrL \) with no functional symbols and a nullary predicate symbol for every propositional variable in \( \Gamma \). Denote by \( \widehat{P} \) the predicate symbol corresponding to the variable \( P \).

  Given the propositional formula \( \varphi \) from \( \Gamma \), let \( \widehat \varphi \) be the formula over \( \mscrL \) in which the variables are replaced with the predicate symbols from \( \mscrL \). Then \( \widehat \varphi \) is closed, and thus its interpretation does not depend on the variable assignment.

  Given an arbitrary propositional interpretation \( I \), let \( \widehat{I} \) be a first-order interpretation over the empty domain\fnote{See \fullref{rem:empty_first_order_structures} regarding empty domains.} \( \varnothing \) such that \( \widehat{I}(\widehat{P}) = I(P) \). Let \( \mscrX = (\varnothing, \widehat{I}) \). Then
  \begin{equation*}
    \Bracks{\widehat{P}}_\mscrX = \Bracks{\widehat{P}}_I.
  \end{equation*}
\end{remark}

\begin{definition}\label{def:first_order_equation}\mimprovised
  A \term{first-order equation} over some fixed \hyperref[def:first_order_language]{language} with \term{left side} the \hyperref[def:first_order_syntax/term]{term} \( \tau \) and \term{right side} \( \sigma \) is the \hyperref[def:first_order_syntax/formula]{formula} \( (\tau \syneq \sigma) \).

  Suppose that the \hyperref[def:first_order_syntax/formula_free_variables]{free variables} (which are actually all the variables) of this formula are among \( \synx_1, \ldots, \synx_n \). Then our goal is to find values \( x_1, \ldots, x_n \) in the domain of some \hyperref[def:first_order_structure]{structure} \( \mscrX \) such that
  \begin{equation*}
    \Bracks{\tau}_\mscrX(x_1, \ldots, x_n) = \Bracks{\sigma}_\mscrX(x_1, \ldots, x_n).
  \end{equation*}

  We call any such tuple \( (x_1, \ldots, x_n) \) in any structure \( \mscrX \) a \term{solution} of the equation.
\end{definition}

\begin{example}\label{ex:equations}
  A remarkable portion of mathematics concerns the study of different types of equations, and many of them are more general than \hyperref[def:first_order_equation]{equations in first-order logic}. The reason for this interest is that equations provide a simple way to specify rich semantic structure using simple syntactic objects.

  \begin{itemize}
    \item Matrix theory can be regarded as the study of linear equations. See \fullref{sec:matrices_over_rings} and \fullref{sec:matrices_over_fields}.
    \item Differential equations is aptly named since it studies equations in functional spaces concerning functions and their derivatives.
    \item Roots of generalized derivatives are studied in optimization. See \fullref{sec:nonsmooth_derivatives}.
    \item Fixed points of functions are studied in different branches of mathematics. See \fullref{thm:banach_fixed_point_theorem} or \fullref{thm:knaster_tarski_theorem}.
    \item \hyperref[def:algebraic_equation]{Algebraic equations} are useful across different branches of mathematics.
    \item \hyperref[def:recurrence_relations]{Recurrence relations} describe combinatorial sequences. See \fullref{sec:recurrence_relations}.
  \end{itemize}
\end{example}

\begin{remark}\label{rem:first_order_entailment_without_closed_formulas}
  \todo{Commutativity counterexample}
\end{remark}

\begin{definition}\label{def:first_order_semantics}
  Fix a first-order logic language \( \mscrL \). We introduce notions analogous to \hyperref[def:propositional_entailment]{propositional semantics}:
  \begin{thmenum}
    \thmitem{def:first_order_semantics/assignment_entailment}\mcite[def. 2.1.23(i)]{Hinman2005Logic} We say that the set of \hyperref[def:first_order_syntax/formula]{first-order formulas} \enquote{\( \Gamma \) \term{entails} \( \psi \) in the \hyperref[def:first_order_structure]{structure} \( \mscrX = (X, I) \) under the \hyperref[def:first_order_valuation/variable_assignment]{variable assignment} \( v \)} if, whenever \( \Bracks{\varphi}_v = T \) for all \( \varphi \in \Gamma \), it follows that \( \Bracks{\psi}_v = T \).

    \thmitem{def:first_order_semantics/entailment}\mcite[def. 2.2.1(iv)]{Hinman2005Logic} If \( \Gamma \) entails \( \psi \) under every variable assignment \( v \) in every structure \( \mscrX \), we simply say that \enquote{\( \Gamma \) entails \( \psi \)} and write \( \Gamma \vDash \psi \).

    As in the case of propositional formulas, we prefer the notation \( \gamma_1, \ldots, \gamma_n \vDash \psi \) to \( \set{ \gamma_1, \ldots, \gamma_n } \vDash \psi \).

    If \( \Gamma \) does not entail \( \varphi \) under at least one variable assignment, we write \( \Gamma \not\vDash \varphi \).

    As in the case of propositional formulas, we prefer the notation \( \gamma_1, \ldots, \gamma_n \vDash \psi \) to \( \set{ \gamma_1, \ldots, \gamma_n } \vDash \psi \).

    \thmitem{def:first_order_semantics/equivalence}\mcite[def. 2.2.1(ii)]{Hinman2005Logic} As in the case with \hyperref[def:semantic_equivalence]{propositional semantical equivalence}, we say that \( \varphi \) and \( \psi \) are \term{semantically equivalent} and write \( \varphi \gleichstark \psi \) if both \( \varphi \vDash \psi \) and \( \psi \vDash \varphi \).

    \thmitem{def:first_order_semantics/validity}\mcite[def. 2.2.1(i)]{Hinman2005Logic} We say that the formula \( \varphi \) is \term{universally valid} if \( \vDash \varphi \).
  \end{thmenum}
\end{definition}
\begin{comments}
  \item It is actually important that we haven't defined entailment within a structure, because it can only be well-defined for \hyperref[def:first_order_syntax/closed_formula]{closed formulas} for reasons outlined in \fullref{rem:first_order_satisfiability_bivalence}.

  \item The concept of universal validity is more general than \hyperref[def:propositional_tautology]{propositional tautologies} in the sense that we define validity for arbitrary formulas that may or may not be closed.

  \item \Fullref{thm:intuitionistic_equivalences} hold for first-order entailment as well.
\end{comments}

\begin{definition}\label{def:first_order_substitution}
  As in \hyperref[sec:propositional_logic]{propositional logic}, we sometimes want to perform substitution, however we have different types of syntactic objects (terms and formulas) which have different substitution rules. The notion of free and bound variables further complicates us --- see for example the problems outlined in \fullref{rem:first_order_substitution_renaming_justification}. It is not only difficult, but also it is of no practical use to define substitution of a first-order subformula inside another formula as it is done in \fullref{def:propositional_substitution}. Instead, we concern ourselves with substituting variables --- propositional variables with first-order formulas and first-order variables with first-order terms. Furthermore, since this does not complicate us, we allow substituting arbitrary terms rather than only first-order variables.

  While substituting a propositional variable is the syntactic analog to applying \hyperref[def:boolean_function]{Boolean functions} to different variables or propositional constants, substituting a first-order variable can express applying \hyperref[def:function]{arbitrary functions} to different first-order variables or arbitrary constants. For example, in a suitable language, we can apply \( \log(x) \) to the constant \( e \) by substituting \( x \) with \( e \) to obtain the closed term \( \log(e) \).

  Where applicable, \term{simultaneous substitution} is defined via the same trick as in \fullref{def:propositional_substitution}.

  \begin{thmenum}
    \thmitem{def:first_order_substitution/propositional} Let \( \varphi \) be a \hyperref[def:propositional_syntax/formula]{propositional formula} with variables \( \op*{Var}(\varphi) = \set{ P_1, \ldots, P_n } \). For brevity, denote \( V \coloneqq \op*{Var}(\varphi) \). Let \( \Theta = \set{ \theta_1, \ldots, \theta_n } \) be a set of \hyperref[def:first_order_syntax/formula]{first-order formulas}.

    It does not make sense to replace a single propositional variable with a single formula. Furthermore, a first-order formula \( \theta_k \) cannot possibly contain any of the propositional variables \( P_1, \ldots, P_n \). This allows us to introduce a simplification of the simultaneous substitution based on \eqref{eq:def:propositional_substitution/single} as follows:
    \begin{equation}\label{eq:def:first_order_substitution/propositional}
      \varphi[V \mapsto \Theta] \coloneqq \begin{cases}
        \varphi,                                                    &\varphi \in \set{ \syntop, \synbot } \\
        \theta_k,                                                   &\varphi = \theta_k \T{for some} k = 1, \ldots, n \\
        \synneg \psi[V \mapsto \Theta],                                &\varphi = \synneg \psi \\
        \psi_1[V \mapsto \Theta] \syncirc \psi_2[V \mapsto \Theta], &\varphi = \psi_1 \syncirc \psi_2, {\syncirc} \in \op*{Conn}.
      \end{cases}
    \end{equation}

    As in \fullref{def:propositional_substitution}, it is not strictly necessary for any of the variables to belong to \( \op*{Var}(\varphi) \).

    \thmitem{def:first_order_substitution/term_in_term}\mcite[def. 2.2.19]{Hinman2005Logic} We define the substitution of the \hyperref[def:first_order_syntax/term]{first-order term} \( \kappa \) with \( \mu \) in the term \( \tau \) as follows:
    \begin{equation}\label{eq:def:first_order_substitution/term_in_term}
      \tau[\kappa \mapsto \mu] \coloneqq \begin{cases}
        \mu,                                                               &\tau = \kappa, \\
        \tau,                                                              &\tau \neq \kappa \T{and} \tau \in \op*{Var}, \\
        f(\rho_1[\kappa \mapsto \mu], \ldots, \rho_n[\kappa \mapsto \mu]), &\tau \neq \kappa \T{and} \tau = f(\rho_1, \ldots, \rho_n).
      \end{cases}
    \end{equation}

    It is not strictly necessary for \( \kappa \) to be a \hyperref[def:first_order_syntax/subterm]{subterm} of \( \tau \).

    \thmitem{def:first_order_substitution/term_in_formula}\mimprovised This last case is more complicated. We define the substitution of the term \( \kappa \) with the term \( \mu \) in the first-order formula \( \varphi \) as follows:
    \begin{equation}\label{eq:def:first_order_substitution/term_in_formula}
      \varphi[\kappa \mapsto \mu] \coloneqq \begin{cases}
        \varphi,                                                           &\varphi \in \set{ \syntop, \synbot }, \\
        p(\tau_1[\kappa \mapsto \mu], \ldots, \tau_n[\kappa \mapsto \mu]), &\varphi = p(\tau_1, \ldots, \tau_n), \\
        \tau_1[\kappa \mapsto \mu] \syneq \tau_2[\kappa \mapsto \mu],      &\varphi = \tau_1 \syneq \tau_2, \\
        \synneg \psi[\kappa \mapsto \mu],                                     &\varphi = \synneg \psi, \\
        \psi_1[\kappa \mapsto \mu] \syncirc \psi_2[\synx \mapsto \mu],       &\varphi = \psi_1 \syncirc \psi_2, {\syncirc} \in \op*{Conn}, \\
        (\dagger),                                                         &\varphi = \quantifier Q {\synx} \psi, Q \in \op*{Quan},
      \end{cases}
    \end{equation}
    where
    \begin{empheq}[left=(\dagger) \coloneqq \empheqlbrace]{align}
      &\varphi,                                                                        && \synx \in \op*{Var}(\kappa), \label{eq:def:first_order_substitution/term_in_formula/quantifiers/trivial} \\
      &\quantifier Q {\synx} \parens[\Big]{\psi[\kappa \mapsto \mu]},                    && \synx \not\in \op*{Var}(\kappa) \cup \op*{Var}(\mu), \label{eq:def:first_order_substitution/term_in_formula/quantifiers/direct} \\
      &\quantifier Q {\syny} \parens[\Big]{\psi[\synx \mapsto \syny][\kappa \mapsto \mu]}, && \synx \not\in \op*{Var}(\kappa) \T{and} \synx \in \op*{Var}(\mu) \T{and} &\label{eq:def:first_order_substitution/term_in_formula/quantifiers/renaming} \\
                                                                                      &&& \syny \not\in \op*{Var}(\kappa) \cup \op*{Var}(\mu) \cup \op*{Free}(\psi). \nonumber
    \end{empheq}
  \end{thmenum}
\end{definition}
\begin{comments}
  \item In \eqref{eq:def:first_order_substitution/term_in_formula/quantifiers/renaming} we chose a fresh variable \( \syny \). As discussed in \fullref{def:variable_identifier}, we can use the well-ordering on the set \( \op*{Var} \) of variables and deterministically choose \( \syny \) to be the smallest variable not present in \( \varphi \). This rule is called \term{renaming of the bound variables} \( \synx \) to \( \syny \) and is done to mitigate capturing as described in \fullref{rem:first_order_substitution_renaming_justification/capturing}.

  \item We could avoid the rule for renaming, however renaming both free and bound variables is natural and is often done in practice (i.e. outside formal logic). For example, consider the \hyperref[def:peano_arithmetic]{Peano arithmetic} formula \enquote{there exists \( n \) such that \( nm \) is even}. Note that the bound variable \( n \) is renamed to \( k \) and the free variable \( m \) to \( n \) in the larger formula \enquote{for every \( n \) there exists \( k \) such that \( kn \) is even}.

  \item The rule \eqref{eq:def:first_order_substitution/term_in_formula/quantifiers/trivial} may seem redundant, but when doing inductive proofs (e.g. our proof of \fullref{thm:renaming_assignment_compatibility}), we usually need to separately consider the cases where \( \synx \in \op*{Var}(\kappa) \) and \( \synx \not\in \op*{Var}(\kappa) \setminus \op*{Var}(\mu) \) and the rule \eqref{eq:def:first_order_substitution/term_in_formula/quantifiers/direct} being trivial simplifies the proofs.

  \item See \fullref{rem:first_order_substitution_parentheses} regarding the additional parentheses in \eqref{eq:def:first_order_substitution/term_in_formula/quantifiers/renaming}.

  \item See \fullref{ex:first_order_substitution} for examples of applying the different quantifier rules.

  \item We use the shorthand convention from \fullref{rem:substitution_notation}.

  \item These algorithms can be found as \identifier{logic.substitution.substitute_in_term} and \identifier{logic.substitution.substitute_in_formula} in \cite{notebook:code}.
\end{comments}

\begin{remark}\label{rem:first_order_substitution_renaming_justification}
  The renaming rule \eqref{eq:def:first_order_substitution/term_in_formula/quantifiers/renaming} is designed to mitigate the following two problems when compared to \eqref{eq:def:first_order_substitution/term_in_formula/quantifiers/direct}:

  \begin{thmenum}
    \thmitem{rem:first_order_substitution_renaming_justification/capturing} Renaming mitigates \enquote{capturing} free variables as in
    \begin{equation*}
      \parens[\Big]{ \qforall \syny p(\synx, \syny) }[\synx \mapsto \syny] = \qforall \syny p(\syny, \syny)
    \end{equation*}
    by instead producing, up to a choice of fresh variables, the formula
    \begin{equation*}
      \parens[\Big]{ \qforall \syny p(\synx, \syny) }[\synx \mapsto \syny] = \qforall \synz p(\syny, \synz).
    \end{equation*}

    \thmitem{rem:first_order_substitution_renaming_justification/colliding} Renaming mitigates \enquote{colliding} multiple bound variables as in
    \begin{equation*}
      \parens[\Big]{ \qforall \synx \qforall \syny p(\synx, \syny) }[\synx \mapsto \syny] = \qforall \synx \qforall \syny p(\syny, \syny)
    \end{equation*}
    by instead producing, up to a choice of fresh variables, the formula
    \begin{equation*}
      \parens[\Big]{ \qforall \synx \qforall \syny p(\synx, \syny) }[\synx \mapsto \syny] = \qforall \synz \qforall \sigma p(\synz, \sigma).
    \end{equation*}
  \end{thmenum}
\end{remark}

\begin{remark}\label{rem:first_order_substitution_parentheses}
  When performing \hyperref[def:first_order_substitution]{substitution}, it is sometimes convenient to add additional parentheses to avoid ambiguity. For example, while parentheses around quantifier expressions are not necessary by the syntax of first-order logic, adding such parentheses helps avoid the ambiguity in
  \begin{equation*}
    \qforall \synx p(\synx, \syny) [\syny \mapsto \synz].
  \end{equation*}

  Instead, we either write
  \begin{equation*}
    \parens[\Big]{ \qforall \synx p(\synx, \syny) } [\syny \mapsto \synz]
  \end{equation*}
  or
  \begin{equation*}
    \qforall \synx \parens[\Big]{ p(\synx, \syny)[\syny \mapsto \synz] }.
  \end{equation*}

  This convention is only part of the metasyntax and the parentheses are not part of the syntax of the formulas themselves.
\end{remark}

\begin{example}\label{ex:first_order_substitution}
  The following term substitutions should justify the distinct cases in \eqref{eq:def:first_order_substitution/term_in_formula}:
  \begin{thmenum}
    \thmitem{ex:first_order_substitution/1} The trivial case without actual substitution:
    \begin{equation*}
      \parens[\Big]{\qforall \synx p(\synx, \syny)}[\synx \mapsto \syny]
      \reloset {\eqref{eq:def:first_order_substitution/term_in_formula/quantifiers/trivial}} =
      \qforall \synx p(\synx, \syny).
    \end{equation*}

    \Fullref{ex:first_order_substitution/5} demonstrates that this does not work for nested substitution.

    \thmitem{ex:first_order_substitution/2} A simple substitution without renaming:
    \begin{align*}
      &\phantom{{}={}}
      \parens[\Big]{\qforall \synx p(\synx, \syny)}[\syny \mapsto \synz]
      \reloset {\eqref{eq:def:first_order_substitution/term_in_formula/quantifiers/direct}} = \\ &=
      \qforall \synx \parens[\Big]{p(\synx, \syny)[\syny \mapsto \synz]}
      \reloset {\eqref{eq:def:first_order_substitution/term_in_formula}} = \\ &=
      \qforall \synx p(\synx, \synz).
    \end{align*}

    \thmitem{ex:first_order_substitution/3} A simple renaming without actual substitution:
    \begin{align*}
      &\phantom{{}={}}
      \parens[\Big]{\qforall \synx p(\synx, \syny)}[\syny \mapsto \synx]
      \reloset {\eqref{eq:def:first_order_substitution/term_in_formula/quantifiers/renaming}} = \\ &=
      \qforall \synz \parens[\Big]{p(\synx, \syny)[\synx \mapsto \synz][\syny \mapsto \synx]}
      \reloset {\eqref{eq:def:first_order_substitution/term_in_formula}} = \\ &=
      \qforall \synz p(\synz, \synx).
    \end{align*}

    \thmitem{ex:first_order_substitution/4} \Fullref{ex:first_order_substitution/3}, but with \( \mu \) in \eqref{eq:def:first_order_substitution/term_in_formula} containing \( \synx \) indirectly:
    \begin{align*}
      &\phantom{{}={}}
      \parens[\Big]{\qforall \synx p(\synx, \syny)}[\syny \mapsto f(\synx)]
      \reloset {\eqref{eq:def:first_order_substitution/term_in_formula/quantifiers/renaming}} = \\ &=
      \qforall \synz \parens[\Big]{p(\synx, \syny)[\synx \mapsto \synz][\syny \mapsto f(\synx)]}
      \reloset {\eqref{eq:def:first_order_substitution/term_in_formula}} = \\ &=
      \qforall \synz p(\synz, f(\synx)).
    \end{align*}

    \thmitem{ex:first_order_substitution/5} Only renaming with multiple quantifiers which shows the limitations of \eqref{eq:def:first_order_substitution/term_in_formula/quantifiers/trivial} as a way to avoid unnecessary renaming:
    \begin{align*}
      &\phantom{{}={}}
      \parens[\Big]{\qforall \syny \qexists \synx p(\synx, \syny)}[\synx \mapsto \syny]
      \reloset {\eqref{eq:def:first_order_substitution/term_in_formula/quantifiers/renaming}} = \\ &=
      \qforall \synz \parens*{ \parens[\Big]{ \qforall \synx p(\synx, \syny) }[\syny \mapsto \synz][\synx \mapsto \syny]}
      \reloset {\eqref{eq:def:first_order_substitution/term_in_formula}} = \\ &=
      \qforall \synz \parens*{ \parens[\Big]{ \qforall \synx p(\synx, \synz) }[\synx \mapsto \syny]}
      \reloset {\eqref{eq:def:first_order_substitution/term_in_formula/quantifiers/trivial}} = \\ &=
      \qforall \synz \qforall \synx p(\synx, \synz).
    \end{align*}

    \thmitem{ex:first_order_substitution/6} Both renaming and substitution with multiple quantifiers:
    \begin{align*}
      &\phantom{{}={}}
      \parens[\Big]{ \qforall \syny (p(\synx, \syny) \synvee \qforall \synx p(\synx, \syny)) }[\synx \mapsto \syny]
      \reloset {\eqref{eq:def:first_order_substitution/term_in_formula/quantifiers/renaming}} = \\ &=
      \qforall \synz \parens*{ \parens[\Big]{ p(\synx, \syny) \synvee \qexists \synx p(\synx, \syny) }[\syny \mapsto \synz][\synx \mapsto \syny] }
      \reloset {\eqref{eq:def:first_order_substitution/term_in_formula}} = \\ &=
      \qforall \synz \parens*{ p(\syny, \synz) \synvee \parens[\Big]{ \qexists \synx p(\synx, \syny) }[\syny \mapsto \synz][\synx \mapsto \syny] }
      \reloset {\eqref{eq:def:first_order_substitution/term_in_formula/quantifiers/direct}} = \\ &=
      \qforall \synz \parens*{ p(\syny, \synz) \synvee \parens[\Big]{ \qexists \synx p(\synx, \synz) }[\synx \mapsto \syny] }
      \reloset {\eqref{eq:def:first_order_substitution/term_in_formula/quantifiers/trivial}} = \\ &=
      \parens[\Big]{ \qforall \synz p(\syny, \synz) } \synvee \parens[\Big]{ \qexists \synx p(\synx, \synz) }.
    \end{align*}

    \thmitem{ex:first_order_substitution/7} Substitution of more general terms than variables with renaming of term's variables:
    \begin{align*}
      &\phantom{{}={}}
      \parens*{\qforall \synx p(\synx, \syny, f(\syny))}[f(\syny) \mapsto g(\syny, \synx)]
      \reloset {\eqref{eq:def:first_order_substitution/term_in_formula/quantifiers/renaming}} = \\ &=
      \qforall \synz \parens[\Big]{p(\synx, \syny, f(\syny))[\synx \mapsto \synz][f(\syny) \mapsto g(\syny, \synx)]}
      \reloset {\eqref{eq:def:first_order_substitution/term_in_formula}} = \\ &=
      \qforall \synz p(\synz, \syny, g(\syny, \synx)).
    \end{align*}
  \end{thmenum}
\end{example}

\begin{proposition}\label{thm:renaming_assignment_compatibility}
  We will show how \hyperref[rem:first_order_substitution_renaming_justification]{syntactic renaming} is compatible with a certain \enquote{semantic renaming}.

  Fix a \hyperref[def:first_order_language]{first-order language} \( \mscrL \), a \hyperref[def:first_order_structure]{structure} \( \mscrX = (X, I) \) on \( \mscrL \) and a \hyperref[def:first_order_valuation/variable_assignment]{variable assignment} \( v \) in \( \mscrX \).

  \begin{thmenum}
    \thmitem{thm:renaming_assignment_compatibility/terms} For any term \( \tau \) and any two variables \( \synx \) and \( \syny \), we have
    \begin{equation}\label{eq:thm:renaming_assignment_compatibility/terms}
      \Bracks{\tau}_{v_{\synx \mapsto \syny}}
      =
      \Bracks[\Big]{\tau[\synx \mapsto \syny]}_v.
    \end{equation}

    \thmitem{thm:renaming_assignment_compatibility/formulas} For any formula \( \varphi \), any variable \( \synx \) and any other variable \( \syny \) not in \( \op*{Var}(\varphi) \) we have
    \begin{equation}\label{eq:thm:renaming_assignment_compatibility/formulas}
      \Bracks{\varphi}_{v_{\synx \mapsto \syny}}
      =
      \Bracks[\Big]{\varphi[\synx \mapsto \syny]}_v.
    \end{equation}
  \end{thmenum}
\end{proposition}
\begin{proof}
  In both cases, we use \fullref{thm:induction_on_rooted_trees} on the definition of the substitution.

  \SubProofOf{thm:renaming_assignment_compatibility/terms}

  \begin{itemize}
    \item If \( \tau = \synx \), then
    \begin{equation*}
      \tau[\synx \mapsto \syny] = \synx[\synx \mapsto \syny] = \syny
    \end{equation*}
    and \eqref{eq:thm:renaming_assignment_compatibility/terms} follows directly.

    \item If \( \tau \) is a variable and \( \tau \neq \synx \), then
    \begin{equation*}
      \tau[\synx \mapsto \syny] = \tau
    \end{equation*}
    and \eqref{eq:thm:renaming_assignment_compatibility/terms} again holds trivially.

    \item If \( \tau = f(\kappa_1, \ldots, \kappa_n) \) and if the inductive hypothesis holds for \( \kappa_1, \ldots, \kappa_n \), then
    \begin{balign*}
      \Bracks[\Big]{ \tau[\synx \mapsto \syny] }_v
      &=
      \Bracks[\Big]{ f(\kappa_1[\synx \mapsto \syny], \ldots, \kappa_n[\synx \mapsto \syny]) }_v
      = \\ &=
      I(f) \parens[\Bigg]{ \Bracks[\Big]{ \kappa_1[\synx \mapsto \syny] }_v, \ldots, \Bracks[\Big]{ \kappa_1[\synx \mapsto \syny] }_v }
      \reloset {\T{ind.}} = \\ &=
      I(f) \parens[\Big]{ \Bracks{\kappa_1}_{v_{\synx \mapsto \syny}}, \ldots, \Bracks{\kappa_n}_{v_{\synx \mapsto \syny}} }
      = \\ &=
      \Bracks[\Big]{ f(\kappa_1, \ldots, \kappa_n) }_{\synx \mapsto \syny}
      = \\ &=
      \Bracks{\tau}_{\synx \mapsto \syny}.
    \end{balign*}
  \end{itemize}

  In all cases, \eqref{eq:thm:renaming_assignment_compatibility/terms} holds.

  \SubProofOf{thm:renaming_assignment_compatibility/formulas}
  \hfill
  \begin{itemize}
    \item If \( \varphi \in \set{ \syntop, \synbot } \), then \( \varphi \) has no subterms and thus \eqref{eq:thm:renaming_assignment_compatibility/formulas} holds vacuously.

    \item If \( \varphi = p(\tau_1, \ldots, \tau_n) \), then since \eqref{eq:thm:renaming_assignment_compatibility/terms} holds for all \( \tau_k \), we have
    \begin{equation*}
      \Bracks[\Big]{ \tau_k[\synx \mapsto \syny] }_v = \Bracks{\tau_k}_{v_{\synx \mapsto \syny}}
    \end{equation*}
    and thus
    \begin{equation*}
      I(p)\parens[\Big]{ \Bracks[\Big]{ \tau_1[\synx \mapsto \syny] }_v, \ldots, \Bracks[\Big]{ \tau_1[\synx \mapsto \syny] }_v }
      \reloset {\T{ind.}} =
      I(p)\parens[\Big]{ \Bracks{\tau_1}_{v_{\synx \mapsto \syny}}, \ldots, \Bracks{\tau_n}_{v_{\synx \mapsto \syny}} }.
    \end{equation*}

    Therefore,
    \begin{balign*}
      \Bracks[\Big]{ \varphi[\synx \mapsto \syny] }_v
      &=
      \Bracks[\Big]{ p(\tau_1[\synx \mapsto \syny], \ldots, \tau_n[\synx \mapsto \syny]) }_v
      = \\ &=
      \Bracks{ p(\tau_1, \ldots, \tau_n) }_{v_{\synx \mapsto \syny}}
      = \\ &=
      \Bracks{\varphi}_{v_{\synx \mapsto \syny}}.
    \end{balign*}

    \item The case \( \varphi = \tau_1 \syneq \tau_2 \) is proved analogously.

    \item The cases \( \varphi = \synneg \psi \) and \( \varphi = \psi_1 \syncirc \psi_2 \) are proved in an straightforward manner.

    \item Let \( \varphi = \qforall \synz \psi \), where the inductive hypothesis holds for \( \psi \). We consider three cases
    \begin{itemize}
      \item Suppose that \( \synz = \synx \). By definition, we have
      \begin{equation*}
        \varphi[\synx \mapsto \syny]
        =
        \varphi,
      \end{equation*}
      hence \eqref{eq:thm:renaming_assignment_compatibility/formulas} holds trivially.

      \item Suppose that \( \synz \neq \synx \). It follows that
      \begin{equation*}
        \varphi[\synx \mapsto \syny]
        =
        \qforall \synz \parens[\Big]{ \psi[\synx \mapsto \syny] }.
      \end{equation*}

      \begin{itemize}
        \item If \( \Bracks[\Big]{\varphi[\synx \mapsto \syny]}_v = T \), by definition of \hyperref[def:first_order_valuation/formula_valuation]{quantifier formula valuation}, for any \( x \in X \) we have
        \begin{equation}\label{eq:thm:renaming_assignment_compatibility/formulas/true_modified_assignment}
          \underbrace{\Bracks[\Big]{\qforall \synz \parens[\Big]{ \psi[\synx \mapsto \syny] } }_v}_{\varphi[\synx \mapsto \syny]}
          =
          \Bracks[\Big]{\psi[\synx \mapsto \syny]}_{v_{\synz \mapsto x}}
          =
          T.
        \end{equation}

        On the other hand, by the inductive hypothesis,
        \begin{equation*}
          \Bracks[\Big]{ \psi[\synx \mapsto \syny] }_v = \Bracks{\psi}_{v_{\synx \mapsto \syny}}
        \end{equation*}
        and, as a special case, for any \( x \in X \),
        \begin{equation}\label{eq:thm:renaming_assignment_compatibility/formulas/ind_hyp_modified_assignment}
          \Bracks[\Big]{ \psi[\synx \mapsto \syny] }_{v_{\synz \mapsto x}} = \Bracks{\psi}_{v_{\synx \mapsto \syny, \synz \mapsto x}}.
        \end{equation}

        Combining \eqref{eq:thm:renaming_assignment_compatibility/formulas/true_modified_assignment} and \eqref{eq:thm:renaming_assignment_compatibility/formulas/ind_hyp_modified_assignment}, we obtain
        \begin{equation*}
          \Bracks[\Big]{ \varphi[\synx \mapsto \syny] }_v
          =
          \Bracks[\Big]{ \psi[\synx \mapsto \syny] }_{v_{\synz \mapsto x}}
          =
          \underbrace{\Bracks{\psi}_{v_{\synx \mapsto \syny, \synz \mapsto x}}}_{T \T*{for all} x \in X}
          =
          \Bracks{\varphi}_{v_{\synx \mapsto \syny}},
        \end{equation*}
        which proves the case.

        \item If \( \Bracks[\Big]{\varphi[\synx \mapsto \syny]}_v = F \), then there exists \( x \in X \) such that
        \begin{equation*}
          \Bracks[\Big]{\psi[\synx \mapsto \syny]}_{v_{\synz \mapsto x}} = F.
        \end{equation*}

        Since \eqref{eq:thm:renaming_assignment_compatibility/formulas/ind_hyp_modified_assignment} holds by the inductive hypothesis, we have
        \begin{equation*}
          \Bracks{\psi}_{v_{\synx \mapsto \syny, \synz \mapsto x}} = F
        \end{equation*}
        for the same \( x \).

        It follows that \( \Bracks{\varphi}_{v_{\synx \mapsto \syny}} = F \), which proves the case.
      \end{itemize}
    \end{itemize}

    \item We can prove the case \( \varphi = \qexists \synz \psi \) using double negation on the previous case.
  \end{itemize}

  In all cases, \eqref{eq:thm:renaming_assignment_compatibility/formulas} holds.
\end{proof}

\begin{proposition}\label{thm:first_order_substitution_equivalence}
  Analogously to \fullref{thm:propositional_substitution_equivalence}, we will show that all substitutions defined in \fullref{def:first_order_substitution} types of substitution preserve the corresponding \hyperref[def:first_order_semantics]{semantics}.

  By induction, this proposition also holds for \hyperref[def:propositional_substitution]{simultaneous substitution}.

  Fix a \hyperref[def:first_order_structure]{structure} \( \mscrX = (X, I) \) and a \hyperref[def:first_order_valuation/variable_assignment]{variable assignment} \( v \).

  \begin{thmenum}
    \thmitem{thm:first_order_substitution_equivalence/propositional} As in \fullref{def:first_order_substitution/propositional}, let \( \varphi \) be a \hyperref[def:propositional_syntax/formula]{propositional formula} with variables \( {V = \set{ P_1, \ldots, P_n }} \) and let \( \Theta = \set{ \theta_1, \ldots, \theta_n } \) be a set of \hyperref[def:first_order_syntax/formula]{first-order formulas}.

    Furthermore, let \( J \) be a \hyperref[def:propositional_valuation/interpretation]{propositional interpretation} such that, for all \( k = 1, \ldots, n \),
    \begin{equation}\label{eq:thm:first_order_substitution_equivalence/propositional/compatibility}
      \Bracks{P_k}_J = \Bracks{\theta_k}_v.
    \end{equation}

    Then
    \begin{equation}\label{eq:thm:first_order_substitution_equivalence/propositional}
      \Bracks[\Big]{ \varphi[V \mapsto \Theta] }_v = \Bracks{\varphi}_J.
    \end{equation}

    In particular, \( \vDash \varphi \) in the sense of \fullref{def:propositional_tautology} implies \( \vDash \varphi[V \mapsto \Theta] \) in the sense of \fullref{def:first_order_semantics/validity}.

    \thmitem{thm:first_order_substitution_equivalence/term_in_term} Let \( \tau \) be a \hyperref[def:first_order_syntax/term]{first-order term} and let \( \kappa \) be a \hyperref[def:first_order_syntax/subterm]{subterm} of \( \tau \). Let \( \mu \) be another term such that
    \begin{equation}\label{eq:thm:first_order_substitution_equivalence/term_in_term/compatibility}
      \Bracks{\mu}_v = \Bracks{\kappa}_v.
    \end{equation}

    Then
    \begin{equation}\label{eq:thm:first_order_substitution_equivalence/term_in_term}
      \Bracks[\Big]{\tau[\kappa \mapsto \mu]}_v = \Bracks{\tau}_v.
    \end{equation}

    \thmitem{thm:first_order_substitution_equivalence/term_in_formula} Let \( \varphi \) be a \hyperref[def:first_order_syntax/formula]{first-order formula} and let \( \kappa \) be a \hyperref[def:first_order_syntax/formula_terms]{term} of \( \varphi \). Let \( \mu \) be another term such that
    \begin{equation}\label{eq:thm:first_order_substitution_equivalence/term_in_formula/compatibility}
      \Bracks{\mu}_v = \Bracks{\kappa}_v.
    \end{equation}

    Then
    \begin{equation}\label{eq:thm:first_order_substitution_equivalence/term_in_formula}
      \Bracks{\varphi[\kappa \mapsto \mu]}_v = \Bracks{\varphi}_v.
    \end{equation}
  \end{thmenum}
\end{proposition}
\begin{proof}
  In all cases, we use \fullref{thm:induction_on_rooted_trees} by the definition of the substitution. The inductive hypothesis for a formula is that the proposition holds for arbitrary substitutions and valuations.

  \SubProofOf{thm:first_order_substitution_equivalence/propositional} Let \( \varphi \) be a propositional formula.
  \begin{itemize}
    \item If \( \varphi \in \set{ \syntop, \synbot } \), no substitution is performed and thus \eqref{eq:thm:first_order_substitution_equivalence/propositional} holds trivially.

    \item If \( \varphi = P_k \) for some \( k = 1, \ldots, n \), then follows \eqref{eq:thm:first_order_substitution_equivalence/propositional} from \eqref{eq:thm:first_order_substitution_equivalence/propositional/compatibility}.

    \item If \( \varphi = \synneg \psi \) and if the inductive hypothesis holds for \( \psi \), then
    \begin{equation*}
      \Bracks[\Big]{ \psi[V \mapsto \Theta] }_v
      =
      \oline{\Bracks[\Big]{ \psi[V \mapsto \Theta] }_v}
      \reloset {\T{ind.}} =
      \oline{\Bracks{\psi}_J}
      =
      \Bracks{\varphi}_J.
    \end{equation*}

    \item If \( \varphi = \psi_1 \syncirc \psi_2, {\syncirc} \in \op*{Conn} \) and if the inductive hypothesis holds for both \( \psi_1 \) and \( \psi_2 \), then
    \begin{equation*}
      \Bracks[\Big]{ \psi[V \mapsto \Theta] }_v
      =
      \Bracks[\Big]{ \psi_1[V \mapsto \Theta] }_v \relcirc \Bracks[\Big]{ \psi_2[V \mapsto \Theta] }_v
      \reloset {\T{ind.}} =
      \Bracks{\psi_1}_J \relcirc \Bracks{\psi_2}_J
      =
      \Bracks{\varphi}_J.
    \end{equation*}
  \end{itemize}

  In all cases, \eqref{eq:thm:first_order_substitution_equivalence/propositional} holds.

  \SubProofOf{thm:first_order_substitution_equivalence/term_in_term} The proof is identical to that of \fullref{thm:renaming_assignment_compatibility/terms}.

  \SubProofOf{thm:first_order_substitution_equivalence/term_in_formula} The proof is identical to that of \fullref{thm:renaming_assignment_compatibility/formulas} except for the special cases where \hyperref[rem:first_order_substitution_renaming_justification]{renaming} occurs, i.e. \( \varphi = \qforall \synx \psi \) and \( \varphi = \qexists \synx \psi \), where
  \begin{itemize}
    \item \( \synx \in \op*{Free}(\mu) \).
    \item \( \syny \not\in \op*{Var}(\kappa) \cup \op*{Var}(\mu) \cup \op*{Var}(\psi) \).
    \item The inductive hypothesis holds for \( \psi \).
  \end{itemize}

  We will only show the case \( \varphi = \qforall \synx \psi \) since the existential case is handled similarly. Since \( \synx \in \op*{Free}(\mu) \), we have
  \begin{equation*}
    \varphi[\kappa \mapsto \mu]
    =
    \qforall \syny \parens[\Big]{ \psi[\synx \mapsto \syny][\kappa \mapsto \mu] },
  \end{equation*}
  which does not allow us to use the inductive hypothesis directly.

  We proceed to prove the statement by nested induction on the number of quantifiers. We have already shown the case of \( 0 \) quantifiers. Suppose that the statement holds for all formulas with strictly less than \( n \) quantifiers and suppose that \( \varphi \) has exactly \( n \) quantifiers.

  Furthermore, for formulas with \( n \) quantifiers with \( \synforall \) as the outermost one, the non-renaming cases \eqref{eq:def:first_order_substitution/term_in_formula/quantifiers/trivial} and \eqref{eq:def:first_order_substitution/term_in_formula/quantifiers/direct} hold. Therefore, since \( \syny \not\in \op*{Free}(\mu) \),
  \begin{equation}\label{eq:thm:first_order_substitution_equivalence/term_in_formula/nested_induction}
    \begin{aligned}
      &\phantom{{}={}}
      \Bracks{\varphi[\kappa \mapsto \mu]}_v
      = \\ &=
      \Bracks[\Big]{\qforall \syny \parens[\Big]{ \psi[\synx \mapsto \syny][\kappa \mapsto \mu] }}_v
      \reloset {\eqref{eq:def:first_order_substitution/term_in_formula/quantifiers/direct}} = \\ &=
      \Bracks[\Big]{ \parens*{ \qforall \syny \parens[\Big]{ \psi[\synx \mapsto \syny] } }[\kappa \mapsto \mu] }_v
      \reloset {\T{ind.}} = \\ &=
      \Bracks[\Big]{\qforall \syny \parens[\Big]{ \psi[\synx \mapsto \syny] } }_v,
    \end{aligned}
  \end{equation}
  where we have implicitly used that \( \psi \) has \( n - 1 \) quantifiers.

  On the other hand, due to \fullref{thm:renaming_assignment_compatibility/formulas},
  \begin{equation*}
    \Bracks[\Big]{ \psi[\synx \mapsto \syny] }_v = \Bracks{\psi}_{v_{\synx \mapsto \syny}}
  \end{equation*}
  and, in particular, for any \( x \in X \),
  \begin{equation*}
    \Bracks[\Big]{ \psi[\synx \mapsto \syny] }_{v_{\syny \mapsto x}}
    =
    \Bracks{\psi}_{v_{\synx \mapsto \syny,\syny \mapsto x}}
    =
    \Bracks{\psi}_{v_{\synx \mapsto x}},
  \end{equation*}
  where the last equality holds because \( \syny \not\in \op*{Var}(\psi) \).

  Hence,
  \begin{equation*}
    \underbrace{ \Bracks[\Big]{\qforall \syny \parens[\Big]{ \psi[\synx \mapsto \syny] } }_v }_{\Bracks{\varphi[\synx \mapsto \syny]}_v}
    =
    \underbrace{ \Bracks[\Big]{\qforall \synx \psi }_{v_{\synx \mapsto \syny}} }_{\Bracks{\varphi}_{v_{\synx \mapsto \syny}}}.
  \end{equation*}

  This proves \eqref{eq:thm:first_order_substitution_equivalence/term_in_formula}.
\end{proof}

\begin{concept}\label{con:formula_defined_predicate}
  As explained in \fullref{rem:first_order_formula_conventions/necessary_signature}, we avoid adding more predicates than necessary to a language. For this reason, we sometimes use \term{formula-defined predicates}. For example, if \( \leq \) is a \hyperref[def:partially_ordered_set]{partial order} symbol and we want to have a predicate for whether \( \synx \) is the \hyperref[def:extremal_points/top_and_bottom]{bottom element}, we can define the formula
  \begin{equation}\taglabel[\op{IsBottom}]{con:formula_defined_predicate/bottom}
    \ref{con:formula_defined_predicate/bottom}[\synx] \coloneqq \qforall \syny (\synx \leq \syny).
  \end{equation}

  Note that \( [\synx] \) is only a notational convenience for highlighting which variables are free, the actual formula is named \( \op{IsBottom} \). This is consistent with \fullref{rem:substitution_notation} which allows us to write \( \op{IsBottom}[\syny] \) rather than \( \op{IsBottom}[\synx \mapsto \syny] \) to verify if \( \syny \) is a bottom element
\end{concept}

\begin{proposition}\label{thm:first_order_quantifiers_are_dual}
  For any formula \( \varphi \) and any variable \( \synx \) over \( \mscrL \), we have the following equivalences:
  \begin{align}
    \synneg \qforall \synx \varphi &\gleichstark \qexists \synx \synneg \varphi \label{eq:thm:first_order_quantifiers_are_dual/negation_of_universal} \\
    \synneg \qexists \synx \varphi &\gleichstark \qforall \synx \synneg \varphi \label{eq:thm:first_order_quantifiers_are_dual/negation_of_existential}
  \end{align}
\end{proposition}
\begin{proof}
  The two equivalences are connected using \hyperref[thm:classical_equivalences/double_negation]{double negation}. We will only prove \eqref{eq:thm:first_order_quantifiers_are_dual/negation_of_universal}.

  Let \( \mscrX = (X, I) \) be a structure over \( \mscrL \) and let \( v \) be a variable assignment. Then
  \begin{align*}
    \Bracks{ \synneg \qforall \synx \varphi }_v
    &=
    \oline{\Bracks{ \qforall \synx \varphi }_v}
    = \\ &=
    \oline{\bigwedge\set{ \Bracks{\varphi}_{v_{\synx \mapsto x}} \given x \in X }}
    \reloset {\eqref{eq:thm:de_morgans_laws/complement_of_meet}} = \\ &=
    \bigvee\set*{ \oline{\Bracks{\varphi}_{v_{\synx \mapsto x}}} \given x \in X }
    = \\ &=
    \bigvee\set{ \Bracks{\synneg \varphi}_{v_{\synx \mapsto x}} \given x \in X }
    = \\ &=
    \Bracks{\qexists \synx \synneg \varphi}_v.
  \end{align*}
\end{proof}

\begin{corollary}\label{thm:relativized_first_order_quantifiers_are_dual}
  \Fullref{thm:first_order_quantifiers_are_dual} also holds for \hyperref[rem:first_order_formula_conventions/relativization]{relativized} formulas:
  \begin{align}
    \synneg \qforall {p(\synx, \cdots)} \varphi &\gleichstark \qexists {p(\synx, \cdots)} \synneg \varphi \label{eq:thm:relativized_first_order_quantifiers_are_dual/negation_of_universal} \\
    \synneg \qexists {p(\synx, \cdots)} \varphi &\gleichstark \qforall {p(\synx, \cdots)} \synneg \varphi \label{eq:thm:relativized_first_order_quantifiers_are_dual/negation_of_existential}
  \end{align}
\end{corollary}
\begin{proof}
  We will only show \eqref{eq:thm:relativized_first_order_quantifiers_are_dual/negation_of_universal}:
  \begin{align*}
    \synneg \qforall {p(\synx, \cdots)} \varphi
    &=
    \synneg \qforall \synx \parens[\Big]{ p(\synx, \cdots) \synimplies \varphi }
    \gleichstark \\ \reloset {\eqref{eq:thm:classical_equivalences/conditional_as_disjunction}} \gleichstark
    \synneg \qforall \synx \parens[\Big]{ \synneg p(\synx, \cdots) \synvee \varphi }
    \gleichstark \\ \reloset {\eqref{eq:thm:first_order_quantifiers_are_dual/negation_of_universal}} \gleichstark
    \qexists \synx \parens[\Big]{ p(\synx, \cdots) \synwedge \synneg \varphi }
  \end{align*}
\end{proof}

\begin{proposition}\label{thm:pulling_quantifiers_out}
  For \( Q \in \op*{Quan} \) and \( \syncirc \in \set{ \synvee, \synwedge } \), if \( \syny \) is not free in \( \varphi \) nor \( \psi \), then
  \begin{equation}\label{thm:pulling_quantifiers_out/first}
    \parens[\Big]{ \quantifier Q \synx \varphi } \syncirc \psi \gleichstark \quantifier Q \syny \parens[\Big]{ \varphi[\synx \mapsto \syny] \syncirc \psi }\phantom{.}
  \end{equation}
  and similarly
  \begin{equation}\label{thm:pulling_quantifiers_out/second}
    \varphi \syncirc \parens[\Big]{ \quantifier Q \synx \psi } \gleichstark \quantifier Q \syny \parens[\Big]{ \varphi \syncirc \psi[\synx \mapsto \syny] }.
  \end{equation}
\end{proposition}
\begin{proof}
  We will only consider one special case of \eqref{thm:pulling_quantifiers_out/first}:
  \begin{equation}\label{thm:pulling_quantifiers_out/first_proof}
    \parens[\Big]{ \qforall \synx \varphi } \synwedge \psi \gleichstark \qforall \synx \parens[\Big]{ \varphi[\synx \mapsto \syny] \synwedge \psi }
  \end{equation}

  The other cases have similar proofs.

  Fix some \hyperref[def:first_order_structure]{structure} \( \mscrX = (X, I) \) and some \hyperref[def:first_order_valuation/variable_assignment]{variable assignment} \( v: \op*{Var} \to X \). Since \( \syny \) is not free in \( \psi \),
  \begin{balign*}
    \Bracks[\Big]{ \qforall \syny \parens[\Big]{ \varphi[\synx \mapsto \syny] \synwedge \psi } }_v
    &=
    \bigvee\set[\Big]{ \Bracks[\Big]{\varphi[\synx \mapsto \syny]}_{v_{\syny \mapsto x}} \synwedge \Bracks{\psi}_{v_{\syny \mapsto x}} \given* x \in X }
    \reloset {\eqref{eq:thm:renaming_assignment_compatibility/formulas}} = \\ &=
    \bigvee\set[\Big]{ \Bracks{\varphi}_{v_{\synx \mapsto \syny,\syny \mapsto x}} \synwedge \Bracks{\psi}_{v_{\syny \mapsto x}} \given* x \in X }
    = \\ &=
    \bigvee\set[\Big]{ \Bracks{\varphi}_{v_{\synx \mapsto x}} \synwedge \Bracks{\psi}_{v_{\syny \mapsto x}} \given* x \in X }
    = \\ &=
    \bigvee\set[\Big]{ \Bracks{\varphi}_{v_{\synx \mapsto x}} \synwedge \Bracks{\psi}_v \given* x \in X }
    \reloset {\eqref{eq:thm:def:boolean_algebra/distributive/meet_over_join}} = \\ &=
    \bigvee\set[\Big]{ \Bracks{\varphi}_{v_{\synx \mapsto x}} \given* x \in X } \synwedge \Bracks{\psi}_v
    = \\ &=
    \Bracks[\Big]{ \qforall \synx \varphi }_v \synwedge \Bracks{\psi}_v
    = \\ &=
    \Bracks[\Big]{ \parens[\Big]{ \qforall \synx \varphi } \synwedge \psi }_v.
  \end{balign*}

  Generalizing on variable assignments and structures, we conclude that \eqref{thm:pulling_quantifiers_out/first_proof} holds.
\end{proof}

\begin{definition}\label{def:prenex_normal_form}\mcite[def. 2.2.33]{Hinman2005Logic}
  We say that the formula \( \varphi \) over \( \mscrL \) is in \term{prenex normal form} if it can be split as \( \varphi = \psi \theta \), where \( \psi \) contains only quantifiers and their variables (and dots) and \( \theta \) is a formula without quantifiers.

  We call \( \psi \) the \term{quantifier prefix} of \( \varphi \) and \( \theta \) --- the \term{matrix} of \( \varphi \).
\end{definition}

\begin{algorithm}[Prenex normal form conversion]\label{alg:prenex_normal_form_conversion}
  We will describe an algorithm that takes a formula \( \varphi \) over \( \mscrL \) and outputs another formula \( \theta \) over \( \mscrL \) that is \hyperref[def:first_order_semantics/equivalence]{semantically equivalent}, but is in \hyperref[def:prenex_normal_form]{prenex normal form}.

  \begin{thmenum}
    \thmitem{alg:prenex_normal_form/conditionals} First, we remove the conditional \( \synimplies \) and biconditional \( \syniff \). Define
    \begin{equation*}
      P_1(\varphi) \coloneqq \begin{cases}
        \synneg P_1(\psi) \synvee P_1(\theta),                                                                          &\varphi = \psi \synimplies \theta, \\
        \parens[\Big]{ \synneg P_1(\psi) \synvee P_1(\theta) } \synwedge \parens[\Big]{ P_1(\psi) \synvee \synneg P_1(\theta) }, &\varphi = \psi \syniff \theta, \\
        \hyperref[rem:straightforward_traversal]{\T{straightforward traversal}},                                  &\T{otherwise.}
      \end{cases}
    \end{equation*}

    \Fullref{thm:classical_equivalences/conditional_as_disjunction}, \fullref{thm:propositional_derivations/biconditional_as_conjunction} and \fullref{thm:first_order_substitution_equivalence/propositional} together ensure that \( P_1(\varphi) \gleichstark \varphi \) for any formula \( \varphi \).

    \thmitem{alg:prenex_normal_form/negation} Second, we \enquote{push} negations inwards. Define
    \begin{equation*}
      P_2(\varphi) \coloneqq \begin{cases}
        \qexists \synx P_2(\synneg \psi),                                             &\varphi = \synneg \qforall \synx \psi, \\
        \qforall \synx P_2(\synneg \psi),                                             &\varphi = \synneg \qexists \synx \psi, \\
        P_2(\synneg \psi) \synwedge P_2(\synneg \theta),                                  &\varphi = \synneg (\psi \synvee \theta), \\
        P_2(\synneg \psi) \synvee P_2(\synneg \theta),                                    &\varphi = \synneg (\psi \synwedge \theta), \\
        P_2(\psi),                                                               &\varphi = \synneg \synneg \psi, \\
        \synneg P_2(\psi),                                                          &\T{otherwise if} \varphi = \synneg \psi, \\
        \hyperref[rem:straightforward_traversal]{\T{straightforward traversal}}, &\T{otherwise.}
      \end{cases}
    \end{equation*}

    \Fullref{thm:de_morgans_laws}, \fullref{thm:classical_equivalences/double_negation}, \fullref{thm:first_order_substitution_equivalence/propositional} and \fullref{thm:first_order_quantifiers_are_dual} ensure that \( P_2(\varphi) \gleichstark \varphi \) for any formula \( \varphi \)  not containing \( \synimplies \) or \( \syniff \).

    \thmitem{alg:prenex_normal_form/quantifiers} Finally, we \enquote{pull} quantifiers outwards. Define
    \begin{equation*}
      P_3(\varphi) \coloneqq \begin{cases}
        \varphi,                                                                           &\op*{Bound}(\varphi) = \varnothing, \\
        P_3\parens[\Big]{ P_3(\psi) \syncirc P_3(\theta) },                                &\varphi = \psi \syncirc \theta \T{and} \op*{Bound}(\varphi) \neq \varnothing, \\
        \quantifier Q \syny P_3 \parens[\Big]{ \varphi[\synx \mapsto \syny] \syncirc \theta }, &\varphi = \parens[\Big]{ \quantifier Q \synx \psi } \syncirc \theta, \\
                                                                                           &Q \in \op*{Quan}, \quad \syncirc \in \set{ \synvee, \synwedge }, \\
                                                                                           &\syny \not\in \op*{Free}(\psi) \cup \op*{Free}(\theta), \\
        \quantifier Q \syny P_3 \parens[\Big]{ \varphi \syncirc \theta[\synx \mapsto \syny] }, &\varphi = \psi \syncirc \parens[\Big]{ \quantifier Q \synx \theta }, \\
                                                                                           &Q \in \op*{Quan}, \quad \syncirc \in \set{ \synvee, \synwedge }, \\
                                                                                           &\syny \not\in \op*{Free}(\psi) \cup \op*{Free}(\theta), \\
        \hyperref[rem:straightforward_traversal]{\T{straightforward traversal}},           &\T{otherwise.}
      \end{cases}
    \end{equation*}

    \Fullref{thm:pulling_quantifiers_out} ensures that \( P_3(\varphi) \gleichstark \varphi \) for any formula \( \varphi \) not containing \( \synimplies \) or \( \syniff \). It is not necessarily in prenex normal form because it does nothing to \( \varphi = \synneg \qforall \synx \psi \).
  \end{thmenum}

  It remains to compose the operators. As per our comments above, the formula \( P_3(P_2(P_1(\varphi))) \) is in prenex normal form and is semantically equivalent to \( \varphi \).
\end{algorithm}
\begin{comments}
  \item The first rule in the definition of \( P_3 \) ensures that the recursion is well-founded. Checking whether the formula has bound variables is a way to verify that it contains no quantifiers.
  \item This algorithm can be found as \identifier{logic.pnf.to_pnf} in \cite{notebook:code}.
\end{comments}

\begin{proposition}\label{thm:quantifier_satisfiability}
  Let \( \mscrL \) be a first-order language, let \( \varphi \) be a formula, let \( \synx \) be any variable and let \( \tau \) be any term in \( \mscrL \). The following hold:

  \begin{thmenum}
    \thmitem{thm:quantifier_satisfiability/universal} \( \qforall \synx \varphi \vDash \varphi[\synx \mapsto \tau] \),

    \thmitem{thm:quantifier_satisfiability/existential} \( \varphi[\synx \mapsto \tau] \vDash \qexists \synx \varphi \).
  \end{thmenum}
\end{proposition}
\begin{comments}
  \item See also \fullref{def:first_order_natural_deduction_system/terms} for inference rules corresponding to this proposition.
\end{comments}
\begin{proof}
  The proof is very straightforward, but the technical details make it look a bit more complicated.

  First note that if the formulas on the left are unsatisfiable, the proof is trivial. Hence, we will assume that they are satisfiable.

  \SubProofOf{thm:quantifier_satisfiability/universal} Suppose that \( \Bracks{\qforall \synx \varphi}_v = T \) for some variable assignment in \( \mscrX \), and let \( t \coloneqq \Bracks{\tau}_v \).

  To avoid the case where \( \synx \in \op*{Var}(\varphi) \) for the purposes for \fullref{thm:renaming_assignment_compatibility}, we substitute it with \( \syny \) that is not a variable in neither \( \tau \) nor \( \varphi \). Then
  \begin{align*}
    \Bracks[\Big]{\varphi[\synx \mapsto \tau]}_v
    &=
    \Bracks[\Big]{\varphi[\synx \mapsto \syny][\syny \mapsto \tau]}_v
    = \\ &=
    \Bracks[\Big]{\varphi[\synx \mapsto \syny][\syny \mapsto \tau]}_{v_{\syny \mapsto t}}
    \reloset {\ref{thm:first_order_substitution_equivalence/term_in_formula}} = \\ &=
    \Bracks[\Big]{\varphi[\synx \mapsto \syny]}_{v_{\syny \mapsto t}}
    \reloset {\ref{thm:renaming_assignment_compatibility/formulas}} = \\ &=
    \Bracks{\varphi}_{v_{\synx \mapsto \syny, \syny \mapsto t}}
    = \\ &=
    \Bracks{\varphi}_{v_{\synx \mapsto t}}.
  \end{align*}

  Since \( \Bracks{\qforall \synx \varphi} = T \), by definition of valuation of \( \synforall \) we have
  \begin{equation*}
    \Bracks{\varphi}_{v_{\synx \mapsto t}} = T.
  \end{equation*}
  and thus
  \begin{equation*}
    \Bracks[\Big]{\varphi[\synx \mapsto \tau]}_v = T.
  \end{equation*}

  Therefore,
  \begin{equation*}
    \qforall \synx \varphi \T{entails} \varphi[\synx \mapsto \tau] \T{under} v.
  \end{equation*}

  Both \( v \) and \( \mscrX \) were chosen arbitrarily, implying
  \begin{equation*}
    \qforall \synx \varphi \vDash \varphi[\synx \mapsto \tau].
  \end{equation*}

  \SubProofOf{thm:quantifier_satisfiability/existential} If \( \Bracks{\varphi[\synx \mapsto \tau]} = T \), then, as shown previously,
  \begin{equation*}
    \Bracks[\Big]{\varphi[\synx \mapsto \tau]}_v = \Bracks{\varphi}_{v_{\synx \mapsto t}}.
  \end{equation*}

  Hence,
  \begin{equation*}
    \Bracks{\qexists \synx \varphi} = \Bracks{\varphi}_{v_{\synx \mapsto t}} = T.
  \end{equation*}

  Therefore,
  \begin{equation*}
    \varphi[\synx \mapsto \tau] \T{entails} \varphi \T{under} v,
  \end{equation*}
  and, generalizing,
  \begin{equation*}
    \varphi[\synx \mapsto \tau] \vDash \varphi.
  \end{equation*}
\end{proof}

\begin{proposition}\label{thm:existential_quantifier_removal}
  Let \( \varphi \) be a formula in the first-order language \( \mscrL \), and let \( \widetilde \mscrL \) be an extension of \( \mscrL \). Clearly \( \varphi \) is a formula in \( \widetilde \mscrL \).

  Suppose that \( c \) is some constant in \( \widetilde \mscrL \) that does not occur in \( \varphi \). Then there exists a structure \( \mscrX = (X, I_X) \) for \( \mscrL \) and an assignment \( v \) in \( \mscrX \) such that
  \begin{equation*}
    \Bracks{\qexists \synx \varphi}_v = T
  \end{equation*}
  if and only if there exists a structure \( \mscrY = (Y, I_Y) \) for \( \widetilde \mscrL \) and an assignment \( w \) in \( \mscrY \) such that
  \begin{equation*}
    \Bracks[\Big]{\varphi[\synx \mapsto c]}_w = T
  \end{equation*}
\end{proposition}
\begin{comments}
  \item The sufficiency statement is a weaker converse to \fullref{thm:quantifier_satisfiability/existential}.
  \item We can always adjoin a single constant \( c \) to \( \mscrL \) to obtain \( \widetilde \mscrL \) (this is actually our proof of sufficiency). This allows us to eliminate existential quantifiers from formulas at the cost of a more general first-order language.
\end{comments}
\begin{proof}
  \SufficiencySubProof Suppose that, for some assignment \( v \) in \( \mscrX \), \( \Bracks{\qexists \synx \varphi}_v = T \). Let \( x_0 \) be a value in \( X \) such that \( \Bracks{\varphi}_{v_{\synx \to x_0}} = T \).

  Construct a language \( \widetilde \mscrL \) with all predicate and functional symbols of \( \mscrL \), along with a new constant symbol \( c \).

  Define a structure \( \mscrY = (X, I_Y) \), where
  \begin{equation*}
    \widetilde I_Y(a) \coloneqq \begin{cases}
      x_0,    &a = c, \\
      I_X(a), &a \in \op*{Func}, \\
      I_X(a), &a \in \op*{Pred}
    \end{cases}
  \end{equation*}

  This structure has the same domain as \( \mscrX \), hence \( v \) is also a variable assignment in \( \mscrY \).

  Since \( \Bracks{c}_{v_{\synx \mapsto x_0}} = \Bracks{\synx}_{v_{\synx \mapsto x_0}} \), \fullref{thm:first_order_substitution_equivalence/term_in_formula} implies that
  \begin{equation*}
    \Bracks[\Big]{\varphi[\synx \mapsto c]}_{v_{\synx \mapsto x_0}} = \Bracks{\varphi}_{v_{\synx \mapsto x_0}} = T.
  \end{equation*}

  But, since \( \varphi[\synx \mapsto c] \) contains no free instances of \( \synx \),
  \begin{equation*}
    \Bracks[\Big]{\varphi[\synx \mapsto c]}_v = \Bracks[\Big]{\varphi[\synx \mapsto c]}_{v_{\synx \mapsto x_0}}.
  \end{equation*}

  Therefore, \( \Bracks{\qexists \synx \varphi}_v = T \) in \( \mscrX \) implies \( \Bracks{\varphi[\synx \mapsto c]}_v = T \) in \( \mscrY \).

  \NecessitySubProof Follows from \fullref{thm:quantifier_satisfiability/existential}.
\end{proof}

\begin{example}\label{ex:existential_quantifier_removal_monoids}
  The \hyperref[def:monoid/theory]{theory of monoids} features a simple demonstration of \fullref{thm:existential_quantifier_removal}.

  One of the axioms for monoids states, in prose, that there exists a distinguished element that is a two-sided identity. This can be written as follows:
  \begin{equation*}
    \qexists \syny \qforall \synx (\syny \cdot \synx \syneq \synx \synwedge \synx \cdot \syny \syneq \synx).
  \end{equation*}

  Instead, for \fullref{def:monoid/theory} we chose to include a constant symbol \( e \) in the language and use the axiom
  \begin{equation*}
    \qforall \synx (e \cdot \synx \syneq \synx \synwedge \synx \cdot e \syneq \synx).
  \end{equation*}
\end{example}

\begin{theorem}[First-order semantic deduction theorem]\label{thm:semantic_deduction_theorem}
  For arbitrary first-order formulas, the entailment \( \Gamma, \varphi \vDash \psi \) holds if and only if \( \Gamma \vDash \varphi \to \psi \) holds.
\end{theorem}
\begin{comments}
  \item This theorem also holds for propositional formulas with the adaptation outlined in \fullref{rem:propositional_logic_as_first_order_logic}.
  \item Compare this result with \fullref{thm:syntactic_deduction_theorem}.
\end{comments}
\begin{proof}
  \SufficiencySubProof Suppose that \( \Gamma, \varphi \vDash \psi \). Let \( \mscrX = (X, I) \) be some structure and let \( v \) be a variable assignment in \( \mscrX = (X, I) \) such that \( \Bracks{\theta}_v = T \) for every \( \theta \in \Gamma \).

  \begin{itemize}
    \item If \( \Bracks{\varphi}_v = T \), by assumption it follows that \( \Bracks{\psi}_v = T \), and
    \begin{equation*}
      \Bracks{\varphi \synimplies \psi}_v
      =
      \Bracks{\varphi}_v \rightarrow \Bracks{\psi}_v
      =
      T \rightarrow T
      =
      T.
    \end{equation*}

    \item If \( \Bracks{\varphi}_v = F \), then
    \begin{equation*}
      \Bracks{\varphi \synimplies \psi}_v
      =
      F \rightarrow \Bracks{\psi}_v
      =
      T.
    \end{equation*}
  \end{itemize}

  In both cases we conclude that \( \Gamma \) entails \( \varphi \synimplies \psi \) under \( v \). Generalizing on \( v \) and \( \mscrX \), we conclude that \( \Gamma \vDash \varphi \synimplies \psi \).

  \NecessitySubProof Conversely, suppose that \( \Gamma \vDash \varphi \synimplies \psi \). Let \( v \) again be a variable assignment in \( \mscrX = (X, I) \) such that \( \Bracks{\theta}_v = T \) for every \( \theta \in \Gamma \). Then
  \begin{equation*}
    \Bracks{\varphi \synimplies \psi}_v = T.
  \end{equation*}

  If we additionally suppose that \( \Bracks{\varphi}_v = T \), it follows from \fullref{def:standard_boolean_functions} that \( \Bracks{\psi}_v = T \).

  Generalizing on \( v \) and \( \mscrX \), we conclude that \( \Gamma, \varphi \vDash \psi \).
\end{proof}

\paragraph{Natural deduction}

\begin{definition}\label{def:first_order_natural_deduction_system}\mcite[def. 2.1.1]{TroelstraSchwichtenberg2000Proofs}
  If we wish to work with first-order logic rather than merely propositional logic, we must extend the \hyperref[def:propositional_natural_deduction_systems]{classical propositional natural deduction system}. We call this, very simply, the (classical) \term{first-order natural deduction system}.

  \begin{thmenum}
    \thmitem{def:first_order_natural_deduction_system/eigenvariables} We first add the following two \hyperref[def:inference_rule]{inference rules} for quantification:

    \begin{minipage}{0.45\textwidth}
      \begin{equation*}\taglabel[\ensuremath{ \synforall^+ }]{eq:def:first_order_natural_deduction_system/forall/intro}
        \begin{prooftree}
          \hypo{ \varphi }
          \infer1[\ref{eq:def:first_order_natural_deduction_system/forall/intro}]{ \qforall \synx \varphi }
        \end{prooftree}
      \end{equation*}
    \end{minipage}
    \hfill
    \begin{minipage}{0.45\textwidth}
      \begin{equation*}\taglabel[\ensuremath{ \synexists^- }]{eq:def:first_order_natural_deduction_system/exists/elim}
        \begin{prooftree}
          \hypo{ \qexists \synx \varphi }
          \hypo{ [\varphi]^n }
          \ellipsis {} { \psi }
          \infer[left label=\( n \)]2[\ref{eq:def:first_order_natural_deduction_system/exists/elim}]{ \psi }
        \end{prooftree}
      \end{equation*}
    \end{minipage}

    Here \( \synx \) is a variable that is not free in any undischarged assumption in our proof of \( \varphi \) (it may be free in \( \varphi \) as long as \( \varphi \) itself is discharged). A variable \( \synx \) satisfying these conditions is called an \term{eigenvariable} of the rule.

    These rules are the primary motivation for inference rules accepting proof trees rather than only formulas --- see \fullref{def:inference_rule} and \fullref{def:deduction_system/rule}. See \fullref{ex:def:first_order_natural_deduction_system/eigenvariables/invalid_universal} for why this condition is important.

    \thmitem{def:first_order_natural_deduction_system/terms} We add two \hyperref[def:inference_rule]{inference rules}, where \( \tau \) is an arbitrary term:

    \begin{minipage}{0.45\textwidth}
      \begin{equation*}\taglabel[\ensuremath{ \synforall^- }]{eq:def:first_order_natural_deduction_system/forall/elim}
        \begin{prooftree}
          \hypo{ \qforall \synx \varphi }
          \infer1[\ref{eq:def:first_order_natural_deduction_system/forall/elim}]{ \varphi[\synx \mapsto \tau] }
        \end{prooftree}
      \end{equation*}
    \end{minipage}
    \hfill
    \begin{minipage}{0.45\textwidth}
      \begin{equation*}\taglabel[\ensuremath{ \synexists^+ }]{eq:def:first_order_natural_deduction_system/exists/intro}
        \begin{prooftree}
          \hypo{ \varphi[\synx \mapsto \tau] }
          \infer1[\ref{eq:def:first_order_natural_deduction_system/exists/intro}]{ \qexists \synx \varphi }
        \end{prooftree}
      \end{equation*}
    \end{minipage}

    Compare this to \fullref{thm:quantifier_satisfiability}.

    \thmitem{def:first_order_natural_deduction_system/equality} Finally, we also add three rules for formal equality:

    \begin{minipage}{0.3\textwidth}
      \begin{equation*}\taglabel[\ensuremath{ \syneq^+ }]{eq:def:first_order_natural_deduction_system/equality/intro}
        \begin{prooftree}
          \infer0[\ref{eq:def:first_order_natural_deduction_system/equality/intro}]{ \tau \syneq \tau }
        \end{prooftree}
      \end{equation*}
    \end{minipage}
    \hfill
    \begin{minipage}{0.3\textwidth}
      \begin{equation*}\taglabel[\ensuremath{ \syneq_L^- }]{eq:def:first_order_natural_deduction_system/equality/elim_left}
        \begin{prooftree}
          \hypo{ \tau \syneq \sigma }
          \hypo{ \varphi[\synx \mapsto \tau] }
          \infer2[\ref{eq:def:first_order_natural_deduction_system/equality/elim_left}]{ \varphi[\synx \mapsto \sigma] }
        \end{prooftree}
      \end{equation*}
    \end{minipage}
    \hfill
    \begin{minipage}{0.3\textwidth}
      \begin{equation*}\taglabel[\ensuremath{ \syneq_L^+ }]{eq:def:first_order_natural_deduction_system/equality/elim_right}
        \begin{prooftree}
          \hypo{ \tau \syneq \sigma }
          \hypo{ \varphi[\synx \mapsto \sigma] }
          \infer2[\ref{eq:def:first_order_natural_deduction_system/equality/elim_right}]{ \varphi[\synx \mapsto \tau] }
        \end{prooftree}
      \end{equation*}
    \end{minipage}
  \end{thmenum}
\end{definition}

\begin{example}\label{ex:def:first_order_natural_deduction_system/eigenvariables}
  \hfill
  \begin{thmenum}
    \thmitem{ex:def:first_order_natural_deduction_system/eigenvariables/invalid_universal_closure} We explicitly forbid the syntactic equivalent of \fullref{thm:implicit_universal_quantification} in order to avoid invalid proofs like \fullref{ex:def:first_order_natural_deduction_system/eigenvariables/invalid_universal}. Consider the proof
    \begin{equation*}
      \begin{prooftree}
        \hypo{ \varphi }
        \infer1[\ref{eq:def:first_order_natural_deduction_system/forall/intro}]{ \qforall \synx \varphi }
      \end{prooftree}
    \end{equation*}

    The problem here is that \( \varphi \) is itself an undischarged assumption, hence \eqref{eq:def:first_order_natural_deduction_system/forall/intro} is actually inapplicable here, and the proof is invalid.

    \thmitem{ex:def:first_order_natural_deduction_system/eigenvariables/invalid_universal} To see why the eigenvariable conditions in \fullref{def:first_order_natural_deduction_system/eigenvariables} are essential, consider the following proof of \( \qforall \synx \varphi \) from \( \qexists \synx \varphi \):
    \begin{equation*}
      \begin{prooftree}
        \hypo{ \qexists \synx \varphi }

        \hypo{ [\varphi]^1 }
        \infer1[\ref{eq:def:first_order_natural_deduction_system/forall/intro}]{ \qforall \synx \varphi }

        \infer[left label=\( 1 \)]2[\ref{eq:def:first_order_natural_deduction_system/exists/elim}]{ \qforall \synx \varphi }
      \end{prooftree}
    \end{equation*}

    This proof relies on \fullref{ex:def:first_order_natural_deduction_system/eigenvariables/invalid_universal_closure}, which we have already demonstrated to be invalid.

    \thmitem{ex:def:first_order_natural_deduction_system/eigenvariables/invalid_existence} Another invalid proof, in case \( \synx \in \op*{Free}(\varphi) \), is
    \begin{equation*}
      \begin{prooftree}
        \hypo{ \qexists \synx \varphi }

        \hypo{ [\varphi]^1 }
        \infer1{ \varphi }

        \infer[left label=\( 1 \)]2[\ref{eq:def:first_order_natural_deduction_system/exists/elim}]{ \varphi }
      \end{prooftree}
    \end{equation*}

    \thmitem{ex:def:first_order_natural_deduction_system/eigenvariables/universal_implies_existence} On the other hand, \( \qexists \synx \varphi \) can easily be derived from \( \qforall \synx \varphi \):
    \begin{equation*}
      \begin{prooftree}
        \hypo{ \qforall \synx \varphi }
        \infer1[\ref{eq:def:first_order_natural_deduction_system/forall/elim}]{ \varphi = \varphi[\synx \mapsto \synx] }
        \infer1[\ref{eq:def:first_order_natural_deduction_system/exists/intro}]{ \qexists \synx \varphi }
      \end{prooftree}
    \end{equation*}

    \thmitem{ex:def:first_order_natural_deduction_system/eigenvariables/universal_implies_universal} It is also valid to perform the completely meaningless derivation:
    \begin{equation*}
      \begin{prooftree}
        \hypo{ \qforall \synx \varphi }
        \infer1[\ref{eq:def:first_order_natural_deduction_system/forall/elim}]{ \varphi = \varphi[\synx \mapsto \synx] }
        \infer1[\ref{eq:def:first_order_natural_deduction_system/forall/intro}]{ \qforall \synx \varphi }
      \end{prooftree}
    \end{equation*}
  \end{thmenum}
\end{example}

\begin{proposition}\label{thm:syntactic_first_order_quantifiers_are_dual}
  For any formula \( \varphi \) and any variable \( \synx \) over \( \mscrL \), we have the following interderivable pairs:
  \begin{align}
    \synneg \qforall \synx \varphi &\T{and} \qexists \synx \synneg \varphi \label{thm:syntactic_first_order_quantifiers_are_dual/negation_of_universal} \\
    \synneg \qexists \synx \varphi &\T{and} \qforall \synx \synneg \varphi \label{thm:syntactic_first_order_quantifiers_are_dual/negation_of_existential}
  \end{align}
\end{proposition}
\begin{proof}
  We will only show \eqref{thm:syntactic_first_order_quantifiers_are_dual/negation_of_universal}. First,

  \begin{equation*}
    \begin{prooftree}
      \hypo{ \synneg \qforall \synx \varphi }
      \hypo{ [\qforall \synx \varphi]^1 }
      \infer2[\ref{inf:def:propositional_natural_deduction_systems/neg/elim}]{ \synbot }
      \infer[left label=\( 1 \)]1[\eqref{eq:def:propositional_natural_deduction_systems/rules/dne}]{ \qforall \synx \varphi }
      \infer1[\eqref{eq:def:first_order_natural_deduction_system/forall/elim}]{ \varphi }

      \hypo{ [\synneg \varphi]^2 }
      \infer2[\ref{inf:def:propositional_natural_deduction_systems/neg/elim}]{ \synbot }

      \infer[left label=\( 2 \)]1[\ref{inf:def:propositional_natural_deduction_systems/neg/intro}]{ \synneg \varphi }
      \infer1[\eqref{eq:def:first_order_natural_deduction_system/exists/intro}]{ \qexists \synx \synneg \varphi }
    \end{prooftree}
  \end{equation*}

  Conversely,
  \begin{equation*}
    \begin{prooftree}
      \hypo{ \qexists \synx \synneg \varphi }

      \hypo{ [\qforall \synx \varphi]^1 }
      \infer1[\eqref{eq:def:first_order_natural_deduction_system/forall/elim}]{ \varphi }

      \hypo{ [\synneg \varphi]^2 }
      \infer2[\ref{inf:def:propositional_natural_deduction_systems/neg/elim}]{ \synbot }
      \infer[left label=\( 1 \)]1[\ref{inf:def:propositional_natural_deduction_systems/neg/intro}]{ \synneg \qforall \synx \varphi }

      \infer[left label=\( 2 \)]2[\eqref{eq:def:first_order_natural_deduction_system/exists/elim}]{ \synneg \qforall \synx \varphi }
    \end{prooftree}
  \end{equation*}
\end{proof}

\begin{theorem}[Soundness and completeness of first-order logic]\label{thm:classical_first_order_logic_soundness_and_complete}\mcite[thm. 3.4.1]{Hinman2005Logic}
  The \hyperref[def:first_order_natural_deduction_system]{classical first-order natural deduction system} is \hyperref[def:logical_framework/soundness]{sound} and \hyperref[def:logical_framework/completeness]{complete} with respect to \hyperref[def:first_order_semantics]{classical semantics}.
\end{theorem}
\begin{comments}
  \item The completeness part is known as \enquote{G\"odel's completeness theorem} and requires an elaborate proof.
  \item The theorem also applies, with adaptations based on \fullref{rem:propositional_logic_as_first_order_logic}, to the \hyperref[def:propositional_natural_deduction_systems]{classical propositional natural deduction system}.
\end{comments}
