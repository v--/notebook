\section{Fixed-width arithmetic}\label{sec:fixed_width_arithmetic}

\paragraph{Endianness}

\begin{definition}\label{def:endianness}\mimprovised
  To any \hyperref[def:positional_number_system]{digit string} \( d_0 \ldots d_{m-1} \) over a fixed \hyperref[def:positional_number_system]{positional number system}, we assign two total orders:

  \begin{thmenum}
    \thmitem{def:endianness/big} The \term{big-endian} order is based on the implicit \hyperref[def:enumeration]{enumeration} \( d_0, \ldots, d_{m-1} \).

    We say that the digit \( d_i \) is \term[en=more significant (\cite[195]{Knuth1997ArtVol2})]{more significant} than \( d_j \) if \( i < j \). Correspondingly, \( d_0 \) is the most significant digit, while \( d_{m-1} \) --- the least significant.

    \thmitem{def:endianness/little} The \term{little-endian} order instead uses the reverse enumeration \( d_{m-1}, \ldots, d_0 \).

    In this case, \( d_i \) is more significant than \( d_j \) if \( i < j \). Thus, \( d_{m-1} \) is the most significant digit, while \( d_0 \) --- the least significant.
  \end{thmenum}
\end{definition}
\begin{comments}
  \item Endianness is discussed in more detail in \fullref{rem:endianness}.
\end{comments}

\begin{remark}\label{rem:endianness}
  The \hyperref[def:endianness]{endianness} terminology is based on Danny Cohen's article \cite{Cohen1981Endianness} on the order in which computer hardware and communication protocols handle \hyperref[def:bit_string]{bit strings}. Cohen in turn based his terms on Jonathan Swift's \cite{Swift2005GulliverTravels}, where Lilliputs fight wars for which end the eggs should be broken at.

  The following quote from the book first refers to \enquote{endianness}:
  \begin{displayquote}
    It is computed, that eleven Thousand Persons have, at several Times, suffered Death, rather than submit to break their Eggs at the smaller End. Many hundred large Volumes have been published upon this Controversy: But the Books of the Big-Endians have been long forbidden, and the whole Party rendered incapable by Law of holding Employments.
  \end{displayquote}

  When computer processors deal with bit strings, endianness affects how atomic operations are performed. According to the article \cite{IBMDocs:endian_independent_code},
  \begin{displayquote}
    Endianness only makes sense when you're breaking up a multi-byte quantity and are trying to store the bytes at consecutive memory locations. However, if you have a 32-bit register storing a 32-bit value, it makes no sense to talk about endianness. The register is neither big-endian nor little-endian; it's just a register holding a 32-bit value. The rightmost bit is the least significant bit, and the leftmost bit is the most significant bit.

    Some people classify a register as a big-endian, because it stores its most significant byte at the lowest memory address.
  \end{displayquote}
  and later
  \begin{displayquote}
    Endianness doesn't apply to everything. If you do bitwise or bitshift operations on an int, you don't notice the endianness. The machine arranges the multiple bytes, so the least significant byte is still the least significant byte, and the most significant byte is still the most significant byte.
  \end{displayquote}

  The same article also comments on hardware manufacturer's preferences, saying
  \begin{displayquote}
    All processors must be designated as either big-endian or little-endian. For example, the 80x86 processors from Intel\textregistered and their clones are little-endian, while Sun's SPARC, Motorola's 68K, and the PowerPC\textregistered families are all big-endian.
  \end{displayquote}
  and
  \begin{displayquote}
    Little-endian based processors (and their clones) are used in most personal computers and laptops, so the vast majority of desktop computers today are little-endian.
  \end{displayquote}

  Another processor manufacturer, ARM, writes in their documentation, \cite{ARMDocs:endianness}, the following:
  \begin{displayquote}
    ARM cores support both modes, but are most commonly used in, and typically default to little-endian mode. Most Linux distributions for ARM tend to be little-endian only. The x86 architecture is little-endian. The PowerPC or the venerable 68K, on the other hand, are generally big-endian, although the Power architecture can also handle little-endian. Several common file formats and networking protocols specify different endianness. For example, .BMP and .GIF files are little-endian, while .JPG is big-endian, and TCP/IP is big-endian, but USB and PCI are little-endian.
  \end{displayquote}

  According to the comments in \cite{SO:bit_shift_endianness}, endianness affects how some operations on PowerPC depend on endianness. Nevertheless, it is accepted that endianness rarely matters within the scope of a single machine. It is crucial, however, when data is transmitted between different machines. For this reason, file formats and network protocols must be explicit about the endianness which they use.

  We will give a more concrete example of handling integers under different endianness conventions in \fullref{ex:little_endian_arithmetic}, after discussing fixed-width integer encoding.
\end{remark}

\paragraph{Unsigned integer encoding}

\begin{definition}\label{def:ring_of_unsigned_integers}\mimprovised
  Fix a \hyperref[def:positional_number_system]{positional number system} \( s_0 \ldots s_{b-1} \) and \hyperref[def:endianness]{endianness}. For some positive integer \( w \), consider the set \( U_w \) of all \hyperref[def:positional_number_system]{digit strings} of length \( w \).

  We say that the string \( s_{a_0} \ldots s_{a_{w-1}} \) \term{encodes} the nonnegative integer \( n \) if
  \begin{equation*}
    n = \sum_{k=0}^{w-1} a_k b^{e(k)},
  \end{equation*}
  where \( e(k) = k \) in case of little-endian ordering and \( e(k) = w - 1 - k \) in case of big-endian ordering.

  \Fullref{alg:addition_with_carrying} and \fullref{alg:addition_with_carrying}, adapted with respect to endianness, define operations that turn \( U_w \) into a \hyperref[def:ring]{ring} \hyperref[def:algebra_over_semiring/homomorphism]{isomorphic} to the ring \hyperref[def:ring_of_integers_modulo]{\( \BbbZ_{2^w} \)} of integers modulo \( 2^w \).

  We call the elements of \( U_w \) big or little endian \term{unsigned \( w \)-digit integers} and refer to \( w \) as their \term{width}.
\end{definition}
\begin{comments}
  \item This is an extension of the positional natural number strings whose connection to real number expansions is discussed in \fullref{def:positional_notation_as_radix_expansion/natural_numbers}. It allows both varying endianness and relaxes the requirement that the most significant digit must be nonzero. The uniqueness lost from this encoding is regained by imposing a specific width.
\end{comments}
\begin{defproof}
  Fix a positional number system \( s_0, \ldots, s_{b-1} \).

  For a nonnegative integer \( n < b^w \), \fullref{alg:integer_radix_expansion} gives us the radix \( b \) expansion
  \begin{equation*}
    p(X) = \sum_{k=0}^r a_k X^k.
  \end{equation*}

  If \( r \geq w \), then \( a_r b^r \geq b^r \geq b^w \), and thus \( n = p(b) \geq b^w \), which is a contradiction. It remains for \( r \) to be less than \( w \).

  If \( r < w - 1 \), we simply define \( a_k = 0 \) for \( k > r \) to obtain the digit string
  \begin{equation*}
    s_{a_0} s_{a_1} \ldots s_{a_{w-1}}
  \end{equation*}
  encoding \( n \) via little-endian ordering.

  For big-endian ordering, we can simply reverse the string.

  We have thus constructed a digit string for each nonnegative integer \( n < b^w \). Each nonnegative integer has a unique radix \( b \) expansion, thus the numbers \( 0, \ldots, b^w - 1 \) correspond to \( b^w \) digit strings of length \( w \).

  \Fullref{thm:cardinality_product_rule} implies that there is a total of \( b^w \) digit strings of length \( w \), hence we have described all of them.
\end{defproof}

\begin{proposition}\label{thm:nonnegative_integer_expansion_divisibility}
  Fix a radix \( b \geq 2 \) and a divisor \( d \) of \( b \). Let \( w \) and \( n \) be nonnegative integers satisfying \( n < b^w \).

  Then \( d \) divides \( n \) if and only if \( d \) divides the \hyperref[def:endianness]{least significant digit} of the \hyperref[def:fixed_width_nonnegative_integer_encoding]{\( w \)-digit} encoding of \( n \).
\end{proposition}
\begin{proof}
  Consider the encoding \( s_{a_0} \ldots s_{a_{w-1}} \).

  In the case of little-endian ordering, we have
  \begin{equation*}
    n = \sum_{k=0}^{w-1} a_k b^k.
  \end{equation*}

  Since \( d \) divides \( b \), it divides \( b^k \) for \( k > 0 \), and thus also \( a_k b^k \). Therefore, in order to divide \( n \), \( d \) must divide the least significant digit \( a_0 = a_0 \cdot b^0 \), and vice versa.

  In the case of big-endian ordering, we have
  \begin{equation*}
    n = \sum_{k=0}^w a_k b^{w-1-k},
  \end{equation*}
  and thus \( d \) divides \( n \) if and only if \( d \) divides the least significant digit \( a_{w-1} \).
\end{proof}

\begin{corollary}\label{thm:nonnegative_integer_decimal_expansion_parity}
  The \hyperref[def:integers]{positive integer} \( n \) is \hyperref[def:integer_parity]{even} if and only if the \hyperref[def:endianness]{least significant digit} of any \hyperref[def:fixed_width_nonnegative_integer_encoding]{decimal encoding} of \( n \) is even.
\end{corollary}
\begin{proof}
  Follows from \fullref{thm:nonnegative_integer_expansion_divisibility} since \( 2 \) divides \( 10 \).
\end{proof}

\begin{corollary}\label{thm:nonnegative_integer_binary_expansion_parity}
  The \hyperref[def:integers]{positive integer} \( n \) is \hyperref[def:integer_parity]{even} if and only if the \hyperref[def:endianness]{least significant digit} of any \hyperref[def:fixed_width_nonnegative_integer_encoding]{binary encoding} of \( n \) is zero.
\end{corollary}
\begin{proof}
  Follows from \fullref{thm:nonnegative_integer_expansion_divisibility} since \( 2 \) obviously divides \( 2 \).
\end{proof}

\begin{example}\label{ex:little_endian_arithmetic}
  In this example, we will regard \hyperref[def:positional_number_system]{digit string} as lines of \enquote{memory cells}, like those of a physical computer or a \hyperref[def:turing_machine]{Turing machine's tape}.

  \hyperref[def:endianness/little]{Big-endian} ordering corresponds to how digits are encoded in the familiar decimal system --- the number \( 234 \) is \hyperref[def:fixed_width_nonnegative_integer_encoding]{encoded} by the decimal digit string
  \begin{MemoryLine}{3}
    2 & 3 & 4
  \end{MemoryLine}

  \hyperref[def:endianness/little]{Little-endian} ordering, on the other hand, would encode the same number as
  \begin{MemoryLine}{3}
    4 & 3 & 2
  \end{MemoryLine}

  Little-endianness allows certain algorithmic patterns that big-endianness does not. Some of them are discussed in \cite{SESE:little_endianness}.

  For example, consider a long memory line with segments of cells, each segment having the same endianness. The unique \( 8 \)-digit big-endian decimal encoding of \( 234 \) corresponds to the segment
  \begin{MemoryLine}{12}
    \anon & \anon & \MemoryLineArrow 0 & 0 & 0 & 0 & 0 & 2 & 3 & 4 & \anon & \anon
  \end{MemoryLine}

  We have denoted the beginning of the segment with an arrow pointer. Clearly \( 234 \) can be encoded via smaller \( 4 \)-digit segments.

  If we depend on reading the memory line sequentially, we need to scan the entire segment to determine the least significant digit. For algorithms that depend on reading the least significant bit first, like \fullref{alg:addition_with_carrying} or even the evenness check \fullref{thm:nonnegative_integer_binary_expansion_parity}, this requires scanning the whole segment and then possibly going backwards.

  To convert the \( 8 \)-digit segment above to a \( 4 \)-digit segment, one option is to move the pointer and discard the first \( 4 \) digits of the \( 8 \)-digit segment:
  \begin{MemoryLine}{12}
    \anon & \anon & \anon & \anon & \anon & \anon & \MemoryLineArrow 0 & 2 & 3 & 4 & \anon & \anon
  \end{MemoryLine}

  Another option is to move the least significant \( 4 \) digits and discard the rest:
  \begin{MemoryLine}{12}
    \anon & \anon & \MemoryLineArrow 0 & 2 & 3 & 4 & \anon & \anon & \anon & \anon & \anon & \anon
  \end{MemoryLine}

  With little-endian order, this operation is instead \enquote{free}. The original \( 8 \)-digit segment is
  \begin{MemoryLine}{12}
    \anon & \anon & \MemoryLineArrow 4 & 3 & 2 & 0 & 0 & 0 & 0 & 0 & \anon & \anon
  \end{MemoryLine}

  To obtain a \( 4 \)-digit segment, we simply discard the most significant digits:
  \begin{MemoryLine}{12}
    \anon & \anon & \MemoryLineArrow 4 & 3 & 2 & 0 & \anon & \anon & \anon & \anon & \anon & \anon
  \end{MemoryLine}

  This has impact on arithmetic, for example. Adding the above to the segment
  \begin{MemoryLine}{12}
    \anon & \anon & \MemoryLineArrow 6 & 6 & 7 & 9 & \anon & \anon & \anon & \anon & \anon & \anon
  \end{MemoryLine}
  results in the \( 5 \)-digit segment
  \begin{MemoryLine}{12}
    \anon & \anon & \MemoryLineArrow 0 & 0 & 0 & 0 & 1 & \anon & \anon & \anon & \anon & \anon
  \end{MemoryLine}

  We can start summing via \fullref{alg:addition_with_carrying} from the least significant bits. The algorithm states that the sum has at most one digit more than the summands. If we pre-allocate one spare digit to the right, we are guaranteed for the sum to fit, and we can trivially discard unused digits later. Such discarding is not possible with big-endian ordering.
\end{example}

\paragraph{Digit shifting}

\begin{definition}\label{def:digit_shift}
  In the ring \hyperref[def:ring_of_unsigned_integers]{\( U_w \)} of \hyperref[def:endianness/big]{\hi{big-endian}} unsigned \( w \)-digit integers, we can define the \term{left shift} by \( m \) as
  \begin{align}
    d_0 d_1 \cdots d_{w-2} d_{w-1} \ll m &\coloneqq d_m d_{m+1} \cdots d_{w-2} d_{w-1} \underbrace{0 \cdots 0}_{m \T*{zeros}} \label{eq:def:digit_shift/left}
  \intertext{and the corresponding \term{right shift} as}
    d_0 d_1 \cdots d_{w-2} d_{w-1} \gg m &\coloneqq \underbrace{0 \cdots 0}_{m \T*{zeros}} d_0 d_1 \cdots d_{w-2-m} d_{w-1-m} \label{eq:def:digit_shift/right}
  \end{align}

  For little-endian integers, we must swap left and right shifts.
\end{definition}
\begin{comments}
  \item We generalize bit shifts as defined in the C programming language standard, \cite{ISO:9899:2018}. Endianness is not discussed in the standard, but the discussion in \cite{SO:bit_shift_endianness} suggests that bit shifting is based on bit significance rather than bit position.

  \item Digit shifts are closely related to shifts of \hyperref[def:sequence]{infinite} and \hyperref[def:doubly_infinite_sequence]{doubly infinite sequences}, defined in \fullref{def:shift_operator}.
\end{comments}

\begin{proposition}\label{thm:digit_shift_arithmetic}
  If a base \( b \) \hyperref[def:positional_number_system]{digit string} of width \( w \) \hyperref[def:fixed_width_nonnegative_integer_encoding]{encodes} the integer \( n \), the left \hyperref[def:digit_shift]{digit shift} by \( m \) encodes \( \rem(n \cdot b^m, b^w) \), while the right shift encodes \( \rem(n, b^m) \).
\end{proposition}
\begin{proof}
  We will only consider big-endian encoding in the proof; for little-endianness, the proof then follows by \fullref{thm:preorder_duality}.

  Fix a positional number system \( s_0 \ldots s_{b-1} \) and a string \( s_{a_0} \ldots s_{a_{w-1}} \). Let
  \begin{equation*}
    n \coloneqq \sum_{k=0}^{w-1} a_k b^k.
  \end{equation*}

  Then the left shift \( s_{a_0} \ldots s_{a_{w-1}} \ll m \) encodes
  \begin{equation*}
    \sum_{k=m}^{w-1} a_{k-m} b^k = b^m \cdot \sum_{k=m}^{w-1} a_{k-m} b^{k-m}.
  \end{equation*}

  We have \( \rem(b^{w+k}, b_w) = 0 \) for \( k \geq 0 \), thus the above equals
  \begin{equation*}
    \rem\parens*{ b^m \cdot \sum_{k=m}^{w+m-1} a_{k-m} b^{k-m}, b^w } = \rem(n \cdot b^m, b^w).
  \end{equation*}

  The right shift encodes
  \begin{equation*}
    \sum_{k=0}^{w-m-1} a_{k-m} b^k,
  \end{equation*}
  which by the same reasoning equals \( \rem(n, b^m) \).
\end{proof}

\paragraph{Bitwise operations}

\begin{definition}\label{def:bitwise_operations}\mcite[13]{Rosen2019DiscreteMathematics}
  For every \( n \)-ary \hyperref[def:boolean_function]{Boolean function} \( f(x_1, \ldots, x_n) \), we define the corresponding \term{bitwise} function taking on \( n \) \hyperref[def:bit_string]{bit strings} of the same width and applying \( f \) to their \( k \)-th digits.
\end{definition}
\begin{comments}
  \item For the Boolean functions in \fullref{def:standard_boolean_functions}, we obtain the unary \enquote{bitwise not}, the binary \enquote{bitwise or}, \enquote{bitwise and} and so forth.

  \item Bitwise operations do not depend on \hyperref[def:endianness]{endianness}.
\end{comments}

\begin{proposition}\label{thm:disjoint_sum_via_bitwise_xor}
  \hyperref[def:bitwise_operations]{Bitwise xor} allows expressing any \hyperref[def:bit_string]{bit string} as
  \begin{equation*}
    d_0 \ldots d_{n-1} = \bigoplus_{k=0}^{n-1} \underbrace{0 \ldots 0}_{k \T*{zeros}} d_k 0 \ldots 0.
  \end{equation*}
\end{proposition}
\begin{proof}
  Bitwise xor corresponds to addition in \( \BbbF_2 \), thus \( d_k \oplus 0 = d_k \). The \( k \)-th bit of the \( k \)-th summand is \( d_k \) and for all other summands it is \( 0 \), thus the sum of the \( k \)-th bits is \( d_k \).
\end{proof}

\begin{example}\label{ex:thm:disjoint_sum_via_bitwise_xor}
  We can regard \fullref{thm:disjoint_sum_via_bitwise_xor} as a syntactic analog of evaluating an \hyperref[def:integer_radix_expansion]{integer radix expansion}.

  Indeed, if \( d_0 \ldots d_{n-1} \) encodes \( n \), we have
  \begin{equation*}
    n = \sum_{k=0}^{n-1} d_k 2^k.
  \end{equation*}

  On the other hand, the power \( 2^k \) is encoded by
  \begin{equation*}
    \underbrace{0 \ldots 0}_{k \T*{zeros}} 1 0 \ldots 0,
  \end{equation*}
  and thus \( d_k 2^k \) is encoded by
  \begin{equation*}
    \underbrace{0 \ldots 0}_{k \T*{zeros}} d_k 0 \ldots 0.
  \end{equation*}

  Summing the above for all \( k \) given an encoding of \( n \), as expected.
\end{example}
