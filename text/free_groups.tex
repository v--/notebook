\section{Free groups}\label{sec:free_groups}

\paragraph{Free monoids}

\begin{definition}\label{def:free_monoid}\mcite[48]{MacLane1998Categories}
  We associate with every \hyperref[def:set]{plain set} \( A \) its \term{free monoid} defined as the \hyperref[def:formal_language/kleene_star]{Kleene star} \( A^* \) with \hyperref[def:formal_language/concatenation]{concatenation} as the monoid operation.

  Denote by \( \iota_A: A \to A^* \) the canonical inclusion function, which sends every member of \( A \) into the corresponding single-symbol string in the \hyperref[def:free_monoid]{free monoid} \( A^* \).
\end{definition}
\begin{defproof}
  Concatenation is clearly associative and the empty string \( \bnfes \) is an \hyperref[def:monoid]{neutral element} under concatenation.
\end{defproof}

\begin{theorem}[Free monoid universal property]\label{thm:free_monoid_universal_property}
  The \hyperref[def:free_monoid]{free monoid} \( A^* \) is the unique up to an isomorphism monoid that satisfies the following \hyperref[rem:universal_mapping_property]{universal mapping property}:
  \begin{displayquote}
    For every monoid \( M \) and every function \( f: A \to M \), there exists a unique \hyperref[def:monoid/homomorphism]{monoid homomorphism} \( \widetilde{f}: A^* \to M \) such that the following diagram commutes:
    \begin{equation}\label{eq:thm:free_monoid_universal_property/diagram}
      \begin{aligned}
        \includegraphics[page=1]{output/thm__free_monoid_universal_property}
      \end{aligned}
    \end{equation}
  \end{displayquote}
\end{theorem}
\begin{comments}
  \item Via \fullref{rem:universal_mapping_property}, \( (\Anon*)^* \) becomes \hyperref[def:category_adjunction]{left adjoint} to the \hyperref[def:concrete_category]{forgetful functor}
  \begin{equation*}
    U: \cat{Mon} \to \cat{Set}.
  \end{equation*}
\end{comments}
\begin{proof}
  In order for \( \widetilde{f} \) to be a monoid homomorphism, it must satisfy \( \widetilde{f}(\bnfes) = e_M \) and
  \begin{equation*}
    \widetilde{f}(a_1 a_2 \ldots a_n) = \widetilde{f}(a_1) \cdot \widetilde{f}(a_2) \cdot \ldots \cdot \widetilde{f}(a_n).
  \end{equation*}

  Via induction on \( n \) we can see that every two such functions are equal because their action is determined entirely by the empty string and the action of \( f \) on the symbols from \( A \).

  This suggests the recursive definition
  \begin{equation*}
    \widetilde{f}(w) \coloneqq \begin{cases}
      e_M,                         &w = \bnfes, \\
      f(a) \cdot \widetilde{f}(w') &w = a \cdot w'.
    \end{cases}
  \end{equation*}
\end{proof}

\paragraph{Free groups}

\begin{definition}\label{def:free_group}\mimprovised
  Let \( A \) be a \hyperref[def:set]{plain set}. We will now construct the \term{free group} \( F(A) \) of \( A \).

  Let \( + \) and \( - \) be a arbitrary plain sets not in \( A \). Consider the \hyperref[def:disjoint_union]{disjoint union} \( U \coloneqq A \times \set{ +, - } \), whose members we will denote by \( a^+ \) and \( a^- \).

  Consider the \hyperref[def:formal_language/kleene_star]{Kleene star} \( U^* \). We say that a string in \( U^* \) is \term{reduced} if there exists no symbol \( a \in A \) such that either \( a^+ a^- \) or \( a^- a^+ \) is a substring of \( w \).

  Define via \hyperref[con:evaluation]{pattern matching} the reduction function
  \begin{equation*}
    \red(w) \coloneqq \begin{cases}
      p \cdot \red(s), &w = p a^+ a^- s \T{or} w = p a^- a^+ s, \T{where} p \T{is reduced} \\
      w,               &\T{otherwise.}
    \end{cases}
  \end{equation*}

  It is well-defined because \( s \) is always shorter than \( w \) and thus \( \red(w) \) recursively applies itself finitely many times. Furthermore, the condition for \( p \) to be reduced ensures that the leftmost pair always gets eliminated, making the process deterministic.

  Clearly \( w \) is reduced if and only if \( w = \red(w) \).

  We define the \term{free group} \( F(A) \) as the set of reduced strings over \( U \), with the operation \( v \star w \coloneqq \red(vw) \).

  The identity is the empty string and the inverse \( w^{-1} \) of \( w \) can be characterized recursively as
  \begin{equation*}
    w^{-1} = \begin{cases}
      \bnfes &w = \bnfes \\
      w^{-1} a^-  &w = a^+ v \\
      w^{-1} a^+  &w = a^- v
    \end{cases}
  \end{equation*}

  The canonical inclusion is
  \begin{equation*}
    \begin{aligned}
      &\iota_A: A \to F(A) \\
      &\iota_A(a) \coloneqq a^+.
    \end{aligned}
  \end{equation*}
\end{definition}
\begin{comments}
  \item Compare this definition to the much less syntactic definition of free abelian groups as \hyperref[def:free_semimodule]{free modules} over \( \BbbZ \).
\end{comments}

\begin{theorem}[Free group universal property]\label{thm:free_group_universal_property}
  The \hyperref[def:free_group]{free group} \( F(A) \) is the unique up to an isomorphism group that satisfies the following \hyperref[rem:universal_mapping_property]{universal mapping property}:
  \begin{displayquote}
    For every group \( G \) and every function \( f: A \to G \), there exists a unique \hyperref[def:group/homomorphism]{group homomorphism} \( \widetilde{f}: F(A) \to G \) such that the following diagram commutes:
    \begin{equation}\label{eq:thm:free_group_universal_property/diagram}
      \begin{aligned}
        \includegraphics[page=1]{output/thm__free_group_universal_property}
      \end{aligned}
    \end{equation}
  \end{displayquote}
\end{theorem}
\begin{comments}
  \item Via \fullref{rem:universal_mapping_property}, \( F \) becomes \hyperref[def:category_adjunction]{left adjoint} to the \hyperref[def:concrete_category]{forgetful functor}
  \begin{equation*}
    U: \cat{Grp} \to \cat{Set}.
  \end{equation*}
\end{comments}
\begin{proof}
  A group homomorphism is a monoid homomorphism, hence we can utilize our reasoning for free monoids to extend the definition from \fullref{thm:free_monoid_universal_property} to
  \begin{equation*}
    \widetilde{f}(w) \coloneqq \begin{cases}
      \bnfes,                      &w = \bnfes, \\
      f(a) \cdot \widetilde{f}(w')      &w = a^+ \cdot w', \\
      f(a)^{-1} \cdot \widetilde{f}(w') &w = a^- \cdot w'.
    \end{cases}
  \end{equation*}
\end{proof}

\begin{corollary}\label{thm:injective_group_homomorphisms_are_monomorphisms}
  Injective group homomorphisms are \hyperref[def:morphism_invertibility/left_cancellative]{categorical monomorphisms}.
\end{corollary}
\begin{proof}
  Follows from \fullref{thm:free_group_universal_property} and \fullref{thm:first_order_categorical_invertibility/injective}.
\end{proof}

\paragraph{Free products}

\begin{remark}\label{rem:group_presentation_syntax}
  \hyperref[def:object_presentation/cardinality]{Finitely-presented} groups have a special syntax: a group with generators \( a_1, \ldots, a_n \) and relators \( a_{i_1} \sim a_{j_1}, \ldots, a_{i_m} \sim a_{j_m} \) can be written in a syntax resembling that of \hyperref[def:group/generated]{generated subgroups}:
  \begin{equation*}
    \braket{ a_1, \ldots, a_n \given a_{i_1} = a_{j_1}, \ldots, a_{i_m} = a_{j_m} }.
  \end{equation*}

  We simply list the elements we want to see and which of them we want to be equal.
\end{remark}

\begin{example}\label{ex:free_group_with_uncountably_many_subgroups}\mcite{MathSE:countable_group_uncountably_many_distinct_subgroup}
  Consider some sequence \( x_1, x_2, \ldots \), as well as the \hyperref[def:free_group]{free group} with \hyperref[rem:group_presentation_syntax]{presentation}
  \begin{equation*}
    F_\infty = \braket{ x_1, x_2, x_3, \ldots }.
  \end{equation*}

  For any set \( N \) of indices, we also have the subgroup
  \begin{equation*}
    F_N \coloneqq \braket{ x_i \given i \in N }.
  \end{equation*}

  \Fullref{thm:cantor_power_set_theorem} implies that there is an uncountable amount of such subgroups. Some of them are isomorphic (if the sets of indices are \hyperref[def:equinumerosity]{equinumerous}), but they are nonetheless distinct.

  Therefore, \( F_\infty \) is a countable group with uncountably many subgroups.
\end{example}

\begin{definition}\label{def:group_free_product}\mcite[323]{Knapp2016BasicAlgebra}
  We define the \term{free product} of a nonempty pairwise disjoint family of groups \( \seq{ \braket{S_k \mid R_k} }_{k \in \mscrK} \) as the group with presentation
  \begin{equation*}
    \Ast_{k \in \mscrK} X_k \coloneqq \braket*{ \bigcup_{k \in \mscrK} S_k \given* \bigcup_{k \in \mscrK} R_k }.
  \end{equation*}

  If the constituent groups are not disjoint, we may instead use \hyperref[def:disjoint_union]{disjoint unions} as
  \small
  \begin{equation*}
    \Ast_{k \in \mscrK} X_k \coloneqq \braket*{ \coprod_{k \in \mscrK} S_k \given* \set[\Big]{ (k, x_1) (k, x_2) \ldots (k, x_n) \given x_1 x_2 \ldots x_n \in R_k } }.
  \end{equation*}
  \normalsize

  For every index \( m \in \mscrK \), we define the canonical embedding
  \begin{equation*}
    \begin{aligned}
       &\iota_m: X_m \to \Ast_{k \in \mscrK} X_k \\
       &\iota_m(x) \coloneqq (m, x).
    \end{aligned}
  \end{equation*}
\end{definition}

\begin{proposition}\label{thm:group_coproduct}
  The \hyperref[def:discrete_category_limits]{categorical coproduct} of the family \( \seq{ G_k }_{k \in \mscrK} \) in the category \hyperref[def:group/category]{\( \cat{Grp} \)} of groups is their \hyperref[def:group_free_product]{free product} \( \Ast_{k \in \mscrK} G_k \).
\end{proposition}
\begin{proof}
  Let \( (A, \alpha) \) be a \hyperref[def:category_of_cones/cocone]{cocone} for the discrete diagram \( \seq{ G_k }_{k \in \mscrK} \). We want to define a group homomorphism \( l: \Ast_{k \in \mscrK} G_k \to A \) such that, for every \( m \in \mscrK \),
  \begin{equation*}
    \alpha_m(x) = l_A(\iota_m(x)).
  \end{equation*}

  This suggests the definition
  \begin{equation*}
    l_A\parens[\Big]{ \iota_{k_1}(x_1) \iota_{k_2}(x_2) \ldots \iota_{k_n}(x_n) } \coloneqq \alpha_{k_1}(x_1) \cdot \alpha_{k_2}(x_k) \cdot \ldots \cdot \alpha_{k_n}(x_n).
  \end{equation*}
\end{proof}

\paragraph{Cyclic groups}

\begin{definition}\label{def:cyclic_group}\mcite[43]{Jacobson1985AlgebraPart1}
  We say that a \hyperref[def:group]{group} is \term[ru=циклическая группа (\cite[97]{Тыртышников2017Алгебра})]{cyclic} if it can be \hyperref[def:object_presentation]{generated} by a single element.

  \Fullref{thm:cyclic_group_classification} implies that, up to an isomorphism, there is only one cyclic group for every possible order. When referring to an abstract cyclic group, we will use the notation \( C_n \) for groups of finite cardinality \( n \) or \( C_\infty \) for (countably) infinite groups.
\end{definition}
\begin{comments}
  \item Given an ambient group \( G \) and some element \( g \in G \), the \term{cyclic subgroup} of \( g \) is the cyclic group isomorphic to the \hyperref[def:group/submodel]{generated subgroup} of \( G \).

  \item As shown in \fullref{thm:cyclic_group_isomorphic_to_integers_modulo_n}, cyclic groups are isomorphic to certain groups of integers, however it is still useful to have cyclic groups as a separate concept.
\end{comments}

\begin{proposition}\label{thm:cyclic_group_classification}
  Fix a \hyperref[def:cyclic_group]{cyclic group} \( G \) and a symbol \( a \) not in \( G \).

  \begin{thmenum}
    \thmitem{thm:cyclic_group_classification/finite} If \( G \) has finite cardinality \( n \), it is isomorphic to the \hyperref[def:free_group]{free group}
    \begin{equation}\label{eq:thm:cyclic_group_classification/finite}
      \braket{ a \given a^n = 1 }.
    \end{equation}

    \thmitem{thm:cyclic_group_classification/infinite} If \( G \) has infinite cardinality \( \infty \), it is isomorphic to
    \begin{equation}\label{eq:thm:cyclic_group_classification/infinite}
      \braket{ a }.
    \end{equation}
  \end{thmenum}
\end{proposition}
\begin{proof}
  Let \( G \) be a cyclic group with generator \( b \).

  \SubProofOf{thm:cyclic_group_classification/finite} If \( G \) has \( n \) elements, then \( b^n = 1 \), and thus \( a \mapsto b \) is a group isomorphism of \( \braket{ a \given a^n = 1 } \) and \( G \).

  \SubProofOf{thm:cyclic_group_classification/infinite} Suppose that \( G \) is infinite. There are only countably many powers of \( b \), one for each integer, so the cardinality of \( G \) is \( \aleph_0 \). The map \( a \mapsto b \) is an isomorphism of \( \braket{ a } \) and \( G \).
\end{proof}

\begin{proposition}\label{thm:cyclic_subgroup_classification}
  Fix a finite \hyperref[def:cyclic_group]{cyclic group} \( G \) with generator \( a \) and a subgroup \( H \) of \( G \). Let \( s \) be the smallest positive integer such that \( a^s \) is in \( H \).

  \begin{thmenum}
    \thmitem{thm:cyclic_subgroup_classification/cyclic} \( H \) is a cyclic group with generator \( a^s \).

    \thmitem{thm:cyclic_subgroup_classification/finite} If \( G \) has finite cardinality \( n \), then the cardinality \( m \) of \( H \) divides \( n \) and \( s = n / m \).

    \thmitem{thm:cyclic_subgroup_classification/infinite} If \( G \) has infinite cardinality, then so does \( H \).
  \end{thmenum}
\end{proposition}
\begin{comments}
  \item \Fullref{thm:cyclic_subgroup_characterization} provides a converse, allowing to conclude that a group is cyclic based on how many subgroups it has.
\end{comments}
\begin{proof}
  \SubProofOf{thm:cyclic_subgroup_classification/cyclic} Let \( s \) be the smallest positive integer such that \( a^s \) is in \( H \). For every member \( a^k \) of \( H \), \fullref{alg:integer_division} gives us nonnegative \( q \) and \( r \), where \( r < s \), such that
  \begin{equation*}
    k = qs + r.
  \end{equation*}

  Then
  \begin{equation*}
    a^k = (a^s)^q \cdot a^r.
  \end{equation*}

  Since \( a^s \) belongs to \( H \), so does is integer power \( a^{sq} \). Then its inverse also belongs to \( H \), as well as
  \begin{equation*}
    a^r = a^k a^{-sq}.
  \end{equation*}

  But we have assumed that \( r < s \), which contradicts the minimality of \( s \) unless \( r = 0 \).

  Therefore, every member of \( H \) is a power of \( a^s \), that is, \( H \) is cyclic with generator \( a^s \).

  \SubProofOf{thm:cyclic_subgroup_classification/finite} Suppose that \( G \) has \( n \) elements and that \( H \) has \( m \) elements.

  \Fullref{thm:lagranges_subgroup_theorem} implies that \( m \) divides \( G \).

  Since \( a^{sm} = (a^s)^m = e \), it follows that \( sm \) is a multiple of \( n \). But \( H \) is cyclic of cardinality \( m \), hence \( a^{sk} \neq e \) whenever \( 0 < k < m \). So \( sm \) is the smallest multiple of \( n \), hence \( n \) itself. Therefore, \( s = n / m \).

  \SubProofOf{thm:cyclic_subgroup_classification/infinite} Suppose that \( G \) is countable.

  Suppose also that \( H \) is finite of cardinality \( m \). Then \( a^m = e \). For every nonnegative integer \( k \), \fullref{alg:integer_division} gives us nonnegative \( q \) and \( r \) such that \( k = mq + r \) and \( 0 \leq r < m \).

  Then
  \begin{equation*}
    a^k = (a^m)^q \cdot a^r = e^q \cdot a^r = a^r.
  \end{equation*}

  Therefore, \( G \) itself must have only \( m \) elements. The obtained contradiction shows that \( H \) is infinite.
\end{proof}

\begin{proposition}\label{thm:def:cyclic_group}
  \hyperref[def:cyclic_group]{Cyclic groups} have the following basic properties:
  \begin{thmenum}
    \thmitem{thm:def:cyclic_group/direct_sum} The \hyperref[def:semimodule_direct_sum]{direct sum} \( C_n \oplus C_m \) of two cyclic groups is cyclic if and only if \( n \) and \( m \) are \hyperref[def:coprime_elements]{coprime}.

    \thmitem{thm:def:cyclic_group/generators} A member of \( C_n \) generates it if and only if it has \hyperref[def:group_element_order]{order} \( n \).

    \thmitem{thm:def:cyclic_group/generators_cardinality} The set of \hyperref[def:object_presentation]{generators} of \( C_n \) has cardinality \( \varphi(n) \), where \( \varphi \) is \hyperref[def:eulers_totient_function]{Euler's totient function}.
  \end{thmenum}
\end{proposition}
\begin{proof}
  \SubProofOf{thm:def:cyclic_group/direct_sum} Let \( a \) be a generator of \( C_n \) and \( b \) --- of \( C_m \).

  \SufficiencySubProof* Suppose that \( (a^i, b^j) \) generates \( C_n \oplus C_m \).

  Let \( d \) be a common divisor of \( n \) and \( m \). \Fullref{thm:common_divisor_to_multiple_lemma} implies that \( s \coloneqq nm / d \) is a common multiple. \Fullref{thm:def:group_element_order/neutral} implies that \( a^s = a^0 \) and \( b^s = b^0 \).

  Then
  \begin{equation*}
    (a^i, b^j)^s = (a^{is}, b^{js}) = (a^0, b^0).
  \end{equation*}

  The order of \( (a^i, b^j) \) is thus at most \( s \).

  But it is a generator of \( C_n \oplus C_m \), hence its order is \( nm \). Then \( d = 1 \) and \( n \) and \( m \) are coprime.

  \NecessitySubProof* Conversely, suppose that \( n \) and \( m \) are coprime.

  Suppose that, for some \( j < i \) we have
  \begin{equation*}
    (a, b)^i = (a, b)^j.
  \end{equation*}

  \Fullref{thm:def:group_element_order/neutral} implies that both \( n \) and \( m \) divide \( i - j \). Then their \hyperref[def:lcm]{least common multiple} \( l \) also divides \( i - j \). But since \( n \) and \( m \) are coprime, \( l = nm \), and thus \( nm \) divides \( i - j \).

  Hence, if \( 0 \leq j < i < nm \),
  \begin{equation*}
    (a, b)^i \neq (a, b)^j.
  \end{equation*}

  Then the direct sum \( C_n \oplus C_m \) has at least \( nm \) elements. But by definition it has at most \( nm \) elements. Therefore, \( C_n \oplus C_m \) is a cyclic group of order \( nm \) generated by \( (a, b) \).

  \SubProofOf{thm:def:cyclic_group/generators} By definition of \( \ord(a) \), the subgroup \( \braket{ a } \) has \( \ord(a) \) distinct elements.

  Then \( x \) is a generator of \( C_n \) if and only if \( \ord(a) \) coincides with the cardinality \( n \) of \( C_n \).

  \SubProofOf{thm:def:cyclic_group/generators_cardinality} Let \( a \) be a generator of \( C_n \). \Fullref{thm:def:group_element_order/power} implies that the order of \( a^m \) is \( n / \gcd(n, m) \). This order equals \( n \) if and only if \( n \) and \( m \) are coprime. Therefore, there are \( \varphi(n) \) generators of \( C_n \).
\end{proof}
