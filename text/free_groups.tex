\subsection{Free groups}\label{subsec:free_groups}

\paragraph{Free monoids}

\begin{definition}\label{def:free_monoid}\mcite[48]{MacLane1998}
  We associate with every \hyperref[def:set]{plain set} \( A \) its \term{free monoid} as the \hyperref[def:formal_language/kleene_star]{Kleene star} \( A^* \) with \hyperref[def:formal_language/concatenation]{concatenation} as the monoid operation.

  Denote by \( \iota_A: A \to A^* \) the canonical inclusion function, which sends every member of \( A \) into the corresponding single-symbol word in the \hyperref[def:free_monoid]{free monoid} \( A^* \).
\end{definition}
\begin{defproof}
  Concatenation is clearly associative and the empty word \( \varepsilon \) is an \hyperref[def:monoid]{neutral element} under concatenation.
\end{defproof}

\begin{theorem}[Free monoid universal property]\label{thm:free_monoid_universal_property}
  The \hyperref[def:free_monoid]{free monoid} \( A^* \) is the unique up to an isomorphism monoid that satisfies the following \hyperref[rem:universal_mapping_property]{universal mapping property}:
  \begin{displayquote}
    For every monoid \( M \) and every function \( f: A \to M \), there exists a unique \hyperref[def:monoid/homomorphism]{monoid homomorphism} \( \widetilde{f}: A^* \to M \) such that the following diagram commutes:
    \begin{equation}\label{eq:thm:free_monoid_universal_property/diagram}
      \begin{aligned}
        \includegraphics[page=1]{output/thm__free_monoid_universal_property}
      \end{aligned}
    \end{equation}
  \end{displayquote}
\end{theorem}
\begin{comments}
  \item Via \fullref{rem:universal_mapping_property}, \( (\Anon*)^* \) becomes \hyperref[def:category_adjunction]{left adjoint} to the \hyperref[def:concrete_category]{forgetful functor}
  \begin{equation*}
    U: \cat{Mon} \to \cat{Set}.
  \end{equation*}
\end{comments}
\begin{proof}
  In order for \( \widetilde{f} \) to be a monoid homomorphism, it must satisfy \( \widetilde{f}(\varepsilon) = e_M \) and
  \begin{equation*}
    \widetilde{f}(a_1 a_2 \ldots a_n) = \widetilde{f}(a_1) \cdot \widetilde{f}(a_2) \cdot \ldots \cdot \widetilde{f}(a_n).
  \end{equation*}

  Every two such functions are equal because their action is determined entirely by the empty word and the action of \( f \) on the symbols from \( A \).

  This suggests the recursive definition
  \begin{equation*}
    \widetilde{f}(w) \coloneqq \begin{cases}
      e_M,                         &w = \varepsilon, \\
      f(a) \cdot \widetilde{f}(w') &w = a \cdot w'.
    \end{cases}
  \end{equation*}
\end{proof}

\paragraph{Free groups}

\begin{definition}\label{def:free_group}\mimprovised
  Let \( A \) be a \hyperref[def:set]{plain set}. We will now construct the \term{free group} \( F(A) \) of \( A \).

  Let \( + \) and \( - \) be a arbitrary plain sets not in \( A \). Consider the \hyperref[def:disjoint_union]{disjoint union} \( U \coloneqq A \times \set{ +, - } \), whose members we will denote by \( a^+ \) and \( a^- \).

  Consider the \hyperref[def:formal_language/kleene_star]{Kleene star} \( U^* \). We say that a word in \( U^* \) is \term{reduced} if there exists no symbol \( a \in A \) such that either \( a^+ a^- \) or \( a^- a^+ \) is a subword of \( w \).

  Define via \hyperref[rem:evaluation]{pattern matching} the reduction function
  \begin{equation*}
    \red(w) \coloneqq \begin{cases}
      p \cdot \red(s), &w = p a^+ a^- s \T{or} w = p a^- a^+ s, \T{where} p \T{is reduced} \\
      w,               &\T{otherwise.}
    \end{cases}
  \end{equation*}

  It is well-defined because \( s \) is always shorter than \( w \) and thus \( \red(w) \) recursively applies itself finitely many times. Furthermore, the condition for \( p \) to be reduced ensures that the leftmost pair always gets eliminated, making the process deterministic.

  Clearly \( w \) is reduced if and only if \( w = \red(w) \).

  We define the \term{free group} \( F(A) \) as the set of reduced words over \( U \), with the operation \( v \star w \coloneqq \red(vw) \).

  The identity is the empty word and the inverse \( w^{-1} \) of \( w \) can be characterized recursively as
  \begin{equation*}
    w^{-1} = \begin{cases}
      \varepsilon &w = \varepsilon \\
      w^{-1} a^-  &w = a^+ v \\
      w^{-1} a^+  &w = a^- v
    \end{cases}
  \end{equation*}

  The canonical inclusion is
  \begin{equation*}
    \begin{aligned}
      &\iota_A: A \to F(A) \\
      &\iota_A(a) \coloneqq a^+.
    \end{aligned}
  \end{equation*}
\end{definition}
\begin{comments}
  \item Compare this definition to the much less syntactic definition of free abelian groups as \hyperref[def:free_semimodule]{free modules} over \( \BbbZ \).
\end{comments}

\begin{theorem}[Free group universal property]\label{thm:free_group_universal_property}
  The \hyperref[def:free_group]{free group} \( F(A) \) is the unique up to an isomorphism group that satisfies the following \hyperref[rem:universal_mapping_property]{universal mapping property}:
  \begin{displayquote}
    For every group \( G \) and every function \( f: A \to G \), there exists a unique \hyperref[def:group/homomorphism]{group homomorphism} \( \widetilde{f}: F(A) \to G \) such that the following diagram commutes:
    \begin{equation}\label{eq:thm:free_group_universal_property/diagram}
      \begin{aligned}
        \includegraphics[page=1]{output/thm__free_group_universal_property}
      \end{aligned}
    \end{equation}
  \end{displayquote}
\end{theorem}
\begin{comments}
  \item Via \fullref{rem:universal_mapping_property}, \( F \) becomes \hyperref[def:category_adjunction]{left adjoint} to the \hyperref[def:concrete_category]{forgetful functor}
  \begin{equation*}
    U: \cat{Grp} \to \cat{Set}.
  \end{equation*}
\end{comments}
\begin{proof}
  A group homomorphism is a monoid homomorphism, hence we can utilize our reasoning for free monoids to extend the definition from \fullref{thm:free_monoid_universal_property} to
  \begin{equation*}
    \widetilde{f}(w) \coloneqq \begin{cases}
      \varepsilon,                      &w = \varepsilon, \\
      f(a) \cdot \widetilde{f}(w')      &w = a^+ \cdot w', \\
      f(a)^{-1} \cdot \widetilde{f}(w') &w = a^- \cdot w'.
    \end{cases}
  \end{equation*}
\end{proof}

\begin{corollary}\label{thm:injective_group_homomorphisms_are_monomorphisms}
  Injective group homomorphisms are \hyperref[def:morphism_invertibility/left_cancellative]{categorical monomorphisms}.
\end{corollary}
\begin{proof}
  Follows from \fullref{thm:free_group_universal_property} and \fullref{thm:first_order_categorical_invertibility/injective}.
\end{proof}

\paragraph{Group presentations}

\begin{definition}\label{def:group_presentation}\mcite[314]{Knapp2016BasicAlgebra}
  Let \( A \) be a \hyperref[def:set]{plain set} with \hyperref[def:free_group]{free group} \( F(A) \) and let \( R \subseteq F(A) \) be a set of words in \( F(A) \). Denote by \( N(R) \) the \hyperref[def:normal_closure]{normal closure} \( R \).

  Then we can define the group with a set of \term{generators} \( A \) and a set of \term{relators} \( R \) as
  \begin{equation}\label{eq:def:group_presentation/presentation}
    \braket{ A \mid R } \coloneqq F(A) / N(R).
  \end{equation}

  If \( R = \varnothing \), there are no relators, and we use the following notation for the free group:
  \begin{equation}\label{eq:def:group_presentation/free}
    \braket{ A } \coloneqq F(A)
  \end{equation}

  Note that we use similar notation compared to generated subgroups discussed in \fullref{def:group/submodel}. The former defines a new group operation from scratch, while the latter uses an existing group operation and is restricted by this operation.

  If, for a given group \( G \) and subsets \( A \) and \( R \) of \( G \), we have
  \begin{equation*}
    G \cong \braket{ A \given R },
  \end{equation*}
  we call the pair \( (A, R) \) a \term{presentation} of \( G \).

  We say that \( G \) is \term{finitely generated} if both \( A \) and \( R \) are \hyperref[def:set_finiteness]{finite sets}; if only \( R \) is finite, we call \( G \) \term{finitely presented}.
\end{definition}
\begin{comments}
  \item Compare this to \hyperref[def:module_presentation]{module presentations} and \hyperref[def:algebra_presentation]{algebra presentations}.
\end{comments}

\begin{proposition}\label{thm:group_presentation_existence}\mcite[prop. 7.7]{Knapp2016BasicAlgebra}
  Every group has at least one \hyperref[def:group_presentation]{presentation}.
\end{proposition}
\begin{proof}
  Let \( U: \cat{Grp} \to \cat{Set} \) be the \hyperref[def:concrete_category]{forgetful functor} sending a group to its domain. Fix an arbitrary group \( G \) and let \( A \coloneqq U(G) \) be the underlying set.

  By \fullref{thm:free_group_universal_property}, there exists a unique group homomorphism \( \varphi: F(A) \to G \) such that
  \begin{equation*}
    U(\varphi) \bincirc \iota_A = \id_A.
  \end{equation*}

  By \fullref{thm:kernel_is_normal_subgroup}, the kernel \( \ker \varphi \) is a normal subgroup of \( A \), hence by \fullref{thm:quotient_structure_universal_property},
  \begin{equation*}
    G = \varphi[F(A)] \cong F(A) / \ker \varphi = \braket{ A \mid \ker \varphi }.
  \end{equation*}
\end{proof}

\begin{example}\label{ex:free_group_with_uncountably_many_subgroups}\mcite{MathSE:countable_group_uncountably_many_distinct_subgroup}
  Consider some sequence \( x_1, x_2, \ldots \), as well as the \hyperref[def:free_group]{free group} with \hyperref[def:group_presentation]{presentation}
  \begin{equation*}
    F_\infty = \braket{ x_1, x_2, x_3, \ldots }.
  \end{equation*}

  For any set \( N \) of indices, we also have the subgroup
  \begin{equation*}
    F_N \coloneqq \braket{ x_i \given i \in N }.
  \end{equation*}

  \Fullref{thm:cantor_power_set_theorem} implies that there is an uncountable amount of such subgroups. Some of them are isomorphic (if the sets of indices are \hyperref[def:equinumerosity]{equinumerous}), but they are nonetheless distinct.

  Therefore, \( F_\infty \) is a countable group with uncountably many subgroups.
\end{example}

\begin{definition}\label{def:group_free_product}\mcite[323]{Knapp2016BasicAlgebra}
  We define the \term{free product} of a nonempty pairwise disjoint family of groups \( \seq{ \braket{S_k \mid R_k} }_{k \in \mscrK} \) as the group with presentation
  \begin{equation*}
    \Ast_{k \in \mscrK} X_k \coloneqq \braket*{ \bigcup_{k \in \mscrK} S_k \given* \bigcup_{k \in \mscrK} R_k }.
  \end{equation*}

  If the constituent groups are not disjoint, we may instead use \hyperref[def:disjoint_union]{disjoint unions} as
  \small
  \begin{equation*}
    \Ast_{k \in \mscrK} X_k \coloneqq \braket*{ \coprod_{k \in \mscrK} S_k \given* \set[\Big]{ (k, x_1) (k, x_2) \ldots (k, x_n) \given x_1 x_2 \ldots x_n \in R_k } }.
  \end{equation*}
  \normalsize

  For every index \( m \in \mscrK \), we define the canonical embedding
  \begin{equation*}
    \begin{aligned}
       &\iota_m: X_m \to \Ast_{k \in \mscrK} X_k \\
       &\iota_m(x) \coloneqq (m, x).
    \end{aligned}
  \end{equation*}
\end{definition}

\begin{proposition}\label{thm:group_coproduct}
  The \hyperref[def:discrete_category_limits]{categorical coproduct} of the family \( \seq{ G_k }_{k \in \mscrK} \) in the category \hyperref[def:group/category]{\( \cat{Grp} \)} of groups is their \hyperref[def:group_free_product]{free product} \( \Ast_{k \in \mscrK} G_k \).
\end{proposition}
\begin{proof}
  Let \( (A, \alpha) \) be a \hyperref[def:category_of_cones/cocone]{cocone} for the discrete diagram \( \seq{ G_k }_{k \in \mscrK} \). We want to define a group homomorphism \( l: \Ast_{k \in \mscrK} G_k \to A \) such that, for every \( m \in \mscrK \),
  \begin{equation*}
    \alpha_m(x) = l_A(\iota_m(x)).
  \end{equation*}

  This suggests the definition
  \begin{equation*}
    l_A\parens[\Big]{ (k_1, x_1) (k_2, x_2) \ldots (k_n, x_n) } \coloneqq \alpha_{k_1}(x_1) \cdot \alpha_{k_2}(x_k) \cdot \ldots \cdot \alpha_{k_n}(x_n).
  \end{equation*}
\end{proof}

\paragraph{Cyclic groups}

\begin{definition}\label{def:cyclic_group}\mcite[def. II.4.7]{Aluffi2009}
  For a singleton alphabet \( \set{ a } \), we define the \term[ru=(бесконечная) циклическая группа (\cite[sec. 2.11]{Тыртышников2007})]{infinite cyclic group} as
  \begin{equation}\label{def:cyclic_group/infinite}
    C_\infty \coloneqq \braket{ a }
  \end{equation}
  and, for positive integers \( n \), the \term{finite cyclic group} of \term{order} \( n \) as
  \begin{equation}\label{def:cyclic_group/finite}
    C_n \coloneqq \braket{ a \given a^n }.
  \end{equation}

  We use the same notation with no regard to the alphabet because all cyclic groups of the same order are \hyperref[def:group/homomorphism]{isomorphic}.
\end{definition}
\begin{comments}
  \item Given an ambient group \( G \) and some element \( g \in G \), the \term{cyclic subgroup} of \( g \) is the cyclic group isomorphic to the \hyperref[def:group/submodel]{generated subgroup} of \( G \).

  \item As shown in \fullref{thm:cyclic_group_isomorphic_to_integers_modulo_n}, cyclic groups are isomorphic to certain groups of integers, however it is still useful to have cyclic groups as a separate concept.
\end{comments}

\begin{proposition}\label{thm:def:cyclic_group}
  \hyperref[def:cyclic_group]{Cyclic groups} have the following basic properties:
  \begin{thmenum}
    \thmitem{thm:def:cyclic_group/subgroups} The \hyperref[def:cyclic_group]{cyclic group} \( C_n \) has a subgroup of order \( m \) if and only if \( m \) divides \( n \).
    \thmitem{thm:def:cyclic_group/direct_sum} The \hyperref[def:first_order_direct_product]{direct sum} \( C_m \oplus C_n \) of two \hyperref[def:cyclic_group]{cyclic groups} is cyclic if and only if \( m \) and \( n \) are \hyperref[def:coprime_elements]{coprime}.
  \end{thmenum}
\end{proposition}
\begin{proof}
  \SubProofOf{thm:def:cyclic_group/subgroups}
  \SufficiencySubProof* If \( H \) is a subgroup of \( C_n = \braket{ a \given a^n } \), by \fullref{thm:lagranges_subgroup_theorem}, its order divides \( n \).

  \NecessitySubProof* Conversely, let \( m \mid n \). Then the following is a cyclic subgroup of exactly \( m \) elements:
  \begin{equation*}
    \set{ e, a^{\ifrac n m}, \cdots, a^{(m - 1) \cdot \ifrac n m} }.
  \end{equation*}

  \SubProofOf{thm:def:cyclic_group/direct_sum} Consider the product \( C_m \times C_n \), where \( a \) is the generator of \( C_m \) and \( b \) is the generator of \( C_n \). The order of the element \( (a, e) \) is \( m \) and the order of \( (e, b) \) is \( n \). The order of
  \begin{equation*}
    (a, b) = (a, e) \cdot (e, b)
  \end{equation*}
  is the least common multiple of \( m \) and \( n \), which equals \( mn \) if and only if \( m \) and \( n \) are coprime.
\end{proof}

\begin{definition}\label{def:group_element_order}\mcite[130]{Knapp2016BasicAlgebra}
  The \term[ru=степень (\cite[sec. 2.10]{Тыртышников2007})]{order} \( \ord(x) \) of a member \( x \) of a group is the smallest positive integer \( n \) such that \( x^n = e \), i.e. order of the \hyperref[def:cyclic_group]{cyclic subgroup} \hyperref[def:group/submodel]{generated} by \( x \).
\end{definition}

\begin{proposition}\label{thm:def:group_element_order}
  For finite groups, the orders of any group element in the sense of \fullref{def:group_element_order} divides the order of the group in the sense of \fullref{def:group_order}.
\end{proposition}
\begin{proof}
  Follows from \fullref{thm:lagranges_subgroup_theorem}.
\end{proof}
