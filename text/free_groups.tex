\section{Free groups}\label{sec:free_groups}

\paragraph{Free monoids}

\begin{definition}\label{def:free_monoid}\mcite[48]{MacLane1998CategoryTheory}
  We associate with every \hyperref[def:set]{plain set} \( A \) its \term{free monoid} defined as the \hyperref[def:formal_language/kleene_star]{Kleene star} \( A^* \) with \hyperref[def:formal_language/concatenation]{concatenation} as the monoid operation.

  Denote by \( \iota_A: A \to A^* \) the canonical inclusion function, which sends every member of \( A \) into the corresponding single-symbol string in the \hyperref[def:free_monoid]{free monoid} \( A^* \).
\end{definition}
\begin{comments}
  \item Free monoids allow describing the possible operations performed in a monoid, and \fullref{thm:free_monoid_universal_property} allows to actually perform these operations in some monoid. Thus, free monoids are an instance of the \hyperref[con:syntax_semantics_duality]{syntax-semantics duality}.
\end{comments}
\begin{defproof}
  Concatenation is clearly associative and the empty string \( \bnfes \) is an \hyperref[def:monoid]{neutral element} under concatenation.
\end{defproof}

\begin{theorem}[Free monoid universal property]\label{thm:free_monoid_universal_property}
  The \hyperref[def:free_monoid]{free monoid} \( A^* \) is the unique up to an isomorphism monoid that satisfies the following \hyperref[rem:universal_mapping_property]{universal mapping property}:
  \begin{displayquote}
    For every monoid \( M \) and every function \( f: A \to M \), there exists a unique \hyperref[def:monoid/homomorphism]{monoid homomorphism} \( \widetilde{f}: A^* \to M \) such that the following diagram commutes:
    \begin{equation}\label{eq:thm:free_monoid_universal_property/diagram}
      \begin{aligned}
        \includegraphics[page=1]{output/thm__free_monoid_universal_property}
      \end{aligned}
    \end{equation}
  \end{displayquote}
\end{theorem}
\begin{comments}
  \item Via \fullref{rem:universal_mapping_property}, \( (\Anon*)^* \) becomes \hyperref[def:category_adjunction]{left adjoint} to the \hyperref[def:concrete_category]{forgetful functor}
  \begin{equation*}
    U: \cat{Mon} \to \cat{Set}.
  \end{equation*}
\end{comments}
\begin{proof}
  In order for \( \widetilde{f} \) to be a monoid homomorphism, it must satisfy \( \widetilde{f}(\bnfes) = e_M \) and
  \begin{equation*}
    \widetilde{f}(a_1 a_2 \ldots a_n) = \widetilde{f}(a_1) \cdot \widetilde{f}(a_2) \cdot \ldots \cdot \widetilde{f}(a_n).
  \end{equation*}

  Via induction on \( n \) we can see that every two such functions are equal because their action is determined entirely by the empty string and the action of \( f \) on the symbols from \( A \).

  This suggests the recursive definition
  \begin{equation*}
    \widetilde{f}(w) \coloneqq \begin{cases}
      e_M,                         &w = \bnfes, \\
      f(a) \cdot \widetilde{f}(w') &w = a \cdot w'.
    \end{cases}
  \end{equation*}
\end{proof}

\begin{proposition}\label{thm:def:free_monoid}
  \hyperref[def:free_monoid]{Free monoids} have the following basic properties:
  \begin{thmenum}
    \thmitem{thm:def:free_monoid/cancellative} Every free monoid is \hyperref[def:binary_operation/cancellative]{cancellative}.
  \end{thmenum}
\end{proposition}
\begin{proof}
  \SubProofOf{thm:def:free_monoid/cancellative} Consider the free monoid \( F(A) \) and suppose that \( uw = vw \) for some words in \( F(A) \). We will use induction on the length of \( w \), simultaneously for all \( u \) and \( v \), to show that we can cancel \( w \).
  \begin{itemize}
    \item If \( w \) is the empty string, clearly \( u = v \).
    \item Suppose that we can cancel strings of length \( n - 1 \), and that \( w \) has length \( n \). Let \( w = ps \), where both \( p \) and \( s \) are nonempty. Then both have length less than \( n \), hence we can cancel \( s \) in \( ups = vps \), and then cancel \( p \) in \( up = vp \).
  \end{itemize}

  We can similarly show that \( wu = wv \) implies \( u = v \).
\end{proof}

\paragraph{Free products}

\begin{definition}\label{def:monoid_free_product}\mimprovised
  Fix a family \( \seq{ M_k }_{k \in \mscrK} \) of \hyperref[def:monoid]{monoids}. We will construct a monoid \( \Ast_{k \in \mscrK} M_k \), which we will call the \term[ru=свободное произведение (\cite[def. X.5.2]{Ляпин1960Полугруппы}), en=free product (\cite[323]{Knapp2016BasicAlgebra})]{free product}, based on the (set-theoretic) \hyperref[def:disjoint_union]{disjoint union} of the individual monoids. In \fullref{thm:monoid_coproduct}, we will show that it is their \hyperref[def:discrete_category_limits]{categorical coproduct}.

  For each \( k \) in \( \mscrK \), denote by \( \delta_k \) the inclusion of \( M_k \) into the disjoint union \( \coprod_{k \in \mscrK} M_k \) and by \( \varphi_k \) --- the inclusion into \( F(M_k) \). Similarly, denote by \( \varphi \) the inclusion of \( \coprod_{k \in \mscrK} M_k \) into its \hyperref[def:free_monoid]{free monoid}.

  \Fullref{thm:free_monoid_universal_property} applied to \( \varphi \bincirc \delta_k \) gives us a homomorphism \( \zeta_k \), and applied to \( \id_{M_k} \) gives us a homomorphism \( \psi_k \) such that the following diagram commutes:
  \begin{equation*}
    \begin{aligned}
      \includegraphics[page=1]{output/def__monoid_free_product}
    \end{aligned}
  \end{equation*}

  Define the relation \( {\sim} \) on \( F(\coprod_{k \in \mscrK} M_k) \) via the following rules (one rule for each \( k \) in \( \mscrK \)):
  \begin{equation*}\taglabel[\ensuremath{ \logic{FreeProd}_k }]{inf:def:monoid_free_product}
    \begin{prooftree}
      \hypo{ v \in F(M_k) }
      \hypo{ w \in F(M_k) }
      \hypo{ \psi_k(v) = \psi_k(w) }
      \infer3[\ref{inf:def:monoid_free_product}]{ \zeta_k(v) \sim \zeta_k(w) }.
    \end{prooftree}
  \end{equation*}

  Thus, if the strings \( v \) and \( w \) from \( F(M_k) \) evaluate to the same value in \( M_k \), we want their embeddings into \( F(\coprod_{k \in \mscrK} M_k) \) to also evaluate to the same value.

  Finally, we define the free product itself as the \hyperref[def:first_order_quotient]{quotient}
  \begin{equation}\label{eq:def:monoid_free_product/product}
    \Ast_{k \in \mscrK} M_k \coloneqq F\parens[\Big]{ \coprod_{k \in \mscrK} A_k } / {\cong},
  \end{equation}
  where \( {\cong} \) is the congruence \hyperref[def:first_order_generated_congruence]{generated} by \( {\sim} \).

  For every index \( k \) in \( \mscrK \), we define the canonical embedding \( \iota: M_k \to \Ast_{k \in \mscrK} M_k \) as the composition
  \begin{equation}\label{eq:def:monoid_free_product/embedding}
    \begin{aligned}
      \includegraphics[page=2]{output/def__monoid_free_product}
    \end{aligned}
  \end{equation}
\end{definition}
\begin{comments}
  \item We base our construction on \enquote{free semimonoid products} discussed by \incite[def. X.5.2]{Ляпин1960Полугруппы}, but with many formal details added.
\end{comments}

\begin{proposition}\label{thm:monoid_coproduct}
  The \hyperref[def:discrete_category_limits]{categorical coproduct} of the family \( \seq{ M_k }_{k \in \mscrK} \) in the category \hyperref[def:monoid/category]{\( \cat{Mon} \)} of monoids is their \hyperref[def:monoid_free_product]{free product} \( \Ast_{k \in \mscrK} M_k \).
\end{proposition}
\begin{proof}
  Let \( (H, \alpha) \) be a \hyperref[def:category_of_cones/cocone]{cocone} for the discrete diagram \( \seq{ M_k }_{k \in \mscrK} \). We need to define a homomorphism \( l_H: \Ast_{k \in \mscrK} M_k \to H \) such that the following diagram commutes:
  \begin{equation*}
    \begin{aligned}
      \includegraphics[page=1]{output/thm__monoid_coproduct}
    \end{aligned}
  \end{equation*}

  Starting with \( \varphi_k \), we use \fullref{thm:free_monoid_universal_property} to obtain a homomorphism \( \chi_k \), from it again via \fullref{thm:free_monoid_universal_property} --- a homomorphism \( \chi \), and, finally, via \fullref{thm:quotient_structure_universal_property} --- a homomorphism \( l_H \) such that the following diagram commutes:
  \begin{equation*}
    \begin{aligned}
      \includegraphics[page=2]{output/thm__monoid_coproduct}
    \end{aligned}
  \end{equation*}

  We have obtained the desired homomorphism \( l_H \), however we need to justify our application of \fullref{thm:quotient_structure_universal_property} --- to show that \( v \cong w \) implies \( \chi(v) = \chi(w) \).

  To prove this, we will use \fullref{thm:induction_on_recursively_defined_relations} on \( v \cong w \):
  \begin{itemize}
    \item If \( v \cong w \) due to \ref{inf:def:first_order_generated_congruence/base}, then \( v \sim w \), and \ref{inf:def:monoid_free_product} implies that there exist strings \( v' \) and \( w' \) in \( F(M_k) \)  such that \( v = \zeta_k(v') \), \( w = \zeta_k(w') \) and \( \psi_k(v') = \psi_k(w') \), where \( \psi_k \) is the left inverse of \( \varphi_k \) built in \fullref{def:monoid_free_product}.

    Then
    \begin{equation*}
      \chi(v)
      =
      \chi(\zeta_k(v'))
      =
      \chi_k(v')
      =
      \alpha_k(\psi_k(v')).
    \end{equation*}

    The gist of this equality is that the string \( v' \) of members of \( G_k \), when evaluated by \( \chi_k \), gives the same result as evaluating the symbols in \( G_k \) and then mapping them via \( \alpha_k \). The string \( v \) is simply an inclusion of \( v' \) into the free monoid of the disjoint union of the individual monoids.

    Since \( \psi_k(v') = \psi_k(w') \), this allows us to conclude that \( \chi(v) = \chi(w) \).

    \item Otherwise, if \( v \cong w \) due to \ref{inf:def:first_order_generated_congruence/rec}, and there exist elements \( v_1 \), \( v_2 \), \( w_1 \) and \( w_2 \) of \( F(\coprod_{k \in \mscrK} M_k) \) such that \( v = v_1 \cdot v_2 \) and \( v_1 \cong v_2 \), and similarly \( w = w_1 \cdot w_2 \) and \( w_1 \cong w_2 \).

    Suppose that the inductive hypothesis holds for \( v_1 \cong v_2 \) and \( w_1 \cong w_2 \). Then
    \begin{equation*}
      \chi(v) = \chi(v_1) \cdot \chi(v_2) = \chi(w_1) \cdot \chi(w_2) = \chi(w).
    \end{equation*}
  \end{itemize}

  Finally, uniqueness of \( l_H \) follows from uniqueness given by \fullref{thm:quotient_structure_universal_property}. This concludes the proof.
\end{proof}

\paragraph{Free groups}

\begin{definition}\label{def:free_group}\mimprovised
  Let \( A \) be a \hyperref[def:set]{plain set}. We will now construct the \term{free group} \( F(A) \) of \( A \).

  Let \( + \) and \( - \) be a arbitrary plain sets not in \( A \). Consider the \hyperref[def:disjoint_union]{disjoint union} \( U \coloneqq A \times \set{ +, - } \), whose members we will denote by \( a^+ \) and \( a^- \).

  Consider the \hyperref[def:formal_language/kleene_star]{Kleene star} \( U^* \). We say that a string in \( U^* \) is \term{reduced} if there exists no symbol \( a \in A \) such that either \( a^+ a^- \) or \( a^- a^+ \) is a substring of \( w \).

  Define via \hyperref[con:evaluation]{pattern matching} the reduction function
  \begin{equation*}
    \red(w) \coloneqq \begin{cases}
      p \cdot \red(s), &w = p a^+ a^- s \T{or} w = p a^- a^+ s, \T{where} p \T{is reduced} \\
      w,               &\T{otherwise.}
    \end{cases}
  \end{equation*}

  It is well-defined because \( s \) is always shorter than \( w \) and thus \( \red(w) \) recursively applies itself finitely many times. Furthermore, the condition for \( p \) to be reduced ensures that the leftmost pair always gets eliminated, making the process deterministic.

  Clearly \( w \) is reduced if and only if \( w = \red(w) \).

  We define the \term{free group} \( F(A) \) as the set of reduced strings over \( U \), with the operation \( v \star w \coloneqq \red(vw) \).

  The identity is the empty string and the inverse \( w^{-1} \) of \( w \) can be characterized recursively as
  \begin{equation*}
    w^{-1} = \begin{cases}
      \bnfes &w = \bnfes \\
      w^{-1} a^-  &w = a^+ v \\
      w^{-1} a^+  &w = a^- v
    \end{cases}
  \end{equation*}

  The canonical inclusion is
  \begin{equation*}
    \begin{aligned}
      &\iota_A: A \to F(A) \\
      &\iota_A(a) \coloneqq a^+.
    \end{aligned}
  \end{equation*}
\end{definition}
\begin{comments}
  \item Compare this definition to the much less syntactic definition of free abelian groups as \hyperref[def:free_semimodule]{free modules} over \( \BbbZ \).
\end{comments}

\begin{theorem}[Free group universal property]\label{thm:free_group_universal_property}
  The \hyperref[def:free_group]{free group} \( F(A) \) is the unique up to an isomorphism group that satisfies the following \hyperref[rem:universal_mapping_property]{universal mapping property}:
  \begin{displayquote}
    For every group \( G \) and every function \( f: A \to G \), there exists a unique \hyperref[def:group/homomorphism]{group homomorphism} \( \widetilde{f}: F(A) \to G \) such that the following diagram commutes:
    \begin{equation}\label{eq:thm:free_group_universal_property/diagram}
      \begin{aligned}
        \includegraphics[page=1]{output/thm__free_group_universal_property}
      \end{aligned}
    \end{equation}
  \end{displayquote}
\end{theorem}
\begin{comments}
  \item Via \fullref{rem:universal_mapping_property}, \( F \) becomes \hyperref[def:category_adjunction]{left adjoint} to the \hyperref[def:concrete_category]{forgetful functor}
  \begin{equation*}
    U: \cat{Grp} \to \cat{Set}.
  \end{equation*}
\end{comments}
\begin{proof}
  A group homomorphism is a monoid homomorphism, hence we can utilize our reasoning for free monoids to extend the definition from \fullref{thm:free_monoid_universal_property} to
  \begin{equation*}
    \widetilde{f}(w) \coloneqq \begin{cases}
      \bnfes,                      &w = \bnfes, \\
      f(a) \cdot \widetilde{f}(w')      &w = a^+ \cdot w', \\
      f(a)^{-1} \cdot \widetilde{f}(w') &w = a^- \cdot w'.
    \end{cases}
  \end{equation*}
\end{proof}

\begin{corollary}\label{thm:injective_group_homomorphisms_are_monomorphisms}
  Injective group homomorphisms are \hyperref[def:morphism_invertibility/left_cancellative]{categorical monomorphisms}.
\end{corollary}
\begin{proof}
  Follows from \fullref{thm:free_group_universal_property} and \fullref{thm:first_order_categorical_invertibility/injective}.
\end{proof}

\begin{example}\label{ex:free_group_with_uncountably_many_subgroups}\mcite{MathSE:countable_group_uncountably_many_distinct_subgroup}
  Consider some sequence \( x_1, x_2, \ldots \), and denote by \( F_\infty \) the \hyperref[def:free_group]{free group} generated by it.

  For any set \( N \) of indices, we also have the subgroup
  \begin{equation*}
    F_N \coloneqq \braket{ x_i \given i \in N }.
  \end{equation*}

  \Fullref{thm:cantor_power_set_theorem} implies that there is an uncountable amount of such subgroups. Some of them are isomorphic (if the sets of indices are \hyperref[def:equinumerosity]{equinumerous}), but they are nonetheless distinct.

  Therefore, \( F_\infty \) is a countable group with uncountably many subgroups.
\end{example}

\begin{proposition}\label{thm:group_coproduct}
  The \hyperref[def:discrete_category_limits]{categorical coproduct} of the family \( \seq{ G_k }_{k \in \mscrK} \) in the category \hyperref[def:group/category]{\( \cat{Grp} \)} of groups is their \hyperref[def:monoid_free_product]{free product} \( \Ast_{k \in \mscrK} G_k \).
\end{proposition}
\begin{proof}
  Our proof of \fullref{thm:monoid_coproduct} can be reused, but with \fullref{thm:free_group_universal_property} instead of \fullref{thm:free_monoid_universal_property}.
\end{proof}

\paragraph{Group presentations}

\begin{definition}\label{def:group_presentation}\mimprovised
  Let \( G \) be a \hyperref[def:group]{group}. Let \( A \) be a \hyperref[def:set]{plain set} and let \( {\sim} \) be a binary relation on the \hyperref[def:free_group]{free group} \( F(A) \).

  If \( G \) is isomorphic to the \hyperref[def:group/quotient]{quotient} \( F(A) / {\cong} \), where \( {\cong} \) is the \hyperref[def:first_order_congruence]{congruence} \hyperref[def:first_order_generated_congruence]{generated} by \( {\sim} \), we call the pair \( (A, {\sim}) \) a \term[en=(group) presentation (\cite[90]{Bourbaki1998Algebra1to3})]{presentation} of \( G \).

  If there exists a presentation \( (A, {\sim}) \) of \( G \) such that \( A \) is finite, we say that \( G \) is \term[en=finitely generated (\cite[90]{Bourbaki1998Algebra1to3})]{finitely generated}\fnote{There is no terminology clash for \enquote{finitely generated group} --- a group \( G \) has a presentation with finitely many generators if and only if it has a finite subset whose \hyperref[def:group/generated]{generated subgroup} is \( G \).}. If there exists a presentation \( (A, \sim) \) where both \( A \) and \( {\sim} \) are finite, we say that the group is \term[en=finitely presented (\cite[90]{Bourbaki1998Algebra1to3})]{finitely presented}.

  Following \incite[90]{Bourbaki1998Algebra1to3}, for finite presentations we will introduce a notation similar to that for \hyperref[def:group/generated]{generated subgroups} --- if \( a_1, \ldots, a_n \) is an enumeration of \( A \) and \( (a_{i_1}, a_{j_1}), \ldots, (a_{i_m}, a_{j_m}) \) be is an enumeration of \( {\sim} \), we denote the presentation by
  \begin{equation}\label{eq:def:group_presentation/finite}
    \braket{ a_1, \ldots, a_n \given a_{i_1} = a_{j_1}, \ldots, a_{i_m} = a_{j_m} }.
  \end{equation}
\end{definition}
\begin{comments}
  \item We base our notation on and terminology \cite[90]{Bourbaki1998Algebra1to3}, but for the quotient we use a congruence relation instead of a set generating a normal subgroup.

  \item Even though we have defined presentations in the context of an existing group, the notation \eqref{eq:def:group_presentation/finite} makes it easy to create groups \enquote{from scratch} based on a set of symbols and desired relations between them. We will use this, for example, in \fullref{thm:cyclic_monoid_classification}.
\end{comments}

\begin{proposition}\label{thm:group_presentation_existence}
  Every \hyperref[def:group]{group} has at least one \hyperref[def:group_presentation]{presentation}.
\end{proposition}
\begin{proof}
  Fix a group \( G \). Consider the embedding \( \iota: G \to F(G) \) into the \hyperref[def:free_group]{free group} of \( G \).

  \Fullref{thm:free_group_universal_property} implies that the identity \( {\id_G: G \to G} \) extends to a unique homomorphism \( \varphi: F(G) \to G \) such that the following diagram commutes:
  \begin{equation*}
    \begin{aligned}
      \includegraphics[page=1]{output/thm__group_presentation_existence}
    \end{aligned}
  \end{equation*}

  The homomorphism \( \varphi \) \hyperref[con:evaluation_homomorphism]{evaluates} each string in \( F(G) \) to an actual value in \( G \).

  \Fullref{thm:group_zero_morphisms/isomorphism} implies that the quotient of \( F(G) / \ker \varphi \) is isomorphic to \( G \). \Fullref{thm:normal_subgroups_and_congruences} implies that the relation \( {\cong} \) defined as
  \begin{equation*}
    x \cong y \T{when} xy^{-1} \in \ker \varphi
  \end{equation*}
  is a \hyperref[def:first_order_congruence]{congruence}.

  Therefore, \( (G, \cong) \) is a presentation of \( G \).
\end{proof}

\paragraph{Cyclic groups}

\begin{definition}\label{def:cyclic_group}\mcite[43]{Jacobson1985BasicAlgebraI}
  We say that a \hyperref[def:group]{group} is \term[ru=циклична группа (\cite[379]{Обрешков1962ВисшаАлгебра}), ru=циклическая группа (\cite[97]{Тыртышников2017ОсновыАлгебры})]{cyclic} if it can be \hyperref[def:group_presentation]{generated} by a single element.

  \Fullref{thm:cyclic_monoid_classification} implies that, up to an isomorphism, there is only one cyclic group for every possible order. When referring to an abstract cyclic group, we will use the notation \( C_n \) for groups of finite cardinality \( n \) or \( C_\infty \) for (countably) infinite groups.
\end{definition}

\begin{definition}\label{def:cyclic_monoid}\mcite[exerc. II.4.6]{MacLane1998CategoryTheory}
  In analogy with \hyperref[def:cyclic_group]{cyclic groups}, we say that a \hyperref[def:monoid]{monoid} is \term{cyclic} if it can be \hyperref[def:group_presentation]{generated} by a single element.
\end{definition}
\begin{comments}
  \item As we will see in \fullref{thm:cyclic_monoid_classification}, all cyclic monoids are groups, and there is only one, up to an isomorphism, infinite cyclic monoid. This explains why cyclic monoids are rarely discussed.
\end{comments}

\begin{lemma}\label{thm:cyclic_monoid_cancellative}
  Every \hyperref[def:cyclic_monoid]{cyclic monoid} is \hyperref[def:binary_operation/cancellative]{cancellative}.
\end{lemma}
\begin{proof}
  Let \( M \) be a cyclic monoid generated by \( a \). By induction on \( k \), we will show that \( a^k a^m = a^k a^l \) implies \( a^m = a^l \).
  \begin{itemize}
    \item The case \( k = 0 \) is trivial.
    \item If \( k > 1 \) and if the inductive hypothesis holds for \( s < k \), then
    \begin{equation*}
      a^k \cdot a^m = a^k \cdot a^l
    \end{equation*}
    is equivalent to
    \begin{equation*}
      a^{k - 1} \cdot a \cdot a^m = a^{k - 1} \cdot a \cdot a^l.
    \end{equation*}

    The inductive hypothesis for \( s = k - 1 \) implies that
    \begin{equation*}
      a \cdot a^m = a \cdot a^l
    \end{equation*}
    and the hypothesis for \( s = 1 \) implies that
    \begin{equation*}
      a^m = a^l.
    \end{equation*}
  \end{itemize}
\end{proof}

\begin{lemma}\label{thm:cyclic_group_generator_order}
  A group with \( n \) elements is \hyperref[def:cyclic_group]{cyclic} if and only if it contains an element of \hyperref[def:group_element_order]{order} \( n \).
\end{lemma}
\begin{comments}
  \item By \fullref{thm:def:group_element_order/group_order}, such an element is necessarily a generator.
\end{comments}
\begin{proof}
  \SufficiencySubProof Let \( G \) be a cyclic group with \( n \) elements generated by \( a \). We will show that \( a \) has order \( n \).

  Since \( G \) has \( n \) elements, by \fullref{thm:def:group_element_order/group_order}, the order of \( a \) divides \( n \).

  Suppose that \( a^m = a^n \) for some \( m < n \). Without loss of generality, let \( m \) be the smallest such value.

  If \( m > 0 \), \fullref{alg:integer_division} gives us nonnegative integers \( q \) and \( r \) such that \( n = qm + r \) and \( 0 \leq r < m \). Then
  \begin{equation*}
    a^n
    \reloset {\eqref{eq:thm:semigroup_exponentiation_properties/multiplication}} =
    a^{qm} \cdot a^r
    \reloset {\eqref{eq:thm:semigroup_exponentiation_properties/repeated}} =
    (a^m)^q \cdot a^r
    =
    e^q \cdot a^r
    =
    e \cdot a^r
    =
    a^r.
  \end{equation*}

  Since \( r < m \), this contradicts the minimality of \( m \). It remains for \( m \) to be \( 0 \).

  It follows that the order of \( a \) is \( n \).

  \NecessitySubProof Let \( a \) be an element of \( G \) of order \( n \). \Fullref{thm:def:group_element_order/group_order} implies that the order of \( a \) divides \( n \). Then \( G \) coincides with the subgroup generated by \( a \).
\end{proof}

\begin{proposition}\label{thm:cyclic_monoid_classification}
  We will classify \hyperref[def:cyclic_group]{cyclic groups} and \hyperref[def:cyclic_monoid]{cyclic monoids}.

  Fix a set \( a \), which we will use as a \hyperref[def:formal_language/symbol]{formal symbol}.

  \begin{thmenum}
    \thmitem{thm:cyclic_monoid_classification/finite_monoid} A cyclic \hi{monoid} is a cyclic \hi{group} if and only if it is finite.

    \thmitem{thm:cyclic_monoid_classification/finite_group} A finite group \( G \) with \( n \) elements is cyclic if and only if the following is a \hyperref[def:group_presentation]{presentation} of \( G \)\fnote{As per \fullref{def:formal_language/empty_string}, we denote the empty string by \( \bnfes \)}:
    \begin{equation}\label{eq:thm:cyclic_monoid_classification/finite_group}
      \braket{ a \given a^n = \bnfes }.
    \end{equation}

    \thmitem{thm:cyclic_monoid_classification/infinite_monoid} An infinite monoid is cyclic if and only if it is isomorphic to the \hyperref[def:free_monoid]{free monoid} \( F(\set{ a }) \).

    \thmitem{thm:cyclic_monoid_classification/infinite_group} An infinite group is cyclic if and only if it is isomorphic to the \hyperref[def:free_group]{free group} \( \braket{ a } \)\fnote{As stated, the proposition conflates the free group \( F(\set{ a }) \) and the group with presentation \( \braket{ a } = F(\set{ a }) / {=} \), but they are clearly isomorphic}.
  \end{thmenum}
\end{proposition}
\begin{comments}
  \item As shown in \fullref{thm:cyclic_group_isomorphic_to_integers_modulo_n}, cyclic groups are also isomorphic to either all integers to the integers modulo some fixed \( n \).
\end{comments}
\begin{proof}
  \SubProofOf{thm:cyclic_monoid_classification/finite_monoid}

  \SufficiencySubProof* Let \( M \) be a cyclic monoid. Suppose that it is a cyclic group. We will show that it is finite.

  Fix elements \( b \) and \( c \), where \( b \) generates \( M \) as a monoid and \( c \) that generates \( M \) as a group.

  Every element of \( M \) is a nonnegative integer power of \( b \). Let \( m \) be the smallest integer such that \( b^m = c^{-1} \) and let \( k \) be the smallest integer such that \( c^k = b \). Then
  \begin{equation*}
    b^{mk + 1} = (b^m)^k \cdot b = (c^{-1})^k \cdot b = c^{-k} \cdot b = c^{-k} \cdot c^k = e_M.
  \end{equation*}

  Then \( M \) has at most \( n \coloneqq mk + 1 \) elements. Indeed, for every positive integer \( s \), \fullref{alg:integer_division} gives us nonnegative integers \( q \) and \( r \) such that \( s = qn + r \) and \( 0 \leq r < n \), where
  \begin{equation*}
    b^s
    =
    (b^n)^q \cdot b^r
    =
    e^q \cdot b^r
    =
    b^r.
  \end{equation*}

  Thus, for every \( s > n \), \( b^s = b^r \) for the remainder \( r < n \).

  \NecessitySubProof* Let \( M \) be a finite cyclic monoid. Since it is cyclic, \fullref{thm:cyclic_monoid_cancellative} implies that it is cancellative. Then \fullref{thm:monoid_inverse_unique} implies that it is a group. If \( b \) generates \( M \) as a monoid, it also generates it as a group, so \( M \) is a cyclic group.

  \SubProofOf{thm:cyclic_monoid_classification/finite_group} Denote the group \eqref{eq:thm:cyclic_monoid_classification/finite_group} by \( C_n \).

  We will show that \( G \) is cyclic if and only if it is isomorphic to \( C_n \). First, however, we need to show that \( C_n \) is a cyclic group with \( n \) elements.

  \SubProof*{Proof that \( C_n \) is cyclic of order \( n \)} Every element of \( C_n \) is a power of \( a \), hence \( C_n \) is cyclic. We will show that it has \( n \) elements.

  Denote by \( \psi \) the right inverse of the projection map \( \pi: F(\set{ a }) \to C_n \). Then \( {\cong} \) is the congruence generated by the relation \( {\sim} \) whose only ordered pair is \( (\psi(a)^n, \psi(a)^0) \).

  We will show via \fullref{thm:induction_on_recursively_defined_relations} on \( \psi(a)^{m + k} \cong \psi(a)^m \) that \( n \) divides \( k \).
  \begin{itemize}
    \item If \( \psi(a)^{m + k} \cong \psi(a)^m \) due to \ref{inf:def:first_order_generated_congruence/base}, then \( \psi(a)^{m + k} \sim \psi(a)^m \), which implies that \( m + k = n \) and \( m = 0 \), hence \( k = n \).

    \item Otherwise, \( \psi(a)^{m + k} \cong \psi(a)^m \) due to \ref{inf:def:first_order_generated_congruence/rec}, and there exists positive integers \( s \) and \( r \) such the inductive hypothesis holds for \( \psi(a)^s \cong \psi(a)^r \) and \( \psi(a)^{m + k - s} \cong \psi(a)^{m - r} \).

    Then \( n \) divides \( s - r \) and \( (m + k - s) - (m - r) = k - (s - r) \), hence also their sum \( k \).
  \end{itemize}

  Thus, \( a^{m + k} = a^m \) implies that \( k \) divides \( n \). In particular, if \( a^n = a^m \) for some nonnegative integer \( m < n \), their difference \( n - m \) divides \( n \), hence \( m = 0 \).

  Therefore, \( e, a, a^2, \ldots, a^{n - 1} \) are distinct elements of \( C_n \). But \( a^n = e_{C_n} \), and hence \( a^{n+k} = a^k \), implying that \( C_n \) has at most \( n \) elements.

  Therefore, \( C_n \) is a cyclic group generated by \( a \) with exactly \( n \) elements.

  \SufficiencySubProof* Suppose that \( G \) is cyclic with generator \( b \). We will show that it is isomorphic to \( C_n \).

  Consider the homomorphism
  \begin{equation*}
    \begin{aligned}
      &\varphi: C_n \to G \\
      &\varphi(a^k) \coloneqq b^k.
    \end{aligned}
  \end{equation*}

  To see that it is injective, suppose that \( \varphi(a^k) = \varphi(a^m) \). Then \( b^k = b^m \), and hence \( b^{k-m} = e_G \). \Fullref{thm:def:group_element_order/neutral} implies that \( n \) divides \( k - m \). We can then use \fullref{thm:def:group_element_order/neutral} again to conclude that \( a^k = a^m \).

  To see that it is surjective, simply observe that \( a^k \) is in the preimage of \( b^k \).

  Therefore, \( \varphi \) is an isomorphism.

  \NecessitySubProof* Suppose that \( G \) is isomorphic to \( C_n \), and let \( \varphi: C_n \to G \) be an isomorphism. We claim that \( \varphi(a) \) generates \( G \).

  Indeed, for every element \( g \) of \( G \), let \( k \) be the smallest nonnegative integer such that \( a^k = \varphi^{-1}(g) \). Then
  \begin{equation*}
    g = \varphi(a^k) = \varphi(a)^k.
  \end{equation*}

  Therefore, \( G \) is cyclic with generator \( \varphi(a) \).

  \SubProofOf{thm:cyclic_monoid_classification/infinite_monoid} Similar to \fullref{thm:cyclic_monoid_classification/finite_group}, but simpler since there is no factorization of the free monoid.

  \SubProofOf{thm:cyclic_monoid_classification/infinite_group} Similar to \fullref{thm:cyclic_monoid_classification/finite_group}, but simpler since there is no factorization of the gree group.
\end{proof}

\begin{proposition}\label{thm:cyclic_subgroup_classification}
  Fix a finite \hyperref[def:cyclic_group]{cyclic group} \( G \) with generator \( a \) and a subgroup \( H \) of \( G \). Let \( s \) be the smallest positive integer such that \( a^s \) is in \( H \).

  \begin{thmenum}
    \thmitem{thm:cyclic_subgroup_classification/cyclic} \( H \) is a cyclic group with generator \( a^s \).

    \thmitem{thm:cyclic_subgroup_classification/finite} If \( G \) has finite cardinality \( n \), then the cardinality \( m \) of \( H \) divides \( n \) and \( s = n / m \).

    \thmitem{thm:cyclic_subgroup_classification/infinite} If \( G \) has infinite cardinality, then so does \( H \).
  \end{thmenum}
\end{proposition}
\begin{proof}
  \SubProofOf{thm:cyclic_subgroup_classification/cyclic} Let \( s \) be the smallest positive integer such that \( a^s \) is in \( H \). For every member \( a^k \) of \( H \), \fullref{alg:integer_division} gives us nonnegative \( q \) and \( r \), where \( r < s \), such that
  \begin{equation*}
    k = qs + r.
  \end{equation*}

  Then
  \begin{equation*}
    a^k = (a^s)^q \cdot a^r.
  \end{equation*}

  Since \( a^s \) belongs to \( H \), so does is integer power \( a^{sq} \). Then its inverse also belongs to \( H \), as well as
  \begin{equation*}
    a^r = a^k a^{-sq}.
  \end{equation*}

  But we have assumed that \( r < s \), which contradicts the minimality of \( s \) unless \( r = 0 \).

  Therefore, every member of \( H \) is a power of \( a^s \), that is, \( H \) is cyclic with generator \( a^s \).

  \SubProofOf{thm:cyclic_subgroup_classification/finite} Suppose that \( G \) has \( n \) elements and that \( H \) has \( m \) elements.

  \Fullref{thm:lagranges_subgroup_theorem} implies that \( m \) divides \( G \).

  Since \( a^{sm} = (a^s)^m = e \), it follows that \( sm \) is a multiple of \( n \). But \( H \) is cyclic of cardinality \( m \), hence \( a^{sk} \neq e \) whenever \( 0 < k < m \). So \( sm \) is the smallest multiple of \( n \), hence \( n \) itself. Therefore, \( s = n / m \).

  \SubProofOf{thm:cyclic_subgroup_classification/infinite} Suppose that \( G \) is countable.

  Suppose also that \( H \) is finite of cardinality \( m \). Then \( a^m = e \). For every nonnegative integer \( k \), \fullref{alg:integer_division} gives us nonnegative \( q \) and \( r \) such that \( k = mq + r \) and \( 0 \leq r < m \).

  Then
  \begin{equation*}
    a^k = (a^m)^q \cdot a^r = e^q \cdot a^r = a^r.
  \end{equation*}

  Therefore, \( G \) itself must have only \( m \) elements. The obtained contradiction shows that \( H \) is infinite.
\end{proof}

\begin{proposition}\label{thm:def:cyclic_group}
  \hyperref[def:cyclic_group]{Cyclic groups} have the following basic properties:
  \begin{thmenum}
    \thmitem{thm:def:cyclic_group/direct_sum} The \hyperref[def:semimodule_direct_sum]{direct sum} \( C_n \oplus C_m \) of two cyclic groups is cyclic if and only if \( n \) and \( m \) are \hyperref[def:coprime_elements]{coprime}.

    \thmitem{thm:def:cyclic_group/generators} A member of \( C_n \) generates it if and only if it has \hyperref[def:group_element_order]{order} \( n \).

    \thmitem{thm:def:cyclic_group/generators_cardinality} The set of \hyperref[def:group_presentation]{generators} of \( C_n \) has cardinality \( \varphi(n) \), where \( \varphi \) is \hyperref[def:eulers_totient_function]{Euler's totient function}.
  \end{thmenum}
\end{proposition}
\begin{proof}
  \SubProofOf{thm:def:cyclic_group/direct_sum} Let \( a \) be a generator of \( C_n \) and \( b \) --- of \( C_m \).

  \SufficiencySubProof* Suppose that \( (a^i, b^j) \) generates \( C_n \oplus C_m \).

  Let \( d \) be a common divisor of \( n \) and \( m \). \Fullref{thm:common_divisor_to_multiple_lemma} implies that \( s \coloneqq nm / d \) is a common multiple. \Fullref{thm:def:group_element_order/neutral} implies that \( a^s = a^0 \) and \( b^s = b^0 \).

  Then
  \begin{equation*}
    (a^i, b^j)^s = (a^{is}, b^{js}) = (a^0, b^0).
  \end{equation*}

  The order of \( (a^i, b^j) \) is thus at most \( s \).

  But it is a generator of \( C_n \oplus C_m \), hence its order is \( nm \). Then \( d = 1 \) and \( n \) and \( m \) are coprime.

  \NecessitySubProof* Conversely, suppose that \( n \) and \( m \) are coprime.

  Suppose that, for some \( j < i \) we have
  \begin{equation*}
    (a, b)^i = (a, b)^j.
  \end{equation*}

  \Fullref{thm:def:group_element_order/neutral} implies that both \( n \) and \( m \) divide \( i - j \). Then their \hyperref[def:lcm]{least common multiple} \( l \) also divides \( i - j \). But since \( n \) and \( m \) are coprime, \( l = nm \), and thus \( nm \) divides \( i - j \).

  Hence, if \( 0 \leq j < i < nm \),
  \begin{equation*}
    (a, b)^i \neq (a, b)^j.
  \end{equation*}

  Then the direct sum \( C_n \oplus C_m \) has at least \( nm \) elements. But by definition it has at most \( nm \) elements. Therefore, \( C_n \oplus C_m \) is a cyclic group of order \( nm \) generated by \( (a, b) \).

  \SubProofOf{thm:def:cyclic_group/generators} By definition of \( \ord(a) \), the subgroup \( \braket{ a } \) has \( \ord(a) \) distinct elements.

  Then \( x \) is a generator of \( C_n \) if and only if \( \ord(a) \) coincides with the cardinality \( n \) of \( C_n \).

  \SubProofOf{thm:def:cyclic_group/generators_cardinality} Let \( a \) be a generator of \( C_n \). \Fullref{thm:def:group_element_order/power} implies that the order of \( a^m \) is \( n / \gcd(n, m) \). This order equals \( n \) if and only if \( n \) and \( m \) are coprime. Therefore, there are \( \varphi(n) \) generators of \( C_n \).
\end{proof}
